%% Generated by Sphinx.
\def\sphinxdocclass{jupyterBook}
\documentclass[letterpaper,10pt,english]{jupyterBook}
\ifdefined\pdfpxdimen
   \let\sphinxpxdimen\pdfpxdimen\else\newdimen\sphinxpxdimen
\fi \sphinxpxdimen=.75bp\relax
\ifdefined\pdfimageresolution
    \pdfimageresolution= \numexpr \dimexpr1in\relax/\sphinxpxdimen\relax
\fi
%% let collapsable pdf bookmarks panel have high depth per default
\PassOptionsToPackage{bookmarksdepth=5}{hyperref}
%% turn off hyperref patch of \index as sphinx.xdy xindy module takes care of
%% suitable \hyperpage mark-up, working around hyperref-xindy incompatibility
\PassOptionsToPackage{hyperindex=false}{hyperref}
%% memoir class requires extra handling
\makeatletter\@ifclassloaded{memoir}
{\ifdefined\memhyperindexfalse\memhyperindexfalse\fi}{}\makeatother

\PassOptionsToPackage{warn}{textcomp}

\catcode`^^^^00a0\active\protected\def^^^^00a0{\leavevmode\nobreak\ }
\usepackage{cmap}
\usepackage{fontspec}
\defaultfontfeatures[\rmfamily,\sffamily,\ttfamily]{}
\usepackage{amsmath,amssymb,amstext}
\usepackage{polyglossia}
\setmainlanguage{english}



\setmainfont{FreeSerif}[
  Extension      = .otf,
  UprightFont    = *,
  ItalicFont     = *Italic,
  BoldFont       = *Bold,
  BoldItalicFont = *BoldItalic
]
\setsansfont{FreeSans}[
  Extension      = .otf,
  UprightFont    = *,
  ItalicFont     = *Oblique,
  BoldFont       = *Bold,
  BoldItalicFont = *BoldOblique,
]
\setmonofont{FreeMono}[
  Extension      = .otf,
  UprightFont    = *,
  ItalicFont     = *Oblique,
  BoldFont       = *Bold,
  BoldItalicFont = *BoldOblique,
]



\usepackage[Bjarne]{fncychap}
\usepackage[,numfigreset=0,mathnumfig]{sphinx}

\fvset{fontsize=\small}
\usepackage{geometry}


% Include hyperref last.
\usepackage{hyperref}
% Fix anchor placement for figures with captions.
\usepackage{hypcap}% it must be loaded after hyperref.
% Set up styles of URL: it should be placed after hyperref.
\urlstyle{same}

\addto\captionsenglish{\renewcommand{\contentsname}{Cadernos}}

\usepackage{sphinxmessages}



         \usepackage[Latin,Greek]{ucharclasses}
        \usepackage{unicode-math}
        % fixing title of the toc
        \addto\captionsenglish{\renewcommand{\contentsname}{Contents}}
        

\title{Métodos Numéricos para Ciências Computacionais e Engenharia}
\date{Dec 24, 2021}
\release{}
\author{Prof.\@{} Gustavo Oliveira}
\newcommand{\sphinxlogo}{\vbox{}}
\renewcommand{\releasename}{}
\makeindex
\begin{document}

\pagestyle{empty}
\sphinxmaketitle
\pagestyle{plain}
\sphinxtableofcontents
\pagestyle{normal}
\phantomsection\label{\detokenize{conteudo::doc}}


\sphinxAtStartPar
Conteúdo para formação complementar empregável em cursos de graduação introdutórios sobre métodos numéricos para aprendizagem ativa baseada em problemas.

\sphinxAtStartPar
Este conteúdo é ensinado na disciplina \sphinxstyleemphasis{Cálculo Numérico} (GDCOC0072) ministrada pelo \sphinxhref{https://gcpeixoto.github.io}{Prof. Gustavo Oliveira} (UFPB/CI/DCC). Todo material é desenvolvido no âmbito do \sphinxhref{http://numbiosis.ci.ufpb.br/pt/inicio/}{Projeto Numbiosis} com suporte do Programa Institucional de Monitoria.

\sphinxAtStartPar
Aplicações específicas em Javascript que simulam diversos métodos do curso foram desenvolvidas pelo egresso \sphinxhref{https://github.com/Vnicius}{Vinícius Veríssimo} e estão disponíveis \sphinxhref{https://vnicius.github.io/numbiosis/}{aqui}.


\part{Programa geral do curso}
\label{\detokenize{conteudo:programa-geral-do-curso}}
\sphinxAtStartPar
\sphinxstyleemphasis{Notebooks} com notas de aula, exemplos resolvidos, algoritmos e exercícios de programação.


\chapter{I. Introdução}
\label{\detokenize{conteudo:i-introducao}}\begin{itemize}
\item {} 
\sphinxAtStartPar
{\hyperref[\detokenize{aula-00-modelagem-programacao::doc}]{\sphinxcrossref{\DUrole{doc,std,std-doc}{Aula 00 \sphinxhyphen{} Modelagem}}}}

\item {} 
\sphinxAtStartPar
{\hyperref[\detokenize{aula-01-ponto-flutuante::doc}]{\sphinxcrossref{\DUrole{doc,std,std-doc}{Aula 01 \sphinxhyphen{} Ponto flutuante}}}}

\item {} 
\sphinxAtStartPar
{\hyperref[\detokenize{aula-02-erros::doc}]{\sphinxcrossref{\DUrole{doc,std,std-doc}{Aula 02 \sphinxhyphen{} Erros numéricos}}}}

\end{itemize}


\chapter{II. Determinação de raízes}
\label{\detokenize{conteudo:ii-determinacao-de-raizes}}\begin{itemize}
\item {} 
\sphinxAtStartPar
{\hyperref[\detokenize{aula-03-analise-grafica::doc}]{\sphinxcrossref{\DUrole{doc,std,std-doc}{Aula 03 \sphinxhyphen{} Análise gráfica}}}}

\item {} 
\sphinxAtStartPar
{\hyperref[\detokenize{aula-04-bissecao::doc}]{\sphinxcrossref{\DUrole{doc,std,std-doc}{Aula 04 \sphinxhyphen{} Bisseção}}}}

\item {} 
\sphinxAtStartPar
{\hyperref[\detokenize{aula-05-ponto-fixo::doc}]{\sphinxcrossref{\DUrole{doc,std,std-doc}{Aula 05 \sphinxhyphen{} Ponto fixo}}}}

\item {} 
\sphinxAtStartPar
{\hyperref[\detokenize{aula-06-newton::doc}]{\sphinxcrossref{\DUrole{doc,std,std-doc}{Aula 06 \sphinxhyphen{} Newton}}}}

\item {} 
\sphinxAtStartPar
{\hyperref[\detokenize{aula-07-secante::doc}]{\sphinxcrossref{\DUrole{doc,std,std-doc}{Aula 07 \sphinxhyphen{} Secante}}}}

\item {} 
\sphinxAtStartPar
{\hyperref[\detokenize{aula-08-muller::doc}]{\sphinxcrossref{\DUrole{doc,std,std-doc}{Aula 08 \sphinxhyphen{} Müller}}}}

\end{itemize}


\chapter{III. Solução de sistemas de equações}
\label{\detokenize{conteudo:iii-solucao-de-sistemas-de-equacoes}}\begin{itemize}
\item {} 
\sphinxAtStartPar
{\hyperref[\detokenize{aula-09-eliminacao-gauss::doc}]{\sphinxcrossref{\DUrole{doc,std,std-doc}{Aula 09 \sphinxhyphen{} Eliminação Gaussiana}}}}

\item {} 
\sphinxAtStartPar
{\hyperref[\detokenize{aula-10-fatoracao-lu::doc}]{\sphinxcrossref{\DUrole{doc,std,std-doc}{Aula 10 \sphinxhyphen{} Fatoração LU}}}}

\item {} 
\sphinxAtStartPar
{\hyperref[\detokenize{aula-11-cholesky::doc}]{\sphinxcrossref{\DUrole{doc,std,std-doc}{Aula 11 \sphinxhyphen{} Cholesky}}}}

\item {} 
\sphinxAtStartPar
{\hyperref[\detokenize{aula-12-jacobi::doc}]{\sphinxcrossref{\DUrole{doc,std,std-doc}{Aula 12 \sphinxhyphen{} Jacobi}}}}

\item {} 
\sphinxAtStartPar
{\hyperref[\detokenize{aula-13-newton-nao-linear::doc}]{\sphinxcrossref{\DUrole{doc,std,std-doc}{Aula 13 \sphinxhyphen{} Newton não\sphinxhyphen{}linear}}}}

\end{itemize}


\chapter{IV. Interpolação e ajuste de curvas}
\label{\detokenize{conteudo:iv-interpolacao-e-ajuste-de-curvas}}\begin{itemize}
\item {} 
\sphinxAtStartPar
{\hyperref[\detokenize{aula-14-interpolacao-lagrange::doc}]{\sphinxcrossref{\DUrole{doc,std,std-doc}{Aula 14 \sphinxhyphen{} Interpolação de Lagrange}}}}

\item {} 
\sphinxAtStartPar
{\hyperref[\detokenize{aula-15-interpolacao-newton::doc}]{\sphinxcrossref{\DUrole{doc,std,std-doc}{Aula 15 \sphinxhyphen{} Interpolação de Newton}}}}

\item {} 
\sphinxAtStartPar
{\hyperref[\detokenize{aula-16-minimos-quadrados::doc}]{\sphinxcrossref{\DUrole{doc,std,std-doc}{Aula 16 \sphinxhyphen{} Mínimos quadrados}}}}

\item {} 
\sphinxAtStartPar
{\hyperref[\detokenize{aula-17-ajusteNaoLinear::doc}]{\sphinxcrossref{\DUrole{doc,std,std-doc}{Aula 17 \sphinxhyphen{} Ajuste não linear}}}}

\end{itemize}


\chapter{V. Integração e diferenciação numérica}
\label{\detokenize{conteudo:v-integracao-e-diferenciacao-numerica}}\begin{itemize}
\item {} 
\sphinxAtStartPar
{\hyperref[\detokenize{aula-18-integracao-newtonCotes::doc}]{\sphinxcrossref{\DUrole{doc,std,std-doc}{Aula 18 \sphinxhyphen{} Integração por Newton\sphinxhyphen{}Cotes}}}}

\item {} 
\sphinxAtStartPar
{\hyperref[\detokenize{aula-19-quadratura-gaussiana::doc}]{\sphinxcrossref{\DUrole{doc,std,std-doc}{Aula 19 \sphinxhyphen{} Quadratura Gaussiana}}}}

\item {} 
\sphinxAtStartPar
{\hyperref[\detokenize{aula-20-diferenciacao-numerica::doc}]{\sphinxcrossref{\DUrole{doc,std,std-doc}{Aula 20 \sphinxhyphen{} Diferenciação numérica}}}}

\end{itemize}


\chapter{VI. Métodos numéricos para EDOs}
\label{\detokenize{conteudo:vi-metodos-numericos-para-edos}}\begin{itemize}
\item {} 
\sphinxAtStartPar
{\hyperref[\detokenize{aula-21-solucoes-edo::doc}]{\sphinxcrossref{\DUrole{doc,std,std-doc}{Aula 21 \sphinxhyphen{} Solução numérica de EDOs}}}}

\item {} 
\sphinxAtStartPar
{\hyperref[\detokenize{aula-22-metodo-euler::doc}]{\sphinxcrossref{\DUrole{doc,std,std-doc}{Aula 22 \sphinxhyphen{} Método de Euler}}}}

\item {} 
\sphinxAtStartPar
{\hyperref[\detokenize{aula-23-taylor-rungeKutta::doc}]{\sphinxcrossref{\DUrole{doc,std,std-doc}{Aula 23 \sphinxhyphen{} Métodos de Taylor e Runge\sphinxhyphen{}Kutta de 2a. ordem}}}}

\end{itemize}


\part{\sphinxstyleemphasis{Code sessions}}
\label{\detokenize{conteudo:code-sessions}}
\sphinxAtStartPar
\sphinxstyleemphasis{Notebooks} com um compêndio de funções de utilidade predefinidas em módulos Python para resolução direta de problemas aplicados às ciências matemáticas, computacionais e engenharias.
\begin{itemize}
\item {} 
\sphinxAtStartPar
{\hyperref[\detokenize{codeSession-1-bisect::doc}]{\sphinxcrossref{\DUrole{doc,std,std-doc}{Code session 1 \sphinxhyphen{} bisect}}}}

\item {} 
\sphinxAtStartPar
{\hyperref[\detokenize{codeSession-2-newton::doc}]{\sphinxcrossref{\DUrole{doc,std,std-doc}{Code session 2 \sphinxhyphen{} newton}}}}

\item {} 
\sphinxAtStartPar
{\hyperref[\detokenize{codeSession-3-polyval::doc}]{\sphinxcrossref{\DUrole{doc,std,std-doc}{Code session 3 \sphinxhyphen{} roots}}}}

\item {} 
\sphinxAtStartPar
{\hyperref[\detokenize{codeSession-4-fsolve::doc}]{\sphinxcrossref{\DUrole{doc,std,std-doc}{Code session 4 \sphinxhyphen{} fsolve}}}}

\item {} 
\sphinxAtStartPar
{\hyperref[\detokenize{codeSession-5-solve::doc}]{\sphinxcrossref{\DUrole{doc,std,std-doc}{Code session 5 \sphinxhyphen{} solve}}}}

\item {} 
\sphinxAtStartPar
{\hyperref[\detokenize{codeSession-6-interp::doc}]{\sphinxcrossref{\DUrole{doc,std,std-doc}{Code session 6 \sphinxhyphen{} interp}}}}

\item {} 
\sphinxAtStartPar
{\hyperref[\detokenize{codeSession-7-fit::doc}]{\sphinxcrossref{\DUrole{doc,std,std-doc}{Code session 7 \sphinxhyphen{} fit}}}}

\item {} 
\sphinxAtStartPar
{\hyperref[\detokenize{codeSession-8-integrate::doc}]{\sphinxcrossref{\DUrole{doc,std,std-doc}{Code session 8 \sphinxhyphen{} integrate}}}}

\item {} 
\sphinxAtStartPar
\DUrole{xref,myst}{Code session 9 \sphinxhyphen{} solve\_ivp}

\end{itemize}


\part{\sphinxstyleemphasis{Listas de exercícios}}
\label{\detokenize{conteudo:listas-de-exercicios}}
\sphinxAtStartPar
\sphinxstyleemphasis{Notebooks} contendo solucionário matemático e computacional de exercícios gerais de menor complexidade.
\begin{itemize}
\item {} 
\sphinxAtStartPar
{\hyperref[\detokenize{lista-1-solucoes::doc}]{\sphinxcrossref{\DUrole{doc,std,std-doc}{Lista 1}}}}

\item {} 
\sphinxAtStartPar
{\hyperref[\detokenize{lista-2-solucoes::doc}]{\sphinxcrossref{\DUrole{doc,std,std-doc}{Lista 2}}}}

\item {} 
\sphinxAtStartPar
{\hyperref[\detokenize{lista-3-solucoes::doc}]{\sphinxcrossref{\DUrole{doc,std,std-doc}{Lista 3}}}}

\item {} 
\sphinxAtStartPar
{\hyperref[\detokenize{lista-4-solucoes::doc}]{\sphinxcrossref{\DUrole{doc,std,std-doc}{Lista 4}}}}

\item {} 
\sphinxAtStartPar
{\hyperref[\detokenize{lista-5-solucoes::doc}]{\sphinxcrossref{\DUrole{doc,std,std-doc}{Lista 5}}}}

\item {} 
\sphinxAtStartPar
{\hyperref[\detokenize{lista-6-solucoes::doc}]{\sphinxcrossref{\DUrole{doc,std,std-doc}{Lista 6}}}}

\end{itemize}


\part{\sphinxstyleemphasis{Conteúdo Extra}}
\label{\detokenize{conteudo:conteudo-extra}}
\sphinxAtStartPar
\sphinxstyleemphasis{Notebooks} com conteúdos complementares não contemplados no curso regular.
\begin{itemize}
\item {} 
\sphinxAtStartPar
{\hyperref[\detokenize{extra/extra-pontoFlutuante::doc}]{\sphinxcrossref{\DUrole{doc,std,std-doc}{Números em ponto flutuante e seus problemas}}}}

\item {} 
\sphinxAtStartPar
{\hyperref[\detokenize{extra/extra-malhasNumericas::doc}]{\sphinxcrossref{\DUrole{doc,std,std-doc}{Malhas numéricas}}}}

\item {} 
\sphinxAtStartPar
\DUrole{xref,myst}{Campos de direção para EDOs}

\item {} 
\sphinxAtStartPar
{\hyperref[\detokenize{extra/extra-eulerMelhorado::doc}]{\sphinxcrossref{\DUrole{doc,std,std-doc}{Melhoramentos do método de Euler}}}}

\item {} 
\sphinxAtStartPar
{\hyperref[\detokenize{extra/extra-estabilidadeEuler::doc}]{\sphinxcrossref{\DUrole{doc,std,std-doc}{Estabilidade do método de Euler}}}}

\item {} 
\sphinxAtStartPar
{\hyperref[\detokenize{extra/extra-eulerImplicito::doc}]{\sphinxcrossref{\DUrole{doc,std,std-doc}{Método de Euler implícito}}}}

\item {} 
\sphinxAtStartPar
{\hyperref[\detokenize{extra/extra-multistep-adamsBashfort::doc}]{\sphinxcrossref{\DUrole{doc,std,std-doc}{Métodos de múltiplos passos: \sphinxstyleemphasis{Adams\sphinxhyphen{}Bashfort}}}}}

\item {} 
\sphinxAtStartPar
{\hyperref[\detokenize{extra/extra-edo-superior::doc}]{\sphinxcrossref{\DUrole{doc,std,std-doc}{EDOs de ordem superior}}}}

\item {} 
\sphinxAtStartPar
\DUrole{xref,myst}{Sistemas de EDOs}

\item {} 
\sphinxAtStartPar
{\hyperref[\detokenize{extra/extra-fft::doc}]{\sphinxcrossref{\DUrole{doc,std,std-doc}{Transformada de Fourier}}}}

\item {} 
\sphinxAtStartPar
{\hyperref[\detokenize{extra/extra-numba::doc}]{\sphinxcrossref{\DUrole{doc,std,std-doc}{Otimização de código}}}}

\end{itemize}


\part{Como contribuir?}
\label{\detokenize{conteudo:como-contribuir}}
\sphinxAtStartPar
O projeto Numbiosis não recebe financiamento direto para bolsas. Todo o conteúdo é desenvolvido pelo Prof. Gustavo Oliveira e alunos (monitores e/ou tutores bolsistas ou voluntários, bem como aqueles que se matriculam no curso e contribuem com melhorias). O material é revisado constantemente, mas possui suporte limitado.

\sphinxAtStartPar
Você é estudante da UFPB e gostaria de contribuir com o projeto? Entre em contato com o Prof. Gustavo.


\chapter{Alguns temas abertos no âmbito do projeto Numbiosis}
\label{\detokenize{conteudo:alguns-temas-abertos-no-ambito-do-projeto-numbiosis}}\begin{itemize}
\item {} 
\sphinxAtStartPar
Implementação de gráficos interativos para visualização 3D de processos iterativos.

\item {} 
\sphinxAtStartPar
Desenvolvimento de APIs para integração de códigos da base Numbiosis em Github/Gitlab.

\item {} 
\sphinxAtStartPar
Configuração de web server JupyterHub para hospedagem de materiais para mini\sphinxhyphen{}cursos remotos.

\item {} 
\sphinxAtStartPar
Desenvolvimento de códigos demonstrativos em Python para aplicações em Engenharias.

\item {} 
\sphinxAtStartPar
Geração de material didático portável (projeto de ensino).

\item {} 
\sphinxAtStartPar
Integração de ferramentas de \sphinxstyleemphasis{autograding}.

\item {} 
\sphinxAtStartPar
Programação orientada a objetos para criação de \sphinxstyleemphasis{smart courses} (módulos para geração de questões customizadas, avaliações e compilações em Latex).

\end{itemize}


\part{Iniciação científica}
\label{\detokenize{conteudo:iniciacao-cientifica}}
\sphinxAtStartPar
Consulte projetos nos horizontes estratégicos do \DUrole{xref,myst}{TRIL Lab} no CI/UFPB. Algumas temas de interesse são:
\begin{itemize}
\item {} 
\sphinxAtStartPar
Computação científica para aplicações em engenharia

\item {} 
\sphinxAtStartPar
Ciência de dados para o setor energético

\item {} 
\sphinxAtStartPar
Dinâmica dos fluidos computacional

\end{itemize}


\part{Cadernos}


\chapter{Modelagem Matemática em Ciências Computacionais e Engenharias}
\label{\detokenize{aula-00-modelagem-programacao:modelagem-matematica-em-ciencias-computacionais-e-engenharias}}\label{\detokenize{aula-00-modelagem-programacao::doc}}
\sphinxAtStartPar
Ao deparar\sphinxhyphen{}se com um problema real, profissionais envolvidos nos diversos setores da economia buscam as melhores estratégias existentes em seu campo de atuação para resolvê\sphinxhyphen{}lo. Supondo que um problema específico possua apenas uma solução e que pudéssemos assemelhá\sphinxhyphen{}lo à uma questão específica, a solução desse problema equivaleria à resposta para a pergunta e a forma de respondê\sphinxhyphen{}la à estratégia seguida para buscar a solução, a qual também poderíamos chamar de \sphinxstyleemphasis{método}.

\sphinxAtStartPar
Alguns exemplos de perguntas que eventualmente surgiriam em domínios específicos do conhecimento são:
\begin{itemize}
\item {} 
\sphinxAtStartPar
que carga máxima seria admissível para um elevador de edifício\sphinxhyphen{}garagem de 15 andares, de maneira que seus cabos, mecanismos e estrutura de elevação não sejam danificados por excesso de tensão? (Engenharia Civil)

\item {} 
\sphinxAtStartPar
quantos mililitros de um fármaco anti\sphinxhyphen{}alérgico poderiam ser injetados por minuto na corrente sanguínea de um indivíduo de modo que não haja efeitos colaterais de overdose? (Biologia)

\item {} 
\sphinxAtStartPar
que magnitude mínima de pressão seria necessária para mover um fluido refrigerante por um nanotubo de um componente eletrônico aquecido de maneira a resfriá\sphinxhyphen{}lo em 2\% de sua temperatura operacional? (Engenharia Mecânica)

\end{itemize}

\sphinxAtStartPar
Evidentemente, perguntas como essas podem ser extremamente difíceis de responder, sendo necessário o envolvimento de uma equipe multidisciplinar, com experiência em variados assuntos. A primeira das perguntas pode envolver profissionais especializados em ciência dos materiais, eletrônica e pneumática; a segunda, matemáticos aplicados, cardiologistas e químicos; a terceira, físicos, analistas de energia e estatísticos.

\sphinxAtStartPar
Resolver um problema científico no mundo atual requer não apenas conhecimento teórico e prático, mas também outras habilidades como a capacidade de pensar computacionalmente e resolver problemas utilizando o enorme potencial da computação de alto desempenho. Na verdade, a maioria dos problemas atuais não pode ser resolvida sem a intervenção de métodos numéricos. Há, certamente, um amplo leque de opções metodológicas para resolvê\sphinxhyphen{}los. Entretanto, em linhas gerais, são quatro as fases de resolução de um problema aplicado, como mostra a  \hyperref[\detokenize{aula-00-modelagem-programacao:passos-modelagem}]{Fig.\@ \ref{\detokenize{aula-00-modelagem-programacao:passos-modelagem}}}.

\begin{figure}[htbp]
\centering
\capstart

\noindent\sphinxincludegraphics[width=700\sphinxpxdimen]{{passos-modelagem}.png}
\caption{As quatro fases fundamentais para a resolução de um problema aplicado nas ciências computacionais e engenharias.}\label{\detokenize{aula-00-modelagem-programacao:passos-modelagem}}\end{figure}


\section{Modelos matemáticos}
\label{\detokenize{aula-00-modelagem-programacao:modelos-matematicos}}
\sphinxAtStartPar
Um modelo matemático pode ser definido, de forma geral, como uma formulação ou equação que expressa as características essenciais de um sistema ou processo físico em termos matemáticos. A fórmula pode variar de uma simples relação algébrica a um conjunto grande e complicado de equações diferenciais.

\sphinxAtStartPar
Por exemplo, com base em suas observações, Newton formulou sua segunda lei do movimento. Se escrevermos a taxa de variação temporal da velocidade pela derivada \(\frac{dv}{dt}\) (em \(m/s)\), um modelo matemático que obtemos para a segunda lei de Newton é
\begin{equation*}
\begin{split}\dfrac{dv}{dt} = \dfrac{F}{m},\end{split}
\end{equation*}
\sphinxAtStartPar
onde \(F\) é a força resultante (em \(N\)) agindo sobre o corpo e \(m\) a massa (em \(kg\)).

\sphinxAtStartPar
Este modelo matemático, assim como vários outros, possui as seguintes características:
\begin{itemize}
\item {} 
\sphinxAtStartPar
descrevem um processo ou sistema natural em termos matemáticos;

\item {} 
\sphinxAtStartPar
representam uma idealização (simplificação) da realidade. Isto é, o modelo ignora alguns “detalhes” do processo natural e se concentra em suas manifestações essenciais;

\item {} 
\sphinxAtStartPar
produzem resultados que podem ser reproduzidos e usados para propósitos de previsão. Por exemplo, se a força sobre um corpo e a sua massa forem conhecidas, o modelo pode ser usado para estimar a aceleração \(a=\frac{dv}{dt}\) do corpo.

\end{itemize}

\sphinxAtStartPar
Consideremos um paraquedista em queda livre. Durante seu movimento, duas forças principais atuam sobre ele. A força gravitacional \(F_G\), com sentido para baixo, e a força da resistência do ar (arrasto) \(F_D\), em sentido oposto. Se o sentido positivo for conferido à força gravitacional, podemos modelar a força resultante como
\begin{equation*}
\begin{split}F = F_G - F_D = mg - cv,\end{split}
\end{equation*}
\sphinxAtStartPar
onde \(g\) é a constante gravitacional e \(c\) o \sphinxstyleemphasis{coeficiente de arrasto}, medido em \(kg/s\). Vale ressaltar que ao assumirmos \(F_D = cv\), estamos dizendo que a força de arrasto é linearmente proporcional à velocidade. Entretanto, na realidade, esta relação é não\sphinxhyphen{}linear.

\sphinxAtStartPar
Dessa maneira, podemos chegar a um modelo mais completo substituindo a força resultante assim obtendo:
\begin{equation*}
\begin{split}\dfrac{dv}{dt} = \dfrac{mg - cv}{m} = g - \dfrac{c}{m}v.\end{split}
\end{equation*}
\sphinxAtStartPar
Esta \sphinxstyleemphasis{equação diferencial ordinária} (EDO) possui uma solução geral que pode ser encontrada por técnicas analíticas. Uma solução particular para esta EDO é obtida ao impormos uma \sphinxstyleemphasis{condição inicial}. Visto que o paraquedista está em repouso antes da queda, temos que \(v=0\) quando \(t=0\). Usando esta informação, concluímos que o perfil de velocidade é dado por
\begin{equation*}
\begin{split}v(t) = \dfrac{gm}{c}(1 - e^{-(c/m)t}).\end{split}
\end{equation*}
\sphinxAtStartPar
Como veremos adiante em um estado de caso real apresentado por Yan Ferreira, esta função cresce exponencialmente até atingir uma estabilização na \sphinxstyleemphasis{velocidade terminal}.


\section{O salto de paraquedas de Yan e Celso}
\label{\detokenize{aula-00-modelagem-programacao:o-salto-de-paraquedas-de-yan-e-celso}}
\sphinxAtStartPar
Como exemplo, veremos a análise do salto de paraquedas de Yan com seu irmão Celso. Vamos calcular a aceleração que seria atingida por Yan desde o salto até o momento da abertura de seu paraquedas. Na época do salto, Yan estava com 65 kg e o ar apresentava um coeficiente de arrasto estimado em 12,5 kg/s.

\sphinxAtStartPar
Utilizando a fórmula dada, podemos calcular a velocidade atingida por Yan em relação ao tempo. A seguir, criamos uma função para calcular \(v(t)\) nos 10 primeiros segundos do salto, que foi o tempo que Yan permaneceu em queda até a abertura do paraquedas.

\begin{sphinxVerbatim}[commandchars=\\\{\}]
\PYG{c+c1}{\PYGZsh{} velocidade no salto de Yan}
\PYG{k+kn}{import} \PYG{n+nn}{numpy} \PYG{k}{as} \PYG{n+nn}{np}
\PYG{k+kn}{import} \PYG{n+nn}{matplotlib}\PYG{n+nn}{.}\PYG{n+nn}{pyplot} \PYG{k}{as} \PYG{n+nn}{plt}


\PYG{n}{arr\PYGZus{}yan} \PYG{o}{=} \PYG{l+m+mi}{11}\PYG{o}{*}\PYG{p}{[}\PYG{l+m+mi}{0}\PYG{p}{]}
\PYG{n}{contador} \PYG{o}{=} \PYG{l+m+mi}{0}
\PYG{k}{while} \PYG{p}{(}\PYG{n}{contador} \PYG{o}{\PYGZlt{}}\PYG{o}{=} \PYG{l+m+mi}{10}\PYG{p}{)}\PYG{p}{:}
    \PYG{n}{a} \PYG{o}{=} \PYG{p}{(}\PYG{p}{(}\PYG{l+m+mf}{9.8} \PYG{o}{*} \PYG{l+m+mi}{65}\PYG{p}{)}\PYG{o}{/}\PYG{l+m+mf}{12.5}\PYG{p}{)}\PYG{o}{*}\PYG{p}{(}\PYG{l+m+mi}{1} \PYG{o}{\PYGZhy{}} \PYG{n}{np}\PYG{o}{.}\PYG{n}{exp}\PYG{p}{(}\PYG{o}{\PYGZhy{}}\PYG{p}{(}\PYG{l+m+mf}{12.5}\PYG{o}{/}\PYG{l+m+mi}{65}\PYG{p}{)}\PYG{o}{*}\PYG{n}{contador}\PYG{p}{)}\PYG{p}{)}   
    \PYG{n}{arr\PYGZus{}yan}\PYG{p}{[}\PYG{n}{contador}\PYG{p}{]} \PYG{o}{=} \PYG{n}{a}
    \PYG{n+nb}{print}\PYG{p}{(}\PYG{l+s+s2}{\PYGZdq{}}\PYG{l+s+si}{\PYGZob{}0:.4f\PYGZcb{}}\PYG{l+s+s2}{ m/s}\PYG{l+s+s2}{\PYGZdq{}}\PYG{o}{.}\PYG{n}{format}\PYG{p}{(}\PYG{n}{a}\PYG{p}{)}\PYG{p}{)}
    \PYG{n}{contador} \PYG{o}{+}\PYG{o}{=} \PYG{l+m+mi}{1}
\PYG{k}{else}\PYG{p}{:}
    \PYG{n+nb}{print}\PYG{p}{(}\PYG{l+s+s1}{\PYGZsq{}}\PYG{l+s+s1}{==\PYGZgt{} Abertura do paraquedas.}\PYG{l+s+s1}{\PYGZsq{}}\PYG{p}{)}
    
\PYG{n}{plt}\PYG{o}{.}\PYG{n}{plot}\PYG{p}{(}\PYG{n}{arr\PYGZus{}yan}\PYG{p}{,}\PYG{l+s+s1}{\PYGZsq{}}\PYG{l+s+s1}{o\PYGZhy{}b}\PYG{l+s+s1}{\PYGZsq{}}\PYG{p}{)}
\PYG{n}{plt}\PYG{o}{.}\PYG{n}{xlabel}\PYG{p}{(}\PYG{l+s+s1}{\PYGZsq{}}\PYG{l+s+s1}{tempo}\PYG{l+s+s1}{\PYGZsq{}}\PYG{p}{)}
\PYG{n}{plt}\PYG{o}{.}\PYG{n}{ylabel}\PYG{p}{(}\PYG{l+s+s1}{\PYGZsq{}}\PYG{l+s+s1}{velocidade}\PYG{l+s+s1}{\PYGZsq{}}\PYG{p}{)}
\PYG{n}{plt}\PYG{o}{.}\PYG{n}{grid}\PYG{p}{(}\PYG{p}{)}
\PYG{n}{plt}\PYG{o}{.}\PYG{n}{axhline}\PYG{p}{(}\PYG{n}{a}\PYG{p}{,}\PYG{n}{color}\PYG{o}{=}\PYG{l+s+s1}{\PYGZsq{}}\PYG{l+s+s1}{r}\PYG{l+s+s1}{\PYGZsq{}}\PYG{p}{,}\PYG{n}{ls}\PYG{o}{=}\PYG{l+s+s1}{\PYGZsq{}}\PYG{l+s+s1}{\PYGZhy{}\PYGZhy{}}\PYG{l+s+s1}{\PYGZsq{}}\PYG{p}{)}\PYG{p}{;}
\end{sphinxVerbatim}

\begin{sphinxVerbatim}[commandchars=\\\{\}]
0.0000 m/s
8.9153 m/s
16.2709 m/s
22.3397 m/s
27.3467 m/s
31.4778 m/s
34.8861 m/s
37.6982 m/s
40.0183 m/s
41.9325 m/s
43.5119 m/s
==\PYGZgt{} Abertura do paraquedas.
\end{sphinxVerbatim}

\noindent\sphinxincludegraphics{{aula-00-modelagem-programacao_3_1}.png}

\sphinxAtStartPar
Porém, Celso, irmão de Yan, também saltou com ele, em separado. Celso, tem mais 20kg a mais do que Yan. Então, vamos ver como a massa influenciou a velocidade no salto de Celso e comparas as curvas.

\begin{sphinxVerbatim}[commandchars=\\\{\}]
\PYG{c+c1}{\PYGZsh{} velocidade no salto de Yan}


\PYG{k+kn}{from} \PYG{n+nn}{math} \PYG{k+kn}{import} \PYG{n}{exp}
\PYG{k+kn}{import} \PYG{n+nn}{matplotlib}\PYG{n+nn}{.}\PYG{n+nn}{pyplot} \PYG{k}{as} \PYG{n+nn}{plt}

\PYG{n}{arr\PYGZus{}celso} \PYG{o}{=} \PYG{n}{arr\PYGZus{}yan}\PYG{p}{[}\PYG{p}{:}\PYG{p}{]}
\PYG{n}{contador} \PYG{o}{=} \PYG{l+m+mi}{0}
\PYG{k}{while} \PYG{p}{(}\PYG{n}{contador} \PYG{o}{\PYGZlt{}}\PYG{o}{=} \PYG{l+m+mi}{10}\PYG{p}{)}\PYG{p}{:}       
    \PYG{n}{b} \PYG{o}{=} \PYG{p}{(}\PYG{p}{(}\PYG{l+m+mf}{9.8} \PYG{o}{*} \PYG{l+m+mi}{85}\PYG{p}{)}\PYG{o}{/}\PYG{l+m+mf}{12.5}\PYG{p}{)}\PYG{o}{*}\PYG{p}{(}\PYG{l+m+mi}{1}\PYG{o}{\PYGZhy{}} \PYG{n}{exp}\PYG{p}{(}\PYG{o}{\PYGZhy{}}\PYG{p}{(}\PYG{l+m+mf}{12.5}\PYG{o}{/}\PYG{l+m+mi}{85}\PYG{p}{)}\PYG{o}{*}\PYG{n}{contador}\PYG{p}{)}\PYG{p}{)}   
    \PYG{n}{arr\PYGZus{}celso}\PYG{p}{[}\PYG{n}{contador}\PYG{p}{]} \PYG{o}{=} \PYG{n}{b}
    \PYG{n+nb}{print}\PYG{p}{(}\PYG{l+s+s2}{\PYGZdq{}}\PYG{l+s+si}{\PYGZob{}0:.4f\PYGZcb{}}\PYG{l+s+s2}{ m/s}\PYG{l+s+s2}{\PYGZdq{}}\PYG{o}{.}\PYG{n}{format}\PYG{p}{(}\PYG{n}{b}\PYG{p}{)}\PYG{p}{)}
    \PYG{n}{contador} \PYG{o}{+}\PYG{o}{=} \PYG{l+m+mi}{1}
\PYG{k}{else}\PYG{p}{:}
    \PYG{n+nb}{print}\PYG{p}{(}\PYG{l+s+s1}{\PYGZsq{}}\PYG{l+s+s1}{==\PYGZgt{} Abertura do paraquedas}\PYG{l+s+s1}{\PYGZsq{}}\PYG{p}{)}
    
\PYG{n}{plt}\PYG{o}{.}\PYG{n}{plot}\PYG{p}{(}\PYG{n}{arr\PYGZus{}yan}\PYG{p}{,}\PYG{l+s+s1}{\PYGZsq{}}\PYG{l+s+s1}{o\PYGZhy{}b}\PYG{l+s+s1}{\PYGZsq{}}\PYG{p}{)}
\PYG{n}{plt}\PYG{o}{.}\PYG{n}{plot}\PYG{p}{(}\PYG{n}{arr\PYGZus{}celso}\PYG{p}{,}\PYG{l+s+s1}{\PYGZsq{}}\PYG{l+s+s1}{o\PYGZhy{}g}\PYG{l+s+s1}{\PYGZsq{}}\PYG{p}{)}
\PYG{n}{plt}\PYG{o}{.}\PYG{n}{xlabel}\PYG{p}{(}\PYG{l+s+s1}{\PYGZsq{}}\PYG{l+s+s1}{tempo}\PYG{l+s+s1}{\PYGZsq{}}\PYG{p}{)}
\PYG{n}{plt}\PYG{o}{.}\PYG{n}{ylabel}\PYG{p}{(}\PYG{l+s+s1}{\PYGZsq{}}\PYG{l+s+s1}{velocidade}\PYG{l+s+s1}{\PYGZsq{}}\PYG{p}{)}
\PYG{n}{plt}\PYG{o}{.}\PYG{n}{grid}\PYG{p}{(}\PYG{p}{)}
\PYG{n}{plt}\PYG{o}{.}\PYG{n}{axhline}\PYG{p}{(}\PYG{n}{a}\PYG{p}{,}\PYG{n}{color}\PYG{o}{=}\PYG{l+s+s1}{\PYGZsq{}}\PYG{l+s+s1}{r}\PYG{l+s+s1}{\PYGZsq{}}\PYG{p}{,}\PYG{n}{ls}\PYG{o}{=}\PYG{l+s+s1}{\PYGZsq{}}\PYG{l+s+s1}{\PYGZhy{}\PYGZhy{}}\PYG{l+s+s1}{\PYGZsq{}}\PYG{p}{)}
\PYG{n}{plt}\PYG{o}{.}\PYG{n}{axhline}\PYG{p}{(}\PYG{n}{b}\PYG{p}{,}\PYG{n}{color}\PYG{o}{=}\PYG{l+s+s1}{\PYGZsq{}}\PYG{l+s+s1}{m}\PYG{l+s+s1}{\PYGZsq{}}\PYG{p}{,}\PYG{n}{ls}\PYG{o}{=}\PYG{l+s+s1}{\PYGZsq{}}\PYG{l+s+s1}{\PYGZhy{}\PYGZhy{}}\PYG{l+s+s1}{\PYGZsq{}}\PYG{p}{)}\PYG{p}{;}
\end{sphinxVerbatim}

\begin{sphinxVerbatim}[commandchars=\\\{\}]
0.0000 m/s
9.1135 m/s
16.9806 m/s
23.7719 m/s
29.6344 m/s
34.6952 m/s
39.0638 m/s
42.8351 m/s
46.0905 m/s
48.9008 m/s
51.3268 m/s
==\PYGZgt{} Abertura do paraquedas
\end{sphinxVerbatim}

\noindent\sphinxincludegraphics{{aula-00-modelagem-programacao_5_1}.png}

\sphinxAtStartPar
Podemos observar que quanto maior a massa, maior é a velocidade atingida. Por esta razão, Celso chegou ao solo antes de Yan.

\sphinxAtStartPar
Este breve exemplo nos mostra a implementação de um modelo matemático em Python, onde utilizamos partes de programação \sphinxstyleemphasis{estruturada} e \sphinxstyleemphasis{modular}. Estruturada, no sentido das instruções e modular no sentido de que aproveitamos os pacotes (ou \sphinxstylestrong{módulos}) \sphinxstyleemphasis{math} e \sphinxstyleemphasis{numpy} para invocarmos a função exponencial \sphinxcode{\sphinxupquote{exp}} e os as funções para plotagens gráficas.


\subsection{Deslocamento até a abertura do paraquedas}
\label{\detokenize{aula-00-modelagem-programacao:deslocamento-ate-a-abertura-do-paraquedas}}
\sphinxAtStartPar
Sabemos da Física e do Cálculo que o deslocamento é a integral da velocidade com relação ao tempo. Portanto, se \(D_Y\) e \(D_C\) foram os deslocamentos de Yan e Celso em seus saltos, podemo usar a integral:
\begin{equation*}
\begin{split}D = \int_0^{10} v(t) \, dt,\end{split}
\end{equation*}
\sphinxAtStartPar
em cada caso para computar esses deslocamentos. No Python, podemos fazer isso com o código abaixo (que você entenderá mais tarde como fazer).

\begin{sphinxVerbatim}[commandchars=\\\{\}]
\PYG{k+kn}{import} \PYG{n+nn}{sympy} \PYG{k}{as} \PYG{n+nn}{sy}

\PYG{n}{t}\PYG{p}{,}\PYG{n}{g}\PYG{p}{,}\PYG{n}{m}\PYG{p}{,}\PYG{n}{c} \PYG{o}{=} \PYG{n}{sy}\PYG{o}{.}\PYG{n}{symbols}\PYG{p}{(}\PYG{l+s+s1}{\PYGZsq{}}\PYG{l+s+s1}{t g m c}\PYG{l+s+s1}{\PYGZsq{}}\PYG{p}{)}
\PYG{n}{v} \PYG{o}{=} \PYG{n}{g}\PYG{o}{*}\PYG{n}{m}\PYG{o}{/}\PYG{n}{c}\PYG{o}{*}\PYG{p}{(}\PYG{l+m+mi}{1} \PYG{o}{\PYGZhy{}} \PYG{n}{sy}\PYG{o}{.}\PYG{n}{exp}\PYG{p}{(}\PYG{o}{\PYGZhy{}}\PYG{n}{c}\PYG{o}{/}\PYG{n}{m}\PYG{o}{*}\PYG{n}{t}\PYG{p}{)}\PYG{p}{)}
\PYG{n}{s1} \PYG{o}{=} \PYG{n}{sy}\PYG{o}{.}\PYG{n}{integrate}\PYG{p}{(}\PYG{n}{v}\PYG{p}{,}\PYG{p}{(}\PYG{n}{t}\PYG{p}{,}\PYG{l+m+mi}{0}\PYG{p}{,}\PYG{l+m+mi}{10}\PYG{p}{)}\PYG{p}{)}\PYG{o}{.}\PYG{n}{subs}\PYG{p}{(}\PYG{p}{\PYGZob{}}\PYG{l+s+s1}{\PYGZsq{}}\PYG{l+s+s1}{m}\PYG{l+s+s1}{\PYGZsq{}}\PYG{p}{:}\PYG{l+m+mf}{65.0}\PYG{p}{,}\PYG{l+s+s1}{\PYGZsq{}}\PYG{l+s+s1}{g}\PYG{l+s+s1}{\PYGZsq{}}\PYG{p}{:}\PYG{l+m+mf}{9.8}\PYG{p}{,}\PYG{l+s+s1}{\PYGZsq{}}\PYG{l+s+s1}{c}\PYG{l+s+s1}{\PYGZsq{}}\PYG{p}{:}\PYG{l+m+mf}{12.5}\PYG{p}{\PYGZcb{}}\PYG{p}{)}
\PYG{n}{s2} \PYG{o}{=} \PYG{n}{sy}\PYG{o}{.}\PYG{n}{integrate}\PYG{p}{(}\PYG{n}{v}\PYG{p}{,}\PYG{p}{(}\PYG{n}{t}\PYG{p}{,}\PYG{l+m+mi}{0}\PYG{p}{,}\PYG{l+m+mi}{10}\PYG{p}{)}\PYG{p}{)}\PYG{o}{.}\PYG{n}{subs}\PYG{p}{(}\PYG{p}{\PYGZob{}}\PYG{l+s+s1}{\PYGZsq{}}\PYG{l+s+s1}{m}\PYG{l+s+s1}{\PYGZsq{}}\PYG{p}{:}\PYG{l+m+mf}{85.0}\PYG{p}{,}\PYG{l+s+s1}{\PYGZsq{}}\PYG{l+s+s1}{g}\PYG{l+s+s1}{\PYGZsq{}}\PYG{p}{:}\PYG{l+m+mf}{9.8}\PYG{p}{,}\PYG{l+s+s1}{\PYGZsq{}}\PYG{l+s+s1}{c}\PYG{l+s+s1}{\PYGZsq{}}\PYG{p}{:}\PYG{l+m+mf}{12.5}\PYG{p}{\PYGZcb{}}\PYG{p}{)}
\PYG{n+nb}{print}\PYG{p}{(}\PYG{l+s+s2}{\PYGZdq{}}\PYG{l+s+s2}{Yan voou incríveis DY = }\PYG{l+s+si}{\PYGZob{}0:.2f\PYGZcb{}}\PYG{l+s+s2}{ metros em 10 segundos!}\PYG{l+s+s2}{\PYGZdq{}}\PYG{o}{.}\PYG{n}{format}\PYG{p}{(}\PYG{n}{s1}\PYG{p}{)}\PYG{p}{)}
\PYG{n+nb}{print}\PYG{p}{(}\PYG{l+s+s2}{\PYGZdq{}}\PYG{l+s+s2}{Celso voou incríveis DC = }\PYG{l+s+si}{\PYGZob{}0:.2f\PYGZcb{}}\PYG{l+s+s2}{ metros em 10 segundos!}\PYG{l+s+s2}{\PYGZdq{}}\PYG{o}{.}\PYG{n}{format}\PYG{p}{(}\PYG{n}{s2}\PYG{p}{)}\PYG{p}{)}
\end{sphinxVerbatim}

\begin{sphinxVerbatim}[commandchars=\\\{\}]
Yan voou incríveis DY = 283.34 metros em 10 segundos!
Celso voou incríveis DC = 317.38 metros em 10 segundos!
\end{sphinxVerbatim}


\section{Programação estruturada e modular}
\label{\detokenize{aula-00-modelagem-programacao:programacao-estruturada-e-modular}}

\subsection{Prgramação estruturada}
\label{\detokenize{aula-00-modelagem-programacao:prgramacao-estruturada}}
\sphinxAtStartPar
A idéia central por trás da programação estruturada é que qualquer algoritmo numérico pode ser composto de três estruturas de controle fundamentais: \sphinxstyleemphasis{sequencia}, \sphinxstyleemphasis{seleção} e \sphinxstyleemphasis{repetição}.

\sphinxAtStartPar
Nos primórdios da computação, os programadores usualmente não prestavam muita atenção ao fato de o programa ser claro e fácil de entender. Hoje, é reconhecido que existem muitos benefícios em escrever um código bem organizado e bem estruturado. Além do benefício óbvio de tornar o software mais fácil de ser compartilhado, isso também ajuda a garantir um desenvolvimento de programa mais eficiente.

\sphinxAtStartPar
Portanto, algoritmos bem estruturados são, invariavelmente, fáceis de corrigir e testar, resultando em programas que têm um tempo de desenvolvimento e atualização menor. Embora a programação estruturada seja flexível o suficiente para permitir criatividade e expressões pessoais, suas regras impõem restrições suficientes para garantir um código final de mais qualidade, mais limpo e mais elegante, quando comparada à versão não estruturada.


\subsection{Programação modular}
\label{\detokenize{aula-00-modelagem-programacao:programacao-modular}}
\sphinxAtStartPar
Na programação modular, a idéia é que cada módulo desenvolva uma tarefa específica e tenha um único ponto de entrada e um único ponto de saída, de modo que o desenvolvedor possa reutilizá\sphinxhyphen{}lo invariavelmente em várias aplicações.

\sphinxAtStartPar
Dividir tarefas ou objetivos em partes mais simples é uma maneira de torná\sphinxhyphen{}los mais fáceis de tratar. Pensando dessa maneira, os programadores começaram a dividir grandes problemas em subproblemas menores, ou \sphinxstylestrong{módulos}, que podem ser desenvolvidos de forma separada e até mesmo por pessoas diferentes, sem que isso interfira no resultado final.

\sphinxAtStartPar
Hoje em dia, todas as linguagens de programação modernas, tais como C++, Java, Javascript e a própria Python utilizam módulos (também conhecidos como \sphinxstyleemphasis{pacotes} ou \sphinxstyleemphasis{bibliotecas}). Algumas características diferenciais da programação modular são a manutençao facilitada e a reusabilidade do código em programas posteriores.

\sphinxAtStartPar
Abaixo, mostramos um exemplo avançado de como criar um módulo em Python para lidar com pontos na Geometria Plana.
Nosso módulo poderia ser salvo em um arquivo chamado \sphinxcode{\sphinxupquote{ponto.py}}, por exemplo e utilizado em programas próprios que viermos a desenvolver. Neste exemplo, a \sphinxstyleemphasis{classe} \sphinxcode{\sphinxupquote{Ponto}} possui funções para realizar as seguintes operações:
\begin{itemize}
\item {} 
\sphinxAtStartPar
criar um novo ponto;

\item {} 
\sphinxAtStartPar
calcular a distância Euclidiana entre dois pontos;

\item {} 
\sphinxAtStartPar
calcular a área de um triângulo pela fórmula de Heron e

\item {} 
\sphinxAtStartPar
imprimir o valor da área de um triângulo.

\end{itemize}

\begin{sphinxVerbatim}[commandchars=\\\{\}]
\PYG{l+s+sd}{\PYGZdq{}\PYGZdq{}\PYGZdq{}}
\PYG{l+s+sd}{Módulo: ponto.py}
\PYG{l+s+sd}{Exemplo de programação modular. }
\PYG{l+s+sd}{Classe para trabalhar com pontos do espaço 2D.}
\PYG{l+s+sd}{\PYGZdq{}\PYGZdq{}\PYGZdq{}}

\PYG{k+kn}{import} \PYG{n+nn}{numpy} \PYG{k}{as} \PYG{n+nn}{np} 
\PYG{k+kn}{import} \PYG{n+nn}{matplotlib}\PYG{n+nn}{.}\PYG{n+nn}{pyplot} \PYG{k}{as} \PYG{n+nn}{plt}

\PYG{k}{class} \PYG{n+nc}{Ponto}\PYG{p}{:}

    \PYG{c+c1}{\PYGZsh{} inicialização de um ponto arbitrário com coordenadas (xp,yp)}
    \PYG{k}{def} \PYG{n+nf+fm}{\PYGZus{}\PYGZus{}init\PYGZus{}\PYGZus{}}\PYG{p}{(}\PYG{n+nb+bp}{self}\PYG{p}{,} \PYG{n}{xp}\PYG{p}{,} \PYG{n}{yp}\PYG{p}{)}\PYG{p}{:}
        \PYG{n+nb+bp}{self}\PYG{o}{.}\PYG{n}{x} \PYG{o}{=} \PYG{n}{xp}
        \PYG{n+nb+bp}{self}\PYG{o}{.}\PYG{n}{y} \PYG{o}{=} \PYG{n}{yp}

    \PYG{c+c1}{\PYGZsh{} fórmula da distância entre dois pontos}
    \PYG{k}{def} \PYG{n+nf}{dist}\PYG{p}{(}\PYG{n}{P1}\PYG{p}{,}\PYG{n}{P2}\PYG{p}{)}\PYG{p}{:}                                        
        \PYG{k}{return} \PYG{p}{(} \PYG{p}{(}\PYG{n}{P2}\PYG{o}{.}\PYG{n}{x} \PYG{o}{\PYGZhy{}} \PYG{n}{P1}\PYG{o}{.}\PYG{n}{x}\PYG{p}{)}\PYG{o}{*}\PYG{o}{*}\PYG{l+m+mi}{2} \PYG{o}{+} \PYG{p}{(}\PYG{n}{P2}\PYG{o}{.}\PYG{n}{y} \PYG{o}{\PYGZhy{}} \PYG{n}{P1}\PYG{o}{.}\PYG{n}{y}\PYG{p}{)}\PYG{o}{*}\PYG{o}{*}\PYG{l+m+mi}{2} \PYG{p}{)}\PYG{o}{*}\PYG{o}{*}\PYG{l+m+mf}{0.5}
    
    \PYG{c+c1}{\PYGZsh{} área de um triângulo ABC pela fórmula de Heron }
    \PYG{k}{def} \PYG{n+nf}{area\PYGZus{}heron}\PYG{p}{(}\PYG{n}{P1}\PYG{p}{,}\PYG{n}{P2}\PYG{p}{,}\PYG{n}{P3}\PYG{p}{)}\PYG{p}{:}
                
        \PYG{n}{a} \PYG{o}{=} \PYG{n}{Ponto}\PYG{o}{.}\PYG{n}{dist}\PYG{p}{(}\PYG{n}{P1}\PYG{p}{,}\PYG{n}{P2}\PYG{p}{)}  \PYG{c+c1}{\PYGZsh{} comprimento |AB|}
        \PYG{n}{b} \PYG{o}{=} \PYG{n}{Ponto}\PYG{o}{.}\PYG{n}{dist}\PYG{p}{(}\PYG{n}{P2}\PYG{p}{,}\PYG{n}{P3}\PYG{p}{)}  \PYG{c+c1}{\PYGZsh{} comprimento |BC|}
        \PYG{n}{c} \PYG{o}{=} \PYG{n}{Ponto}\PYG{o}{.}\PYG{n}{dist}\PYG{p}{(}\PYG{n}{P3}\PYG{p}{,}\PYG{n}{P1}\PYG{p}{)}  \PYG{c+c1}{\PYGZsh{} comprimento |CA|}
        
        \PYG{n}{p} \PYG{o}{=} \PYG{l+m+mf}{0.5}\PYG{o}{*}\PYG{p}{(}\PYG{n}{a} \PYG{o}{+} \PYG{n}{b} \PYG{o}{+} \PYG{n}{c}\PYG{p}{)}  \PYG{c+c1}{\PYGZsh{} semiperímetro }
        
        \PYG{n}{A} \PYG{o}{=} \PYG{p}{(} \PYG{n}{p}\PYG{o}{*}\PYG{p}{(}\PYG{n}{p} \PYG{o}{\PYGZhy{}} \PYG{n}{a}\PYG{p}{)}\PYG{o}{*}\PYG{p}{(}\PYG{n}{p} \PYG{o}{\PYGZhy{}} \PYG{n}{b}\PYG{p}{)}\PYG{o}{*}\PYG{p}{(}\PYG{n}{p} \PYG{o}{\PYGZhy{}} \PYG{n}{c}\PYG{p}{)} \PYG{p}{)}\PYG{o}{*}\PYG{o}{*}\PYG{l+m+mf}{0.5}  \PYG{c+c1}{\PYGZsh{} área}
        
        \PYG{k}{return} \PYG{n}{A}
    
    \PYG{k}{def} \PYG{n+nf}{imprime\PYGZus{}area\PYGZus{}triangulo}\PYG{p}{(}\PYG{n}{P1}\PYG{p}{,}\PYG{n}{P2}\PYG{p}{,}\PYG{n}{P3}\PYG{p}{)}\PYG{p}{:}
        
        \PYG{n}{txt} \PYG{o}{=} \PYG{l+s+s1}{\PYGZsq{}}\PYG{l+s+s1}{Area do triângulo P1 = (}\PYG{l+s+si}{\PYGZob{}0\PYGZcb{}}\PYG{l+s+s1}{.}\PYG{l+s+si}{\PYGZob{}1\PYGZcb{}}\PYG{l+s+s1}{); P2 = (}\PYG{l+s+si}{\PYGZob{}2\PYGZcb{}}\PYG{l+s+s1}{.}\PYG{l+s+si}{\PYGZob{}3\PYGZcb{}}\PYG{l+s+s1}{); P3 = (}\PYG{l+s+si}{\PYGZob{}4\PYGZcb{}}\PYG{l+s+s1}{.}\PYG{l+s+si}{\PYGZob{}5\PYGZcb{}}\PYG{l+s+s1}{) :: A = }\PYG{l+s+si}{\PYGZob{}6\PYGZcb{}}\PYG{l+s+s1}{\PYGZsq{}}        
        \PYG{n}{area} \PYG{o}{=} \PYG{n}{Ponto}\PYG{o}{.}\PYG{n}{area\PYGZus{}heron}\PYG{p}{(}\PYG{n}{P1}\PYG{p}{,}\PYG{n}{P2}\PYG{p}{,}\PYG{n}{P3}\PYG{p}{)}
        
        \PYG{n+nb}{print}\PYG{p}{(}\PYG{n}{txt}\PYG{o}{.}\PYG{n}{format}\PYG{p}{(}\PYG{n}{P1}\PYG{o}{.}\PYG{n}{x}\PYG{p}{,}\PYG{n}{P1}\PYG{o}{.}\PYG{n}{y}\PYG{p}{,}\PYG{n}{P2}\PYG{o}{.}\PYG{n}{x}\PYG{p}{,}\PYG{n}{P2}\PYG{o}{.}\PYG{n}{y}\PYG{p}{,}\PYG{n}{P3}\PYG{o}{.}\PYG{n}{x}\PYG{p}{,}\PYG{n}{P3}\PYG{o}{.}\PYG{n}{y}\PYG{p}{,}\PYG{n}{area}\PYG{p}{)}\PYG{p}{)}
    
\end{sphinxVerbatim}


\subsubsection{Exemplo: usando a classe \sphinxstyleliteralintitle{\sphinxupquote{ponto.py}} para calcular a área de um triângulo retângulo}
\label{\detokenize{aula-00-modelagem-programacao:exemplo-usando-a-classe-ponto-py-para-calcular-a-area-de-um-triangulo-retangulo}}
\begin{sphinxVerbatim}[commandchars=\\\{\}]
\PYG{c+c1}{\PYGZsh{} Cálculo da área para o triângulo }
\PYG{c+c1}{\PYGZsh{} P1 = (0,0);  P2 = (1,0);  P3 = (0,1)}

\PYG{n}{P1} \PYG{o}{=} \PYG{n}{Ponto}\PYG{p}{(}\PYG{l+m+mf}{0.0}\PYG{p}{,}\PYG{l+m+mf}{0.0}\PYG{p}{)}
\PYG{n}{P2} \PYG{o}{=} \PYG{n}{Ponto}\PYG{p}{(}\PYG{l+m+mf}{1.0}\PYG{p}{,}\PYG{l+m+mf}{0.0}\PYG{p}{)}
\PYG{n}{P3} \PYG{o}{=} \PYG{n}{Ponto}\PYG{p}{(}\PYG{l+m+mf}{0.0}\PYG{p}{,}\PYG{l+m+mf}{1.0}\PYG{p}{)}
\PYG{n}{Ponto}\PYG{o}{.}\PYG{n}{imprime\PYGZus{}area\PYGZus{}triangulo}\PYG{p}{(}\PYG{n}{P1}\PYG{p}{,}\PYG{n}{P2}\PYG{p}{,}\PYG{n}{P3}\PYG{p}{)}
\end{sphinxVerbatim}

\begin{sphinxVerbatim}[commandchars=\\\{\}]
Area do triângulo P1 = (0.0.0.0); P2 = (1.0.0.0); P3 = (0.0.1.0) :: A = 0.49999999999999983
\end{sphinxVerbatim}

\begin{sphinxVerbatim}[commandchars=\\\{\}]
\PYG{c+c1}{\PYGZsh{} plotagem do triângulo}
\PYG{n}{P} \PYG{o}{=} \PYG{n}{np}\PYG{o}{.}\PYG{n}{array}\PYG{p}{(}\PYG{p}{[}\PYG{p}{[}\PYG{n}{P1}\PYG{o}{.}\PYG{n}{x}\PYG{p}{,}\PYG{n}{P1}\PYG{o}{.}\PYG{n}{y}\PYG{p}{]}\PYG{p}{,}\PYG{p}{[}\PYG{n}{P2}\PYG{o}{.}\PYG{n}{x}\PYG{p}{,}\PYG{n}{P2}\PYG{o}{.}\PYG{n}{y}\PYG{p}{]}\PYG{p}{,}\PYG{p}{[}\PYG{n}{P3}\PYG{o}{.}\PYG{n}{x}\PYG{p}{,}\PYG{n}{P3}\PYG{o}{.}\PYG{n}{y}\PYG{p}{]}\PYG{p}{]}\PYG{p}{)}
\PYG{n}{plt}\PYG{o}{.}\PYG{n}{figure}\PYG{p}{(}\PYG{p}{)}
\PYG{n}{pol} \PYG{o}{=} \PYG{n}{plt}\PYG{o}{.}\PYG{n}{Polygon}\PYG{p}{(}\PYG{n}{P}\PYG{p}{,}\PYG{n}{color}\PYG{o}{=}\PYG{l+s+s1}{\PYGZsq{}}\PYG{l+s+s1}{red}\PYG{l+s+s1}{\PYGZsq{}}\PYG{p}{,}\PYG{n}{alpha}\PYG{o}{=}\PYG{l+m+mf}{0.4}\PYG{p}{)}
\PYG{n}{plt}\PYG{o}{.}\PYG{n}{gca}\PYG{p}{(}\PYG{p}{)}\PYG{o}{.}\PYG{n}{add\PYGZus{}patch}\PYG{p}{(}\PYG{n}{pol}\PYG{p}{)}\PYG{p}{;}
\end{sphinxVerbatim}

\noindent\sphinxincludegraphics{{aula-00-modelagem-programacao_13_0}.png}


\subsubsection{Exemplo: usando a classe \sphinxstyleliteralintitle{\sphinxupquote{ponto.py}} para calcular a área de um triângulo qualquer}
\label{\detokenize{aula-00-modelagem-programacao:exemplo-usando-a-classe-ponto-py-para-calcular-a-area-de-um-triangulo-qualquer}}
\begin{sphinxVerbatim}[commandchars=\\\{\}]
\PYG{c+c1}{\PYGZsh{} Cálculo da área para o triângulo }
\PYG{c+c1}{\PYGZsh{} P1 = (4,2);  P2 = (3,2);  P3 = (2,\PYGZhy{}3)}

\PYG{n}{P4} \PYG{o}{=} \PYG{n}{Ponto}\PYG{p}{(}\PYG{l+m+mf}{4.0}\PYG{p}{,}\PYG{l+m+mf}{2.0}\PYG{p}{)}
\PYG{n}{P5} \PYG{o}{=} \PYG{n}{Ponto}\PYG{p}{(}\PYG{l+m+mf}{1.5}\PYG{p}{,}\PYG{l+m+mf}{1.5}\PYG{p}{)}
\PYG{n}{P6} \PYG{o}{=} \PYG{n}{Ponto}\PYG{p}{(}\PYG{l+m+mf}{2.0}\PYG{p}{,}\PYG{o}{\PYGZhy{}}\PYG{l+m+mf}{3.0}\PYG{p}{)}
\PYG{n}{Ponto}\PYG{o}{.}\PYG{n}{imprime\PYGZus{}area\PYGZus{}triangulo}\PYG{p}{(}\PYG{n}{P4}\PYG{p}{,}\PYG{n}{P5}\PYG{p}{,}\PYG{n}{P6}\PYG{p}{)}

\PYG{n}{plt}\PYG{o}{.}\PYG{n}{figure}\PYG{p}{(}\PYG{p}{)}
\PYG{n}{plt}\PYG{o}{.}\PYG{n}{scatter}\PYG{p}{(}\PYG{n}{P4}\PYG{o}{.}\PYG{n}{x}\PYG{p}{,}\PYG{n}{P4}\PYG{o}{.}\PYG{n}{y}\PYG{p}{,}\PYG{n}{color}\PYG{o}{=}\PYG{l+s+s1}{\PYGZsq{}}\PYG{l+s+s1}{blue}\PYG{l+s+s1}{\PYGZsq{}}\PYG{p}{)}
\PYG{n}{plt}\PYG{o}{.}\PYG{n}{scatter}\PYG{p}{(}\PYG{n}{P5}\PYG{o}{.}\PYG{n}{x}\PYG{p}{,}\PYG{n}{P5}\PYG{o}{.}\PYG{n}{y}\PYG{p}{,}\PYG{n}{color}\PYG{o}{=}\PYG{l+s+s1}{\PYGZsq{}}\PYG{l+s+s1}{blue}\PYG{l+s+s1}{\PYGZsq{}}\PYG{p}{)}
\PYG{n}{plt}\PYG{o}{.}\PYG{n}{scatter}\PYG{p}{(}\PYG{n}{P6}\PYG{o}{.}\PYG{n}{x}\PYG{p}{,}\PYG{n}{P6}\PYG{o}{.}\PYG{n}{y}\PYG{p}{,}\PYG{n}{color}\PYG{o}{=}\PYG{l+s+s1}{\PYGZsq{}}\PYG{l+s+s1}{blue}\PYG{l+s+s1}{\PYGZsq{}}\PYG{p}{)}

\PYG{n}{Px} \PYG{o}{=} \PYG{p}{[}\PYG{n}{P4}\PYG{o}{.}\PYG{n}{x}\PYG{p}{,}\PYG{n}{P5}\PYG{o}{.}\PYG{n}{x}\PYG{p}{,}\PYG{n}{P6}\PYG{o}{.}\PYG{n}{x}\PYG{p}{]}
\PYG{n}{Py} \PYG{o}{=} \PYG{p}{[}\PYG{n}{P4}\PYG{o}{.}\PYG{n}{y}\PYG{p}{,}\PYG{n}{P5}\PYG{o}{.}\PYG{n}{y}\PYG{p}{,}\PYG{n}{P6}\PYG{o}{.}\PYG{n}{y}\PYG{p}{]}
\PYG{n}{plt}\PYG{o}{.}\PYG{n}{fill}\PYG{p}{(}\PYG{n}{Px}\PYG{p}{,}\PYG{n}{Py}\PYG{p}{,}\PYG{n}{color}\PYG{o}{=}\PYG{l+s+s1}{\PYGZsq{}}\PYG{l+s+s1}{red}\PYG{l+s+s1}{\PYGZsq{}}\PYG{p}{,}\PYG{n}{alpha}\PYG{o}{=}\PYG{l+m+mf}{0.4}\PYG{p}{)}\PYG{p}{;}
\end{sphinxVerbatim}

\begin{sphinxVerbatim}[commandchars=\\\{\}]
Area do triângulo P1 = (4.0.2.0); P2 = (1.5.1.5); P3 = (2.0.\PYGZhy{}3.0) :: A = 5.75
\end{sphinxVerbatim}

\noindent\sphinxincludegraphics{{aula-00-modelagem-programacao_15_1}.png}


\section{O quarto paradigma da ciência}
\label{\detokenize{aula-00-modelagem-programacao:o-quarto-paradigma-da-ciencia}}
\sphinxAtStartPar
\sphinxstylestrong{Nota:} o texto desta seção pode ser encontrado em sua forma expandida neste \sphinxhref{https://tril.ci.ufpb.br/2021/10/16/engenharia-computacional-para-um-mundo-centrado-em-dados/}{link}.

\sphinxAtStartPar
Durante uma palestra proferida na Califórnia em 2009 para o \sphinxstyleemphasis{Computer Science and Telecommunications Board – National Research Council (NRC\sphinxhyphen{} CSTB)}, o renomado cientista da computação Jerry Nicholas “Jim” Gray (1944 – 2012), pontuou o surgimento do quarto paradigma da ciência. Ao utilizar o termo “eScience”, Gray queria dizer que a exploração científica seria grandemente influenciada pelo uso intensivo dos dados nos anos vindouros.

\sphinxAtStartPar
A evolução dos supercomputadores nesses últimos vinte anos elevou a capacidade de compreensão da natureza e permitiu que a ciência subisse mais um degrau na escada do conhecimento. A fim de entender o que é o quarto paradigma da ciência, vejamos quais são os três primeiros retrocedendo no tempo.

\sphinxAtStartPar
Há alguns milhares de anos, a ciência era essencialmente \sphinxstylestrong{empírica}. Tentava\sphinxhyphen{}se compreender os fenômenos naturais pela observação. Há algumas centenas de anos, equações, modelos e generalizações possibilitaram que a ciência se tornasse \sphinxstylestrong{teórica}, ou seja, passou\sphinxhyphen{}se a descrever com clareza e tecnicidade como um certo fenômeno funcionava. De algumas décadas para cá, as simulações de alta complexidade revelaram o terceiro pilar da ciência, tornando\sphinxhyphen{}a \sphinxstylestrong{computacional}. Atualmente, o amálgama entre teoria, experimentação e computação gerou uma vertente de \sphinxstylestrong{exploração dos dados}. Em outras palavras, poderíamos dizer que a ciência agora possui uma quarta faceta. Ela se tornou “datificada”.

\sphinxAtStartPar
Os dados são exploráveis porque um dia puderam ser capturados, mensurados, processados e simulados. Diante disso, a ciência atual é:
\begin{enumerate}
\sphinxsetlistlabels{\arabic}{enumi}{enumii}{}{.}%
\item {} 
\sphinxAtStartPar
Experimental;

\item {} 
\sphinxAtStartPar
Teórica;

\item {} 
\sphinxAtStartPar
Computacional e

\item {} 
\sphinxAtStartPar
Datificada,

\end{enumerate}

\sphinxAtStartPar
sendo o último paradigma a consumação dos três anteriores.


\subsection{Ciência e Engenharia Computacional}
\label{\detokenize{aula-00-modelagem-programacao:ciencia-e-engenharia-computacional}}
\sphinxAtStartPar
A \sphinxstylestrong{Ciência e Engenharia Computacional}, internacionalmente conhecida pelo acrônimo CSE (\sphinxstyleemphasis{Computational Science and Engineering}) é uma área interdisciplinar que compreende a ciência da computação, matemática aplicada, biologia e outras áreas do núcleo STEM (\sphinxstyleemphasis{Science, Technology, Engineering, and Mathematics}) voltada à resolução de problemas práticos das engenharias. A característica peculiar da CSE é o uso de métodos numéricos e sua integração com modelos matemáticos para subsidiar simulações computacionais e a resolução de equações diferenciais ordinárias ou parciais.

\sphinxAtStartPar
Equações diferenciais descrevem uma ampla variedade de fenômenos físicos, desde a absorção de um fármaco pelo organismo humano em escala nanométrica, até as ondas de choque macroscópicas causadas pela explosão de uma dinamite. Estes são apenas dois exemplos de situações que interessariam a indústria de biotecnologia e de construção ocorrendo em escalas extremamente distintas. Portanto, a CSE é afeita a problemas do mundo real caracterizados por multi\sphinxhyphen{}escalas pelos quais se compreende a dinâmica intrínseca de um fenômeno complexo modelável analiticamente.


\subsection{Por que a CSE importa em um mundo de dados?}
\label{\detokenize{aula-00-modelagem-programacao:por-que-a-cse-importa-em-um-mundo-de-dados}}
\sphinxAtStartPar
Embora em muitas áreas já se saiba com riqueza de detalhes como mecanismos intrínsecos de sistemas dinâmicos funcionam, em outras, este não é o caso. A indisponibilidade de dados em certos domínios do conhecimento decorre, principalmente, da dificuldade de capturá\sphinxhyphen{}los e coletá\sphinxhyphen{}los, seja pela inexistência de infraestrutura tecnológica, seja pelo alto investimento necessário para obtê\sphinxhyphen{}los. Um exemplo é a exploração da subsuperfície terreste. Dominar o conhecimento acerca da mecânica das rochas, a salinidade de aquíferos, o potencial geotérmico, ou a constituição química do gás natural nas partes mais baixas do planeta equivale a adentrar dezenas, centenas ou milhares de quilômetros na litosfera. Essa não é uma tarefa fácil. Pelo contrário, há tantas variáveis, riscos e custos envolvidos que sua execução pode ser inviabilizada.

\sphinxAtStartPar
A CSE entra em cena para preencher lacunas que os “dados”, por si só, não conseguem fechar. Uma vez que a exploração de dados só é exequível quando os próprios dados estão disponíveis, seria impossível explorar dados acerca de algo desconhecido. Claramente, seria paradoxal. Para dar outra ilustração, tomemos o exemplo das imagens digitais. A tecnologia atual provê condições suficientes para tirarmos fotos de milhares de objetos, seres e indivíduos (p.ex. uma formiga, um prédio, um planeta) sem qualquer dificuldade.

\sphinxAtStartPar
Processar imagens, hoje em dia, é um dos grandes carros\sphinxhyphen{}chefe para que modelos de inteligência artificial sejam implementados com precisão. Mas a manipulação de imagens só é possível por causa do progresso na física, óptica álgebra linear – afinal, imagens são matrizes – e ciência dos materiais – para construir carcaças de celulares, câmeras DSLR e satélites.

\sphinxAtStartPar
Aqui, cabe uma pergunta: será que no processos de fabricação desses componentes materiais, ninguém apelou para simulações computacionais? É bem improvável, porque nenhum gestor aprovaria o envio de um satélite responsável por fotografar o oceano Atlântico para o espaço sem ter uma margem aceitável de confiabilidade de que ele orbitaria corretamente e cumpriria seu propósito com segurança. E como se adquire essa confiabilidade? Vai\sphinxhyphen{}se para um laboratório, cria\sphinxhyphen{}se um protótipo e faz\sphinxhyphen{}se 1, 2, 10, 1.000, 100.000 simulações, até que se reduzam ao máximo as incertezas.

\sphinxAtStartPar
Estando em algum lugar entre o terceiro e o quarto paradigma, a CSE não apenas intermedia a análise de incertezas, como também simula processos, assim gerando economia de recursos. A CSE usa o primeiro, o segundo e o terceiro na transição para o quarto paradigma e, naturalmente, agrega valor a processos de um mundo que vive em transição tecnológica.


\subsection{Engenharia centrada em dados: um novo ramo na árvore do futuro?}
\label{\detokenize{aula-00-modelagem-programacao:engenharia-centrada-em-dados-um-novo-ramo-na-arvore-do-futuro}}
\sphinxAtStartPar
Mark Girolami, um professor de Engenharia Civil em Cambridge, liderou um grande programa de ciência de dados e inteligência artificial no \sphinxhref{https://turing.ac.uk/}{Alan Turing Institute} entre os anos de 2017 e 2020. Em sua percepção, a ciência de dados já havia se impregnado nas engenharias de tal forma que a inauguração de um novo termo para descrever essa interseção crescente seria justificável. Ele o chamou de engenharia centrada em dados (data\sphinxhyphen{}centric engineering, ou DCE).

\sphinxAtStartPar
Segundo Girolami, a DCE é explicada por um desenvolvimento substancial que impacta as engenharias, profissões associadas, suas práticas e também a política. Ao relembrar que a engenharia erigiu\sphinxhyphen{}se sobre dados desde seus primórdios, citou em seu artigo uma clássica fala de Lord Kelvin de 1889:
\begin{quote}

\sphinxAtStartPar
“Quando você pode medir o que está falando e expressar isso em números, você sabe alguma coisa sobre aquilo; quando não o pode expressar em números, seu conhecimento é escasso e insatisfatório; pode ser o princípio do conhecimento, mas você, em seus pensamentos, empurrou parcamente a fronteira da ciência.” {[}tradução livre{]}
\end{quote}

\sphinxAtStartPar
Em suma, “medir é saber”. Girolami então pontuou que dados derivados da observação e medição experimentais foram os responsáveis por conduzir o desenvolvimento da filosofia natural e impulsionar o estado\sphinxhyphen{}da\sphinxhyphen{}técnica da engenharia por todo o século XIX. Em suas palavras,
\begin{quote}

\sphinxAtStartPar
“os dados sempre estiveram no coração da ciência e da prática na engenharia”.
\end{quote}

\sphinxAtStartPar
Com a irrupção da DCE, uma pergunta que se faz é: seria a DCE um novo ramo na árvore do futuro da ciência de dados? Enquanto uma resposta objetiva é aguardada, poderíamos intuir, com base na opinião de vários experts que se reuniram no DCEng Summit, realizado no último setembro, em Londres, que as engenharias não serão mais as mesmas daqui para a frente. É consensual que o big data abriu enormes oportunidades para praticamente todas as áreas da engenharia – Aeronáutica, Civil, Mecânica, Offshore, entre outras – haja vista o nível de detalhamento provido por muitos bancos de dados quanto no que tange à compreensão de variados fenômenos que foram observados e medidos experimentalmente ao longo de décadas. Entretanto, uma gama de desafios acompanha essa evolução furtivamente. Ética e privacidade na gestão de dados, lentidão da difusão tecnológica em países de baixa renda e carência de profissionais qualificados e currículos contemporâneos são alguns deles.

\sphinxAtStartPar
Ley et al., emitindo pontos de vista sobre como a DCE se projeta em suas áreas de atuação – estatística, engenharia e desenvolvimento de software –, concluíram que:
\begin{enumerate}
\sphinxsetlistlabels{\arabic}{enumi}{enumii}{}{.}%
\item {} 
\sphinxAtStartPar
o pensamento centrado em dados tornou\sphinxhyphen{}se necessário em vários domínios do conhecimento e a riqueza por eles disponibilizada acelerará a pesquisa no âmbito da engenharia de maneira imensurável;

\item {} 
\sphinxAtStartPar
dados solitários não valem muita coisa e podem transmitir mensagens equivocadas se não forem analisados com cuidado e geridos de forma segura;

\item {} 
\sphinxAtStartPar
pelo fato de a educação baseada em dados ser uma habilidade indispensável para a formação de futuros engenheiros, as universidades, empresas e gestores devem se mobilizar para assegurar um currículo interdisciplinar que forme profissionais com “mente aberta” e explore habilidades flexíveis capazes de lidar com dados.

\end{enumerate}

\sphinxAtStartPar
Na Engenharia Mecânica, em particular, o conceito de \sphinxstylestrong{ciência de dados mecanicista} (\sphinxstyleemphasis{mechanistic data science}, MDS) e sua incorporação na educação de engenheiros, bem como de estudantes de nível médio nos Estados Unidos foi recentemente debatida no \sphinxhref{http://16.usnccm.org/SC16-002}{16th USCCM} e na conferência \sphinxhref{https://mmldt.eng.ucsd.edu/short-course}{MMLDT\sphinxhyphen{}CSET 2021}. A proposta da MDS é explanada, por exemplo, nesta \sphinxhref{https://www.imagwiki.nibib.nih.gov/sites/default/files/liu\_pdf\_usnccm2019\_lecture1.pdf}{apresentação} do Prof. Wing Kam Liu da Northwestern University.


\subsection{Engenharia computacional no enfrentamento da Covid\sphinxhyphen{}19}
\label{\detokenize{aula-00-modelagem-programacao:engenharia-computacional-no-enfrentamento-da-covid-19}}
\sphinxAtStartPar
A pandemia da Covid\sphinxhyphen{}19 desencadeou enormes desafios para a comunidade global. Concomitantemente, pesquisadores de diversas áreas mobilizaram\sphinxhyphen{}se para apresentar estratégias de enfrentamento à doença e propostas para mitigação dos riscos de contaminação do vírus SARS\sphinxhyphen{}CoV\sphinxhyphen{}2. A CSE não ficou de fora. A dinâmica dos fluidos computacional (computational fluid dynamics, CFD) contribuiu magnificamente para a elaboração de protocolos, equipamentos de proteção individual e coletiva, bem como para controle de engenharia em ambientes hospitalares. A seguir, parafraseamos três objetivos reportados pela Siemens Digital Industries Software em estudos de caso reais guiados por CFD:

\sphinxAtStartPar
entender a dinâmica espaço\sphinxhyphen{}temporal de partículas virais exaladas por seres humanos na forma de gotículas e aerossóis;
aperfeiçoar o projeto de sistemas de ventilação interior, bem como de exaustão e filtração visando conforto e segurança hospitalar;
projetar equipamentos respiratórios, dispositivos para esterilização, purificação e produção de vacinas.


\section{Considerações finais}
\label{\detokenize{aula-00-modelagem-programacao:consideracoes-finais}}
\sphinxAtStartPar
A datificação está se consolidando como o quarto paradigma da ciência e a engenharia será cada vez mais influenciada por uma cultura baseada em dados. No \DUrole{xref,myst}{TRIL Lab}, defendemos uma formação estratégica e trabalhamos para que nossos alunos e colaboradores se adaptem a um cenário que demanda cada vez mais habilidades flexíveis e interdisciplinaridade. Em um mundo centrado em dados, a engenharia computacional terá um leque incomensurável de oportunidades, compreendendo, modelando e resolvendo problemas de engenharia do mundo real.

\sphinxAtStartPar
O conteúdo deste livro tem o objetivo de proporcionar a nossos estudantes uma formação moderna e ampla em métodos numéricos com aplicações às ciências computacionais e engenharias. Entretanto, qualquer perfil profissional que se beneficie da computação científica terá nele um estímulo adicional para enveredar\sphinxhyphen{}se pelo frutífero e versátil universo das técnicas fundamentais da engenharia computacional.


\chapter{Conversão numérica e ponto flutuante}
\label{\detokenize{aula-01-ponto-flutuante:conversao-numerica-e-ponto-flutuante}}\label{\detokenize{aula-01-ponto-flutuante::doc}}
\begin{sphinxVerbatim}[commandchars=\\\{\}]
\PYG{o}{\PYGZpc{}}\PYG{k}{matplotlib} inline
\end{sphinxVerbatim}


\section{Sistema binário}
\label{\detokenize{aula-01-ponto-flutuante:sistema-binario}}
\sphinxAtStartPar
Simples exercícios de conversão numérica para introduzi\sphinxhyphen{}lo à computação numérica com Python.


\subsection{Exercícios de conversão numérica}
\label{\detokenize{aula-01-ponto-flutuante:exercicios-de-conversao-numerica}}
\begin{sphinxVerbatim}[commandchars=\\\{\}]
\PYG{c+c1}{\PYGZsh{} (100)\PYGZus{}2 \PYGZhy{}\PYGZgt{} base 10}
\PYG{n}{c} \PYG{o}{=} \PYG{n+nb}{int}\PYG{p}{(}\PYG{l+s+s1}{\PYGZsq{}}\PYG{l+s+s1}{100}\PYG{l+s+s1}{\PYGZsq{}}\PYG{p}{,}\PYG{n}{base}\PYG{o}{=}\PYG{l+m+mi}{2}\PYG{p}{)}
\PYG{n+nb}{print}\PYG{p}{(}\PYG{n}{c}\PYG{p}{)}

\PYG{c+c1}{\PYGZsh{} representação  }
\PYG{n+nb}{print}\PYG{p}{(}\PYG{l+m+mi}{1}\PYG{o}{*}\PYG{l+m+mi}{2}\PYG{o}{*}\PYG{o}{*}\PYG{l+m+mi}{2} \PYG{o}{+} \PYG{l+m+mi}{0}\PYG{o}{*}\PYG{l+m+mi}{2}\PYG{o}{*}\PYG{o}{*}\PYG{l+m+mi}{1} \PYG{o}{+} \PYG{l+m+mi}{0}\PYG{o}{*}\PYG{l+m+mi}{2}\PYG{o}{*}\PYG{o}{*}\PYG{l+m+mi}{0}\PYG{p}{)}

\PYG{c+c1}{\PYGZsh{} (4)\PYGZus{}10 \PYGZhy{}\PYGZgt{} base 2}
\PYG{c+c1}{\PYGZsh{} obs: note que \PYGZsq{}0b\PYGZsq{} indica que o número é binário}
\PYG{n}{c} \PYG{o}{=} \PYG{n+nb}{bin}\PYG{p}{(}\PYG{l+m+mi}{4}\PYG{p}{)}
\PYG{n+nb}{print}\PYG{p}{(}\PYG{n}{c}\PYG{p}{)}
\end{sphinxVerbatim}

\begin{sphinxVerbatim}[commandchars=\\\{\}]
4
4
0b100
\end{sphinxVerbatim}

\begin{sphinxVerbatim}[commandchars=\\\{\}]
\PYG{c+c1}{\PYGZsh{} (222)\PYGZus{}8}
\PYG{n}{c} \PYG{o}{=} \PYG{n+nb}{int}\PYG{p}{(}\PYG{l+s+s1}{\PYGZsq{}}\PYG{l+s+s1}{222}\PYG{l+s+s1}{\PYGZsq{}}\PYG{p}{,}\PYG{n}{base}\PYG{o}{=}\PYG{l+m+mi}{8}\PYG{p}{)}
\PYG{n+nb}{print}\PYG{p}{(}\PYG{n}{c}\PYG{p}{)}

\PYG{c+c1}{\PYGZsh{} representação  }
\PYG{n+nb}{print}\PYG{p}{(}\PYG{l+m+mi}{2}\PYG{o}{*}\PYG{l+m+mi}{8}\PYG{o}{*}\PYG{o}{*}\PYG{l+m+mi}{2} \PYG{o}{+} \PYG{l+m+mi}{2}\PYG{o}{*}\PYG{l+m+mi}{8}\PYG{o}{*}\PYG{o}{*}\PYG{l+m+mi}{1} \PYG{o}{+} \PYG{l+m+mi}{2}\PYG{o}{*}\PYG{l+m+mi}{8}\PYG{o}{*}\PYG{o}{*}\PYG{l+m+mi}{0}\PYG{p}{)}

\PYG{c+c1}{\PYGZsh{} (146)\PYGZus{}10 \PYGZhy{}\PYGZgt{} base 8}

\PYG{n}{c} \PYG{o}{=} \PYG{n+nb}{oct}\PYG{p}{(}\PYG{l+m+mi}{146}\PYG{p}{)}
\PYG{c+c1}{\PYGZsh{} obs: note que \PYGZsq{}0o\PYGZsq{} indica que o número é octal}
\PYG{n+nb}{print}\PYG{p}{(}\PYG{n}{c}\PYG{p}{)}
\end{sphinxVerbatim}

\begin{sphinxVerbatim}[commandchars=\\\{\}]
146
146
0o222
\end{sphinxVerbatim}

\begin{sphinxVerbatim}[commandchars=\\\{\}]
\PYG{c+c1}{\PYGZsh{} (2AE4)\PYGZus{}16}
\PYG{n}{c} \PYG{o}{=} \PYG{n+nb}{int}\PYG{p}{(}\PYG{l+s+s1}{\PYGZsq{}}\PYG{l+s+s1}{2ae4}\PYG{l+s+s1}{\PYGZsq{}}\PYG{p}{,}\PYG{n}{base}\PYG{o}{=}\PYG{l+m+mi}{16}\PYG{p}{)}
\PYG{n+nb}{print}\PYG{p}{(}\PYG{n}{c}\PYG{p}{)}

\PYG{c+c1}{\PYGZsh{} representação  }
\PYG{c+c1}{\PYGZsh{} obs: A = 10; E = 14}
\PYG{n+nb}{print}\PYG{p}{(}\PYG{l+m+mi}{2}\PYG{o}{*}\PYG{l+m+mi}{16}\PYG{o}{*}\PYG{o}{*}\PYG{l+m+mi}{3} \PYG{o}{+} \PYG{l+m+mi}{10}\PYG{o}{*}\PYG{l+m+mi}{16}\PYG{o}{*}\PYG{o}{*}\PYG{l+m+mi}{2} \PYG{o}{+} \PYG{l+m+mi}{14}\PYG{o}{*}\PYG{l+m+mi}{16}\PYG{o}{*}\PYG{o}{*}\PYG{l+m+mi}{1} \PYG{o}{+} \PYG{l+m+mi}{4}\PYG{o}{*}\PYG{l+m+mi}{166}\PYG{o}{*}\PYG{o}{*}\PYG{l+m+mi}{0}\PYG{p}{)}

\PYG{c+c1}{\PYGZsh{} (146)\PYGZus{}10 \PYGZhy{}\PYGZgt{} base 8}

\PYG{n}{c} \PYG{o}{=} \PYG{n+nb}{oct}\PYG{p}{(}\PYG{l+m+mi}{146}\PYG{p}{)}
\PYG{c+c1}{\PYGZsh{} obs: note que \PYGZsq{}0o\PYGZsq{} indica que o número é octal}
\PYG{n+nb}{print}\PYG{p}{(}\PYG{n}{c}\PYG{p}{)}
\end{sphinxVerbatim}

\begin{sphinxVerbatim}[commandchars=\\\{\}]
10980
10980
0o222
\end{sphinxVerbatim}

\begin{sphinxVerbatim}[commandchars=\\\{\}]
\PYG{l+s+s2}{\PYGZdq{}}\PYG{l+s+s2}{Brincando com Python e divisões sucessivas}\PYG{l+s+s2}{\PYGZdq{}}

\PYG{n+nb}{print}\PYG{p}{(}\PYG{l+s+s1}{\PYGZsq{}}\PYG{l+s+s1}{Esquema de divisões sucessivas:}\PYG{l+s+se}{\PYGZbs{}n}\PYG{l+s+s1}{\PYGZsq{}}\PYG{p}{)}

\PYG{n+nb}{print}\PYG{p}{(}\PYG{n+nb}{str}\PYG{p}{(}\PYG{l+m+mi}{4}\PYG{p}{)} \PYG{o}{+} \PYG{l+s+s1}{\PYGZsq{}}\PYG{l+s+s1}{ | 2}\PYG{l+s+s1}{\PYGZsq{}}\PYG{p}{)}
\PYG{n+nb}{print}\PYG{p}{(} \PYG{n+nb}{str}\PYG{p}{(} \PYG{n+nb}{len}\PYG{p}{(}\PYG{n+nb}{str}\PYG{p}{(}\PYG{l+m+mi}{4}\PYG{p}{)}\PYG{p}{)}\PYG{o}{*}\PYG{l+s+s1}{\PYGZsq{}}\PYG{l+s+s1}{ }\PYG{l+s+s1}{\PYGZsq{}}\PYG{p}{)} \PYG{o}{+} \PYG{l+s+s1}{\PYGZsq{}}\PYG{l+s+s1}{  –––}\PYG{l+s+s1}{\PYGZsq{}}\PYG{p}{)}
\PYG{n+nb}{print}\PYG{p}{(} \PYG{n+nb}{str}\PYG{p}{(} \PYG{l+m+mi}{4} \PYG{o}{\PYGZpc{}} \PYG{l+m+mi}{2}\PYG{p}{)} \PYG{o}{+} \PYG{l+s+s1}{\PYGZsq{}}\PYG{l+s+s1}{   }\PYG{l+s+s1}{\PYGZsq{}} \PYG{o}{+} \PYG{n+nb}{str}\PYG{p}{(}\PYG{l+m+mi}{4} \PYG{o}{/}\PYG{o}{/} \PYG{l+m+mi}{2}\PYG{p}{)} \PYG{o}{+} \PYG{l+s+s1}{\PYGZsq{}}\PYG{l+s+s1}{ | 2}\PYG{l+s+s1}{\PYGZsq{}} \PYG{p}{)}
\PYG{n+nb}{print}\PYG{p}{(} \PYG{n+nb}{str}\PYG{p}{(} \PYG{l+m+mi}{5}\PYG{o}{*}\PYG{n+nb}{len}\PYG{p}{(}\PYG{n+nb}{str}\PYG{p}{(}\PYG{l+m+mi}{4}\PYG{p}{)}\PYG{p}{)}\PYG{o}{*}\PYG{l+s+s1}{\PYGZsq{}}\PYG{l+s+s1}{ }\PYG{l+s+s1}{\PYGZsq{}}\PYG{p}{)} \PYG{o}{+} \PYG{l+s+s1}{\PYGZsq{}}\PYG{l+s+s1}{  –––}\PYG{l+s+s1}{\PYGZsq{}}\PYG{p}{)}
\PYG{n+nb}{print}\PYG{p}{(} \PYG{n+nb}{str}\PYG{p}{(} \PYG{l+m+mi}{4}\PYG{o}{*}\PYG{n+nb}{len}\PYG{p}{(}\PYG{n+nb}{str}\PYG{p}{(}\PYG{l+m+mi}{4}\PYG{p}{)}\PYG{p}{)}\PYG{o}{*}\PYG{l+s+s1}{\PYGZsq{}}\PYG{l+s+s1}{ }\PYG{l+s+s1}{\PYGZsq{}}\PYG{p}{)} \PYG{o}{+} \PYG{n+nb}{str}\PYG{p}{(}\PYG{l+m+mi}{4} \PYG{o}{\PYGZpc{}} \PYG{l+m+mi}{2} \PYG{o}{\PYGZpc{}} \PYG{l+m+mi}{2}\PYG{p}{)} \PYG{o}{+} \PYG{l+s+s1}{\PYGZsq{}}\PYG{l+s+s1}{   }\PYG{l+s+s1}{\PYGZsq{}} \PYG{o}{+} \PYG{n+nb}{str}\PYG{p}{(}\PYG{l+m+mi}{4} \PYG{o}{/}\PYG{o}{/} \PYG{l+m+mi}{2} \PYG{o}{/}\PYG{o}{/} \PYG{l+m+mi}{2}\PYG{p}{)}\PYG{p}{)}
\end{sphinxVerbatim}

\begin{sphinxVerbatim}[commandchars=\\\{\}]
Esquema de divisões sucessivas:

4 | 2
   –––
0   2 | 2
       –––
    0   1
\end{sphinxVerbatim}

\sphinxAtStartPar
\sphinxstylestrong{Exercício:} estude a codificação do esquema acima. O que os operadores \sphinxcode{\sphinxupquote{//}} e \sphinxcode{\sphinxupquote{\%}} estão fazendo?


\section{Máquina binária}
\label{\detokenize{aula-01-ponto-flutuante:maquina-binaria}}
\sphinxAtStartPar
O código abaixo é um protótipo para implementação de uma máquina binária. Uma versão muito mais robusta e melhor implementada pode ser vista aqui: \sphinxurl{https://vnicius.github.io/numbiosis/conversor/index.html}.

\begin{sphinxVerbatim}[commandchars=\\\{\}]
\PYG{l+s+sd}{\PYGZdq{}\PYGZdq{}\PYGZdq{}}
\PYG{l+s+sd}{Converte inteiro para binário}
\PYG{l+s+sd}{por divisões sucessivas.}
\PYG{l+s+sd}{! Confronte com a função residente \PYGZsq{}bin()\PYGZsq{}}
\PYG{l+s+sd}{\PYGZdq{}\PYGZdq{}\PYGZdq{}}


\PYG{k}{def} \PYG{n+nf}{int2bin}\PYG{p}{(}\PYG{n}{N}\PYG{p}{)}\PYG{p}{:}

    \PYG{n}{b} \PYG{o}{=} \PYG{p}{[}\PYG{p}{]} \PYG{c+c1}{\PYGZsh{} lista auxiliar}

    \PYG{c+c1}{\PYGZsh{} divisões sucessivas}
    \PYG{k}{while} \PYG{n}{N} \PYG{o}{\PYGZgt{}}\PYG{o}{=} \PYG{l+m+mi}{2}\PYG{p}{:}
        \PYG{n}{b}\PYG{o}{.}\PYG{n}{append}\PYG{p}{(}\PYG{n}{N} \PYG{o}{\PYGZpc{}} \PYG{l+m+mi}{2}\PYG{p}{)}
        \PYG{n}{N} \PYG{o}{=} \PYG{n}{N}\PYG{o}{/}\PYG{o}{/}\PYG{l+m+mi}{2}

    \PYG{n}{b}\PYG{o}{.}\PYG{n}{append}\PYG{p}{(}\PYG{n}{N}\PYG{p}{)}
    \PYG{n}{b}\PYG{o}{.}\PYG{n}{reverse}\PYG{p}{(}\PYG{p}{)}
    \PYG{n}{b} \PYG{o}{=} \PYG{p}{[}\PYG{n+nb}{str}\PYG{p}{(}\PYG{n}{i}\PYG{p}{)} \PYG{k}{for} \PYG{n}{i} \PYG{o+ow}{in} \PYG{n}{b}\PYG{p}{]} \PYG{c+c1}{\PYGZsh{} converte para string}
    \PYG{n}{s} \PYG{o}{=} \PYG{l+s+s1}{\PYGZsq{}}\PYG{l+s+s1}{\PYGZsq{}}
    \PYG{n}{s} \PYG{o}{=} \PYG{n}{s}\PYG{o}{.}\PYG{n}{join}\PYG{p}{(}\PYG{n}{b}\PYG{p}{)}

    \PYG{k}{return} \PYG{n}{s} \PYG{c+c1}{\PYGZsh{} retorna string}


\PYG{l+s+sd}{\PYGZdq{}\PYGZdq{}\PYGZdq{}}
\PYG{l+s+sd}{Converte parte fracionária para binário}
\PYG{l+s+sd}{por multiplicações sucessivas.}
\PYG{l+s+sd}{\PYGZdq{}\PYGZdq{}\PYGZdq{}}
\PYG{k}{def} \PYG{n+nf}{frac2bin}\PYG{p}{(}\PYG{n}{Q}\PYG{p}{)}\PYG{p}{:}

    \PYG{n}{count} \PYG{o}{=} \PYG{l+m+mi}{0} \PYG{c+c1}{\PYGZsh{} contador (limite manual posto em 10!)}
    \PYG{n}{b} \PYG{o}{=} \PYG{p}{[}\PYG{p}{]}  \PYG{c+c1}{\PYGZsh{} lista auxiliar}

    \PYG{c+c1}{\PYGZsh{} multiplicações sucessivas}
    \PYG{n}{Q} \PYG{o}{*}\PYG{o}{=} \PYG{l+m+mi}{2}
    \PYG{k}{while} \PYG{n}{Q} \PYG{o}{\PYGZgt{}} \PYG{l+m+mi}{0} \PYG{o+ow}{and} \PYG{n}{count} \PYG{o}{\PYGZlt{}}\PYG{o}{=} \PYG{l+m+mi}{10}\PYG{p}{:}
        \PYG{k}{if} \PYG{n}{Q} \PYG{o}{\PYGZgt{}} \PYG{l+m+mi}{1}\PYG{p}{:}
            \PYG{n}{Q} \PYG{o}{=} \PYG{n}{Q}\PYG{o}{\PYGZhy{}}\PYG{l+m+mi}{1}
            \PYG{n}{b}\PYG{o}{.}\PYG{n}{append}\PYG{p}{(}\PYG{l+m+mi}{1}\PYG{p}{)}
        \PYG{k}{else}\PYG{p}{:}
            \PYG{n}{b}\PYG{o}{.}\PYG{n}{append}\PYG{p}{(}\PYG{l+m+mi}{0}\PYG{p}{)}
        \PYG{n}{Q} \PYG{o}{*}\PYG{o}{=} \PYG{l+m+mi}{2}
        \PYG{n}{count} \PYG{o}{+}\PYG{o}{=} \PYG{l+m+mi}{1}

    \PYG{n}{b} \PYG{o}{=} \PYG{p}{[}\PYG{n+nb}{str}\PYG{p}{(}\PYG{n}{i}\PYG{p}{)} \PYG{k}{for} \PYG{n}{i} \PYG{o+ow}{in} \PYG{n}{b}\PYG{p}{]} \PYG{c+c1}{\PYGZsh{} converte para string}
    \PYG{n}{s} \PYG{o}{=} \PYG{l+s+s1}{\PYGZsq{}}\PYG{l+s+s1}{\PYGZsq{}}
    \PYG{n}{s} \PYG{o}{=} \PYG{n}{s}\PYG{o}{.}\PYG{n}{join}\PYG{p}{(}\PYG{n}{b}\PYG{p}{)}

    \PYG{k}{return} \PYG{n}{s} \PYG{c+c1}{\PYGZsh{} retorna string}


\PYG{k}{def} \PYG{n+nf}{convert}\PYG{p}{(}\PYG{n}{app}\PYG{p}{,}\PYG{n}{btn}\PYG{p}{)}\PYG{p}{:}
    \PYG{n+nb}{print}\PYG{p}{(}\PYG{n}{btn}\PYG{p}{)}



\PYG{c+c1}{\PYGZsh{} Função principal}
\PYG{k}{def} \PYG{n+nf}{main}\PYG{p}{(}\PYG{p}{)}\PYG{p}{:}

    \PYG{c+c1}{\PYGZsh{} Pré\PYGZhy{}criação da interface com usuário}

    \PYG{c+c1}{\PYGZsh{} todo: tratamento de exceção no tipo de entrada}
    \PYG{c+c1}{\PYGZsh{}       contagem de casas decimais no caso de dízimas}
         \PYG{n+nb}{print}\PYG{p}{(}\PYG{l+s+s1}{\PYGZsq{}}\PYG{l+s+s1}{*** MÁQUINA BINÁRIA ***}\PYG{l+s+s1}{\PYGZsq{}}\PYG{p}{)}
    \PYG{c+c1}{\PYGZsh{}     N = input(\PYGZsq{}Selecione a parte inteira:\PYGZbs{}n\PYGZsq{})}
    \PYG{c+c1}{\PYGZsh{}     Q = input(\PYGZsq{}Selecione a parte fracionária:\PYGZbs{}n\PYGZsq{})}
    \PYG{c+c1}{\PYGZsh{}     print(\PYGZsq{}Seu número é: \PYGZsq{} + int2bin( int(N) ) + \PYGZsq{}.\PYGZsq{} + frac2bin( float(Q) )  + \PYGZsq{}.\PYGZsq{})}
    \PYG{c+c1}{\PYGZsh{}     print(\PYGZsq{}*** ***\PYGZsq{})}


\PYG{k}{if} \PYG{n+nv+vm}{\PYGZus{}\PYGZus{}name\PYGZus{}\PYGZus{}} \PYG{o}{==} \PYG{l+s+s2}{\PYGZdq{}}\PYG{l+s+s2}{\PYGZus{}\PYGZus{}main\PYGZus{}\PYGZus{}}\PYG{l+s+s2}{\PYGZdq{}}\PYG{p}{:}
    \PYG{n}{main}\PYG{p}{(}\PYG{p}{)}
\end{sphinxVerbatim}

\begin{sphinxVerbatim}[commandchars=\\\{\}]
*** MÁQUINA BINÁRIA ***
\end{sphinxVerbatim}


\section{Sistema de ponto flutuante}
\label{\detokenize{aula-01-ponto-flutuante:sistema-de-ponto-flutuante}}

\subsection{A reta “perfurada”}
\label{\detokenize{aula-01-ponto-flutuante:a-reta-perfurada}}
\sphinxAtStartPar
Como temos estudado, a matemática computacional opera no domínio \(\mathbb{F}\), de pontos flutuantes, ao invés de trabalhar com números reais (conjunto \(\mathbb{R}\)). Vejamos um exemplo:

\sphinxAtStartPar
\sphinxstylestrong{Exemplo}: Considere o sistema de ponto flutuante \(\mathbb{F}(2,3,-1,2)\). Determinemos todos os seus números representáveis:

\sphinxAtStartPar
Como a base é \(2\), os dígitos possíveis são \(0\) e \(1\) com mantissas:
\begin{itemize}
\item {} 
\sphinxAtStartPar
\(0.100\)

\item {} 
\sphinxAtStartPar
\(0.101\)

\item {} 
\sphinxAtStartPar
\(0.110\)

\item {} 
\sphinxAtStartPar
\(0.111\)

\end{itemize}

\sphinxAtStartPar
Para cada expoente no conjunto \(e=\{-1,0,1,2\}\), obteremos 16 números positivos, a saber:
\begin{itemize}
\item {} 
\sphinxAtStartPar
\((0.100 \times 2^{-1})_{2} = (0.01)_2 = 0.2^0 + 0.2^{-1} + 1.2^{-2} = 1/4\)

\item {} 
\sphinxAtStartPar
\((0.100 \times 2^{0})_{2} = (0.1)_2 = 0.2^0 + 1.2^{-1} = 1/2\)

\item {} 
\sphinxAtStartPar
\((0.100 \times 2^{1})_{2} = (1.0)_2 = 1.2^0 + 0.2^{-1} = 1\)

\item {} 
\sphinxAtStartPar
\((0.100 \times 2^{2})_{2} = (10.0)_2 = 1.2^1 + 0.2^{1} + 0.2^{-1} = 2\)

\item {} 
\sphinxAtStartPar
\((0.101 \times 2^{-1})_{2} = (0.0101)_2 = 0.2^0 + 0.2^{-1} + 1.2^{-2} + 0.2^{-3} + 1.2^{-4}= 5/16\)

\item {} 
\sphinxAtStartPar
\((0.101 \times 2^{0})_{2} = (0.101)_2 = 0.2^0 + 1.2^{-1} + 0.2^{-2} + 1.2^{-3} = 5/8\)

\item {} 
\sphinxAtStartPar
\((0.101 \times 2^{1})_{2} = (1.01)_2 = 1.2^0 + 0.2^{-1} + 1.2^{-2} = 1\)

\item {} 
\sphinxAtStartPar
\((0.101 \times 2^{2})_{2} = (10.1)_2 = 1.2^1 + 0.2^{1} + 0.2^{-1} = 2\)

\end{itemize}

\sphinxAtStartPar
(…)

\sphinxAtStartPar
Fazendo as contas para os números restantes, obtemos a seguinte tabela:


\begin{savenotes}\sphinxattablestart
\centering
\begin{tabulary}{\linewidth}[t]{|T|T|T|T|T|T|}
\hline

\sphinxAtStartPar

&\sphinxstyletheadfamily 
\sphinxAtStartPar
m
&\sphinxstyletheadfamily 
\sphinxAtStartPar
0.100
&\sphinxstyletheadfamily 
\sphinxAtStartPar
0.101
&\sphinxstyletheadfamily 
\sphinxAtStartPar
0.110
&\sphinxstyletheadfamily 
\sphinxAtStartPar
0.111
\\
\hline
\sphinxAtStartPar
\sphinxstylestrong{e}
&
\sphinxAtStartPar

&
\sphinxAtStartPar

&
\sphinxAtStartPar

&
\sphinxAtStartPar

&
\sphinxAtStartPar

\\
\hline
\sphinxAtStartPar
\sphinxhyphen{}1
&
\sphinxAtStartPar

&
\sphinxAtStartPar
1/4
&
\sphinxAtStartPar
5/16
&
\sphinxAtStartPar
3/8
&
\sphinxAtStartPar
7/16
\\
\hline
\sphinxAtStartPar
0
&
\sphinxAtStartPar

&
\sphinxAtStartPar
1/2
&
\sphinxAtStartPar
5/8
&
\sphinxAtStartPar
3/4
&
\sphinxAtStartPar
7/8
\\
\hline
\sphinxAtStartPar
1
&
\sphinxAtStartPar

&
\sphinxAtStartPar
1
&
\sphinxAtStartPar
5/4
&
\sphinxAtStartPar
3/2
&
\sphinxAtStartPar
7/4
\\
\hline
\sphinxAtStartPar
2
&
\sphinxAtStartPar

&
\sphinxAtStartPar
2
&
\sphinxAtStartPar
5/2
&
\sphinxAtStartPar
3
&
\sphinxAtStartPar
7/2
\\
\hline
\end{tabulary}
\par
\sphinxattableend\end{savenotes}

\sphinxAtStartPar
Na reta real, esses valores ficariam dispostos da seguinte forma:

\begin{sphinxVerbatim}[commandchars=\\\{\}]
\PYG{k+kn}{from} \PYG{n+nn}{matplotlib}\PYG{n+nn}{.}\PYG{n+nn}{pyplot} \PYG{k+kn}{import} \PYG{n}{plot}
\PYG{n}{x} \PYG{o}{=} \PYG{p}{[}\PYG{l+m+mi}{1}\PYG{o}{/}\PYG{l+m+mi}{4}\PYG{p}{,}\PYG{l+m+mi}{1}\PYG{o}{/}\PYG{l+m+mi}{2}\PYG{p}{,}\PYG{l+m+mi}{1}\PYG{p}{,}\PYG{l+m+mi}{2}\PYG{p}{,}\PYG{l+m+mi}{5}\PYG{o}{/}\PYG{l+m+mi}{16}\PYG{p}{,}\PYG{l+m+mi}{5}\PYG{o}{/}\PYG{l+m+mi}{8}\PYG{p}{,}\PYG{l+m+mi}{5}\PYG{o}{/}\PYG{l+m+mi}{4}\PYG{p}{,}\PYG{l+m+mi}{5}\PYG{o}{/}\PYG{l+m+mi}{2}\PYG{p}{,}\PYG{l+m+mi}{3}\PYG{o}{/}\PYG{l+m+mi}{8}\PYG{p}{,}\PYG{l+m+mi}{3}\PYG{o}{/}\PYG{l+m+mi}{4}\PYG{p}{,}\PYG{l+m+mi}{3}\PYG{o}{/}\PYG{l+m+mi}{2}\PYG{p}{,}\PYG{l+m+mi}{3}\PYG{p}{,}\PYG{l+m+mi}{7}\PYG{o}{/}\PYG{l+m+mi}{16}\PYG{p}{,}\PYG{l+m+mi}{7}\PYG{o}{/}\PYG{l+m+mi}{8}\PYG{p}{,}\PYG{l+m+mi}{7}\PYG{o}{/}\PYG{l+m+mi}{4}\PYG{p}{,}\PYG{l+m+mi}{7}\PYG{o}{/}\PYG{l+m+mi}{2}\PYG{p}{]}
\PYG{n}{x} \PYG{o}{=} \PYG{n+nb}{sorted}\PYG{p}{(}\PYG{n}{x}\PYG{p}{)}

\PYG{n}{plot}\PYG{p}{(}\PYG{n}{x}\PYG{p}{,}\PYG{l+m+mi}{16}\PYG{o}{*}\PYG{p}{[}\PYG{l+m+mi}{0}\PYG{p}{]}\PYG{p}{,}\PYG{l+s+s1}{\PYGZsq{}}\PYG{l+s+s1}{:}\PYG{l+s+s1}{\PYGZsq{}}\PYG{p}{)}
\PYG{n}{plot}\PYG{p}{(}\PYG{n}{x}\PYG{p}{,}\PYG{l+m+mi}{16}\PYG{o}{*}\PYG{p}{[}\PYG{l+m+mi}{0}\PYG{p}{]}\PYG{p}{,}\PYG{l+s+s1}{\PYGZsq{}}\PYG{l+s+s1}{o}\PYG{l+s+s1}{\PYGZsq{}}\PYG{p}{)}\PYG{p}{;}
\end{sphinxVerbatim}

\noindent\sphinxincludegraphics{{aula-01-ponto-flutuante_11_0}.png}

\sphinxAtStartPar
Isto é, \(\mathbb{F}\) é uma reta “perfurada”, para a qual apenas 16 números positivos, 16 simétricos destes e mais o 0 são representáveis. Logo, o conjunto contém apenas 33 elementos.


\section{Simulador de \protect\(\mathbb{F}\protect\)}
\label{\detokenize{aula-01-ponto-flutuante:simulador-de-mathbb-f}}
\begin{sphinxVerbatim}[commandchars=\\\{\}]
\PYG{k+kn}{import} \PYG{n+nn}{numpy} \PYG{k}{as} \PYG{n+nn}{np}
\PYG{k+kn}{import} \PYG{n+nn}{matplotlib}\PYG{n+nn}{.}\PYG{n+nn}{pyplot} \PYG{k}{as} \PYG{n+nn}{plt}

\PYG{o}{\PYGZpc{}}\PYG{k}{matplotlib} inline

\PYG{k}{def} \PYG{n+nf}{simulacao\PYGZus{}F}\PYG{p}{(}\PYG{n}{b}\PYG{p}{,}\PYG{n}{t}\PYG{p}{,}\PYG{n}{L}\PYG{p}{,}\PYG{n}{U}\PYG{p}{)}\PYG{p}{:}
    \PYG{n}{x} \PYG{o}{=} \PYG{p}{[}\PYG{p}{]}
    \PYG{n}{epsm} \PYG{o}{=} \PYG{n}{b}\PYG{o}{*}\PYG{o}{*}\PYG{p}{(}\PYG{l+m+mi}{1}\PYG{o}{\PYGZhy{}}\PYG{n}{t}\PYG{p}{)} \PYG{c+c1}{\PYGZsh{} epsilon de máquina}
    \PYG{n}{M} \PYG{o}{=} \PYG{n}{np}\PYG{o}{.}\PYG{n}{arange}\PYG{p}{(}\PYG{l+m+mf}{1.}\PYG{p}{,}\PYG{n}{b}\PYG{o}{\PYGZhy{}}\PYG{n}{epsm}\PYG{p}{,}\PYG{n}{epsm}\PYG{p}{)}
    \PYG{n+nb}{print}\PYG{p}{(}\PYG{n}{M}\PYG{p}{)}

    \PYG{n}{E} \PYG{o}{=} \PYG{l+m+mi}{1}
    \PYG{k}{for} \PYG{n}{e} \PYG{o+ow}{in} \PYG{n+nb}{range}\PYG{p}{(}\PYG{l+m+mi}{0}\PYG{p}{,}\PYG{n}{U}\PYG{o}{+}\PYG{l+m+mi}{1}\PYG{p}{)}\PYG{p}{:}
        \PYG{n}{x} \PYG{o}{=} \PYG{n}{np}\PYG{o}{.}\PYG{n}{concatenate}\PYG{p}{(}\PYG{p}{[}\PYG{n}{x}\PYG{p}{,}\PYG{n}{M}\PYG{o}{*}\PYG{n}{E}\PYG{p}{]}\PYG{p}{)}
        \PYG{n}{E} \PYG{o}{*}\PYG{o}{=} \PYG{n}{b}    
    \PYG{n}{E} \PYG{o}{=} \PYG{n}{b}\PYG{o}{*}\PYG{o}{*}\PYG{p}{(}\PYG{o}{\PYGZhy{}}\PYG{l+m+mi}{1}\PYG{p}{)}
    
    \PYG{n}{y} \PYG{o}{=} \PYG{p}{[}\PYG{p}{]}
    \PYG{k}{for} \PYG{n}{e} \PYG{o+ow}{in} \PYG{n+nb}{range}\PYG{p}{(}\PYG{o}{\PYGZhy{}}\PYG{l+m+mi}{1}\PYG{p}{,}\PYG{n}{L}\PYG{o}{\PYGZhy{}}\PYG{l+m+mi}{1}\PYG{p}{,}\PYG{o}{\PYGZhy{}}\PYG{l+m+mi}{1}\PYG{p}{)}\PYG{p}{:}
        \PYG{n}{y} \PYG{o}{=} \PYG{n}{np}\PYG{o}{.}\PYG{n}{concatenate}\PYG{p}{(}\PYG{p}{[}\PYG{n}{y}\PYG{p}{,}\PYG{n}{M}\PYG{o}{*}\PYG{n}{E}\PYG{p}{]}\PYG{p}{)}
        \PYG{n}{E} \PYG{o}{/}\PYG{o}{=} \PYG{n}{b}    
    \PYG{n}{yy} \PYG{o}{=} \PYG{n}{np}\PYG{o}{.}\PYG{n}{asarray}\PYG{p}{(}\PYG{n}{y}\PYG{p}{)}
    \PYG{n}{xx} \PYG{o}{=} \PYG{n}{np}\PYG{o}{.}\PYG{n}{asarray}\PYG{p}{(}\PYG{n}{x}\PYG{p}{)}    
    \PYG{n}{x} \PYG{o}{=} \PYG{n}{np}\PYG{o}{.}\PYG{n}{concatenate}\PYG{p}{(}\PYG{p}{[}\PYG{n}{yy}\PYG{p}{,}\PYG{n}{np}\PYG{o}{.}\PYG{n}{array}\PYG{p}{(}\PYG{p}{[}\PYG{l+m+mf}{0.}\PYG{p}{]}\PYG{p}{)}\PYG{p}{,}\PYG{n}{xx}\PYG{p}{]}\PYG{p}{)}
    \PYG{k}{return} \PYG{n}{x}

\PYG{n}{Y} \PYG{o}{=} \PYG{n}{simulacao\PYGZus{}F}\PYG{p}{(}\PYG{l+m+mi}{2}\PYG{p}{,}\PYG{l+m+mi}{2}\PYG{p}{,}\PYG{o}{\PYGZhy{}}\PYG{l+m+mi}{3}\PYG{p}{,}\PYG{l+m+mi}{2}\PYG{p}{)}
\PYG{n}{X} \PYG{o}{=} \PYG{n}{np}\PYG{o}{.}\PYG{n}{zeros}\PYG{p}{(}\PYG{n}{Y}\PYG{o}{.}\PYG{n}{shape}\PYG{p}{)}

\PYG{n}{plt}\PYG{o}{.}\PYG{n}{scatter}\PYG{p}{(}\PYG{n}{Y}\PYG{p}{,}\PYG{n}{X}\PYG{p}{,}\PYG{n}{c}\PYG{o}{=}\PYG{l+s+s1}{\PYGZsq{}}\PYG{l+s+s1}{r}\PYG{l+s+s1}{\PYGZsq{}}\PYG{p}{,}\PYG{n}{marker}\PYG{o}{=}\PYG{l+s+s1}{\PYGZsq{}}\PYG{l+s+s1}{+}\PYG{l+s+s1}{\PYGZsq{}}\PYG{p}{)}\PYG{p}{;}
\end{sphinxVerbatim}

\begin{sphinxVerbatim}[commandchars=\\\{\}]
[1.]
\end{sphinxVerbatim}

\noindent\sphinxincludegraphics{{aula-01-ponto-flutuante_14_1}.png}


\section{Limites de máquina para ponto flutuante}
\label{\detokenize{aula-01-ponto-flutuante:limites-de-maquina-para-ponto-flutuante}}
\begin{sphinxVerbatim}[commandchars=\\\{\}]
\PYG{k+kn}{import} \PYG{n+nn}{numpy} \PYG{k}{as} \PYG{n+nn}{np} 

\PYG{c+c1}{\PYGZsh{} limites de máquina para ponto flutuante}
\PYG{c+c1}{\PYGZsh{}help(np.finfo)}

\PYG{c+c1}{\PYGZsh{} epsilon de máquina para tipo float (64 bits)}
\PYG{n+nb}{print}\PYG{p}{(}\PYG{l+s+s1}{\PYGZsq{}}\PYG{l+s+s1}{Epsilon de máquina do numpy \PYGZhy{} 64 bits}\PYG{l+s+s1}{\PYGZsq{}}\PYG{p}{)}
\PYG{n+nb}{print}\PYG{p}{(}\PYG{n}{np}\PYG{o}{.}\PYG{n}{finfo}\PYG{p}{(}\PYG{n+nb}{float}\PYG{p}{)}\PYG{o}{.}\PYG{n}{eps}\PYG{p}{)}

\PYG{c+c1}{\PYGZsh{} função para calculo do epsilon: erro relativo}
\PYG{k}{def} \PYG{n+nf}{eps\PYGZus{}mach}\PYG{p}{(}\PYG{n}{func}\PYG{o}{=}\PYG{n+nb}{float}\PYG{p}{)}\PYG{p}{:}
    \PYG{n}{eps} \PYG{o}{=} \PYG{n}{func}\PYG{p}{(}\PYG{l+m+mi}{1}\PYG{p}{)}
    \PYG{k}{while} \PYG{n}{func}\PYG{p}{(}\PYG{l+m+mi}{1}\PYG{p}{)} \PYG{o}{+} \PYG{n}{func}\PYG{p}{(}\PYG{n}{eps}\PYG{p}{)} \PYG{o}{!=} \PYG{n}{func}\PYG{p}{(}\PYG{l+m+mi}{1}\PYG{p}{)}\PYG{p}{:}
        \PYG{n}{epsf} \PYG{o}{=} \PYG{n}{eps}
        \PYG{n}{eps} \PYG{o}{=} \PYG{n}{func}\PYG{p}{(}\PYG{n}{eps}\PYG{p}{)} \PYG{o}{/} \PYG{n}{func}\PYG{p}{(}\PYG{l+m+mi}{2}\PYG{p}{)}
    \PYG{k}{return} \PYG{n}{epsf}

\PYG{c+c1}{\PYGZsh{} número máximo representável }
\PYG{n+nb}{print}\PYG{p}{(}\PYG{l+s+s1}{\PYGZsq{}}\PYG{l+s+s1}{número máximo representável}\PYG{l+s+s1}{\PYGZsq{}}\PYG{p}{)}
\PYG{n+nb}{print}\PYG{p}{(}\PYG{n}{np}\PYG{o}{.}\PYG{n}{finfo}\PYG{p}{(}\PYG{n+nb}{float}\PYG{p}{)}\PYG{o}{.}\PYG{n}{max}\PYG{p}{)}

\PYG{c+c1}{\PYGZsh{} número mínimo representável }
\PYG{n+nb}{print}\PYG{p}{(}\PYG{l+s+s1}{\PYGZsq{}}\PYG{l+s+s1}{número mínimo representável}\PYG{l+s+s1}{\PYGZsq{}}\PYG{p}{)} 
\PYG{n+nb}{print}\PYG{p}{(}\PYG{n}{np}\PYG{o}{.}\PYG{n}{finfo}\PYG{p}{(}\PYG{n+nb}{float}\PYG{p}{)}\PYG{o}{.}\PYG{n}{min}\PYG{p}{)}

\PYG{c+c1}{\PYGZsh{} número de bits no expoente }
\PYG{n+nb}{print}\PYG{p}{(}\PYG{l+s+s1}{\PYGZsq{}}\PYG{l+s+s1}{número de bits no expoente}\PYG{l+s+s1}{\PYGZsq{}}\PYG{p}{)} 
\PYG{n+nb}{print}\PYG{p}{(}\PYG{n}{np}\PYG{o}{.}\PYG{n}{finfo}\PYG{p}{(}\PYG{n+nb}{float}\PYG{p}{)}\PYG{o}{.}\PYG{n}{nexp}\PYG{p}{)}

\PYG{c+c1}{\PYGZsh{} número de bits na mantissa}
\PYG{n+nb}{print}\PYG{p}{(}\PYG{l+s+s1}{\PYGZsq{}}\PYG{l+s+s1}{número de bits na mantissa}\PYG{l+s+s1}{\PYGZsq{}}\PYG{p}{)}
\PYG{n+nb}{print}\PYG{p}{(}\PYG{n}{np}\PYG{o}{.}\PYG{n}{finfo}\PYG{p}{(}\PYG{n+nb}{float}\PYG{p}{)}\PYG{o}{.}\PYG{n}{nmant}\PYG{p}{)}
\end{sphinxVerbatim}

\begin{sphinxVerbatim}[commandchars=\\\{\}]
Epsilon de máquina do numpy \PYGZhy{} 64 bits
2.220446049250313e\PYGZhy{}16
número máximo representável
1.7976931348623157e+308
número mínimo representável
\PYGZhy{}1.7976931348623157e+308
número de bits no expoente
11
número de bits na mantissa
52
\end{sphinxVerbatim}

\begin{sphinxVerbatim}[commandchars=\\\{\}]
\PYG{k+kn}{from} \PYG{n+nn}{matplotlib}\PYG{n+nn}{.}\PYG{n+nn}{pyplot} \PYG{k+kn}{import} \PYG{n}{plot}

\PYG{n}{x} \PYG{o}{=} \PYG{n}{np}\PYG{o}{.}\PYG{n}{linspace}\PYG{p}{(}\PYG{l+m+mf}{1e\PYGZhy{}15}\PYG{p}{,}\PYG{l+m+mf}{1e\PYGZhy{}20}\PYG{p}{,}\PYG{n}{num}\PYG{o}{=}\PYG{l+m+mi}{100}\PYG{p}{)}
\PYG{n}{f} \PYG{o}{=} \PYG{p}{(}\PYG{p}{(}\PYG{l+m+mi}{1}\PYG{o}{+}\PYG{n}{x}\PYG{p}{)}\PYG{o}{\PYGZhy{}}\PYG{l+m+mi}{1}\PYG{p}{)}\PYG{o}{/}\PYG{n}{x}
\PYG{n}{plot}\PYG{p}{(}\PYG{n}{x}\PYG{p}{,}\PYG{n}{f}\PYG{p}{)}\PYG{p}{;}
\end{sphinxVerbatim}

\noindent\sphinxincludegraphics{{aula-01-ponto-flutuante_17_0}.png}


\chapter{Erros numéricos e seus efeitos}
\label{\detokenize{aula-02-erros:erros-numericos-e-seus-efeitos}}\label{\detokenize{aula-02-erros::doc}}
\begin{sphinxVerbatim}[commandchars=\\\{\}]
\PYG{o}{\PYGZpc{}}\PYG{k}{matplotlib} inline
\end{sphinxVerbatim}


\section{Motivação}
\label{\detokenize{aula-02-erros:motivacao}}
\sphinxAtStartPar
\sphinxstylestrong{Exemplo}: Avaliar o polinômio \(P(x) = x^3 - 6x^2 + 4x - 0.1\)
no ponto \(x=5.24\) e comparar com o resultado exato.

\sphinxAtStartPar
Vamos fazer o seguinte:
\begin{enumerate}
\sphinxsetlistlabels{\arabic}{enumi}{enumii}{}{.}%
\item {} 
\sphinxAtStartPar
Com uma calculadora, computar o valor de \(P(5.24)\) e assuma que este é seu valor exato.

\item {} 
\sphinxAtStartPar
Calcular \(P(5.24)\) usando arredondamento com dois dígitos de precisão.

\end{enumerate}

\sphinxAtStartPar
\sphinxstylestrong{Passo 1}

\sphinxAtStartPar
Faça as suas contas! Suponhamos que seja \sphinxhyphen{}0.007776.

\sphinxAtStartPar
\sphinxstylestrong{Passo 2}

\sphinxAtStartPar
Vamos “imitar” as contas feitas na mão…

\begin{sphinxVerbatim}[commandchars=\\\{\}]
\PYG{c+c1}{\PYGZsh{} parcelas }

\PYG{n}{p1} \PYG{o}{=} \PYG{l+m+mf}{5.24}\PYG{o}{*}\PYG{o}{*}\PYG{l+m+mi}{3}
\PYG{n+nb}{print}\PYG{p}{(}\PYG{l+s+s1}{\PYGZsq{}}\PYG{l+s+s1}{p1: }\PYG{l+s+si}{\PYGZob{}0:.20g\PYGZcb{}}\PYG{l+s+s1}{\PYGZsq{}}\PYG{o}{.}\PYG{n}{format}\PYG{p}{(}\PYG{n}{p1}\PYG{p}{)}\PYG{p}{)} \PYG{c+c1}{\PYGZsh{} 20 dígitos significativos}
\PYG{n+nb}{print}\PYG{p}{(}\PYG{l+s+s1}{\PYGZsq{}}\PYG{l+s+s1}{p1 (com arredondamento): }\PYG{l+s+si}{\PYGZob{}0:.2f\PYGZcb{}}\PYG{l+s+s1}{\PYGZsq{}}\PYG{o}{.}\PYG{n}{format}\PYG{p}{(}\PYG{n}{p1}\PYG{p}{)}\PYG{p}{)} 

\PYG{n+nb}{print}\PYG{p}{(}\PYG{l+s+s1}{\PYGZsq{}}\PYG{l+s+se}{\PYGZbs{}n}\PYG{l+s+s1}{\PYGZsq{}}\PYG{p}{)}

\PYG{n}{p2} \PYG{o}{=} \PYG{o}{\PYGZhy{}} \PYG{l+m+mi}{6}\PYG{o}{*}\PYG{l+m+mf}{5.24}\PYG{o}{*}\PYG{o}{*}\PYG{l+m+mi}{2}
\PYG{n+nb}{print}\PYG{p}{(}\PYG{l+s+s1}{\PYGZsq{}}\PYG{l+s+s1}{p2: }\PYG{l+s+si}{\PYGZob{}0:.20g\PYGZcb{}}\PYG{l+s+s1}{\PYGZsq{}}\PYG{o}{.}\PYG{n}{format}\PYG{p}{(}\PYG{n}{p2}\PYG{p}{)}\PYG{p}{)}
\PYG{n+nb}{print}\PYG{p}{(}\PYG{l+s+s1}{\PYGZsq{}}\PYG{l+s+s1}{p2 (com arredondamento): }\PYG{l+s+si}{\PYGZob{}0:.2f\PYGZcb{}}\PYG{l+s+s1}{\PYGZsq{}}\PYG{o}{.}\PYG{n}{format}\PYG{p}{(}\PYG{n}{p2}\PYG{p}{)}\PYG{p}{)}

\PYG{n+nb}{print}\PYG{p}{(}\PYG{l+s+s1}{\PYGZsq{}}\PYG{l+s+se}{\PYGZbs{}n}\PYG{l+s+s1}{\PYGZsq{}}\PYG{p}{)}

\PYG{n}{p3} \PYG{o}{=} \PYG{l+m+mi}{4}\PYG{o}{*}\PYG{l+m+mf}{5.24}
\PYG{n+nb}{print}\PYG{p}{(}\PYG{l+s+s1}{\PYGZsq{}}\PYG{l+s+s1}{p3: }\PYG{l+s+si}{\PYGZob{}0:.20g\PYGZcb{}}\PYG{l+s+s1}{\PYGZsq{}}\PYG{o}{.}\PYG{n}{format}\PYG{p}{(}\PYG{n}{p3}\PYG{p}{)}\PYG{p}{)}
\PYG{n+nb}{print}\PYG{p}{(}\PYG{l+s+s1}{\PYGZsq{}}\PYG{l+s+s1}{p3 (com arredondamento): }\PYG{l+s+si}{\PYGZob{}0:.2f\PYGZcb{}}\PYG{l+s+s1}{\PYGZsq{}}\PYG{o}{.}\PYG{n}{format}\PYG{p}{(}\PYG{n}{p3}\PYG{p}{)}\PYG{p}{)}

\PYG{n+nb}{print}\PYG{p}{(}\PYG{l+s+s1}{\PYGZsq{}}\PYG{l+s+se}{\PYGZbs{}n}\PYG{l+s+s1}{\PYGZsq{}}\PYG{p}{)}

\PYG{n}{p4} \PYG{o}{=} \PYG{o}{\PYGZhy{}} \PYG{l+m+mf}{0.1}
\PYG{n+nb}{print}\PYG{p}{(}\PYG{l+s+s1}{\PYGZsq{}}\PYG{l+s+s1}{p4: }\PYG{l+s+si}{\PYGZob{}0:.20g\PYGZcb{}}\PYG{l+s+s1}{\PYGZsq{}}\PYG{o}{.}\PYG{n}{format}\PYG{p}{(}\PYG{n}{p4}\PYG{p}{)}\PYG{p}{)}
\PYG{n+nb}{print}\PYG{p}{(}\PYG{l+s+s1}{\PYGZsq{}}\PYG{l+s+s1}{p4 (com arredondamento): }\PYG{l+s+si}{\PYGZob{}0:.2f\PYGZcb{}}\PYG{l+s+s1}{\PYGZsq{}}\PYG{o}{.}\PYG{n}{format}\PYG{p}{(}\PYG{n}{p4}\PYG{p}{)}\PYG{p}{)}

\PYG{n+nb}{print}\PYG{p}{(}\PYG{l+s+s1}{\PYGZsq{}}\PYG{l+s+se}{\PYGZbs{}n}\PYG{l+s+s1}{\PYGZsq{}}\PYG{p}{)}

\PYG{n}{Px} \PYG{o}{=} \PYG{n}{p1} \PYG{o}{+} \PYG{n}{p2} \PYG{o}{+} \PYG{n}{p3} \PYG{o}{+} \PYG{n}{p4}
\PYG{n+nb}{print}\PYG{p}{(}\PYG{l+s+s1}{\PYGZsq{}}\PYG{l+s+s1}{Px: }\PYG{l+s+si}{\PYGZob{}0:.20g\PYGZcb{}}\PYG{l+s+s1}{\PYGZsq{}}\PYG{o}{.}\PYG{n}{format}\PYG{p}{(}\PYG{n}{Px}\PYG{p}{)}\PYG{p}{)}
\PYG{n+nb}{print}\PYG{p}{(}\PYG{l+s+s1}{\PYGZsq{}}\PYG{l+s+s1}{Px: (com arredondamento): }\PYG{l+s+si}{\PYGZob{}0:.2f\PYGZcb{}}\PYG{l+s+s1}{\PYGZsq{}}\PYG{o}{.}\PYG{n}{format}\PYG{p}{(}\PYG{n}{Px}\PYG{p}{)}\PYG{p}{)}
\end{sphinxVerbatim}

\begin{sphinxVerbatim}[commandchars=\\\{\}]
p1: 143.87782400000000393
p1 (com arredondamento): 143.88


p2: \PYGZhy{}164.74560000000002447
p2 (com arredondamento): \PYGZhy{}164.75


p3: 20.960000000000000853
p3 (com arredondamento): 20.96


p4: \PYGZhy{}0.10000000000000000555
p4 (com arredondamento): \PYGZhy{}0.10


Px: \PYGZhy{}0.0077760000000196838332
Px: (com arredondamento): \PYGZhy{}0.01
\end{sphinxVerbatim}

\sphinxAtStartPar
\sphinxstylestrong{Conclusão:} o cálculo com dois dígitos afeta o resultado drasticamente!

\sphinxAtStartPar
Agora, vamos comparar o resultado de se avaliar \(P(5.24)\) com as duas formas do polinômio e 16 dígitos de precisão:

\begin{sphinxVerbatim}[commandchars=\\\{\}]
\PYG{c+c1}{\PYGZsh{} ponto de análise}
\PYG{n}{x} \PYG{o}{=} \PYG{l+m+mf}{5.24}

\PYG{c+c1}{\PYGZsh{} P1(x) }
\PYG{n}{P1x} \PYG{o}{=} \PYG{n}{x}\PYG{o}{*}\PYG{o}{*}\PYG{l+m+mi}{3} \PYG{o}{\PYGZhy{}} \PYG{l+m+mi}{6}\PYG{o}{*}\PYG{n}{x}\PYG{o}{*}\PYG{o}{*}\PYG{l+m+mi}{2} \PYG{o}{+} \PYG{l+m+mi}{4}\PYG{o}{*}\PYG{n}{x} \PYG{o}{\PYGZhy{}} \PYG{l+m+mf}{0.1} 
\PYG{n+nb}{print}\PYG{p}{(}\PYG{l+s+s1}{\PYGZsq{}}\PYG{l+s+si}{\PYGZob{}0:.16f\PYGZcb{}}\PYG{l+s+s1}{\PYGZsq{}}\PYG{o}{.}\PYG{n}{format}\PYG{p}{(}\PYG{n}{P1x}\PYG{p}{)}\PYG{p}{)}

\PYG{c+c1}{\PYGZsh{} P2(x) }
\PYG{n}{P2x} \PYG{o}{=} \PYG{n}{x}\PYG{o}{*}\PYG{p}{(}\PYG{n}{x}\PYG{o}{*}\PYG{p}{(}\PYG{n}{x} \PYG{o}{\PYGZhy{}} \PYG{l+m+mi}{6}\PYG{p}{)} \PYG{o}{+} \PYG{l+m+mi}{4}\PYG{p}{)} \PYG{o}{\PYGZhy{}} \PYG{l+m+mf}{0.1} \PYG{c+c1}{\PYGZsh{} forma estruturada (forma de Hörner)}
\PYG{n+nb}{print}\PYG{p}{(}\PYG{l+s+s1}{\PYGZsq{}}\PYG{l+s+si}{\PYGZob{}0:.16f\PYGZcb{}}\PYG{l+s+s1}{\PYGZsq{}}\PYG{o}{.}\PYG{n}{format}\PYG{p}{(}\PYG{n}{P2x}\PYG{p}{)}\PYG{p}{)}
\end{sphinxVerbatim}

\begin{sphinxVerbatim}[commandchars=\\\{\}]
\PYGZhy{}0.0077760000000197
\PYGZhy{}0.0077759999999939
\end{sphinxVerbatim}

\sphinxAtStartPar
O que temos acima? Dois valores levemente distintos. Se computarmos os erros absoluto e relativo entre esses valores e nosso valor supostamente assumido como exato, teríamos:

\sphinxAtStartPar
\sphinxstylestrong{Erros absolutos}

\begin{sphinxVerbatim}[commandchars=\\\{\}]
\PYG{n}{x\PYGZus{}exato} \PYG{o}{=} \PYG{o}{\PYGZhy{}}\PYG{l+m+mf}{0.007776}
\PYG{n}{EA1} \PYG{o}{=} \PYG{n+nb}{abs}\PYG{p}{(}\PYG{n}{P1x} \PYG{o}{\PYGZhy{}} \PYG{n}{x\PYGZus{}exato}\PYG{p}{)}
\PYG{n+nb}{print}\PYG{p}{(}\PYG{n}{EA1}\PYG{p}{)}

\PYG{n}{EA2} \PYG{o}{=} \PYG{n+nb}{abs}\PYG{p}{(}\PYG{n}{P2x} \PYG{o}{\PYGZhy{}} \PYG{n}{x\PYGZus{}exato}\PYG{p}{)}
\PYG{n+nb}{print}\PYG{p}{(}\PYG{n}{EA2}\PYG{p}{)}
\end{sphinxVerbatim}

\begin{sphinxVerbatim}[commandchars=\\\{\}]
1.968390728190883e\PYGZhy{}14
6.1287780406260595e\PYGZhy{}15
\end{sphinxVerbatim}

\sphinxAtStartPar
Claro que \(EA_1 > EA_2\). Entretanto, podemos verificar pelo seguinte teste lógico:

\begin{sphinxVerbatim}[commandchars=\\\{\}]
\PYG{c+c1}{\PYGZsh{} teste é verdadeiro}
\PYG{n}{EA1} \PYG{o}{\PYGZgt{}} \PYG{n}{EA2}
\end{sphinxVerbatim}

\begin{sphinxVerbatim}[commandchars=\\\{\}]
True
\end{sphinxVerbatim}

\sphinxAtStartPar
\sphinxstylestrong{Erros relativos}

\sphinxAtStartPar
Os erros relativos também podem ser computados como:

\begin{sphinxVerbatim}[commandchars=\\\{\}]
\PYG{n}{ER1} \PYG{o}{=} \PYG{n}{EA1}\PYG{o}{/}\PYG{n+nb}{abs}\PYG{p}{(}\PYG{n}{x\PYGZus{}exato}\PYG{p}{)}
\PYG{n+nb}{print}\PYG{p}{(}\PYG{n}{ER1}\PYG{p}{)}

\PYG{n}{ER2} \PYG{o}{=} \PYG{n}{EA2}\PYG{o}{/}\PYG{n+nb}{abs}\PYG{p}{(}\PYG{n}{x\PYGZus{}exato}\PYG{p}{)}
\PYG{n+nb}{print}\PYG{p}{(}\PYG{n}{ER2}\PYG{p}{)}
\end{sphinxVerbatim}

\begin{sphinxVerbatim}[commandchars=\\\{\}]
2.5313666772002096e\PYGZhy{}12
7.881659002862731e\PYGZhy{}13
\end{sphinxVerbatim}

\sphinxAtStartPar
\sphinxstylestrong{Gráfico de \(P(x)\)}

\begin{sphinxVerbatim}[commandchars=\\\{\}]
\PYG{k+kn}{import} \PYG{n+nn}{numpy} \PYG{k}{as} \PYG{n+nn}{np}
\PYG{k+kn}{import} \PYG{n+nn}{matplotlib}\PYG{n+nn}{.}\PYG{n+nn}{pyplot} \PYG{k}{as} \PYG{n+nn}{plt}

\PYG{c+c1}{\PYGZsh{} eixo x com 20 pontos}
\PYG{n}{x} \PYG{o}{=} \PYG{n}{np}\PYG{o}{.}\PYG{n}{linspace}\PYG{p}{(}\PYG{o}{\PYGZhy{}}\PYG{l+m+mi}{3}\PYG{p}{,}\PYG{l+m+mi}{3}\PYG{p}{,}\PYG{n}{num}\PYG{o}{=}\PYG{l+m+mi}{20}\PYG{p}{,}\PYG{n}{endpoint}\PYG{o}{=}\PYG{k+kc}{True}\PYG{p}{)}

\PYG{c+c1}{\PYGZsh{} plotagem de P1(x) e P2(x)}
\PYG{n}{P1x} \PYG{o}{=} \PYG{k}{lambda} \PYG{n}{x}\PYG{p}{:} \PYG{n}{x}\PYG{o}{*}\PYG{o}{*}\PYG{l+m+mi}{3} \PYG{o}{\PYGZhy{}} \PYG{l+m+mi}{6}\PYG{o}{*}\PYG{n}{x}\PYG{o}{*}\PYG{o}{*}\PYG{l+m+mi}{2} \PYG{o}{+} \PYG{l+m+mi}{4}\PYG{o}{*}\PYG{n}{x} \PYG{o}{\PYGZhy{}} \PYG{l+m+mf}{0.1}
\PYG{n}{P2x} \PYG{o}{=} \PYG{k}{lambda} \PYG{n}{x}\PYG{p}{:} \PYG{n}{x}\PYG{o}{*}\PYG{p}{(}\PYG{n}{x}\PYG{o}{*}\PYG{p}{(}\PYG{n}{x} \PYG{o}{\PYGZhy{}} \PYG{l+m+mi}{6}\PYG{p}{)} \PYG{o}{+} \PYG{l+m+mi}{4}\PYG{p}{)} \PYG{o}{\PYGZhy{}} \PYG{l+m+mf}{0.1}
\PYG{n}{plt}\PYG{o}{.}\PYG{n}{plot}\PYG{p}{(}\PYG{n}{x}\PYG{p}{,}\PYG{n}{P1x}\PYG{p}{(}\PYG{n}{x}\PYG{p}{)}\PYG{p}{,}\PYG{l+s+s1}{\PYGZsq{}}\PYG{l+s+s1}{r}\PYG{l+s+s1}{\PYGZsq{}}\PYG{p}{,}\PYG{n}{x}\PYG{p}{,}\PYG{n}{P2x}\PYG{p}{(}\PYG{n}{x}\PYG{p}{)}\PYG{p}{,}\PYG{l+s+s1}{\PYGZsq{}}\PYG{l+s+s1}{bo}\PYG{l+s+s1}{\PYGZsq{}}\PYG{p}{)}\PYG{p}{;}
\end{sphinxVerbatim}

\noindent\sphinxincludegraphics{{aula-02-erros_15_0}.png}


\subsection{Função de Airy}
\label{\detokenize{aula-02-erros:funcao-de-airy}}
\sphinxAtStartPar
A função de Airy é solução da equação de Schrödinger da mecânica quântica. Muda o comportamento de oscilatório para exponencial.

\sphinxAtStartPar
Abaixo, vamos criar uma função aproximada (perturbada) para a função de Airy (assumindo\sphinxhyphen{}a como uma aproximação daquela que é exata) e outra para calcular diretamente o erro relativo para valores dessas funções.

\begin{sphinxVerbatim}[commandchars=\\\{\}]
\PYG{k+kn}{from} \PYG{n+nn}{scipy} \PYG{k+kn}{import} \PYG{n}{special}
\PYG{k+kn}{import} \PYG{n+nn}{matplotlib}\PYG{n+nn}{.}\PYG{n+nn}{pyplot} \PYG{k}{as} \PYG{n+nn}{plt} 

\PYG{c+c1}{\PYGZsh{} eixo x }
\PYG{n}{x} \PYG{o}{=} \PYG{n}{np}\PYG{o}{.}\PYG{n}{linspace}\PYG{p}{(}\PYG{o}{\PYGZhy{}}\PYG{l+m+mi}{10}\PYG{p}{,} \PYG{o}{\PYGZhy{}}\PYG{l+m+mi}{2}\PYG{p}{,} \PYG{l+m+mi}{100}\PYG{p}{)}

\PYG{c+c1}{\PYGZsh{} funções de Airy e derivadas (solução exata)}
\PYG{n}{ai}\PYG{p}{,} \PYG{n}{aip}\PYG{p}{,} \PYG{n}{bi}\PYG{p}{,} \PYG{n}{bip} \PYG{o}{=} \PYG{n}{special}\PYG{o}{.}\PYG{n}{airy}\PYG{p}{(}\PYG{n}{x}\PYG{p}{)}

\PYG{c+c1}{\PYGZsh{} função de Airy (fazendo papel de solução aproximada)}
\PYG{n}{ai2} \PYG{o}{=} \PYG{l+m+mf}{1.1}\PYG{o}{*}\PYG{n}{ai} \PYG{o}{+} \PYG{l+m+mf}{0.05}\PYG{o}{*}\PYG{n}{np}\PYG{o}{.}\PYG{n}{cos}\PYG{p}{(}\PYG{n}{x}\PYG{p}{)} 
\end{sphinxVerbatim}

\sphinxAtStartPar
Podemos usar o conceito de \sphinxstyleemphasis{função anônima} para calcular diretamente o \sphinxstylestrong{erro relativo percentual} para cada ponto \(x\):
\begin{equation*}
\begin{split}ER_p(x) = \frac{\mid \ f_{aprox}(x) - f_{ex}(x) \ \mid}{\mid \ f_{ex}(x) \ \mid},\end{split}
\end{equation*}
\sphinxAtStartPar
onde \(f_{aprox}(x)\) é o valor da função aproximada (de Airy) e
onde \(f_{ex}(x)\) é o valor da função exata (de Airy).

\begin{sphinxVerbatim}[commandchars=\\\{\}]
\PYG{c+c1}{\PYGZsh{} define função anônima para erro relativo}
\PYG{n}{r} \PYG{o}{=} \PYG{k}{lambda} \PYG{n}{fex}\PYG{p}{,}\PYG{n}{faprox}\PYG{p}{:} \PYG{p}{(}\PYG{n}{np}\PYG{o}{.}\PYG{n}{abs}\PYG{p}{(}\PYG{n}{fex}\PYG{o}{\PYGZhy{}}\PYG{n}{faprox}\PYG{p}{)}\PYG{o}{/}\PYG{n}{np}\PYG{o}{.}\PYG{n}{abs}\PYG{p}{(}\PYG{n}{fex}\PYG{p}{)}\PYG{p}{)}\PYG{o}{/}\PYG{l+m+mi}{100}

\PYG{c+c1}{\PYGZsh{} calcula erro relativo para função de Airy e sua aproximação}
\PYG{n}{rel} \PYG{o}{=} \PYG{n}{r}\PYG{p}{(}\PYG{n}{ai}\PYG{p}{,}\PYG{n}{ai2}\PYG{p}{)}
\end{sphinxVerbatim}

\sphinxAtStartPar
A seguir, mostramos a plotagem das funções exatas e aproximadas, bem como do erro relativo pontual.

\begin{sphinxVerbatim}[commandchars=\\\{\}]
\PYG{c+c1}{\PYGZsh{} plotagens }
\PYG{n}{plt}\PYG{o}{.}\PYG{n}{plot}\PYG{p}{(}\PYG{n}{x}\PYG{p}{,} \PYG{n}{ai}\PYG{p}{,} \PYG{l+s+s1}{\PYGZsq{}}\PYG{l+s+s1}{r}\PYG{l+s+s1}{\PYGZsq{}}\PYG{p}{,} \PYG{n}{label}\PYG{o}{=}\PYG{l+s+s1}{\PYGZsq{}}\PYG{l+s+s1}{sol exata}\PYG{l+s+s1}{\PYGZsq{}}\PYG{p}{)}
\PYG{n}{plt}\PYG{o}{.}\PYG{n}{plot}\PYG{p}{(}\PYG{n}{x}\PYG{p}{,} \PYG{n}{ai2}\PYG{p}{,} \PYG{l+s+s1}{\PYGZsq{}}\PYG{l+s+s1}{b}\PYG{l+s+s1}{\PYGZsq{}}\PYG{p}{,} \PYG{n}{label}\PYG{o}{=}\PYG{l+s+s1}{\PYGZsq{}}\PYG{l+s+s1}{sol aprox}\PYG{l+s+s1}{\PYGZsq{}}\PYG{p}{)}
\PYG{n}{plt}\PYG{o}{.}\PYG{n}{grid}\PYG{p}{(}\PYG{p}{)}
\PYG{n}{plt}\PYG{o}{.}\PYG{n}{legend}\PYG{p}{(}\PYG{n}{loc}\PYG{o}{=}\PYG{l+s+s1}{\PYGZsq{}}\PYG{l+s+s1}{upper right}\PYG{l+s+s1}{\PYGZsq{}}\PYG{p}{)}
\PYG{n}{plt}\PYG{o}{.}\PYG{n}{show}\PYG{p}{(}\PYG{p}{)}

\PYG{n}{plt}\PYG{o}{.}\PYG{n}{plot}\PYG{p}{(}\PYG{n}{x}\PYG{p}{,}\PYG{n}{rel}\PYG{p}{,}\PYG{l+s+s1}{\PYGZsq{}}\PYG{l+s+s1}{\PYGZhy{}}\PYG{l+s+s1}{\PYGZsq{}}\PYG{p}{,} \PYG{n}{label}\PYG{o}{=}\PYG{l+s+s1}{\PYGZsq{}}\PYG{l+s+s1}{err rel }\PYG{l+s+s1}{\PYGZpc{}}\PYG{l+s+s1}{\PYGZsq{}}\PYG{p}{)}
\PYG{n}{plt}\PYG{o}{.}\PYG{n}{grid}\PYG{p}{(}\PYG{p}{)}
\PYG{n}{plt}\PYG{o}{.}\PYG{n}{legend}\PYG{p}{(}\PYG{n}{loc}\PYG{o}{=}\PYG{l+s+s1}{\PYGZsq{}}\PYG{l+s+s1}{upper right}\PYG{l+s+s1}{\PYGZsq{}}\PYG{p}{)}\PYG{p}{;}
\end{sphinxVerbatim}

\noindent\sphinxincludegraphics{{aula-02-erros_21_0}.png}

\noindent\sphinxincludegraphics{{aula-02-erros_21_1}.png}


\section{Erro de cancelamento}
\label{\detokenize{aula-02-erros:erro-de-cancelamento}}
\sphinxAtStartPar
Ocorre quando números de grandezas próximas são subtraídos. No exemplo, a seguir, induzimos uma divisão por zero usando o valor do épsilon de máquina \(\epsilon_m\) ao fazer
\begin{equation*}
\begin{split}\dfrac{1}{(1 + 0.25\epsilon_m) - 1}\end{split}
\end{equation*}
\sphinxAtStartPar
Isto ocorre porque o denominador sofre um \sphinxstyleemphasis{cancelamento subtrativo}, quando, para a matemática precisa, deveria valer \(0.25\epsilon_m\).


\section{Propagação de erros}
\label{\detokenize{aula-02-erros:propagacao-de-erros}}
\sphinxAtStartPar
Vamos comparar duas situações. Calcular
\begin{equation*}
\begin{split}e^{-v} = \sum_{i=0}^{\infty} (-1)^i \frac{v^i}{i!}\end{split}
\end{equation*}
\sphinxAtStartPar
e comparar com a identidade \$\(e^{-v} = \dfrac{1}{e^v}.\)\$

\begin{sphinxVerbatim}[commandchars=\\\{\}]
\PYG{c+c1}{\PYGZsh{} somatória (primeiros 20 termos)}
\PYG{n}{v} \PYG{o}{=} \PYG{l+m+mf}{5.25}
\PYG{n}{s} \PYG{o}{=} \PYG{l+m+mi}{0}
\PYG{k}{for} \PYG{n}{i} \PYG{o+ow}{in} \PYG{n+nb}{range}\PYG{p}{(}\PYG{l+m+mi}{20}\PYG{p}{)}\PYG{p}{:}    
    \PYG{n+nb}{print}\PYG{p}{(}\PYG{l+s+s1}{\PYGZsq{}}\PYG{l+s+si}{\PYGZob{}0:5g\PYGZcb{}}\PYG{l+s+s1}{\PYGZsq{}}\PYG{o}{.}\PYG{n}{format}\PYG{p}{(}\PYG{n}{s}\PYG{p}{)}\PYG{p}{)}    
    \PYG{n}{s} \PYG{o}{+}\PYG{o}{=} \PYG{p}{(}\PYG{p}{(}\PYG{o}{\PYGZhy{}}\PYG{l+m+mi}{1}\PYG{p}{)}\PYG{o}{*}\PYG{o}{*}\PYG{n}{i}\PYG{o}{*}\PYG{n}{v}\PYG{o}{*}\PYG{o}{*}\PYG{n}{i}\PYG{p}{)}\PYG{o}{/}\PYG{n}{np}\PYG{o}{.}\PYG{n}{math}\PYG{o}{.}\PYG{n}{factorial}\PYG{p}{(}\PYG{n}{i}\PYG{p}{)}

\PYG{n+nb}{print}\PYG{p}{(}\PYG{l+s+s1}{\PYGZsq{}}\PYG{l+s+se}{\PYGZbs{}n}\PYG{l+s+s1}{caso 1: }\PYG{l+s+si}{\PYGZob{}0:5g\PYGZcb{}}\PYG{l+s+s1}{\PYGZsq{}}\PYG{o}{.}\PYG{n}{format}\PYG{p}{(}\PYG{n}{s}\PYG{p}{)}\PYG{p}{)}    

\PYG{n+nb}{print}\PYG{p}{(}\PYG{l+s+s1}{\PYGZsq{}}\PYG{l+s+s1}{caso 2: }\PYG{l+s+si}{\PYGZob{}0:5g\PYGZcb{}}\PYG{l+s+s1}{\PYGZsq{}}\PYG{o}{.}\PYG{n}{format}\PYG{p}{(}\PYG{l+m+mi}{1}\PYG{o}{/}\PYG{n}{np}\PYG{o}{.}\PYG{n}{exp}\PYG{p}{(}\PYG{n}{v}\PYG{p}{)}\PYG{p}{)}\PYG{p}{)}
\end{sphinxVerbatim}

\begin{sphinxVerbatim}[commandchars=\\\{\}]
    0
    1
\PYGZhy{}4.25
9.53125
\PYGZhy{}14.5859
17.0679
\PYGZhy{}16.1686
12.9133
\PYGZhy{}8.89814
5.41562
\PYGZhy{}2.93407
1.44952
\PYGZhy{}0.642652
0.272671
\PYGZhy{}0.0969786
0.0416401
\PYGZhy{}0.00687642
0.00904307
0.00412676
0.00556069

caso 1: 0.00516447
caso 2: 0.00524752
\end{sphinxVerbatim}


\chapter{Análise gráfica, localização e refinamento de raízes}
\label{\detokenize{aula-03-analise-grafica:analise-grafica-localizacao-e-refinamento-de-raizes}}\label{\detokenize{aula-03-analise-grafica::doc}}

\section{Estudo de caso: salto do paraquedista}
\label{\detokenize{aula-03-analise-grafica:estudo-de-caso-salto-do-paraquedista}}
\begin{sphinxVerbatim}[commandchars=\\\{\}]
\PYG{o}{\PYGZpc{}}\PYG{k}{matplotlib} inline 
\end{sphinxVerbatim}

\begin{sphinxVerbatim}[commandchars=\\\{\}]
\PYG{k+kn}{import} \PYG{n+nn}{numpy} \PYG{k}{as} \PYG{n+nn}{np}
\PYG{k+kn}{import} \PYG{n+nn}{matplotlib}\PYG{n+nn}{.}\PYG{n+nn}{pyplot} \PYG{k}{as} \PYG{n+nn}{plt} 

\PYG{c+c1}{\PYGZsh{} parametros }
\PYG{n}{t} \PYG{o}{=} \PYG{l+m+mf}{12.0}
\PYG{n}{v} \PYG{o}{=} \PYG{l+m+mf}{42.0}
\PYG{n}{m} \PYG{o}{=} \PYG{l+m+mf}{70.0}
\PYG{n}{g} \PYG{o}{=} \PYG{l+m+mf}{9.81}
\end{sphinxVerbatim}

\begin{sphinxVerbatim}[commandchars=\\\{\}]
\PYG{c+c1}{\PYGZsh{} localizacao}
\PYG{n}{a}\PYG{p}{,}\PYG{n}{b} \PYG{o}{=} \PYG{l+m+mi}{1}\PYG{p}{,}\PYG{l+m+mi}{20}
\PYG{n}{c} \PYG{o}{=} \PYG{n}{np}\PYG{o}{.}\PYG{n}{linspace}\PYG{p}{(}\PYG{n}{a}\PYG{p}{,}\PYG{n}{b}\PYG{p}{,}\PYG{l+m+mi}{20}\PYG{p}{)}
\PYG{n}{f} \PYG{o}{=} \PYG{n}{g}\PYG{o}{*}\PYG{n}{m}\PYG{o}{/}\PYG{n}{c}\PYG{o}{*}\PYG{p}{(}\PYG{l+m+mi}{1} \PYG{o}{\PYGZhy{}} \PYG{n}{np}\PYG{o}{.}\PYG{n}{exp}\PYG{p}{(}\PYG{o}{\PYGZhy{}}\PYG{n}{c}\PYG{o}{/}\PYG{n}{m}\PYG{o}{*}\PYG{n}{t}\PYG{p}{)}\PYG{p}{)} \PYG{o}{\PYGZhy{}} \PYG{n}{v}

\PYG{n}{plt}\PYG{o}{.}\PYG{n}{plot}\PYG{p}{(}\PYG{n}{c}\PYG{p}{,}\PYG{n}{f}\PYG{p}{,}\PYG{l+s+s1}{\PYGZsq{}}\PYG{l+s+s1}{k\PYGZhy{}\PYGZhy{}o}\PYG{l+s+s1}{\PYGZsq{}}\PYG{p}{)}\PYG{p}{;}
\PYG{n}{plt}\PYG{o}{.}\PYG{n}{xlabel}\PYG{p}{(}\PYG{l+s+s1}{\PYGZsq{}}\PYG{l+s+s1}{c}\PYG{l+s+s1}{\PYGZsq{}}\PYG{p}{)}
\PYG{n}{plt}\PYG{o}{.}\PYG{n}{ylabel}\PYG{p}{(}\PYG{l+s+s1}{\PYGZsq{}}\PYG{l+s+s1}{f(c)}\PYG{l+s+s1}{\PYGZsq{}}\PYG{p}{)}
\PYG{n}{plt}\PYG{o}{.}\PYG{n}{title}\PYG{p}{(}\PYG{l+s+s1}{\PYGZsq{}}\PYG{l+s+s1}{Variação do coef. de arrasto}\PYG{l+s+s1}{\PYGZsq{}}\PYG{p}{)}
\PYG{n}{plt}\PYG{o}{.}\PYG{n}{grid}\PYG{p}{(}\PYG{k+kc}{True}\PYG{p}{)}
\end{sphinxVerbatim}

\noindent\sphinxincludegraphics{{aula-03-analise-grafica_4_0}.png}

\begin{sphinxVerbatim}[commandchars=\\\{\}]
\PYG{c+c1}{\PYGZsh{} refinamento}
\PYG{n}{a}\PYG{p}{,}\PYG{n}{b} \PYG{o}{=} \PYG{l+m+mi}{10}\PYG{p}{,}\PYG{l+m+mi}{20}
\PYG{n}{c} \PYG{o}{=} \PYG{n}{np}\PYG{o}{.}\PYG{n}{linspace}\PYG{p}{(}\PYG{n}{a}\PYG{p}{,}\PYG{n}{b}\PYG{p}{,}\PYG{l+m+mi}{100}\PYG{p}{)}
\PYG{n}{f} \PYG{o}{=} \PYG{n}{g}\PYG{o}{*}\PYG{n}{m}\PYG{o}{/}\PYG{n}{c}\PYG{o}{*}\PYG{p}{(}\PYG{l+m+mi}{1} \PYG{o}{\PYGZhy{}} \PYG{n}{np}\PYG{o}{.}\PYG{n}{exp}\PYG{p}{(}\PYG{o}{\PYGZhy{}}\PYG{n}{c}\PYG{o}{/}\PYG{n}{m}\PYG{o}{*}\PYG{n}{t}\PYG{p}{)}\PYG{p}{)} \PYG{o}{\PYGZhy{}} \PYG{n}{v}

\PYG{n}{plt}\PYG{o}{.}\PYG{n}{plot}\PYG{p}{(}\PYG{n}{c}\PYG{p}{,}\PYG{n}{f}\PYG{p}{)}
\PYG{n}{plt}\PYG{o}{.}\PYG{n}{plot}\PYG{p}{(}\PYG{n}{c}\PYG{p}{,}\PYG{l+m+mi}{0}\PYG{o}{*}\PYG{n}{c}\PYG{p}{,}\PYG{l+s+s1}{\PYGZsq{}}\PYG{l+s+s1}{r\PYGZhy{}\PYGZhy{}}\PYG{l+s+s1}{\PYGZsq{}}\PYG{p}{)}
\PYG{n}{plt}\PYG{o}{.}\PYG{n}{grid}\PYG{p}{(}\PYG{p}{)}
\end{sphinxVerbatim}

\noindent\sphinxincludegraphics{{aula-03-analise-grafica_5_0}.png}

\begin{sphinxVerbatim}[commandchars=\\\{\}]
\PYG{c+c1}{\PYGZsh{} refinamento}
\PYG{n}{a}\PYG{p}{,}\PYG{n}{b} \PYG{o}{=} \PYG{l+m+mi}{14}\PYG{p}{,}\PYG{l+m+mi}{16}
\PYG{n}{c} \PYG{o}{=} \PYG{n}{np}\PYG{o}{.}\PYG{n}{linspace}\PYG{p}{(}\PYG{n}{a}\PYG{p}{,}\PYG{n}{b}\PYG{p}{,}\PYG{l+m+mi}{100}\PYG{p}{)}
\PYG{n}{f} \PYG{o}{=} \PYG{n}{g}\PYG{o}{*}\PYG{n}{m}\PYG{o}{/}\PYG{n}{c}\PYG{o}{*}\PYG{p}{(}\PYG{l+m+mi}{1} \PYG{o}{\PYGZhy{}} \PYG{n}{np}\PYG{o}{.}\PYG{n}{exp}\PYG{p}{(}\PYG{o}{\PYGZhy{}}\PYG{n}{c}\PYG{o}{/}\PYG{n}{m}\PYG{o}{*}\PYG{n}{t}\PYG{p}{)}\PYG{p}{)} \PYG{o}{\PYGZhy{}} \PYG{n}{v}

\PYG{n}{plt}\PYG{o}{.}\PYG{n}{plot}\PYG{p}{(}\PYG{n}{c}\PYG{p}{,}\PYG{n}{f}\PYG{p}{)}
\PYG{n}{plt}\PYG{o}{.}\PYG{n}{plot}\PYG{p}{(}\PYG{n}{c}\PYG{p}{,}\PYG{l+m+mi}{0}\PYG{o}{*}\PYG{n}{c}\PYG{p}{,}\PYG{l+s+s1}{\PYGZsq{}}\PYG{l+s+s1}{r\PYGZhy{}\PYGZhy{}}\PYG{l+s+s1}{\PYGZsq{}}\PYG{p}{)}
\PYG{n}{plt}\PYG{o}{.}\PYG{n}{grid}\PYG{p}{(}\PYG{p}{)}
\end{sphinxVerbatim}

\noindent\sphinxincludegraphics{{aula-03-analise-grafica_6_0}.png}

\begin{sphinxVerbatim}[commandchars=\\\{\}]
\PYG{c+c1}{\PYGZsh{} refinamento}
\PYG{n}{a}\PYG{p}{,}\PYG{n}{b} \PYG{o}{=} \PYG{l+m+mf}{14.75}\PYG{p}{,}\PYG{l+m+mf}{15.5}
\PYG{n}{c} \PYG{o}{=} \PYG{n}{np}\PYG{o}{.}\PYG{n}{linspace}\PYG{p}{(}\PYG{n}{a}\PYG{p}{,}\PYG{n}{b}\PYG{p}{,}\PYG{l+m+mi}{100}\PYG{p}{)}
\PYG{n}{f} \PYG{o}{=} \PYG{n}{g}\PYG{o}{*}\PYG{n}{m}\PYG{o}{/}\PYG{n}{c}\PYG{o}{*}\PYG{p}{(}\PYG{l+m+mi}{1} \PYG{o}{\PYGZhy{}} \PYG{n}{np}\PYG{o}{.}\PYG{n}{exp}\PYG{p}{(}\PYG{o}{\PYGZhy{}}\PYG{n}{c}\PYG{o}{/}\PYG{n}{m}\PYG{o}{*}\PYG{n}{t}\PYG{p}{)}\PYG{p}{)} \PYG{o}{\PYGZhy{}} \PYG{n}{v}

\PYG{n}{plt}\PYG{o}{.}\PYG{n}{plot}\PYG{p}{(}\PYG{n}{c}\PYG{p}{,}\PYG{n}{f}\PYG{p}{)}
\PYG{n}{plt}\PYG{o}{.}\PYG{n}{plot}\PYG{p}{(}\PYG{n}{c}\PYG{p}{,}\PYG{l+m+mi}{0}\PYG{o}{*}\PYG{n}{c}\PYG{p}{,}\PYG{l+s+s1}{\PYGZsq{}}\PYG{l+s+s1}{r\PYGZhy{}\PYGZhy{}}\PYG{l+s+s1}{\PYGZsq{}}\PYG{p}{)}
\PYG{n}{plt}\PYG{o}{.}\PYG{n}{grid}\PYG{p}{(}\PYG{p}{)}
\end{sphinxVerbatim}

\noindent\sphinxincludegraphics{{aula-03-analise-grafica_7_0}.png}

\begin{sphinxVerbatim}[commandchars=\\\{\}]
\PYG{c+c1}{\PYGZsh{} refinamento}
\PYG{n}{a}\PYG{p}{,}\PYG{n}{b} \PYG{o}{=} \PYG{l+m+mf}{15.1}\PYG{p}{,}\PYG{l+m+mf}{15.2}
\PYG{n}{c} \PYG{o}{=} \PYG{n}{np}\PYG{o}{.}\PYG{n}{linspace}\PYG{p}{(}\PYG{n}{a}\PYG{p}{,}\PYG{n}{b}\PYG{p}{,}\PYG{l+m+mi}{100}\PYG{p}{)}
\PYG{n}{f} \PYG{o}{=} \PYG{n}{g}\PYG{o}{*}\PYG{n}{m}\PYG{o}{/}\PYG{n}{c}\PYG{o}{*}\PYG{p}{(}\PYG{l+m+mi}{1} \PYG{o}{\PYGZhy{}} \PYG{n}{np}\PYG{o}{.}\PYG{n}{exp}\PYG{p}{(}\PYG{o}{\PYGZhy{}}\PYG{n}{c}\PYG{o}{/}\PYG{n}{m}\PYG{o}{*}\PYG{n}{t}\PYG{p}{)}\PYG{p}{)} \PYG{o}{\PYGZhy{}} \PYG{n}{v}

\PYG{n}{plt}\PYG{o}{.}\PYG{n}{plot}\PYG{p}{(}\PYG{n}{c}\PYG{p}{,}\PYG{n}{f}\PYG{p}{)}
\PYG{n}{plt}\PYG{o}{.}\PYG{n}{plot}\PYG{p}{(}\PYG{n}{c}\PYG{p}{,}\PYG{l+m+mi}{0}\PYG{o}{*}\PYG{n}{c}\PYG{p}{,}\PYG{l+s+s1}{\PYGZsq{}}\PYG{l+s+s1}{r\PYGZhy{}\PYGZhy{}}\PYG{l+s+s1}{\PYGZsq{}}\PYG{p}{)}
\PYG{n}{plt}\PYG{o}{.}\PYG{n}{grid}\PYG{p}{(}\PYG{p}{)}
\end{sphinxVerbatim}

\noindent\sphinxincludegraphics{{aula-03-analise-grafica_8_0}.png}

\begin{sphinxVerbatim}[commandchars=\\\{\}]
\PYG{c+c1}{\PYGZsh{} refinamento}
\PYG{n}{a}\PYG{p}{,}\PYG{n}{b} \PYG{o}{=} \PYG{l+m+mf}{15.12}\PYG{p}{,}\PYG{l+m+mf}{15.14}
\PYG{n}{c} \PYG{o}{=} \PYG{n}{np}\PYG{o}{.}\PYG{n}{linspace}\PYG{p}{(}\PYG{n}{a}\PYG{p}{,}\PYG{n}{b}\PYG{p}{,}\PYG{l+m+mi}{100}\PYG{p}{)}
\PYG{n}{f} \PYG{o}{=} \PYG{n}{g}\PYG{o}{*}\PYG{n}{m}\PYG{o}{/}\PYG{n}{c}\PYG{o}{*}\PYG{p}{(}\PYG{l+m+mi}{1} \PYG{o}{\PYGZhy{}} \PYG{n}{np}\PYG{o}{.}\PYG{n}{exp}\PYG{p}{(}\PYG{o}{\PYGZhy{}}\PYG{n}{c}\PYG{o}{/}\PYG{n}{m}\PYG{o}{*}\PYG{n}{t}\PYG{p}{)}\PYG{p}{)} \PYG{o}{\PYGZhy{}} \PYG{n}{v}

\PYG{n}{plt}\PYG{o}{.}\PYG{n}{plot}\PYG{p}{(}\PYG{n}{c}\PYG{p}{,}\PYG{n}{f}\PYG{p}{)}
\PYG{n}{plt}\PYG{o}{.}\PYG{n}{plot}\PYG{p}{(}\PYG{n}{c}\PYG{p}{,}\PYG{l+m+mi}{0}\PYG{o}{*}\PYG{n}{c}\PYG{p}{,}\PYG{l+s+s1}{\PYGZsq{}}\PYG{l+s+s1}{r\PYGZhy{}\PYGZhy{}}\PYG{l+s+s1}{\PYGZsq{}}\PYG{p}{)}
\PYG{n}{plt}\PYG{o}{.}\PYG{n}{axvline}\PYG{p}{(}\PYG{n}{x}\PYG{o}{=}\PYG{l+m+mf}{15.1278}\PYG{p}{)}
\PYG{n}{plt}\PYG{o}{.}\PYG{n}{axvline}\PYG{p}{(}\PYG{n}{x}\PYG{o}{=}\PYG{l+m+mf}{15.1275}\PYG{p}{)}
\PYG{n}{plt}\PYG{o}{.}\PYG{n}{grid}\PYG{p}{(}\PYG{p}{)}
\end{sphinxVerbatim}

\noindent\sphinxincludegraphics{{aula-03-analise-grafica_9_0}.png}


\section{Métodos iterativos}
\label{\detokenize{aula-03-analise-grafica:metodos-iterativos}}
\sphinxAtStartPar
A primeira estrutura de controle fundamental para computação numérica é o laço \sphinxcode{\sphinxupquote{for}}. Vamos ver como usá\sphinxhyphen{}lo para computar somatórias e produtórios.

\sphinxAtStartPar
Por exemplo, vamos computar:

\sphinxAtStartPar
\(s = \sum_{i=1}^n i = i + i + \ldots + i = s_1 + s_2 + \ldots + s_{n}\), para um valor \(n\) finito.

\sphinxAtStartPar
Podemos fazer isto da seguinte maneira:

\begin{sphinxVerbatim}[commandchars=\\\{\}]
\PYG{n}{n} \PYG{o}{=} \PYG{l+m+mi}{10}

\PYG{n}{s} \PYG{o}{=} \PYG{l+m+mf}{0.0}
\PYG{k}{for} \PYG{n}{i} \PYG{o+ow}{in} \PYG{n+nb}{range}\PYG{p}{(}\PYG{n}{n}\PYG{p}{)}\PYG{p}{:}
    \PYG{n}{s} \PYG{o}{+}\PYG{o}{=} \PYG{n}{i}
\end{sphinxVerbatim}

\sphinxAtStartPar
Detalhando… Considere o par \((i,s_i)\). Há soma abaixo?

\begin{sphinxVerbatim}[commandchars=\\\{\}]
\PYG{n}{s} \PYG{o}{=} \PYG{l+m+mf}{0.}
\PYG{k}{for} \PYG{n}{i} \PYG{o+ow}{in} \PYG{n+nb}{range}\PYG{p}{(}\PYG{n}{n}\PYG{p}{)}\PYG{p}{:}
    \PYG{n+nb}{print}\PYG{p}{(}\PYG{n}{i}\PYG{o}{+}\PYG{l+m+mi}{1}\PYG{p}{,}\PYG{n}{s}\PYG{p}{)}
\end{sphinxVerbatim}

\begin{sphinxVerbatim}[commandchars=\\\{\}]
1 0.0
2 0.0
3 0.0
4 0.0
5 0.0
6 0.0
7 0.0
8 0.0
9 0.0
10 0.0
\end{sphinxVerbatim}

\begin{sphinxVerbatim}[commandchars=\\\{\}]
\PYG{c+c1}{\PYGZsh{} Aqui há um incremento constante de 1}
\PYG{n}{s} \PYG{o}{=} \PYG{l+m+mf}{0.}
\PYG{k}{for} \PYG{n}{i} \PYG{o+ow}{in} \PYG{n+nb}{range}\PYG{p}{(}\PYG{n}{n}\PYG{p}{)}\PYG{p}{:}
    \PYG{n}{s} \PYG{o}{=} \PYG{n}{s} \PYG{o}{+} \PYG{l+m+mi}{1}
    \PYG{n+nb}{print}\PYG{p}{(}\PYG{n}{i}\PYG{o}{+}\PYG{l+m+mi}{1}\PYG{p}{,}\PYG{n}{s}\PYG{p}{)}
\end{sphinxVerbatim}

\begin{sphinxVerbatim}[commandchars=\\\{\}]
1 1.0
2 2.0
3 3.0
4 4.0
5 5.0
6 6.0
7 7.0
8 8.0
9 9.0
10 10.0
\end{sphinxVerbatim}

\sphinxAtStartPar
O exemplo anterior também poderia ser reproduzido como:

\begin{sphinxVerbatim}[commandchars=\\\{\}]
\PYG{n}{s} \PYG{o}{=} \PYG{l+m+mf}{0.}
\PYG{k}{for} \PYG{n}{i} \PYG{o+ow}{in} \PYG{n+nb}{range}\PYG{p}{(}\PYG{n}{n}\PYG{p}{)}\PYG{p}{:}
    \PYG{n}{s} \PYG{o}{+}\PYG{o}{=} \PYG{l+m+mi}{1} \PYG{c+c1}{\PYGZsh{} produz o mesmo que s \PYGZlt{}\PYGZhy{} s + 1}
    \PYG{n+nb}{print}\PYG{p}{(}\PYG{n}{i}\PYG{o}{+}\PYG{l+m+mi}{1}\PYG{p}{,}\PYG{n}{s}\PYG{p}{)}
\end{sphinxVerbatim}

\begin{sphinxVerbatim}[commandchars=\\\{\}]
1 1.0
2 2.0
3 3.0
4 4.0
5 5.0
6 6.0
7 7.0
8 8.0
9 9.0
10 10.0
\end{sphinxVerbatim}

\sphinxAtStartPar
No próximo exemplo, o incremento não é mais constante:

\begin{sphinxVerbatim}[commandchars=\\\{\}]
\PYG{n}{s} \PYG{o}{=} \PYG{l+m+mf}{0.}
\PYG{k}{for} \PYG{n}{i} \PYG{o+ow}{in} \PYG{n+nb}{range}\PYG{p}{(}\PYG{n}{n}\PYG{p}{)}\PYG{p}{:}
    \PYG{n}{s} \PYG{o}{+}\PYG{o}{=} \PYG{n}{i}
    \PYG{n+nb}{print}\PYG{p}{(}\PYG{n}{i}\PYG{o}{+}\PYG{l+m+mi}{1}\PYG{p}{,}\PYG{n}{s}\PYG{p}{)}
\end{sphinxVerbatim}

\begin{sphinxVerbatim}[commandchars=\\\{\}]
1 0.0
2 1.0
3 3.0
4 6.0
5 10.0
6 15.0
7 21.0
8 28.0
9 36.0
10 45.0
\end{sphinxVerbatim}


\section{Determinação de raízes por força bruta}
\label{\detokenize{aula-03-analise-grafica:determinacao-de-raizes-por-forca-bruta}}
\sphinxAtStartPar
No computador, sabemos que uma função matemática \(f(x)\) pode ser representada de duas formas principais:
\begin{itemize}
\item {} 
\sphinxAtStartPar
através de uma função programada (em Python, por exemplo) que retorna o valor da função para um dado argumento

\item {} 
\sphinxAtStartPar
uma coleção de pontos \((x,f(x))\) na forma de uma tabela.

\end{itemize}

\sphinxAtStartPar
A segunda forma é bem mais útil para análise gráfica. Esta forma é também adequada para resolver problemas de determinação de raízes e de otimização com simplicidade. No primeiro caso, basta pesquisar todos os pontos e procurar onde a função cruza o eixo \(x\), como fizemos anteriormente. No segundo caso, buscamos um ponto de mínimo ou máximo local, ou global.

\sphinxAtStartPar
Abordagens que seguem esse caminho podem chegar a examinar uma grande quantidade de pontos. Por essa razão, são chamados de métodos de \sphinxstyleemphasis{força bruta}, isto é, não seguem uma técnica elaborada.


\subsection{Algoritmo numérico}
\label{\detokenize{aula-03-analise-grafica:algoritmo-numerico}}
\sphinxAtStartPar
Em geral, queremos resolver o problema \(f(x) = 0\) especialmente quando \(f\) é não\sphinxhyphen{}linear. Para isso, desejamos encontrar os \(x\) onde \(f\) cruza o eixo. Um algoritmo em força bruta deverá percorrer todos os pontos sobre a curva e verificar se um ponto está abaixo do eixo e seu sucessor imediato está acima, ou vice\sphinxhyphen{}versa. Se isto ocorrer, então deve haver uma raiz neste intervalo.

\sphinxAtStartPar
\sphinxstylestrong{Algoritmo.} Dado um conjunto de \(n+1\) pontos \((x_i,y_i)\), \(y_i = f(x_i), \, i = 0,\ldots,n\), onde \(x_0 < \ldots < x_n\). Verificamos se \(y_i < 0\) e se \(y_{i+1} > 0\). Uma expressão compacta para esta checagem é o teste \(y_i \, y_{i+1} < 0\). Se o produto for negativo, então a raiz de \(f\) está no intervalo \([x_i,x_{i+1}]\). Assumindo uma variação linear entre os pontos, temos a aproximação
\begin{equation*}
\begin{split}f(x) \approx \dfrac{ y_{i+1} - y_i }{ x_{i+1} - x_i }(x - x_i) + y_i.\end{split}
\end{equation*}
\sphinxAtStartPar
Logo, \(f(x) = 0\) implica que a raiz é
\begin{equation*}
\begin{split}x = x_i - \dfrac{ x_{i+1} - x_i }{ y_{i+1} - y_i }y_i.\end{split}
\end{equation*}
\sphinxAtStartPar
\sphinxstylestrong{Exemplo.} Encontre a raiz da função \(f(x) = \exp(-x^2)\cos(3x)\) usando o algoritmo de força bruta.

\sphinxAtStartPar
Vamos plotar esta função apenas para visualizar seu comportamento.

\begin{sphinxVerbatim}[commandchars=\\\{\}]
\PYG{k+kn}{from} \PYG{n+nn}{numpy} \PYG{k+kn}{import} \PYG{n}{exp}\PYG{p}{,} \PYG{n}{cos}

\PYG{n}{f} \PYG{o}{=} \PYG{k}{lambda} \PYG{n}{x}\PYG{p}{:} \PYG{n}{exp}\PYG{p}{(}\PYG{o}{\PYGZhy{}}\PYG{n}{x}\PYG{o}{*}\PYG{o}{*}\PYG{l+m+mi}{2}\PYG{p}{)}\PYG{o}{*}\PYG{n}{cos}\PYG{p}{(}\PYG{l+m+mi}{3}\PYG{o}{*}\PYG{n}{x}\PYG{p}{)}
\PYG{n}{x} \PYG{o}{=} \PYG{n}{np}\PYG{o}{.}\PYG{n}{linspace}\PYG{p}{(}\PYG{l+m+mi}{0}\PYG{p}{,}\PYG{l+m+mi}{4}\PYG{p}{,}\PYG{l+m+mi}{1000}\PYG{p}{)}
\PYG{n}{plt}\PYG{o}{.}\PYG{n}{plot}\PYG{p}{(}\PYG{n}{x}\PYG{p}{,}\PYG{n}{f}\PYG{p}{(}\PYG{n}{x}\PYG{p}{)}\PYG{p}{)}\PYG{p}{;} \PYG{n}{plt}\PYG{o}{.}\PYG{n}{grid}\PYG{p}{(}\PYG{p}{)}
\end{sphinxVerbatim}

\noindent\sphinxincludegraphics{{aula-03-analise-grafica_24_0}.png}

\sphinxAtStartPar
Nesta plotagem, vemos claramente que a função possui duas raizes: uma próxima de \(x = 0.5\) e outra em \(x = \pi/6\).

\sphinxAtStartPar
Implementemos o algoritmo.

\begin{sphinxVerbatim}[commandchars=\\\{\}]
\PYG{k}{def} \PYG{n+nf}{forca\PYGZus{}bruta}\PYG{p}{(}\PYG{n}{f}\PYG{p}{,}\PYG{n}{a}\PYG{p}{,}\PYG{n}{b}\PYG{p}{,}\PYG{n}{n}\PYG{p}{)}\PYG{p}{:}
    \PYG{k+kn}{from} \PYG{n+nn}{numpy} \PYG{k+kn}{import} \PYG{n}{linspace}
    \PYG{n}{x} \PYG{o}{=} \PYG{n}{linspace}\PYG{p}{(}\PYG{n}{a}\PYG{p}{,}\PYG{n}{b}\PYG{p}{,}\PYG{n}{n}\PYG{p}{)}
    \PYG{n}{y} \PYG{o}{=} \PYG{n}{f}\PYG{p}{(}\PYG{n}{x}\PYG{p}{)}
    \PYG{n}{raizes} \PYG{o}{=} \PYG{p}{[}\PYG{p}{]}
    \PYG{k}{for} \PYG{n}{i} \PYG{o+ow}{in} \PYG{n+nb}{range}\PYG{p}{(}\PYG{n}{n}\PYG{o}{\PYGZhy{}}\PYG{l+m+mi}{1}\PYG{p}{)}\PYG{p}{:}
        \PYG{k}{if} \PYG{n}{y}\PYG{p}{[}\PYG{n}{i}\PYG{p}{]}\PYG{o}{*}\PYG{n}{y}\PYG{p}{[}\PYG{n}{i}\PYG{o}{+}\PYG{l+m+mi}{1}\PYG{p}{]} \PYG{o}{\PYGZlt{}} \PYG{l+m+mi}{0}\PYG{p}{:}
            \PYG{n}{raiz} \PYG{o}{=} \PYG{n}{x}\PYG{p}{[}\PYG{n}{i}\PYG{p}{]} \PYG{o}{\PYGZhy{}} \PYG{p}{(}\PYG{n}{x}\PYG{p}{[}\PYG{n}{i}\PYG{o}{+}\PYG{l+m+mi}{1}\PYG{p}{]} \PYG{o}{\PYGZhy{}} \PYG{n}{x}\PYG{p}{[}\PYG{n}{i}\PYG{p}{]}\PYG{p}{)}\PYG{o}{/}\PYG{p}{(}\PYG{n}{y}\PYG{p}{[}\PYG{n}{i}\PYG{o}{+}\PYG{l+m+mi}{1}\PYG{p}{]} \PYG{o}{\PYGZhy{}} \PYG{n}{y}\PYG{p}{[}\PYG{n}{i}\PYG{p}{]}\PYG{p}{)}\PYG{o}{*}\PYG{n}{y}\PYG{p}{[}\PYG{n}{i}\PYG{p}{]}
            \PYG{n}{raizes}\PYG{o}{.}\PYG{n}{append}\PYG{p}{(}\PYG{n}{raiz}\PYG{p}{)}
    \PYG{k}{if} \PYG{n+nb}{len}\PYG{p}{(}\PYG{n}{raizes}\PYG{p}{)} \PYG{o}{==} \PYG{l+m+mi}{0}\PYG{p}{:}               
        \PYG{n+nb}{print}\PYG{p}{(}\PYG{l+s+s1}{\PYGZsq{}}\PYG{l+s+s1}{Nenhuma raiz foi encontrada}\PYG{l+s+s1}{\PYGZsq{}}\PYG{p}{)}
    \PYG{k}{return} \PYG{n}{raizes}
\end{sphinxVerbatim}

\sphinxAtStartPar
Agora aplicamos o algoritmo na mesma função.

\begin{sphinxVerbatim}[commandchars=\\\{\}]
\PYG{n}{a}\PYG{p}{,}\PYG{n}{b}\PYG{p}{,}\PYG{n}{n} \PYG{o}{=} \PYG{l+m+mi}{0}\PYG{p}{,}\PYG{l+m+mi}{4}\PYG{p}{,}\PYG{l+m+mi}{1000}
\PYG{n}{raizes} \PYG{o}{=} \PYG{n}{forca\PYGZus{}bruta}\PYG{p}{(}\PYG{n}{f}\PYG{p}{,}\PYG{n}{a}\PYG{p}{,}\PYG{n}{b}\PYG{p}{,}\PYG{n}{n}\PYG{p}{)}
\PYG{n+nb}{print}\PYG{p}{(}\PYG{n}{raizes}\PYG{p}{)}
\end{sphinxVerbatim}

\begin{sphinxVerbatim}[commandchars=\\\{\}]
[0.5236017411236913, 1.5708070694852787, 2.6180048381439596, 3.665219264613299]
\end{sphinxVerbatim}

\sphinxAtStartPar
Temos, na verdade, 4 raízes! Plotemos o gráfico ampliado no intervalo {[}2.5,3.8{]}.

\begin{sphinxVerbatim}[commandchars=\\\{\}]
\PYG{n}{x2} \PYG{o}{=} \PYG{n}{np}\PYG{o}{.}\PYG{n}{linspace}\PYG{p}{(}\PYG{l+m+mf}{2.5}\PYG{p}{,}\PYG{l+m+mf}{3.8}\PYG{p}{,}\PYG{l+m+mi}{100}\PYG{p}{)}
\PYG{n}{plt}\PYG{o}{.}\PYG{n}{plot}\PYG{p}{(}\PYG{n}{x2}\PYG{p}{,}\PYG{n}{f}\PYG{p}{(}\PYG{n}{x2}\PYG{p}{)}\PYG{p}{)}\PYG{p}{;} \PYG{n}{plt}\PYG{o}{.}\PYG{n}{grid}\PYG{p}{(}\PYG{p}{)}
\end{sphinxVerbatim}

\noindent\sphinxincludegraphics{{aula-03-analise-grafica_30_0}.png}

\sphinxAtStartPar
Conseguimos enxergar mais uma raiz. Agora, plotemos um pouco mais ampliado entre {[}3.6,3.7{]}.

\begin{sphinxVerbatim}[commandchars=\\\{\}]
\PYG{n}{x3} \PYG{o}{=} \PYG{n}{np}\PYG{o}{.}\PYG{n}{linspace}\PYG{p}{(}\PYG{l+m+mf}{3.6}\PYG{p}{,}\PYG{l+m+mf}{3.7}\PYG{p}{,}\PYG{l+m+mi}{100}\PYG{p}{)}
\PYG{n}{plt}\PYG{o}{.}\PYG{n}{plot}\PYG{p}{(}\PYG{n}{x3}\PYG{p}{,}\PYG{n}{f}\PYG{p}{(}\PYG{n}{x3}\PYG{p}{)}\PYG{p}{)}\PYG{p}{;} \PYG{n}{plt}\PYG{o}{.}\PYG{n}{grid}\PYG{p}{(}\PYG{p}{)}
\end{sphinxVerbatim}

\noindent\sphinxincludegraphics{{aula-03-analise-grafica_32_0}.png}

\sphinxAtStartPar
Dessa forma, podemos identificar que, de fato existe uma quarta raiz.

\sphinxAtStartPar
Este exemplo mostrou uma aplicação do método de força bruta para determinação de raízes. Para finalizar, podemos embelezar o gráfico.

\begin{sphinxVerbatim}[commandchars=\\\{\}]
\PYG{n}{r} \PYG{o}{=} \PYG{n}{np}\PYG{o}{.}\PYG{n}{array}\PYG{p}{(}\PYG{n}{raizes}\PYG{p}{)} \PYG{c+c1}{\PYGZsh{} vetoriza a lista}
\PYG{n}{plt}\PYG{o}{.}\PYG{n}{plot}\PYG{p}{(}\PYG{n}{x}\PYG{p}{,}\PYG{l+m+mi}{0}\PYG{o}{*}\PYG{n}{f}\PYG{p}{(}\PYG{n}{x}\PYG{p}{)}\PYG{p}{,}\PYG{l+s+s1}{\PYGZsq{}}\PYG{l+s+s1}{r:}\PYG{l+s+s1}{\PYGZsq{}}\PYG{p}{,}\PYG{n}{x}\PYG{p}{,}\PYG{n}{f}\PYG{p}{(}\PYG{n}{x}\PYG{p}{)}\PYG{p}{,}\PYG{l+s+s1}{\PYGZsq{}}\PYG{l+s+s1}{\PYGZhy{}}\PYG{l+s+s1}{\PYGZsq{}}\PYG{p}{,}\PYG{n}{r}\PYG{p}{,}\PYG{n}{np}\PYG{o}{.}\PYG{n}{zeros}\PYG{p}{(}\PYG{l+m+mi}{4}\PYG{p}{)}\PYG{p}{,}\PYG{l+s+s1}{\PYGZsq{}}\PYG{l+s+s1}{ok}\PYG{l+s+s1}{\PYGZsq{}}\PYG{p}{,}\PYG{p}{)}
\PYG{n}{plt}\PYG{o}{.}\PYG{n}{xlabel}\PYG{p}{(}\PYG{l+s+s1}{\PYGZsq{}}\PYG{l+s+s1}{\PYGZdl{}x\PYGZdl{}}\PYG{l+s+s1}{\PYGZsq{}}\PYG{p}{,}\PYG{n}{fontsize}\PYG{o}{=}\PYG{l+m+mi}{14}\PYG{p}{)}
\PYG{n}{plt}\PYG{o}{.}\PYG{n}{ylabel}\PYG{p}{(}\PYG{l+s+s1}{\PYGZsq{}}\PYG{l+s+s1}{\PYGZdl{}f(x)\PYGZdl{}}\PYG{l+s+s1}{\PYGZsq{}}\PYG{p}{,}\PYG{n}{fontsize}\PYG{o}{=}\PYG{l+m+mi}{14}\PYG{p}{)}        
\PYG{n}{plt}\PYG{o}{.}\PYG{n}{grid}\PYG{p}{(}\PYG{p}{)}
\PYG{n}{plt}\PYG{o}{.}\PYG{n}{title}\PYG{p}{(}\PYG{l+s+s1}{\PYGZsq{}}\PYG{l+s+s1}{Raízes de \PYGZdl{}}\PYG{l+s+s1}{\PYGZbs{}}\PYG{l+s+s1}{exp(\PYGZhy{}x\PYGZca{}2)}\PYG{l+s+s1}{\PYGZbs{}}\PYG{l+s+s1}{cos(3x)\PYGZdl{}}\PYG{l+s+s1}{\PYGZsq{}}\PYG{p}{)}\PYG{p}{;}
\end{sphinxVerbatim}

\noindent\sphinxincludegraphics{{aula-03-analise-grafica_35_0}.png}


\chapter{Implementações do método da bisseção}
\label{\detokenize{aula-04-bissecao:implementacoes-do-metodo-da-bissecao}}\label{\detokenize{aula-04-bissecao::doc}}
\begin{sphinxVerbatim}[commandchars=\\\{\}]
\PYG{o}{\PYGZpc{}}\PYG{k}{matplotlib} inline
\end{sphinxVerbatim}

\begin{sphinxVerbatim}[commandchars=\\\{\}]
\PYG{l+s+sd}{\PYGZdq{}\PYGZdq{}\PYGZdq{}MB Metodo da bissecao para funcoes unidimensionais}
\PYG{l+s+sd}{entrada: }
\PYG{l+s+sd}{    f: uma string dependendo de x, i.e., a funcao}
\PYG{l+s+sd}{           (e.g., \PYGZsq{}x\PYGZca{}2 + 1\PYGZsq{}, \PYGZsq{}x\PYGZca{}2*cos(x)\PYGZsq{}, etc.) }
\PYG{l+s+sd}{    a: limite inferior do dominio}
\PYG{l+s+sd}{    b: limite superior do dominio}
\PYG{l+s+sd}{    tol: tolerancia    }
\PYG{l+s+sd}{    N: numero maximo de iteracoes do metodo}

\PYG{l+s+sd}{saida: }
\PYG{l+s+sd}{    xm: raiz da funcao}
\PYG{l+s+sd}{\PYGZdq{}\PYGZdq{}\PYGZdq{}}

\PYG{k+kn}{import} \PYG{n+nn}{inspect}\PYG{o}{,} \PYG{n+nn}{re}

\PYG{k}{def} \PYG{n+nf}{bissecao}\PYG{p}{(}\PYG{n}{f}\PYG{p}{,}\PYG{n}{a}\PYG{p}{,}\PYG{n}{b}\PYG{p}{,}\PYG{n}{tol}\PYG{p}{,}\PYG{n}{N}\PYG{p}{,}\PYG{n}{var}\PYG{p}{)}\PYG{p}{:}
        
    \PYG{c+c1}{\PYGZsh{} TODO identificar a variável usada na função }
    \PYG{c+c1}{\PYGZsh{}      Aqui, tentei assumir que apenas uma era usada (e.g. \PYGZsq{}x\PYGZsq{}),}
    \PYG{c+c1}{\PYGZsh{}      mas foi complicado generalizar quando há objeto numpy}
    \PYG{c+c1}{\PYGZsh{}var = re.search(\PYGZsq{}[a\PYGZhy{}zA\PYGZhy{}Z]+\PYGZsq{},f)}
    \PYG{c+c1}{\PYGZsh{}var = var.group()}

    \PYG{c+c1}{\PYGZsh{} cria função anônima}
    \PYG{n}{f} \PYG{o}{=} \PYG{n+nb}{eval}\PYG{p}{(}\PYG{l+s+s1}{\PYGZsq{}}\PYG{l+s+s1}{lambda }\PYG{l+s+s1}{\PYGZsq{}} \PYG{o}{+} \PYG{n}{var} \PYG{o}{+} \PYG{l+s+s1}{\PYGZsq{}}\PYG{l+s+s1}{ :}\PYG{l+s+s1}{\PYGZsq{}} \PYG{o}{+} \PYG{n}{f}\PYG{p}{)}

    \PYG{c+c1}{\PYGZsh{} Se função não for de uma variável, lança erro.}
    \PYG{c+c1}{\PYGZsh{} Mais aplicável se o caso geral fosse implementado.        }
    \PYG{k}{if} \PYG{n+nb}{len}\PYG{p}{(}\PYG{n}{inspect}\PYG{o}{.}\PYG{n}{getfullargspec}\PYG{p}{(}\PYG{n}{f}\PYG{p}{)}\PYG{o}{.}\PYG{n}{args}\PYG{p}{)} \PYG{o}{\PYGZhy{}} \PYG{l+m+mi}{1} \PYG{o}{\PYGZgt{}} \PYG{l+m+mi}{0}\PYG{p}{:}    
        \PYG{k}{raise} \PYG{n+ne}{ValueError}\PYG{p}{(}\PYG{l+s+s1}{\PYGZsq{}}\PYG{l+s+s1}{O código é válido apenas para uma variável.}\PYG{l+s+s1}{\PYGZsq{}}\PYG{p}{)}

    \PYG{c+c1}{\PYGZsh{} calcula valor da função nos extremos}
    \PYG{n}{fa} \PYG{o}{=} \PYG{n}{f}\PYG{p}{(}\PYG{n}{a}\PYG{p}{)} 
    \PYG{n}{fb} \PYG{o}{=} \PYG{n}{f}\PYG{p}{(}\PYG{n}{b}\PYG{p}{)}
    
    \PYG{c+c1}{\PYGZsh{} verifica sinal da função para o intervalo passado     }
    \PYG{k}{if} \PYG{n}{fa}\PYG{o}{*}\PYG{n}{fb} \PYG{o}{\PYGZgt{}}\PYG{o}{=} \PYG{l+m+mi}{0}\PYG{p}{:}
        \PYG{k}{raise} \PYG{n+ne}{ValueError}\PYG{p}{(}\PYG{l+s+s1}{\PYGZsq{}}\PYG{l+s+s1}{A função deve ter sinais opostos em a e b!}\PYG{l+s+s1}{\PYGZsq{}}\PYG{p}{)}
    
    \PYG{c+c1}{\PYGZsh{} flag usada para prevenir a obtenção da raiz }
    \PYG{c+c1}{\PYGZsh{} antes de o intervalo ter sido }
    \PYG{c+c1}{\PYGZsh{} suficientemente reduzido}
    \PYG{n}{done} \PYG{o}{=} \PYG{l+m+mi}{0}\PYG{p}{;}

    \PYG{c+c1}{\PYGZsh{} loop principal}

    \PYG{c+c1}{\PYGZsh{} bisecta o intervalo}
    \PYG{n}{xm} \PYG{o}{=} \PYG{p}{(}\PYG{n}{a}\PYG{o}{+}\PYG{n}{b}\PYG{p}{)}\PYG{o}{/}\PYG{l+m+mi}{2}

    \PYG{n}{i} \PYG{o}{=} \PYG{l+m+mi}{1} \PYG{c+c1}{\PYGZsh{} contador }

    \PYG{k}{while} \PYG{n+nb}{abs}\PYG{p}{(}\PYG{n}{a}\PYG{o}{\PYGZhy{}}\PYG{n}{b}\PYG{p}{)} \PYG{o}{\PYGZgt{}} \PYG{n}{tol} \PYG{o+ow}{and} \PYG{p}{(} \PYG{o+ow}{not} \PYG{n}{done} \PYG{o+ow}{or} \PYG{n}{N} \PYG{o}{!=} \PYG{l+m+mi}{0} \PYG{p}{)}\PYG{p}{:}
    \PYG{c+c1}{\PYGZsh{} avalia a função no ponto médio}
        \PYG{n}{fxm} \PYG{o}{=} \PYG{n}{f}\PYG{p}{(}\PYG{n}{xm}\PYG{p}{)}
        \PYG{n+nb}{print}\PYG{p}{(}\PYG{l+s+s2}{\PYGZdq{}}\PYG{l+s+s2}{(i = }\PYG{l+s+si}{\PYGZob{}0:d\PYGZcb{}}\PYG{l+s+s2}{) f(xm)=}\PYG{l+s+si}{\PYGZob{}1:f\PYGZcb{}}\PYG{l+s+s2}{ | f(a)=}\PYG{l+s+si}{\PYGZob{}2:f\PYGZcb{}}\PYG{l+s+s2}{ | f(b)=}\PYG{l+s+si}{\PYGZob{}3:f\PYGZcb{}}\PYG{l+s+s2}{\PYGZdq{}}\PYG{o}{.}\PYG{n}{format}\PYG{p}{(}\PYG{n}{i}\PYG{p}{,}\PYG{n}{fxm}\PYG{p}{,}\PYG{n}{fa}\PYG{p}{,}\PYG{n}{fb}\PYG{p}{)}\PYG{p}{)}
   
        \PYG{k}{if} \PYG{n}{fa}\PYG{o}{*}\PYG{n}{fxm} \PYG{o}{\PYGZlt{}} \PYG{l+m+mi}{0}\PYG{p}{:}       \PYG{c+c1}{\PYGZsh{} Raiz esta à esquerda de xm}
            \PYG{n}{b} \PYG{o}{=} \PYG{n}{xm}
            \PYG{n}{fb} \PYG{o}{=} \PYG{n}{fxm}
            \PYG{n}{xm} \PYG{o}{=} \PYG{p}{(}\PYG{n}{a}\PYG{o}{+}\PYG{n}{b}\PYG{p}{)}\PYG{o}{/}\PYG{l+m+mi}{2}
        \PYG{k}{elif} \PYG{n}{fxm}\PYG{o}{*}\PYG{n}{fb} \PYG{o}{\PYGZlt{}} \PYG{l+m+mi}{0}\PYG{p}{:}     \PYG{c+c1}{\PYGZsh{} Raiz esta à direita de xm}
            \PYG{n}{a} \PYG{o}{=} \PYG{n}{xm}
            \PYG{n}{fa} \PYG{o}{=} \PYG{n}{fxm}
            \PYG{n}{xm} \PYG{o}{=} \PYG{p}{(}\PYG{n}{a}\PYG{o}{+}\PYG{n}{b}\PYG{p}{)}\PYG{o}{/}\PYG{l+m+mi}{2}
        \PYG{k}{else}\PYG{p}{:}               \PYG{c+c1}{\PYGZsh{} Achamos a raiz}
            \PYG{n}{done} \PYG{o}{=} \PYG{l+m+mi}{1}
    
        \PYG{n}{N} \PYG{o}{\PYGZhy{}}\PYG{o}{=} \PYG{l+m+mi}{1}              \PYG{c+c1}{\PYGZsh{} Atualiza passo}
        \PYG{n}{i} \PYG{o}{+}\PYG{o}{=} \PYG{l+m+mi}{1}              \PYG{c+c1}{\PYGZsh{} Atualiza contador}
            
    \PYG{n+nb}{print}\PYG{p}{(}\PYG{l+s+s2}{\PYGZdq{}}\PYG{l+s+s2}{Solução encontrada: }\PYG{l+s+si}{\PYGZob{}0\PYGZcb{}}\PYG{l+s+s2}{\PYGZdq{}}\PYG{o}{.}\PYG{n}{format}\PYG{p}{(}\PYG{n}{xm}\PYG{p}{)}\PYG{p}{)}

    \PYG{k}{return} \PYG{n}{xm}
\end{sphinxVerbatim}

\begin{sphinxVerbatim}[commandchars=\\\{\}]
\PYG{k+kn}{import} \PYG{n+nn}{numpy} \PYG{k}{as} \PYG{n+nn}{np}
\PYG{k+kn}{import} \PYG{n+nn}{matplotlib}\PYG{n+nn}{.}\PYG{n+nn}{pyplot} \PYG{k}{as} \PYG{n+nn}{plt}
\PYG{k+kn}{from} \PYG{n+nn}{scipy}\PYG{n+nn}{.}\PYG{n+nn}{optimize} \PYG{k+kn}{import} \PYG{n}{bisect}\PYG{p}{,} \PYG{n}{newton}

\PYG{c+c1}{\PYGZsh{} Dados de entrada}

\PYG{n}{t} \PYG{o}{=} \PYG{n}{np}\PYG{o}{.}\PYG{n}{arange}\PYG{p}{(}\PYG{l+m+mi}{0}\PYG{p}{,}\PYG{l+m+mi}{520}\PYG{p}{,}\PYG{l+m+mi}{1}\PYG{p}{)}   \PYG{c+c1}{\PYGZsh{} tempo [s]}
\PYG{n}{c} \PYG{o}{=} \PYG{l+m+mf}{1.46}   \PYG{c+c1}{\PYGZsh{} coeficiente de arrasto [kg/s]}
\PYG{n}{m} \PYG{o}{=} \PYG{l+m+mi}{90}     \PYG{c+c1}{\PYGZsh{} massa [kg]}
\PYG{n}{g} \PYG{o}{=} \PYG{l+m+mf}{9.81}   \PYG{c+c1}{\PYGZsh{} constante de gravidade [m/s2]}

\PYG{c+c1}{\PYGZsh{} Dados de saída}

\PYG{c+c1}{\PYGZsh{}\PYGZsh{} velocidade terminal [m/s]}
\PYG{n}{v\PYGZus{}ms1} \PYG{o}{=} \PYG{p}{(}\PYG{n}{g}\PYG{o}{*}\PYG{n}{m}\PYG{o}{/}\PYG{n}{c}\PYG{p}{)}\PYG{o}{*}\PYG{p}{(}\PYG{l+m+mi}{1} \PYG{o}{\PYGZhy{}} \PYG{n}{np}\PYG{o}{.}\PYG{n}{exp}\PYG{p}{(}\PYG{p}{(}\PYG{o}{\PYGZhy{}}\PYG{n}{c}\PYG{o}{/}\PYG{n}{m}\PYG{p}{)}\PYG{o}{*}\PYG{n}{t}\PYG{p}{)}\PYG{p}{)}

\PYG{c+c1}{\PYGZsh{} velocidade terminal [km/h]}
\PYG{n}{v\PYGZus{}kh1} \PYG{o}{=} \PYG{p}{(}\PYG{l+m+mi}{1}\PYG{o}{/}\PYG{l+m+mf}{3.6}\PYG{p}{)}\PYG{o}{*}\PYG{n}{v\PYGZus{}ms1}\PYG{p}{;}

\PYG{c+c1}{\PYGZsh{} gráfico tempo x velocidade}
\PYG{n}{plt}\PYG{o}{.}\PYG{n}{figure}
\PYG{n}{plt}\PYG{o}{.}\PYG{n}{plot}\PYG{p}{(}\PYG{n}{t}\PYG{p}{,}\PYG{n}{v\PYGZus{}ms1}\PYG{p}{)}
\PYG{n}{plt}\PYG{o}{.}\PYG{n}{xlabel}\PYG{p}{(}\PYG{l+s+s1}{\PYGZsq{}}\PYG{l+s+s1}{t [s]}\PYG{l+s+s1}{\PYGZsq{}}\PYG{p}{)}
\PYG{n}{plt}\PYG{o}{.}\PYG{n}{ylabel}\PYG{p}{(}\PYG{l+s+s1}{\PYGZsq{}}\PYG{l+s+s1}{v [m/s]}\PYG{l+s+s1}{\PYGZsq{}}\PYG{p}{)}
\PYG{n}{plt}\PYG{o}{.}\PYG{n}{title}\PYG{p}{(}\PYG{l+s+s1}{\PYGZsq{}}\PYG{l+s+s1}{Velocidade terminal \PYGZhy{} paraquedista}\PYG{l+s+s1}{\PYGZsq{}}\PYG{p}{)}\PYG{p}{;}
\end{sphinxVerbatim}

\noindent\sphinxincludegraphics{{aula-04-bissecao_3_0}.png}

\begin{sphinxVerbatim}[commandchars=\\\{\}]
\PYG{k+kn}{import} \PYG{n+nn}{sympy} \PYG{k}{as} \PYG{n+nn}{sp}

\PYG{n}{time} \PYG{o}{=} \PYG{l+m+mi}{12}     \PYG{c+c1}{\PYGZsh{} tempo [s]}
\PYG{n}{mass} \PYG{o}{=} \PYG{l+m+mi}{70}     \PYG{c+c1}{\PYGZsh{} massa [kg]}
\PYG{n}{vel} \PYG{o}{=} \PYG{l+m+mi}{42}     \PYG{c+c1}{\PYGZsh{} velocidade [m/s]}
\PYG{n}{grav} \PYG{o}{=} \PYG{l+m+mf}{9.81}   \PYG{c+c1}{\PYGZsh{} constante de gravidade [m/s2]}

\PYG{c+c1}{\PYGZsh{} defino variáveis simbólicas}
\PYG{n}{g}\PYG{p}{,}\PYG{n}{m}\PYG{p}{,}\PYG{n}{t}\PYG{p}{,}\PYG{n}{v}\PYG{p}{,}\PYG{n}{c} \PYG{o}{=} \PYG{n}{sp}\PYG{o}{.}\PYG{n}{symbols}\PYG{p}{(}\PYG{l+s+s1}{\PYGZsq{}}\PYG{l+s+s1}{g,m,t,v,c}\PYG{l+s+s1}{\PYGZsq{}}\PYG{p}{)}

\PYG{c+c1}{\PYGZsh{} expressão geral}
\PYG{n}{f\PYGZus{}g} \PYG{o}{=} \PYG{p}{(}\PYG{n}{g}\PYG{o}{*}\PYG{n}{m}\PYG{o}{/}\PYG{n}{c}\PYG{p}{)}\PYG{o}{*}\PYG{p}{(}\PYG{l+m+mi}{1} \PYG{o}{\PYGZhy{}} \PYG{n}{sp}\PYG{o}{.}\PYG{n}{exp}\PYG{p}{(}\PYG{p}{(}\PYG{o}{\PYGZhy{}}\PYG{n}{c}\PYG{o}{/}\PYG{n}{m}\PYG{p}{)}\PYG{o}{*}\PYG{n}{t}\PYG{p}{)}\PYG{p}{)} \PYG{o}{\PYGZhy{}} \PYG{n}{v}

\PYG{c+c1}{\PYGZsh{} expressão particular com valores substituídos }
\PYG{c+c1}{\PYGZsh{} convertida para string}
\PYG{n}{f\PYGZus{}s} \PYG{o}{=} \PYG{n+nb}{str}\PYG{p}{(}\PYG{n}{f\PYGZus{}g}\PYG{o}{.}\PYG{n}{subs}\PYG{p}{(}\PYG{p}{\PYGZob{}}\PYG{l+s+s1}{\PYGZsq{}}\PYG{l+s+s1}{g}\PYG{l+s+s1}{\PYGZsq{}}\PYG{p}{:}\PYG{n}{grav}\PYG{p}{,}\PYG{l+s+s1}{\PYGZsq{}}\PYG{l+s+s1}{m}\PYG{l+s+s1}{\PYGZsq{}}\PYG{p}{:}\PYG{n}{mass}\PYG{p}{,}\PYG{l+s+s1}{\PYGZsq{}}\PYG{l+s+s1}{v}\PYG{l+s+s1}{\PYGZsq{}}\PYG{p}{:}\PYG{n}{vel}\PYG{p}{,}\PYG{l+s+s1}{\PYGZsq{}}\PYG{l+s+s1}{t}\PYG{l+s+s1}{\PYGZsq{}}\PYG{p}{:}\PYG{n}{time}\PYG{p}{\PYGZcb{}}\PYG{p}{)}\PYG{p}{)}

\PYG{c+c1}{\PYGZsh{} TODO}
\PYG{c+c1}{\PYGZsh{} para esta função, teremos que substituir \PYGZsq{}exp\PYGZsq{} por \PYGZsq{}np.exp\PYGZsq{})}
\PYG{n+nb}{print}\PYG{p}{(}\PYG{l+s+s1}{\PYGZsq{}}\PYG{l+s+s1}{f(c) = }\PYG{l+s+s1}{\PYGZsq{}} \PYG{o}{+} \PYG{n}{f\PYGZus{}s} \PYG{o}{+} \PYG{l+s+s1}{\PYGZsq{}}\PYG{l+s+se}{\PYGZbs{}n}\PYG{l+s+s1}{\PYGZsq{}}\PYG{p}{)}
\PYG{n}{f\PYGZus{}s} \PYG{o}{=} \PYG{l+s+s1}{\PYGZsq{}}\PYG{l+s+s1}{\PYGZhy{}42 + 686.7*(1 \PYGZhy{} np.exp(\PYGZhy{}6*c/35))/c}\PYG{l+s+s1}{\PYGZsq{}}

\PYG{c+c1}{\PYGZsh{} resolve bisseção}
\PYG{n}{xm} \PYG{o}{=} \PYG{n}{bissecao}\PYG{p}{(}\PYG{n}{f\PYGZus{}s}\PYG{p}{,}\PYG{l+m+mi}{12}\PYG{p}{,}\PYG{l+m+mi}{16}\PYG{p}{,}\PYG{l+m+mf}{1e\PYGZhy{}5}\PYG{p}{,}\PYG{l+m+mi}{100}\PYG{p}{,}\PYG{l+s+s1}{\PYGZsq{}}\PYG{l+s+s1}{c}\PYG{l+s+s1}{\PYGZsq{}}\PYG{p}{)}
\end{sphinxVerbatim}

\begin{sphinxVerbatim}[commandchars=\\\{\}]
f(c) = \PYGZhy{}42 + 686.7*(1 \PYGZhy{} exp(\PYGZhy{}6*c/35))/c

(i = 1) f(xm)=2.600284 | f(a)=7.910578 | f(b)=\PYGZhy{}1.844622
(i = 2) f(xm)=0.281205 | f(a)=2.600284 | f(b)=\PYGZhy{}1.844622
(i = 3) f(xm)=\PYGZhy{}0.804573 | f(a)=0.281205 | f(b)=\PYGZhy{}1.844622
(i = 4) f(xm)=\PYGZhy{}0.267556 | f(a)=0.281205 | f(b)=\PYGZhy{}0.804573
(i = 5) f(xm)=0.005337 | f(a)=0.281205 | f(b)=\PYGZhy{}0.267556
(i = 6) f(xm)=\PYGZhy{}0.131479 | f(a)=0.005337 | f(b)=\PYGZhy{}0.267556
(i = 7) f(xm)=\PYGZhy{}0.063164 | f(a)=0.005337 | f(b)=\PYGZhy{}0.131479
(i = 8) f(xm)=\PYGZhy{}0.028937 | f(a)=0.005337 | f(b)=\PYGZhy{}0.063164
(i = 9) f(xm)=\PYGZhy{}0.011806 | f(a)=0.005337 | f(b)=\PYGZhy{}0.028937
(i = 10) f(xm)=\PYGZhy{}0.003236 | f(a)=0.005337 | f(b)=\PYGZhy{}0.011806
(i = 11) f(xm)=0.001050 | f(a)=0.005337 | f(b)=\PYGZhy{}0.003236
(i = 12) f(xm)=\PYGZhy{}0.001093 | f(a)=0.001050 | f(b)=\PYGZhy{}0.003236
(i = 13) f(xm)=\PYGZhy{}0.000022 | f(a)=0.001050 | f(b)=\PYGZhy{}0.001093
(i = 14) f(xm)=0.000514 | f(a)=0.001050 | f(b)=\PYGZhy{}0.000022
(i = 15) f(xm)=0.000246 | f(a)=0.000514 | f(b)=\PYGZhy{}0.000022
(i = 16) f(xm)=0.000112 | f(a)=0.000246 | f(b)=\PYGZhy{}0.000022
(i = 17) f(xm)=0.000045 | f(a)=0.000112 | f(b)=\PYGZhy{}0.000022
(i = 18) f(xm)=0.000012 | f(a)=0.000045 | f(b)=\PYGZhy{}0.000022
(i = 19) f(xm)=\PYGZhy{}0.000005 | f(a)=0.000012 | f(b)=\PYGZhy{}0.000022
Solução encontrada: 15.127429962158203
\end{sphinxVerbatim}


\section{Tarefas}
\label{\detokenize{aula-04-bissecao:tarefas}}\begin{itemize}
\item {} 
\sphinxAtStartPar
Melhore o código Python tratando os TODOs:

\end{itemize}

\sphinxAtStartPar
Tente generalizar o código da bisseção para que identifique automaticamente a variável de entrada utilizada pelo usuário (use expressões regulares e remova o argumento \sphinxcode{\sphinxupquote{var}} da definição da função).

\sphinxAtStartPar
Note que o trecho simbólico abaixo foi necessário para substituir a função da chamada \sphinxcode{\sphinxupquote{exp}}, não interpretada por \sphinxcode{\sphinxupquote{eval}} por uma nova string que usasse \sphinxcode{\sphinxupquote{np.exp}}.

\begin{sphinxVerbatim}[commandchars=\\\{\}]
\PYG{c+c1}{\PYGZsh{} TODO}
\PYG{c+c1}{\PYGZsh{} para esta função, teremos que substituir \PYGZsq{}exp\PYGZsq{} por \PYGZsq{}np.exp\PYGZsq{})}
\PYG{n+nb}{print}\PYG{p}{(}\PYG{l+s+s1}{\PYGZsq{}}\PYG{l+s+s1}{f(c) = }\PYG{l+s+s1}{\PYGZsq{}} \PYG{o}{+} \PYG{n}{f\PYGZus{}s} \PYG{o}{+} \PYG{l+s+s1}{\PYGZsq{}}\PYG{l+s+se}{\PYGZbs{}n}\PYG{l+s+s1}{\PYGZsq{}}\PYG{p}{)}
\PYG{n}{f\PYGZus{}s} \PYG{o}{=} \PYG{l+s+s1}{\PYGZsq{}}\PYG{l+s+s1}{\PYGZhy{}42 + 686.7*(1 \PYGZhy{} np.exp(\PYGZhy{}6*c/35))/c}\PYG{l+s+s1}{\PYGZsq{}}
\end{sphinxVerbatim}

\sphinxAtStartPar
Tente fazer as correções necessárias no código. \sphinxstylestrong{Sugestão:} verifique a função \sphinxcode{\sphinxupquote{sympy.core.evalf}} do módulo \sphinxcode{\sphinxupquote{sympy}})
\begin{itemize}
\item {} 
\sphinxAtStartPar
Adicione mecanismos de plotagem no código Python

\item {} 
\sphinxAtStartPar
Crie um código em Javascript para adicionarmos na página do projeto Numbiosis com o máximo possível de GUI (labels + input data).

\item {} 
\sphinxAtStartPar
Teste a implementação com um problema realista.

\end{itemize}

\sphinxAtStartPar
\sphinxstylestrong{Problema sugerido:}
Uma reação química reversível
\begin{equation*}
\begin{split}2A+B \iff C\end{split}
\end{equation*}
\sphinxAtStartPar
pode ser caracterizada pela relação de equilíbrio
\begin{equation*}
\begin{split}K = \dfrac{c_c}{c_a^2c_b},\end{split}
\end{equation*}
\sphinxAtStartPar
onde a nomenclatura \(c_i\) representa a concentração do constituinte \(i\). Suponha que definamos uma variável \(x\) como o número de moles de \(C\) que são produzidos. A conservação da massa pode ser usada para reformular a relação de equilíbrio como
\begin{equation*}
\begin{split}K = \dfrac{(c_{c,0} + x)}{(c_{a,0} - 2x)^2 (c_{b,0} - x),}\end{split}
\end{equation*}
\sphinxAtStartPar
onde o subscrito \(0\) designa a concentração inicial de cada constituinte. Se \(K = 0,016\), \(c_{a,0} = 42\), \(c_{b,0} = 28\) e \(c_{c,0} = 4\), determine o valor de \(x\).

\sphinxAtStartPar
(a) Obtenha a solução graficamente.

\sphinxAtStartPar
(b) Com base em (a), resolva a raiz com suposições iniciais de \(x_l = 0\) e \(x_u = 20\), com critério de erro de \(\epsilon_s = 0,5\%\). (Vide clipping \sphinxstyleemphasis{Definições de erro} para entender \(\epsilon_s\).)

\sphinxAtStartPar
© Use o método da bisseção.


\section{Tarefa: Falsa Posição}
\label{\detokenize{aula-04-bissecao:tarefa-falsa-posicao}}
\sphinxAtStartPar
Programe uma nova função para executar o método da falsa posição ou estenda o código anterior para uma nova função que contemple os dois casos (sugestão: use \sphinxcode{\sphinxupquote{switch... case...}}).


\chapter{Método da Iteração Linear (Ponto Fixo)}
\label{\detokenize{aula-05-ponto-fixo:metodo-da-iteracao-linear-ponto-fixo}}\label{\detokenize{aula-05-ponto-fixo::doc}}
\sphinxAtStartPar
Este notebook explora aspectos do método da \sphinxstyleemphasis{iteração linear}, ou também chamado de método do \sphinxstyleemphasis{ponto fixo}.

\begin{sphinxVerbatim}[commandchars=\\\{\}]
\PYG{k+kn}{import} \PYG{n+nn}{numpy} \PYG{k}{as} \PYG{n+nn}{np}
\PYG{k+kn}{import} \PYG{n+nn}{matplotlib}\PYG{n+nn}{.}\PYG{n+nn}{pyplot} \PYG{k}{as} \PYG{n+nn}{plt} 
\PYG{o}{\PYGZpc{}}\PYG{k}{matplotlib} inline
\end{sphinxVerbatim}


\section{Exemplo}
\label{\detokenize{aula-05-ponto-fixo:exemplo}}
\sphinxAtStartPar
Estudamos a função \(f(x) = x^2 + x - 6\).

\begin{sphinxVerbatim}[commandchars=\\\{\}]
\PYG{n}{x} \PYG{o}{=} \PYG{n}{np}\PYG{o}{.}\PYG{n}{linspace}\PYG{p}{(}\PYG{o}{\PYGZhy{}}\PYG{l+m+mi}{4}\PYG{p}{,}\PYG{l+m+mi}{4}\PYG{p}{,}\PYG{l+m+mi}{50}\PYG{p}{)}
\PYG{n}{f} \PYG{o}{=} \PYG{k}{lambda} \PYG{n}{x}\PYG{p}{:} \PYG{n}{x}\PYG{o}{*}\PYG{o}{*}\PYG{l+m+mi}{2} \PYG{o}{+} \PYG{n}{x} \PYG{o}{\PYGZhy{}} \PYG{l+m+mi}{6}

\PYG{n}{xr} \PYG{o}{=} \PYG{n}{np}\PYG{o}{.}\PYG{n}{roots}\PYG{p}{(}\PYG{p}{[}\PYG{l+m+mi}{1}\PYG{p}{,}\PYG{l+m+mi}{1}\PYG{p}{,}\PYG{o}{\PYGZhy{}}\PYG{l+m+mi}{6}\PYG{p}{]}\PYG{p}{)}
\PYG{n+nb}{print}\PYG{p}{(}\PYG{l+s+s1}{\PYGZsq{}}\PYG{l+s+s1}{Raízes: x1 = }\PYG{l+s+si}{\PYGZob{}:f\PYGZcb{}}\PYG{l+s+s1}{, x2 = }\PYG{l+s+si}{\PYGZob{}:f\PYGZcb{}}\PYG{l+s+s1}{\PYGZsq{}}\PYG{o}{.}\PYG{n}{format}\PYG{p}{(}\PYG{n}{xr}\PYG{p}{[}\PYG{l+m+mi}{0}\PYG{p}{]}\PYG{p}{,} \PYG{n}{xr}\PYG{p}{[}\PYG{l+m+mi}{1}\PYG{p}{]}\PYG{p}{)}\PYG{p}{)}

\PYG{c+c1}{\PYGZsh{} função de iteração}
\PYG{n}{g1} \PYG{o}{=} \PYG{k}{lambda} \PYG{n}{x}\PYG{p}{:} \PYG{l+m+mi}{6} \PYG{o}{\PYGZhy{}} \PYG{n}{x}\PYG{o}{*}\PYG{o}{*}\PYG{l+m+mi}{2}

\PYG{n}{plt}\PYG{o}{.}\PYG{n}{plot}\PYG{p}{(}\PYG{n}{x}\PYG{p}{,}\PYG{n}{f}\PYG{p}{(}\PYG{n}{x}\PYG{p}{)}\PYG{p}{,}\PYG{n}{label}\PYG{o}{=}\PYG{l+s+s1}{\PYGZsq{}}\PYG{l+s+s1}{\PYGZdl{}f(x)\PYGZdl{}}\PYG{l+s+s1}{\PYGZsq{}}\PYG{p}{)}\PYG{p}{;}
\PYG{n}{plt}\PYG{o}{.}\PYG{n}{plot}\PYG{p}{(}\PYG{n}{x}\PYG{p}{,}\PYG{n}{x}\PYG{p}{,}\PYG{l+s+s1}{\PYGZsq{}}\PYG{l+s+s1}{k\PYGZhy{}\PYGZhy{}}\PYG{l+s+s1}{\PYGZsq{}}\PYG{p}{,}\PYG{n}{label}\PYG{o}{=}\PYG{l+s+s1}{\PYGZsq{}}\PYG{l+s+s1}{\PYGZdl{}y=x\PYGZdl{}}\PYG{l+s+s1}{\PYGZsq{}}\PYG{p}{)}\PYG{p}{;}
\PYG{n}{plt}\PYG{o}{.}\PYG{n}{plot}\PYG{p}{(}\PYG{n}{x}\PYG{p}{,}\PYG{l+m+mi}{0}\PYG{o}{*}\PYG{n}{x}\PYG{p}{,}\PYG{l+s+s1}{\PYGZsq{}}\PYG{l+s+s1}{c\PYGZhy{}.}\PYG{l+s+s1}{\PYGZsq{}}\PYG{p}{,}\PYG{n}{label}\PYG{o}{=}\PYG{l+s+s1}{\PYGZsq{}}\PYG{l+s+s1}{\PYGZdl{}y=0\PYGZdl{}}\PYG{l+s+s1}{\PYGZsq{}}\PYG{p}{)}\PYG{p}{;}
\PYG{n}{plt}\PYG{o}{.}\PYG{n}{plot}\PYG{p}{(}\PYG{n}{x}\PYG{p}{,}\PYG{n}{g1}\PYG{p}{(}\PYG{n}{x}\PYG{p}{)}\PYG{p}{,}\PYG{l+s+s1}{\PYGZsq{}}\PYG{l+s+s1}{r\PYGZhy{}\PYGZhy{}}\PYG{l+s+s1}{\PYGZsq{}}\PYG{p}{,}\PYG{n}{label}\PYG{o}{=}\PYG{l+s+s1}{\PYGZsq{}}\PYG{l+s+s1}{\PYGZdl{}g\PYGZus{}1(x)\PYGZdl{}}\PYG{l+s+s1}{\PYGZsq{}}\PYG{p}{)}\PYG{p}{;}

\PYG{n}{plt}\PYG{o}{.}\PYG{n}{axvline}\PYG{p}{(}\PYG{o}{\PYGZhy{}}\PYG{l+m+mi}{3}\PYG{p}{,}\PYG{o}{\PYGZhy{}}\PYG{l+m+mi}{5}\PYG{p}{,}\PYG{l+m+mi}{10}\PYG{p}{,}\PYG{n}{color}\PYG{o}{=}\PYG{l+s+s1}{\PYGZsq{}}\PYG{l+s+s1}{m}\PYG{l+s+s1}{\PYGZsq{}}\PYG{p}{)}\PYG{p}{;}
\PYG{n}{plt}\PYG{o}{.}\PYG{n}{axvline}\PYG{p}{(}\PYG{l+m+mi}{2}\PYG{p}{,}\PYG{o}{\PYGZhy{}}\PYG{l+m+mi}{5}\PYG{p}{,}\PYG{l+m+mi}{10}\PYG{p}{,}\PYG{n}{color}\PYG{o}{=}\PYG{l+s+s1}{\PYGZsq{}}\PYG{l+s+s1}{m}\PYG{l+s+s1}{\PYGZsq{}}\PYG{p}{)}\PYG{p}{;}
\PYG{n}{plt}\PYG{o}{.}\PYG{n}{legend}\PYG{p}{(}\PYG{n}{loc}\PYG{o}{=}\PYG{l+s+s1}{\PYGZsq{}}\PYG{l+s+s1}{best}\PYG{l+s+s1}{\PYGZsq{}}\PYG{p}{)}\PYG{p}{;}
\end{sphinxVerbatim}

\begin{sphinxVerbatim}[commandchars=\\\{\}]
Raízes: x1 = \PYGZhy{}3.000000, x2 = 2.000000
\end{sphinxVerbatim}

\noindent\sphinxincludegraphics{{aula-05-ponto-fixo_4_1}.png}


\section{Exemplo}
\label{\detokenize{aula-05-ponto-fixo:id1}}
\sphinxAtStartPar
Estudamos a função \(f(x) = \exp(x) -x\)

\begin{sphinxVerbatim}[commandchars=\\\{\}]
\PYG{n}{x2} \PYG{o}{=} \PYG{n}{np}\PYG{o}{.}\PYG{n}{linspace}\PYG{p}{(}\PYG{o}{\PYGZhy{}}\PYG{l+m+mi}{1}\PYG{p}{,}\PYG{l+m+mi}{1}\PYG{p}{,}\PYG{l+m+mi}{50}\PYG{p}{)}

\PYG{n}{f2} \PYG{o}{=} \PYG{k}{lambda} \PYG{n}{x}\PYG{p}{:} \PYG{n}{np}\PYG{o}{.}\PYG{n}{exp}\PYG{p}{(}\PYG{o}{\PYGZhy{}}\PYG{n}{x}\PYG{p}{)} \PYG{o}{\PYGZhy{}} \PYG{n}{x}
\PYG{n}{g2} \PYG{o}{=} \PYG{k}{lambda} \PYG{n}{x}\PYG{p}{:}  \PYG{n}{np}\PYG{o}{.}\PYG{n}{exp}\PYG{p}{(}\PYG{o}{\PYGZhy{}}\PYG{n}{x}\PYG{p}{)}

\PYG{n}{plt}\PYG{o}{.}\PYG{n}{plot}\PYG{p}{(}\PYG{n}{x2}\PYG{p}{,}\PYG{n}{f2}\PYG{p}{(}\PYG{n}{x2}\PYG{p}{)}\PYG{p}{,}\PYG{n}{x2}\PYG{p}{,}\PYG{n}{g2}\PYG{p}{(}\PYG{n}{x2}\PYG{p}{)}\PYG{p}{,}\PYG{l+s+s1}{\PYGZsq{}}\PYG{l+s+s1}{r}\PYG{l+s+s1}{\PYGZsq{}}\PYG{p}{,}\PYG{n}{x2}\PYG{p}{,}\PYG{n}{x2}\PYG{p}{,}\PYG{l+s+s1}{\PYGZsq{}}\PYG{l+s+s1}{c}\PYG{l+s+s1}{\PYGZsq{}}\PYG{p}{)}
\PYG{n}{plt}\PYG{o}{.}\PYG{n}{grid}\PYG{p}{(}\PYG{k+kc}{True}\PYG{p}{)}
\end{sphinxVerbatim}

\noindent\sphinxincludegraphics{{aula-05-ponto-fixo_6_0}.png}


\section{Implementação do método do ponto fixo}
\label{\detokenize{aula-05-ponto-fixo:implementacao-do-metodo-do-ponto-fixo}}
\begin{sphinxVerbatim}[commandchars=\\\{\}]
\PYG{k}{def} \PYG{n+nf}{ponto\PYGZus{}fixo}\PYG{p}{(}\PYG{n}{x0}\PYG{p}{,}\PYG{n}{f}\PYG{p}{,}\PYG{n}{g}\PYG{p}{,}\PYG{n}{tol}\PYG{p}{,}\PYG{n}{N}\PYG{p}{,}\PYG{n}{vis}\PYG{p}{)}\PYG{p}{:}
    \PYG{l+s+sd}{\PYGZdq{}\PYGZdq{}\PYGZdq{} }
\PYG{l+s+sd}{    Resolve problema de determinacao de raizes pelo }
\PYG{l+s+sd}{    metodo do ponto fixo (iteracao linear).}
\PYG{l+s+sd}{    }
\PYG{l+s+sd}{    entrada: }
\PYG{l+s+sd}{    }
\PYG{l+s+sd}{       x0  \PYGZhy{} aproximacao inicial        (float)}
\PYG{l+s+sd}{        f  \PYGZhy{} funcao a ser resolvida     (str)}
\PYG{l+s+sd}{        g  \PYGZhy{} funcao de iteracao         (str)}
\PYG{l+s+sd}{       tol \PYGZhy{} tolerancia                 (float)}
\PYG{l+s+sd}{        N  \PYGZhy{} numero maximo de iteracoes (int)}
\PYG{l+s+sd}{      vis  \PYGZhy{} flag para plotagem         (bool)}
\PYG{l+s+sd}{      }
\PYG{l+s+sd}{    saida:  }
\PYG{l+s+sd}{    }
\PYG{l+s+sd}{       x   \PYGZhy{} raiz aproximada para f     (float)      }
\PYG{l+s+sd}{    \PYGZdq{}\PYGZdq{}\PYGZdq{}}
    \PYG{k+kn}{from} \PYG{n+nn}{numpy} \PYG{k+kn}{import} \PYG{n}{linspace}
    \PYG{k+kn}{from} \PYG{n+nn}{matplotlib}\PYG{n+nn}{.}\PYG{n+nn}{pyplot} \PYG{k+kn}{import} \PYG{n}{plot}\PYG{p}{,}\PYG{n}{legend}

    \PYG{c+c1}{\PYGZsh{} funcoes}
    \PYG{n}{f} \PYG{o}{=} \PYG{n+nb}{eval}\PYG{p}{(}\PYG{l+s+s1}{\PYGZsq{}}\PYG{l+s+s1}{lambda x:}\PYG{l+s+s1}{\PYGZsq{}} \PYG{o}{+} \PYG{n}{f}\PYG{p}{)}
    \PYG{n}{g} \PYG{o}{=} \PYG{n+nb}{eval}\PYG{p}{(}\PYG{l+s+s1}{\PYGZsq{}}\PYG{l+s+s1}{lambda x:}\PYG{l+s+s1}{\PYGZsq{}} \PYG{o}{+} \PYG{n}{g}\PYG{p}{)}
    
    \PYG{c+c1}{\PYGZsh{} inicializacao}
    \PYG{n}{it} \PYG{o}{=} \PYG{l+m+mi}{0} \PYG{c+c1}{\PYGZsh{} contador }
    \PYG{n}{x}\PYG{p}{,} \PYG{n}{xn} \PYG{o}{=} \PYG{n}{x0}\PYG{p}{,} \PYG{n}{x0} \PYG{o}{+} \PYG{l+m+mi}{1} \PYG{c+c1}{\PYGZsh{} iteradas atual, anterior}
    
    \PYG{n}{e} \PYG{o}{=} \PYG{n+nb}{abs}\PYG{p}{(}\PYG{n}{x}\PYG{o}{\PYGZhy{}}\PYG{n}{xn}\PYG{p}{)}\PYG{o}{/}\PYG{n+nb}{abs}\PYG{p}{(}\PYG{n}{x}\PYG{p}{)} \PYG{c+c1}{\PYGZsh{} erro    }

    \PYG{c+c1}{\PYGZsh{} tabela}
    \PYG{n+nb}{print}\PYG{p}{(}\PYG{l+s+s1}{\PYGZsq{}}\PYG{l+s+s1}{i}\PYG{l+s+se}{\PYGZbs{}t}\PYG{l+s+s1}{ x}\PYG{l+s+se}{\PYGZbs{}t}\PYG{l+s+se}{\PYGZbs{}t}\PYG{l+s+s1}{ f(x)}\PYG{l+s+se}{\PYGZbs{}t}\PYG{l+s+se}{\PYGZbs{}t}\PYG{l+s+s1}{ ER}\PYG{l+s+s1}{\PYGZsq{}}\PYG{p}{)}
    \PYG{n+nb}{print}\PYG{p}{(}\PYG{l+s+s1}{\PYGZsq{}}\PYG{l+s+si}{\PYGZob{}0:d\PYGZcb{}}\PYG{l+s+se}{\PYGZbs{}t}\PYG{l+s+s1}{ }\PYG{l+s+si}{\PYGZob{}1:f\PYGZcb{}}\PYG{l+s+se}{\PYGZbs{}t}\PYG{l+s+s1}{ }\PYG{l+s+si}{\PYGZob{}2:f\PYGZcb{}}\PYG{l+s+se}{\PYGZbs{}t}\PYG{l+s+s1}{ }\PYG{l+s+si}{\PYGZob{}3:e\PYGZcb{}}\PYG{l+s+s1}{\PYGZsq{}}\PYG{o}{.}\PYG{n}{format}\PYG{p}{(}\PYG{n}{it}\PYG{p}{,}\PYG{n}{x}\PYG{p}{,}\PYG{n}{f}\PYG{p}{(}\PYG{n}{x}\PYG{p}{)}\PYG{p}{,}\PYG{n}{e}\PYG{p}{)}\PYG{p}{)}       
    
    \PYG{c+c1}{\PYGZsh{} laco}
    \PYG{k}{while} \PYG{n}{e} \PYG{o}{\PYGZgt{}}\PYG{o}{=} \PYG{n}{tol} \PYG{o+ow}{and} \PYG{n}{it} \PYG{o}{\PYGZlt{}}\PYG{o}{=} \PYG{n}{N}\PYG{p}{:}
        \PYG{n}{it} \PYG{o}{+}\PYG{o}{=} \PYG{l+m+mi}{1}    
        \PYG{n}{xn} \PYG{o}{=} \PYG{n}{x}                             
        \PYG{n}{x} \PYG{o}{=} \PYG{n}{g}\PYG{p}{(}\PYG{n}{xn}\PYG{p}{)}               
        \PYG{n}{e} \PYG{o}{=} \PYG{n+nb}{abs}\PYG{p}{(}\PYG{n}{x}\PYG{o}{\PYGZhy{}}\PYG{n}{xn}\PYG{p}{)}\PYG{o}{/}\PYG{n+nb}{abs}\PYG{p}{(}\PYG{n}{x}\PYG{p}{)}         
        \PYG{n+nb}{print}\PYG{p}{(}\PYG{l+s+s1}{\PYGZsq{}}\PYG{l+s+si}{\PYGZob{}0:d\PYGZcb{}}\PYG{l+s+se}{\PYGZbs{}t}\PYG{l+s+s1}{ }\PYG{l+s+si}{\PYGZob{}1:f\PYGZcb{}}\PYG{l+s+se}{\PYGZbs{}t}\PYG{l+s+s1}{ }\PYG{l+s+si}{\PYGZob{}2:f\PYGZcb{}}\PYG{l+s+se}{\PYGZbs{}t}\PYG{l+s+s1}{ }\PYG{l+s+si}{\PYGZob{}3:e\PYGZcb{}}\PYG{l+s+s1}{\PYGZsq{}}\PYG{o}{.}\PYG{n}{format}\PYG{p}{(}\PYG{n}{it}\PYG{p}{,}\PYG{n}{x}\PYG{p}{,}\PYG{n}{f}\PYG{p}{(}\PYG{n}{x}\PYG{p}{)}\PYG{p}{,}\PYG{n}{e}\PYG{p}{)}\PYG{p}{)}      
        
        \PYG{k}{if} \PYG{n}{it} \PYG{o}{\PYGZgt{}} \PYG{n}{N}\PYG{p}{:}
            \PYG{n+nb}{print}\PYG{p}{(}\PYG{l+s+s1}{\PYGZsq{}}\PYG{l+s+s1}{Solução nao alcancada com N iteracoes.}\PYG{l+s+s1}{\PYGZsq{}}\PYG{p}{)}
            \PYG{k}{break}
       
    \PYG{k}{if} \PYG{n}{vis} \PYG{o}{==} \PYG{k+kc}{True}\PYG{p}{:}
        \PYG{n}{dx} \PYG{o}{=} \PYG{l+m+mi}{2}\PYG{o}{*}\PYG{n}{x}
        \PYG{n}{dom} \PYG{o}{=} \PYG{n}{linspace}\PYG{p}{(}\PYG{n}{x} \PYG{o}{\PYGZhy{}} \PYG{n}{dx}\PYG{p}{,}\PYG{n}{x} \PYG{o}{+} \PYG{n}{dx}\PYG{p}{,}\PYG{l+m+mi}{30}\PYG{p}{)}
        \PYG{n}{plot}\PYG{p}{(}\PYG{n}{dom}\PYG{p}{,}\PYG{n}{f}\PYG{p}{(}\PYG{n}{dom}\PYG{p}{)}\PYG{p}{,}\PYG{n}{label}\PYG{o}{=}\PYG{l+s+s1}{\PYGZsq{}}\PYG{l+s+s1}{\PYGZdl{}f(x)\PYGZdl{}}\PYG{l+s+s1}{\PYGZsq{}}\PYG{p}{)}
        \PYG{n}{plot}\PYG{p}{(}\PYG{n}{dom}\PYG{p}{,}\PYG{n}{dom}\PYG{o}{*}\PYG{l+m+mi}{0}\PYG{p}{,}\PYG{n}{label}\PYG{o}{=}\PYG{l+s+s1}{\PYGZsq{}}\PYG{l+s+s1}{\PYGZdl{}y=0\PYGZdl{}}\PYG{l+s+s1}{\PYGZsq{}}\PYG{p}{)}
        \PYG{n}{plot}\PYG{p}{(}\PYG{n}{dom}\PYG{p}{,}\PYG{n}{g}\PYG{p}{(}\PYG{n}{dom}\PYG{p}{)}\PYG{p}{,}\PYG{n}{label}\PYG{o}{=}\PYG{l+s+s1}{\PYGZsq{}}\PYG{l+s+s1}{\PYGZdl{}g(x)\PYGZdl{}}\PYG{l+s+s1}{\PYGZsq{}}\PYG{p}{)}
        \PYG{n}{plot}\PYG{p}{(}\PYG{n}{dom}\PYG{p}{,}\PYG{n}{dom}\PYG{p}{,}\PYG{n}{label}\PYG{o}{=}\PYG{l+s+s1}{\PYGZsq{}}\PYG{l+s+s1}{\PYGZdl{}y=x\PYGZdl{}}\PYG{l+s+s1}{\PYGZsq{}}\PYG{p}{)}
        \PYG{n}{legend}\PYG{p}{(}\PYG{p}{)}
        
    \PYG{k}{return} \PYG{n}{x}
\end{sphinxVerbatim}


\section{Estudo de caso: \protect\(f(x) = x^2 + x - 6\protect\)}
\label{\detokenize{aula-05-ponto-fixo:estudo-de-caso-f-x-x-2-x-6}}
\sphinxAtStartPar
Função de iteração: \(g(x) = \sqrt{6 - x}\)

\begin{sphinxVerbatim}[commandchars=\\\{\}]
\PYG{n}{f} \PYG{o}{=} \PYG{l+s+s1}{\PYGZsq{}}\PYG{l+s+s1}{x**2 + x \PYGZhy{} 6}\PYG{l+s+s1}{\PYGZsq{}}
\PYG{n}{g} \PYG{o}{=} \PYG{l+s+s1}{\PYGZsq{}}\PYG{l+s+s1}{(6 \PYGZhy{} x)**(1/2)}\PYG{l+s+s1}{\PYGZsq{}}

\PYG{n}{x0} \PYG{o}{=} \PYG{l+m+mf}{0.1}
\PYG{n}{tol} \PYG{o}{=} \PYG{l+m+mf}{1e\PYGZhy{}5}
\PYG{n}{N} \PYG{o}{=} \PYG{l+m+mi}{100}

\PYG{n}{ponto\PYGZus{}fixo}\PYG{p}{(}\PYG{n}{x0}\PYG{p}{,}\PYG{n}{f}\PYG{p}{,}\PYG{n}{g}\PYG{p}{,}\PYG{n}{tol}\PYG{p}{,}\PYG{n}{N}\PYG{p}{,}\PYG{k+kc}{True}\PYG{p}{)}
\end{sphinxVerbatim}

\begin{sphinxVerbatim}[commandchars=\\\{\}]
i	 x		 f(x)		 ER
0	 0.100000	 \PYGZhy{}5.890000	 1.000000e+01
1	 2.428992	 2.328992	 9.588307e\PYGZhy{}01
2	 1.889711	 \PYGZhy{}0.539280	 2.853771e\PYGZhy{}01
3	 2.027385	 0.137674	 6.790695e\PYGZhy{}02
4	 1.993142	 \PYGZhy{}0.034243	 1.718024e\PYGZhy{}02
5	 2.001714	 0.008572	 4.282174e\PYGZhy{}03
6	 1.999572	 \PYGZhy{}0.002142	 1.071346e\PYGZhy{}03
7	 2.000107	 0.000536	 2.677864e\PYGZhy{}04
8	 1.999973	 \PYGZhy{}0.000134	 6.694973e\PYGZhy{}05
9	 2.000007	 0.000033	 1.673724e\PYGZhy{}05
10	 1.999998	 \PYGZhy{}0.000008	 4.184321e\PYGZhy{}06
\end{sphinxVerbatim}

\begin{sphinxVerbatim}[commandchars=\\\{\}]
1.9999983262723453
\end{sphinxVerbatim}

\noindent\sphinxincludegraphics{{aula-05-ponto-fixo_10_2}.png}

\sphinxAtStartPar
Função de iteração: \(g(x) = -\sqrt{6 - x}\)

\begin{sphinxVerbatim}[commandchars=\\\{\}]
\PYG{n}{f} \PYG{o}{=} \PYG{l+s+s1}{\PYGZsq{}}\PYG{l+s+s1}{x**2 + x \PYGZhy{} 6}\PYG{l+s+s1}{\PYGZsq{}}
\PYG{n}{g} \PYG{o}{=} \PYG{l+s+s1}{\PYGZsq{}}\PYG{l+s+s1}{\PYGZhy{}(6 \PYGZhy{} x)**(1/2)}\PYG{l+s+s1}{\PYGZsq{}}

\PYG{n}{x0} \PYG{o}{=} \PYG{l+m+mf}{0.1}
\PYG{n}{tol} \PYG{o}{=} \PYG{l+m+mf}{1e\PYGZhy{}5}
\PYG{n}{N} \PYG{o}{=} \PYG{l+m+mi}{100}

\PYG{n}{ponto\PYGZus{}fixo}\PYG{p}{(}\PYG{n}{x0}\PYG{p}{,}\PYG{n}{f}\PYG{p}{,}\PYG{n}{g}\PYG{p}{,}\PYG{n}{tol}\PYG{p}{,}\PYG{n}{N}\PYG{p}{,}\PYG{k+kc}{True}\PYG{p}{)}
\end{sphinxVerbatim}

\begin{sphinxVerbatim}[commandchars=\\\{\}]
i	 x		 f(x)		 ER
0	 0.100000	 \PYGZhy{}5.890000	 1.000000e+01
1	 \PYGZhy{}2.428992	 \PYGZhy{}2.528992	 1.041169e+00
2	 \PYGZhy{}2.903273	 \PYGZhy{}0.474281	 1.633608e\PYGZhy{}01
3	 \PYGZhy{}2.983835	 \PYGZhy{}0.080563	 2.699970e\PYGZhy{}02
4	 \PYGZhy{}2.997305	 \PYGZhy{}0.013469	 4.493853e\PYGZhy{}03
5	 \PYGZhy{}2.999551	 \PYGZhy{}0.002246	 7.488072e\PYGZhy{}04
6	 \PYGZhy{}2.999925	 \PYGZhy{}0.000374	 1.247965e\PYGZhy{}04
7	 \PYGZhy{}2.999988	 \PYGZhy{}0.000062	 2.079929e\PYGZhy{}05
8	 \PYGZhy{}2.999998	 \PYGZhy{}0.000010	 3.466545e\PYGZhy{}06
\end{sphinxVerbatim}

\begin{sphinxVerbatim}[commandchars=\\\{\}]
\PYGZhy{}2.9999979200736955
\end{sphinxVerbatim}

\noindent\sphinxincludegraphics{{aula-05-ponto-fixo_12_2}.png}


\chapter{Implementação do método de Newton}
\label{\detokenize{aula-06-newton:implementacao-do-metodo-de-newton}}\label{\detokenize{aula-06-newton::doc}}
\begin{sphinxVerbatim}[commandchars=\\\{\}]
\PYG{o}{\PYGZpc{}}\PYG{k}{matplotlib} inline
\end{sphinxVerbatim}
\begin{itemize}
\item {} 
\sphinxAtStartPar
Analisar a dependência da estimativa inicial.

\item {} 
\sphinxAtStartPar
Executar o código duas vezes: para \(x_0=0.0\) e \(x_0=1.0\) em \(f(x) = -0.9x^2 + 1.7x + 2.5\).

\end{itemize}

\begin{sphinxVerbatim}[commandchars=\\\{\}]
\PYG{c+c1}{\PYGZsh{} Método de Newton}

\PYG{k+kn}{from} \PYG{n+nn}{numpy} \PYG{k+kn}{import} \PYG{n}{linspace}
\PYG{k+kn}{from} \PYG{n+nn}{matplotlib}\PYG{n+nn}{.}\PYG{n+nn}{pyplot} \PYG{k+kn}{import} \PYG{n}{plot}

\PYG{k}{def} \PYG{n+nf}{newton}\PYG{p}{(}\PYG{n}{x0}\PYG{p}{,}\PYG{n}{f}\PYG{p}{,}\PYG{n}{df}\PYG{p}{,}\PYG{n}{tol}\PYG{p}{,}\PYG{n}{nmax}\PYG{p}{,}\PYG{n}{var}\PYG{p}{,}\PYG{n}{plotar}\PYG{p}{)}\PYG{p}{:}

    \PYG{n}{f} \PYG{o}{=} \PYG{n+nb}{eval}\PYG{p}{(}\PYG{l+s+s1}{\PYGZsq{}}\PYG{l+s+s1}{lambda x:}\PYG{l+s+s1}{\PYGZsq{}} \PYG{o}{+} \PYG{n}{f}\PYG{p}{)}
    \PYG{n}{df} \PYG{o}{=} \PYG{n+nb}{eval}\PYG{p}{(}\PYG{l+s+s1}{\PYGZsq{}}\PYG{l+s+s1}{lambda x:}\PYG{l+s+s1}{\PYGZsq{}} \PYG{o}{+} \PYG{n}{df}\PYG{p}{)}

    \PYG{n}{it} \PYG{o}{=} \PYG{l+m+mi}{0} \PYG{c+c1}{\PYGZsh{} contador de iteracoes}

    \PYG{c+c1}{\PYGZsh{} imprime estimativa inicial}
    \PYG{n+nb}{print}\PYG{p}{(}\PYG{l+s+s1}{\PYGZsq{}}\PYG{l+s+s1}{Estimativa inicial: x0 = }\PYG{l+s+si}{\PYGZob{}0\PYGZcb{}}\PYG{l+s+se}{\PYGZbs{}n}\PYG{l+s+s1}{\PYGZsq{}}\PYG{o}{.}\PYG{n}{format}\PYG{p}{(}\PYG{n}{x0}\PYG{p}{)}\PYG{p}{)}  

    \PYG{c+c1}{\PYGZsh{} Loop }
    \PYG{k}{for} \PYG{n}{i} \PYG{o+ow}{in} \PYG{n+nb}{range}\PYG{p}{(}\PYG{l+m+mi}{0}\PYG{p}{,}\PYG{n}{nmax}\PYG{p}{)}\PYG{p}{:}
        
        \PYG{n}{x} \PYG{o}{=} \PYG{n}{x0} \PYG{o}{\PYGZhy{}} \PYG{n}{f}\PYG{p}{(}\PYG{n}{x0}\PYG{p}{)}\PYG{o}{/}\PYG{n}{df}\PYG{p}{(}\PYG{n}{x0}\PYG{p}{)} \PYG{c+c1}{\PYGZsh{} funcao de iteracao }
        
        \PYG{n}{e} \PYG{o}{=} \PYG{n+nb}{abs}\PYG{p}{(}\PYG{n}{x}\PYG{o}{\PYGZhy{}}\PYG{n}{x0}\PYG{p}{)}\PYG{o}{/}\PYG{n+nb}{abs}\PYG{p}{(}\PYG{n}{x}\PYG{p}{)} \PYG{c+c1}{\PYGZsh{} erro}
        
        \PYG{c+c1}{\PYGZsh{} tabela}
        \PYG{n+nb}{print}\PYG{p}{(}\PYG{l+s+s1}{\PYGZsq{}}\PYG{l+s+si}{\PYGZob{}0:d\PYGZcb{}}\PYG{l+s+s1}{  }\PYG{l+s+si}{\PYGZob{}1:f\PYGZcb{}}\PYG{l+s+s1}{  }\PYG{l+s+si}{\PYGZob{}2:f\PYGZcb{}}\PYG{l+s+s1}{  }\PYG{l+s+si}{\PYGZob{}3:f\PYGZcb{}}\PYG{l+s+s1}{  }\PYG{l+s+si}{\PYGZob{}4:e\PYGZcb{}}\PYG{l+s+s1}{\PYGZsq{}}\PYG{o}{.}\PYG{n}{format}\PYG{p}{(}\PYG{n}{i}\PYG{p}{,}\PYG{n}{x}\PYG{p}{,}\PYG{n}{f}\PYG{p}{(}\PYG{n}{x}\PYG{p}{)}\PYG{p}{,}\PYG{n}{df}\PYG{p}{(}\PYG{n}{x}\PYG{p}{)}\PYG{p}{,}\PYG{n}{e}\PYG{p}{)}\PYG{p}{)}
        
        \PYG{k}{if} \PYG{n}{e} \PYG{o}{\PYGZlt{}} \PYG{n}{tol}\PYG{p}{:}
            \PYG{k}{break}
        \PYG{n}{x0} \PYG{o}{=} \PYG{n}{x}                
        
    \PYG{k}{if} \PYG{n}{i} \PYG{o}{==} \PYG{n}{nmax}\PYG{p}{:}
        \PYG{n+nb}{print}\PYG{p}{(}\PYG{l+s+s1}{\PYGZsq{}}\PYG{l+s+s1}{Solução não obtida em }\PYG{l+s+si}{\PYGZob{}0:d\PYGZcb{}}\PYG{l+s+s1}{ iterações}\PYG{l+s+s1}{\PYGZsq{}}\PYG{o}{.}\PYG{n}{format}\PYG{p}{(}\PYG{n}{nmax}\PYG{p}{)}\PYG{p}{)}
    \PYG{k}{else}\PYG{p}{:}
        \PYG{n+nb}{print}\PYG{p}{(}\PYG{l+s+s1}{\PYGZsq{}}\PYG{l+s+s1}{Solução obtida: x = }\PYG{l+s+si}{\PYGZob{}0:.10f\PYGZcb{}}\PYG{l+s+s1}{\PYGZsq{}}\PYG{o}{.}\PYG{n}{format}\PYG{p}{(}\PYG{n}{x}\PYG{p}{)}\PYG{p}{)}

    \PYG{c+c1}{\PYGZsh{} plotagem}
    \PYG{k}{if} \PYG{n}{plotar}\PYG{p}{:}        
        \PYG{n}{delta} \PYG{o}{=} \PYG{l+m+mi}{3}\PYG{o}{*}\PYG{n}{x}
        \PYG{n}{dom} \PYG{o}{=} \PYG{n}{linspace}\PYG{p}{(}\PYG{n}{x}\PYG{o}{\PYGZhy{}}\PYG{n}{delta}\PYG{p}{,}\PYG{n}{x}\PYG{o}{+}\PYG{n}{delta}\PYG{p}{,}\PYG{l+m+mi}{30}\PYG{p}{)}
        \PYG{n}{plot}\PYG{p}{(}\PYG{n}{dom}\PYG{p}{,}\PYG{n}{f}\PYG{p}{(}\PYG{n}{dom}\PYG{p}{)}\PYG{p}{,}\PYG{n}{x}\PYG{p}{,}\PYG{n}{f}\PYG{p}{(}\PYG{n}{x}\PYG{p}{)}\PYG{p}{,}\PYG{l+s+s1}{\PYGZsq{}}\PYG{l+s+s1}{ro}\PYG{l+s+s1}{\PYGZsq{}}\PYG{p}{)}

    \PYG{k}{return} \PYG{n}{x}
      
    
\PYG{c+c1}{\PYGZsh{} parametros    }
\PYG{n}{x0} \PYG{o}{=} \PYG{l+m+mf}{0.} \PYG{c+c1}{\PYGZsh{} estimativa inicial}
\PYG{n}{tol} \PYG{o}{=} \PYG{l+m+mf}{1e\PYGZhy{}3} \PYG{c+c1}{\PYGZsh{} tolerancia}
\PYG{n}{nmax} \PYG{o}{=} \PYG{l+m+mi}{100} \PYG{c+c1}{\PYGZsh{} numero maximo de iteracoes}
\PYG{n}{f} \PYG{o}{=} \PYG{l+s+s1}{\PYGZsq{}}\PYG{l+s+s1}{\PYGZhy{}0.9*x**2 + 1.7*x + 2.5}\PYG{l+s+s1}{\PYGZsq{}}   \PYG{c+c1}{\PYGZsh{} funcao}
\PYG{n}{df} \PYG{o}{=} \PYG{l+s+s1}{\PYGZsq{}}\PYG{l+s+s1}{\PYGZhy{}1.8*x + 1.7}\PYG{l+s+s1}{\PYGZsq{}}   \PYG{c+c1}{\PYGZsh{} derivada dafuncao}
\PYG{n}{var} \PYG{o}{=} \PYG{l+s+s1}{\PYGZsq{}}\PYG{l+s+s1}{x}\PYG{l+s+s1}{\PYGZsq{}}
\PYG{n}{plotar} \PYG{o}{=} \PYG{k+kc}{True}

\PYG{c+c1}{\PYGZsh{} chamada da função}
\PYG{n}{xm} \PYG{o}{=} \PYG{n}{newton}\PYG{p}{(}\PYG{n}{x0}\PYG{p}{,}\PYG{n}{f}\PYG{p}{,}\PYG{n}{df}\PYG{p}{,}\PYG{n}{tol}\PYG{p}{,}\PYG{n}{nmax}\PYG{p}{,}\PYG{n}{var}\PYG{p}{,}\PYG{n}{plotar}\PYG{p}{)}
\end{sphinxVerbatim}

\begin{sphinxVerbatim}[commandchars=\\\{\}]
Estimativa inicial: x0 = 0.0

0  \PYGZhy{}1.470588  \PYGZhy{}1.946367  4.347059  1.000000e+00
1  \PYGZhy{}1.022845  \PYGZhy{}0.180427  3.541121  4.377432e\PYGZhy{}01
2  \PYGZhy{}0.971893  \PYGZhy{}0.002336  3.449407  5.242539e\PYGZhy{}02
3  \PYGZhy{}0.971216  \PYGZhy{}0.000000  3.448188  6.974331e\PYGZhy{}04
Solução obtida: x = \PYGZhy{}0.9712156364
\end{sphinxVerbatim}

\noindent\sphinxincludegraphics{{aula-06-newton_3_1}.png}

\begin{sphinxVerbatim}[commandchars=\\\{\}]
\PYG{c+c1}{\PYGZsh{} chamada da função}
\PYG{n}{xm} \PYG{o}{=} \PYG{n}{newton}\PYG{p}{(}\PYG{l+m+mf}{1.0}\PYG{p}{,}\PYG{n}{f}\PYG{p}{,}\PYG{n}{df}\PYG{p}{,}\PYG{n}{tol}\PYG{p}{,}\PYG{n}{nmax}\PYG{p}{,}\PYG{n}{var}\PYG{p}{,}\PYG{n}{plotar}\PYG{p}{)}
\end{sphinxVerbatim}

\begin{sphinxVerbatim}[commandchars=\\\{\}]
Estimativa inicial: x0 = 1.0

0  34.000000  \PYGZhy{}980.100000  \PYGZhy{}59.500000  9.705882e\PYGZhy{}01
1  17.527731  \PYGZhy{}244.202079  \PYGZhy{}29.849916  9.397833e\PYGZhy{}01
2  9.346734  \PYGZhy{}60.235844  \PYGZhy{}15.124121  8.752787e\PYGZhy{}01
3  5.363967  \PYGZhy{}14.276187  \PYGZhy{}7.955141  7.425039e\PYGZhy{}01
4  3.569381  \PYGZhy{}2.898486  \PYGZhy{}4.724886  5.027724e\PYGZhy{}01
5  2.955930  \PYGZhy{}0.338690  \PYGZhy{}3.620674  2.075323e\PYGZhy{}01
6  2.862387  \PYGZhy{}0.007875  \PYGZhy{}3.452297  3.268017e\PYGZhy{}02
7  2.860106  \PYGZhy{}0.000005  \PYGZhy{}3.448190  7.975862e\PYGZhy{}04
Solução obtida: x = 2.8601057637
\end{sphinxVerbatim}

\noindent\sphinxincludegraphics{{aula-06-newton_4_1}.png}


\section{Desafio}
\label{\detokenize{aula-06-newton:desafio}}\begin{enumerate}
\sphinxsetlistlabels{\arabic}{enumi}{enumii}{}{.}%
\item {} 
\sphinxAtStartPar
Generalize o código acima para que a expressão da derivada seja calculada diretamenta e não manualmente. (dica: use computação simbólica)

\item {} 
\sphinxAtStartPar
Resolva o problema aplicado abaixo com este método ou desenvolva o seu para resolver e compare com a função residente do \sphinxcode{\sphinxupquote{scipy}}.

\end{enumerate}


\subsection{Problema aplicado}
\label{\detokenize{aula-06-newton:problema-aplicado}}
\sphinxAtStartPar
Um jogador de futebol americano está prestes a fazer um lançamento para outro jogador de seu time. O lançador tem uma altura de 1,82 m e o outro jogador está afastado de 18,2 m. A expressão que descreve o movimento da bola é a familiar equação da física que descreve o movimento de um projétil:
\begin{equation*}
\begin{split}y = x\tan(\theta) - \dfrac{1}{2}\dfrac{x^2 g}{v_0^2}\dfrac{1}{\cos^2(\theta)} + h,\end{split}
\end{equation*}
\sphinxAtStartPar
onde \(x\) e \(y\) são as distâncias horizontal e verical, respectivamente, \(g=9,8 \, m/s^2\) é a aceleração da gravidade, \(v_0\) é a velocidade inicial da bola quando deixa a mão do lançador e \(\theta\) é o Ângulo que a bola faz com o eixo horizontal nesse mesmo instante. Para \(v_0 = 15,2 \, m/s\), \(x = 18,2 \, m\), \(h = 1,82 \, m\) e \(y = 2,1 \, m\), determine o ângulo \(\theta\) no qual o jogador deve lançar a bola.


\subsection{Solução por função residente}
\label{\detokenize{aula-06-newton:solucao-por-funcao-residente}}\begin{itemize}
\item {} 
\sphinxAtStartPar
Importar módulos

\item {} 
\sphinxAtStartPar
Definir função \(f(\theta)\)

\end{itemize}

\begin{sphinxVerbatim}[commandchars=\\\{\}]
\PYG{k+kn}{from} \PYG{n+nn}{scipy}\PYG{n+nn}{.}\PYG{n+nn}{optimize} \PYG{k+kn}{import} \PYG{n}{newton} 
\PYG{k+kn}{import} \PYG{n+nn}{numpy} \PYG{k}{as} \PYG{n+nn}{np}
\PYG{k+kn}{import} \PYG{n+nn}{matplotlib}\PYG{n+nn}{.}\PYG{n+nn}{pyplot} \PYG{k}{as} \PYG{n+nn}{plt}

\PYG{n}{v0} \PYG{o}{=} \PYG{l+m+mf}{15.2}
\PYG{n}{x} \PYG{o}{=} \PYG{l+m+mf}{18.2}
\PYG{n}{h} \PYG{o}{=} \PYG{l+m+mf}{1.82}
\PYG{n}{y} \PYG{o}{=} \PYG{l+m+mf}{2.1}
\PYG{n}{g} \PYG{o}{=} \PYG{l+m+mf}{9.8}

\PYG{c+c1}{\PYGZsh{} f(theta) = 0}
\PYG{n}{f} \PYG{o}{=} \PYG{k}{lambda} \PYG{n}{theta}\PYG{p}{:} \PYG{n}{x}\PYG{o}{*}\PYG{n}{np}\PYG{o}{.}\PYG{n}{tan}\PYG{p}{(}\PYG{n}{theta}\PYG{p}{)} \PYG{o}{\PYGZhy{}} \PYG{l+m+mf}{0.5}\PYG{o}{*}\PYG{p}{(}\PYG{n}{x}\PYG{o}{*}\PYG{o}{*}\PYG{l+m+mi}{2}\PYG{o}{*}\PYG{n}{g}\PYG{o}{/}\PYG{n}{v0}\PYG{o}{*}\PYG{o}{*}\PYG{l+m+mi}{2}\PYG{p}{)}\PYG{o}{*}\PYG{p}{(}\PYG{l+m+mi}{1}\PYG{o}{/}\PYG{p}{(}\PYG{n}{np}\PYG{o}{.}\PYG{n}{cos}\PYG{p}{(}\PYG{n}{theta}\PYG{p}{)}\PYG{o}{*}\PYG{o}{*}\PYG{l+m+mi}{2}\PYG{p}{)}\PYG{p}{)} \PYG{o}{+} \PYG{n}{h} \PYG{o}{\PYGZhy{}} \PYG{n}{y}
\end{sphinxVerbatim}


\section{Localização}
\label{\detokenize{aula-06-newton:localizacao}}
\begin{sphinxVerbatim}[commandchars=\\\{\}]
\PYG{n}{th} \PYG{o}{=} \PYG{n}{np}\PYG{o}{.}\PYG{n}{linspace}\PYG{p}{(}\PYG{l+m+mf}{0.1}\PYG{p}{,}\PYG{l+m+mf}{1.4}\PYG{p}{,}\PYG{l+m+mi}{100}\PYG{p}{,}\PYG{k+kc}{True}\PYG{p}{)}
\PYG{n}{plt}\PYG{o}{.}\PYG{n}{plot}\PYG{p}{(}\PYG{n}{th}\PYG{p}{,}\PYG{n}{f}\PYG{p}{(}\PYG{n}{th}\PYG{p}{)}\PYG{p}{)}
\PYG{n}{plt}\PYG{o}{.}\PYG{n}{xlabel}\PYG{p}{(}\PYG{l+s+s1}{\PYGZsq{}}\PYG{l+s+s1}{angulo}\PYG{l+s+s1}{\PYGZsq{}}\PYG{p}{)}\PYG{p}{;}
\end{sphinxVerbatim}

\noindent\sphinxincludegraphics{{aula-06-newton_10_0}.png}

\begin{sphinxVerbatim}[commandchars=\\\{\}]
\PYG{n}{th} \PYG{o}{=} \PYG{n}{np}\PYG{o}{.}\PYG{n}{linspace}\PYG{p}{(}\PYG{l+m+mf}{0.4}\PYG{p}{,}\PYG{l+m+mf}{0.8}\PYG{p}{,}\PYG{l+m+mi}{100}\PYG{p}{,}\PYG{k+kc}{True}\PYG{p}{)}
\PYG{n}{plt}\PYG{o}{.}\PYG{n}{plot}\PYG{p}{(}\PYG{n}{th}\PYG{p}{,}\PYG{n}{f}\PYG{p}{(}\PYG{n}{th}\PYG{p}{)}\PYG{p}{)}
\PYG{n}{plt}\PYG{o}{.}\PYG{n}{xlabel}\PYG{p}{(}\PYG{l+s+s1}{\PYGZsq{}}\PYG{l+s+s1}{angulo}\PYG{l+s+s1}{\PYGZsq{}}\PYG{p}{)}\PYG{p}{;}
\end{sphinxVerbatim}

\noindent\sphinxincludegraphics{{aula-06-newton_11_0}.png}

\begin{sphinxVerbatim}[commandchars=\\\{\}]
\PYG{n}{th} \PYG{o}{=} \PYG{n}{np}\PYG{o}{.}\PYG{n}{linspace}\PYG{p}{(}\PYG{l+m+mf}{0.45}\PYG{p}{,}\PYG{l+m+mf}{0.47}\PYG{p}{,}\PYG{l+m+mi}{100}\PYG{p}{,}\PYG{k+kc}{True}\PYG{p}{)}
\PYG{n}{plt}\PYG{o}{.}\PYG{n}{plot}\PYG{p}{(}\PYG{n}{th}\PYG{p}{,}\PYG{n}{f}\PYG{p}{(}\PYG{n}{th}\PYG{p}{)}\PYG{p}{)}
\PYG{n}{plt}\PYG{o}{.}\PYG{n}{xlabel}\PYG{p}{(}\PYG{l+s+s1}{\PYGZsq{}}\PYG{l+s+s1}{angulo}\PYG{l+s+s1}{\PYGZsq{}}\PYG{p}{)}\PYG{p}{;}
\end{sphinxVerbatim}

\noindent\sphinxincludegraphics{{aula-06-newton_12_0}.png}


\section{Refinamento}
\label{\detokenize{aula-06-newton:refinamento}}
\sphinxAtStartPar
Para a função residente, nem é preciso fazer um processo de localização prolongado de modo a entrar com uma estimativa inicial muito próxima. Plotar a função até um intervalo razoável já é suficiente para ter uma noção sobre onde a raiz está.

\sphinxAtStartPar
Quanto à escolha da estimativa inicial, ainda que seja “mal feita”, o método poderá encontrar a raiz de modo rápido, pois sua programação é robusta.

\sphinxAtStartPar
Vejamos então, qual é a raiz com uma estimativa inicial de 0.47.

\begin{sphinxVerbatim}[commandchars=\\\{\}]
\PYG{n}{ang} \PYG{o}{=} \PYG{n}{newton}\PYG{p}{(}\PYG{n}{f}\PYG{p}{,}\PYG{l+m+mf}{0.47}\PYG{p}{)}
\PYG{n}{ang}
\end{sphinxVerbatim}

\begin{sphinxVerbatim}[commandchars=\\\{\}]
0.4608834641642987
\end{sphinxVerbatim}

\begin{sphinxVerbatim}[commandchars=\\\{\}]
\PYG{n}{np}\PYG{o}{.}\PYG{n}{rad2deg}\PYG{p}{(}\PYG{n}{ang}\PYG{p}{)}
\end{sphinxVerbatim}

\begin{sphinxVerbatim}[commandchars=\\\{\}]
26.406677343983237
\end{sphinxVerbatim}


\chapter{Implementação do método da secante}
\label{\detokenize{aula-07-secante:implementacao-do-metodo-da-secante}}\label{\detokenize{aula-07-secante::doc}}
\begin{sphinxVerbatim}[commandchars=\\\{\}]
\PYG{o}{\PYGZpc{}}\PYG{k}{matplotlib} inline
\end{sphinxVerbatim}

\begin{sphinxVerbatim}[commandchars=\\\{\}]
\PYG{c+c1}{\PYGZsh{} Método da Secante}

\PYG{k+kn}{from} \PYG{n+nn}{numpy} \PYG{k+kn}{import} \PYG{n}{linspace}
\PYG{k+kn}{from} \PYG{n+nn}{matplotlib}\PYG{n+nn}{.}\PYG{n+nn}{pyplot} \PYG{k+kn}{import} \PYG{n}{plot}

\PYG{k}{def} \PYG{n+nf}{secante}\PYG{p}{(}\PYG{n}{xa}\PYG{p}{,}\PYG{n}{xb}\PYG{p}{,}\PYG{n}{f}\PYG{p}{,}\PYG{n}{tol}\PYG{p}{,}\PYG{n}{nmax}\PYG{p}{,}\PYG{n}{var}\PYG{p}{,}\PYG{n}{plotar}\PYG{p}{)}\PYG{p}{:}

    \PYG{n}{f} \PYG{o}{=} \PYG{n+nb}{eval}\PYG{p}{(}\PYG{l+s+s1}{\PYGZsq{}}\PYG{l+s+s1}{lambda x:}\PYG{l+s+s1}{\PYGZsq{}} \PYG{o}{+} \PYG{n}{f}\PYG{p}{)}

    \PYG{c+c1}{\PYGZsh{} imprime estimativas iniciais}
    \PYG{n+nb}{print}\PYG{p}{(}\PYG{l+s+s1}{\PYGZsq{}}\PYG{l+s+s1}{Estimativas iniciais: xa = }\PYG{l+s+si}{\PYGZob{}0\PYGZcb{}}\PYG{l+s+s1}{; xb = }\PYG{l+s+si}{\PYGZob{}1\PYGZcb{}}\PYG{l+s+s1}{ }\PYG{l+s+se}{\PYGZbs{}n}\PYG{l+s+s1}{\PYGZsq{}}\PYG{o}{.}\PYG{n}{format}\PYG{p}{(}\PYG{n}{xa}\PYG{p}{,}\PYG{n}{xb}\PYG{p}{)}\PYG{p}{)}  

    \PYG{c+c1}{\PYGZsh{} Loop }
    \PYG{k}{for} \PYG{n}{i} \PYG{o+ow}{in} \PYG{n+nb}{range}\PYG{p}{(}\PYG{l+m+mi}{0}\PYG{p}{,}\PYG{n}{nmax}\PYG{p}{)}\PYG{p}{:}
        
        \PYG{n}{x} \PYG{o}{=} \PYG{p}{(}\PYG{n}{xa}\PYG{o}{*}\PYG{n}{f}\PYG{p}{(}\PYG{n}{xb}\PYG{p}{)} \PYG{o}{\PYGZhy{}} \PYG{n}{xb}\PYG{o}{*}\PYG{n}{f}\PYG{p}{(}\PYG{n}{xa}\PYG{p}{)}\PYG{p}{)}\PYG{o}{/}\PYG{p}{(}\PYG{n}{f}\PYG{p}{(}\PYG{n}{xb}\PYG{p}{)} \PYG{o}{\PYGZhy{}} \PYG{n}{f}\PYG{p}{(}\PYG{n}{xa}\PYG{p}{)}\PYG{p}{)}
                        
        \PYG{n}{e} \PYG{o}{=} \PYG{n+nb}{abs}\PYG{p}{(}\PYG{n}{x}\PYG{o}{\PYGZhy{}}\PYG{n}{xb}\PYG{p}{)}\PYG{o}{/}\PYG{n+nb}{abs}\PYG{p}{(}\PYG{n}{x}\PYG{p}{)} \PYG{c+c1}{\PYGZsh{} erro}
        
        \PYG{c+c1}{\PYGZsh{} tabela}
        \PYG{n+nb}{print}\PYG{p}{(}\PYG{l+s+s1}{\PYGZsq{}}\PYG{l+s+si}{\PYGZob{}0:d\PYGZcb{}}\PYG{l+s+s1}{  }\PYG{l+s+si}{\PYGZob{}1:f\PYGZcb{}}\PYG{l+s+s1}{  }\PYG{l+s+si}{\PYGZob{}2:f\PYGZcb{}}\PYG{l+s+s1}{  }\PYG{l+s+si}{\PYGZob{}3:e\PYGZcb{}}\PYG{l+s+s1}{\PYGZsq{}}\PYG{o}{.}\PYG{n}{format}\PYG{p}{(}\PYG{n}{i}\PYG{p}{,}\PYG{n}{x}\PYG{p}{,}\PYG{n}{f}\PYG{p}{(}\PYG{n}{x}\PYG{p}{)}\PYG{p}{,}\PYG{n}{e}\PYG{p}{)}\PYG{p}{)}
        
        \PYG{k}{if} \PYG{n}{e} \PYG{o}{\PYGZlt{}} \PYG{n}{tol}\PYG{p}{:}
            \PYG{k}{break}
        \PYG{n}{xa} \PYG{o}{=} \PYG{n}{xb}
        \PYG{n}{xb} \PYG{o}{=} \PYG{n}{x}
        
    \PYG{k}{if} \PYG{n}{i} \PYG{o}{==} \PYG{n}{nmax}\PYG{p}{:}
        \PYG{n+nb}{print}\PYG{p}{(}\PYG{l+s+s1}{\PYGZsq{}}\PYG{l+s+s1}{Solução não obtida em }\PYG{l+s+si}{\PYGZob{}0:d\PYGZcb{}}\PYG{l+s+s1}{ iterações}\PYG{l+s+s1}{\PYGZsq{}}\PYG{o}{.}\PYG{n}{format}\PYG{p}{(}\PYG{n}{nmax}\PYG{p}{)}\PYG{p}{)}
    \PYG{k}{else}\PYG{p}{:}
        \PYG{n+nb}{print}\PYG{p}{(}\PYG{l+s+s1}{\PYGZsq{}}\PYG{l+s+s1}{Solução obtida: x = }\PYG{l+s+si}{\PYGZob{}0:.10f\PYGZcb{}}\PYG{l+s+s1}{\PYGZsq{}}\PYG{o}{.}\PYG{n}{format}\PYG{p}{(}\PYG{n}{x}\PYG{p}{)}\PYG{p}{)}

    \PYG{c+c1}{\PYGZsh{} plotagem}
    \PYG{k}{if} \PYG{n}{plotar}\PYG{p}{:}        
        \PYG{n}{delta} \PYG{o}{=} \PYG{l+m+mi}{3}\PYG{o}{*}\PYG{n}{x}
        \PYG{n}{dom} \PYG{o}{=} \PYG{n}{linspace}\PYG{p}{(}\PYG{n}{x}\PYG{o}{\PYGZhy{}}\PYG{n}{delta}\PYG{p}{,}\PYG{n}{x}\PYG{o}{+}\PYG{n}{delta}\PYG{p}{,}\PYG{l+m+mi}{30}\PYG{p}{)}
        \PYG{n}{plot}\PYG{p}{(}\PYG{n}{dom}\PYG{p}{,}\PYG{n}{f}\PYG{p}{(}\PYG{n}{dom}\PYG{p}{)}\PYG{p}{,}\PYG{n}{x}\PYG{p}{,}\PYG{n}{f}\PYG{p}{(}\PYG{n}{x}\PYG{p}{)}\PYG{p}{,}\PYG{l+s+s1}{\PYGZsq{}}\PYG{l+s+s1}{ro}\PYG{l+s+s1}{\PYGZsq{}}\PYG{p}{)}

    \PYG{k}{return} \PYG{n}{x}
      
    
\PYG{c+c1}{\PYGZsh{} parametros    }
\PYG{n}{xa} \PYG{o}{=} \PYG{l+m+mf}{1.0} \PYG{c+c1}{\PYGZsh{} estimativa inicial 1}
\PYG{n}{xb} \PYG{o}{=} \PYG{l+m+mf}{2.0} \PYG{c+c1}{\PYGZsh{} estimativa inicial 2}
\PYG{n}{tol} \PYG{o}{=} \PYG{l+m+mf}{1e\PYGZhy{}3} \PYG{c+c1}{\PYGZsh{} tolerancia}
\PYG{n}{nmax} \PYG{o}{=} \PYG{l+m+mi}{100} \PYG{c+c1}{\PYGZsh{} numero maximo de iteracoes}
\PYG{n}{f} \PYG{o}{=} \PYG{l+s+s1}{\PYGZsq{}}\PYG{l+s+s1}{\PYGZhy{}0.9*x**2 + 1.7*x + 2.5}\PYG{l+s+s1}{\PYGZsq{}}   \PYG{c+c1}{\PYGZsh{} funcao}
\PYG{n}{var} \PYG{o}{=} \PYG{l+s+s1}{\PYGZsq{}}\PYG{l+s+s1}{x}\PYG{l+s+s1}{\PYGZsq{}}
\PYG{n}{plotar} \PYG{o}{=} \PYG{k+kc}{True}

\PYG{c+c1}{\PYGZsh{} chamada da função}
\PYG{n}{xm} \PYG{o}{=} \PYG{n}{secante}\PYG{p}{(}\PYG{n}{xa}\PYG{p}{,}\PYG{n}{xb}\PYG{p}{,}\PYG{n}{f}\PYG{p}{,}\PYG{n}{tol}\PYG{p}{,}\PYG{n}{nmax}\PYG{p}{,}\PYG{n}{var}\PYG{p}{,}\PYG{n}{plotar}\PYG{p}{)}
\end{sphinxVerbatim}

\begin{sphinxVerbatim}[commandchars=\\\{\}]
Estimativas iniciais: xa = 1.0; xb = 2.0 

0  4.300000  \PYGZhy{}6.831000  5.348837e\PYGZhy{}01
1  2.579345  0.897168  6.670898e\PYGZhy{}01
2  2.779097  0.273423  7.187654e\PYGZhy{}02
3  2.866660  \PYGZhy{}0.022642  3.054518e\PYGZhy{}02
4  2.859963  0.000487  2.341478e\PYGZhy{}03
5  2.860104  0.000001  4.933559e\PYGZhy{}05
Solução obtida: x = 2.8601041641
\end{sphinxVerbatim}

\noindent\sphinxincludegraphics{{aula-07-secante_2_1}.png}


\section{Problema}
\label{\detokenize{aula-07-secante:problema}}
\sphinxAtStartPar
Determinar a raiz positiva da equação: \(f(x) = \sqrt{x} - 5e^{-x}\), pelo método das secantes com erro inferior a \(10^{-2}\).


\subsection{Resolução}
\label{\detokenize{aula-07-secante:resolucao}}
\sphinxAtStartPar
Para obtermos os valores iniciais \(x_0\) e \(x_1\) necessários para iniciar o processo iterativo, dividimos a equação original \(f(x) = 0\) em outras duas \(y_1\) e \(y_2\), com \(y_1 = \sqrt{x}\) e \(y_2(x) = e^{-x}\), que colocadas no mesmo gráfico, produzem uma interseção próximo a \(x = 1.5\). Assim, podemos escolher duas estimativas iniciais próximas deste valor. Podemos escolher \(x_0 = 1.4\) e \(x_1=1.5\).

\begin{sphinxVerbatim}[commandchars=\\\{\}]
\PYG{k+kn}{from} \PYG{n+nn}{numpy} \PYG{k+kn}{import} \PYG{n}{sqrt}\PYG{p}{,} \PYG{n}{exp}
\PYG{k+kn}{from} \PYG{n+nn}{matplotlib}\PYG{n+nn}{.}\PYG{n+nn}{pyplot} \PYG{k+kn}{import} \PYG{n}{plot}\PYG{p}{,}\PYG{n}{legend}

\PYG{n}{fx} \PYG{o}{=} \PYG{k}{lambda} \PYG{n}{x}\PYG{p}{:} \PYG{n}{sqrt}\PYG{p}{(}\PYG{n}{x}\PYG{p}{)} \PYG{o}{\PYGZhy{}} \PYG{l+m+mi}{5}\PYG{o}{*}\PYG{n}{exp}\PYG{p}{(}\PYG{o}{\PYGZhy{}}\PYG{n}{x}\PYG{p}{)} 

\PYG{n}{x} \PYG{o}{=} \PYG{n}{linspace}\PYG{p}{(}\PYG{l+m+mi}{0}\PYG{p}{,}\PYG{l+m+mi}{3}\PYG{p}{,}\PYG{l+m+mi}{100}\PYG{p}{)}
\PYG{n}{plot}\PYG{p}{(}\PYG{n}{x}\PYG{p}{,}\PYG{n}{fx}\PYG{p}{(}\PYG{n}{x}\PYG{p}{)}\PYG{p}{,}\PYG{n}{label}\PYG{o}{=}\PYG{l+s+s1}{\PYGZsq{}}\PYG{l+s+s1}{\PYGZdl{}f(x) = x\PYGZca{}}\PYG{l+s+s1}{\PYGZob{}}\PYG{l+s+s1}{1/2\PYGZcb{} \PYGZhy{} 5e\PYGZca{}}\PYG{l+s+s1}{\PYGZob{}}\PYG{l+s+s1}{\PYGZhy{}x\PYGZcb{}\PYGZdl{}}\PYG{l+s+s1}{\PYGZsq{}}\PYG{p}{)}\PYG{p}{;}
\PYG{n}{plot}\PYG{p}{(}\PYG{n}{x}\PYG{p}{,}\PYG{n}{sqrt}\PYG{p}{(}\PYG{n}{x}\PYG{p}{)}\PYG{p}{,}\PYG{n}{label}\PYG{o}{=}\PYG{l+s+s1}{\PYGZsq{}}\PYG{l+s+s1}{\PYGZdl{}y\PYGZus{}1(x) = x\PYGZca{}}\PYG{l+s+s1}{\PYGZob{}}\PYG{l+s+s1}{1/2\PYGZcb{}\PYGZdl{}}\PYG{l+s+s1}{\PYGZsq{}}\PYG{p}{)}\PYG{p}{;}
\PYG{n}{plot}\PYG{p}{(}\PYG{n}{x}\PYG{p}{,}\PYG{l+m+mi}{5}\PYG{o}{*}\PYG{n}{exp}\PYG{p}{(}\PYG{o}{\PYGZhy{}}\PYG{n}{x}\PYG{p}{)}\PYG{p}{,}\PYG{n}{label}\PYG{o}{=}\PYG{l+s+s1}{\PYGZsq{}}\PYG{l+s+s1}{\PYGZdl{}y\PYGZus{}2(x) = 5e\PYGZca{}}\PYG{l+s+s1}{\PYGZob{}}\PYG{l+s+s1}{\PYGZhy{}x\PYGZcb{}\PYGZdl{}}\PYG{l+s+s1}{\PYGZsq{}}\PYG{p}{)}\PYG{p}{;}
\PYG{n}{plot}\PYG{p}{(}\PYG{n}{x}\PYG{p}{,}\PYG{n}{fx}\PYG{p}{(}\PYG{n}{x}\PYG{p}{)}\PYG{o}{*}\PYG{l+m+mi}{0}\PYG{p}{,}\PYG{l+s+s1}{\PYGZsq{}}\PYG{l+s+s1}{\PYGZhy{}\PYGZhy{}}\PYG{l+s+s1}{\PYGZsq{}}\PYG{p}{)}\PYG{p}{;}
\PYG{n}{legend}\PYG{p}{(}\PYG{p}{)}\PYG{p}{;}
\end{sphinxVerbatim}

\noindent\sphinxincludegraphics{{aula-07-secante_6_0}.png}

\sphinxAtStartPar
Vejamos o valor de \(f(x=1.5)\).

\begin{sphinxVerbatim}[commandchars=\\\{\}]
\PYG{n}{fx}\PYG{p}{(}\PYG{l+m+mf}{1.5}\PYG{p}{)}
\end{sphinxVerbatim}

\begin{sphinxVerbatim}[commandchars=\\\{\}]
0.10909407064943988
\end{sphinxVerbatim}

\sphinxAtStartPar
Vamos montar uma função anônima para computar o valor da interseção da secante com o eixo \(x\), a saber:

\begin{sphinxVerbatim}[commandchars=\\\{\}]
\PYG{n}{xm} \PYG{o}{=} \PYG{k}{lambda} \PYG{n}{a}\PYG{p}{,}\PYG{n}{b}\PYG{p}{:} \PYG{p}{(} \PYG{n}{a}\PYG{o}{*}\PYG{n}{fx}\PYG{p}{(}\PYG{n}{b}\PYG{p}{)} \PYG{o}{\PYGZhy{}} \PYG{n}{b}\PYG{o}{*}\PYG{n}{fx}\PYG{p}{(}\PYG{n}{a}\PYG{p}{)} \PYG{p}{)} \PYG{o}{/} \PYG{p}{(}\PYG{n}{fx}\PYG{p}{(}\PYG{n}{b}\PYG{p}{)} \PYG{o}{\PYGZhy{}} \PYG{n}{fx}\PYG{p}{(}\PYG{n}{a}\PYG{p}{)} \PYG{p}{)}
\end{sphinxVerbatim}

\sphinxAtStartPar
Vamos usar os nosso valores estimados:

\begin{sphinxVerbatim}[commandchars=\\\{\}]
\PYG{n}{x0} \PYG{o}{=} \PYG{l+m+mf}{1.4}
\PYG{n}{x1} \PYG{o}{=} \PYG{l+m+mf}{1.5}
\PYG{n}{x2} \PYG{o}{=} \PYG{n+nb}{round}\PYG{p}{(}\PYG{n}{xm}\PYG{p}{(}\PYG{n}{x0}\PYG{p}{,}\PYG{n}{x1}\PYG{p}{)}\PYG{p}{,}\PYG{l+m+mi}{3}\PYG{p}{)} 
\PYG{n+nb}{print}\PYG{p}{(}\PYG{n}{x2}\PYG{p}{)}
\end{sphinxVerbatim}

\begin{sphinxVerbatim}[commandchars=\\\{\}]
1.431
\end{sphinxVerbatim}

\sphinxAtStartPar
Agora, usamos este novo valor e o anterior.

\begin{sphinxVerbatim}[commandchars=\\\{\}]
\PYG{n}{x3} \PYG{o}{=} \PYG{n+nb}{round}\PYG{p}{(}\PYG{n}{xm}\PYG{p}{(}\PYG{n}{x1}\PYG{p}{,}\PYG{n}{x2}\PYG{p}{)}\PYG{p}{,}\PYG{l+m+mi}{3}\PYG{p}{)}
\PYG{n+nb}{print}\PYG{p}{(}\PYG{n}{x3}\PYG{p}{)}
\end{sphinxVerbatim}

\begin{sphinxVerbatim}[commandchars=\\\{\}]
1.43
\end{sphinxVerbatim}

\sphinxAtStartPar
Calculemos o erro relativo entre as estimativas \(x_1\) e \(x_2\):

\begin{sphinxVerbatim}[commandchars=\\\{\}]
\PYG{n}{err} \PYG{o}{=} \PYG{k}{lambda} \PYG{n}{a}\PYG{p}{,}\PYG{n}{b}\PYG{p}{:} \PYG{n+nb}{abs}\PYG{p}{(}\PYG{n}{a} \PYG{o}{\PYGZhy{}} \PYG{n}{b}\PYG{p}{)}\PYG{o}{/}\PYG{n+nb}{abs}\PYG{p}{(}\PYG{n}{a}\PYG{p}{)}
\PYG{n}{e1} \PYG{o}{=} \PYG{n}{err}\PYG{p}{(}\PYG{n}{x2}\PYG{p}{,}\PYG{n}{x1}\PYG{p}{)}
\PYG{n+nb}{print}\PYG{p}{(}\PYG{n+nb}{round}\PYG{p}{(}\PYG{n}{e1}\PYG{p}{,}\PYG{l+m+mi}{3}\PYG{p}{)}\PYG{p}{)}
\PYG{n+nb}{print}\PYG{p}{(}\PYG{l+s+s2}{\PYGZdq{}}\PYG{l+s+si}{\PYGZob{}0:e\PYGZcb{}}\PYG{l+s+s2}{\PYGZdq{}}\PYG{o}{.}\PYG{n}{format}\PYG{p}{(}\PYG{n}{e1}\PYG{p}{)}\PYG{p}{)}
\end{sphinxVerbatim}

\begin{sphinxVerbatim}[commandchars=\\\{\}]
0.048
4.821803e\PYGZhy{}02
\end{sphinxVerbatim}

\sphinxAtStartPar
Agora, calculemos o erro relativo entre as estimativas \(x_2\) e \(x_3\):

\begin{sphinxVerbatim}[commandchars=\\\{\}]
\PYG{n}{e2} \PYG{o}{=} \PYG{n}{err}\PYG{p}{(}\PYG{n}{x3}\PYG{p}{,}\PYG{n}{x2}\PYG{p}{)}
\PYG{n+nb}{print}\PYG{p}{(}\PYG{n+nb}{round}\PYG{p}{(}\PYG{n}{e2}\PYG{p}{,}\PYG{l+m+mi}{3}\PYG{p}{)}\PYG{p}{)}
\PYG{n+nb}{print}\PYG{p}{(}\PYG{l+s+s2}{\PYGZdq{}}\PYG{l+s+si}{\PYGZob{}0:e\PYGZcb{}}\PYG{l+s+s2}{\PYGZdq{}}\PYG{o}{.}\PYG{n}{format}\PYG{p}{(}\PYG{n}{e2}\PYG{p}{)}\PYG{p}{)}
\end{sphinxVerbatim}

\begin{sphinxVerbatim}[commandchars=\\\{\}]
0.001
6.993007e\PYGZhy{}04
\end{sphinxVerbatim}

\sphinxAtStartPar
O erro está diminuindo. Além disso, o valor da raiz está se estabilizando em torno de 1.430. Isto significa que as estimativas iniciais foram muito boas. Com efeito, o uso das interseções proporcionou uma boa escolha.


\chapter{Raízes de polinômios com o método de Müller}
\label{\detokenize{aula-08-muller:raizes-de-polinomios-com-o-metodo-de-muller}}\label{\detokenize{aula-08-muller::doc}}
\begin{sphinxVerbatim}[commandchars=\\\{\}]
\PYG{o}{\PYGZpc{}}\PYG{k}{matplotlib} inline
\end{sphinxVerbatim}


\section{Computação com polinômios}
\label{\detokenize{aula-08-muller:computacao-com-polinomios}}
\sphinxAtStartPar
Como exemplo, vamos implementar a forma aninhada de um polinômio de grau 3 (também conhecida como \sphinxstylestrong{forma de Hörner})

\begin{sphinxVerbatim}[commandchars=\\\{\}]
\PYG{k+kn}{import} \PYG{n+nn}{sympy} \PYG{k}{as} \PYG{n+nn}{sy}
\PYG{k+kn}{import} \PYG{n+nn}{numpy} \PYG{k}{as} \PYG{n+nn}{np}
\PYG{k+kn}{import} \PYG{n+nn}{matplotlib}\PYG{n+nn}{.}\PYG{n+nn}{pyplot} \PYG{k}{as} \PYG{n+nn}{plt}

\PYG{n}{sy}\PYG{o}{.}\PYG{n}{init\PYGZus{}printing}\PYG{p}{(}\PYG{p}{)}

\PYG{c+c1}{\PYGZsh{} escreve a forma aninhada de um polinomio de grau n}

\PYG{c+c1}{\PYGZsh{} grau do polinomio}
\PYG{n}{n} \PYG{o}{=} \PYG{l+m+mi}{3}

\PYG{c+c1}{\PYGZsh{} variavel independente}
\PYG{n}{x} \PYG{o}{=} \PYG{n}{sy}\PYG{o}{.}\PYG{n}{Symbol}\PYG{p}{(}\PYG{l+s+s1}{\PYGZsq{}}\PYG{l+s+s1}{x}\PYG{l+s+s1}{\PYGZsq{}}\PYG{p}{)}\PYG{p}{;}       

\PYG{c+c1}{\PYGZsh{} coeficientes do polinomio}
\PYG{n}{a} \PYG{o}{=} \PYG{p}{[} \PYG{n}{sy}\PYG{o}{.}\PYG{n}{Symbol}\PYG{p}{(}\PYG{l+s+s1}{\PYGZsq{}}\PYG{l+s+s1}{a}\PYG{l+s+s1}{\PYGZsq{}}\PYG{o}{+} \PYG{n+nb}{str}\PYG{p}{(}\PYG{n}{i}\PYG{p}{)}\PYG{p}{)} \PYGZbs{}
     \PYG{k}{for} \PYG{n}{i} \PYG{o+ow}{in} \PYG{n+nb}{range}\PYG{p}{(}\PYG{l+m+mi}{0}\PYG{p}{,}\PYG{n}{n}\PYG{o}{+}\PYG{l+m+mi}{1}\PYG{p}{)} \PYG{p}{]} 

\PYG{c+c1}{\PYGZsh{} forma aninhada simbolica}
\PYG{n}{p}\PYG{p}{,} \PYG{n}{dp} \PYG{o}{=} \PYG{l+m+mi}{0}\PYG{p}{,} \PYG{l+m+mi}{0}
\PYG{k}{for} \PYG{n}{j} \PYG{o+ow}{in} \PYG{n+nb}{range}\PYG{p}{(}\PYG{n}{n}\PYG{p}{,}\PYG{o}{\PYGZhy{}}\PYG{l+m+mi}{1}\PYG{p}{,}\PYG{o}{\PYGZhy{}}\PYG{l+m+mi}{1}\PYG{p}{)}\PYG{p}{:}
    \PYG{n}{dp} \PYG{o}{=} \PYG{n}{dp}\PYG{o}{*}\PYG{n}{x} \PYG{o}{+} \PYG{n}{p}
    \PYG{n}{p}  \PYG{o}{=} \PYG{n}{a}\PYG{p}{[}\PYG{n}{j}\PYG{p}{]} \PYG{o}{+} \PYG{n}{p}\PYG{o}{*}\PYG{n}{x}
    
\PYG{c+c1}{\PYGZsh{} determinacao de derivada de modo simbolico }
\PYG{n}{dp2} \PYG{o}{=} \PYG{n}{sy}\PYG{o}{.}\PYG{n}{diff}\PYG{p}{(}\PYG{n}{p}\PYG{p}{,}\PYG{n}{x}\PYG{p}{)}
\end{sphinxVerbatim}

\sphinxAtStartPar
Imprimindo o polinômio simbólico

\begin{sphinxVerbatim}[commandchars=\\\{\}]
\PYG{n}{p}
\end{sphinxVerbatim}

\noindent\sphinxincludegraphics{{aula-08-muller_5_0}.png}

\sphinxAtStartPar
Imprimindo a derivada simbólica do polinômio implementada pelo usuário

\begin{sphinxVerbatim}[commandchars=\\\{\}]
\PYG{n}{dp}
\end{sphinxVerbatim}

\noindent\sphinxincludegraphics{{aula-08-muller_7_0}.png}

\sphinxAtStartPar
Imprimindo a derivada simbólica do polinômio pela função residente \sphinxcode{\sphinxupquote{diff}}

\begin{sphinxVerbatim}[commandchars=\\\{\}]
\PYG{n}{dp2}
\end{sphinxVerbatim}

\noindent\sphinxincludegraphics{{aula-08-muller_9_0}.png}

\sphinxAtStartPar
Verificando igualdade

\begin{sphinxVerbatim}[commandchars=\\\{\}]
\PYG{n}{dp} \PYG{o}{==} \PYG{n}{dp2}
\end{sphinxVerbatim}

\begin{sphinxVerbatim}[commandchars=\\\{\}]
True
\end{sphinxVerbatim}


\section{Calculando raízes de polinômios}
\label{\detokenize{aula-08-muller:calculando-raizes-de-polinomios}}
\begin{sphinxVerbatim}[commandchars=\\\{\}]
\PYG{c+c1}{\PYGZsh{} define valores dos coeficientes aj para o polinômio}
\PYG{c+c1}{\PYGZsh{} na ordem a0 + a1x + a2x**2 + ...}
\PYG{n}{v} \PYG{o}{=} \PYG{p}{[}\PYG{o}{\PYGZhy{}}\PYG{l+m+mi}{1}\PYG{p}{,} \PYG{l+m+mf}{2.2}\PYG{p}{,}\PYG{l+m+mf}{3.5}\PYG{p}{,}\PYG{l+m+mi}{4}\PYG{p}{]}

\PYG{c+c1}{\PYGZsh{} escreve o polinômio}
\PYG{n}{pn} \PYG{o}{=} \PYG{n}{p}\PYG{o}{.}\PYG{n}{subs}\PYG{p}{(}\PYG{n+nb}{dict}\PYG{p}{(}\PYG{n+nb}{zip}\PYG{p}{(}\PYG{n}{a}\PYG{p}{,}\PYG{n}{v}\PYG{p}{)}\PYG{p}{)}\PYG{p}{)}
\PYG{n+nb}{print}\PYG{p}{(}\PYG{n}{pn}\PYG{p}{)}
\end{sphinxVerbatim}

\begin{sphinxVerbatim}[commandchars=\\\{\}]
x*(x*(4*x + 3.5) + 2.2) \PYGZhy{} 1
\end{sphinxVerbatim}

\sphinxAtStartPar
Calcula todas as raízes do polinômio \sphinxcode{\sphinxupquote{pn}}

\begin{sphinxVerbatim}[commandchars=\\\{\}]
\PYG{n}{rc} \PYG{o}{=} \PYG{n}{sy}\PYG{o}{.}\PYG{n}{roots}\PYG{p}{(}\PYG{n}{pn}\PYG{p}{,}\PYG{n}{x}\PYG{p}{,}\PYG{n}{multiple}\PYG{o}{=}\PYG{k+kc}{True}\PYG{p}{)}
\PYG{n}{rc}
\end{sphinxVerbatim}

\noindent\sphinxincludegraphics{{aula-08-muller_15_0}.png}

\sphinxAtStartPar
Calcula apenas as raízes reais de \sphinxcode{\sphinxupquote{pn}}

\begin{sphinxVerbatim}[commandchars=\\\{\}]
\PYG{n}{rr} \PYG{o}{=} \PYG{n}{sy}\PYG{o}{.}\PYG{n}{roots}\PYG{p}{(}\PYG{n}{pn}\PYG{p}{,}\PYG{n}{x}\PYG{p}{,}\PYG{n}{multiple}\PYG{o}{=}\PYG{k+kc}{True}\PYG{p}{,}\PYG{n+nb}{filter}\PYG{o}{=}\PYG{l+s+s1}{\PYGZsq{}}\PYG{l+s+s1}{R}\PYG{l+s+s1}{\PYGZsq{}}\PYG{p}{)}
\PYG{n}{rr}
\end{sphinxVerbatim}

\noindent\sphinxincludegraphics{{aula-08-muller_17_0}.png}


\section{Avaliando polinômios}
\label{\detokenize{aula-08-muller:avaliando-polinomios}}
\sphinxAtStartPar
Podemos avaliar polinômios usando a função \sphinxcode{\sphinxupquote{polyval}} do \sphinxstyleemphasis{Numpy}. Entretanto, como ela recebe coeficientes do maior para o menor grau, para mantermos a consistência com nosso polinômio anterior, devemos converter a lista \sphinxcode{\sphinxupquote{v}} para um objeto \sphinxcode{\sphinxupquote{array}} e fazer uma inversão (\sphinxcode{\sphinxupquote{flip}}).

\begin{sphinxVerbatim}[commandchars=\\\{\}]
\PYG{n}{vi} \PYG{o}{=} \PYG{n}{np}\PYG{o}{.}\PYG{n}{flip}\PYG{p}{(}\PYG{n}{np}\PYG{o}{.}\PYG{n}{asarray}\PYG{p}{(}\PYG{n}{v}\PYG{p}{)}\PYG{p}{,}\PYG{n}{axis}\PYG{o}{=}\PYG{l+m+mi}{0}\PYG{p}{)}
\PYG{n}{vi}
\end{sphinxVerbatim}

\begin{sphinxVerbatim}[commandchars=\\\{\}]
array([ 4. ,  3.5,  2.2, \PYGZhy{}1. ])
\end{sphinxVerbatim}

\sphinxAtStartPar
Agora, vamos avaliar o polinômio em \(x=\pi\)

\begin{sphinxVerbatim}[commandchars=\\\{\}]
\PYG{n}{xi} \PYG{o}{=} \PYG{n}{np}\PYG{o}{.}\PYG{n}{pi}
\PYG{n}{np}\PYG{o}{.}\PYG{n}{polyval}\PYG{p}{(}\PYG{n}{vi}\PYG{p}{,}\PYG{n}{xi}\PYG{p}{)}
\end{sphinxVerbatim}

\noindent\sphinxincludegraphics{{aula-08-muller_21_0}.png}

\sphinxAtStartPar
Note que se avaliássemos o polinômio em um ponto arbitrário, a forma impressa é idêntica àquela que obtivemos anteriormente.

\begin{sphinxVerbatim}[commandchars=\\\{\}]
\PYG{n}{np}\PYG{o}{.}\PYG{n}{polyval}\PYG{p}{(}\PYG{n}{vi}\PYG{p}{,}\PYG{n}{x}\PYG{p}{)}
\end{sphinxVerbatim}

\noindent\sphinxincludegraphics{{aula-08-muller_23_0}.png}

\sphinxAtStartPar
Agora vamos plotar o polinômio. Antes, vamos converter nosso polinômio para uma função a fim de a avaliarmos em um intervalo. Vamos escolher o intervalo \(-1 \le x \le 1\)

\begin{sphinxVerbatim}[commandchars=\\\{\}]
\PYG{c+c1}{\PYGZsh{} converte para função}
\PYG{n}{f} \PYG{o}{=} \PYG{n}{sy}\PYG{o}{.}\PYG{n}{lambdify}\PYG{p}{(}\PYG{n}{x}\PYG{p}{,}\PYG{n}{pn}\PYG{p}{)}

\PYG{c+c1}{\PYGZsh{} intervalo}
\PYG{n}{xf} \PYG{o}{=} \PYG{n}{np}\PYG{o}{.}\PYG{n}{linspace}\PYG{p}{(}\PYG{o}{\PYGZhy{}}\PYG{l+m+mi}{1}\PYG{p}{,}\PYG{l+m+mi}{1}\PYG{p}{,}\PYG{n}{num}\PYG{o}{=}\PYG{l+m+mi}{20}\PYG{p}{,}\PYG{n}{endpoint}\PYG{o}{=}\PYG{k+kc}{True}\PYG{p}{)}

\PYG{c+c1}{\PYGZsh{} plotagem do polinômio com destaque para a raiz real}
\PYG{n}{plt}\PYG{o}{.}\PYG{n}{axhline}\PYG{p}{(}\PYG{n}{y}\PYG{o}{=}\PYG{n}{f}\PYG{p}{(}\PYG{n}{rr}\PYG{p}{[}\PYG{l+m+mi}{0}\PYG{p}{]}\PYG{p}{)}\PYG{p}{,}\PYG{n}{c}\PYG{o}{=}\PYG{l+s+s1}{\PYGZsq{}}\PYG{l+s+s1}{r}\PYG{l+s+s1}{\PYGZsq{}}\PYG{p}{,}\PYG{n}{ls}\PYG{o}{=}\PYG{l+s+s1}{\PYGZsq{}}\PYG{l+s+s1}{\PYGZhy{}\PYGZhy{}}\PYG{l+s+s1}{\PYGZsq{}}\PYG{p}{)}
\PYG{n}{plt}\PYG{o}{.}\PYG{n}{axvline}\PYG{p}{(}\PYG{n}{x}\PYG{o}{=}\PYG{n}{rr}\PYG{p}{[}\PYG{l+m+mi}{0}\PYG{p}{]}\PYG{p}{,}\PYG{n}{c}\PYG{o}{=}\PYG{l+s+s1}{\PYGZsq{}}\PYG{l+s+s1}{r}\PYG{l+s+s1}{\PYGZsq{}}\PYG{p}{,}\PYG{n}{ls}\PYG{o}{=}\PYG{l+s+s1}{\PYGZsq{}}\PYG{l+s+s1}{\PYGZhy{}\PYGZhy{}}\PYG{l+s+s1}{\PYGZsq{}}\PYG{p}{)}
\PYG{n}{plt}\PYG{o}{.}\PYG{n}{plot}\PYG{p}{(}\PYG{n}{xf}\PYG{p}{,}\PYG{n}{f}\PYG{p}{(}\PYG{n}{xf}\PYG{p}{)}\PYG{p}{)}
\PYG{n}{plt}\PYG{o}{.}\PYG{n}{plot}\PYG{p}{(}\PYG{n}{rr}\PYG{p}{[}\PYG{l+m+mi}{0}\PYG{p}{]}\PYG{p}{,}\PYG{n}{f}\PYG{p}{(}\PYG{n}{rr}\PYG{p}{[}\PYG{l+m+mi}{0}\PYG{p}{]}\PYG{p}{)}\PYG{p}{,}\PYG{l+s+s1}{\PYGZsq{}}\PYG{l+s+s1}{ro}\PYG{l+s+s1}{\PYGZsq{}}\PYG{p}{)}
\PYG{n}{plt}\PYG{o}{.}\PYG{n}{xlabel}\PYG{p}{(}\PYG{l+s+s1}{\PYGZsq{}}\PYG{l+s+s1}{\PYGZdl{}x\PYGZdl{}}\PYG{l+s+s1}{\PYGZsq{}}\PYG{p}{)}
\PYG{n}{plt}\PYG{o}{.}\PYG{n}{ylabel}\PYG{p}{(}\PYG{l+s+s1}{\PYGZsq{}}\PYG{l+s+s1}{\PYGZdl{}P(x)\PYGZdl{}}\PYG{l+s+s1}{\PYGZsq{}}\PYG{p}{)}\PYG{p}{;}
\end{sphinxVerbatim}

\noindent\sphinxincludegraphics{{aula-08-muller_25_0}.png}


\section{Implementação: Método de Müller}
\label{\detokenize{aula-08-muller:implementacao-metodo-de-muller}}
\begin{sphinxVerbatim}[commandchars=\\\{\}]
\PYG{c+c1}{\PYGZsh{} \PYGZbs{}TODO caso complexo (verificar aritmética)}
\PYG{k}{def} \PYG{n+nf}{metodo\PYGZus{}muller}\PYG{p}{(}\PYG{n}{f}\PYG{p}{,}\PYG{n}{x0}\PYG{p}{,}\PYG{n}{dx}\PYG{p}{,}\PYG{n}{EPS}\PYG{p}{,}\PYG{n}{N}\PYG{p}{)}\PYG{p}{:}
    \PYG{l+s+sd}{\PYGZdq{}\PYGZdq{}\PYGZdq{}}
\PYG{l+s+sd}{        Busca aproximação para raiz da função f}
\PYG{l+s+sd}{        pelo método de Muller.}
\PYG{l+s+sd}{        }
\PYG{l+s+sd}{        ENTRADA: }
\PYG{l+s+sd}{             f: função; ex. f = lambda x: x\PYGZca{}3 + 2*x}
\PYG{l+s+sd}{            x0: estimativa inicial }
\PYG{l+s+sd}{             h: incremento (produz valores vizinhos)}
\PYG{l+s+sd}{           EPS: erro}
\PYG{l+s+sd}{             N: iterações}
\PYG{l+s+sd}{        SAÍDA: }
\PYG{l+s+sd}{             x: aproximação de raiz de f}
\PYG{l+s+sd}{    \PYGZdq{}\PYGZdq{}\PYGZdq{}}
    
    \PYG{k}{if} \PYG{n}{N} \PYG{o}{\PYGZlt{}} \PYG{l+m+mi}{3}\PYG{p}{:}
        \PYG{k}{raise}\PYG{p}{(}\PYG{l+s+s2}{\PYGZdq{}}\PYG{l+s+s2}{N deve ser maior do que 3}\PYG{l+s+s2}{\PYGZdq{}}\PYG{p}{)}
         
    \PYG{c+c1}{\PYGZsh{} escolhendo os dois pontos adicionais }
    \PYG{c+c1}{\PYGZsh{} na vizinhança de x0 para ter as 3}
    \PYG{c+c1}{\PYGZsh{} estimativas iniciais}
    \PYG{n}{x1} \PYG{o}{=} \PYG{n}{x0} \PYG{o}{\PYGZhy{}} \PYG{n}{dx}
    \PYG{n}{x2} \PYG{o}{=} \PYG{n}{x0} \PYG{o}{+} \PYG{n}{dx}
    
    \PYG{n}{h0} \PYG{o}{=} \PYG{n}{x1} \PYG{o}{\PYGZhy{}} \PYG{n}{x0}
    \PYG{n}{h1} \PYG{o}{=} \PYG{n}{x2} \PYG{o}{\PYGZhy{}} \PYG{n}{x1}
    \PYG{n}{d0} \PYG{o}{=} \PYG{p}{(}\PYG{n}{f}\PYG{p}{(}\PYG{n}{x1}\PYG{p}{)} \PYG{o}{\PYGZhy{}} \PYG{n}{f}\PYG{p}{(}\PYG{n}{x0}\PYG{p}{)}\PYG{p}{)}\PYG{o}{/}\PYG{n}{h0}
    \PYG{n}{d1} \PYG{o}{=} \PYG{p}{(}\PYG{n}{f}\PYG{p}{(}\PYG{n}{x2}\PYG{p}{)} \PYG{o}{\PYGZhy{}} \PYG{n}{f}\PYG{p}{(}\PYG{n}{x1}\PYG{p}{)}\PYG{p}{)}\PYG{o}{/}\PYG{n}{h1}
    \PYG{n}{d} \PYG{o}{=} \PYG{p}{(}\PYG{n}{d1} \PYG{o}{\PYGZhy{}} \PYG{n}{d0}\PYG{p}{)}\PYG{o}{/}\PYG{p}{(}\PYG{n}{h1} \PYG{o}{+} \PYG{n}{h0}\PYG{p}{)}
    \PYG{n}{i} \PYG{o}{=} \PYG{l+m+mi}{3}
    \PYG{k}{while} \PYG{n}{i} \PYG{o}{\PYGZlt{}}\PYG{o}{=} \PYG{n}{N}\PYG{p}{:}

        \PYG{n}{b} \PYG{o}{=} \PYG{n}{d1} \PYG{o}{+} \PYG{n}{h1}\PYG{o}{*}\PYG{n}{d}
        
        \PYG{c+c1}{\PYGZsh{} discriminante }
        \PYG{n}{D} \PYG{o}{=} \PYG{p}{(}\PYG{n}{b}\PYG{o}{*}\PYG{o}{*}\PYG{l+m+mi}{2} \PYG{o}{\PYGZhy{}} \PYG{l+m+mi}{4}\PYG{o}{*}\PYG{n}{f}\PYG{p}{(}\PYG{n}{x2}\PYG{p}{)}\PYG{o}{*}\PYG{n}{d}\PYG{p}{)}\PYG{o}{*}\PYG{o}{*}\PYG{l+m+mf}{0.5}        
        
        \PYG{c+c1}{\PYGZsh{} Verificando o denominador:}
        \PYG{c+c1}{\PYGZsh{} Esta condição irá definir o maior denominador}
        \PYG{c+c1}{\PYGZsh{} haja vista que b + sgn(b)D.}
        \PYG{c+c1}{\PYGZsh{} (critério de sgn(b))}
        \PYG{k}{if} \PYG{n+nb}{abs}\PYG{p}{(}\PYG{n}{b} \PYG{o}{\PYGZhy{}} \PYG{n}{D}\PYG{p}{)} \PYG{o}{\PYGZlt{}} \PYG{n+nb}{abs}\PYG{p}{(}\PYG{n}{b} \PYG{o}{+} \PYG{n}{D}\PYG{p}{)}\PYG{p}{:}
            \PYG{n}{E} \PYG{o}{=} \PYG{n}{b} \PYG{o}{+} \PYG{n}{D}
        \PYG{k}{else}\PYG{p}{:}
            \PYG{n}{E} \PYG{o}{=} \PYG{n}{b} \PYG{o}{\PYGZhy{}} \PYG{n}{D}
        
        \PYG{n}{h} \PYG{o}{=} \PYG{o}{\PYGZhy{}}\PYG{l+m+mi}{2}\PYG{o}{*}\PYG{n}{f}\PYG{p}{(}\PYG{n}{x2}\PYG{p}{)}\PYG{o}{/}\PYG{n}{E}
        \PYG{n}{x} \PYG{o}{=} \PYG{n}{x2} \PYG{o}{+} \PYG{n}{h}
        \PYG{k}{if} \PYG{n+nb}{abs}\PYG{p}{(}\PYG{n}{h}\PYG{p}{)} \PYG{o}{\PYGZlt{}} \PYG{n}{EPS}\PYG{p}{:}
            \PYG{k}{return} \PYG{n}{x}
        
        \PYG{c+c1}{\PYGZsh{} atualização}
        \PYG{n}{x0} \PYG{o}{=} \PYG{n}{x1}
        \PYG{n}{x1} \PYG{o}{=} \PYG{n}{x2}
        \PYG{n}{x2} \PYG{o}{=} \PYG{n}{x}        
        \PYG{n}{h0} \PYG{o}{=} \PYG{n}{x1} \PYG{o}{\PYGZhy{}} \PYG{n}{x0}
        \PYG{n}{h1} \PYG{o}{=} \PYG{n}{x2} \PYG{o}{\PYGZhy{}} \PYG{n}{x1}
        \PYG{n}{d0} \PYG{o}{=} \PYG{p}{(}\PYG{n}{f}\PYG{p}{(}\PYG{n}{x1}\PYG{p}{)} \PYG{o}{\PYGZhy{}} \PYG{n}{f}\PYG{p}{(}\PYG{n}{x0}\PYG{p}{)}\PYG{p}{)}\PYG{o}{/}\PYG{n}{h0}
        \PYG{n}{d1} \PYG{o}{=} \PYG{p}{(}\PYG{n}{f}\PYG{p}{(}\PYG{n}{x2}\PYG{p}{)} \PYG{o}{\PYGZhy{}} \PYG{n}{f}\PYG{p}{(}\PYG{n}{x1}\PYG{p}{)}\PYG{p}{)}\PYG{o}{/}\PYG{n}{h1}
        \PYG{n}{d} \PYG{o}{=} \PYG{p}{(}\PYG{n}{d1} \PYG{o}{\PYGZhy{}} \PYG{n}{d0}\PYG{p}{)}\PYG{o}{/}\PYG{p}{(}\PYG{n}{h1} \PYG{o}{+} \PYG{n}{h0}\PYG{p}{)}
        
        \PYG{n}{i} \PYG{o}{+}\PYG{o}{=} \PYG{l+m+mi}{1}
\end{sphinxVerbatim}


\section{Exemplos}
\label{\detokenize{aula-08-muller:exemplos}}
\sphinxAtStartPar
\sphinxstylestrong{Exemplo}. Determinando raízes para o polinômio \(P(x) = 4x^3 + 3.5x^2 + 2.2x - 1\) com estimativas iniciais \(x_0 = 0.5\) \(x_1 = 1.0\) e \(x_2 = 1.5\), \(\epsilon = 10^{-5}\) e \(N = 100\). Notemos que o segundo argumento da função desempenha o papel de \(x_1\) e o terceiro argumento opera como um “raio” de comprimento \(dx = 0.5\) que fará com que \(x_0 = x_1 - dx\) e \(x_2 = x_1 + dx\). Isto decorre de como o a função foi programada. Veja o código anterior.

\begin{sphinxVerbatim}[commandchars=\\\{\}]
\PYG{n}{f} \PYG{o}{=} \PYG{k}{lambda} \PYG{n}{x}\PYG{p}{:} \PYG{l+m+mi}{4}\PYG{o}{*}\PYG{n}{x}\PYG{o}{*}\PYG{o}{*}\PYG{l+m+mi}{3} \PYG{o}{+} \PYG{l+m+mf}{3.5}\PYG{o}{*}\PYG{n}{x}\PYG{o}{*}\PYG{o}{*}\PYG{l+m+mi}{2} \PYG{o}{+} \PYG{l+m+mf}{2.2}\PYG{o}{*}\PYG{n}{x} \PYG{o}{\PYGZhy{}} \PYG{l+m+mi}{1}

\PYG{n}{x0} \PYG{o}{=} \PYG{n}{metodo\PYGZus{}muller}\PYG{p}{(}\PYG{n}{f}\PYG{p}{,}\PYG{l+m+mf}{1.0}\PYG{p}{,}\PYG{l+m+mf}{0.5}\PYG{p}{,}\PYG{l+m+mf}{1e\PYGZhy{}5}\PYG{p}{,}\PYG{l+m+mi}{100}\PYG{p}{)}
\end{sphinxVerbatim}

\begin{sphinxVerbatim}[commandchars=\\\{\}]
\PYG{n}{X} \PYG{o}{=} \PYG{n}{np}\PYG{o}{.}\PYG{n}{linspace}\PYG{p}{(}\PYG{l+m+mf}{0.2}\PYG{p}{,}\PYG{l+m+mf}{1.7}\PYG{p}{,}\PYG{l+m+mi}{100}\PYG{p}{)}
\PYG{n}{plt}\PYG{o}{.}\PYG{n}{plot}\PYG{p}{(}\PYG{n}{X}\PYG{p}{,}\PYG{n}{f}\PYG{p}{(}\PYG{n}{X}\PYG{p}{)}\PYG{p}{)}
\PYG{n}{plt}\PYG{o}{.}\PYG{n}{plot}\PYG{p}{(}\PYG{n}{X}\PYG{p}{,}\PYG{l+m+mi}{0}\PYG{o}{*}\PYG{n}{f}\PYG{p}{(}\PYG{n}{X}\PYG{p}{)}\PYG{p}{)}
\PYG{n}{plt}\PYG{o}{.}\PYG{n}{axvline}\PYG{p}{(}\PYG{n}{x}\PYG{o}{=}\PYG{n}{x0}\PYG{o}{.}\PYG{n}{real}\PYG{p}{,}\PYG{n}{c}\PYG{o}{=}\PYG{l+s+s1}{\PYGZsq{}}\PYG{l+s+s1}{r}\PYG{l+s+s1}{\PYGZsq{}}\PYG{p}{,}\PYG{n}{ls}\PYG{o}{=}\PYG{l+s+s1}{\PYGZsq{}}\PYG{l+s+s1}{\PYGZhy{}\PYGZhy{}}\PYG{l+s+s1}{\PYGZsq{}}\PYG{p}{)}\PYG{p}{;}
\end{sphinxVerbatim}

\noindent\sphinxincludegraphics{{aula-08-muller_31_0}.png}

\sphinxAtStartPar
\sphinxstylestrong{Exemplo}. Determinando raízes para o polinômio \(P(x) = x^4 - 3x^3 + x^2 + x + 1\) com estimativas iniciais \(x_0 = -0.5\) \(x_1 = 0.0\) e \(x_2 = 0.5\), \(\epsilon = 10^{-5}\) e \(N = 100\).

\begin{sphinxVerbatim}[commandchars=\\\{\}]
\PYG{n}{f2} \PYG{o}{=} \PYG{k}{lambda} \PYG{n}{x}\PYG{p}{:} \PYG{n}{x}\PYG{o}{*}\PYG{o}{*}\PYG{l+m+mi}{4} \PYG{o}{\PYGZhy{}} \PYG{l+m+mi}{3}\PYG{o}{*}\PYG{n}{x}\PYG{o}{*}\PYG{o}{*}\PYG{l+m+mi}{3} \PYG{o}{+} \PYG{n}{x}\PYG{o}{*}\PYG{o}{*}\PYG{l+m+mi}{2} \PYG{o}{+} \PYG{n}{x} \PYG{o}{+} \PYG{l+m+mi}{1}

\PYG{n}{metodo\PYGZus{}muller}\PYG{p}{(}\PYG{n}{f2}\PYG{p}{,}\PYG{o}{\PYGZhy{}}\PYG{l+m+mf}{0.5}\PYG{p}{,}\PYG{l+m+mf}{0.5}\PYG{p}{,}\PYG{l+m+mf}{1e\PYGZhy{}5}\PYG{p}{,}\PYG{l+m+mi}{100}\PYG{p}{)}
\end{sphinxVerbatim}

\begin{sphinxVerbatim}[commandchars=\\\{\}]
(\PYGZhy{}0.3390928377617365\PYGZhy{}0.4466300999972928j)
\end{sphinxVerbatim}

\sphinxAtStartPar
Com essas estimativas a raiz é um número complexo. Escolhamos agora estimativas diferentes:
\begin{itemize}
\item {} 
\sphinxAtStartPar
Caso 1: \(x_0 = 0.5\) \(x_1 = 1.0\) e \(x_2 = 1.5\)

\item {} 
\sphinxAtStartPar
Caso 2: \(x_0 = 1.5\) \(x_1 = 2.0\) e \(x_2 = 2.5\)

\end{itemize}

\begin{sphinxVerbatim}[commandchars=\\\{\}]
\PYG{c+c1}{\PYGZsh{} caso 1}
\PYG{n}{c1} \PYG{o}{=} \PYG{n}{metodo\PYGZus{}muller}\PYG{p}{(}\PYG{n}{f2}\PYG{p}{,}\PYG{l+m+mf}{1.5}\PYG{p}{,}\PYG{l+m+mf}{0.5}\PYG{p}{,}\PYG{l+m+mf}{1e\PYGZhy{}5}\PYG{p}{,}\PYG{l+m+mi}{100}\PYG{p}{)}
\PYG{n+nb}{print}\PYG{p}{(}\PYG{n}{c1}\PYG{p}{)}
\end{sphinxVerbatim}

\begin{sphinxVerbatim}[commandchars=\\\{\}]
1.3893906833348133
\end{sphinxVerbatim}

\begin{sphinxVerbatim}[commandchars=\\\{\}]
\PYG{c+c1}{\PYGZsh{} caso 2}
\PYG{n}{c2} \PYG{o}{=} \PYG{n}{metodo\PYGZus{}muller}\PYG{p}{(}\PYG{n}{f2}\PYG{p}{,}\PYG{l+m+mf}{2.0}\PYG{p}{,}\PYG{l+m+mf}{0.5}\PYG{p}{,}\PYG{l+m+mf}{1e\PYGZhy{}5}\PYG{p}{,}\PYG{l+m+mi}{100}\PYG{p}{)}
\PYG{n+nb}{print}\PYG{p}{(}\PYG{n}{c2}\PYG{p}{)}
\end{sphinxVerbatim}

\begin{sphinxVerbatim}[commandchars=\\\{\}]
2.2887949921884836
\end{sphinxVerbatim}

\sphinxAtStartPar
Por que há resultados diferentes? Vamos verificar o gráfico deste polinômio no domínio \([-1,2.8]\).

\begin{sphinxVerbatim}[commandchars=\\\{\}]
\PYG{k+kn}{from} \PYG{n+nn}{matplotlib}\PYG{n+nn}{.}\PYG{n+nn}{pyplot} \PYG{k+kn}{import} \PYG{n}{plot}\PYG{p}{,}\PYG{n}{legend}
\PYG{k+kn}{from} \PYG{n+nn}{numpy} \PYG{k+kn}{import} \PYG{n}{linspace}\PYG{p}{,} \PYG{n}{where}\PYG{p}{,} \PYG{n}{logical\PYGZus{}and}

\PYG{n}{x} \PYG{o}{=} \PYG{n}{linspace}\PYG{p}{(}\PYG{o}{\PYGZhy{}}\PYG{l+m+mi}{1}\PYG{p}{,}\PYG{l+m+mf}{2.8}\PYG{p}{,}\PYG{l+m+mi}{50}\PYG{p}{)}
\PYG{n}{plot}\PYG{p}{(}\PYG{n}{x}\PYG{p}{,}\PYG{n}{f2}\PYG{p}{(}\PYG{n}{x}\PYG{p}{)}\PYG{p}{)}
\PYG{n}{plot}\PYG{p}{(}\PYG{n}{x}\PYG{p}{,}\PYG{l+m+mi}{0}\PYG{o}{*}\PYG{n}{f2}\PYG{p}{(}\PYG{n}{x}\PYG{p}{)}\PYG{p}{,}\PYG{l+s+s1}{\PYGZsq{}}\PYG{l+s+s1}{:}\PYG{l+s+s1}{\PYGZsq{}}\PYG{p}{)}
\PYG{n}{xi} \PYG{o}{=} \PYG{n}{where}\PYG{p}{(} \PYG{n}{logical\PYGZus{}and}\PYG{p}{(}\PYG{n}{x} \PYG{o}{\PYGZgt{}}\PYG{o}{=} \PYG{o}{\PYGZhy{}}\PYG{l+m+mf}{0.5}\PYG{p}{,}\PYG{n}{x} \PYG{o}{\PYGZlt{}}\PYG{o}{=} \PYG{l+m+mf}{0.5}\PYG{p}{)} \PYG{p}{)}
\PYG{n}{xi} \PYG{o}{=} \PYG{n}{x}\PYG{p}{[}\PYG{n}{xi}\PYG{p}{]}
\PYG{n}{plot}\PYG{p}{(}\PYG{n}{xi}\PYG{p}{,}\PYG{l+m+mi}{0}\PYG{o}{*}\PYG{n}{xi}\PYG{p}{,}\PYG{l+s+s1}{\PYGZsq{}}\PYG{l+s+s1}{\PYGZhy{}g}\PYG{l+s+s1}{\PYGZsq{}}\PYG{p}{,}\PYG{n}{label}\PYG{o}{=}\PYG{l+s+s1}{\PYGZsq{}}\PYG{l+s+s1}{intervalo onde a raiz é complexa}\PYG{l+s+s1}{\PYGZsq{}}\PYG{p}{)}
\PYG{n}{plot}\PYG{p}{(}\PYG{n}{c1}\PYG{p}{,}\PYG{l+m+mi}{0}\PYG{p}{,}\PYG{l+s+s1}{\PYGZsq{}}\PYG{l+s+s1}{or}\PYG{l+s+s1}{\PYGZsq{}}\PYG{p}{,}\PYG{n}{c2}\PYG{p}{,}\PYG{l+m+mi}{0}\PYG{p}{,}\PYG{l+s+s1}{\PYGZsq{}}\PYG{l+s+s1}{or}\PYG{l+s+s1}{\PYGZsq{}}\PYG{p}{,}\PYG{n}{label}\PYG{o}{=}\PYG{l+s+s1}{\PYGZsq{}}\PYG{l+s+s1}{raiz real}\PYG{l+s+s1}{\PYGZsq{}}\PYG{p}{)}
\PYG{n}{legend}\PYG{p}{(}\PYG{p}{)}\PYG{p}{;}
\end{sphinxVerbatim}

\noindent\sphinxincludegraphics{{aula-08-muller_38_0}.png}

\sphinxAtStartPar
Na primeira escolha de estimativas iniciais, obtivemos uma raiz complexa porque no intervalo \([-0.5,0.5]\), o polinômio não intersecta o eixo \(x\). Nos outros dois casos, temos as duas raízes reais do polinômio.


\chapter{Álgebra linear com Python: Eliminação Gaussiana e Condicionamento}
\label{\detokenize{aula-09-eliminacao-gauss:algebra-linear-com-python-eliminacao-gaussiana-e-condicionamento}}\label{\detokenize{aula-09-eliminacao-gauss::doc}}
\begin{sphinxVerbatim}[commandchars=\\\{\}]
\PYG{o}{\PYGZpc{}}\PYG{k}{matplotlib} inline
\end{sphinxVerbatim}


\section{Solução de sistemas lineares}
\label{\detokenize{aula-09-eliminacao-gauss:solucao-de-sistemas-lineares}}
\sphinxAtStartPar
Métodos adequados para a resolução de sistemas lineares e realizar operações no escopo da Álgebra Linear são encontrados no submódulo \sphinxcode{\sphinxupquote{linalg}} do \sphinxcode{\sphinxupquote{scipy}}. Importamos essas funcionalidades com:

\begin{sphinxVerbatim}[commandchars=\\\{\}]
\PYG{k+kn}{from} \PYG{n+nn}{scipy} \PYG{k+kn}{import} \PYG{n}{linalg}
\end{sphinxVerbatim}

\sphinxAtStartPar
Vamos calcular a solução do sistema linear \({\bf A}{\bf x} = {\bf b}\) com
\begin{equation*}
\begin{split}{\bf A} = \begin{bmatrix}
4   & -2  & -3   & 6   &  \\
-6  & 7   & 6.5  & -6  &  \\
1   & 7.5 & 6.25 & 5.5 &  \\
-12 & 22  & 15.5 & -1 
\end{bmatrix},
\quad
{\bf b} = \begin{bmatrix}
12   \\
-6.5 \\
16   \\
17 
\end{bmatrix}\end{split}
\end{equation*}
\sphinxAtStartPar
Vamos importar os módulos e escrever a matriz \({\bf A}\).

\begin{sphinxVerbatim}[commandchars=\\\{\}]
\PYG{k+kn}{import} \PYG{n+nn}{numpy} \PYG{k}{as} \PYG{n+nn}{np}
\PYG{k+kn}{from} \PYG{n+nn}{scipy} \PYG{k+kn}{import} \PYG{n}{linalg}

\PYG{n}{A} \PYG{o}{=} \PYG{n}{np}\PYG{o}{.}\PYG{n}{array}\PYG{p}{(}\PYG{p}{[}\PYG{p}{[}\PYG{l+m+mi}{4}\PYG{p}{,}\PYG{o}{\PYGZhy{}}\PYG{l+m+mi}{2}\PYG{p}{,}\PYG{o}{\PYGZhy{}}\PYG{l+m+mi}{3}\PYG{p}{,}\PYG{l+m+mi}{6}\PYG{p}{]}\PYG{p}{,}\PYG{p}{[}\PYG{o}{\PYGZhy{}}\PYG{l+m+mi}{6}\PYG{p}{,}\PYG{l+m+mi}{7}\PYG{p}{,}\PYG{l+m+mf}{6.5}\PYG{p}{,}\PYG{o}{\PYGZhy{}}\PYG{l+m+mi}{6}\PYG{p}{]}\PYG{p}{,}\PYG{p}{[}\PYG{l+m+mi}{1}\PYG{p}{,}\PYG{l+m+mf}{7.5}\PYG{p}{,}\PYG{l+m+mf}{6.25}\PYG{p}{,}\PYG{l+m+mf}{5.5}\PYG{p}{]}\PYG{p}{,}\PYG{p}{[}\PYG{o}{\PYGZhy{}}\PYG{l+m+mi}{12}\PYG{p}{,}\PYG{l+m+mi}{22}\PYG{p}{,}\PYG{l+m+mf}{15.5}\PYG{p}{,}\PYG{o}{\PYGZhy{}}\PYG{l+m+mi}{1}\PYG{p}{]}\PYG{p}{]}\PYG{p}{)}
\PYG{n+nb}{print}\PYG{p}{(}\PYG{n}{A}\PYG{p}{)}
\end{sphinxVerbatim}

\begin{sphinxVerbatim}[commandchars=\\\{\}]
[[  4.    \PYGZhy{}2.    \PYGZhy{}3.     6.  ]
 [ \PYGZhy{}6.     7.     6.5   \PYGZhy{}6.  ]
 [  1.     7.5    6.25   5.5 ]
 [\PYGZhy{}12.    22.    15.5   \PYGZhy{}1.  ]]
\end{sphinxVerbatim}

\sphinxAtStartPar
Agora, vamos escrever o vetor \({\bf b}\).

\begin{sphinxVerbatim}[commandchars=\\\{\}]
\PYG{n}{b} \PYG{o}{=} \PYG{n}{np}\PYG{o}{.}\PYG{n}{array}\PYG{p}{(}\PYG{p}{[}\PYG{l+m+mi}{12}\PYG{p}{,}\PYG{o}{\PYGZhy{}}\PYG{l+m+mf}{6.5}\PYG{p}{,}\PYG{l+m+mi}{16}\PYG{p}{,}\PYG{l+m+mi}{17}\PYG{p}{]}\PYG{p}{)}
\PYG{n+nb}{print}\PYG{p}{(}\PYG{n}{b}\PYG{p}{)}
\end{sphinxVerbatim}

\begin{sphinxVerbatim}[commandchars=\\\{\}]
[12.  \PYGZhy{}6.5 16.  17. ]
\end{sphinxVerbatim}

\sphinxAtStartPar
Podemos checar as dimensões com

\begin{sphinxVerbatim}[commandchars=\\\{\}]
\PYG{c+c1}{\PYGZsh{} dimensões de A}
\PYG{n}{A}\PYG{o}{.}\PYG{n}{shape}
\end{sphinxVerbatim}

\begin{sphinxVerbatim}[commandchars=\\\{\}]
(4, 4)
\end{sphinxVerbatim}

\begin{sphinxVerbatim}[commandchars=\\\{\}]
\PYG{c+c1}{\PYGZsh{} dimensão de b}
\PYG{n}{b}\PYG{o}{.}\PYG{n}{shape}
\end{sphinxVerbatim}

\begin{sphinxVerbatim}[commandchars=\\\{\}]
(4,)
\end{sphinxVerbatim}

\sphinxAtStartPar
A solução do sistema pode ser obtida através do método \sphinxcode{\sphinxupquote{linalg.solve}}.

\begin{sphinxVerbatim}[commandchars=\\\{\}]
\PYG{n}{x} \PYG{o}{=} \PYG{n}{linalg}\PYG{o}{.}\PYG{n}{solve}\PYG{p}{(}\PYG{n}{A}\PYG{p}{,}\PYG{n}{b}\PYG{p}{)}
\PYG{n+nb}{print}\PYG{p}{(}\PYG{n}{x}\PYG{p}{)}
\end{sphinxVerbatim}

\begin{sphinxVerbatim}[commandchars=\\\{\}]
[ 2.   4.  \PYGZhy{}3.   0.5]
\end{sphinxVerbatim}


\section{Inversão de matrizes}
\label{\detokenize{aula-09-eliminacao-gauss:inversao-de-matrizes}}
\sphinxAtStartPar
Matematicamente, a solução do sistema anterior é dada por \({\bf x} = {\bf A}^{-1}{\bf b}\). Podemos até invocar a matriz inversa aqui com \sphinxcode{\sphinxupquote{linalg.inv(A).dot(b)}} e a solução é a mesma que no caso anterior.

\begin{sphinxVerbatim}[commandchars=\\\{\}]
\PYG{n}{x2} \PYG{o}{=} \PYG{n}{linalg}\PYG{o}{.}\PYG{n}{inv}\PYG{p}{(}\PYG{n}{A}\PYG{p}{)}\PYG{o}{.}\PYG{n}{dot}\PYG{p}{(}\PYG{n}{b}\PYG{p}{)}
\PYG{n+nb}{print}\PYG{p}{(}\PYG{n}{x2}\PYG{p}{)}
\end{sphinxVerbatim}

\begin{sphinxVerbatim}[commandchars=\\\{\}]
[ 2.   4.  \PYGZhy{}3.   0.5]
\end{sphinxVerbatim}

\sphinxAtStartPar
Por outro lado, este método é ineficiente. Computacionalmente, a inversão de matrizes \sphinxstylestrong{não} é aconselhável.


\section{Verificação da solução}
\label{\detokenize{aula-09-eliminacao-gauss:verificacao-da-solucao}}
\sphinxAtStartPar
Podemos usar o fato de que \({\bf A}{\bf A}^{-1}{\bf b} - {\bf b} = {\bf 0}\).

\begin{sphinxVerbatim}[commandchars=\\\{\}]
\PYG{n}{x3} \PYG{o}{=} \PYG{n}{A}\PYG{o}{.}\PYG{n}{dot}\PYG{p}{(}\PYG{n}{linalg}\PYG{o}{.}\PYG{n}{inv}\PYG{p}{(}\PYG{n}{A}\PYG{p}{)}\PYG{o}{.}\PYG{n}{dot}\PYG{p}{(}\PYG{n}{b}\PYG{p}{)}\PYG{p}{)} \PYG{o}{\PYGZhy{}} \PYG{n}{b}
\PYG{n+nb}{print}\PYG{p}{(}\PYG{n}{x3}\PYG{p}{)}
\end{sphinxVerbatim}

\begin{sphinxVerbatim}[commandchars=\\\{\}]
[ 5.32907052e\PYGZhy{}15 \PYGZhy{}3.19744231e\PYGZhy{}14 \PYGZhy{}4.44089210e\PYGZhy{}14 \PYGZhy{}1.20792265e\PYGZhy{}13]
\end{sphinxVerbatim}

\sphinxAtStartPar
Note que o vetor é próximo do vetor nulo, mas não identicamente nulo.

\sphinxAtStartPar
Podemos também computar a \sphinxstylestrong{norma do resíduo (erro)}: \(||{\bf r}|| = ||{\bf b} - {\bf A}{\bf x}|| =  \langle {\bf b} - {\bf A}{\bf x}, {\bf b} - {\bf A}{\bf x} \rangle^{1/2}\)

\begin{sphinxVerbatim}[commandchars=\\\{\}]
\PYG{n}{r} \PYG{o}{=} \PYG{n}{b} \PYG{o}{\PYGZhy{}} \PYG{n}{A}\PYG{o}{.}\PYG{n}{dot}\PYG{p}{(}\PYG{n}{x}\PYG{p}{)} 
\PYG{n}{np}\PYG{o}{.}\PYG{n}{sqrt}\PYG{p}{(}\PYG{n}{r}\PYG{o}{.}\PYG{n}{dot}\PYG{p}{(}\PYG{n}{r}\PYG{p}{)}\PYG{p}{)}
\end{sphinxVerbatim}

\begin{sphinxVerbatim}[commandchars=\\\{\}]
1.464821375527116e\PYGZhy{}14
\end{sphinxVerbatim}

\sphinxAtStartPar
Como a norma do resíduo é próxima de zero, a solução do sistema linear é assumida como correta.


\chapter{Eliminação Gaussiana}
\label{\detokenize{aula-09-eliminacao-gauss:eliminacao-gaussiana}}
\sphinxAtStartPar
A Eliminação Gaussiana (EG) é um algoritmo utilizado para resolver sistemas de equações lineares ao reduzir a matriz plena associada do sistema a uma matriz triangular. Este processo também é chamado de \sphinxstyleemphasis{escalonamento}. Abaixo, usaremos uma matriz genérica 3x3 para exemplificação.


\section{Passos}
\label{\detokenize{aula-09-eliminacao-gauss:passos}}\begin{itemize}
\item {} 
\sphinxAtStartPar
Escrever o sistema linear na forma de \sphinxstyleemphasis{matriz estendida} usando os coeficientes das variáveis como elementos da matriz e o vetor independente como sendo a última coluna;

\end{itemize}
\begin{equation*}
\begin{split}\begin{array}{c}a_{11}x_1+a_{12}x_2+a_{13}x_3 = b_1 \quad(L_1)\\
a_{21}x_1+a_{22}x_2+a_{23}x_3 = b_2 \quad(L_2)\\
a_{31}x_1+a_{32}x_2+a_{33}x_3 = b_3 \quad(L_3)\\
\downarrow\\
\left[
\begin{array}{ccc|c}
	a_{11}&a_{12}&a_{13}&b_{1}\\
	a_{21}&a_{22}&a_{23}&b_{2}\\
	a_{31}&a_{32}&a_{33}&b_{3}
\end{array}
\right]
\end{array}\end{split}
\end{equation*}\begin{itemize}
\item {} 
\sphinxAtStartPar
Realizar operações elementares de combinação linear e permutação entre linhas;
\begin{itemize}
\item {} 
\sphinxAtStartPar
Multiplicação por escalar:\$\$
\textbackslash{}begin\{array\}\{c\}
L\_2 \textbackslash{}leftarrow L\_2 .w\textbackslash{} \textbackslash{}Rightarrow\textbackslash{}

\end{itemize}

\end{itemize}
\label{equation:aula-09-eliminacao-gauss:38e1a23a-0d2e-4caa-bad5-c0868ba5d450}\begin{bmatrix}
a_{11}&a_{12}\\
a_{21}&a_{22}\\
\end{bmatrix}\label{equation:aula-09-eliminacao-gauss:c882cf6d-3ae7-470d-90a1-5ff4ff69ea44}\begin{bmatrix}
a_{11}&a_{12}\\
w.a_{21}&w.a_{22}\\
\end{bmatrix}
\sphinxAtStartPar
\textbackslash{}end\{array\}
\$\$

\begin{sphinxVerbatim}[commandchars=\\\{\}]
\PYGZhy{} Combinação linear:    
\end{sphinxVerbatim}
\begin{equation*}
\begin{split}
\begin{array}{c}
L_2 \leftarrow L_2 - L_1.w\ \Rightarrow\ 
\begin{bmatrix}
a_{11}&a_{12}\\
a_{21}&a_{22}\\
\end{bmatrix} \rightarrow
\begin{bmatrix}
a_{11}&a_{12}\\
a_{21}-a_{11}.w&a_{22}-a_{12}.w\\
\end{bmatrix}
\end{array}
\end{split}
\end{equation*}
\begin{sphinxVerbatim}[commandchars=\\\{\}]
\PYGZhy{} Permutação:
\end{sphinxVerbatim}
\begin{equation*}
\begin{split}
\begin{array}{c}
L_2 \leftarrow L_1\ e\ L_1\leftarrow L_2\ \Rightarrow\ 
\begin{bmatrix}
a_{11}&a_{12}\\
a_{21}&a_{22}\\
\end{bmatrix} \rightarrow
\begin{bmatrix}
a_{21}&a_{22}\\
a_{11}&a_{12}\\
\end{bmatrix}
\end{array}
\end{split}
\end{equation*}\begin{itemize}
\item {} 
\sphinxAtStartPar
Anular todos os elementos na porção triangular inferior da matriz original, isto é, todas as entradas exatamente abaixo das entradas dispostas na diagonal principal;

\end{itemize}
\begin{equation*}
\begin{split}\left[
\begin{array}{ccc|c}
	a_{11}&a_{12}&a_{13}&b_{1}\\
	a_{21}&a_{22}&a_{23}&b_{2}\\
	a_{31}&a_{32}&a_{33}&b_{3}
\end{array}
\right]\ \rightarrow\ 
\left[
\begin{array}{ccc|c}
	a_{11}^{(k)}&a_{12}^{(k)}&a_{13}^{(k)}&b_{1}^{(k)}\\
	0&a_{22}^{(k)}&a_{23}^{(k)}&b_{2}^{(k)}\\
	0&0&a_{33}^{(k)}&b_{3}^{(k)}
\end{array}
\right]\end{split}
\end{equation*}\begin{itemize}
\item {} 
\sphinxAtStartPar
A partir da forma triangular, realizar a substituição regressiva.

\end{itemize}
\begin{equation*}
\begin{split}
\left[
\begin{array}{ccc|c}
	a_{11}^{(k)}&a_{12}^{(k)}&a_{13}^{(k)}&b_{1}^{(k)}\\
	0&a_{22}^{(k)}&a_{23}^{(k)}&b_{2}^{(k)}\\
	0&0&a_{33}^{(k)}&b_{3}^{(k)}
\end{array}
\right]\ \rightarrow\ 
\left\{
\begin{array}{c}
x_3 = \frac{b_3^{(k)}}{a_{33}^{(k)}},\ a_{33}^{(k)} \neq 0\\\\
x_2 = \frac{b_2^{(k)}-a_{23}^{(k)}.x_3}{a_{22}^{(k)}},\ a_{22}^{(k)} \neq 0\\\\
x_1 = \frac{b_1^{(k)}-a_{12}^{(k)}.x_2-a_{13}^{(k)}.x_3}{a_{11}^{(k)}},\ a_{11}^{(k)} \neq 0\\\\
\end{array}
\right.
\end{split}
\end{equation*}
\sphinxAtStartPar
Vejamos um exemplo numérico de como funciona a Eliminação Gaussiana.

\begin{sphinxVerbatim}[commandchars=\\\{\}]
\PYG{c+c1}{\PYGZsh{} matriz}
\PYG{n}{M} \PYG{o}{=} \PYG{n}{np}\PYG{o}{.}\PYG{n}{array}\PYG{p}{(}\PYG{p}{[}\PYG{p}{[}\PYG{l+m+mf}{1.0}\PYG{p}{,}\PYG{l+m+mf}{1.5}\PYG{p}{,}\PYG{o}{\PYGZhy{}}\PYG{l+m+mf}{2.0}\PYG{p}{]}\PYG{p}{,}\PYG{p}{[}\PYG{l+m+mf}{2.0}\PYG{p}{,}\PYG{l+m+mf}{1.0}\PYG{p}{,}\PYG{o}{\PYGZhy{}}\PYG{l+m+mf}{1.0}\PYG{p}{]}\PYG{p}{,}\PYG{p}{[}\PYG{l+m+mf}{3.0}\PYG{p}{,}\PYG{o}{\PYGZhy{}}\PYG{l+m+mf}{1.0}\PYG{p}{,}\PYG{l+m+mf}{2.0}\PYG{p}{]}\PYG{p}{]}\PYG{p}{)}
\PYG{n+nb}{print}\PYG{p}{(}\PYG{n}{M}\PYG{p}{)}
\end{sphinxVerbatim}

\begin{sphinxVerbatim}[commandchars=\\\{\}]
[[ 1.   1.5 \PYGZhy{}2. ]
 [ 2.   1.  \PYGZhy{}1. ]
 [ 3.  \PYGZhy{}1.   2. ]]
\end{sphinxVerbatim}

\begin{sphinxVerbatim}[commandchars=\\\{\}]
\PYG{c+c1}{\PYGZsh{} zeramento da segunda linha }
\PYG{n}{m1} \PYG{o}{=} \PYG{n}{M}\PYG{p}{[}\PYG{l+m+mi}{1}\PYG{p}{,}\PYG{l+m+mi}{0}\PYG{p}{]}\PYG{o}{/}\PYG{n}{M}\PYG{p}{[}\PYG{l+m+mi}{0}\PYG{p}{,}\PYG{l+m+mi}{0}\PYG{p}{]}
\PYG{n}{M}\PYG{p}{[}\PYG{l+m+mi}{1}\PYG{p}{,}\PYG{p}{:}\PYG{p}{]} \PYG{o}{+}\PYG{o}{=} \PYG{o}{\PYGZhy{}} \PYG{n}{m1}\PYG{o}{*}\PYG{n}{M}\PYG{p}{[}\PYG{l+m+mi}{0}\PYG{p}{,}\PYG{p}{:}\PYG{p}{]}
\PYG{n+nb}{print}\PYG{p}{(}\PYG{n}{M}\PYG{p}{)}
\end{sphinxVerbatim}

\begin{sphinxVerbatim}[commandchars=\\\{\}]
[[ 1.   1.5 \PYGZhy{}2. ]
 [ 0.  \PYGZhy{}2.   3. ]
 [ 3.  \PYGZhy{}1.   2. ]]
\end{sphinxVerbatim}

\begin{sphinxVerbatim}[commandchars=\\\{\}]
\PYG{c+c1}{\PYGZsh{} zeramento da terceira linha}
\PYG{n}{m2} \PYG{o}{=} \PYG{n}{M}\PYG{p}{[}\PYG{l+m+mi}{2}\PYG{p}{,}\PYG{l+m+mi}{0}\PYG{p}{]}\PYG{o}{/}\PYG{n}{M}\PYG{p}{[}\PYG{l+m+mi}{0}\PYG{p}{,}\PYG{l+m+mi}{0}\PYG{p}{]}
\PYG{n}{M}\PYG{p}{[}\PYG{l+m+mi}{2}\PYG{p}{,}\PYG{p}{:}\PYG{p}{]} \PYG{o}{+}\PYG{o}{=} \PYG{o}{\PYGZhy{}} \PYG{n}{m2}\PYG{o}{*}\PYG{n}{M}\PYG{p}{[}\PYG{l+m+mi}{0}\PYG{p}{,}\PYG{p}{:}\PYG{p}{]}
\PYG{n+nb}{print}\PYG{p}{(}\PYG{n}{M}\PYG{p}{)}
\end{sphinxVerbatim}

\begin{sphinxVerbatim}[commandchars=\\\{\}]
[[ 1.   1.5 \PYGZhy{}2. ]
 [ 0.  \PYGZhy{}2.   3. ]
 [ 0.  \PYGZhy{}5.5  8. ]]
\end{sphinxVerbatim}

\begin{sphinxVerbatim}[commandchars=\\\{\}]
\PYG{c+c1}{\PYGZsh{} eliminação}
\PYG{n}{M} \PYG{o}{=} \PYG{n}{np}\PYG{o}{.}\PYG{n}{array}\PYG{p}{(}\PYG{p}{[}\PYG{p}{[}\PYG{l+m+mf}{1.0}\PYG{p}{,}\PYG{l+m+mf}{1.5}\PYG{p}{,}\PYG{o}{\PYGZhy{}}\PYG{l+m+mf}{2.0}\PYG{p}{]}\PYG{p}{,}\PYG{p}{[}\PYG{l+m+mf}{2.0}\PYG{p}{,}\PYG{l+m+mf}{1.0}\PYG{p}{,}\PYG{o}{\PYGZhy{}}\PYG{l+m+mf}{1.0}\PYG{p}{]}\PYG{p}{,}\PYG{p}{[}\PYG{l+m+mf}{3.0}\PYG{p}{,}\PYG{o}{\PYGZhy{}}\PYG{l+m+mf}{1.0}\PYG{p}{,}\PYG{l+m+mf}{2.0}\PYG{p}{]}\PYG{p}{]}\PYG{p}{)}
\PYG{n}{b} \PYG{o}{=} \PYG{n}{np}\PYG{o}{.}\PYG{n}{array}\PYG{p}{(}\PYG{p}{[}\PYG{o}{\PYGZhy{}}\PYG{l+m+mi}{2}\PYG{p}{,}\PYG{l+m+mi}{3}\PYG{p}{,}\PYG{l+m+mi}{1}\PYG{p}{]}\PYG{p}{)}

\PYG{n}{m}\PYG{p}{,}\PYG{n}{n} \PYG{o}{=} \PYG{n}{M}\PYG{o}{.}\PYG{n}{shape}
\PYG{k}{for} \PYG{n}{i} \PYG{o+ow}{in} \PYG{n+nb}{range}\PYG{p}{(}\PYG{n}{m}\PYG{p}{)}\PYG{p}{:}
    \PYG{k}{for} \PYG{n}{j} \PYG{o+ow}{in} \PYG{n+nb}{range}\PYG{p}{(}\PYG{n}{i}\PYG{o}{+}\PYG{l+m+mi}{1}\PYG{p}{,}\PYG{n}{n}\PYG{p}{)}\PYG{p}{:}
        \PYG{n}{pivo} \PYG{o}{=} \PYG{n}{M}\PYG{p}{[}\PYG{n}{j}\PYG{p}{,}\PYG{n}{i}\PYG{p}{]}\PYG{o}{/}\PYG{n}{M}\PYG{p}{[}\PYG{n}{i}\PYG{p}{,}\PYG{n}{i}\PYG{p}{]}                        
        \PYG{k}{for} \PYG{n}{k} \PYG{o+ow}{in} \PYG{n+nb}{range}\PYG{p}{(}\PYG{n}{m}\PYG{p}{)}\PYG{p}{:}
            \PYG{n}{M}\PYG{p}{[}\PYG{n}{j}\PYG{p}{,}\PYG{n}{k}\PYG{p}{]} \PYG{o}{+}\PYG{o}{=} \PYG{o}{\PYGZhy{}}\PYG{n}{pivo}\PYG{o}{*}\PYG{n}{M}\PYG{p}{[}\PYG{n}{i}\PYG{p}{,}\PYG{n}{k}\PYG{p}{]}

\PYG{n+nb}{print}\PYG{p}{(}\PYG{n}{M}\PYG{p}{)}            
\end{sphinxVerbatim}

\begin{sphinxVerbatim}[commandchars=\\\{\}]
[[ 1.    1.5  \PYGZhy{}2.  ]
 [ 0.   \PYGZhy{}2.    3.  ]
 [ 0.    0.   \PYGZhy{}0.25]]
\end{sphinxVerbatim}

\begin{sphinxVerbatim}[commandchars=\\\{\}]
\PYG{c+c1}{\PYGZsh{} função simples para eliminação}
\PYG{k}{def} \PYG{n+nf}{eliminacao}\PYG{p}{(}\PYG{n}{M}\PYG{p}{)}\PYG{p}{:}
    \PYG{n}{m}\PYG{p}{,}\PYG{n}{n} \PYG{o}{=} \PYG{n}{M}\PYG{o}{.}\PYG{n}{shape}
    \PYG{k}{for} \PYG{n}{i} \PYG{o+ow}{in} \PYG{n+nb}{range}\PYG{p}{(}\PYG{n}{m}\PYG{p}{)}\PYG{p}{:}
        \PYG{k}{for} \PYG{n}{j} \PYG{o+ow}{in} \PYG{n+nb}{range}\PYG{p}{(}\PYG{n}{i}\PYG{o}{+}\PYG{l+m+mi}{1}\PYG{p}{,}\PYG{n}{n}\PYG{p}{)}\PYG{p}{:}
            \PYG{n}{pivo} \PYG{o}{=} \PYG{n}{M}\PYG{p}{[}\PYG{n}{j}\PYG{p}{,}\PYG{n}{i}\PYG{p}{]}\PYG{o}{/}\PYG{n}{M}\PYG{p}{[}\PYG{n}{i}\PYG{p}{,}\PYG{n}{i}\PYG{p}{]}                        
            \PYG{k}{for} \PYG{n}{k} \PYG{o+ow}{in} \PYG{n+nb}{range}\PYG{p}{(}\PYG{n}{m}\PYG{p}{)}\PYG{p}{:}
                \PYG{n}{M}\PYG{p}{[}\PYG{n}{j}\PYG{p}{,}\PYG{n}{k}\PYG{p}{]} \PYG{o}{+}\PYG{o}{=} \PYG{o}{\PYGZhy{}}\PYG{n}{pivo}\PYG{o}{*}\PYG{n}{M}\PYG{p}{[}\PYG{n}{i}\PYG{p}{,}\PYG{n}{k}\PYG{p}{]}
    \PYG{k}{return} \PYG{n}{M}
\end{sphinxVerbatim}

\begin{sphinxVerbatim}[commandchars=\\\{\}]
\PYG{c+c1}{\PYGZsh{} matriz randômica 5x5}
\PYG{n}{M2} \PYG{o}{=} \PYG{n}{np}\PYG{o}{.}\PYG{n}{random}\PYG{o}{.}\PYG{n}{rand}\PYG{p}{(}\PYG{l+m+mi}{5}\PYG{p}{,}\PYG{l+m+mi}{5}\PYG{p}{)}
\PYG{n+nb}{print}\PYG{p}{(}\PYG{n}{eliminacao}\PYG{p}{(}\PYG{n}{M2}\PYG{p}{)}\PYG{p}{)}
\end{sphinxVerbatim}

\begin{sphinxVerbatim}[commandchars=\\\{\}]
[[ 4.42389663e\PYGZhy{}01  6.77549133e\PYGZhy{}01  6.62310314e\PYGZhy{}01  8.92758362e\PYGZhy{}01
   8.67074978e\PYGZhy{}01]
 [ 0.00000000e+00 \PYGZhy{}8.94767973e\PYGZhy{}01 \PYGZhy{}1.17650434e+00 \PYGZhy{}1.23381565e+00
  \PYGZhy{}1.54014118e+00]
 [\PYGZhy{}1.11022302e\PYGZhy{}16  0.00000000e+00  1.14215572e\PYGZhy{}01 \PYGZhy{}6.53327797e\PYGZhy{}01
  \PYGZhy{}5.87564401e\PYGZhy{}02]
 [\PYGZhy{}2.58946634e\PYGZhy{}16  0.00000000e+00  0.00000000e+00 \PYGZhy{}1.60708761e+00
  \PYGZhy{}2.20279294e\PYGZhy{}01]
 [ 3.32024317e\PYGZhy{}17  0.00000000e+00  0.00000000e+00  0.00000000e+00
  \PYGZhy{}9.80998229e\PYGZhy{}01]]
\end{sphinxVerbatim}


\section{Condicionamento}
\label{\detokenize{aula-09-eliminacao-gauss:condicionamento}}
\sphinxAtStartPar
Vamos ver como pequenas “perturbações” em matrizes podem provocar mudanças drásticas
nas soluções de sistemas. Isto ocorre quando temos um problema \sphinxstyleemphasis{mal condicionado}.

\begin{sphinxVerbatim}[commandchars=\\\{\}]
\PYG{n}{A1} \PYG{o}{=} \PYG{n}{np}\PYG{o}{.}\PYG{n}{array}\PYG{p}{(}\PYG{p}{[}\PYG{p}{[}\PYG{l+m+mi}{1}\PYG{p}{,}\PYG{l+m+mi}{2}\PYG{p}{]}\PYG{p}{,}\PYG{p}{[}\PYG{l+m+mf}{1.1}\PYG{p}{,}\PYG{l+m+mi}{2}\PYG{p}{]}\PYG{p}{]}\PYG{p}{)}
\PYG{n}{b1} \PYG{o}{=} \PYG{n}{np}\PYG{o}{.}\PYG{n}{array}\PYG{p}{(}\PYG{p}{[}\PYG{l+m+mi}{10}\PYG{p}{,}\PYG{l+m+mf}{10.4}\PYG{p}{]}\PYG{p}{)}
\PYG{n+nb}{print}\PYG{p}{(}\PYG{l+s+s1}{\PYGZsq{}}\PYG{l+s+s1}{matriz}\PYG{l+s+s1}{\PYGZsq{}}\PYG{p}{)}
\PYG{n+nb}{print}\PYG{p}{(}\PYG{n}{A1}\PYG{p}{)}
\PYG{n+nb}{print}\PYG{p}{(}\PYG{l+s+s1}{\PYGZsq{}}\PYG{l+s+s1}{vetor}\PYG{l+s+s1}{\PYGZsq{}}\PYG{p}{)}
\PYG{n+nb}{print}\PYG{p}{(}\PYG{n}{b1}\PYG{p}{)}
\end{sphinxVerbatim}

\begin{sphinxVerbatim}[commandchars=\\\{\}]
matriz
[[1.  2. ]
 [1.1 2. ]]
vetor
[10.  10.4]
\end{sphinxVerbatim}

\begin{sphinxVerbatim}[commandchars=\\\{\}]
\PYG{c+c1}{\PYGZsh{} solução do sistema A1x1 = b1}
\PYG{n}{x1} \PYG{o}{=} \PYG{n}{linalg}\PYG{o}{.}\PYG{n}{solve}\PYG{p}{(}\PYG{n}{A1}\PYG{p}{,}\PYG{n}{b1}\PYG{p}{)}
\PYG{n+nb}{print}\PYG{p}{(}\PYG{n}{x1}\PYG{p}{)}
\end{sphinxVerbatim}

\begin{sphinxVerbatim}[commandchars=\\\{\}]
[4. 3.]
\end{sphinxVerbatim}

\begin{sphinxVerbatim}[commandchars=\\\{\}]
\PYG{n}{d} \PYG{o}{=} \PYG{l+m+mf}{0.045}
\PYG{n}{A2} \PYG{o}{=} \PYG{n}{np}\PYG{o}{.}\PYG{n}{array}\PYG{p}{(}\PYG{p}{[}\PYG{p}{[}\PYG{l+m+mi}{1}\PYG{p}{,}\PYG{l+m+mi}{2}\PYG{p}{]}\PYG{p}{,}\PYG{p}{[}\PYG{l+m+mf}{1.1}\PYG{o}{\PYGZhy{}}\PYG{n}{d}\PYG{p}{,}\PYG{l+m+mi}{2}\PYG{p}{]}\PYG{p}{]}\PYG{p}{)}
\PYG{n}{b2} \PYG{o}{=} \PYG{n}{np}\PYG{o}{.}\PYG{n}{array}\PYG{p}{(}\PYG{p}{[}\PYG{l+m+mi}{10}\PYG{p}{,}\PYG{l+m+mf}{10.4}\PYG{p}{]}\PYG{p}{)}
\PYG{n+nb}{print}\PYG{p}{(}\PYG{l+s+s1}{\PYGZsq{}}\PYG{l+s+s1}{matriz}\PYG{l+s+s1}{\PYGZsq{}}\PYG{p}{)}
\PYG{n+nb}{print}\PYG{p}{(}\PYG{n}{A2}\PYG{p}{)}
\PYG{n+nb}{print}\PYG{p}{(}\PYG{l+s+s1}{\PYGZsq{}}\PYG{l+s+s1}{vetor}\PYG{l+s+s1}{\PYGZsq{}}\PYG{p}{)}
\PYG{n+nb}{print}\PYG{p}{(}\PYG{n}{b2}\PYG{p}{)}
\end{sphinxVerbatim}

\begin{sphinxVerbatim}[commandchars=\\\{\}]
matriz
[[1.    2.   ]
 [1.055 2.   ]]
vetor
[10.  10.4]
\end{sphinxVerbatim}

\begin{sphinxVerbatim}[commandchars=\\\{\}]
\PYG{c+c1}{\PYGZsh{} solução do sistema perturbado A2x1 = b2}
\PYG{n}{x2} \PYG{o}{=} \PYG{n}{linalg}\PYG{o}{.}\PYG{n}{solve}\PYG{p}{(}\PYG{n}{A2}\PYG{p}{,}\PYG{n}{b2}\PYG{p}{)}
\PYG{n+nb}{print}\PYG{p}{(}\PYG{n}{x2}\PYG{p}{)}
\end{sphinxVerbatim}

\begin{sphinxVerbatim}[commandchars=\\\{\}]
[7.27272727 1.36363636]
\end{sphinxVerbatim}

\sphinxAtStartPar
A solução muda drasticamente aqui! Isto se deve à quase dependência linear em que a matriz se encontra. Ou seja, \(\det({\bf A}_2) \approx 0\).

\begin{sphinxVerbatim}[commandchars=\\\{\}]
\PYG{n+nb}{print}\PYG{p}{(}\PYG{n}{linalg}\PYG{o}{.}\PYG{n}{det}\PYG{p}{(}\PYG{n}{A1}\PYG{p}{)}\PYG{p}{,}\PYG{n}{linalg}\PYG{o}{.}\PYG{n}{det}\PYG{p}{(}\PYG{n}{A2}\PYG{p}{)}\PYG{p}{)}
\end{sphinxVerbatim}

\begin{sphinxVerbatim}[commandchars=\\\{\}]
\PYGZhy{}0.20000000000000018 \PYGZhy{}0.11000000000000032
\end{sphinxVerbatim}

\begin{sphinxVerbatim}[commandchars=\\\{\}]
\PYG{n}{linalg}\PYG{o}{.}\PYG{n}{norm}\PYG{p}{(}\PYG{n}{A}\PYG{p}{)}\PYG{o}{*}\PYG{n}{linalg}\PYG{o}{.}\PYG{n}{norm}\PYG{p}{(}\PYG{n}{linalg}\PYG{o}{.}\PYG{n}{inv}\PYG{p}{(}\PYG{n}{A}\PYG{p}{)}\PYG{p}{)}
\end{sphinxVerbatim}

\begin{sphinxVerbatim}[commandchars=\\\{\}]
169.28388045827452
\end{sphinxVerbatim}


\section{Método de Gauss\sphinxhyphen{}Jordan}
\label{\detokenize{aula-09-eliminacao-gauss:metodo-de-gauss-jordan}}
\sphinxAtStartPar
O método de Gauss\sphinxhyphen{}Jordan é uma variação da eliminação de Gauss, em que não apenas as entradas da porção inferior da matriz plena do sistema são anuladas, mas também as entradas da porção superior, resultando em uma matriz diagonal.

\sphinxAtStartPar
Além disso, todas as linhas são normalizadas através da sua divisão pelo respectivo elemento pivô. Por exemplo, \(a_{11}\) é o elemento pivô da primeira equação, \(a_{22}\) da segunda, e assim por diante). A partir daí, a obtenção dos valores das variáveis é imediata.

\sphinxAtStartPar
O método é melhor ilustrado no seguinte exemplo.

\begin{sphinxVerbatim}[commandchars=\\\{\}]
\PYG{c+c1}{\PYGZsh{} Matriz aumentada}
\PYG{n}{AB} \PYG{o}{=} \PYG{n}{np}\PYG{o}{.}\PYG{n}{array}\PYG{p}{(}\PYG{p}{[}\PYG{p}{[}\PYG{l+m+mf}{3.} \PYG{p}{,} \PYG{o}{\PYGZhy{}}\PYG{l+m+mf}{0.1}\PYG{p}{,} \PYG{o}{\PYGZhy{}}\PYG{l+m+mf}{0.2}\PYG{p}{,} \PYG{l+m+mf}{7.85}\PYG{p}{]}\PYG{p}{,} \PYG{p}{[}\PYG{l+m+mf}{0.1}\PYG{p}{,} \PYG{l+m+mf}{7.}\PYG{p}{,} \PYG{o}{\PYGZhy{}}\PYG{l+m+mf}{0.3}\PYG{p}{,} \PYG{o}{\PYGZhy{}}\PYG{l+m+mf}{19.3}\PYG{p}{]}\PYG{p}{,} \PYG{p}{[}\PYG{l+m+mf}{0.3}\PYG{p}{,} \PYG{o}{\PYGZhy{}}\PYG{l+m+mf}{0.2}\PYG{p}{,} \PYG{l+m+mf}{10.}\PYG{p}{,} \PYG{l+m+mf}{71.4}\PYG{p}{]}\PYG{p}{]}\PYG{p}{)}
\PYG{n+nb}{print}\PYG{p}{(}\PYG{n}{AB}\PYG{p}{)}
\end{sphinxVerbatim}

\begin{sphinxVerbatim}[commandchars=\\\{\}]
[[  3.    \PYGZhy{}0.1   \PYGZhy{}0.2    7.85]
 [  0.1    7.    \PYGZhy{}0.3  \PYGZhy{}19.3 ]
 [  0.3   \PYGZhy{}0.2   10.    71.4 ]]
\end{sphinxVerbatim}

\begin{sphinxVerbatim}[commandchars=\\\{\}]
\PYG{c+c1}{\PYGZsh{} Normalização da 1a. linha}
\PYG{n}{AB}\PYG{p}{[}\PYG{l+m+mi}{0}\PYG{p}{,}\PYG{p}{:}\PYG{p}{]} \PYG{o}{/}\PYG{o}{=} \PYG{n}{AB}\PYG{p}{[}\PYG{l+m+mi}{0}\PYG{p}{,}\PYG{l+m+mi}{0}\PYG{p}{]} \PYG{c+c1}{\PYGZsh{} L1 \PYGZlt{}\PYGZhy{} L1/a11}
\PYG{n+nb}{print}\PYG{p}{(}\PYG{n}{AB}\PYG{p}{)}
\end{sphinxVerbatim}

\begin{sphinxVerbatim}[commandchars=\\\{\}]
[[ 1.00000000e+00 \PYGZhy{}3.33333333e\PYGZhy{}02 \PYGZhy{}6.66666667e\PYGZhy{}02  2.61666667e+00]
 [ 1.00000000e\PYGZhy{}01  7.00000000e+00 \PYGZhy{}3.00000000e\PYGZhy{}01 \PYGZhy{}1.93000000e+01]
 [ 3.00000000e\PYGZhy{}01 \PYGZhy{}2.00000000e\PYGZhy{}01  1.00000000e+01  7.14000000e+01]]
\end{sphinxVerbatim}

\begin{sphinxVerbatim}[commandchars=\\\{\}]
\PYG{c+c1}{\PYGZsh{} Eliminação de x1 da 2a. e 3a. linhas}
\PYG{n}{m1} \PYG{o}{=} \PYG{n}{AB}\PYG{p}{[}\PYG{l+m+mi}{1}\PYG{p}{,}\PYG{l+m+mi}{0}\PYG{p}{]}
\PYG{n}{AB}\PYG{p}{[}\PYG{l+m+mi}{1}\PYG{p}{,}\PYG{p}{:}\PYG{p}{]} \PYG{o}{\PYGZhy{}}\PYG{o}{=} \PYG{n}{m1}\PYG{o}{*}\PYG{n}{AB}\PYG{p}{[}\PYG{l+m+mi}{0}\PYG{p}{,}\PYG{p}{:}\PYG{p}{]} \PYG{c+c1}{\PYGZsh{} L2 \PYGZlt{}\PYGZhy{} L2 \PYGZhy{} m1*L1}
\PYG{n}{m2} \PYG{o}{=} \PYG{n}{AB}\PYG{p}{[}\PYG{l+m+mi}{2}\PYG{p}{,}\PYG{l+m+mi}{0}\PYG{p}{]}
\PYG{n}{AB}\PYG{p}{[}\PYG{l+m+mi}{2}\PYG{p}{,}\PYG{p}{:}\PYG{p}{]} \PYG{o}{\PYGZhy{}}\PYG{o}{=} \PYG{n}{m2}\PYG{o}{*}\PYG{n}{AB}\PYG{p}{[}\PYG{l+m+mi}{0}\PYG{p}{,}\PYG{p}{:}\PYG{p}{]} \PYG{c+c1}{\PYGZsh{} L3 \PYGZlt{}\PYGZhy{} L3 \PYGZhy{} m2*L1}
\PYG{n+nb}{print}\PYG{p}{(}\PYG{n}{AB}\PYG{p}{)}
\end{sphinxVerbatim}

\begin{sphinxVerbatim}[commandchars=\\\{\}]
[[ 1.00000000e+00 \PYGZhy{}3.33333333e\PYGZhy{}02 \PYGZhy{}6.66666667e\PYGZhy{}02  2.61666667e+00]
 [ 0.00000000e+00  7.00333333e+00 \PYGZhy{}2.93333333e\PYGZhy{}01 \PYGZhy{}1.95616667e+01]
 [ 0.00000000e+00 \PYGZhy{}1.90000000e\PYGZhy{}01  1.00200000e+01  7.06150000e+01]]
\end{sphinxVerbatim}

\begin{sphinxVerbatim}[commandchars=\\\{\}]
\PYG{c+c1}{\PYGZsh{} Normalização da 2a. linha}
\PYG{n}{AB}\PYG{p}{[}\PYG{l+m+mi}{1}\PYG{p}{,}\PYG{p}{:}\PYG{p}{]} \PYG{o}{/}\PYG{o}{=} \PYG{n}{AB}\PYG{p}{[}\PYG{l+m+mi}{1}\PYG{p}{,}\PYG{l+m+mi}{1}\PYG{p}{]} \PYG{c+c1}{\PYGZsh{} L2 \PYGZlt{}\PYGZhy{} L2/a22}
\PYG{n+nb}{print}\PYG{p}{(}\PYG{n}{AB}\PYG{p}{)}
\end{sphinxVerbatim}

\begin{sphinxVerbatim}[commandchars=\\\{\}]
[[ 1.00000000e+00 \PYGZhy{}3.33333333e\PYGZhy{}02 \PYGZhy{}6.66666667e\PYGZhy{}02  2.61666667e+00]
 [ 0.00000000e+00  1.00000000e+00 \PYGZhy{}4.18848168e\PYGZhy{}02 \PYGZhy{}2.79319372e+00]
 [ 0.00000000e+00 \PYGZhy{}1.90000000e\PYGZhy{}01  1.00200000e+01  7.06150000e+01]]
\end{sphinxVerbatim}

\begin{sphinxVerbatim}[commandchars=\\\{\}]
\PYG{c+c1}{\PYGZsh{} Eliminação de x2 da 1a. e 3a. linhas}
\PYG{n}{m3} \PYG{o}{=} \PYG{n}{AB}\PYG{p}{[}\PYG{l+m+mi}{0}\PYG{p}{,}\PYG{l+m+mi}{1}\PYG{p}{]}
\PYG{n}{AB}\PYG{p}{[}\PYG{l+m+mi}{0}\PYG{p}{,}\PYG{p}{:}\PYG{p}{]} \PYG{o}{\PYGZhy{}}\PYG{o}{=} \PYG{n}{m3}\PYG{o}{*}\PYG{n}{AB}\PYG{p}{[}\PYG{l+m+mi}{1}\PYG{p}{,}\PYG{p}{:}\PYG{p}{]} \PYG{c+c1}{\PYGZsh{} L1 \PYGZlt{}\PYGZhy{} L1 \PYGZhy{} m3*L2  }
\PYG{n}{m4} \PYG{o}{=} \PYG{n}{AB}\PYG{p}{[}\PYG{l+m+mi}{2}\PYG{p}{,}\PYG{l+m+mi}{1}\PYG{p}{]}
\PYG{n}{AB}\PYG{p}{[}\PYG{l+m+mi}{2}\PYG{p}{,}\PYG{p}{:}\PYG{p}{]} \PYG{o}{\PYGZhy{}}\PYG{o}{=} \PYG{n}{m4}\PYG{o}{*}\PYG{n}{AB}\PYG{p}{[}\PYG{l+m+mi}{1}\PYG{p}{,}\PYG{p}{:}\PYG{p}{]} \PYG{c+c1}{\PYGZsh{} L3 \PYGZlt{}\PYGZhy{} L3 \PYGZhy{} m4*L2 }
\PYG{n+nb}{print}\PYG{p}{(}\PYG{n}{AB}\PYG{p}{)}
\end{sphinxVerbatim}

\begin{sphinxVerbatim}[commandchars=\\\{\}]
[[ 1.00000000e+00  0.00000000e+00 \PYGZhy{}6.80628272e\PYGZhy{}02  2.52356021e+00]
 [ 0.00000000e+00  1.00000000e+00 \PYGZhy{}4.18848168e\PYGZhy{}02 \PYGZhy{}2.79319372e+00]
 [ 0.00000000e+00  0.00000000e+00  1.00120419e+01  7.00842932e+01]]
\end{sphinxVerbatim}

\begin{sphinxVerbatim}[commandchars=\\\{\}]
\PYG{c+c1}{\PYGZsh{} Normalização da 3a. linha}
\PYG{n}{AB}\PYG{p}{[}\PYG{l+m+mi}{2}\PYG{p}{,}\PYG{p}{:}\PYG{p}{]} \PYG{o}{/}\PYG{o}{=} \PYG{n}{AB}\PYG{p}{[}\PYG{l+m+mi}{2}\PYG{p}{,}\PYG{l+m+mi}{2}\PYG{p}{]} \PYG{c+c1}{\PYGZsh{} L3 \PYGZlt{}\PYGZhy{} L3/a33}
\PYG{n+nb}{print}\PYG{p}{(}\PYG{n}{AB}\PYG{p}{)}
\end{sphinxVerbatim}

\begin{sphinxVerbatim}[commandchars=\\\{\}]
[[ 1.          0.         \PYGZhy{}0.06806283  2.52356021]
 [ 0.          1.         \PYGZhy{}0.04188482 \PYGZhy{}2.79319372]
 [ 0.          0.          1.          7.        ]]
\end{sphinxVerbatim}

\begin{sphinxVerbatim}[commandchars=\\\{\}]
\PYG{c+c1}{\PYGZsh{} Eliminação de x3 da 1a. e 2a. linhas}
\PYG{n}{m5} \PYG{o}{=} \PYG{n}{AB}\PYG{p}{[}\PYG{l+m+mi}{0}\PYG{p}{,}\PYG{l+m+mi}{2}\PYG{p}{]}
\PYG{n}{AB}\PYG{p}{[}\PYG{l+m+mi}{0}\PYG{p}{,}\PYG{p}{:}\PYG{p}{]} \PYG{o}{\PYGZhy{}}\PYG{o}{=} \PYG{n}{m5}\PYG{o}{*}\PYG{n}{AB}\PYG{p}{[}\PYG{l+m+mi}{2}\PYG{p}{,}\PYG{p}{:}\PYG{p}{]} \PYG{c+c1}{\PYGZsh{} L1 \PYGZlt{}\PYGZhy{} L1 \PYGZhy{} m5*L3}
\PYG{n}{m6} \PYG{o}{=} \PYG{n}{AB}\PYG{p}{[}\PYG{l+m+mi}{1}\PYG{p}{,}\PYG{l+m+mi}{2}\PYG{p}{]}
\PYG{n}{AB}\PYG{p}{[}\PYG{l+m+mi}{1}\PYG{p}{,}\PYG{p}{:}\PYG{p}{]} \PYG{o}{\PYGZhy{}}\PYG{o}{=} \PYG{n}{m6}\PYG{o}{*}\PYG{n}{AB}\PYG{p}{[}\PYG{l+m+mi}{2}\PYG{p}{,}\PYG{p}{:}\PYG{p}{]} \PYG{c+c1}{\PYGZsh{} L2 \PYGZlt{}\PYGZhy{} L2 \PYGZhy{} m5*L3}
\PYG{n+nb}{print}\PYG{p}{(}\PYG{n}{AB}\PYG{p}{)}
\end{sphinxVerbatim}

\begin{sphinxVerbatim}[commandchars=\\\{\}]
[[ 1.   0.   0.   3. ]
 [ 0.   1.   0.  \PYGZhy{}2.5]
 [ 0.   0.   1.   7. ]]
\end{sphinxVerbatim}

\sphinxAtStartPar
Do último resultado, vemos que a matriz identidade é obtida, apontando para o vetor solução \([3 \ \ -2.5 \ \ 7]^T\).


\section{Notas}
\label{\detokenize{aula-09-eliminacao-gauss:notas}}
\sphinxAtStartPar
Para aqueles acostumados com a notação para matrizes do Matlab, o método \sphinxcode{\sphinxupquote{np.mat}} pode ajudar. No exemplo a seguir, as linhas da matriz são passadas como uma expressão do tipo \sphinxcode{\sphinxupquote{str}} e separadas com \sphinxcode{\sphinxupquote{;}}.

\begin{sphinxVerbatim}[commandchars=\\\{\}]
\PYG{c+c1}{\PYGZsh{} cria matriz}
\PYG{n}{np}\PYG{o}{.}\PYG{n}{array}\PYG{p}{(}\PYG{n}{np}\PYG{o}{.}\PYG{n}{mat}\PYG{p}{(}\PYG{l+s+s1}{\PYGZsq{}}\PYG{l+s+s1}{1 2; 3 4}\PYG{l+s+s1}{\PYGZsq{}}\PYG{p}{)}\PYG{p}{)}
\end{sphinxVerbatim}

\begin{sphinxVerbatim}[commandchars=\\\{\}]
array([[1, 2],
       [3, 4]])
\end{sphinxVerbatim}

\begin{sphinxVerbatim}[commandchars=\\\{\}]
\PYG{n}{np}\PYG{o}{.}\PYG{n}{array}\PYG{p}{(}\PYG{n}{np}\PYG{o}{.}\PYG{n}{mat}\PYG{p}{(}\PYG{l+s+s1}{\PYGZsq{}}\PYG{l+s+s1}{1 2 3}\PYG{l+s+s1}{\PYGZsq{}}\PYG{p}{)}\PYG{p}{)}
\end{sphinxVerbatim}

\begin{sphinxVerbatim}[commandchars=\\\{\}]
array([[1, 2, 3]])
\end{sphinxVerbatim}


\section{Tarefa}
\label{\detokenize{aula-09-eliminacao-gauss:tarefa}}
\sphinxAtStartPar
Implemente o algoritmo pleno para a eliminação gaussiana. Verifique exceções (erros que devem ser observados para evitar falhas no algoritmo, tal como pivôs nulos e matrizes singulares, por exemplo) e use o seu método para resolver os sistemas lineares da forma \({\bf A}{\bf x} = {\bf b}\) da lista de exercícios.


\chapter{Fatoração LU com Python}
\label{\detokenize{aula-10-fatoracao-lu:fatoracao-lu-com-python}}\label{\detokenize{aula-10-fatoracao-lu::doc}}

\section{Decomposição LU}
\label{\detokenize{aula-10-fatoracao-lu:decomposicao-lu}}
\sphinxAtStartPar
Suponha o sistema de equações
\begin{equation*}
\begin{split}AX = B\end{split}
\end{equation*}
\sphinxAtStartPar
A motivação para a decomposição LU baseia\sphinxhyphen{}se na observação de que sistemas triangulares são mais fáceis de resolver. Semelhantemente à Eliminação Gaussiana, a decomposição LU (que, na verdade, é uma segunda abordagem da própria Eliminação Gaussiana), explora a ideia de “fatoração” de matrizes, em que a matriz original do sistema é fatorada (“quebrada”) como
\begin{equation*}
\begin{split}A = LU,\end{split}
\end{equation*}
\sphinxAtStartPar
onde \(L\) é uma matriz triangular inferior e \(U\) é triangular superior. Nosso objetivo é determinar \(L\) e \(U\), de maneira que o vetor \(X\) seja obtido através da resolução de um par de sistemas cujas matrizes são triangulares.


\subsection{Exemplo}
\label{\detokenize{aula-10-fatoracao-lu:exemplo}}
\sphinxAtStartPar
Consideraremos que as matrizes triangulares inferiores \(L\) sempre terão a sua diagonal principal formada por entradas iguais a 1. Este tipo de fatoração é conhecido como \sphinxstyleemphasis{Fatoração de Doolittle}.
\begin{equation*}
\begin{split}{\bf A} = \begin{bmatrix}
1 & 2 & 4\\
3 & 8 & 14\\
2 & 6 & 13
\end{bmatrix} = LU\end{split}
\end{equation*}
\sphinxAtStartPar
onde
\begin{equation*}
\begin{split}{\bf L} = \begin{bmatrix}
1 & 0 & 0\\
L_{21} & 1 & 0\\
L_{31} & L_{32} & 1
\end{bmatrix}
\quad \text{e} \quad
{\bf U} = \begin{bmatrix}
U_{11} & U_{12} & U_{13}\\
0 & U_{22} & U_{23}\\
0 & 0 & U_{33}
\end{bmatrix}
\end{split}
\end{equation*}
\sphinxAtStartPar
Multiplicando \(LU\) e definindo a resposta igual a \(A\) temos:
\begin{equation*}
\begin{split}\begin{bmatrix}
U_{11} & U_{12} & U_{13}\\
L_{21} U_{11} & L_{21} U_{12} + U_{22} & L_{21} U_{13} + U_{23}\\
L_{31} U_{11} & L_{31} U_{12} + L_{32} U_{22} & L_{31} U_{13} + L_{32} U_{23} + U_{33}
\end{bmatrix} = \begin{bmatrix}
1 & 2 & 4\\
3 & 8 & 14\\
2 & 6 & 13
\end{bmatrix}
\end{split}
\end{equation*}
\sphinxAtStartPar
Agora, através de substituição de recorrência, facilmente encontramos \(L\) e \(U\).
\begin{equation*}
\begin{split}
{\bf A} = \begin{bmatrix}
1 & 2 & 4\\
3 & 8 & 14\\
2 & 6 & 13
\end{bmatrix} = \begin{bmatrix}
1 & 0 & 0\\
3 & 1 & 0\\
2 & 1 & 1
\end{bmatrix} \begin{bmatrix}
1 & 2 & 4\\
0 & 2 & 2\\
0 & 0 & 3
\end{bmatrix}
\end{split}
\end{equation*}

\section{Usando a decomposição LU para resolver sistemas de equações}
\label{\detokenize{aula-10-fatoracao-lu:usando-a-decomposicao-lu-para-resolver-sistemas-de-equacoes}}
\sphinxAtStartPar
Uma vez decomposta a matriz \(A\) no produto \(LU\), podemos obter a solução \(X\) de forma direta. O procedimento, equivalente à substituição de recorrência mencionada anteriormente, pode ser resumido como segue: dada \(A\), encontre \(L\) e \(U\) tal que \(A = LU\), ou seja, \((LU)X = B\). Em outras palavras:
\begin{itemize}
\item {} 
\sphinxAtStartPar
Defina \(Y = UX\).

\item {} 
\sphinxAtStartPar
Resolva o sistema triangular \(LY = B\) para \(Y\).

\item {} 
\sphinxAtStartPar
Finalmente, resolva o sistema triangular \(UX = Y\) para \(X\).

\end{itemize}

\sphinxAtStartPar
O benefício desta abordagem é a resolução de somente sistemas triangulares. Por outro lado, o custo empregado é termos de resolver dois sistemas.


\subsection{Exemplo}
\label{\detokenize{aula-10-fatoracao-lu:id1}}
\sphinxAtStartPar
Encontre a solução \(X = \begin{bmatrix} x_1 \\ x_2 \\ x_3 \end{bmatrix}\) do sistema
\begin{equation*}
\begin{split}
\begin{bmatrix} 
1 & 2 & 4\\ 
3 & 8 & 14\\ 
2 & 6 & 13 
\end{bmatrix} 
\begin{bmatrix} 
x_1 \\ x_2 \\ x_3 
\end{bmatrix} 
= 
\begin{bmatrix} 
3 \\ 13 \\ 4
\end{bmatrix}.
\end{split}
\end{equation*}\begin{itemize}
\item {} 
\sphinxAtStartPar
As matrizes \(L\) e \(U\) já foram obtidas anteriormente.

\end{itemize}
\begin{equation*}
\begin{split}L = \begin{bmatrix}
1 & 0 & 0\\
3 & 1 & 0\\
2 & 1 & 1
\end{bmatrix},
\quad 
U = \begin{bmatrix}
1 & 2 & 4\\
0 & 2 & 2\\
0 & 0 & 3
\end{bmatrix}\end{split}
\end{equation*}\begin{itemize}
\item {} 
\sphinxAtStartPar
A próxima etapa é resolver \(LY = B\), para o vetor \(Y = \begin{bmatrix} y_1 \\ y_2 \\ y_3 \end{bmatrix}\).

\end{itemize}
\begin{equation*}
\begin{split}LY = \begin{bmatrix}
1 & 0 & 0\\
3 & 1 & 0\\
2 & 1 & 1
\end{bmatrix}
\begin{bmatrix}
x_1 \\ x_2 \\ x_3
\end{bmatrix} =
\begin{bmatrix}
3 \\ 13 \\ 4
\end{bmatrix} = B\end{split}
\end{equation*}
\sphinxAtStartPar
Este sistema pode ser resolvido por substituição direta, obtendo \(Y = \begin{bmatrix} 3 \\ 4 \\ -6 \end{bmatrix}\).
\begin{itemize}
\item {} 
\sphinxAtStartPar
Agora que encontramos \(Y\), concluímos o procedimento resolvendo \(UX = Y\) para \(X\). Ou seja, resolvemos:

\end{itemize}
\begin{equation*}
\begin{split}
UX = \begin{bmatrix}
1 & 2 & 4\\
0 & 2 & 2\\
0 & 0 & 3
\end{bmatrix}
\begin{bmatrix} x_1 \\ x_2 \\ x_3 \end{bmatrix} = 
\begin{bmatrix} 3 \\ 4 \\ -6 \end{bmatrix}
\end{split}
\end{equation*}
\sphinxAtStartPar
Realizando a substituição regressiva (baixo para cima; da direita para a esquerda), obtemos a solução do problema.
\begin{equation*}
\begin{split}X = \begin{bmatrix} 3 \\ 4 \\ -2 \end{bmatrix}\end{split}
\end{equation*}
\sphinxAtStartPar
Abaixo, temos uma implementação de uma fatoração LU sem pivoteamento.

\begin{sphinxVerbatim}[commandchars=\\\{\}]
\PYG{k+kn}{import} \PYG{n+nn}{numpy} \PYG{k}{as} \PYG{n+nn}{np} 

\PYG{k}{def} \PYG{n+nf}{lu\PYGZus{}nopivot}\PYG{p}{(}\PYG{n}{A}\PYG{p}{)}\PYG{p}{:}
    \PYG{l+s+sd}{\PYGZsq{}\PYGZsq{}\PYGZsq{}}
\PYG{l+s+sd}{    Realiza fatoração LU para a matriz A}
\PYG{l+s+sd}{    }
\PYG{l+s+sd}{    entrada: }
\PYG{l+s+sd}{        A  \PYGZhy{}  matriz:  array (nxn) }
\PYG{l+s+sd}{    }
\PYG{l+s+sd}{    saida: }
\PYG{l+s+sd}{        L,U  \PYGZhy{} matriz triangular inferior, superior : array (nxn)         }
\PYG{l+s+sd}{    \PYGZsq{}\PYGZsq{}\PYGZsq{}}
        
    \PYG{n}{n} \PYG{o}{=} \PYG{n}{np}\PYG{o}{.}\PYG{n}{shape}\PYG{p}{(}\PYG{n}{A}\PYG{p}{)}\PYG{p}{[}\PYG{l+m+mi}{0}\PYG{p}{]} \PYG{c+c1}{\PYGZsh{} dimensao}
    \PYG{n}{L} \PYG{o}{=} \PYG{n}{np}\PYG{o}{.}\PYG{n}{eye}\PYG{p}{(}\PYG{n}{n}\PYG{p}{)} \PYG{c+c1}{\PYGZsh{} auxiliar }
    
    \PYG{k}{for} \PYG{n}{k} \PYG{o+ow}{in} \PYG{n}{np}\PYG{o}{.}\PYG{n}{arange}\PYG{p}{(}\PYG{n}{n}\PYG{p}{)}\PYG{p}{:}
        \PYG{n}{L}\PYG{p}{[}\PYG{n}{k}\PYG{o}{+}\PYG{l+m+mi}{1}\PYG{p}{:}\PYG{n}{n}\PYG{p}{,}\PYG{n}{k}\PYG{p}{]} \PYG{o}{=} \PYG{n}{A}\PYG{p}{[}\PYG{n}{k}\PYG{o}{+}\PYG{l+m+mi}{1}\PYG{p}{:}\PYG{n}{n}\PYG{p}{,}\PYG{n}{k}\PYG{p}{]}\PYG{o}{/}\PYG{n}{A}\PYG{p}{[}\PYG{n}{k}\PYG{p}{,}\PYG{n}{k}\PYG{p}{]}        
        \PYG{k}{for} \PYG{n}{l} \PYG{o+ow}{in} \PYG{n}{np}\PYG{o}{.}\PYG{n}{arange}\PYG{p}{(}\PYG{n}{k}\PYG{o}{+}\PYG{l+m+mi}{1}\PYG{p}{,}\PYG{n}{n}\PYG{p}{)}\PYG{p}{:}
            \PYG{n}{A}\PYG{p}{[}\PYG{n}{l}\PYG{p}{,}\PYG{p}{:}\PYG{p}{]} \PYG{o}{=} \PYG{n}{A}\PYG{p}{[}\PYG{n}{l}\PYG{p}{,}\PYG{p}{:}\PYG{p}{]} \PYG{o}{\PYGZhy{}} \PYG{n}{np}\PYG{o}{.}\PYG{n}{dot}\PYG{p}{(}\PYG{n}{L}\PYG{p}{[}\PYG{n}{l}\PYG{p}{,}\PYG{n}{k}\PYG{p}{]}\PYG{p}{,}\PYG{n}{A}\PYG{p}{[}\PYG{n}{k}\PYG{p}{,}\PYG{p}{:}\PYG{p}{]}\PYG{p}{)}
            
    \PYG{n}{U} \PYG{o}{=} \PYG{n}{A}
    \PYG{k}{return} \PYG{p}{(}\PYG{n}{L}\PYG{p}{,}\PYG{n}{U}\PYG{p}{)}
\end{sphinxVerbatim}

\sphinxAtStartPar
\sphinxstylestrong{Exemplo:} Fatoração \(\bf{ LU }\) da matriz
\begin{equation*}
\begin{split}{\bf A} =
\begin{bmatrix}
4 & -2 & -3 &  6 \\  
1 &  4 &  2 &  3 \\
2 & -3 &  3 & -2 \\ 
1 &  5 &  3 &  4
\end{bmatrix}\end{split}
\end{equation*}
\begin{sphinxVerbatim}[commandchars=\\\{\}]
\PYG{n}{A} \PYG{o}{=} \PYG{n}{np}\PYG{o}{.}\PYG{n}{array}\PYG{p}{(}\PYG{p}{[}\PYG{p}{[} \PYG{l+m+mf}{4.}\PYG{p}{,} \PYG{o}{\PYGZhy{}}\PYG{l+m+mf}{2.}\PYG{p}{,} \PYG{o}{\PYGZhy{}}\PYG{l+m+mf}{3.}\PYG{p}{,}  \PYG{l+m+mf}{6.}\PYG{p}{]}\PYG{p}{,}\PYG{p}{[} \PYG{l+m+mf}{1.}\PYG{p}{,}  \PYG{l+m+mf}{4.}\PYG{p}{,}  \PYG{l+m+mf}{2.}\PYG{p}{,}  \PYG{l+m+mf}{3.}\PYG{p}{]}\PYG{p}{,}\PYG{p}{[} \PYG{l+m+mf}{2.}\PYG{p}{,}  \PYG{o}{\PYGZhy{}}\PYG{l+m+mf}{3.}\PYG{p}{,}  \PYG{l+m+mf}{3.}\PYG{p}{,} \PYG{o}{\PYGZhy{}}\PYG{l+m+mf}{2.}\PYG{p}{]}\PYG{p}{,}\PYG{p}{[} \PYG{l+m+mf}{1.}\PYG{p}{,}  \PYG{l+m+mf}{5.}\PYG{p}{,}  \PYG{l+m+mf}{3.}\PYG{p}{,}  \PYG{l+m+mf}{4.}\PYG{p}{]}\PYG{p}{]}\PYG{p}{)}

\PYG{n}{L}\PYG{p}{,}\PYG{n}{U} \PYG{o}{=} \PYG{n}{lu\PYGZus{}nopivot}\PYG{p}{(}\PYG{n}{A}\PYG{p}{)}
\end{sphinxVerbatim}

\begin{sphinxVerbatim}[commandchars=\\\{\}]
\PYG{n}{L}
\end{sphinxVerbatim}

\begin{sphinxVerbatim}[commandchars=\\\{\}]
array([[ 1.        ,  0.        ,  0.        ,  0.        ],
       [ 0.25      ,  1.        ,  0.        ,  0.        ],
       [ 0.5       , \PYGZhy{}0.44444444,  1.        ,  0.        ],
       [ 0.25      ,  1.22222222,  0.06796117,  1.        ]])
\end{sphinxVerbatim}

\begin{sphinxVerbatim}[commandchars=\\\{\}]
\PYG{n}{U}
\end{sphinxVerbatim}

\begin{sphinxVerbatim}[commandchars=\\\{\}]
array([[ 4.        , \PYGZhy{}2.        , \PYGZhy{}3.        ,  6.        ],
       [ 0.        ,  4.5       ,  2.75      ,  1.5       ],
       [ 0.        ,  0.        ,  5.72222222, \PYGZhy{}4.33333333],
       [ 0.        ,  0.        ,  0.        ,  0.96116505]])
\end{sphinxVerbatim}


\chapter{Fatoração de Cholesky}
\label{\detokenize{aula-11-cholesky:fatoracao-de-cholesky}}\label{\detokenize{aula-11-cholesky::doc}}
\begin{sphinxVerbatim}[commandchars=\\\{\}]
\PYG{o}{\PYGZpc{}}\PYG{k}{matplotlib} inline
\end{sphinxVerbatim}


\section{Matrizes positivas definidas}
\label{\detokenize{aula-11-cholesky:matrizes-positivas-definidas}}
\sphinxAtStartPar
\sphinxstylestrong{Definição (baseada em autovalores)} uma matriz \({\bf A}\) é \sphinxstyleemphasis{positiva definida} se todos os seus autovalores são positivos (\(\lambda > 0\)).

\sphinxAtStartPar
Entretanto, não é conveniente computar todos os autovalores de uma matriz para saber se ela é ou não positiva definida. Há meios mais rápidos de fazer este teste como o da “energia”.

\sphinxAtStartPar
\sphinxstylestrong{Definição (baseada em energia)}: uma matriz \({\bf A}\) é \sphinxstyleemphasis{positiva definida} se \({\bf x}^T {\bf A} {\bf x} > 0\) para todo vetor não\sphinxhyphen{}nulo \({\bf x}\).

\sphinxAtStartPar
Para a ordem \(n=2\), temos os eguinte resultado:
\begin{equation*}
\begin{split}{\bf x}^T {\bf A} {\bf x} = 
\left[\begin{matrix}x_{1} & x_{2}\end{matrix}\right]
\left[\begin{matrix}a_{11} & a_{12}\\a_{12} & a_{22}\end{matrix}\right]
\left[\begin{matrix}x_{1}\\x_{2}\end{matrix}\right]
= a_{11} x_{1}^{2} + 2 a_{12} x_{1} x_{2} + a_{22} x_{2}^{2} > 0\end{split}
\end{equation*}
\sphinxAtStartPar
Em muitas aplicações, este número é a “energia” no sistema.

\sphinxAtStartPar
O código abaixo mostra que esta multiplicação produz uma forma quadrática para a ordem \(n\).

\begin{sphinxVerbatim}[commandchars=\\\{\}]
\PYG{k+kn}{import} \PYG{n+nn}{sympy} \PYG{k}{as} \PYG{n+nn}{sp}
\PYG{n}{sp}\PYG{o}{.}\PYG{n}{init\PYGZus{}printing}\PYG{p}{(}\PYG{n}{use\PYGZus{}unicode}\PYG{o}{=}\PYG{k+kc}{True}\PYG{p}{)}

\PYG{n}{x1}\PYG{p}{,}\PYG{n}{x2}\PYG{p}{,}\PYG{n}{x3} \PYG{o}{=} \PYG{n}{sp}\PYG{o}{.}\PYG{n}{symbols}\PYG{p}{(}\PYG{l+s+s1}{\PYGZsq{}}\PYG{l+s+s1}{x1,x2,x3}\PYG{l+s+s1}{\PYGZsq{}}\PYG{p}{)}

\PYG{c+c1}{\PYGZsh{} ordem do sistema}
\PYG{n}{n} \PYG{o}{=} \PYG{l+m+mi}{3}

\PYG{n}{x} \PYG{o}{=} \PYG{n}{sp}\PYG{o}{.}\PYG{n}{zeros}\PYG{p}{(}\PYG{l+m+mi}{1}\PYG{p}{,}\PYG{n}{n}\PYG{p}{)}
\PYG{n}{A} \PYG{o}{=} \PYG{n}{sp}\PYG{o}{.}\PYG{n}{zeros}\PYG{p}{(}\PYG{n}{n}\PYG{p}{,}\PYG{n}{n}\PYG{p}{)}
\PYG{n}{aux} \PYG{o}{=} \PYG{l+m+mi}{0}\PYG{o}{*}\PYG{n}{A}\PYG{p}{;}
\PYG{k}{for} \PYG{n}{i} \PYG{o+ow}{in} \PYG{n+nb}{range}\PYG{p}{(}\PYG{n}{n}\PYG{p}{)}\PYG{p}{:}
    \PYG{n}{x}\PYG{p}{[}\PYG{n}{i}\PYG{p}{]} \PYG{o}{=} \PYG{n}{sp}\PYG{o}{.}\PYG{n}{symbols}\PYG{p}{(}\PYG{l+s+s1}{\PYGZsq{}}\PYG{l+s+s1}{x}\PYG{l+s+s1}{\PYGZsq{}} \PYG{o}{+} \PYG{n+nb}{str}\PYG{p}{(}\PYG{n}{i}\PYG{o}{+}\PYG{l+m+mi}{1}\PYG{p}{)}\PYG{p}{)}
    \PYG{k}{for} \PYG{n}{j} \PYG{o+ow}{in} \PYG{n+nb}{range}\PYG{p}{(}\PYG{n}{n}\PYG{p}{)}\PYG{p}{:}    
        \PYG{k}{if} \PYG{n}{i} \PYG{o}{==} \PYG{n}{j}\PYG{p}{:} \PYG{c+c1}{\PYGZsh{} diagonal}
            \PYG{n}{A}\PYG{p}{[}\PYG{n}{i}\PYG{p}{,}\PYG{n}{i}\PYG{p}{]} \PYG{o}{=} \PYG{n}{sp}\PYG{o}{.}\PYG{n}{symbols}\PYG{p}{(}\PYG{l+s+s1}{\PYGZsq{}}\PYG{l+s+s1}{a}\PYG{l+s+s1}{\PYGZsq{}} \PYG{o}{+} \PYG{n+nb}{str}\PYG{p}{(}\PYG{n}{i}\PYG{o}{+}\PYG{l+m+mi}{1}\PYG{p}{)} \PYG{o}{+} \PYG{n+nb}{str}\PYG{p}{(}\PYG{n}{i}\PYG{o}{+}\PYG{l+m+mi}{1}\PYG{p}{)}\PYG{p}{)}        
        \PYG{k}{elif} \PYG{n}{j} \PYG{o}{\PYGZgt{}} \PYG{n}{i}\PYG{p}{:} \PYG{c+c1}{\PYGZsh{} triang. superior}
            \PYG{n}{aux}\PYG{p}{[}\PYG{n}{i}\PYG{p}{,}\PYG{n}{j}\PYG{p}{]} \PYG{o}{=} \PYG{n}{sp}\PYG{o}{.}\PYG{n}{symbols}\PYG{p}{(}\PYG{l+s+s1}{\PYGZsq{}}\PYG{l+s+s1}{a}\PYG{l+s+s1}{\PYGZsq{}} \PYG{o}{+} \PYG{n+nb}{str}\PYG{p}{(}\PYG{n}{i}\PYG{o}{+}\PYG{l+m+mi}{1}\PYG{p}{)} \PYG{o}{+} \PYG{n+nb}{str}\PYG{p}{(}\PYG{n}{j}\PYG{o}{+}\PYG{l+m+mi}{1}\PYG{p}{)}\PYG{p}{)}        

\PYG{c+c1}{\PYGZsh{} matriz}
\PYG{n}{A} \PYG{o}{=} \PYG{n}{A} \PYG{o}{+} \PYG{n}{aux}\PYG{o}{.}\PYG{n}{T} \PYG{o}{+} \PYG{n}{aux} \PYG{c+c1}{\PYGZsh{} compõe D + L + U}
\PYG{n}{A} \PYG{o}{=} \PYG{n}{sp}\PYG{o}{.}\PYG{n}{Matrix}\PYG{p}{(}\PYG{n}{A}\PYG{p}{)}

\PYG{c+c1}{\PYGZsh{} vetor}
\PYG{n}{x} \PYG{o}{=} \PYG{n}{sp}\PYG{o}{.}\PYG{n}{Matrix}\PYG{p}{(}\PYG{n}{x}\PYG{p}{)}
\PYG{n}{x} \PYG{o}{=} \PYG{n}{x}\PYG{o}{.}\PYG{n}{T}

\PYG{c+c1}{\PYGZsh{} matriz positiva definida}
\PYG{n}{c} \PYG{o}{=} \PYG{p}{(}\PYG{n}{x}\PYG{o}{.}\PYG{n}{T}\PYG{p}{)}\PYG{o}{*}\PYG{n}{A}\PYG{o}{*}\PYG{n}{x}

\PYG{c+c1}{\PYGZsh{} expressãoo quadrática}
\PYG{n}{sp}\PYG{o}{.}\PYG{n}{expand}\PYG{p}{(}\PYG{n}{c}\PYG{p}{[}\PYG{l+m+mi}{0}\PYG{p}{]}\PYG{p}{)}
\end{sphinxVerbatim}

\noindent\sphinxincludegraphics{{aula-11-cholesky_3_0}.png}

\sphinxAtStartPar
Ainda falando sobre o caso \(n=2\), observamos que os autovalores da matriz \({\bf A}\) são positivos se, e somente se
\begin{equation*}
\begin{split}a_{11}>0 \qquad  \text{e} \qquad  a_{11}a_{22} - a_{12}^2 > 0.\end{split}
\end{equation*}
\sphinxAtStartPar
Na verdade, esta regra vale para todos os \sphinxstylestrong{pivôs}. Estes dois últimos valores são os pivôs de uma matriz simétrica 2x2 (verifique a eliminação de Gauss quando aplicada à segunda equação).

\sphinxAtStartPar
A teoria da Álgebra Linear permite\sphinxhyphen{}nos elencar as seguintes declarações, \sphinxstylestrong{todas equivalentes}, acerca da determinação de uma matriz positiva definida \({\bf A}\):
\begin{enumerate}
\sphinxsetlistlabels{\arabic}{enumi}{enumii}{}{.}%
\item {} 
\sphinxAtStartPar
Todos os seus \(n\) pivôs são positivos.

\item {} 
\sphinxAtStartPar
Todos os determinantes menores superiores esquerdos (ou principais) são positivos (veja Critério de Sylvester).

\item {} 
\sphinxAtStartPar
Todos os seus \(n\) autovalores são positivos.

\item {} 
\sphinxAtStartPar
\({\bf x}^T {\bf A}{\bf x} > 0\) para todo vetor não\sphinxhyphen{}nulo \({\bf x}\). (Definição baseada na “energia”).

\item {} 
\sphinxAtStartPar
\({\bf A} = {\bf G}{\bf G}^T\) para uma matriz \({\bf G}\) com colunas independentes.

\end{enumerate}


\subsection{Interpretação geométrica}
\label{\detokenize{aula-11-cholesky:interpretacao-geometrica}}
\sphinxAtStartPar
Matrizes positivas definidas realizam transformações “limitadas” no sentido de “semiplano” do vetor transformado. Por exemplo, se tomarmos um vetor \({\bf x} \in \mathbb{R}^2\) não\sphinxhyphen{}nulo e usarmos o fato de que para uma matriz positiva definida, a inequação \({\bf x}^T{\bf A}{\bf x} > 0\) deve valer, ao chamarmos \({\bf y} = {\bf A}{\bf x}\), a expressão anterior é o produto interno entre \({\bf x}\) e \({\bf y}\), a saber \({\bf x}^T{\bf y}\). Se o produto interno é positivo, já sabemos que os vetores não são ortogonais. Agora, para verificar que eles realmente pertencem a um mesmo semiplano, usaremos a seguinte expressão para o ângulo entre dois vetores:
\begin{equation*}
\begin{split}\cos(\theta) = \dfrac{{\bf x}^T {\bf y}}{||{\bf x}||{\bf y}||}\end{split}
\end{equation*}
\sphinxAtStartPar
Uma vez que a norma (comprimento) de um vetor é sempre um número real positivo, o produto \(||{\bf x}||{\bf y}||\) no denominador acima é um número real positivo. Se \({\bf x}^T {\bf y}\) > 0, então \(\cos(\theta) > 0\), e o efeito geométrico da transformação é
\begin{equation*}
\begin{split}|\theta| < \frac{\pi}{2},\end{split}
\end{equation*}
\sphinxAtStartPar
ou seja, o ângulo entre \({\bf x}\) e \({\bf y}\) é sempre menor do que 90 graus (ou \(\pi/2\) radianos).


\section{Método da Fatoração de Cholesky}
\label{\detokenize{aula-11-cholesky:metodo-da-fatoracao-de-cholesky}}
\sphinxAtStartPar
Trata\sphinxhyphen{}se de um algoritmo para resolução de sistemas lineares \({\bf A}{\bf x} = {\bf b}\) através da decomposição da matriz \({\bf A}\) em dois fatores simétricos, \({\bf G}\) e \({\bf G}^T\). O método é aplicável apenas a sistemas cuja matriz associada é simétrica e positiva definida.


\subsection{Passos}
\label{\detokenize{aula-11-cholesky:passos}}\begin{itemize}
\item {} 
\sphinxAtStartPar
Primeiramente, é necessário checar se a matriz associada ao sistema cumpre os requisitos da fatoração de Cholesky:
\begin{itemize}
\item {} 
\sphinxAtStartPar
\sphinxstylestrong{Simetria}: a matriz é simétrica quando sua transposta é igual a ela própria:
\$\(A=A^T\)\$

\item {} 
\sphinxAtStartPar
\sphinxstylestrong{Definição positiva}: averiguamos se o Critério de Sylvester é satisfeito. Ou seja, verificamos se todos os determinantes menores principais da matriz, constituídos pelas \(k\) primeiras linhas e \(k\) primeiras colunas dela, são maiores do que zero:

\end{itemize}

\end{itemize}
\begin{equation*}
\begin{split}\det({\bf A}_k) > 0\text{, onde } k=1,2,\ldots,n \text{ para matrizes }A_{nxn}\end{split}
\end{equation*}
\sphinxAtStartPar
Concluídas as verificações anteriores, decompomos a matriz \({\bf A}_{nxn}\) em uma triangular inferior \({\bf A}\) e sua transposta \({\bf A}^T\), a qual é triangular superior.  O processo é descrito abaixo:
\begin{itemize}
\item {} 
\sphinxAtStartPar
Para uma matriz \(A_{4x4}\) obtém\sphinxhyphen{}se o fator de Cholesky da seguinte forma:

\end{itemize}
\begin{equation*}
\begin{split}{\bf A} = {\bf G}{\bf G}^T \Rightarrow
 \begin{bmatrix}
a_{11} & a_{12} & a_{13} & a_{14} \\
a_{21} & a_{22} & a_{23} & a_{24} \\
a_{31} & a_{32} & a_{33} & a_{34} \\
a_{41} & a_{42} & a_{43} & a_{44}
 \end{bmatrix} =
 \begin{bmatrix}
g_{11} &             &            &              \\
g_{21} & g_{22} &            &              \\
g_{31} & g_{32} & g_{33} &             \\
g_{41} & g_{42} & g_{43} & g_{44} \\
 \end{bmatrix}
 \begin{bmatrix}
g_{11} & g_{21} & g_{31} &  g_{41} \\
            & g_{22} & g_{32} & g_{42} \\
            &             & g_{33} & g_{43} \\
           &              &             & g_{44} \\
 \end{bmatrix}
\end{split}
\end{equation*}\begin{itemize}
\item {} 
\sphinxAtStartPar
Este sistema pode ser resolvido comparando cada coluna de \({\bf A}\) com a multiplicação de \({\bf G}\) por cada coluna de \({\bf G}^T\) de maneira que \({\bf x}_{kxn} = {\bf G}{\bf G}^T_{kxn}, \, k = 1,2,\ldots, n\). Através dessa sequência são obtidos sistemas simples, em que cada coluna \(k\) terá os valores de \(g_{kn}\)e, após a atualização do iterador \(k \rightarrow k + 1\), não será necessário resolver novamente as \(k − 1\) linhas do próximo sistema gerado.

\end{itemize}

\sphinxAtStartPar
De posse dos valores que geram as matrizes triangulares, é possível seguir para a última etapa, a qual consiste em obter o vetor \({\bf x}\) fazendo
\begin{equation*}
\begin{split}{\bf A}{\bf x}={\bf b} \implies ({\bf G}{\bf G}^T){\bf x}={\bf b} \implies {\bf G}({\bf G}^T{\bf x})={\bf b} \implies {\bf G}{\bf y}=B.\end{split}
\end{equation*}
\sphinxAtStartPar
Aqui, usamos o mesmo processo utilizado na Fatoração LU. Primeiramente, obtemos o vetor \({\bf y}\) após a resolução do sistema \({\bf G}{\bf y} = {\bf b}\). Enfim, obtemos \({\bf x}\) através da relação \({\bf G}^T {\bf x} = {\bf y}\). Fazendo com que, assim, seja obtido o vetor \({\bf x}\) que resolve o sistema linear proposto no inicio.


\section{Algoritmo para a fatoração de Cholesky}
\label{\detokenize{aula-11-cholesky:algoritmo-para-a-fatoracao-de-cholesky}}
\sphinxAtStartPar
O código abaixo é uma implementação de um algoritmo para a fatoração de Cholesky por computação simbólica.

\begin{sphinxVerbatim}[commandchars=\\\{\}]
\PYG{c+c1}{\PYGZsh{} implementação de algoritmo simbólico }
\PYG{c+c1}{\PYGZsh{} para a decomposição de Cholesky }

\PYG{n}{B} \PYG{o}{=} \PYG{n}{A}\PYG{p}{[}\PYG{p}{:}\PYG{p}{,}\PYG{p}{:}\PYG{p}{]} \PYG{c+c1}{\PYGZsh{} faz cópia da matriz A}

\PYG{k}{for} \PYG{n}{k} \PYG{o+ow}{in} \PYG{n+nb}{range}\PYG{p}{(}\PYG{l+m+mi}{0}\PYG{p}{,}\PYG{n}{n}\PYG{p}{)}\PYG{p}{:}
    \PYG{k}{for} \PYG{n}{i} \PYG{o+ow}{in} \PYG{n+nb}{range}\PYG{p}{(}\PYG{l+m+mi}{0}\PYG{p}{,}\PYG{n}{k}\PYG{p}{)}\PYG{p}{:}
        \PYG{n}{s} \PYG{o}{=} \PYG{l+m+mf}{0.}
        \PYG{k}{for} \PYG{n}{j} \PYG{o+ow}{in} \PYG{n+nb}{range}\PYG{p}{(}\PYG{l+m+mi}{0}\PYG{p}{,}\PYG{n}{i}\PYG{p}{)}\PYG{p}{:}
            \PYG{n}{s} \PYG{o}{+}\PYG{o}{=} \PYG{n}{B}\PYG{p}{[}\PYG{n}{i}\PYG{p}{,}\PYG{n}{j}\PYG{p}{]}\PYG{o}{*}\PYG{n}{B}\PYG{p}{[}\PYG{n}{k}\PYG{p}{,}\PYG{n}{j}\PYG{p}{]}
        \PYG{n}{B}\PYG{p}{[}\PYG{n}{k}\PYG{p}{,}\PYG{n}{i}\PYG{p}{]} \PYG{o}{=} \PYG{p}{(}\PYG{n}{B}\PYG{p}{[}\PYG{n}{k}\PYG{p}{,}\PYG{n}{i}\PYG{p}{]} \PYG{o}{\PYGZhy{}} \PYG{n}{s}\PYG{p}{)}\PYG{o}{/}\PYG{n}{B}\PYG{p}{[}\PYG{n}{i}\PYG{p}{,}\PYG{n}{i}\PYG{p}{]}
    \PYG{n}{s} \PYG{o}{=} \PYG{l+m+mf}{0.}
    \PYG{k}{for} \PYG{n}{j} \PYG{o+ow}{in} \PYG{n+nb}{range}\PYG{p}{(}\PYG{l+m+mi}{0}\PYG{p}{,}\PYG{n}{k}\PYG{p}{)}\PYG{p}{:}
        \PYG{n}{s} \PYG{o}{+}\PYG{o}{=} \PYG{n}{s} \PYG{o}{+} \PYG{n}{B}\PYG{p}{[}\PYG{n}{k}\PYG{p}{,}\PYG{n}{j}\PYG{p}{]}\PYG{o}{*}\PYG{n}{B}\PYG{p}{[}\PYG{n}{k}\PYG{p}{,}\PYG{n}{j}\PYG{p}{]}
    \PYG{n}{B}\PYG{p}{[}\PYG{n}{k}\PYG{p}{,}\PYG{n}{k}\PYG{p}{]} \PYG{o}{=} \PYG{n}{sp}\PYG{o}{.}\PYG{n}{sqrt}\PYG{p}{(}\PYG{n}{B}\PYG{p}{[}\PYG{n}{k}\PYG{p}{,}\PYG{n}{k}\PYG{p}{]} \PYG{o}{\PYGZhy{}} \PYG{n}{s}\PYG{p}{)}

\PYG{c+c1}{\PYGZsh{} saída }
\PYG{n}{B}
\end{sphinxVerbatim}

\noindent\sphinxincludegraphics{{aula-11-cholesky_9_0}.png}


\subsection{Tarefa}
\label{\detokenize{aula-11-cholesky:tarefa}}
\sphinxAtStartPar
Converta o código simbólico acima para uma versão numérica (ou implemente a sua própria versão) e aplique\sphinxhyphen{}o na matriz abaixo para encontrar o fator de Cholesky:
\begin{equation*}
\begin{split}\textbf{A} = 
\begin{bmatrix}
6 & 15 & 55 \\
15 & 55 & 225 \\
55 & 225 & 979 
\end{bmatrix}\end{split}
\end{equation*}

\section{Cálculo do fator de Cholesky com Python}
\label{\detokenize{aula-11-cholesky:calculo-do-fator-de-cholesky-com-python}}
\begin{sphinxVerbatim}[commandchars=\\\{\}]
\PYG{k+kn}{import} \PYG{n+nn}{matplotlib}\PYG{n+nn}{.}\PYG{n+nn}{pyplot} \PYG{k}{as} \PYG{n+nn}{plt}
\PYG{k+kn}{from} \PYG{n+nn}{scipy} \PYG{k+kn}{import} \PYG{n}{array}\PYG{p}{,} \PYG{n}{linalg}\PYG{p}{,} \PYG{n}{dot}\PYG{p}{,} \PYG{n}{zeros}

\PYG{c+c1}{\PYGZsh{} matriz}
\PYG{n}{A} \PYG{o}{=} \PYG{n}{array}\PYG{p}{(}\PYG{p}{[}\PYG{p}{[}\PYG{l+m+mi}{16}\PYG{p}{,} \PYG{o}{\PYGZhy{}}\PYG{l+m+mi}{4}\PYG{p}{,} \PYG{l+m+mi}{12}\PYG{p}{,} \PYG{o}{\PYGZhy{}}\PYG{l+m+mi}{4}\PYG{p}{]}\PYG{p}{,}
           \PYG{p}{[}\PYG{o}{\PYGZhy{}}\PYG{l+m+mi}{4}\PYG{p}{,} \PYG{l+m+mi}{2}\PYG{p}{,} \PYG{o}{\PYGZhy{}}\PYG{l+m+mi}{1}\PYG{p}{,} \PYG{l+m+mi}{1}\PYG{p}{]}\PYG{p}{,}
           \PYG{p}{[}\PYG{l+m+mi}{12}\PYG{p}{,} \PYG{o}{\PYGZhy{}}\PYG{l+m+mi}{1}\PYG{p}{,} \PYG{l+m+mi}{14}\PYG{p}{,} \PYG{o}{\PYGZhy{}}\PYG{l+m+mi}{2}\PYG{p}{]}\PYG{p}{,}
           \PYG{p}{[}\PYG{o}{\PYGZhy{}}\PYG{l+m+mi}{4}\PYG{p}{,} \PYG{l+m+mi}{1}\PYG{p}{,} \PYG{o}{\PYGZhy{}}\PYG{l+m+mi}{2}\PYG{p}{,} \PYG{l+m+mi}{83}\PYG{p}{]}\PYG{p}{]}\PYG{p}{)}

\PYG{c+c1}{\PYGZsh{} fator de Cholesky do Scipy}
\PYG{n}{L} \PYG{o}{=} \PYG{n}{linalg}\PYG{o}{.}\PYG{n}{cholesky}\PYG{p}{(}\PYG{n}{A}\PYG{p}{,} \PYG{n}{lower}\PYG{o}{=}\PYG{k+kc}{True}\PYG{p}{,} \PYG{n}{overwrite\PYGZus{}a}\PYG{o}{=}\PYG{k+kc}{False}\PYG{p}{,} \PYG{n}{check\PYGZus{}finite}\PYG{o}{=}\PYG{k+kc}{True}\PYG{p}{)}

\PYG{c+c1}{\PYGZsh{} fator de Cholesky implementado}
\PYG{n}{n} \PYG{o}{=} \PYG{n}{A}\PYG{o}{.}\PYG{n}{shape}\PYG{p}{[}\PYG{l+m+mi}{0}\PYG{p}{]}
\PYG{n}{G} \PYG{o}{=} \PYG{n}{zeros}\PYG{p}{(}\PYG{n}{A}\PYG{o}{.}\PYG{n}{shape}\PYG{p}{,} \PYG{n}{dtype}\PYG{o}{=}\PYG{n+nb}{float}\PYG{p}{)}

\PYG{n+nb}{print}\PYG{p}{(}\PYG{l+s+s1}{\PYGZsq{}}\PYG{l+s+s1}{Matriz A = }\PYG{l+s+se}{\PYGZbs{}n}\PYG{l+s+s1}{\PYGZsq{}}\PYG{p}{,} \PYG{n}{A}\PYG{p}{)}
\PYG{n+nb}{print}\PYG{p}{(}\PYG{l+s+s1}{\PYGZsq{}}\PYG{l+s+s1}{Matriz L = }\PYG{l+s+se}{\PYGZbs{}n}\PYG{l+s+s1}{\PYGZsq{}}\PYG{p}{,} \PYG{n}{L}\PYG{p}{)}
\PYG{n+nb}{print}\PYG{p}{(}\PYG{l+s+s1}{\PYGZsq{}}\PYG{l+s+s1}{Matriz L\PYGZca{}T = }\PYG{l+s+se}{\PYGZbs{}n}\PYG{l+s+s1}{\PYGZsq{}}\PYG{p}{,} \PYG{n}{L}\PYG{o}{.}\PYG{n}{T}\PYG{p}{)}

\PYG{c+c1}{\PYGZsh{} prova real por produto interno}
\PYG{n}{A2} \PYG{o}{=} \PYG{n}{dot}\PYG{p}{(}\PYG{n}{L}\PYG{p}{,} \PYG{n}{L}\PYG{o}{.}\PYG{n}{T}\PYG{p}{)}
\PYG{n+nb}{print}\PYG{p}{(}\PYG{l+s+s1}{\PYGZsq{}}\PYG{l+s+s1}{Matriz LL\PYGZca{}T = }\PYG{l+s+se}{\PYGZbs{}n}\PYG{l+s+s1}{\PYGZsq{}}\PYG{p}{,} \PYG{n}{A2}\PYG{p}{)}

\PYG{c+c1}{\PYGZsh{} prova real usando norma de Frobenius da diferenca de matrizes}
\PYG{n+nb}{print}\PYG{p}{(}\PYG{l+s+s1}{\PYGZsq{}}\PYG{l+s+s1}{Norma || A \PYGZhy{} LL\PYGZca{}T || = }\PYG{l+s+s1}{\PYGZsq{}}\PYG{p}{,} \PYG{n}{linalg}\PYG{o}{.}\PYG{n}{norm}\PYG{p}{(}\PYG{n}{A}\PYG{o}{\PYGZhy{}}\PYG{n}{A2}\PYG{p}{)}\PYG{p}{)}

\PYG{n}{plt}\PYG{o}{.}\PYG{n}{matshow}\PYG{p}{(}\PYG{n}{A}\PYG{p}{,}\PYG{n}{cmap}\PYG{o}{=}\PYG{n}{plt}\PYG{o}{.}\PYG{n}{cm}\PYG{o}{.}\PYG{n}{flag}\PYG{p}{)}\PYG{p}{;}
\PYG{n}{plt}\PYG{o}{.}\PYG{n}{matshow}\PYG{p}{(}\PYG{n}{L}\PYG{p}{,}\PYG{n}{cmap}\PYG{o}{=}\PYG{n}{plt}\PYG{o}{.}\PYG{n}{cm}\PYG{o}{.}\PYG{n}{flag}\PYG{p}{)}\PYG{p}{;}
\PYG{n}{plt}\PYG{o}{.}\PYG{n}{matshow}\PYG{p}{(}\PYG{n}{L}\PYG{o}{.}\PYG{n}{T}\PYG{p}{,}\PYG{n}{cmap}\PYG{o}{=}\PYG{n}{plt}\PYG{o}{.}\PYG{n}{cm}\PYG{o}{.}\PYG{n}{flag}\PYG{p}{)}\PYG{p}{;}
\end{sphinxVerbatim}

\begin{sphinxVerbatim}[commandchars=\\\{\}]
Matriz A = 
 [[16 \PYGZhy{}4 12 \PYGZhy{}4]
 [\PYGZhy{}4  2 \PYGZhy{}1  1]
 [12 \PYGZhy{}1 14 \PYGZhy{}2]
 [\PYGZhy{}4  1 \PYGZhy{}2 83]]
Matriz L = 
 [[ 4.  0.  0.  0.]
 [\PYGZhy{}1.  1.  0.  0.]
 [ 3.  2.  1.  0.]
 [\PYGZhy{}1.  0.  1.  9.]]
Matriz L\PYGZca{}T = 
 [[ 4. \PYGZhy{}1.  3. \PYGZhy{}1.]
 [ 0.  1.  2.  0.]
 [ 0.  0.  1.  1.]
 [ 0.  0.  0.  9.]]
Matriz LL\PYGZca{}T = 
 [[16. \PYGZhy{}4. 12. \PYGZhy{}4.]
 [\PYGZhy{}4.  2. \PYGZhy{}1.  1.]
 [12. \PYGZhy{}1. 14. \PYGZhy{}2.]
 [\PYGZhy{}4.  1. \PYGZhy{}2. 83.]]
Norma || A \PYGZhy{} LL\PYGZca{}T || =  0.0
\end{sphinxVerbatim}

\noindent\sphinxincludegraphics{{aula-11-cholesky_12_1}.png}

\noindent\sphinxincludegraphics{{aula-11-cholesky_12_2}.png}

\noindent\sphinxincludegraphics{{aula-11-cholesky_12_3}.png}


\chapter{Método de Jacobi\sphinxhyphen{}Richardson}
\label{\detokenize{aula-12-jacobi:metodo-de-jacobi-richardson}}\label{\detokenize{aula-12-jacobi::doc}}
\sphinxAtStartPar
O método de Jacobi\sphinxhyphen{}Richardson (MJR) é um método iterativo que busca uma solução aproximada para sistemas lineares. Um método iterativo como tal é também conhecido como \sphinxstyleemphasis{aproximações sucessivas}, em que uma sequência convergente de vetores é desejada. O MJR é de especial interesse em sistemas cujas matrizes são \sphinxstyleemphasis{diagonalmente dominantes}.


\section{Instalações necessárias}
\label{\detokenize{aula-12-jacobi:instalacoes-necessarias}}
\sphinxAtStartPar
Este notebook usa o módulo \sphinxcode{\sphinxupquote{plotly}} para plotagem. Para instalá\sphinxhyphen{}lo via \sphinxcode{\sphinxupquote{conda}} e habilitá\sphinxhyphen{}lo para uso como extensão do Jupyter Lab, execute os comandos abaixo em uma célula:

\begin{sphinxVerbatim}[commandchars=\\\{\}]
import sys
!conda install \PYGZhy{}\PYGZhy{}yes \PYGZhy{}\PYGZhy{}prefix \PYGZob{}sys.prefix\PYGZcb{} nodejs plotly
!jupyter labextension install jupyterlab\PYGZhy{}plotly@4.12.0
\end{sphinxVerbatim}


\section{Descrição do método}
\label{\detokenize{aula-12-jacobi:descricao-do-metodo}}
\sphinxAtStartPar
Suponha um sistema linear nas incógnitas \(x_1, x_2, \dots, x_n\) da seguinte forma:
\begin{equation*}
\begin{split}a_{11} x_{1} + a_{12} x_{2} + \dots + a_{1n} x_{n} = b_1 \\
a_{21} x_{1} + a_{22} x_{2} + \dots + a_{2n} x_{n} = b_2 \\
. \\
. \\
. \\
a_{n1} x_{1} + a_{n2} x_{2} + \dots + a_{nn} x_{n} = b_n\end{split}
\end{equation*}
\sphinxAtStartPar
Suponha também que todos os termos \(a_{ii}\) sejam diferentes de zero \((i = 1, . . . , n)\). Se não for o caso, isso pode, geralmente, ser resolvido permutando equações.

\sphinxAtStartPar
Inicialmente, o método sugere que as variáveis sejam isoladas em cada equação. Assim, escrevemos
\begin{equation*}
\begin{split}
x_1 = \dfrac{1}{a_{11}} [b_1 - a_{12} x_{2} - a_{13} x_{3} - \dots - a_{1n} x_{n}] \\
x_2 = \dfrac{1}{a_{22}} [b_2 - a_{21} x_{1} - a_{23} x_{3} - \dots - a_{2n} x_{n}] \\
. \\
. \\
. \\
x_n = \dfrac{1}{a_{nn}} [b_n - a_{n1} x_{1} - a_{n2} x_{2} - \dots - a_{n \ n-1} x_{n-1}]
\end{split}
\end{equation*}
\sphinxAtStartPar
A partir dessas equações “isoladas” e de um vetor de estimativa inicial (“chute inicial”) \([x^{(0)}_1 \ \ x^{(0)}_2 \ \ \dots \ \  x^{(0)}_n]^T\), criamos o processo iterativo inserindo, à direita, um contador.

\sphinxAtStartPar
No primeiro passo, obtemos \([x^{(1)}_1 \ \ x^{(1)}_2 \ \ \dots \ \ x^{(1)}_n]^T\). Em seguida, estimamos \([x^{(2)}_1 \ \ x^{(2)}_2 \ \  \dots \ \  x^{(2)}_n]^T\). Repetindo o processo, a iterada \(k\) é construída como segue:
\begin{equation*}
\begin{split}
x^{(k+1)}_1 = \dfrac{1}{a_{11}} [b_1 - a_{12} x^{(k)}_{2} - a_{13} x^{(k)}_{3} - \dots - a_{1n} x^{(k)}_{n}] \\
x^{(k+1)}_2 = \dfrac{1}{a_{22}} [b_2 - a_{21} x^{(k)}_{1} - a_{23} x^{(k)}_{3} - \dots - a_{2n} x^{(k)}_{n}] \\
. \\
. \\
. \\
x^{(k+1)}_n = \dfrac{1}{a_{nn}} [b_n - a_{n1} x^{(k)}_{1} - a_{n2} x^{(k)}_{2} - \dots - a_{n \ n-1} x^{(k)}_{n-1}]
\end{split}
\end{equation*}
\sphinxAtStartPar
Logo, para todo \(i = 1, 2, \dots , n\), esperamos que a sequência \(\{ x^{(k+1)}_i \}_k\) convirja para o vetor solução \({\bf x}\) do sistema original, caracterizado por \({\bf A}{\bf x} = {\bf b}\).

\sphinxAtStartPar
Entretanto, a convergência é garantida apenas se houver dominância dos valores da diagonal principal sobre seus pares nas mesmas linhas. Podemos verificar a convergência do processo iterativo acima por meio do \sphinxstyleemphasis{critério das linhas}, embora exista uma maneira mais versátil de fazer esta verificação usando normas de matrizes. Por enquanto, entendamos o critério das linhas.


\section{Critério das linhas}
\label{\detokenize{aula-12-jacobi:criterio-das-linhas}}
\sphinxAtStartPar
O critério das linhas estabelece que
\begin{equation*}
\begin{split}\sum_{\substack{j = 1 \\ j \neq i}} ^n |a_{ij}| < |a_{ii}|, \forall i = 1, 2, \dots , n.\end{split}
\end{equation*}
\sphinxAtStartPar
Em palavras: \sphinxstyleemphasis{“o valor absoluto do termo diagonal na linha \(i\) é maior do que a soma dos valores absolutos de todos os outros termos na mesma linha”}. Se satisfeita, esta equação implica que a matriz \({\bf A}\) é (estritamente) diagonalmente dominante.

\sphinxAtStartPar
É importante observar que o critério das linhas pode deixar de ser satisfeito se houver troca na ordem das equações, e vice\sphinxhyphen{}versa. Todavia, uma troca cuidadosa pode fazer com que o sistema passe a satisfazer o critério. Por teorema, admite\sphinxhyphen{}se que se a matriz do sistema linear satisfaz o critério das linhas, então o algoritmo de Jacobi\sphinxhyphen{}Richardson converge para uma solução aproximada.


\section{Vantagens e desvantagens do método}
\label{\detokenize{aula-12-jacobi:vantagens-e-desvantagens-do-metodo}}
\sphinxAtStartPar
O método tem a vantagem de ser de fácil implementação no computador do que o método de escalonamento e está menos propenso à propagação de erros de arredondamento. Por outro lado, sua desvantagem é não funcionar para todos os casos.


\section{Implementação do método de Jacobi}
\label{\detokenize{aula-12-jacobi:implementacao-do-metodo-de-jacobi}}
\begin{sphinxVerbatim}[commandchars=\\\{\}]
\PYG{l+s+sd}{\PYGZdq{}\PYGZdq{}\PYGZdq{}}
\PYG{l+s+sd}{JACOBI: metodo de Jacobi para solução do sistema Ax=b.}

\PYG{l+s+sd}{x = jacobi(A,b,x0,N)}

\PYG{l+s+sd}{entrada:}
\PYG{l+s+sd}{   A: matriz n x n (tipo: lista de listas)}
\PYG{l+s+sd}{   b:  vetor n x 1 (tipo: lista)}
\PYG{l+s+sd}{  x0:  vetor n x 1, ponto de partida (tipo: lista)}
\PYG{l+s+sd}{   N:  numero de iterações do método}
\PYG{l+s+sd}{saidas:}
\PYG{l+s+sd}{x:  matriz n x N,}
\PYG{l+s+sd}{    contendo a sequencia de aproximações da solução}
\PYG{l+s+sd}{sp:  norma de B. O metodo converge se sp \PYGZlt{} 1.}
\PYG{l+s+sd}{\PYGZdq{}\PYGZdq{}\PYGZdq{}} 
\PYG{k+kn}{import} \PYG{n+nn}{numpy} \PYG{k}{as} \PYG{n+nn}{np}
\PYG{k+kn}{from} \PYG{n+nn}{matplotlib}\PYG{n+nn}{.}\PYG{n+nn}{pyplot} \PYG{k+kn}{import} \PYG{n}{plot}

\PYG{l+s+sd}{\PYGZdq{}\PYGZdq{}\PYGZdq{}}
\PYG{l+s+sd}{JACOBI: metodo de Jacobi para solução do sistema Ax=b.}

\PYG{l+s+sd}{x = jacobi(A,b,x0,N)}

\PYG{l+s+sd}{entrada:}
\PYG{l+s+sd}{   A: matriz n x n (tipo: lista de listas)}
\PYG{l+s+sd}{   b:  vetor n x 1 (tipo: lista)}
\PYG{l+s+sd}{  x0:  vetor n x 1, ponto de partida (tipo: lista)}
\PYG{l+s+sd}{   N:  numero de iterações do método}
\PYG{l+s+sd}{saidas:}
\PYG{l+s+sd}{x:  matriz n x N,}
\PYG{l+s+sd}{    contendo a sequencia de aproximações da solução}
\PYG{l+s+sd}{v:  norma de G. O metodo converge se v \PYGZlt{} 1.}
\PYG{l+s+sd}{\PYGZdq{}\PYGZdq{}\PYGZdq{}} 
\PYG{k}{def} \PYG{n+nf}{jacobi}\PYG{p}{(}\PYG{n}{A}\PYG{p}{,}\PYG{n}{b}\PYG{p}{,}\PYG{n}{x0}\PYG{p}{,}\PYG{n}{N}\PYG{p}{)}\PYG{p}{:}    
    
    \PYG{n}{f} \PYG{o}{=} \PYG{k}{lambda} \PYG{n}{obj}\PYG{p}{:} \PYG{n+nb}{isinstance}\PYG{p}{(}\PYG{n}{obj}\PYG{p}{,}\PYG{n+nb}{list}\PYG{p}{)}     
    \PYG{k}{if} \PYG{o+ow}{not} \PYG{n+nb}{all}\PYG{p}{(}\PYG{p}{[}\PYG{n}{f}\PYG{p}{(}\PYG{n}{A}\PYG{p}{)}\PYG{p}{,}\PYG{n}{f}\PYG{p}{(}\PYG{n}{b}\PYG{p}{)}\PYG{p}{,}\PYG{n}{f}\PYG{p}{(}\PYG{n}{x0}\PYG{p}{)}\PYG{p}{]}\PYG{p}{)}\PYG{p}{:}
        \PYG{k}{raise} \PYG{n+ne}{TypeError}\PYG{p}{(}\PYG{l+s+s1}{\PYGZsq{}}\PYG{l+s+s1}{A, b e x0 devem ser do tipo list.}\PYG{l+s+s1}{\PYGZsq{}}\PYG{p}{)}
    \PYG{k}{else}\PYG{p}{:}
        \PYG{n}{A} \PYG{o}{=} \PYG{n}{np}\PYG{o}{.}\PYG{n}{asarray}\PYG{p}{(}\PYG{n}{A}\PYG{p}{)}
        \PYG{n}{b} \PYG{o}{=} \PYG{n}{np}\PYG{o}{.}\PYG{n}{asarray}\PYG{p}{(}\PYG{n}{b}\PYG{p}{)}
        \PYG{n}{x0} \PYG{o}{=} \PYG{n}{np}\PYG{o}{.}\PYG{n}{asarray}\PYG{p}{(}\PYG{n}{x0}\PYG{p}{)}
            
    \PYG{n}{n} \PYG{o}{=} \PYG{n}{np}\PYG{o}{.}\PYG{n}{size}\PYG{p}{(}\PYG{n}{b}\PYG{p}{)}
    \PYG{n}{x} \PYG{o}{=} \PYG{n}{np}\PYG{o}{.}\PYG{n}{zeros}\PYG{p}{(}\PYG{p}{(}\PYG{n}{n}\PYG{p}{,}\PYG{n}{N}\PYG{p}{)}\PYG{p}{,}\PYG{n}{dtype}\PYG{o}{=}\PYG{n+nb}{float}\PYG{p}{)}
    \PYG{n}{x}\PYG{p}{[}\PYG{p}{:}\PYG{p}{,}\PYG{l+m+mi}{0}\PYG{p}{]} \PYG{o}{=} \PYG{n}{x0}
        
    \PYG{n}{P} \PYG{o}{=} \PYG{n}{np}\PYG{o}{.}\PYG{n}{diag}\PYG{p}{(}\PYG{n}{np}\PYG{o}{.}\PYG{n}{diag}\PYG{p}{(}\PYG{n}{A}\PYG{p}{)}\PYG{p}{)}
    \PYG{n}{Q} \PYG{o}{=} \PYG{n}{P}\PYG{o}{\PYGZhy{}}\PYG{n}{A} \PYG{c+c1}{\PYGZsh{} elementos de fora da diagonal}
    \PYG{n}{G} \PYG{o}{=} \PYG{n}{np}\PYG{o}{.}\PYG{n}{linalg}\PYG{o}{.}\PYG{n}{solve}\PYG{p}{(}\PYG{n}{P}\PYG{p}{,}\PYG{n}{Q}\PYG{p}{)} \PYG{c+c1}{\PYGZsh{} matriz da função de iteração}
    \PYG{n}{c} \PYG{o}{=} \PYG{n}{np}\PYG{o}{.}\PYG{n}{linalg}\PYG{o}{.}\PYG{n}{solve}\PYG{p}{(}\PYG{n}{P}\PYG{p}{,}\PYG{n}{b}\PYG{p}{)} \PYG{c+c1}{\PYGZsh{} vetor da função de iteração}
    \PYG{n}{v} \PYG{o}{=} \PYG{n}{np}\PYG{o}{.}\PYG{n}{linalg}\PYG{o}{.}\PYG{n}{norm}\PYG{p}{(}\PYG{n}{G}\PYG{p}{)}
    
    \PYG{c+c1}{\PYGZsh{}processo iterativo}
    \PYG{n}{X} \PYG{o}{=} \PYG{n}{x0}\PYG{p}{[}\PYG{p}{:}\PYG{p}{]}    
    \PYG{n}{j} \PYG{o}{=} \PYG{l+m+mi}{1}
    \PYG{k}{for} \PYG{n}{j} \PYG{o+ow}{in} \PYG{n+nb}{range}\PYG{p}{(}\PYG{l+m+mi}{1}\PYG{p}{,}\PYG{n}{N}\PYG{p}{)}\PYG{p}{:}        
        \PYG{n}{X} \PYG{o}{=} \PYG{n}{G}\PYG{o}{.}\PYG{n}{dot}\PYG{p}{(}\PYG{n}{X}\PYG{p}{)} \PYG{o}{+} \PYG{n}{c}        
        \PYG{n}{x}\PYG{p}{[}\PYG{p}{:}\PYG{p}{,}\PYG{n}{j}\PYG{p}{]} \PYG{o}{=} \PYG{n}{X}                
        
    \PYG{k}{return} \PYG{n}{x}\PYG{p}{,}\PYG{n}{v}
\end{sphinxVerbatim}

\begin{sphinxVerbatim}[commandchars=\\\{\}]
\PYG{c+c1}{\PYGZsh{} Exemplo 15.2}
\PYG{n}{A} \PYG{o}{=} \PYG{p}{[}\PYG{p}{[}\PYG{l+m+mi}{10}\PYG{p}{,}\PYG{l+m+mi}{2}\PYG{p}{,}\PYG{l+m+mi}{1}\PYG{p}{]}\PYG{p}{,}\PYG{p}{[}\PYG{l+m+mi}{1}\PYG{p}{,}\PYG{l+m+mi}{5}\PYG{p}{,}\PYG{l+m+mi}{1}\PYG{p}{]}\PYG{p}{,}\PYG{p}{[}\PYG{l+m+mi}{2}\PYG{p}{,}\PYG{l+m+mi}{3}\PYG{p}{,}\PYG{l+m+mi}{10}\PYG{p}{]}\PYG{p}{]}
\PYG{n}{b} \PYG{o}{=} \PYG{p}{[}\PYG{l+m+mi}{7}\PYG{p}{,}\PYG{o}{\PYGZhy{}}\PYG{l+m+mi}{8}\PYG{p}{,}\PYG{l+m+mi}{6}\PYG{p}{]}
\PYG{n}{x} \PYG{o}{=} \PYG{p}{[}\PYG{l+m+mi}{1}\PYG{p}{,}\PYG{l+m+mi}{1}\PYG{p}{,}\PYG{l+m+mi}{1}\PYG{p}{]}
\PYG{n}{N} \PYG{o}{=} \PYG{l+m+mi}{10}

\PYG{n}{sol}\PYG{p}{,}\PYG{n}{v} \PYG{o}{=} \PYG{n}{jacobi}\PYG{p}{(}\PYG{n}{A}\PYG{p}{,}\PYG{n}{b}\PYG{p}{,}\PYG{n}{x}\PYG{p}{,}\PYG{n}{N}\PYG{p}{)}
\PYG{n+nb}{print}\PYG{p}{(}\PYG{n}{sol}\PYG{p}{[}\PYG{p}{:}\PYG{p}{,}\PYG{o}{\PYGZhy{}}\PYG{l+m+mi}{1}\PYG{p}{]}\PYG{p}{)}
\PYG{n+nb}{print}\PYG{p}{(}\PYG{n}{v}\PYG{p}{)}
\end{sphinxVerbatim}

\begin{sphinxVerbatim}[commandchars=\\\{\}]
[ 0.99977882 \PYGZhy{}2.00026225  0.9996773 ]
0.5099019513592785
\end{sphinxVerbatim}


\subsection{Tarefa para turma de computação:}
\label{\detokenize{aula-12-jacobi:tarefa-para-turma-de-computacao}}
\sphinxAtStartPar
(Este código depende da biblioteca \sphinxcode{\sphinxupquote{plotly}}, mas não \sphinxstyleemphasis{está otimizado}.)

\sphinxAtStartPar
Desenvolver código para plotagem 3D da convergência para o método de Jacobi.

\begin{sphinxVerbatim}[commandchars=\\\{\}]
\PYG{c+c1}{\PYGZsh{} plotagem 3D da convergência }
\PYG{k+kn}{import} \PYG{n+nn}{plotly}\PYG{n+nn}{.}\PYG{n+nn}{graph\PYGZus{}objs} \PYG{k}{as} \PYG{n+nn}{go}
\PYG{k+kn}{from} \PYG{n+nn}{plotly}\PYG{n+nn}{.}\PYG{n+nn}{offline} \PYG{k+kn}{import} \PYG{n}{iplot}\PYG{p}{,} \PYG{n}{init\PYGZus{}notebook\PYGZus{}mode}
\PYG{n}{init\PYGZus{}notebook\PYGZus{}mode}\PYG{p}{(}\PYG{n}{connected}\PYG{o}{=}\PYG{k+kc}{False}\PYG{p}{)}

\PYG{n}{xp} \PYG{o}{=} \PYG{n}{sol}\PYG{p}{[}\PYG{l+m+mi}{0}\PYG{p}{,}\PYG{p}{:}\PYG{p}{]}
\PYG{n}{yp} \PYG{o}{=} \PYG{n}{sol}\PYG{p}{[}\PYG{l+m+mi}{1}\PYG{p}{,}\PYG{p}{:}\PYG{p}{]}
\PYG{n}{zp} \PYG{o}{=} \PYG{n}{sol}\PYG{p}{[}\PYG{l+m+mi}{2}\PYG{p}{,}\PYG{p}{:}\PYG{p}{]}

\PYG{n}{points} \PYG{o}{=} \PYG{n}{go}\PYG{o}{.}\PYG{n}{Scatter3d}\PYG{p}{(}\PYG{n}{x} \PYG{o}{=} \PYG{n}{xp}\PYG{p}{,} \PYG{n}{y} \PYG{o}{=} \PYG{n}{yp}\PYG{p}{,} \PYG{n}{z} \PYG{o}{=} \PYG{n}{zp}\PYG{p}{,} \PYG{n}{mode} \PYG{o}{=} \PYG{l+s+s1}{\PYGZsq{}}\PYG{l+s+s1}{markers}\PYG{l+s+s1}{\PYGZsq{}}\PYG{p}{,} \PYG{n}{marker} \PYG{o}{=} \PYG{n+nb}{dict}\PYG{p}{(}\PYG{n}{size} \PYG{o}{=} \PYG{l+m+mf}{0.1}\PYG{p}{,}\PYG{n}{color} \PYG{o}{=} \PYG{l+s+s2}{\PYGZdq{}}\PYG{l+s+s2}{rgb(227,26,28)}\PYG{l+s+s2}{\PYGZdq{}}\PYG{p}{)}\PYG{p}{)}

\PYG{n}{vecs} \PYG{o}{=} \PYG{p}{[}\PYG{p}{]}
\PYG{k}{for} \PYG{n}{i} \PYG{o+ow}{in} \PYG{n+nb}{range}\PYG{p}{(}\PYG{n+nb}{len}\PYG{p}{(}\PYG{n}{xp}\PYG{p}{)}\PYG{p}{)}\PYG{p}{:}
    \PYG{n}{v} \PYG{o}{=} \PYG{n}{go}\PYG{o}{.}\PYG{n}{Scatter3d}\PYG{p}{(} \PYG{n}{x} \PYG{o}{=} \PYG{p}{[}\PYG{l+m+mi}{0}\PYG{p}{,}\PYG{n}{xp}\PYG{p}{[}\PYG{n}{i}\PYG{p}{]}\PYG{p}{]}\PYG{p}{,} \PYG{n}{y} \PYG{o}{=} \PYG{p}{[}\PYG{l+m+mi}{0}\PYG{p}{,}\PYG{n}{yp}\PYG{p}{[}\PYG{n}{i}\PYG{p}{]}\PYG{p}{]}\PYG{p}{,} \PYG{n}{z} \PYG{o}{=} \PYG{p}{[}\PYG{l+m+mi}{0}\PYG{p}{,}\PYG{n}{zp}\PYG{p}{[}\PYG{n}{i}\PYG{p}{]}\PYG{p}{]}\PYG{p}{,} \PYG{n}{marker} \PYG{o}{=} \PYG{n+nb}{dict}\PYG{p}{(}\PYG{n}{size} \PYG{o}{=} \PYG{l+m+mf}{0.1}\PYG{p}{,} \PYG{n}{color} \PYG{o}{=} \PYG{l+s+s2}{\PYGZdq{}}\PYG{l+s+s2}{rgb(0,255,0)}\PYG{l+s+s2}{\PYGZdq{}}\PYG{p}{)}\PYG{p}{,}
                       \PYG{n}{line} \PYG{o}{=} \PYG{n+nb}{dict}\PYG{p}{(} \PYG{n}{color} \PYG{o}{=} \PYG{l+s+s2}{\PYGZdq{}}\PYG{l+s+s2}{rgb(0,255,0)}\PYG{l+s+s2}{\PYGZdq{}}\PYG{p}{,} \PYG{n}{width} \PYG{o}{=} \PYG{l+m+mi}{4}\PYG{p}{)} \PYG{p}{)}
    \PYG{n}{vecs}\PYG{o}{.}\PYG{n}{append}\PYG{p}{(}\PYG{n}{v}\PYG{p}{)}


\PYG{n}{cx} \PYG{o}{=} \PYG{n}{np}\PYG{o}{.}\PYG{n}{sum}\PYG{p}{(}\PYG{n}{xp}\PYG{p}{)}\PYG{o}{/}\PYG{n}{np}\PYG{o}{.}\PYG{n}{size}\PYG{p}{(}\PYG{n}{xp}\PYG{p}{)}
\PYG{n}{cy} \PYG{o}{=} \PYG{n}{np}\PYG{o}{.}\PYG{n}{sum}\PYG{p}{(}\PYG{n}{yp}\PYG{p}{)}\PYG{o}{/}\PYG{n}{np}\PYG{o}{.}\PYG{n}{size}\PYG{p}{(}\PYG{n}{yp}\PYG{p}{)}
\PYG{n}{cz} \PYG{o}{=} \PYG{n}{np}\PYG{o}{.}\PYG{n}{sum}\PYG{p}{(}\PYG{n}{zp}\PYG{p}{)}\PYG{o}{/}\PYG{n}{np}\PYG{o}{.}\PYG{n}{size}\PYG{p}{(}\PYG{n}{zp}\PYG{p}{)}

\PYG{n}{vector} \PYG{o}{=} \PYG{n}{go}\PYG{o}{.}\PYG{n}{Scatter3d}\PYG{p}{(} \PYG{n}{x} \PYG{o}{=} \PYG{p}{[}\PYG{l+m+mi}{0}\PYG{p}{,}\PYG{n}{cx}\PYG{p}{]}\PYG{p}{,} \PYG{n}{y} \PYG{o}{=} \PYG{p}{[}\PYG{l+m+mi}{0}\PYG{p}{,}\PYG{n}{cy}\PYG{p}{]}\PYG{p}{,} \PYG{n}{z} \PYG{o}{=} \PYG{p}{[}\PYG{l+m+mi}{0}\PYG{p}{,}\PYG{n}{cz}\PYG{p}{]}\PYG{p}{,} \PYG{n}{marker} \PYG{o}{=} \PYG{n+nb}{dict}\PYG{p}{(}\PYG{n}{size} \PYG{o}{=} \PYG{l+m+mi}{1}\PYG{p}{,} \PYG{n}{color} \PYG{o}{=} \PYG{l+s+s2}{\PYGZdq{}}\PYG{l+s+s2}{rgb(100,100,100)}\PYG{l+s+s2}{\PYGZdq{}}\PYG{p}{)}\PYG{p}{,}
                       \PYG{n}{line} \PYG{o}{=} \PYG{n+nb}{dict}\PYG{p}{(} \PYG{n}{color} \PYG{o}{=} \PYG{l+s+s2}{\PYGZdq{}}\PYG{l+s+s2}{rgb(100,100,100)}\PYG{l+s+s2}{\PYGZdq{}}\PYG{p}{,} \PYG{n}{width} \PYG{o}{=} \PYG{l+m+mi}{4}\PYG{p}{)} \PYG{p}{)}

\PYG{n}{data} \PYG{o}{=} \PYG{p}{[}\PYG{n}{points}\PYG{p}{,}\PYG{n}{vector}\PYG{p}{]} \PYG{o}{+} \PYG{n}{vecs}
\PYG{n}{layout} \PYG{o}{=} \PYG{n}{go}\PYG{o}{.}\PYG{n}{Layout}\PYG{p}{(}\PYG{n}{margin} \PYG{o}{=} \PYG{n+nb}{dict}\PYG{p}{(} \PYG{n}{l} \PYG{o}{=} \PYG{l+m+mi}{0}\PYG{p}{,}\PYG{n}{r} \PYG{o}{=} \PYG{l+m+mi}{0}\PYG{p}{,} \PYG{n}{b} \PYG{o}{=} \PYG{l+m+mi}{0}\PYG{p}{,} \PYG{n}{t} \PYG{o}{=} \PYG{l+m+mi}{0}\PYG{p}{)}\PYG{p}{)}
\PYG{n}{fig} \PYG{o}{=} \PYG{n}{go}\PYG{o}{.}\PYG{n}{Figure}\PYG{p}{(}\PYG{n}{data}\PYG{o}{=}\PYG{n}{data}\PYG{p}{,}\PYG{n}{layout}\PYG{o}{=}\PYG{n}{layout}\PYG{p}{)}


\PYG{c+c1}{\PYGZsh{} vector}

\PYG{n}{fig}\PYG{o}{.}\PYG{n}{add\PYGZus{}cone}\PYG{p}{(}\PYG{n}{x}\PYG{o}{=}\PYG{p}{[}\PYG{n}{cx}\PYG{p}{]}\PYG{p}{,}\PYG{n}{y}\PYG{o}{=}\PYG{p}{[}\PYG{n}{cy}\PYG{p}{]}\PYG{p}{,}\PYG{n}{z}\PYG{o}{=}\PYG{p}{[}\PYG{n}{cz}\PYG{p}{]}\PYG{p}{,}\PYG{n}{u}\PYG{o}{=}\PYG{p}{[}\PYG{n}{cx}\PYG{p}{]}\PYG{p}{,}\PYG{n}{v}\PYG{o}{=}\PYG{p}{[}\PYG{n}{cy}\PYG{p}{]}\PYG{p}{,}\PYG{n}{w}\PYG{o}{=}\PYG{p}{[}\PYG{n}{cz}\PYG{p}{]}\PYG{p}{,}\PYG{n}{sizeref}\PYG{o}{=}\PYG{l+m+mf}{0.1}\PYG{p}{,}\PYG{n}{anchor}\PYG{o}{=}\PYG{l+s+s1}{\PYGZsq{}}\PYG{l+s+s1}{tip}\PYG{l+s+s1}{\PYGZsq{}}\PYG{p}{,}\PYG{n}{colorscale}\PYG{o}{=}\PYG{l+s+s1}{\PYGZsq{}}\PYG{l+s+s1}{gray}\PYG{l+s+s1}{\PYGZsq{}}\PYG{p}{)}

\PYG{k}{for} \PYG{n}{i} \PYG{o+ow}{in} \PYG{n+nb}{range}\PYG{p}{(}\PYG{n+nb}{len}\PYG{p}{(}\PYG{n}{xp}\PYG{p}{)}\PYG{p}{)}\PYG{p}{:}
    \PYG{n}{fig}\PYG{o}{.}\PYG{n}{add\PYGZus{}cone}\PYG{p}{(}\PYG{n}{x}\PYG{o}{=}\PYG{p}{[}\PYG{n}{xp}\PYG{p}{[}\PYG{n}{i}\PYG{p}{]}\PYG{p}{]}\PYG{p}{,}\PYG{n}{y}\PYG{o}{=}\PYG{p}{[}\PYG{n}{yp}\PYG{p}{[}\PYG{n}{i}\PYG{p}{]}\PYG{p}{]}\PYG{p}{,}\PYG{n}{z}\PYG{o}{=}\PYG{p}{[}\PYG{n}{zp}\PYG{p}{[}\PYG{n}{i}\PYG{p}{]}\PYG{p}{]}\PYG{p}{,}\PYG{n}{u}\PYG{o}{=}\PYG{p}{[}\PYG{n}{xp}\PYG{p}{[}\PYG{n}{i}\PYG{p}{]}\PYG{p}{]}\PYG{p}{,}\PYG{n}{v}\PYG{o}{=}\PYG{p}{[}\PYG{n}{yp}\PYG{p}{[}\PYG{n}{i}\PYG{p}{]}\PYG{p}{]}\PYG{p}{,}\PYG{n}{w}\PYG{o}{=}\PYG{p}{[}\PYG{n}{zp}\PYG{p}{[}\PYG{n}{i}\PYG{p}{]}\PYG{p}{]}\PYG{p}{,}\PYG{n}{sizeref}\PYG{o}{=}\PYG{l+m+mf}{0.1}\PYG{p}{,}\PYG{n}{anchor}\PYG{o}{=}\PYG{l+s+s1}{\PYGZsq{}}\PYG{l+s+s1}{tip}\PYG{l+s+s1}{\PYGZsq{}}\PYG{p}{,}\PYG{n}{colorscale}\PYG{o}{=}\PYG{l+s+s1}{\PYGZsq{}}\PYG{l+s+s1}{jet}\PYG{l+s+s1}{\PYGZsq{}}\PYG{p}{)}


\PYG{n}{iplot}\PYG{p}{(}\PYG{n}{fig}\PYG{p}{,} \PYG{n}{show\PYGZus{}link}\PYG{o}{=}\PYG{k+kc}{True}\PYG{p}{,}\PYG{n}{filename}\PYG{o}{=}\PYG{l+s+s1}{\PYGZsq{}}\PYG{l+s+s1}{jacobi\PYGZhy{}3d\PYGZhy{}vectors}\PYG{l+s+s1}{\PYGZsq{}}\PYG{p}{)}
\end{sphinxVerbatim}


\chapter{Método de Newton para sistemas não\sphinxhyphen{}lineares}
\label{\detokenize{aula-13-newton-nao-linear:metodo-de-newton-para-sistemas-nao-lineares}}\label{\detokenize{aula-13-newton-nao-linear::doc}}
\sphinxAtStartPar
Este método determina, a cada iteração, a solução aproximada do sistema não\sphinxhyphen{}linear através de uma linearização das funções\sphinxhyphen{}alvo com a matriz Jacobiana associada ao sistema.


\section{Passos}
\label{\detokenize{aula-13-newton-nao-linear:passos}}
\sphinxAtStartPar
Para o método de Newton não\sphinxhyphen{}linear, basicamente criamos uma espécie de “caminho” onde somamos um vetor de deslocamento \({\bf s}\) às aproximações sucessivas que dá a direção para onde os vetores devem prosseguir a fim de atingir convergência.

\sphinxAtStartPar
\sphinxstylestrong{Obs.:} este processo iterativo usa critérios de parada naturais em algoritmos iterativos.

\sphinxAtStartPar
Para encontrarmos o vetor solução, devemos resolver a equação matricial linearizada
\begin{equation*}
\begin{split}{\bf J}({\bf x}^{(i)}){\bf s}^{(i)} = - {\bf F}({\bf x}^{(i)})\end{split}
\end{equation*}
\sphinxAtStartPar
Em seguida, atualizamos o novo vetor da sequencia como:
\begin{equation*}
\begin{split}{\bf x}^{(i+1)}={\bf x}^{(i)} + {\bf s}^{(i)}.\end{split}
\end{equation*}
\sphinxAtStartPar
Acima, \({\bf J}({\bf x}^{(i)})\) é a matriz Jacobiana formada a partir das derivadas parciais das funções componentes do vetor \({\bf F}\).

\sphinxAtStartPar
No caso de um sistema em que \({\bf F} = [f_1(x_1,x_2) \ \  f_2(x_1,x_2)]^T\), teríamos o sistema abaixo:
\begin{equation*}
\begin{split}
\begin{bmatrix}
\frac{\partial f_1(x_1,x_2)}{\partial x_1}  & \frac{\partial f_1(x_1,x_2)}{\partial x_2} \\
\frac{\partial f_2(x_1,x_2)}{\partial x_1}  & \frac{\partial f_2(x_1,x_2)}{\partial x_2} \\
\end{bmatrix}
\begin{bmatrix}
s_0 \\
s_1 \\
\end{bmatrix}
=
\begin{bmatrix}
-f_1(x_1,x_2)\\
-f_2(x_1,x_2) \\
\end{bmatrix}
\end{split}
\end{equation*}
\begin{sphinxVerbatim}[commandchars=\\\{\}]
\PYG{o}{\PYGZpc{}}\PYG{k}{matplotlib} inline
\PYG{k+kn}{import} \PYG{n+nn}{numpy} \PYG{k}{as} \PYG{n+nn}{np}
\PYG{k+kn}{import} \PYG{n+nn}{matplotlib}\PYG{n+nn}{.}\PYG{n+nn}{pyplot} \PYG{k}{as} \PYG{n+nn}{plt}
\PYG{k+kn}{import} \PYG{n+nn}{sympy} \PYG{k}{as} \PYG{n+nn}{sy}
\PYG{k+kn}{from} \PYG{n+nn}{scipy}\PYG{n+nn}{.}\PYG{n+nn}{optimize} \PYG{k+kn}{import} \PYG{n}{root}
\end{sphinxVerbatim}

\sphinxAtStartPar
No exemplo a seguir, mostramos como podemos resolver um sistema de equações não\sphinxhyphen{}lineares usando o \sphinxcode{\sphinxupquote{scipy}}.

\sphinxAtStartPar
Procuramos as soluções para o sistema não\sphinxhyphen{}linear
\begin{equation*}
\begin{split}\begin{cases}f_1(x,y) &: x^2 + y^2 = 2 \\
f_2(x,y) &: x^2 - \frac{y^2}{9} = 1\end{cases}\end{split}
\end{equation*}
\sphinxAtStartPar
Vamos plotar o gráficos das funções:

\begin{sphinxVerbatim}[commandchars=\\\{\}]
\PYG{n}{x} \PYG{o}{=} \PYG{n}{np}\PYG{o}{.}\PYG{n}{linspace}\PYG{p}{(}\PYG{o}{\PYGZhy{}}\PYG{l+m+mi}{2}\PYG{p}{,}\PYG{l+m+mi}{2}\PYG{p}{,}\PYG{l+m+mi}{50}\PYG{p}{,}\PYG{n}{endpoint}\PYG{o}{=}\PYG{k+kc}{True}\PYG{p}{)}
\PYG{n}{y} \PYG{o}{=} \PYG{n}{x}\PYG{p}{[}\PYG{p}{:}\PYG{p}{]}
\PYG{n}{X}\PYG{p}{,}\PYG{n}{Y} \PYG{o}{=} \PYG{n}{np}\PYG{o}{.}\PYG{n}{meshgrid}\PYG{p}{(}\PYG{n}{x}\PYG{p}{,}\PYG{n}{y}\PYG{p}{)}

\PYG{c+c1}{\PYGZsh{} define funções para plotagem}
\PYG{n}{F1} \PYG{o}{=} \PYG{n}{X}\PYG{o}{*}\PYG{o}{*}\PYG{l+m+mi}{2} \PYG{o}{+} \PYG{n}{Y}\PYG{o}{*}\PYG{o}{*}\PYG{l+m+mi}{2} \PYG{o}{\PYGZhy{}} \PYG{l+m+mi}{2}
\PYG{n}{F2} \PYG{o}{=} \PYG{n}{X}\PYG{o}{*}\PYG{o}{*}\PYG{l+m+mi}{2} \PYG{o}{\PYGZhy{}} \PYG{n}{Y}\PYG{o}{*}\PYG{o}{*}\PYG{l+m+mi}{2}\PYG{o}{/}\PYG{l+m+mi}{9} \PYG{o}{\PYGZhy{}} \PYG{l+m+mi}{1}

\PYG{c+c1}{\PYGZsh{} curvas de nível}
\PYG{n}{C} \PYG{o}{=} \PYG{n}{plt}\PYG{o}{.}\PYG{n}{contour}\PYG{p}{(}\PYG{n}{X}\PYG{p}{,}\PYG{n}{Y}\PYG{p}{,}\PYG{n}{F1}\PYG{p}{,}\PYG{n}{levels}\PYG{o}{=}\PYG{p}{[}\PYG{l+m+mi}{0}\PYG{p}{]}\PYG{p}{,}\PYG{n}{colors}\PYG{o}{=}\PYG{l+s+s1}{\PYGZsq{}}\PYG{l+s+s1}{k}\PYG{l+s+s1}{\PYGZsq{}}\PYG{p}{)}
\PYG{n}{C} \PYG{o}{=} \PYG{n}{plt}\PYG{o}{.}\PYG{n}{contour}\PYG{p}{(}\PYG{n}{X}\PYG{p}{,}\PYG{n}{Y}\PYG{p}{,}\PYG{n}{F2}\PYG{p}{,}\PYG{n}{levels}\PYG{o}{=}\PYG{p}{[}\PYG{l+m+mi}{0}\PYG{p}{]}\PYG{p}{,}\PYG{n}{colors}\PYG{o}{=}\PYG{l+s+s1}{\PYGZsq{}}\PYG{l+s+s1}{g}\PYG{l+s+s1}{\PYGZsq{}}\PYG{p}{)}
\PYG{n}{plt}\PYG{o}{.}\PYG{n}{grid}\PYG{p}{(}\PYG{k+kc}{True}\PYG{p}{)}
\end{sphinxVerbatim}

\noindent\sphinxincludegraphics{{aula-13-newton-nao-linear_6_0}.png}

\sphinxAtStartPar
Pela figura, vemos que existem 4 pontos de interseção entre as curvas e, portanto, 4 soluções, as quais formam o conjunto
\begin{equation*}
\begin{split}S = \{(x_1^{*},y_1^{*}),(x_2^{*},y_2^{*}),(x_3^{*},y_3^{*}),(x_4^{*},y_4^{*}))\end{split}
\end{equation*}
\sphinxAtStartPar
Agora, vamos usar a função \sphinxcode{\sphinxupquote{root}} do \sphinxcode{\sphinxupquote{scipy}} para computar essas soluções com base em estimativas iniciais.

\begin{sphinxVerbatim}[commandchars=\\\{\}]
\PYG{c+c1}{\PYGZsh{} define função para o vetor F(x)}
\PYG{k}{def} \PYG{n+nf}{F}\PYG{p}{(}\PYG{n}{x}\PYG{p}{)}\PYG{p}{:}
    \PYG{k}{return} \PYG{p}{[} \PYG{n}{x}\PYG{p}{[}\PYG{l+m+mi}{0}\PYG{p}{]}\PYG{o}{*}\PYG{o}{*}\PYG{l+m+mi}{2} \PYG{o}{+} \PYG{n}{x}\PYG{p}{[}\PYG{l+m+mi}{1}\PYG{p}{]}\PYG{o}{*}\PYG{o}{*}\PYG{l+m+mi}{2} \PYG{o}{\PYGZhy{}} \PYG{l+m+mi}{2}\PYG{p}{,}
             \PYG{n}{x}\PYG{p}{[}\PYG{l+m+mi}{0}\PYG{p}{]}\PYG{o}{*}\PYG{o}{*}\PYG{l+m+mi}{2} \PYG{o}{\PYGZhy{}} \PYG{n}{x}\PYG{p}{[}\PYG{l+m+mi}{1}\PYG{p}{]}\PYG{o}{*}\PYG{o}{*}\PYG{l+m+mi}{2}\PYG{o}{/}\PYG{l+m+mi}{9} \PYG{o}{\PYGZhy{}} \PYG{l+m+mi}{1} \PYG{p}{]}


\PYG{n}{x}\PYG{p}{,}\PYG{n}{y} \PYG{o}{=} \PYG{n}{sy}\PYG{o}{.}\PYG{n}{symbols}\PYG{p}{(}\PYG{l+s+s1}{\PYGZsq{}}\PYG{l+s+s1}{x,y}\PYG{l+s+s1}{\PYGZsq{}}\PYG{p}{)}

\PYG{c+c1}{\PYGZsh{} usa computação simbólica para determinar a matriz Jacobiana}
\PYG{n}{f1} \PYG{o}{=} \PYG{n}{x}\PYG{o}{*}\PYG{o}{*}\PYG{l+m+mi}{2} \PYG{o}{+} \PYG{n}{y}\PYG{o}{*}\PYG{o}{*}\PYG{l+m+mi}{2} \PYG{o}{\PYGZhy{}} \PYG{l+m+mi}{2}
\PYG{n}{f2} \PYG{o}{=} \PYG{n}{x}\PYG{o}{*}\PYG{o}{*}\PYG{l+m+mi}{2} \PYG{o}{\PYGZhy{}} \PYG{n}{y}\PYG{o}{*}\PYG{o}{*}\PYG{l+m+mi}{2}\PYG{o}{/}\PYG{l+m+mi}{9} \PYG{o}{\PYGZhy{}} \PYG{l+m+mi}{1}

\PYG{c+c1}{\PYGZsh{} gradientes}

\PYG{n}{f1x}\PYG{p}{,}\PYG{n}{f1y} \PYG{o}{=} \PYG{n}{sy}\PYG{o}{.}\PYG{n}{diff}\PYG{p}{(}\PYG{n}{f1}\PYG{p}{,}\PYG{n}{x}\PYG{p}{)}\PYG{p}{,}\PYG{n}{sy}\PYG{o}{.}\PYG{n}{diff}\PYG{p}{(}\PYG{n}{f1}\PYG{p}{,}\PYG{n}{y}\PYG{p}{)}
\PYG{n}{f2x}\PYG{p}{,}\PYG{n}{f2y} \PYG{o}{=} \PYG{n}{sy}\PYG{o}{.}\PYG{n}{diff}\PYG{p}{(}\PYG{n}{f2}\PYG{p}{,}\PYG{n}{x}\PYG{p}{)}\PYG{p}{,}\PYG{n}{sy}\PYG{o}{.}\PYG{n}{diff}\PYG{p}{(}\PYG{n}{f2}\PYG{p}{,}\PYG{n}{y}\PYG{p}{)}

\PYG{c+c1}{\PYGZsh{} imprime derivadas parciais}
\PYG{n+nb}{print}\PYG{p}{(}\PYG{n}{f1x}\PYG{p}{)}
\PYG{n+nb}{print}\PYG{p}{(}\PYG{n}{f1y}\PYG{p}{)}
\PYG{n+nb}{print}\PYG{p}{(}\PYG{n}{f2x}\PYG{p}{)}
\PYG{n+nb}{print}\PYG{p}{(}\PYG{n}{f2y}\PYG{p}{)}

\PYG{c+c1}{\PYGZsh{} monta matriz Jacobiana}
\PYG{k}{def} \PYG{n+nf}{jacobian}\PYG{p}{(}\PYG{n}{x}\PYG{p}{)}\PYG{p}{:}
    \PYG{k}{return} \PYG{n}{np}\PYG{o}{.}\PYG{n}{array}\PYG{p}{(}\PYG{p}{[}\PYG{p}{[}\PYG{l+m+mi}{2}\PYG{o}{*}\PYG{n}{x}\PYG{p}{[}\PYG{l+m+mi}{0}\PYG{p}{]}\PYG{p}{,} \PYG{l+m+mi}{2}\PYG{o}{*}\PYG{n}{x}\PYG{p}{[}\PYG{l+m+mi}{1}\PYG{p}{]}\PYG{p}{]}\PYG{p}{,} \PYG{p}{[}\PYG{l+m+mi}{2}\PYG{o}{*}\PYG{n}{x}\PYG{p}{[}\PYG{l+m+mi}{0}\PYG{p}{]}\PYG{p}{,}\PYG{o}{\PYGZhy{}}\PYG{l+m+mi}{2}\PYG{o}{*}\PYG{n}{x}\PYG{p}{[}\PYG{l+m+mi}{1}\PYG{p}{]}\PYG{o}{/}\PYG{l+m+mi}{9}\PYG{p}{]}\PYG{p}{]}\PYG{p}{)}

\PYG{c+c1}{\PYGZsh{} resolve o sistema não\PYGZhy{}linear por algoritmo de Levenberg\PYGZhy{}Marqardt modificado}

\PYG{n}{inicial} \PYG{o}{=} \PYG{p}{[}\PYG{p}{[}\PYG{l+m+mi}{2}\PYG{p}{,}\PYG{l+m+mi}{2}\PYG{p}{]}\PYG{p}{,}\PYG{p}{[}\PYG{o}{\PYGZhy{}}\PYG{l+m+mi}{2}\PYG{p}{,}\PYG{l+m+mi}{2}\PYG{p}{]}\PYG{p}{,}\PYG{p}{[}\PYG{o}{\PYGZhy{}}\PYG{l+m+mi}{2}\PYG{p}{,}\PYG{o}{\PYGZhy{}}\PYG{l+m+mi}{2}\PYG{p}{]}\PYG{p}{,}\PYG{p}{[}\PYG{l+m+mi}{2}\PYG{p}{,}\PYG{o}{\PYGZhy{}}\PYG{l+m+mi}{2}\PYG{p}{]}\PYG{p}{]}

\PYG{n}{S} \PYG{o}{=} \PYG{p}{[}\PYG{p}{]}
\PYG{n}{i} \PYG{o}{=} \PYG{l+m+mi}{1}
\PYG{k}{for} \PYG{n}{vetor} \PYG{o+ow}{in} \PYG{n}{inicial}\PYG{p}{:} 
    \PYG{n}{aux} \PYG{o}{=} \PYG{n}{root}\PYG{p}{(}\PYG{n}{F}\PYG{p}{,}\PYG{n}{vetor}\PYG{p}{,}\PYG{n}{jac}\PYG{o}{=}\PYG{n}{jacobian}\PYG{p}{,} \PYG{n}{method}\PYG{o}{=}\PYG{l+s+s1}{\PYGZsq{}}\PYG{l+s+s1}{lm}\PYG{l+s+s1}{\PYGZsq{}}\PYG{p}{)}
    \PYG{n}{S}\PYG{o}{.}\PYG{n}{append}\PYG{p}{(}\PYG{n}{aux}\PYG{o}{.}\PYG{n}{x}\PYG{p}{)} 
    \PYG{n}{s} \PYG{o}{=} \PYG{l+s+s1}{\PYGZsq{}}\PYG{l+s+s1}{Solução x(}\PYG{l+s+si}{\PYGZob{}0\PYGZcb{}}\PYG{l+s+s1}{)* encontrada: }\PYG{l+s+si}{\PYGZob{}1\PYGZcb{}}\PYG{l+s+s1}{\PYGZsq{}}
    \PYG{n+nb}{print}\PYG{p}{(}\PYG{n}{s}\PYG{o}{.}\PYG{n}{format}\PYG{p}{(}\PYG{n}{i}\PYG{p}{,}\PYG{n}{aux}\PYG{o}{.}\PYG{n}{x}\PYG{p}{)}\PYG{p}{)}
    \PYG{n}{i} \PYG{o}{+}\PYG{o}{=}\PYG{l+m+mi}{1}
\end{sphinxVerbatim}

\begin{sphinxVerbatim}[commandchars=\\\{\}]
2*x
2*y
2*x
\PYGZhy{}2*y/9
Solução x(1)* encontrada: [1.04880885 0.9486833 ]
Solução x(2)* encontrada: [\PYGZhy{}1.04880885  0.9486833 ]
Solução x(3)* encontrada: [\PYGZhy{}1.04880885 \PYGZhy{}0.9486833 ]
Solução x(4)* encontrada: [ 1.04880885 \PYGZhy{}0.9486833 ]
\end{sphinxVerbatim}

\sphinxAtStartPar
Em seguida, vamos plotar as soluções e as curvas

\begin{sphinxVerbatim}[commandchars=\\\{\}]
\PYG{c+c1}{\PYGZsh{} curvas de nível}
\PYG{n}{C} \PYG{o}{=} \PYG{n}{plt}\PYG{o}{.}\PYG{n}{contour}\PYG{p}{(}\PYG{n}{X}\PYG{p}{,}\PYG{n}{Y}\PYG{p}{,}\PYG{n}{F1}\PYG{p}{,}\PYG{n}{levels}\PYG{o}{=}\PYG{p}{[}\PYG{l+m+mi}{0}\PYG{p}{]}\PYG{p}{,}\PYG{n}{colors}\PYG{o}{=}\PYG{l+s+s1}{\PYGZsq{}}\PYG{l+s+s1}{k}\PYG{l+s+s1}{\PYGZsq{}}\PYG{p}{)}
\PYG{n}{C} \PYG{o}{=} \PYG{n}{plt}\PYG{o}{.}\PYG{n}{contour}\PYG{p}{(}\PYG{n}{X}\PYG{p}{,}\PYG{n}{Y}\PYG{p}{,}\PYG{n}{F2}\PYG{p}{,}\PYG{n}{levels}\PYG{o}{=}\PYG{p}{[}\PYG{l+m+mi}{0}\PYG{p}{]}\PYG{p}{,}\PYG{n}{colors}\PYG{o}{=}\PYG{l+s+s1}{\PYGZsq{}}\PYG{l+s+s1}{g}\PYG{l+s+s1}{\PYGZsq{}}\PYG{p}{)}
\PYG{n}{plt}\PYG{o}{.}\PYG{n}{grid}\PYG{p}{(}\PYG{k+kc}{True}\PYG{p}{)}

\PYG{c+c1}{\PYGZsh{} imprime interseções}
\PYG{k}{for} \PYG{n}{i} \PYG{o+ow}{in} \PYG{n+nb}{range}\PYG{p}{(}\PYG{n+nb}{len}\PYG{p}{(}\PYG{n}{S}\PYG{p}{)}\PYG{p}{)}\PYG{p}{:}
        \PYG{n}{plt}\PYG{o}{.}\PYG{n}{plot}\PYG{p}{(}\PYG{n}{S}\PYG{p}{[}\PYG{n}{i}\PYG{p}{]}\PYG{p}{[}\PYG{l+m+mi}{0}\PYG{p}{]}\PYG{p}{,}\PYG{n}{S}\PYG{p}{[}\PYG{n}{i}\PYG{p}{]}\PYG{p}{[}\PYG{l+m+mi}{1}\PYG{p}{]}\PYG{p}{,}\PYG{l+s+s1}{\PYGZsq{}}\PYG{l+s+s1}{or}\PYG{l+s+s1}{\PYGZsq{}}\PYG{p}{)}
\end{sphinxVerbatim}

\noindent\sphinxincludegraphics{{aula-13-newton-nao-linear_10_0}.png}


\subsection{Exercício:}
\label{\detokenize{aula-13-newton-nao-linear:exercicio}}
\sphinxAtStartPar
Resolva os sistemas não\sphinxhyphen{}lineares da Lista de Exercícios 4 usando a mesma abordagem acima.


\chapter{Nota: Raízes de sistemas não\sphinxhyphen{}lineares}
\label{\detokenize{aula-13-newton-nao-linear:nota-raizes-de-sistemas-nao-lineares}}\begin{itemize}
\item {} 
\sphinxAtStartPar
Uma equação linear tem a forma:

\end{itemize}
\begin{equation*}
\begin{split}f(x) = a_1 x_1 + a_2 x_2 + \ldots + a_n x_n\end{split}
\end{equation*}\begin{itemize}
\item {} 
\sphinxAtStartPar
Uma equação não\sphinxhyphen{}linear possui “produtos de incógnitas”, e.g.

\end{itemize}
\begin{equation*}
\begin{split} f_2(x) = a_1 x_1 x_2 + a_2 x_2^2 + a_n x_nx_1 \end{split}
\end{equation*}\begin{itemize}
\item {} 
\sphinxAtStartPar
Um sistema de equações não\sphinxhyphen{}lineares é composto de várias equações não\sphinxhyphen{}lineares

\end{itemize}
\begin{equation*}
\begin{split} f_1(x_1,x_2,\ldots,x_n) = 0 \end{split}
\end{equation*}\begin{equation*}
\begin{split} f_2(x_1,x_2,\ldots,x_n) = 0 \end{split}
\end{equation*}\begin{equation*}
\begin{split} \vdots \end{split}
\end{equation*}\begin{equation*}
\begin{split} f_n(x_1,x_2,\ldots,x_n) = 0 \end{split}
\end{equation*}\begin{itemize}
\item {} 
\sphinxAtStartPar
A solução do sistema é o vetor \( (x_1^{*},x_2^{*},\ldots,x_n^{*}) \) que satisfaz as \( n \) equações simultaneamente.

\end{itemize}


\section{Iteração de Ponto Fixo para sistemas não\sphinxhyphen{}lineares}
\label{\detokenize{aula-13-newton-nao-linear:iteracao-de-ponto-fixo-para-sistemas-nao-lineares}}\begin{itemize}
\item {} 
\sphinxAtStartPar
Aplicar o algoritmo iterativo em cada componente:

\end{itemize}
\begin{equation*}
\begin{split} x_1^{i+1} = \tilde{f}_1(x_1^{i},x_2^{i},\ldots,x_n^{i}) \end{split}
\end{equation*}\begin{equation*}
\begin{split} x_2^{i+1} = \tilde{f}_2(x_1^{i},x_2^{i},\ldots,x_n^{i}) \end{split}
\end{equation*}\begin{equation*}
\begin{split} \vdots \end{split}
\end{equation*}\begin{equation*}
\begin{split} x_3^{i+1} = \tilde{f}_3(x_1^{i},x_2^{i},\ldots,x_n^{i}) \end{split}
\end{equation*}
\sphinxAtStartPar
As formas funcionais mudam porque devemos isolar a variável \( x_i \).

\sphinxAtStartPar
Exemplo: encontrar a raiz do sistema abaixo:
\begin{equation*}
\begin{split} f_1(x,y) = x^2 + xy - 10 = 0 \end{split}
\end{equation*}\begin{equation*}
\begin{split} f_2(x,y) = y + 3xy^2 - 57 = 0 \end{split}
\end{equation*}
\sphinxAtStartPar
Solução:

\sphinxAtStartPar
Reescrevamos as equações na forma
\begin{equation*}
\begin{split} x = \tilde{f}_1(x,y) = \sqrt{10 - xy} \end{split}
\end{equation*}\begin{equation*}
\begin{split} y = \tilde{f}_2(x,y) = \sqrt{\frac{57-y}{3x}} \end{split}
\end{equation*}
\sphinxAtStartPar
de onde temos a iteração de ponto fixo dada por
\begin{equation*}
\begin{split} x^{i+1} = \sqrt{10 - x^iy^i} \end{split}
\end{equation*}\begin{equation*}
\begin{split} y^{i+1} = \sqrt{\frac{57-y^i}{3x^i}}, \quad i = 0,1,2,\ldots, \end{split}
\end{equation*}
\sphinxAtStartPar
Usando \( (x^0,y^0) = (1.5,3.5) \) como “chute” inicial, computamos
\begin{equation*}
\begin{split} x^{1} = \sqrt{10 - x^0y^0} = \sqrt{10 - 1.5(3.5)} = 2.17945 \end{split}
\end{equation*}\begin{align*}\!\begin{aligned}
y^{1} = \sqrt{\frac{57-y^0}{3x^1}} = \sqrt{\frac{57-3.5}{3(2.17945)}} = 2.86051 $$ (o valor de $ x^1 $ pode ser usado diretamente em vez de $ x^0 $.)\\
$$ x^{2} = \sqrt{10 - x^1y^1} = \sqrt{10 - 2.17945(2.86051)} = 1.94053\\
\end{aligned}\end{align*}\begin{equation*}
\begin{split} y^{2} = \sqrt{\frac{57-y^1}{3x^2}} = \sqrt{\frac{57-2.86051}{3(1.94053)}} = 3.04955 \end{split}
\end{equation*}\begin{equation*}
\begin{split} \ldots \end{split}
\end{equation*}
\sphinxAtStartPar
O processo iterativo converge para a solução \( (x^{*},y^{*}) = (2,3) \).

\sphinxAtStartPar
\sphinxstylestrong{Notas:}
\begin{itemize}
\item {} 
\sphinxAtStartPar
A convergência por iteração de PF depende de como as equações \( \tilde{f}_1,\tilde{f}_2,\ldots,\tilde{f}_n \) são formuladas, bem como de um bom “chute” inicial.

\item {} 
\sphinxAtStartPar
A iteração de PF é bastante restritiva nas soluções de sistemas não\sphinxhyphen{}lineares.

\end{itemize}


\section{Newton\sphinxhyphen{}Raphson para sistema não\sphinxhyphen{}linear}
\label{\detokenize{aula-13-newton-nao-linear:newton-raphson-para-sistema-nao-linear}}\begin{itemize}
\item {} 
\sphinxAtStartPar
Depende de série de Taylor em \( n-\)dimensões.

\item {} 
\sphinxAtStartPar
Para 2 dimensões, por exemplo, o método de Newton\sphinxhyphen{}Raphson pode ser escrito como:

\end{itemize}
\begin{equation*}
\begin{split} u_{i+1} = u_i + (x_{i+1} - x_i)\frac{\partial u_i}{\partial x} +(y_{i+1} - y_i)\frac{\partial u_i}{\partial y} \end{split}
\end{equation*}\begin{equation*}
\begin{split} v_{i+1} = v_i + (x_{i+1} - x_i)\frac{\partial v_i}{\partial x} +(y_{i+1} - y_i)\frac{\partial v_i}{\partial y} \end{split}
\end{equation*}\begin{itemize}
\item {} 
\sphinxAtStartPar
A estimativa da raiz corresponde aos valores de \( x \) e \( y \) para os quais \( u_{i+1} = 0 \) e \( v_{i+1} = 0 \). Então,

\end{itemize}
\begin{equation*}
\begin{split} \frac{\partial u_i}{\partial x} x_{i+1} + 
   \frac{\partial u_i}{\partial y} y_{i+1} = 
   - u_i + \frac{\partial u_i}{\partial x}x_i 
         + \frac{\partial u_i}{\partial y}y_i  \end{split}
\end{equation*}\begin{equation*}
\begin{split} \frac{\partial v_i}{\partial x} x_{i+1} + 
   \frac{\partial v_i}{\partial y} y_{i+1} = 
   - v_i + \frac{\partial v_i}{\partial x}x_i 
         + \frac{\partial v_i}{\partial y}y_i,  \end{split}
\end{equation*}
\sphinxAtStartPar
que é um sistema nas incógnitas \( x_{i+1} \) e \( y_{i+1} \).
\begin{itemize}
\item {} 
\sphinxAtStartPar
Manipulações algébricas permitem solucionar este sistema (e.g. regra de Cramer):

\end{itemize}
\begin{equation*}
\begin{split} x_{i+1} = x_i - J^{-1}\left(u_i\frac{\partial v_i}{\partial y} - v_i\frac{\partial u_i}{\partial y}\right) \end{split}
\end{equation*}\begin{equation*}
\begin{split} y_{i+1} = y_i - J^{-1}\left(v_i\frac{\partial u_i}{\partial x} - u_i\frac{\partial v_i}{\partial x}\right), \end{split}
\end{equation*}
\sphinxAtStartPar
onde
\begin{equation*}
\begin{split} J = \frac{\partial u_i}{\partial x}\frac{\partial v_i}{\partial y} - \frac{\partial u_i}{\partial y}\frac{\partial v_i}{\partial x} \end{split}
\end{equation*}
\sphinxAtStartPar
é o determinante da matriz \sphinxstyleemphasis{Jacobiana} do sistema.

\sphinxAtStartPar
Exemplo: resolver o mesmo sistema do PF

\sphinxAtStartPar
Solução:
\begin{itemize}
\item {} 
\sphinxAtStartPar
Inicialmente, calculemos as derivadas parciais no ponto inicial:

\end{itemize}
\begin{equation*}
\begin{split} \frac{\partial u_0}{\partial x} = 2x_0 + y_0 = 2(1.5) + 3.5 = 6.5, \ \ \ \ \frac{\partial u_0}{\partial y} = x_0 = 1.5 \end{split}
\end{equation*}\begin{equation*}
\begin{split} \frac{\partial v_0}{\partial x} = 3y_0^2 = 3(3.5)^2 = 36.75, \ \ \ \ \frac{\partial v_0}{\partial y} = 1 + 6x_0y_0 = 1 + 6(1.5)(3.5) = 32.5 \end{split}
\end{equation*}\begin{itemize}
\item {} 
\sphinxAtStartPar
O determinante é \( J = 6.5(32.5) - 1.5(36.75) = 156.125 \).

\item {} 
\sphinxAtStartPar
Calculamos os valores da função no ponto inicial

\end{itemize}
\begin{equation*}
\begin{split} u(x_0,y_0) = u_0 = -2.5, \quad v(x_0,y_0) = v_0 = 1.625 \end{split}
\end{equation*}\begin{itemize}
\item {} 
\sphinxAtStartPar
Calculamos os valores no próximo passo, i.e., \( (x^1,y^1) \).

\end{itemize}
\begin{equation*}
\begin{split} x^1 = 1.5 - \frac{\cdots}{156.125} = 2.03603 \end{split}
\end{equation*}\begin{equation*}
\begin{split} y^1 = 3.5 - \frac{\cdots}{156.125} = 2.84388 \end{split}
\end{equation*}\begin{itemize}
\item {} 
\sphinxAtStartPar
O processo iterativo converge para a solução \( (x^{*},y^{*}) = (2,3) \).

\end{itemize}

\sphinxAtStartPar
\sphinxstyleemphasis{Ref.: Chapra \& Canale, sec. 6.6}

\begin{sphinxVerbatim}[commandchars=\\\{\}]
\PYG{o}{\PYGZpc{}}\PYG{k}{matplotlib} inline
\end{sphinxVerbatim}


\chapter{Polinômio Interpolador de Lagrange}
\label{\detokenize{aula-14-interpolacao-lagrange:polinomio-interpolador-de-lagrange}}\label{\detokenize{aula-14-interpolacao-lagrange::doc}}
\sphinxAtStartPar
O polinômio interpolador de Lagrange é representado por:
\begin{equation*}
\begin{split}P_n (x) = \sum _{i = 0}^{n} L_i (x) f(x_i)\end{split}
\end{equation*}
\sphinxAtStartPar
onde
\begin{equation*}
\begin{split}L_i(x) = \prod _{\substack{j = 0 \\ j \neq i}}^n \dfrac{x - x_j}{x_i - x_j}\end{split}
\end{equation*}
\sphinxAtStartPar
Por exemplo, a versão linear (\(n = 1\)) seria
\begin{equation*}
\begin{split}P_1(x) = \dfrac{x − x_1}{x_0 − x_1} f(x_0) + \dfrac{x − x_0}{x_1 − x_0} f(x_1) \end{split}
\end{equation*}
\sphinxAtStartPar
e a versão de segundo grau (\(n = 1\)) seria
\begin{equation*}
\begin{split}P_2(x) = \dfrac{(x − x_1)(x − x_2)}{(x_0 − x_1)(x_0 − x_2)} f(x_0) + \dfrac{(x − x_0)(x − x_2)}{(x_1 − x_0)(x_1 − x_2)} f(x_1) + \dfrac{(x − x_0)(x − x_1)}{(x_2 − x_0)(x_2 − x_1)}f(x_2)\end{split}
\end{equation*}

\chapter{Funções de base de Lagrange}
\label{\detokenize{aula-14-interpolacao-lagrange:funcoes-de-base-de-lagrange}}
\sphinxAtStartPar
Código gerador de funções de base de Lagrange de grau \(n\) por computação simbólica.

\begin{sphinxVerbatim}[commandchars=\\\{\}]
\PYG{k+kn}{from} \PYG{n+nn}{sympy} \PYG{k+kn}{import} \PYG{n}{Symbol}

\PYG{k}{def} \PYG{n+nf}{symbolic\PYGZus{}vector}\PYG{p}{(}\PYG{n}{n}\PYG{p}{,}\PYG{n}{var}\PYG{p}{)}\PYG{p}{:}
    \PYG{l+s+sd}{\PYGZdq{}\PYGZdq{}\PYGZdq{}Cria uma lista com n variáveis simbólicas.}
\PYG{l+s+sd}{    }
\PYG{l+s+sd}{        entrada:}
\PYG{l+s+sd}{            n: numero de pontos}
\PYG{l+s+sd}{            var: uma string (ex. \PYGZsq{}x\PYGZsq{})}
\PYG{l+s+sd}{        saida:}
\PYG{l+s+sd}{            V: [\PYGZsq{}var0\PYGZsq{},\PYGZsq{}var1\PYGZsq{},...,\PYGZsq{}varn\PYGZhy{}1\PYGZsq{}]}
\PYG{l+s+sd}{    \PYGZdq{}\PYGZdq{}\PYGZdq{}}
    \PYG{k}{if} \PYG{o+ow}{not} \PYG{n+nb}{isinstance}\PYG{p}{(}\PYG{n}{var}\PYG{p}{,}\PYG{n+nb}{str}\PYG{p}{)}\PYG{p}{:}
        \PYG{k}{raise} \PYG{n+ne}{TypeError}\PYG{p}{(}\PYG{l+s+sa}{f}\PYG{l+s+s1}{\PYGZsq{}}\PYG{l+s+si}{\PYGZob{}}\PYG{n}{var}\PYG{l+s+si}{\PYGZcb{}}\PYG{l+s+s1}{ must be a string.}\PYG{l+s+s1}{\PYGZsq{}}\PYG{p}{)}                 
        
    \PYG{n}{V} \PYG{o}{=} \PYG{p}{[}\PYG{n}{Symbol}\PYG{p}{(}\PYG{n}{var} \PYG{o}{+} \PYG{n+nb}{str}\PYG{p}{(}\PYG{n}{i}\PYG{p}{)}\PYG{p}{)} \PYG{k}{for} \PYG{n}{i} \PYG{o+ow}{in} \PYG{n+nb}{range}\PYG{p}{(}\PYG{l+m+mi}{0}\PYG{p}{,}\PYG{n}{n}\PYG{p}{)}\PYG{p}{]}
        
    \PYG{k}{return} \PYG{n}{V}

\PYG{k}{def} \PYG{n+nf}{L\PYGZus{}nj}\PYG{p}{(}\PYG{n}{X}\PYG{p}{,}\PYG{n}{j}\PYG{p}{)}\PYG{p}{:}
    \PYG{l+s+sd}{\PYGZdq{}\PYGZdq{}\PYGZdq{} Calcula a função de base de Lagrange L\PYGZus{}\PYGZob{}n,j\PYGZcb{}(x). }
\PYG{l+s+sd}{        }
\PYG{l+s+sd}{        entrada:}
\PYG{l+s+sd}{            X: um vetor contendo variáveis simbólicas}
\PYG{l+s+sd}{    \PYGZdq{}\PYGZdq{}\PYGZdq{}}
    \PYG{c+c1}{\PYGZsh{} pega a variavel base passada e converte para simbólica}
    \PYG{n}{x} \PYG{o}{=} \PYG{n+nb}{str}\PYG{p}{(}\PYG{n}{X}\PYG{p}{[}\PYG{l+m+mi}{1}\PYG{p}{]}\PYG{p}{)}
    \PYG{n}{x} \PYG{o}{=} \PYG{n}{Symbol}\PYG{p}{(}\PYG{n}{x}\PYG{p}{[}\PYG{l+m+mi}{0}\PYG{p}{:}\PYG{o}{\PYGZhy{}}\PYG{l+m+mi}{1}\PYG{p}{]}\PYG{p}{)}
    \PYG{n}{L} \PYG{o}{=} \PYG{l+m+mf}{1.0}    
    \PYG{k}{for} \PYG{n}{i} \PYG{o+ow}{in} \PYG{n+nb}{range}\PYG{p}{(}\PYG{l+m+mi}{0}\PYG{p}{,}\PYG{n+nb}{len}\PYG{p}{(}\PYG{n}{X}\PYG{p}{)}\PYG{p}{)}\PYG{p}{:}
        \PYG{k}{if} \PYG{n}{i} \PYG{o}{!=} \PYG{n}{j}\PYG{p}{:}
            \PYG{n}{L} \PYG{o}{*}\PYG{o}{=} \PYG{p}{(}\PYG{n}{x} \PYG{o}{\PYGZhy{}} \PYG{n}{X}\PYG{p}{[}\PYG{n}{i}\PYG{p}{]}\PYG{p}{)}\PYG{o}{/}\PYG{p}{(}\PYG{n}{X}\PYG{p}{[}\PYG{n}{j}\PYG{p}{]} \PYG{o}{\PYGZhy{}} \PYG{n}{X}\PYG{p}{[}\PYG{n}{i}\PYG{p}{]}\PYG{p}{)}  
            
    \PYG{k}{return} \PYG{n}{L}
\end{sphinxVerbatim}

\begin{sphinxVerbatim}[commandchars=\\\{\}]
\PYG{k+kn}{import} \PYG{n+nn}{numpy} \PYG{k}{as} \PYG{n+nn}{np}
\PYG{k+kn}{import} \PYG{n+nn}{matplotlib}\PYG{n+nn}{.}\PYG{n+nn}{pyplot} \PYG{k}{as} \PYG{n+nn}{plt} 

\PYG{c+c1}{\PYGZsh{} número de nós de interpolação: interpolação de (n\PYGZhy{}1)\PYGZhy{}ésimo grau}
\PYG{n}{n} \PYG{o}{=} \PYG{l+m+mi}{7}

\PYG{c+c1}{\PYGZsh{} domínio de interpolação}
\PYG{n}{x0}\PYG{p}{,}\PYG{n}{x1} \PYG{o}{=} \PYG{o}{\PYGZhy{}}\PYG{l+m+mi}{1}\PYG{p}{,}\PYG{l+m+mi}{1}

\PYG{c+c1}{\PYGZsh{} constroi vetor simbolico em x}
\PYG{n}{X} \PYG{o}{=} \PYG{n}{symbolic\PYGZus{}vector}\PYG{p}{(}\PYG{n}{n}\PYG{p}{,}\PYG{l+s+s1}{\PYGZsq{}}\PYG{l+s+s1}{x}\PYG{l+s+s1}{\PYGZsq{}}\PYG{p}{)}

\PYG{c+c1}{\PYGZsh{} constroi pontos numericos }
\PYG{n}{xp} \PYG{o}{=} \PYG{n}{np}\PYG{o}{.}\PYG{n}{linspace}\PYG{p}{(}\PYG{n}{x0}\PYG{p}{,}\PYG{n}{x1}\PYG{p}{,}\PYG{n}{num}\PYG{o}{=}\PYG{n}{n}\PYG{p}{,}\PYG{n}{endpoint}\PYG{o}{=}\PYG{k+kc}{True}\PYG{p}{)}

\PYG{c+c1}{\PYGZsh{} cria malha numérica}
\PYG{n}{xv} \PYG{o}{=} \PYG{n}{np}\PYG{o}{.}\PYG{n}{linspace}\PYG{p}{(}\PYG{n}{x0}\PYG{p}{,}\PYG{n}{x1}\PYG{p}{)}

\PYG{c+c1}{\PYGZsh{} matriz das funções}
\PYG{n}{Y} \PYG{o}{=} \PYG{n}{np}\PYG{o}{.}\PYG{n}{zeros}\PYG{p}{(}\PYG{p}{(}\PYG{n}{n}\PYG{p}{,}\PYG{n+nb}{len}\PYG{p}{(}\PYG{n}{xv}\PYG{p}{)}\PYG{p}{)}\PYG{p}{)}
\PYG{k}{for} \PYG{n}{i} \PYG{o+ow}{in} \PYG{n+nb}{range}\PYG{p}{(}\PYG{l+m+mi}{0}\PYG{p}{,}\PYG{n}{n}\PYG{p}{)}\PYG{p}{:}
    \PYG{n}{Y}\PYG{p}{[}\PYG{n}{i}\PYG{p}{,}\PYG{p}{]} \PYG{o}{=} \PYG{n}{np}\PYG{o}{.}\PYG{n}{zeros}\PYG{p}{(}\PYG{n}{np}\PYG{o}{.}\PYG{n}{shape}\PYG{p}{(}\PYG{n}{xv}\PYG{p}{)}\PYG{p}{)}

\PYG{c+c1}{\PYGZsh{} montagem de dict para substituição: [xk,x0,x1,x2,...]}
\PYG{n}{k} \PYG{o}{=} \PYG{p}{[}\PYG{n+nb}{str}\PYG{p}{(}\PYG{n}{i}\PYG{p}{)} \PYG{k}{for} \PYG{n}{i} \PYG{o+ow}{in} \PYG{n}{X}\PYG{p}{]}
\PYG{n}{k}\PYG{o}{.}\PYG{n}{insert}\PYG{p}{(}\PYG{l+m+mi}{0}\PYG{p}{,}\PYG{l+s+s1}{\PYGZsq{}}\PYG{l+s+s1}{x}\PYG{l+s+s1}{\PYGZsq{}}\PYG{p}{)}

\PYG{c+c1}{\PYGZsh{} preenche matriz}
\PYG{k}{for} \PYG{n}{i} \PYG{o+ow}{in} \PYG{n+nb}{range}\PYG{p}{(}\PYG{l+m+mi}{0}\PYG{p}{,}\PYG{n}{Y}\PYG{o}{.}\PYG{n}{shape}\PYG{p}{[}\PYG{l+m+mi}{0}\PYG{p}{]}\PYG{p}{)}\PYG{p}{:}
    \PYG{k}{for} \PYG{n}{j} \PYG{o+ow}{in} \PYG{n+nb}{range}\PYG{p}{(}\PYG{l+m+mi}{0}\PYG{p}{,}\PYG{n}{np}\PYG{o}{.}\PYG{n}{size}\PYG{p}{(}\PYG{n}{xv}\PYG{p}{)}\PYG{p}{)}\PYG{p}{:}
        \PYG{n}{v} \PYG{o}{=} \PYG{n+nb}{list}\PYG{p}{(}\PYG{n}{np}\PYG{o}{.}\PYG{n}{concatenate}\PYG{p}{(}\PYG{p}{[}\PYG{n}{np}\PYG{o}{.}\PYG{n}{asarray}\PYG{p}{(}\PYG{p}{[}\PYG{n}{xv}\PYG{p}{[}\PYG{n}{j}\PYG{p}{]}\PYG{p}{]}\PYG{p}{)}\PYG{p}{,}\PYG{n}{xp}\PYG{p}{]}\PYG{p}{)}\PYG{p}{)}
        \PYG{n}{d} \PYG{o}{=} \PYG{n+nb}{dict}\PYG{p}{(}\PYG{n+nb}{zip}\PYG{p}{(}\PYG{n}{k}\PYG{p}{,}\PYG{n}{v}\PYG{p}{)}\PYG{p}{)}
        \PYG{n}{Y}\PYG{p}{[}\PYG{n}{i}\PYG{p}{,}\PYG{n}{j}\PYG{p}{]} \PYG{o}{=} \PYG{n}{L\PYGZus{}nj}\PYG{p}{(}\PYG{n}{X}\PYG{p}{,}\PYG{n}{i}\PYG{p}{)}\PYG{o}{.}\PYG{n}{subs}\PYG{p}{(}\PYG{n}{d}\PYG{p}{)}

\PYG{c+c1}{\PYGZsh{} plotagem das funções}

\PYG{n}{leg} \PYG{o}{=} \PYG{p}{[}\PYG{p}{]}
\PYG{k}{for} \PYG{n}{i} \PYG{o+ow}{in} \PYG{n+nb}{range}\PYG{p}{(}\PYG{l+m+mi}{0}\PYG{p}{,}\PYG{n}{Y}\PYG{o}{.}\PYG{n}{shape}\PYG{p}{[}\PYG{l+m+mi}{0}\PYG{p}{]}\PYG{p}{)}\PYG{p}{:}
    \PYG{n}{plt}\PYG{o}{.}\PYG{n}{plot}\PYG{p}{(}\PYG{n}{xv}\PYG{p}{,}\PYG{n}{Y}\PYG{p}{[}\PYG{n}{i}\PYG{p}{,}\PYG{p}{]}\PYG{p}{)}    
    \PYG{n}{s} \PYG{o}{=} \PYG{l+s+s1}{\PYGZsq{}}\PYG{l+s+s1}{\PYGZdl{}L\PYGZus{}}\PYG{l+s+s1}{\PYGZob{}}\PYG{l+s+s1}{\PYGZsq{}} \PYG{o}{+} \PYG{n+nb}{str}\PYG{p}{(}\PYG{n}{n}\PYG{o}{\PYGZhy{}}\PYG{l+m+mi}{1}\PYG{p}{)} \PYG{o}{+} \PYG{l+s+s1}{\PYGZsq{}}\PYG{l+s+s1}{,}\PYG{l+s+s1}{\PYGZsq{}} \PYG{o}{+} \PYG{n+nb}{str}\PYG{p}{(}\PYG{n}{i}\PYG{p}{)} \PYG{o}{+} \PYG{l+s+s1}{\PYGZsq{}}\PYG{l+s+s1}{\PYGZcb{}\PYGZdl{}}\PYG{l+s+s1}{\PYGZsq{}}
    \PYG{n}{leg}\PYG{o}{.}\PYG{n}{append}\PYG{p}{(}\PYG{n}{s}\PYG{p}{)}

\PYG{n}{plt}\PYG{o}{.}\PYG{n}{grid}\PYG{p}{(}\PYG{p}{)} 

\PYG{c+c1}{\PYGZsh{} nós }
\PYG{n}{plt}\PYG{o}{.}\PYG{n}{scatter}\PYG{p}{(}\PYG{n}{xp}\PYG{p}{,}\PYG{n}{np}\PYG{o}{.}\PYG{n}{zeros}\PYG{p}{(}\PYG{n}{xp}\PYG{o}{.}\PYG{n}{shape}\PYG{p}{)}\PYG{p}{,}\PYG{n}{c}\PYG{o}{=} \PYG{l+s+s1}{\PYGZsq{}}\PYG{l+s+s1}{k}\PYG{l+s+s1}{\PYGZsq{}}\PYG{p}{)}
\PYG{n}{plt}\PYG{o}{.}\PYG{n}{scatter}\PYG{p}{(}\PYG{n}{xp}\PYG{p}{,}\PYG{n}{np}\PYG{o}{.}\PYG{n}{ones}\PYG{p}{(}\PYG{n}{xp}\PYG{o}{.}\PYG{n}{shape}\PYG{p}{)}\PYG{p}{,}\PYG{n}{c}\PYG{o}{=}\PYG{l+s+s1}{\PYGZsq{}}\PYG{l+s+s1}{k}\PYG{l+s+s1}{\PYGZsq{}}\PYG{p}{,}\PYG{n}{marker}\PYG{o}{=}\PYG{l+s+s1}{\PYGZsq{}}\PYG{l+s+s1}{s}\PYG{l+s+s1}{\PYGZsq{}}\PYG{p}{)}
        
       
\PYG{n}{plt}\PYG{o}{.}\PYG{n}{legend}\PYG{p}{(}\PYG{n}{leg}\PYG{p}{,}\PYG{n}{loc}\PYG{o}{=}\PYG{l+s+s1}{\PYGZsq{}}\PYG{l+s+s1}{best}\PYG{l+s+s1}{\PYGZsq{}}\PYG{p}{,}\PYG{n}{bbox\PYGZus{}to\PYGZus{}anchor}\PYG{o}{=}\PYG{p}{(}\PYG{l+m+mf}{0.7}\PYG{p}{,} \PYG{l+m+mf}{0.5}\PYG{p}{,} \PYG{l+m+mf}{0.5}\PYG{p}{,} \PYG{l+m+mf}{0.5}\PYG{p}{)}\PYG{p}{)}
\PYG{n}{plt}\PYG{o}{.}\PYG{n}{title}\PYG{p}{(}\PYG{l+s+s1}{\PYGZsq{}}\PYG{l+s+s1}{Funções de base de Lagrange de ordem }\PYG{l+s+s1}{\PYGZsq{}} \PYG{o}{+} \PYG{n+nb}{str}\PYG{p}{(}\PYG{n}{n}\PYG{o}{\PYGZhy{}}\PYG{l+m+mi}{1}\PYG{p}{)} \PYG{o}{+} \PYG{l+s+s1}{\PYGZsq{}}\PYG{l+s+s1}{ em [}\PYG{l+s+s1}{\PYGZsq{}}\PYG{o}{+}\PYG{n+nb}{str}\PYG{p}{(}\PYG{n}{x0}\PYG{p}{)}\PYG{o}{+}\PYG{l+s+s1}{\PYGZsq{}}\PYG{l+s+s1}{,}\PYG{l+s+s1}{\PYGZsq{}}\PYG{o}{+}\PYG{n+nb}{str}\PYG{p}{(}\PYG{n}{x1}\PYG{p}{)}\PYG{o}{+}\PYG{l+s+s1}{\PYGZsq{}}\PYG{l+s+s1}{]}\PYG{l+s+s1}{\PYGZsq{}}\PYG{p}{)}\PYG{p}{;}
\end{sphinxVerbatim}

\noindent\sphinxincludegraphics{{aula-14-interpolacao-lagrange_5_0}.png}


\chapter{Polinômio Interpolador de Newton (Diferenças Divididas)}
\label{\detokenize{aula-15-interpolacao-newton:polinomio-interpolador-de-newton-diferencas-divididas}}\label{\detokenize{aula-15-interpolacao-newton::doc}}

\section{Interpolação Linear}
\label{\detokenize{aula-15-interpolacao-newton:interpolacao-linear}}
\sphinxAtStartPar
Excluindo\sphinxhyphen{}se o caso de função constante, a forma mais simples de interpolação é ligar dois pontos dados com uma reta. Usando semelhança de triângulos entre nós e valores de função, obtemos
\begin{equation*}
\begin{split} \dfrac{f_1(x) − f(x_0)}{x − x_0} = \dfrac{f(x_1) − f(x_0)}{x_1 − x_0} \end{split}
\end{equation*}
\sphinxAtStartPar
a qual pode ser reorganizada para fornecer
\begin{equation*}
\begin{split} f_1(x) = f(x_0) + \dfrac{f(x_1) − f(x_0)}{x_1 − x_0} (x − x_0) \qquad (1) \end{split}
\end{equation*}
\sphinxAtStartPar
A notação \(f_1(x)\) indica que esse é um polinômio interpolador de primeiro grau. Observe que, além de representar a inclinação da reta ligando os pontos, o termo \([f(x_1) − f(x_0)]/(x_1 − x_0)\) é uma aproximação por diferenças divididas da primeira derivada.


\section{Interpolação Quadrática}
\label{\detokenize{aula-15-interpolacao-newton:interpolacao-quadratica}}
\sphinxAtStartPar
Com três pontos, a interpolação quadrática é obtida a partir de
\begin{equation*}
\begin{split} f_2(x) = b_0 + b_1(x − x_0) + b_2(x − x_0)(x − x_1) \qquad (2) \end{split}
\end{equation*}
\sphinxAtStartPar
Um procedimento simples pode ser usado para determinar os valores dos coeficientes.

\sphinxAtStartPar
Para \(b_0\), a Equação (2) com \(x = x_0\) pode ser usada para calcular
\begin{equation*}
\begin{split} b_0 = f(x_0) \qquad (3) \end{split}
\end{equation*}
\sphinxAtStartPar
A Equação (3) pode ser substituída na Equação (2), a qual pode ser calculada em \(x = x_1\) para
\begin{equation*}
\begin{split} b_1 = \dfrac{f(x_1) − f(x_0)}{x_1 − x_0} \qquad (4) \end{split}
\end{equation*}
\sphinxAtStartPar
Finalmente, as Equações (3) e (4) podem ser substituídas na Equação (2), a qual pode ser calculada em \(x = x_2\) e resolvida (depois de algumas manipulações algébricas) por
\begin{equation*}
\begin{split} b_2 = \dfrac{\dfrac{f(x_2) − f(x_1)}{x_2 − x_1} − \dfrac{f(x_1) − f(x_0)}{x_1 − x_0}}{x_2 − x_0} \qquad (5) \end{split}
\end{equation*}

\section{Forma Geral dos Polinômios Interpoladores de Newton}
\label{\detokenize{aula-15-interpolacao-newton:forma-geral-dos-polinomios-interpoladores-de-newton}}
\sphinxAtStartPar
A análise anterior pode ser generalizada para ajustar um polinômio de grau \(n\) a \(n + 1\) pontos dados. O polinômio de grau \(n\) é
\begin{equation*}
\begin{split} f_n(x) = b_0 + b_1(x − x_0) + \dots + b_n(x − x_0)(x − x_1) \dots (x − x_{n − 1}) \qquad (6) \end{split}
\end{equation*}
\sphinxAtStartPar
Como foi feito anteriormente com as interpolações linear e quadrática, os pontos dados podem ser usados para calcular os coeficientes \(b_0\), \(b_1\), \textbackslash{}dots , \(b_n\). Para um polinômio de grau \(n\), \(n + 1\) pontos dados são necessários: \((x_0, f(x_0)), (x_1, f(x_1)), \dots , (x_n, f(x_n))\). Usamos esses pontos dados e as seguintes equações para calcular os coeficientes
\begin{equation*}
\begin{split}
b_0 = f(x_0) \qquad (7) \\
b_1 = f[x_1, x_0] \qquad (8) \\
b_2 = f[x_2, x_1, x_0] \qquad (9) \\
\vdots \\
b_n = f[x_n, x_{n - 1}, \dots , x_1, x_0] \qquad (10)
\end{split}
\end{equation*}
\sphinxAtStartPar
onde a função com colchetes corresponde a diferenças divididas. Por exemplo, a primeira diferença dividida finita é representada em geral por
\begin{equation*}
\begin{split} f[x_i, x_j] = \dfrac{f(x_i) − f(x_j)}{x_i − x_j} \end{split}
\end{equation*}
\sphinxAtStartPar
A segunda diferença dividida finita, que representa a diferença das duas primeiras diferenças divididas, é expressa em geral por
\begin{equation*}
\begin{split} f[x_i, x_j, x_k] = \dfrac{f[x_i , x_j] − f[x_j, x_k]}{x_i − x_k} \end{split}
\end{equation*}
\sphinxAtStartPar
Analogamente, a n\sphinxhyphen{}ésima diferença dividida é
\begin{equation*}
\begin{split} f[x_n, x_{n − 1}, \dots , x_1, x_0] = \dfrac{f[x_n, x_{n − 1}, \dots , x_1] − f[x_{n − 1}, x_{n − 2}, \dots , x_0]}{x_n − x_0} \end{split}
\end{equation*}

\section{Funções de Base de Newton}
\label{\detokenize{aula-15-interpolacao-newton:funcoes-de-base-de-newton}}
\sphinxAtStartPar
Código gerador de funções de base de Newton de grau \(n\) por computação simbólica.

\begin{sphinxVerbatim}[commandchars=\\\{\}]
\PYG{k+kn}{from} \PYG{n+nn}{sympy} \PYG{k+kn}{import} \PYG{n}{Symbol}

\PYG{k}{def} \PYG{n+nf}{symbolic\PYGZus{}vector}\PYG{p}{(}\PYG{n}{n}\PYG{p}{,}\PYG{n}{var}\PYG{p}{)}\PYG{p}{:}
    \PYG{l+s+sd}{\PYGZdq{}\PYGZdq{}\PYGZdq{}Cria uma lista com n variáveis simbólicas.}
\PYG{l+s+sd}{    }
\PYG{l+s+sd}{        entrada:}
\PYG{l+s+sd}{            n: numero de pontos}
\PYG{l+s+sd}{            var: uma string (ex. \PYGZsq{}x\PYGZsq{})}
\PYG{l+s+sd}{        saida:}
\PYG{l+s+sd}{            V: [\PYGZsq{}var0\PYGZsq{},\PYGZsq{}var1\PYGZsq{},...,\PYGZsq{}varn\PYGZhy{}1\PYGZsq{}]}
\PYG{l+s+sd}{    \PYGZdq{}\PYGZdq{}\PYGZdq{}}
    \PYG{k}{if} \PYG{o+ow}{not} \PYG{n+nb}{isinstance}\PYG{p}{(}\PYG{n}{var}\PYG{p}{,}\PYG{n+nb}{str}\PYG{p}{)}\PYG{p}{:}
        \PYG{k}{raise} \PYG{n+ne}{TypeError}\PYG{p}{(}\PYG{l+s+s2}{\PYGZdq{}}\PYG{l+s+si}{\PYGZob{}0\PYGZcb{}}\PYG{l+s+s2}{ must be a string.}\PYG{l+s+s2}{\PYGZdq{}}\PYG{o}{.}\PYG{n}{format}\PYG{p}{(}\PYG{n}{var}\PYG{p}{)}\PYG{p}{)}                 
        
    \PYG{n}{V} \PYG{o}{=} \PYG{p}{[}\PYG{n}{Symbol}\PYG{p}{(}\PYG{n}{var}  \PYG{o}{+} \PYG{n+nb}{str}\PYG{p}{(}\PYG{n}{i}\PYG{p}{)}\PYG{p}{)} \PYG{k}{for} \PYG{n}{i} \PYG{o+ow}{in} \PYG{n+nb}{range}\PYG{p}{(}\PYG{l+m+mi}{0}\PYG{p}{,}\PYG{n}{n}\PYG{p}{)}\PYG{p}{]}
        
    \PYG{k}{return} \PYG{n}{V}

\PYG{k}{def} \PYG{n+nf}{N\PYGZus{}nj}\PYG{p}{(}\PYG{n}{X}\PYG{p}{,}\PYG{n}{j}\PYG{p}{)}\PYG{p}{:}
    \PYG{l+s+sd}{\PYGZdq{}\PYGZdq{}\PYGZdq{} Calcula a função de base de Newton N\PYGZus{}\PYGZob{}n,j\PYGZcb{}(x). }
\PYG{l+s+sd}{        }
\PYG{l+s+sd}{        entrada:}
\PYG{l+s+sd}{            X: um vetor contendo variáveis simbólicas}
\PYG{l+s+sd}{    \PYGZdq{}\PYGZdq{}\PYGZdq{}}
    \PYG{c+c1}{\PYGZsh{} pega a variavel base passada e converte para simbólica}
    \PYG{n}{x} \PYG{o}{=} \PYG{n}{X}\PYG{p}{[}\PYG{l+m+mi}{1}\PYG{p}{]}
    \PYG{n}{x} \PYG{o}{=} \PYG{n+nb}{str}\PYG{p}{(}\PYG{n}{x}\PYG{p}{)}
    \PYG{n}{x} \PYG{o}{=} \PYG{n}{Symbol}\PYG{p}{(}\PYG{n}{x}\PYG{p}{[}\PYG{l+m+mi}{0}\PYG{p}{:}\PYG{o}{\PYGZhy{}}\PYG{l+m+mi}{1}\PYG{p}{]}\PYG{p}{)}
    \PYG{n}{N} \PYG{o}{=} \PYG{n}{x}\PYG{o}{/}\PYG{n}{x}\PYG{p}{;}
    \PYG{k}{if} \PYG{n}{j} \PYG{o}{\PYGZgt{}} \PYG{l+m+mi}{0}\PYG{p}{:} 
        \PYG{k}{for} \PYG{n}{k} \PYG{o+ow}{in} \PYG{n+nb}{range}\PYG{p}{(}\PYG{l+m+mi}{0}\PYG{p}{,}\PYG{n}{j}\PYG{p}{)}\PYG{p}{:}
            \PYG{n}{N} \PYG{o}{*}\PYG{o}{=} \PYG{p}{(}\PYG{n}{x} \PYG{o}{\PYGZhy{}} \PYG{n}{X}\PYG{p}{[}\PYG{n}{k}\PYG{p}{]}\PYG{p}{)}
            
    \PYG{k}{return} \PYG{n}{N}
\end{sphinxVerbatim}

\begin{sphinxVerbatim}[commandchars=\\\{\}]
\PYG{k+kn}{import} \PYG{n+nn}{numpy} \PYG{k}{as} \PYG{n+nn}{np}
\PYG{k+kn}{import} \PYG{n+nn}{matplotlib}\PYG{n+nn}{.}\PYG{n+nn}{pyplot} \PYG{k}{as} \PYG{n+nn}{plt}
\PYG{o}{\PYGZpc{}}\PYG{k}{matplotlib} inline

\PYG{c+c1}{\PYGZsh{} número de nós de interpolação: interpolação de (n\PYGZhy{}1)\PYGZhy{}ésimo grau}
\PYG{n}{n} \PYG{o}{=} \PYG{l+m+mi}{5}

\PYG{c+c1}{\PYGZsh{} domínio de interpolação}
\PYG{n}{x0}\PYG{p}{,}\PYG{n}{x1} \PYG{o}{=} \PYG{o}{\PYGZhy{}}\PYG{l+m+mi}{1}\PYG{p}{,}\PYG{l+m+mi}{1}

\PYG{c+c1}{\PYGZsh{} constroi vetor simbolico em x}
\PYG{n}{X} \PYG{o}{=} \PYG{n}{symbolic\PYGZus{}vector}\PYG{p}{(}\PYG{n}{n}\PYG{p}{,}\PYG{l+s+s1}{\PYGZsq{}}\PYG{l+s+s1}{x}\PYG{l+s+s1}{\PYGZsq{}}\PYG{p}{)}

\PYG{c+c1}{\PYGZsh{} constroi pontos numericos }
\PYG{n}{xp} \PYG{o}{=} \PYG{n}{np}\PYG{o}{.}\PYG{n}{linspace}\PYG{p}{(}\PYG{n}{x0}\PYG{p}{,}\PYG{n}{x1}\PYG{p}{,}\PYG{n}{num}\PYG{o}{=}\PYG{n}{n}\PYG{p}{,}\PYG{n}{endpoint}\PYG{o}{=}\PYG{k+kc}{True}\PYG{p}{)}

\PYG{c+c1}{\PYGZsh{} cria malha numérica}
\PYG{n}{xv} \PYG{o}{=} \PYG{n}{np}\PYG{o}{.}\PYG{n}{linspace}\PYG{p}{(}\PYG{n}{x0}\PYG{p}{,}\PYG{n}{x1}\PYG{p}{)}

\PYG{c+c1}{\PYGZsh{} matriz das funções}
\PYG{n}{Y} \PYG{o}{=} \PYG{n}{np}\PYG{o}{.}\PYG{n}{zeros}\PYG{p}{(}\PYG{p}{(}\PYG{n}{n}\PYG{p}{,}\PYG{n+nb}{len}\PYG{p}{(}\PYG{n}{xv}\PYG{p}{)}\PYG{p}{)}\PYG{p}{)}
\PYG{k}{for} \PYG{n}{i} \PYG{o+ow}{in} \PYG{n+nb}{range}\PYG{p}{(}\PYG{l+m+mi}{0}\PYG{p}{,}\PYG{n}{n}\PYG{p}{)}\PYG{p}{:}
    \PYG{n}{Y}\PYG{p}{[}\PYG{n}{i}\PYG{p}{,}\PYG{p}{]} \PYG{o}{=} \PYG{n}{np}\PYG{o}{.}\PYG{n}{zeros}\PYG{p}{(}\PYG{n}{np}\PYG{o}{.}\PYG{n}{shape}\PYG{p}{(}\PYG{n}{xv}\PYG{p}{)}\PYG{p}{)}

\PYG{c+c1}{\PYGZsh{} montagem de dict para substituição: [xk,x0,x1,x2,...]}
\PYG{n}{k} \PYG{o}{=} \PYG{p}{[}\PYG{n+nb}{str}\PYG{p}{(}\PYG{n}{i}\PYG{p}{)} \PYG{k}{for} \PYG{n}{i} \PYG{o+ow}{in} \PYG{n}{X}\PYG{p}{]}
\PYG{n}{k}\PYG{o}{.}\PYG{n}{insert}\PYG{p}{(}\PYG{l+m+mi}{0}\PYG{p}{,}\PYG{l+s+s1}{\PYGZsq{}}\PYG{l+s+s1}{x}\PYG{l+s+s1}{\PYGZsq{}}\PYG{p}{)}

\PYG{c+c1}{\PYGZsh{} preenche matriz}
\PYG{k}{for} \PYG{n}{i} \PYG{o+ow}{in} \PYG{n+nb}{range}\PYG{p}{(}\PYG{l+m+mi}{0}\PYG{p}{,}\PYG{n}{Y}\PYG{o}{.}\PYG{n}{shape}\PYG{p}{[}\PYG{l+m+mi}{0}\PYG{p}{]}\PYG{p}{)}\PYG{p}{:}
    \PYG{k}{for} \PYG{n}{j} \PYG{o+ow}{in} \PYG{n+nb}{range}\PYG{p}{(}\PYG{l+m+mi}{0}\PYG{p}{,}\PYG{n}{np}\PYG{o}{.}\PYG{n}{size}\PYG{p}{(}\PYG{n}{xv}\PYG{p}{)}\PYG{p}{)}\PYG{p}{:}
        \PYG{n}{v} \PYG{o}{=} \PYG{n+nb}{list}\PYG{p}{(}\PYG{n}{np}\PYG{o}{.}\PYG{n}{concatenate}\PYG{p}{(}\PYG{p}{[}\PYG{n}{np}\PYG{o}{.}\PYG{n}{asarray}\PYG{p}{(}\PYG{p}{[}\PYG{n}{xv}\PYG{p}{[}\PYG{n}{j}\PYG{p}{]}\PYG{p}{]}\PYG{p}{)}\PYG{p}{,}\PYG{n}{xp}\PYG{p}{]}\PYG{p}{)}\PYG{p}{)}
        \PYG{n}{d} \PYG{o}{=} \PYG{n+nb}{dict}\PYG{p}{(}\PYG{n+nb}{zip}\PYG{p}{(}\PYG{n}{k}\PYG{p}{,}\PYG{n}{v}\PYG{p}{)}\PYG{p}{)}
        \PYG{n}{Y}\PYG{p}{[}\PYG{n}{i}\PYG{p}{,}\PYG{n}{j}\PYG{p}{]} \PYG{o}{=} \PYG{n}{N\PYGZus{}nj}\PYG{p}{(}\PYG{n}{X}\PYG{p}{,}\PYG{n}{i}\PYG{p}{)}\PYG{o}{.}\PYG{n}{subs}\PYG{p}{(}\PYG{n}{d}\PYG{p}{)}

\PYG{c+c1}{\PYGZsh{} plotagem das funções}

\PYG{c+c1}{\PYGZsh{} nós }
\PYG{n}{plt}\PYG{o}{.}\PYG{n}{scatter}\PYG{p}{(}\PYG{n}{xp}\PYG{p}{,}\PYG{n}{np}\PYG{o}{.}\PYG{n}{zeros}\PYG{p}{(}\PYG{n}{xp}\PYG{o}{.}\PYG{n}{shape}\PYG{p}{)}\PYG{p}{,}\PYG{n}{c}\PYG{o}{=} \PYG{l+s+s1}{\PYGZsq{}}\PYG{l+s+s1}{k}\PYG{l+s+s1}{\PYGZsq{}}\PYG{p}{)}

\PYG{n}{leg} \PYG{o}{=} \PYG{p}{[}\PYG{p}{]}
\PYG{k}{for} \PYG{n}{i} \PYG{o+ow}{in} \PYG{n+nb}{range}\PYG{p}{(}\PYG{l+m+mi}{0}\PYG{p}{,}\PYG{n}{Y}\PYG{o}{.}\PYG{n}{shape}\PYG{p}{[}\PYG{l+m+mi}{0}\PYG{p}{]}\PYG{p}{)}\PYG{p}{:}
    \PYG{n}{plt}\PYG{o}{.}\PYG{n}{plot}\PYG{p}{(}\PYG{n}{xv}\PYG{p}{,}\PYG{n}{Y}\PYG{p}{[}\PYG{n}{i}\PYG{p}{,}\PYG{p}{]}\PYG{p}{)}
    \PYG{n}{s} \PYG{o}{=} \PYG{l+s+s1}{\PYGZsq{}}\PYG{l+s+s1}{\PYGZdl{}N\PYGZus{}}\PYG{l+s+s1}{\PYGZob{}}\PYG{l+s+s1}{\PYGZsq{}} \PYG{o}{+} \PYG{n+nb}{str}\PYG{p}{(}\PYG{n}{i}\PYG{p}{)} \PYG{o}{+} \PYG{l+s+s1}{\PYGZsq{}}\PYG{l+s+s1}{\PYGZcb{}(x)\PYGZdl{}}\PYG{l+s+s1}{\PYGZsq{}}
    \PYG{n}{leg}\PYG{o}{.}\PYG{n}{append}\PYG{p}{(}\PYG{n}{s}\PYG{p}{)}
        
\PYG{n}{plt}\PYG{o}{.}\PYG{n}{grid}\PYG{p}{(}\PYG{p}{)}        
\PYG{n}{plt}\PYG{o}{.}\PYG{n}{legend}\PYG{p}{(}\PYG{n}{leg}\PYG{p}{,}\PYG{n}{loc}\PYG{o}{=}\PYG{l+s+s1}{\PYGZsq{}}\PYG{l+s+s1}{best}\PYG{l+s+s1}{\PYGZsq{}}\PYG{p}{)}
\PYG{n}{plt}\PYG{o}{.}\PYG{n}{title}\PYG{p}{(}\PYG{l+s+s1}{\PYGZsq{}}\PYG{l+s+s1}{Funções de base de Newton até ordem }\PYG{l+s+s1}{\PYGZsq{}} \PYG{o}{+} \PYG{n+nb}{str}\PYG{p}{(}\PYG{n}{n}\PYG{o}{\PYGZhy{}}\PYG{l+m+mi}{1}\PYG{p}{)} \PYG{o}{+} \PYG{l+s+s1}{\PYGZsq{}}\PYG{l+s+s1}{ em [}\PYG{l+s+s1}{\PYGZsq{}}\PYG{o}{+}\PYG{n+nb}{str}\PYG{p}{(}\PYG{n}{x0}\PYG{p}{)}\PYG{o}{+}\PYG{l+s+s1}{\PYGZsq{}}\PYG{l+s+s1}{,}\PYG{l+s+s1}{\PYGZsq{}}\PYG{o}{+}\PYG{n+nb}{str}\PYG{p}{(}\PYG{n}{x1}\PYG{p}{)}\PYG{o}{+}\PYG{l+s+s1}{\PYGZsq{}}\PYG{l+s+s1}{]}\PYG{l+s+s1}{\PYGZsq{}}\PYG{p}{)}\PYG{p}{;}
\end{sphinxVerbatim}

\noindent\sphinxincludegraphics{{aula-15-interpolacao-newton_3_0}.png}


\subsection{Exemplo:}
\label{\detokenize{aula-15-interpolacao-newton:exemplo}}
\sphinxAtStartPar
Encontre o polinômio interpolador de Newton de ordem 1 \(P_1(x)\) para a tabela abaixo


\begin{savenotes}\sphinxattablestart
\centering
\begin{tabulary}{\linewidth}[t]{|T|T|}
\hline
\sphinxstyletheadfamily 
\sphinxAtStartPar
x
&\sphinxstyletheadfamily 
\sphinxAtStartPar
y
\\
\hline
\sphinxAtStartPar
\sphinxhyphen{}1
&
\sphinxAtStartPar
4
\\
\hline
\sphinxAtStartPar
0
&
\sphinxAtStartPar
1
\\
\hline
\end{tabulary}
\par
\sphinxattableend\end{savenotes}

\sphinxAtStartPar
Compute o valor de \(P_1(0.35)\).

\begin{sphinxVerbatim}[commandchars=\\\{\}]
\PYG{k+kn}{import} \PYG{n+nn}{numpy} \PYG{k}{as} \PYG{n+nn}{np}
\PYG{k+kn}{import} \PYG{n+nn}{matplotlib}\PYG{n+nn}{.}\PYG{n+nn}{pyplot} \PYG{k}{as} \PYG{n+nn}{plt}

\PYG{c+c1}{\PYGZsh{} interpolação linear}

\PYG{c+c1}{\PYGZsh{} coeficiente a0 = y0}
\PYG{c+c1}{\PYGZsh{} coeficiente a1 = (y1\PYGZhy{}y0)/(x1\PYGZhy{}x0)}

\PYG{c+c1}{\PYGZsh{} pontos}
\PYG{n}{x0}\PYG{p}{,}\PYG{n}{y0} \PYG{o}{=} \PYG{o}{\PYGZhy{}}\PYG{l+m+mi}{1}\PYG{p}{,}\PYG{l+m+mi}{4}
\PYG{n}{x1}\PYG{p}{,}\PYG{n}{y1} \PYG{o}{=}  \PYG{l+m+mi}{0}\PYG{p}{,}\PYG{l+m+mi}{1}

\PYG{c+c1}{\PYGZsh{} ordem 0}
\PYG{n}{a0} \PYG{o}{=} \PYG{n}{y0}

\PYG{c+c1}{\PYGZsh{} interpolador de Newton}
\PYG{n}{a1} \PYG{o}{=} \PYG{p}{(}\PYG{n}{y1}\PYG{o}{\PYGZhy{}}\PYG{n}{y0}\PYG{p}{)}\PYG{o}{/}\PYG{p}{(}\PYG{n}{x1}\PYG{o}{\PYGZhy{}}\PYG{n}{x0}\PYG{p}{)}
\PYG{n}{P1} \PYG{o}{=} \PYG{k}{lambda} \PYG{n}{x}\PYG{p}{:} \PYG{n}{a0} \PYG{o}{+} \PYG{n}{a1}\PYG{o}{*}\PYG{p}{(}\PYG{n}{x}\PYG{o}{\PYGZhy{}}\PYG{n}{x0}\PYG{p}{)}

\PYG{c+c1}{\PYGZsh{} ponto interpolado}
\PYG{n}{xp} \PYG{o}{=} \PYG{o}{\PYGZhy{}}\PYG{l+m+mf}{0.35}
\PYG{n}{yp} \PYG{o}{=} \PYG{n}{P1}\PYG{p}{(}\PYG{n}{xp}\PYG{p}{)}
\PYG{n}{yp}

\PYG{c+c1}{\PYGZsh{} plotagem}

\PYG{c+c1}{\PYGZsh{} nós}
\PYG{n}{plt}\PYG{o}{.}\PYG{n}{plot}\PYG{p}{(}\PYG{p}{[}\PYG{n}{x0}\PYG{p}{,}\PYG{n}{x1}\PYG{p}{]}\PYG{p}{,}\PYG{p}{[}\PYG{l+m+mi}{0}\PYG{p}{,}\PYG{l+m+mi}{0}\PYG{p}{]}\PYG{p}{,}\PYG{l+s+s1}{\PYGZsq{}}\PYG{l+s+s1}{ok}\PYG{l+s+s1}{\PYGZsq{}}\PYG{p}{)}

\PYG{c+c1}{\PYGZsh{} valores nodais}
\PYG{n}{plt}\PYG{o}{.}\PYG{n}{plot}\PYG{p}{(}\PYG{p}{[}\PYG{n}{x0}\PYG{p}{,}\PYG{n}{x1}\PYG{p}{]}\PYG{p}{,}\PYG{p}{[}\PYG{n}{y0}\PYG{p}{,}\PYG{n}{y1}\PYG{p}{]}\PYG{p}{,}\PYG{l+s+s1}{\PYGZsq{}}\PYG{l+s+s1}{ok}\PYG{l+s+s1}{\PYGZsq{}}\PYG{p}{)}

\PYG{c+c1}{\PYGZsh{} interpolador}
\PYG{n}{x} \PYG{o}{=} \PYG{n}{np}\PYG{o}{.}\PYG{n}{linspace}\PYG{p}{(}\PYG{n}{x0}\PYG{p}{,}\PYG{n}{x1}\PYG{p}{,}\PYG{l+m+mi}{30}\PYG{p}{,}\PYG{n}{endpoint}\PYG{o}{=}\PYG{k+kc}{True}\PYG{p}{)}
\PYG{n}{plt}\PYG{o}{.}\PYG{n}{plot}\PYG{p}{(}\PYG{n}{x}\PYG{p}{,}\PYG{n}{P1}\PYG{p}{(}\PYG{n}{x}\PYG{p}{)}\PYG{p}{,}\PYG{n}{label}\PYG{o}{=}\PYG{l+s+s1}{\PYGZsq{}}\PYG{l+s+s1}{\PYGZdl{}P\PYGZus{}1(x)\PYGZdl{}}\PYG{l+s+s1}{\PYGZsq{}}\PYG{p}{)}

\PYG{c+c1}{\PYGZsh{} ponto interpolado}
\PYG{n}{plt}\PYG{o}{.}\PYG{n}{plot}\PYG{p}{(}\PYG{n}{xp}\PYG{p}{,}\PYG{l+m+mi}{0}\PYG{p}{,}\PYG{l+s+s1}{\PYGZsq{}}\PYG{l+s+s1}{sr}\PYG{l+s+s1}{\PYGZsq{}}\PYG{p}{)}
\PYG{n}{plt}\PYG{o}{.}\PYG{n}{plot}\PYG{p}{(}\PYG{n}{xp}\PYG{p}{,}\PYG{n}{yp}\PYG{p}{,}\PYG{l+s+s1}{\PYGZsq{}}\PYG{l+s+s1}{sr}\PYG{l+s+s1}{\PYGZsq{}}\PYG{p}{)}
\PYG{n}{plt}\PYG{o}{.}\PYG{n}{axvline}\PYG{p}{(}\PYG{n}{xp}\PYG{p}{,}\PYG{l+m+mi}{0}\PYG{p}{,}\PYG{n}{yp}\PYG{p}{,}\PYG{n}{c}\PYG{o}{=}\PYG{l+s+s1}{\PYGZsq{}}\PYG{l+s+s1}{r}\PYG{l+s+s1}{\PYGZsq{}}\PYG{p}{,}\PYG{n}{ls}\PYG{o}{=}\PYG{l+s+s1}{\PYGZsq{}}\PYG{l+s+s1}{dashed}\PYG{l+s+s1}{\PYGZsq{}}\PYG{p}{)}

\PYG{n}{plt}\PYG{o}{.}\PYG{n}{grid}\PYG{p}{(}\PYG{p}{)}
\PYG{n}{plt}\PYG{o}{.}\PYG{n}{legend}\PYG{p}{(}\PYG{n}{loc}\PYG{o}{=}\PYG{l+s+s1}{\PYGZsq{}}\PYG{l+s+s1}{best}\PYG{l+s+s1}{\PYGZsq{}}\PYG{p}{)}\PYG{p}{;}
\end{sphinxVerbatim}

\noindent\sphinxincludegraphics{{aula-15-interpolacao-newton_5_0}.png}

\begin{sphinxVerbatim}[commandchars=\\\{\}]
\PYG{n}{P1}\PYG{p}{(}\PYG{n}{xp}\PYG{p}{)}
\end{sphinxVerbatim}

\begin{sphinxVerbatim}[commandchars=\\\{\}]
2.05
\end{sphinxVerbatim}


\subsection{Exemplo:}
\label{\detokenize{aula-15-interpolacao-newton:id1}}
\sphinxAtStartPar
Encontre o polinômio interpolador de Newton de ordem 2 \(P_2(x)\) para a tabela abaixo


\begin{savenotes}\sphinxattablestart
\centering
\begin{tabulary}{\linewidth}[t]{|T|T|}
\hline
\sphinxstyletheadfamily 
\sphinxAtStartPar
x
&\sphinxstyletheadfamily 
\sphinxAtStartPar
y
\\
\hline
\sphinxAtStartPar
\sphinxhyphen{}1
&
\sphinxAtStartPar
4
\\
\hline
\sphinxAtStartPar
0
&
\sphinxAtStartPar
1
\\
\hline
\sphinxAtStartPar
2
&
\sphinxAtStartPar
\sphinxhyphen{}1
\\
\hline
\end{tabulary}
\par
\sphinxattableend\end{savenotes}

\sphinxAtStartPar
Compute o valor de \(P_2(0.35)\).

\begin{sphinxVerbatim}[commandchars=\\\{\}]
\PYG{c+c1}{\PYGZsh{} interpolação quadrática}

\PYG{c+c1}{\PYGZsh{} Usando tabela DD: }
\PYG{c+c1}{\PYGZsh{} https://vnicius.github.io/numbiosis/interpolador\PYGZhy{}newton/index.html}


\PYG{c+c1}{\PYGZsh{} par adicional }
\PYG{n}{x2}\PYG{p}{,}\PYG{n}{y2} \PYG{o}{=} \PYG{l+m+mf}{2.}\PYG{p}{,}\PYG{o}{\PYGZhy{}}\PYG{l+m+mf}{1.}

\PYG{c+c1}{\PYGZsh{} coeficiente a2 = f[x0,x1,x2] = ( f[x1,x2] \PYGZhy{} f[x0,x1] ) / (x2 \PYGZhy{} x0)}
\PYG{c+c1}{\PYGZsh{} a2 = ( (y2\PYGZhy{}y1)/(x2\PYGZhy{}x1) \PYGZhy{} (y1\PYGZhy{}y0)/(x1\PYGZhy{}x0) )/(x2\PYGZhy{}x0)*(xx\PYGZhy{}x0)*(xx\PYGZhy{}x1)}

\PYG{c+c1}{\PYGZsh{} interpolador de Newton}
\PYG{n}{P2} \PYG{o}{=} \PYG{k}{lambda} \PYG{n}{xx}\PYG{p}{:} \PYG{n}{P1}\PYG{p}{(}\PYG{n}{xx}\PYG{p}{)} \PYG{o}{+} \PYG{p}{(} \PYG{p}{(} \PYG{p}{(}\PYG{n}{y2}\PYG{o}{\PYGZhy{}}\PYG{n}{y1}\PYG{p}{)}\PYG{o}{/}\PYG{p}{(}\PYG{n}{x2}\PYG{o}{\PYGZhy{}}\PYG{n}{x1}\PYG{p}{)} \PYG{o}{\PYGZhy{}} \PYG{p}{(}\PYG{n}{y1}\PYG{o}{\PYGZhy{}}\PYG{n}{y0}\PYG{p}{)}\PYG{o}{/}\PYG{p}{(}\PYG{n}{x1}\PYG{o}{\PYGZhy{}}\PYG{n}{x0}\PYG{p}{)} \PYG{p}{)}\PYG{o}{/}\PYG{p}{(}\PYG{n}{x2}\PYG{o}{\PYGZhy{}}\PYG{n}{x0}\PYG{p}{)} \PYG{p}{)}\PYG{o}{*}\PYG{p}{(}\PYG{n}{xx}\PYG{o}{\PYGZhy{}}\PYG{n}{x0}\PYG{p}{)}\PYG{o}{*}\PYG{p}{(}\PYG{n}{xx}\PYG{o}{\PYGZhy{}}\PYG{n}{x1}\PYG{p}{)}


\PYG{c+c1}{\PYGZsh{} ponto interpolado}
\PYG{n}{yp} \PYG{o}{=} \PYG{n}{P2}\PYG{p}{(}\PYG{n}{xp}\PYG{p}{)}
\PYG{n}{yp}

\PYG{c+c1}{\PYGZsh{} plotagem}

\PYG{c+c1}{\PYGZsh{} nós}
\PYG{n}{plt}\PYG{o}{.}\PYG{n}{plot}\PYG{p}{(}\PYG{p}{[}\PYG{n}{x0}\PYG{p}{,}\PYG{n}{x1}\PYG{p}{,}\PYG{n}{x2}\PYG{p}{]}\PYG{p}{,}\PYG{p}{[}\PYG{l+m+mi}{0}\PYG{p}{,}\PYG{l+m+mi}{0}\PYG{p}{,}\PYG{l+m+mi}{0}\PYG{p}{]}\PYG{p}{,}\PYG{l+s+s1}{\PYGZsq{}}\PYG{l+s+s1}{ok}\PYG{l+s+s1}{\PYGZsq{}}\PYG{p}{)}

\PYG{c+c1}{\PYGZsh{} valores nodais}
\PYG{n}{plt}\PYG{o}{.}\PYG{n}{plot}\PYG{p}{(}\PYG{p}{[}\PYG{n}{x0}\PYG{p}{,}\PYG{n}{x1}\PYG{p}{,}\PYG{n}{x2}\PYG{p}{]}\PYG{p}{,}\PYG{p}{[}\PYG{n}{y0}\PYG{p}{,}\PYG{n}{y1}\PYG{p}{,}\PYG{n}{y2}\PYG{p}{]}\PYG{p}{,}\PYG{l+s+s1}{\PYGZsq{}}\PYG{l+s+s1}{ok}\PYG{l+s+s1}{\PYGZsq{}}\PYG{p}{)}

\PYG{c+c1}{\PYGZsh{} interpolador}
\PYG{n}{x} \PYG{o}{=} \PYG{n}{np}\PYG{o}{.}\PYG{n}{linspace}\PYG{p}{(}\PYG{n}{x0}\PYG{p}{,}\PYG{n}{x2}\PYG{p}{,}\PYG{l+m+mi}{30}\PYG{p}{,}\PYG{n}{endpoint}\PYG{o}{=}\PYG{k+kc}{True}\PYG{p}{)}
\PYG{n}{plt}\PYG{o}{.}\PYG{n}{plot}\PYG{p}{(}\PYG{n}{x}\PYG{p}{,}\PYG{n}{P2}\PYG{p}{(}\PYG{n}{x}\PYG{p}{)}\PYG{p}{,}\PYG{n}{label}\PYG{o}{=}\PYG{l+s+s1}{\PYGZsq{}}\PYG{l+s+s1}{\PYGZdl{}P\PYGZus{}2(x)\PYGZdl{}}\PYG{l+s+s1}{\PYGZsq{}}\PYG{p}{)}

\PYG{c+c1}{\PYGZsh{} ponto interpolado}
\PYG{n}{plt}\PYG{o}{.}\PYG{n}{plot}\PYG{p}{(}\PYG{n}{xp}\PYG{p}{,}\PYG{l+m+mi}{0}\PYG{p}{,}\PYG{l+s+s1}{\PYGZsq{}}\PYG{l+s+s1}{sr}\PYG{l+s+s1}{\PYGZsq{}}\PYG{p}{)}
\PYG{n}{plt}\PYG{o}{.}\PYG{n}{plot}\PYG{p}{(}\PYG{n}{xp}\PYG{p}{,}\PYG{n}{yp}\PYG{p}{,}\PYG{l+s+s1}{\PYGZsq{}}\PYG{l+s+s1}{sr}\PYG{l+s+s1}{\PYGZsq{}}\PYG{p}{)}
\PYG{n}{plt}\PYG{o}{.}\PYG{n}{axvline}\PYG{p}{(}\PYG{n}{xp}\PYG{p}{,}\PYG{l+m+mi}{0}\PYG{p}{,}\PYG{n}{yp}\PYG{p}{,}\PYG{n}{c}\PYG{o}{=}\PYG{l+s+s1}{\PYGZsq{}}\PYG{l+s+s1}{r}\PYG{l+s+s1}{\PYGZsq{}}\PYG{p}{,}\PYG{n}{ls}\PYG{o}{=}\PYG{l+s+s1}{\PYGZsq{}}\PYG{l+s+s1}{dashed}\PYG{l+s+s1}{\PYGZsq{}}\PYG{p}{)}

\PYG{n}{plt}\PYG{o}{.}\PYG{n}{grid}\PYG{p}{(}\PYG{p}{)}
\PYG{n}{plt}\PYG{o}{.}\PYG{n}{legend}\PYG{p}{(}\PYG{n}{loc}\PYG{o}{=}\PYG{l+s+s1}{\PYGZsq{}}\PYG{l+s+s1}{best}\PYG{l+s+s1}{\PYGZsq{}}\PYG{p}{)}\PYG{p}{;}
\end{sphinxVerbatim}

\noindent\sphinxincludegraphics{{aula-15-interpolacao-newton_8_0}.png}

\begin{sphinxVerbatim}[commandchars=\\\{\}]
\PYG{n}{P2}\PYG{p}{(}\PYG{n}{xp}\PYG{p}{)}
\end{sphinxVerbatim}

\begin{sphinxVerbatim}[commandchars=\\\{\}]
1.8983333333333332
\end{sphinxVerbatim}


\subsection{Exemplo:}
\label{\detokenize{aula-15-interpolacao-newton:id2}}
\sphinxAtStartPar
Encontre o polinômio interpolador de Newton de ordem 3 \(P_3(x)\) para a tabela abaixo


\begin{savenotes}\sphinxattablestart
\centering
\begin{tabulary}{\linewidth}[t]{|T|T|}
\hline
\sphinxstyletheadfamily 
\sphinxAtStartPar
x
&\sphinxstyletheadfamily 
\sphinxAtStartPar
y
\\
\hline
\sphinxAtStartPar
\sphinxhyphen{}1
&
\sphinxAtStartPar
4
\\
\hline
\sphinxAtStartPar
0
&
\sphinxAtStartPar
1
\\
\hline
\sphinxAtStartPar
2
&
\sphinxAtStartPar
\sphinxhyphen{}1
\\
\hline
\sphinxAtStartPar
3
&
\sphinxAtStartPar
1
\\
\hline
\end{tabulary}
\par
\sphinxattableend\end{savenotes}

\sphinxAtStartPar
Compute o valor de \(P_3(0.35)\).

\begin{sphinxVerbatim}[commandchars=\\\{\}]
\PYG{c+c1}{\PYGZsh{} interpolação quadrática}

\PYG{c+c1}{\PYGZsh{} Usando tabela DD: }
\PYG{c+c1}{\PYGZsh{} https://vnicius.github.io/numbiosis/interpolador\PYGZhy{}newton/index.html}


\PYG{c+c1}{\PYGZsh{} par adicional }
\PYG{n}{x3}\PYG{p}{,}\PYG{n}{y3} \PYG{o}{=} \PYG{l+m+mf}{3.}\PYG{p}{,}\PYG{l+m+mf}{1.}

\PYG{c+c1}{\PYGZsh{} coeficiente a3 = f[x0,x1,x2,x3] }

\PYG{c+c1}{\PYGZsh{} interpolador de Newton}
\PYG{n}{P3} \PYG{o}{=} \PYG{k}{lambda} \PYG{n}{xxx}\PYG{p}{:} \PYG{n}{P2}\PYG{p}{(}\PYG{n}{xxx}\PYG{p}{)} \PYG{o}{+} \PYG{l+m+mi}{1}\PYG{o}{/}\PYG{l+m+mi}{12}\PYG{o}{*}\PYG{p}{(}\PYG{n}{xxx}\PYG{o}{\PYGZhy{}}\PYG{n}{x0}\PYG{p}{)}\PYG{o}{*}\PYG{p}{(}\PYG{n}{xxx}\PYG{o}{\PYGZhy{}}\PYG{n}{x1}\PYG{p}{)}\PYG{o}{*}\PYG{p}{(}\PYG{n}{xxx}\PYG{o}{\PYGZhy{}}\PYG{n}{x2}\PYG{p}{)}


\PYG{c+c1}{\PYGZsh{} ponto interpolado}
\PYG{n}{yp} \PYG{o}{=} \PYG{n}{P3}\PYG{p}{(}\PYG{n}{xp}\PYG{p}{)}
\PYG{n}{yp}

\PYG{c+c1}{\PYGZsh{} plotagem}

\PYG{c+c1}{\PYGZsh{} nós}
\PYG{n}{plt}\PYG{o}{.}\PYG{n}{plot}\PYG{p}{(}\PYG{p}{[}\PYG{n}{x0}\PYG{p}{,}\PYG{n}{x1}\PYG{p}{,}\PYG{n}{x2}\PYG{p}{,}\PYG{n}{x3}\PYG{p}{]}\PYG{p}{,}\PYG{p}{[}\PYG{l+m+mi}{0}\PYG{p}{,}\PYG{l+m+mi}{0}\PYG{p}{,}\PYG{l+m+mi}{0}\PYG{p}{,}\PYG{l+m+mi}{0}\PYG{p}{]}\PYG{p}{,}\PYG{l+s+s1}{\PYGZsq{}}\PYG{l+s+s1}{ok}\PYG{l+s+s1}{\PYGZsq{}}\PYG{p}{)}

\PYG{c+c1}{\PYGZsh{} valores nodais}
\PYG{n}{plt}\PYG{o}{.}\PYG{n}{plot}\PYG{p}{(}\PYG{p}{[}\PYG{n}{x0}\PYG{p}{,}\PYG{n}{x1}\PYG{p}{,}\PYG{n}{x2}\PYG{p}{,}\PYG{n}{x3}\PYG{p}{]}\PYG{p}{,}\PYG{p}{[}\PYG{n}{y0}\PYG{p}{,}\PYG{n}{y1}\PYG{p}{,}\PYG{n}{y2}\PYG{p}{,}\PYG{n}{y3}\PYG{p}{]}\PYG{p}{,}\PYG{l+s+s1}{\PYGZsq{}}\PYG{l+s+s1}{ok}\PYG{l+s+s1}{\PYGZsq{}}\PYG{p}{)}

\PYG{c+c1}{\PYGZsh{} interpolador}
\PYG{n}{x} \PYG{o}{=} \PYG{n}{np}\PYG{o}{.}\PYG{n}{linspace}\PYG{p}{(}\PYG{n}{x0}\PYG{p}{,}\PYG{n}{x3}\PYG{p}{,}\PYG{l+m+mi}{30}\PYG{p}{,}\PYG{n}{endpoint}\PYG{o}{=}\PYG{k+kc}{True}\PYG{p}{)}
\PYG{n}{plt}\PYG{o}{.}\PYG{n}{plot}\PYG{p}{(}\PYG{n}{x}\PYG{p}{,}\PYG{n}{P3}\PYG{p}{(}\PYG{n}{x}\PYG{p}{)}\PYG{p}{,}\PYG{n}{label}\PYG{o}{=}\PYG{l+s+s1}{\PYGZsq{}}\PYG{l+s+s1}{\PYGZdl{}P\PYGZus{}3(x)\PYGZdl{}}\PYG{l+s+s1}{\PYGZsq{}}\PYG{p}{)}

\PYG{c+c1}{\PYGZsh{} ponto interpolado}
\PYG{n}{plt}\PYG{o}{.}\PYG{n}{plot}\PYG{p}{(}\PYG{n}{xp}\PYG{p}{,}\PYG{l+m+mi}{0}\PYG{p}{,}\PYG{l+s+s1}{\PYGZsq{}}\PYG{l+s+s1}{sr}\PYG{l+s+s1}{\PYGZsq{}}\PYG{p}{)}
\PYG{n}{plt}\PYG{o}{.}\PYG{n}{plot}\PYG{p}{(}\PYG{n}{xp}\PYG{p}{,}\PYG{n}{yp}\PYG{p}{,}\PYG{l+s+s1}{\PYGZsq{}}\PYG{l+s+s1}{sr}\PYG{l+s+s1}{\PYGZsq{}}\PYG{p}{)}
\PYG{n}{plt}\PYG{o}{.}\PYG{n}{axvline}\PYG{p}{(}\PYG{n}{xp}\PYG{p}{,}\PYG{l+m+mi}{0}\PYG{p}{,}\PYG{n}{yp}\PYG{p}{,}\PYG{n}{c}\PYG{o}{=}\PYG{l+s+s1}{\PYGZsq{}}\PYG{l+s+s1}{r}\PYG{l+s+s1}{\PYGZsq{}}\PYG{p}{,}\PYG{n}{ls}\PYG{o}{=}\PYG{l+s+s1}{\PYGZsq{}}\PYG{l+s+s1}{dashed}\PYG{l+s+s1}{\PYGZsq{}}\PYG{p}{)}

\PYG{n}{plt}\PYG{o}{.}\PYG{n}{grid}\PYG{p}{(}\PYG{p}{)}
\PYG{n}{plt}\PYG{o}{.}\PYG{n}{legend}\PYG{p}{(}\PYG{n}{loc}\PYG{o}{=}\PYG{l+s+s1}{\PYGZsq{}}\PYG{l+s+s1}{best}\PYG{l+s+s1}{\PYGZsq{}}\PYG{p}{)}\PYG{p}{;}
\end{sphinxVerbatim}

\noindent\sphinxincludegraphics{{aula-15-interpolacao-newton_11_0}.png}

\begin{sphinxVerbatim}[commandchars=\\\{\}]
\PYG{n}{P3}\PYG{p}{(}\PYG{o}{\PYGZhy{}}\PYG{l+m+mf}{0.35}\PYG{p}{)}
\end{sphinxVerbatim}

\begin{sphinxVerbatim}[commandchars=\\\{\}]
1.9428854166666665
\end{sphinxVerbatim}

\begin{sphinxVerbatim}[commandchars=\\\{\}]
\PYG{c+c1}{\PYGZsh{} ponto interpolado}
\PYG{n}{YP} \PYG{o}{=} \PYG{p}{[}\PYG{n}{P1}\PYG{p}{(}\PYG{n}{xp}\PYG{p}{)}\PYG{p}{,}\PYG{n}{P2}\PYG{p}{(}\PYG{n}{xp}\PYG{p}{)}\PYG{p}{,}\PYG{n}{P3}\PYG{p}{(}\PYG{n}{xp}\PYG{p}{)}\PYG{p}{]}

\PYG{c+c1}{\PYGZsh{} plotagem}
\PYG{c+c1}{\PYGZsh{} nós}
\PYG{n}{plt}\PYG{o}{.}\PYG{n}{plot}\PYG{p}{(}\PYG{p}{[}\PYG{n}{x0}\PYG{p}{,}\PYG{n}{x1}\PYG{p}{,}\PYG{n}{x2}\PYG{p}{,}\PYG{n}{x3}\PYG{p}{]}\PYG{p}{,}\PYG{p}{[}\PYG{l+m+mi}{0}\PYG{p}{,}\PYG{l+m+mi}{0}\PYG{p}{,}\PYG{l+m+mi}{0}\PYG{p}{,}\PYG{l+m+mi}{0}\PYG{p}{]}\PYG{p}{,}\PYG{l+s+s1}{\PYGZsq{}}\PYG{l+s+s1}{ok}\PYG{l+s+s1}{\PYGZsq{}}\PYG{p}{)}

\PYG{c+c1}{\PYGZsh{} valores nodais}
\PYG{n}{plt}\PYG{o}{.}\PYG{n}{plot}\PYG{p}{(}\PYG{p}{[}\PYG{n}{x0}\PYG{p}{,}\PYG{n}{x1}\PYG{p}{,}\PYG{n}{x2}\PYG{p}{,}\PYG{n}{x3}\PYG{p}{]}\PYG{p}{,}\PYG{p}{[}\PYG{n}{y0}\PYG{p}{,}\PYG{n}{y1}\PYG{p}{,}\PYG{n}{y2}\PYG{p}{,}\PYG{n}{y3}\PYG{p}{]}\PYG{p}{,}\PYG{l+s+s1}{\PYGZsq{}}\PYG{l+s+s1}{ok}\PYG{l+s+s1}{\PYGZsq{}}\PYG{p}{)}

\PYG{c+c1}{\PYGZsh{} interpoladores}
\PYG{n}{x} \PYG{o}{=} \PYG{n}{np}\PYG{o}{.}\PYG{n}{linspace}\PYG{p}{(}\PYG{n}{x0}\PYG{p}{,}\PYG{n}{x3}\PYG{p}{,}\PYG{l+m+mi}{30}\PYG{p}{,}\PYG{n}{endpoint}\PYG{o}{=}\PYG{k+kc}{True}\PYG{p}{)}
\PYG{n}{plt}\PYG{o}{.}\PYG{n}{plot}\PYG{p}{(}\PYG{n}{x}\PYG{p}{,}\PYG{n}{P1}\PYG{p}{(}\PYG{n}{x}\PYG{p}{)}\PYG{p}{,}\PYG{l+s+s1}{\PYGZsq{}}\PYG{l+s+s1}{b}\PYG{l+s+s1}{\PYGZsq{}}\PYG{p}{,}\PYG{n}{label}\PYG{o}{=}\PYG{l+s+s1}{\PYGZsq{}}\PYG{l+s+s1}{\PYGZdl{}P\PYGZus{}1(x)\PYGZdl{}}\PYG{l+s+s1}{\PYGZsq{}}\PYG{p}{)}
\PYG{n}{plt}\PYG{o}{.}\PYG{n}{plot}\PYG{p}{(}\PYG{n}{x}\PYG{p}{,}\PYG{n}{P2}\PYG{p}{(}\PYG{n}{x}\PYG{p}{)}\PYG{p}{,}\PYG{l+s+s1}{\PYGZsq{}}\PYG{l+s+s1}{c}\PYG{l+s+s1}{\PYGZsq{}}\PYG{p}{,}\PYG{n}{label}\PYG{o}{=}\PYG{l+s+s1}{\PYGZsq{}}\PYG{l+s+s1}{\PYGZdl{}P\PYGZus{}2(x)\PYGZdl{}}\PYG{l+s+s1}{\PYGZsq{}}\PYG{p}{)}
\PYG{n}{plt}\PYG{o}{.}\PYG{n}{plot}\PYG{p}{(}\PYG{n}{x}\PYG{p}{,}\PYG{n}{P3}\PYG{p}{(}\PYG{n}{x}\PYG{p}{)}\PYG{p}{,}\PYG{l+s+s1}{\PYGZsq{}}\PYG{l+s+s1}{g}\PYG{l+s+s1}{\PYGZsq{}}\PYG{p}{,}\PYG{n}{label}\PYG{o}{=}\PYG{l+s+s1}{\PYGZsq{}}\PYG{l+s+s1}{\PYGZdl{}P\PYGZus{}3(x)\PYGZdl{}}\PYG{l+s+s1}{\PYGZsq{}}\PYG{p}{)}

\PYG{c+c1}{\PYGZsh{} ponto interpolado}
\PYG{n}{plt}\PYG{o}{.}\PYG{n}{plot}\PYG{p}{(}\PYG{n}{xp}\PYG{p}{,}\PYG{l+m+mi}{0}\PYG{p}{,}\PYG{l+s+s1}{\PYGZsq{}}\PYG{l+s+s1}{sr}\PYG{l+s+s1}{\PYGZsq{}}\PYG{p}{)}
\PYG{n}{plt}\PYG{o}{.}\PYG{n}{plot}\PYG{p}{(}\PYG{p}{[}\PYG{n}{xp}\PYG{p}{,}\PYG{n}{xp}\PYG{p}{,}\PYG{n}{xp}\PYG{p}{]}\PYG{p}{,}\PYG{n}{YP}\PYG{p}{,}\PYG{l+s+s1}{\PYGZsq{}}\PYG{l+s+s1}{sr}\PYG{l+s+s1}{\PYGZsq{}}\PYG{p}{)}
\PYG{n}{plt}\PYG{o}{.}\PYG{n}{axvline}\PYG{p}{(}\PYG{n}{xp}\PYG{p}{,}\PYG{l+m+mi}{0}\PYG{p}{,}\PYG{n+nb}{max}\PYG{p}{(}\PYG{n}{YP}\PYG{p}{)}\PYG{p}{,}\PYG{n}{c}\PYG{o}{=}\PYG{l+s+s1}{\PYGZsq{}}\PYG{l+s+s1}{r}\PYG{l+s+s1}{\PYGZsq{}}\PYG{p}{,}\PYG{n}{ls}\PYG{o}{=}\PYG{l+s+s1}{\PYGZsq{}}\PYG{l+s+s1}{dashed}\PYG{l+s+s1}{\PYGZsq{}}\PYG{p}{)}

\PYG{n}{plt}\PYG{o}{.}\PYG{n}{grid}\PYG{p}{(}\PYG{p}{)}
\PYG{n}{plt}\PYG{o}{.}\PYG{n}{legend}\PYG{p}{(}\PYG{n}{loc}\PYG{o}{=}\PYG{l+s+s1}{\PYGZsq{}}\PYG{l+s+s1}{best}\PYG{l+s+s1}{\PYGZsq{}}\PYG{p}{)}\PYG{p}{;}
\end{sphinxVerbatim}

\noindent\sphinxincludegraphics{{aula-15-interpolacao-newton_13_0}.png}


\section{Comparação (zoom)}
\label{\detokenize{aula-15-interpolacao-newton:comparacao-zoom}}
\begin{sphinxVerbatim}[commandchars=\\\{\}]
\PYG{c+c1}{\PYGZsh{} interpoladores}
\PYG{n}{x} \PYG{o}{=} \PYG{n}{np}\PYG{o}{.}\PYG{n}{linspace}\PYG{p}{(}\PYG{o}{\PYGZhy{}}\PYG{l+m+mf}{0.4}\PYG{p}{,}\PYG{o}{\PYGZhy{}}\PYG{l+m+mf}{0.3}\PYG{p}{,}\PYG{l+m+mi}{30}\PYG{p}{,}\PYG{n}{endpoint}\PYG{o}{=}\PYG{k+kc}{True}\PYG{p}{)}
\PYG{n}{plt}\PYG{o}{.}\PYG{n}{plot}\PYG{p}{(}\PYG{n}{x}\PYG{p}{,}\PYG{n}{P1}\PYG{p}{(}\PYG{n}{x}\PYG{p}{)}\PYG{p}{,}\PYG{l+s+s1}{\PYGZsq{}}\PYG{l+s+s1}{b}\PYG{l+s+s1}{\PYGZsq{}}\PYG{p}{,}\PYG{n}{label}\PYG{o}{=}\PYG{l+s+s1}{\PYGZsq{}}\PYG{l+s+s1}{\PYGZdl{}P\PYGZus{}1(x)\PYGZdl{}}\PYG{l+s+s1}{\PYGZsq{}}\PYG{p}{)}
\PYG{n}{plt}\PYG{o}{.}\PYG{n}{plot}\PYG{p}{(}\PYG{n}{x}\PYG{p}{,}\PYG{n}{P2}\PYG{p}{(}\PYG{n}{x}\PYG{p}{)}\PYG{p}{,}\PYG{l+s+s1}{\PYGZsq{}}\PYG{l+s+s1}{c}\PYG{l+s+s1}{\PYGZsq{}}\PYG{p}{,}\PYG{n}{label}\PYG{o}{=}\PYG{l+s+s1}{\PYGZsq{}}\PYG{l+s+s1}{\PYGZdl{}P\PYGZus{}2(x)\PYGZdl{}}\PYG{l+s+s1}{\PYGZsq{}}\PYG{p}{)}
\PYG{n}{plt}\PYG{o}{.}\PYG{n}{plot}\PYG{p}{(}\PYG{n}{x}\PYG{p}{,}\PYG{n}{P3}\PYG{p}{(}\PYG{n}{x}\PYG{p}{)}\PYG{p}{,}\PYG{l+s+s1}{\PYGZsq{}}\PYG{l+s+s1}{g}\PYG{l+s+s1}{\PYGZsq{}}\PYG{p}{,}\PYG{n}{label}\PYG{o}{=}\PYG{l+s+s1}{\PYGZsq{}}\PYG{l+s+s1}{\PYGZdl{}P\PYGZus{}3(x)\PYGZdl{}}\PYG{l+s+s1}{\PYGZsq{}}\PYG{p}{)}

\PYG{c+c1}{\PYGZsh{} ponto interpolado}
\PYG{n}{plt}\PYG{o}{.}\PYG{n}{plot}\PYG{p}{(}\PYG{p}{[}\PYG{n}{xp}\PYG{p}{,}\PYG{n}{xp}\PYG{p}{,}\PYG{n}{xp}\PYG{p}{]}\PYG{p}{,}\PYG{n}{YP}\PYG{p}{,}\PYG{l+s+s1}{\PYGZsq{}}\PYG{l+s+s1}{sr}\PYG{l+s+s1}{\PYGZsq{}}\PYG{p}{)}
\PYG{n}{plt}\PYG{o}{.}\PYG{n}{axvline}\PYG{p}{(}\PYG{n}{xp}\PYG{p}{,}\PYG{l+m+mi}{0}\PYG{p}{,}\PYG{n+nb}{max}\PYG{p}{(}\PYG{n}{YP}\PYG{p}{)}\PYG{p}{,}\PYG{n}{c}\PYG{o}{=}\PYG{l+s+s1}{\PYGZsq{}}\PYG{l+s+s1}{r}\PYG{l+s+s1}{\PYGZsq{}}\PYG{p}{,}\PYG{n}{ls}\PYG{o}{=}\PYG{l+s+s1}{\PYGZsq{}}\PYG{l+s+s1}{dashed}\PYG{l+s+s1}{\PYGZsq{}}\PYG{p}{)}

\PYG{n}{plt}\PYG{o}{.}\PYG{n}{grid}\PYG{p}{(}\PYG{p}{)}
\PYG{n}{plt}\PYG{o}{.}\PYG{n}{legend}\PYG{p}{(}\PYG{n}{loc}\PYG{o}{=}\PYG{l+s+s1}{\PYGZsq{}}\PYG{l+s+s1}{best}\PYG{l+s+s1}{\PYGZsq{}}\PYG{p}{)}\PYG{p}{;}
\end{sphinxVerbatim}

\noindent\sphinxincludegraphics{{aula-15-interpolacao-newton_15_0}.png}


\chapter{Ajuste de Curvas por Mínimos Quadrados}
\label{\detokenize{aula-16-minimos-quadrados:ajuste-de-curvas-por-minimos-quadrados}}\label{\detokenize{aula-16-minimos-quadrados::doc}}

\section{Regressão Linear}
\label{\detokenize{aula-16-minimos-quadrados:regressao-linear}}
\sphinxAtStartPar
O exemplo mais simples de aproximação por mínimos quadrados é ajustar uma reta a um conjunto de pares de observações \((x_1, y_1), (x_2, y_2), \dots, (x_n, y_n)\). A expressão matemática do ajuste por uma reta é
\begin{equation*}
\begin{split} y = a_0 + a_1 x + e \qquad (1) \end{split}
\end{equation*}
\sphinxAtStartPar
onde \(a_0\) e \(a_1\) são coeficientes representando a intersecção com o eixo \(y\) e a inclinação, respectivamente, e \(e\) é o erro ou resíduo entre o modelo e a observação, o qual pode ser representado, depois de se reorganizar a equação acima, por
\begin{equation*}
\begin{split} e = y − a_0 − a_1 x \end{split}
\end{equation*}
\sphinxAtStartPar
Portanto, o erro ou resíduo é a discrepância entre o valor verdadeiro de \(y\) e o valor aproximado, \(a_0 + a_1 x\), previsto pela equação linear.


\subsection{Critério para um Melhor Ajuste}
\label{\detokenize{aula-16-minimos-quadrados:criterio-para-um-melhor-ajuste}}
\sphinxAtStartPar
Uma estratégia para ajustar uma melhor reta pelos dados é minimizar a soma dos quadrados dos resíduos entre o \(y\) medido e o \(y\) calculado com o modelo linear
\begin{equation*}
\begin{split} S_r = \sum _{i = 1}^{n} e_i^2 = \sum _{i = 1}^{n} (y_{i,medido} - y_{i,modelo})^2 = \sum _{i = 1}^{n} (y_i - a_0 - a_1 x_i)^2 \qquad (2) \end{split}
\end{equation*}
\sphinxAtStartPar
Esse critério tem diversas vantagens, incluindo o fato de que ele fornece uma única reta para um dado conjunto de dados.


\subsection{Ajuste por Mínimos Quadrados por uma Reta}
\label{\detokenize{aula-16-minimos-quadrados:ajuste-por-minimos-quadrados-por-uma-reta}}
\sphinxAtStartPar
Para determinar os valores de \(a_0\) e \(a_1\), a Equação (2) é derivada com relação a cada coeficiente
\begin{equation*}
\begin{split}
\dfrac{\partial S_r}{\partial a_0} = - 2 \sum (y_i - a_0 - a_1 x_i) \\
\dfrac{\partial S_r}{\partial a_1} = - 2 \sum [(y_i - a_0 - a_1 x_i) x_i]
\end{split}
\end{equation*}
\sphinxAtStartPar
Igualando essas derivadas a zero será obtido um \(S_r\) mínimo
\begin{equation*}
\begin{split}
0 = \sum y_i - \sum a_0 - \sum a_1 x_i \\
0 = \sum y_i x_i - \sum a_0 x_i - \sum a_1 x_i^2
\end{split}
\end{equation*}
\sphinxAtStartPar
Agora, pode\sphinxhyphen{}se expressar essas equações como um conjunto de duas equações lineares simultâneas em duas variáveis (\(a_0\) e \(a_1\))
\begin{equation*}
\begin{split}
n a_0 + \left( \sum x_i \right) a_1 = \sum y_i \\
\left( \sum x_i \right) a_0 + \left( \sum x_i^2 \right) a_1 = \sum x_i y_i
\end{split}
\end{equation*}
\sphinxAtStartPar
Essas são as chamadas equações normais. Elas podem ser resolvidas simultaneamente para obter\sphinxhyphen{}se
\begin{equation*}
\begin{split}
a_1 = \dfrac{n \sum x_i y_i - \sum x_i \sum y_i}{n \sum x_i^2 - (\sum x_i)^2} \qquad (3) \\
a_0 = \overline{y} - a_1 \overline{x} \qquad (4)
\end{split}
\end{equation*}
\sphinxAtStartPar
onde \(\overline{y}\) e \(\overline{x}\) são as médias de \(y\) e \(x\), respectivamente.


\subsection{Quantificação do Erro da Regressão Linear}
\label{\detokenize{aula-16-minimos-quadrados:quantificacao-do-erro-da-regressao-linear}}
\sphinxAtStartPar
Um “desvio padrão” para a reta de regressão pode ser determinado por
\begin{equation*}
\begin{split} s_{y/x} = \sqrt{\dfrac{S_r}{n-2}} \end{split}
\end{equation*}
\sphinxAtStartPar
onde \(s_{y/x}\) é chamado erro padrão da estimativa. O subscrito “\(y/x\)” indica que o erro é para um valor previsto de \(y\) correspondente a um valor particular de \(x\). Além disso, observe que agora estamos dividindo por \(n − 2\) porque duas estimativas provenientes dos dados —\(a_0\) e \(a_1\) — foram usadas para calcular \(S_r\); portanto, perdemos dois graus de liberdade.

\sphinxAtStartPar
Exatamente como no caso do desvio padrão, o erro padrão da estimativa quantifica a dispersão dos dados. Entretanto, erro padrão da estimativa quantifica a dispersão em torno da reta de regressão, em contraste com o desvio padrão original que quantificava a dispersão em torno da média.

\sphinxAtStartPar
A fim de quantificar o “quão bom” é um ajuste de reta, dois conceitos utilizados são os coeficiente de determinação (\(r^2\)) e o coeficiente de correlação (\(r = \sqrt{r^2}\)).
\begin{equation*}
\begin{split} r^2 = \dfrac{S_t - S_r}{S_t} \end{split}
\end{equation*}\begin{equation*}
\begin{split} S_t = \sum _{i = 1}^n (y_i − \overline{y})^2 \end{split}
\end{equation*}
\sphinxAtStartPar
Para um ajuste perfeito, \(S_r = 0\) e \(r = r_2 = 1\), significando que a reta explica 100\% da variação dos dados. Para \(r = r_2 = 0\), \(S_r = St\) e o ajuste não representa nenhuma melhora. Uma formulação alternativa para \(r\) que é mais conveniente para implementação computacional é
\begin{equation*}
\begin{split} r = \dfrac{n \sum x_i y_i - (\sum x_i)(\sum y_i)}{\sqrt{n \sum x_i^2 - (\sum x_i)^2} \sqrt{n \sum y_i^2 - (\sum y_i)^2}} \qquad (5) \end{split}
\end{equation*}

\section{Regressão Polinomial}
\label{\detokenize{aula-16-minimos-quadrados:regressao-polinomial}}
\sphinxAtStartPar
O procedimento dos mínimos quadrados pode ser prontamente estendido para ajustar dados por um polinômio de grau mais alto. Por exemplo, suponha que se queira ajustar um polinômio de segundo grau ou quadrático
\begin{equation*}
\begin{split} y = a_0 + a_1 x + a_2 x_2 + e \qquad (6) \end{split}
\end{equation*}
\sphinxAtStartPar
Nesse caso, a soma dos quadrados dos resíduos é
\begin{equation*}
\begin{split} S_r = \sum _{i = 1}^n (y_i − a_0 − a_1 x_i − a_2 x_i^2)^2 \qquad (7) \end{split}
\end{equation*}
\sphinxAtStartPar
Seguindo um procedimento análogo ao anterior, toma\sphinxhyphen{}se a derivada da Equação (7) com relação a cada um dos coeficientes desconhecidos do polinômio, como em
\begin{equation*}
\begin{split}
\dfrac{\partial S_r}{\partial a_0} = - 2 \sum (y_i - a_0 - a_1 x_i - a_2 x_i^2) \\
\dfrac{\partial S_r}{\partial a_1} = - 2 \sum [(y_i - a_0 - a_1 x_i - a_2 x_i^2) x_i] \\
\dfrac{\partial S_r}{\partial a_2} = - 2 \sum [(y_i - a_0 - a_1 x_i - a_2 x_i^2) x_i^2]
\end{split}
\end{equation*}
\sphinxAtStartPar
Essas equações podem ser igualadas a zero e reorganizadas para determinar o seguinte conjunto de equações normais
\begin{equation*}
\begin{split}
n a_0 + \left( \sum x_i \right) a_1 + \left( \sum x_i^2 \right) a_2 = \sum y_i \\
\left( \sum x_i \right) a_0 + \left( \sum x_i^2 \right) a_1 + \left( \sum x_i^3 \right) a_2 = \sum x_i y_i \\
\left( \sum x_i^2 \right) a_0 + \left( \sum x_i^3 \right) a_1 + \left( \sum x_i^4 \right) a_2 = \sum x_i^2 y_i
\end{split}
\end{equation*}
\sphinxAtStartPar
Nesse caso, vê\sphinxhyphen{}se que o problema de determinar o polinômio de segundo grau por mínimos quadrados é equivalente a resolver um sistema de três equações lineares simultâneas. Na forma matricial, temos
\begin{equation*}
\begin{split}
\begin{bmatrix}
n & \sum x_i & \sum x_i^2 \\
\sum x_i & \sum x_i^2 & \sum x_i^3 \\
\sum x_i^2 & \sum x_i^3 & \sum x_i^4
\end{bmatrix}
\begin{bmatrix}
a_0 \\
a_1 \\
a_2
\end{bmatrix} =
\begin{bmatrix}
\sum y_i \\
\sum x_i y_i \\
\sum x_i^2 y_i
\end{bmatrix} \qquad (8)
\end{split}
\end{equation*}
\sphinxAtStartPar
O caso bidimensional pode ser facilmente estendido para um polinômio de grau \(m\).
\begin{equation*}
\begin{split} y = a_0 + a_1 x + a_2 x^2 + \dots + a_m x^m + e \end{split}
\end{equation*}

\section{Nota}
\label{\detokenize{aula-16-minimos-quadrados:nota}}
\sphinxAtStartPar
O presente texto (conteúdo teórico) é um resumo baseado no livro Métodos Numéricos para Engenharia (Chapra e Canale).


\section{Motivação: Evolução da População Paraibana}
\label{\detokenize{aula-16-minimos-quadrados:motivacao-evolucao-da-populacao-paraibana}}
\sphinxAtStartPar
Este exemplo lê um arquivo CSV, plota o gráfico de dispersão, compara modelos e formata os dados para visualização.

\begin{sphinxVerbatim}[commandchars=\\\{\}]
\PYG{k+kn}{import} \PYG{n+nn}{pandas} \PYG{k}{as} \PYG{n+nn}{pd}
\PYG{k+kn}{from} \PYG{n+nn}{bokeh}\PYG{n+nn}{.}\PYG{n+nn}{io} \PYG{k+kn}{import} \PYG{n}{output\PYGZus{}notebook}
\PYG{k+kn}{from} \PYG{n+nn}{bokeh}\PYG{n+nn}{.}\PYG{n+nn}{plotting} \PYG{k+kn}{import} \PYG{n}{figure}\PYG{p}{,} \PYG{n}{show}
\PYG{k+kn}{from} \PYG{n+nn}{bokeh}\PYG{n+nn}{.}\PYG{n+nn}{models} \PYG{k+kn}{import} \PYG{n}{PrintfTickFormatter}\PYG{p}{,} \PYG{n}{Range1d}

\PYG{k+kn}{from} \PYG{n+nn}{scipy} \PYG{k+kn}{import} \PYG{n}{stats}
\PYG{k+kn}{from} \PYG{n+nn}{scipy} \PYG{k+kn}{import} \PYG{n}{interpolate}

\PYG{c+c1}{\PYGZsh{} tabela de dados}
\PYG{n}{df} \PYG{o}{=} \PYG{n}{pd}\PYG{o}{.}\PYG{n}{read\PYGZus{}csv}\PYG{p}{(}\PYG{l+s+s2}{\PYGZdq{}}\PYG{l+s+s2}{file\PYGZhy{}pop\PYGZhy{}pb.csv}\PYG{l+s+s2}{\PYGZdq{}}\PYG{p}{)}
\PYG{n}{x} \PYG{o}{=} \PYG{n}{df}\PYG{p}{[}\PYG{l+s+s1}{\PYGZsq{}}\PYG{l+s+s1}{ano}\PYG{l+s+s1}{\PYGZsq{}}\PYG{p}{]}
\PYG{n}{y} \PYG{o}{=} \PYG{n}{df}\PYG{p}{[}\PYG{l+s+s1}{\PYGZsq{}}\PYG{l+s+s1}{pop. urbana}\PYG{l+s+s1}{\PYGZsq{}}\PYG{p}{]}

\PYG{c+c1}{\PYGZsh{} grafico de dispersao}
\PYG{n}{f} \PYG{o}{=} \PYG{n}{figure}\PYG{p}{(}\PYG{n}{plot\PYGZus{}width}\PYG{o}{=}\PYG{l+m+mi}{400}\PYG{p}{,} \PYG{n}{plot\PYGZus{}height}\PYG{o}{=}\PYG{l+m+mi}{400}\PYG{p}{)}
\PYG{n}{f}\PYG{o}{.}\PYG{n}{scatter}\PYG{p}{(}\PYG{n}{x}\PYG{p}{,}\PYG{n}{y}\PYG{p}{,}\PYG{n}{fill\PYGZus{}color}\PYG{o}{=}\PYG{l+s+s2}{\PYGZdq{}}\PYG{l+s+s2}{blue}\PYG{l+s+s2}{\PYGZdq{}}\PYG{p}{,}\PYG{n}{radius}\PYG{o}{=}\PYG{l+m+mi}{1}\PYG{p}{,}\PYG{n}{alpha}\PYG{o}{=}\PYG{l+m+mi}{1}\PYG{p}{)}

\PYG{c+c1}{\PYGZsh{} formatacao}
\PYG{n}{f}\PYG{o}{.}\PYG{n}{background\PYGZus{}fill\PYGZus{}color} \PYG{o}{=} \PYG{l+s+s2}{\PYGZdq{}}\PYG{l+s+s2}{blue}\PYG{l+s+s2}{\PYGZdq{}}
\PYG{n}{f}\PYG{o}{.}\PYG{n}{background\PYGZus{}fill\PYGZus{}alpha} \PYG{o}{=} \PYG{l+m+mf}{0.02}
\PYG{n}{f}\PYG{o}{.}\PYG{n}{x\PYGZus{}range}\PYG{o}{=}\PYG{n}{Range1d}\PYG{p}{(}\PYG{l+m+mi}{1960}\PYG{p}{,} \PYG{l+m+mi}{2020}\PYG{p}{)}
\PYG{n}{e} \PYG{o}{=} \PYG{l+m+mf}{5.e5}
\PYG{n}{f}\PYG{o}{.}\PYG{n}{y\PYGZus{}range}\PYG{o}{=}\PYG{n}{Range1d}\PYG{p}{(}\PYG{n}{y}\PYG{o}{.}\PYG{n}{min}\PYG{p}{(}\PYG{p}{)} \PYG{o}{\PYGZhy{}} \PYG{n}{e}\PYG{p}{,}\PYG{n}{y}\PYG{o}{.}\PYG{n}{max}\PYG{p}{(}\PYG{p}{)} \PYG{o}{+} \PYG{n}{e}\PYG{p}{)}
\PYG{n}{f}\PYG{o}{.}\PYG{n}{xaxis}\PYG{o}{.}\PYG{n}{axis\PYGZus{}label} \PYG{o}{=} \PYG{l+s+s2}{\PYGZdq{}}\PYG{l+s+s2}{Ano}\PYG{l+s+s2}{\PYGZdq{}}
\PYG{n}{f}\PYG{o}{.}\PYG{n}{yaxis}\PYG{o}{.}\PYG{n}{axis\PYGZus{}label} \PYG{o}{=} \PYG{l+s+s2}{\PYGZdq{}}\PYG{l+s+s2}{Pop. urbana \PYGZhy{} PB}\PYG{l+s+s2}{\PYGZdq{}}
\PYG{n}{f}\PYG{o}{.}\PYG{n}{yaxis}\PYG{p}{[}\PYG{l+m+mi}{0}\PYG{p}{]}\PYG{o}{.}\PYG{n}{formatter} \PYG{o}{=} \PYG{n}{PrintfTickFormatter}\PYG{p}{(}\PYG{n+nb}{format}\PYG{o}{=}\PYG{l+s+s2}{\PYGZdq{}}\PYG{l+s+si}{\PYGZpc{}1.2e}\PYG{l+s+s2}{\PYGZdq{}}\PYG{p}{)}

\PYG{c+c1}{\PYGZsh{} interpolacao}

\PYG{n}{f}\PYG{o}{.}\PYG{n}{line}\PYG{p}{(}\PYG{n}{x}\PYG{p}{,}\PYG{n}{y}\PYG{p}{,}\PYG{n}{color}\PYG{o}{=}\PYG{l+s+s1}{\PYGZsq{}}\PYG{l+s+s1}{green}\PYG{l+s+s1}{\PYGZsq{}}\PYG{p}{)}

\PYG{c+c1}{\PYGZsh{} estimando o valor da populacao em 1975 e 1982 por interpolacao}
\PYG{n}{p} \PYG{o}{=} \PYG{n}{interpolate}\PYG{o}{.}\PYG{n}{interp1d}\PYG{p}{(}\PYG{n}{x}\PYG{p}{,} \PYG{n}{y}\PYG{p}{)}
\PYG{n}{xx} \PYG{o}{=} \PYG{p}{[}\PYG{l+m+mi}{1975}\PYG{p}{,}\PYG{l+m+mi}{1982}\PYG{p}{]}
\PYG{n}{yy} \PYG{o}{=} \PYG{n}{p}\PYG{p}{(}\PYG{n}{xx}\PYG{p}{)}
\PYG{n+nb}{print}\PYG{p}{(}\PYG{l+s+s2}{\PYGZdq{}}\PYG{l+s+s2}{População obtida por interpolação linear em: 1975 = }\PYG{l+s+si}{\PYGZob{}:d\PYGZcb{}}\PYG{l+s+s2}{; 1982 = }\PYG{l+s+si}{\PYGZob{}:d\PYGZcb{}}\PYG{l+s+s2}{; }\PYG{l+s+s2}{\PYGZdq{}}\PYG{o}{.}\PYG{n}{format}\PYG{p}{(}\PYG{n+nb}{int}\PYG{p}{(}\PYG{n}{yy}\PYG{p}{[}\PYG{l+m+mi}{0}\PYG{p}{]}\PYG{p}{)}\PYG{p}{,}\PYG{n+nb}{int}\PYG{p}{(}\PYG{n}{yy}\PYG{p}{[}\PYG{l+m+mi}{1}\PYG{p}{]}\PYG{p}{)}\PYG{p}{)}\PYG{p}{)}
\PYG{n}{f}\PYG{o}{.}\PYG{n}{square}\PYG{p}{(}\PYG{n}{xx}\PYG{p}{,}\PYG{n}{yy}\PYG{p}{,}\PYG{n}{color}\PYG{o}{=}\PYG{l+s+s1}{\PYGZsq{}}\PYG{l+s+s1}{black}\PYG{l+s+s1}{\PYGZsq{}}\PYG{p}{,}\PYG{n}{line\PYGZus{}width}\PYG{o}{=}\PYG{l+m+mi}{4}\PYG{p}{)}

\PYG{c+c1}{\PYGZsh{} ajuste por minimos quadrados}

\PYG{n}{slope}\PYG{p}{,} \PYG{n}{intercept}\PYG{p}{,} \PYG{n}{r\PYGZus{}value}\PYG{p}{,} \PYG{n}{p\PYGZus{}value}\PYG{p}{,} \PYG{n}{std\PYGZus{}err} \PYG{o}{=} \PYG{n}{stats}\PYG{o}{.}\PYG{n}{linregress}\PYG{p}{(}\PYG{n}{x}\PYG{p}{,}\PYG{n}{y}\PYG{p}{)}
\PYG{n}{px} \PYG{o}{=} \PYG{n}{intercept} \PYG{o}{+} \PYG{n}{slope}\PYG{o}{*}\PYG{n}{x}
\PYG{n}{f}\PYG{o}{.}\PYG{n}{line}\PYG{p}{(}\PYG{n}{x}\PYG{p}{,}\PYG{n}{px}\PYG{p}{,}\PYG{n}{color}\PYG{o}{=}\PYG{l+s+s1}{\PYGZsq{}}\PYG{l+s+s1}{red}\PYG{l+s+s1}{\PYGZsq{}}\PYG{p}{)}

\PYG{n}{output\PYGZus{}notebook}\PYG{p}{(}\PYG{p}{)}
\PYG{n}{show}\PYG{p}{(}\PYG{n}{f}\PYG{p}{)}
\end{sphinxVerbatim}

\begin{sphinxVerbatim}[commandchars=\\\{\}]
População obtida por interpolação linear em: 1975 = 1225580; 1982 = 1552017; 
\end{sphinxVerbatim}

\begin{sphinxVerbatim}[commandchars=\\\{\}]
\PYG{n}{df}\PYG{o}{.}\PYG{n}{set\PYGZus{}index}\PYG{p}{(}\PYG{l+s+s1}{\PYGZsq{}}\PYG{l+s+s1}{ano}\PYG{l+s+s1}{\PYGZsq{}}\PYG{p}{)}
\end{sphinxVerbatim}

\begin{sphinxVerbatim}[commandchars=\\\{\}]
      pop. urbana
ano              
1970      1002156
1980      1449004
1991      2015576
2000      2447212
2010      2902044
\end{sphinxVerbatim}


\section{Exemplo}
\label{\detokenize{aula-16-minimos-quadrados:exemplo}}
\sphinxAtStartPar
Vamos resolver um problema de regressão linear por meio das equações normais passo a passo. Em primeiro lugar, importemos os módulos de computação numérica e de plotagem.

\begin{sphinxVerbatim}[commandchars=\\\{\}]
\PYG{c+c1}{\PYGZsh{} importação de módulos }
\PYG{k+kn}{import} \PYG{n+nn}{numpy} \PYG{k}{as} \PYG{n+nn}{np}
\PYG{k+kn}{import} \PYG{n+nn}{matplotlib}\PYG{n+nn}{.}\PYG{n+nn}{pyplot} \PYG{k}{as} \PYG{n+nn}{plt}
\PYG{k+kn}{import} \PYG{n+nn}{sympy} \PYG{k}{as} \PYG{n+nn}{sy}
\PYG{o}{\PYGZpc{}}\PYG{k}{matplotlib} inline
\end{sphinxVerbatim}

\sphinxAtStartPar
Consideremos a simples tabela abaixo de um experimento fictício que busca correlacionar densidade e diâmetro médio de grãos em alimentos.


\begin{savenotes}\sphinxattablestart
\centering
\begin{tabulary}{\linewidth}[t]{|T|T|}
\hline
\sphinxstyletheadfamily 
\sphinxAtStartPar
densidade
&\sphinxstyletheadfamily 
\sphinxAtStartPar
diâmetro médio de grão
\\
\hline
\sphinxAtStartPar
1.0
&
\sphinxAtStartPar
0.5
\\
\hline
\sphinxAtStartPar
2.1
&
\sphinxAtStartPar
2.5
\\
\hline
\sphinxAtStartPar
3.3
&
\sphinxAtStartPar
2.0
\\
\hline
\sphinxAtStartPar
4.5
&
\sphinxAtStartPar
4.2
\\
\hline
\end{tabulary}
\par
\sphinxattableend\end{savenotes}

\sphinxAtStartPar
Vamos escrever os dados como \sphinxstyleemphasis{arrays}.

\begin{sphinxVerbatim}[commandchars=\\\{\}]
\PYG{c+c1}{\PYGZsh{} tabela de dados}
\PYG{n}{x} \PYG{o}{=} \PYG{n}{np}\PYG{o}{.}\PYG{n}{array}\PYG{p}{(}\PYG{p}{[}\PYG{l+m+mi}{1}\PYG{p}{,}\PYG{l+m+mi}{2}\PYG{p}{,}\PYG{l+m+mi}{3}\PYG{p}{,}\PYG{l+m+mi}{4}\PYG{p}{]}\PYG{p}{)} \PYG{c+c1}{\PYGZsh{} densidade}
\PYG{n}{y} \PYG{o}{=} \PYG{n}{np}\PYG{o}{.}\PYG{n}{array}\PYG{p}{(}\PYG{p}{[}\PYG{l+m+mf}{0.5}\PYG{p}{,}\PYG{l+m+mf}{2.5}\PYG{p}{,}\PYG{l+m+mf}{2.0}\PYG{p}{,}\PYG{l+m+mf}{4.0}\PYG{p}{]}\PYG{p}{)} \PYG{c+c1}{\PYGZsh{} diâmetro}
\end{sphinxVerbatim}

\sphinxAtStartPar
Agora, calculamos os coeficientes linear \(\alpha_0\) e angular \(\alpha_1\) pelas fórmulas das equações normais vistas em aula.

\begin{sphinxVerbatim}[commandchars=\\\{\}]
\PYG{n}{m} \PYG{o}{=} \PYG{n}{np}\PYG{o}{.}\PYG{n}{size}\PYG{p}{(}\PYG{n}{x}\PYG{p}{)}
\PYG{n}{alpha1} \PYG{o}{=} \PYG{p}{(}\PYG{n}{m}\PYG{o}{*}\PYG{n}{np}\PYG{o}{.}\PYG{n}{dot}\PYG{p}{(}\PYG{n}{x}\PYG{p}{,}\PYG{n}{y}\PYG{p}{)} \PYG{o}{\PYGZhy{}} \PYG{n}{np}\PYG{o}{.}\PYG{n}{sum}\PYG{p}{(}\PYG{n}{x}\PYG{p}{)}\PYG{o}{*}\PYG{n}{np}\PYG{o}{.}\PYG{n}{sum}\PYG{p}{(}\PYG{n}{y}\PYG{p}{)}\PYG{p}{)}\PYG{o}{/}\PYG{p}{(}\PYG{n}{m}\PYG{o}{*}\PYG{n}{np}\PYG{o}{.}\PYG{n}{dot}\PYG{p}{(}\PYG{n}{x}\PYG{p}{,}\PYG{n}{x}\PYG{p}{)}\PYG{o}{\PYGZhy{}}\PYG{n}{np}\PYG{o}{.}\PYG{n}{sum}\PYG{p}{(}\PYG{n}{x}\PYG{p}{)}\PYG{o}{*}\PYG{o}{*}\PYG{l+m+mi}{2}\PYG{p}{)}
\PYG{n}{alpha0} \PYG{o}{=} \PYG{n}{np}\PYG{o}{.}\PYG{n}{mean}\PYG{p}{(}\PYG{n}{y}\PYG{p}{)} \PYG{o}{\PYGZhy{}} \PYG{n}{alpha1}\PYG{o}{*}\PYG{n}{np}\PYG{o}{.}\PYG{n}{mean}\PYG{p}{(}\PYG{n}{x}\PYG{p}{)}
\end{sphinxVerbatim}

\sphinxAtStartPar
Podemos agora escrever a equação da reta de regressão usando o \sphinxstyleemphasis{array} \sphinxcode{\sphinxupquote{x}} como abscissa. Este será o nosso \sphinxstyleemphasis{modelo de ajuste}.

\begin{sphinxVerbatim}[commandchars=\\\{\}]
\PYG{n}{y2} \PYG{o}{=} \PYG{n}{alpha0} \PYG{o}{+} \PYG{n}{alpha1}\PYG{o}{*}\PYG{n}{x}
\end{sphinxVerbatim}

\sphinxAtStartPar
Enfim, plotamos o gráfico de dispersão dos valores \sphinxstyleemphasis{medidos} juntamente com o \sphinxstyleemphasis{modelo de ajuste} da seguinte forma:

\begin{sphinxVerbatim}[commandchars=\\\{\}]
\PYG{n}{mod} \PYG{o}{=} \PYG{n}{plt}\PYG{o}{.}\PYG{n}{plot}\PYG{p}{(}\PYG{n}{x}\PYG{p}{,}\PYG{n}{y2}\PYG{p}{,}\PYG{l+s+s1}{\PYGZsq{}}\PYG{l+s+s1}{r:}\PYG{l+s+s1}{\PYGZsq{}}\PYG{p}{)}\PYG{p}{;} \PYG{c+c1}{\PYGZsh{} modelo}
\PYG{n}{med} \PYG{o}{=} \PYG{n}{plt}\PYG{o}{.}\PYG{n}{scatter}\PYG{p}{(}\PYG{n}{x}\PYG{p}{,}\PYG{n}{y}\PYG{p}{,}\PYG{n}{c}\PYG{o}{=}\PYG{l+s+s1}{\PYGZsq{}}\PYG{l+s+s1}{b}\PYG{l+s+s1}{\PYGZsq{}}\PYG{p}{)}\PYG{p}{;} \PYG{c+c1}{\PYGZsh{} medição}
\PYG{n}{plt}\PYG{o}{.}\PYG{n}{legend}\PYG{p}{(}\PYG{p}{\PYGZob{}}\PYG{l+s+s1}{\PYGZsq{}}\PYG{l+s+s1}{modelo de ajuste}\PYG{l+s+s1}{\PYGZsq{}}\PYG{p}{:}\PYG{n}{mod}\PYG{p}{,} \PYG{l+s+s1}{\PYGZsq{}}\PYG{l+s+s1}{medição}\PYG{l+s+s1}{\PYGZsq{}}\PYG{p}{:}\PYG{n}{med}\PYG{p}{\PYGZcb{}}\PYG{p}{)}\PYG{p}{;} \PYG{c+c1}{\PYGZsh{} legenda}

\PYG{c+c1}{\PYGZsh{} esta linha adiciona a equação de ajuste ao gráfico na posição (x,y) = (2.8,0.8)}
\PYG{c+c1}{\PYGZsh{} com fonte tamanho 14 e cor RGB = [0.4,0.5,0.4].}
\PYG{n}{plt}\PYG{o}{.}\PYG{n}{annotate}\PYG{p}{(}\PYG{l+s+s1}{\PYGZsq{}}\PYG{l+s+s1}{y= }\PYG{l+s+si}{\PYGZob{}0:.2f\PYGZcb{}}\PYG{l+s+s1}{ + }\PYG{l+s+si}{\PYGZob{}1:.2f\PYGZcb{}}\PYG{l+s+s1}{x}\PYG{l+s+s1}{\PYGZsq{}}\PYG{o}{.}\PYG{n}{format}\PYG{p}{(}\PYG{n}{alpha0}\PYG{p}{,}\PYG{n}{alpha1}\PYG{p}{)}\PYG{p}{,}\PYG{p}{(}\PYG{l+m+mf}{2.8}\PYG{p}{,}\PYG{l+m+mf}{0.8}\PYG{p}{)}\PYG{p}{,}\PYG{n}{fontsize}\PYG{o}{=}\PYG{l+m+mi}{14}\PYG{p}{,}\PYG{n}{c}\PYG{o}{=}\PYG{p}{[}\PYG{l+m+mf}{0.4}\PYG{p}{,}\PYG{l+m+mf}{0.5}\PYG{p}{,}\PYG{l+m+mf}{0.4}\PYG{p}{]}\PYG{p}{)}\PYG{p}{;}
\end{sphinxVerbatim}

\noindent\sphinxincludegraphics{{aula-16-minimos-quadrados_15_0}.png}

\sphinxAtStartPar
Na prática, podemos calcular regressão linear usando o módulo \sphinxcode{\sphinxupquote{scipy.stats}}. Vide \sphinxstyleemphasis{Code session 7}.


\chapter{Ajuste de curvas: caso não\sphinxhyphen{}linear}
\label{\detokenize{aula-17-ajusteNaoLinear:ajuste-de-curvas-caso-nao-linear}}\label{\detokenize{aula-17-ajusteNaoLinear::doc}}
\begin{sphinxVerbatim}[commandchars=\\\{\}]
\PYG{k+kn}{import} \PYG{n+nn}{numpy} \PYG{k}{as} \PYG{n+nn}{np} 
\PYG{k+kn}{import} \PYG{n+nn}{matplotlib}\PYG{n+nn}{.}\PYG{n+nn}{pyplot} \PYG{k}{as} \PYG{n+nn}{plt}
\PYG{o}{\PYGZpc{}}\PYG{k}{matplotlib} inline
\end{sphinxVerbatim}


\section{Motivação: comportamento de fluidos lei de potência}
\label{\detokenize{aula-17-ajusteNaoLinear:motivacao-comportamento-de-fluidos-lei-de-potencia}}
\begin{sphinxVerbatim}[commandchars=\\\{\}]
\PYG{n}{gamma} \PYG{o}{=} \PYG{n}{np}\PYG{o}{.}\PYG{n}{linspace}\PYG{p}{(}\PYG{l+m+mi}{0}\PYG{p}{,}\PYG{l+m+mi}{1}\PYG{p}{,}\PYG{l+m+mi}{20}\PYG{p}{,}\PYG{k+kc}{True}\PYG{p}{)}
\PYG{n}{n1} \PYG{o}{=} \PYG{l+m+mf}{1.5} \PYG{c+c1}{\PYGZsh{} dilatante}
\PYG{n}{n2} \PYG{o}{=} \PYG{l+m+mf}{0.7} \PYG{c+c1}{\PYGZsh{} pseudoplastico}
\PYG{n}{k} \PYG{o}{=} \PYG{l+m+mf}{0.1}
\PYG{n}{tau1} \PYG{o}{=} \PYG{n}{k}\PYG{o}{*}\PYG{n}{gamma}\PYG{o}{*}\PYG{o}{*}\PYG{n}{n1}
\PYG{n}{tau2} \PYG{o}{=} \PYG{n}{k}\PYG{o}{*}\PYG{n}{gamma}\PYG{o}{*}\PYG{o}{*}\PYG{n}{n2}
\PYG{n}{tau3} \PYG{o}{=} \PYG{n}{k}\PYG{o}{*}\PYG{n}{gamma}

\PYG{n}{plt}\PYG{o}{.}\PYG{n}{plot}\PYG{p}{(}\PYG{n}{gamma}\PYG{p}{,}\PYG{n}{tau1}\PYG{p}{,}\PYG{l+s+s1}{\PYGZsq{}}\PYG{l+s+s1}{o}\PYG{l+s+s1}{\PYGZsq{}}\PYG{p}{,}\PYG{n}{label}\PYG{o}{=}\PYG{l+s+s1}{\PYGZsq{}}\PYG{l+s+s1}{dilatante}\PYG{l+s+s1}{\PYGZsq{}}\PYG{p}{)}
\PYG{n}{plt}\PYG{o}{.}\PYG{n}{plot}\PYG{p}{(}\PYG{n}{gamma}\PYG{p}{,}\PYG{n}{tau2}\PYG{p}{,}\PYG{l+s+s1}{\PYGZsq{}}\PYG{l+s+s1}{o}\PYG{l+s+s1}{\PYGZsq{}}\PYG{p}{,}\PYG{n}{label}\PYG{o}{=}\PYG{l+s+s1}{\PYGZsq{}}\PYG{l+s+s1}{pseudoplástico}\PYG{l+s+s1}{\PYGZsq{}}\PYG{p}{)}
\PYG{n}{plt}\PYG{o}{.}\PYG{n}{plot}\PYG{p}{(}\PYG{n}{gamma}\PYG{p}{,}\PYG{n}{tau3}\PYG{p}{,}\PYG{l+s+s1}{\PYGZsq{}}\PYG{l+s+s1}{\PYGZhy{}o}\PYG{l+s+s1}{\PYGZsq{}}\PYG{p}{,}\PYG{n}{label}\PYG{o}{=}\PYG{l+s+s1}{\PYGZsq{}}\PYG{l+s+s1}{newtoniano}\PYG{l+s+s1}{\PYGZsq{}}\PYG{p}{)}
\PYG{n}{plt}\PYG{o}{.}\PYG{n}{legend}\PYG{p}{(}\PYG{p}{)}
\PYG{n}{plt}\PYG{o}{.}\PYG{n}{xlabel}\PYG{p}{(}\PYG{l+s+s1}{\PYGZsq{}}\PYG{l+s+s1}{\PYGZdl{}}\PYG{l+s+s1}{\PYGZbs{}}\PYG{l+s+s1}{dot}\PYG{l+s+s1}{\PYGZob{}}\PYG{l+s+s1}{\PYGZbs{}}\PYG{l+s+s1}{gamma\PYGZcb{}\PYGZdl{}}\PYG{l+s+s1}{\PYGZsq{}}\PYG{p}{)}
\PYG{n}{plt}\PYG{o}{.}\PYG{n}{ylabel}\PYG{p}{(}\PYG{l+s+s1}{\PYGZsq{}}\PYG{l+s+s1}{\PYGZdl{}}\PYG{l+s+s1}{\PYGZbs{}}\PYG{l+s+s1}{sigma\PYGZcb{}\PYGZdl{}}\PYG{l+s+s1}{\PYGZsq{}}\PYG{p}{)}
\PYG{n}{plt}\PYG{o}{.}\PYG{n}{grid}\PYG{p}{(}\PYG{n}{axis}\PYG{o}{=}\PYG{l+s+s1}{\PYGZsq{}}\PYG{l+s+s1}{x}\PYG{l+s+s1}{\PYGZsq{}}\PYG{p}{)}
\PYG{n}{plt}\PYG{o}{.}\PYG{n}{title}\PYG{p}{(}\PYG{l+s+s1}{\PYGZsq{}}\PYG{l+s+s1}{Dispersão: fluidos lei de potência}\PYG{l+s+s1}{\PYGZsq{}}\PYG{p}{)}
\PYG{n}{plt}\PYG{o}{.}\PYG{n}{tick\PYGZus{}params}\PYG{p}{(}\PYG{n}{axis}\PYG{o}{=}\PYG{l+s+s1}{\PYGZsq{}}\PYG{l+s+s1}{both}\PYG{l+s+s1}{\PYGZsq{}}\PYG{p}{,}\PYG{n}{which}\PYG{o}{=}\PYG{l+s+s1}{\PYGZsq{}}\PYG{l+s+s1}{both}\PYG{l+s+s1}{\PYGZsq{}}\PYG{p}{,}\PYG{n}{labelbottom}\PYG{o}{=}\PYG{k+kc}{False}\PYG{p}{,}\PYG{n}{labelleft}\PYG{o}{=}\PYG{k+kc}{False}\PYG{p}{)}
\end{sphinxVerbatim}

\noindent\sphinxincludegraphics{{aula-17-ajusteNaoLinear_3_0}.png}


\section{Exemplo}
\label{\detokenize{aula-17-ajusteNaoLinear:exemplo}}
\sphinxAtStartPar
Determine os parâmetros \(a\) e \(b\) de modo que \(y = f(x) = ae^{bx}\) ajuste os seguintes dados no sentido de mínimos quadrados:


\begin{savenotes}\sphinxattablestart
\centering
\begin{tabulary}{\linewidth}[t]{|T|T|}
\hline
\sphinxstyletheadfamily 
\sphinxAtStartPar
x
&\sphinxstyletheadfamily 
\sphinxAtStartPar
y
\\
\hline
\sphinxAtStartPar
1.2
&
\sphinxAtStartPar
7.5
\\
\hline
\sphinxAtStartPar
2.8
&
\sphinxAtStartPar
16.1
\\
\hline
\sphinxAtStartPar
4.3
&
\sphinxAtStartPar
38.9
\\
\hline
\sphinxAtStartPar
5.4
&
\sphinxAtStartPar
67.0
\\
\hline
\sphinxAtStartPar
6.8
&
\sphinxAtStartPar
146.6
\\
\hline
\sphinxAtStartPar
7.9
&
\sphinxAtStartPar
266.2
\\
\hline
\end{tabulary}
\par
\sphinxattableend\end{savenotes}

\sphinxAtStartPar
Plotando o gráfico de dispersão:

\begin{sphinxVerbatim}[commandchars=\\\{\}]
\PYG{n}{x} \PYG{o}{=} \PYG{n}{np}\PYG{o}{.}\PYG{n}{array}\PYG{p}{(}\PYG{p}{[}\PYG{l+m+mf}{1.2}\PYG{p}{,}\PYG{l+m+mf}{2.8}\PYG{p}{,}\PYG{l+m+mf}{4.3}\PYG{p}{,}\PYG{l+m+mf}{5.4}\PYG{p}{,}\PYG{l+m+mf}{6.8}\PYG{p}{,}\PYG{l+m+mf}{7.9}\PYG{p}{]}\PYG{p}{)}
\PYG{n}{y} \PYG{o}{=} \PYG{n}{np}\PYG{o}{.}\PYG{n}{array}\PYG{p}{(}\PYG{p}{[}\PYG{l+m+mf}{7.5}\PYG{p}{,}\PYG{l+m+mf}{16.1}\PYG{p}{,}\PYG{l+m+mf}{38.9}\PYG{p}{,}\PYG{l+m+mf}{67.0}\PYG{p}{,}\PYG{l+m+mf}{146.6}\PYG{p}{,}\PYG{l+m+mf}{266.2}\PYG{p}{]}\PYG{p}{)}
\PYG{n}{plt}\PYG{o}{.}\PYG{n}{plot}\PYG{p}{(}\PYG{n}{x}\PYG{p}{,}\PYG{n}{y}\PYG{p}{,}\PYG{l+s+s1}{\PYGZsq{}}\PYG{l+s+s1}{o}\PYG{l+s+s1}{\PYGZsq{}}\PYG{p}{)}\PYG{p}{;}
\end{sphinxVerbatim}

\noindent\sphinxincludegraphics{{aula-17-ajusteNaoLinear_6_0}.png}

\sphinxAtStartPar
\sphinxstylestrong{Teste de alinhamento:} façamos a linearização
\begin{equation*}
\begin{split}Y = Z + bx,\end{split}
\end{equation*}
\sphinxAtStartPar
onde \(Y = \log(y)\) e \(Z = \log(a)\).

\sphinxAtStartPar
Plotemos a dispersão \((x,Y)\).

\begin{sphinxVerbatim}[commandchars=\\\{\}]
\PYG{n}{Y} \PYG{o}{=} \PYG{n}{np}\PYG{o}{.}\PYG{n}{log}\PYG{p}{(}\PYG{n}{y}\PYG{p}{)}
\PYG{n}{plt}\PYG{o}{.}\PYG{n}{plot}\PYG{p}{(}\PYG{n}{x}\PYG{p}{,}\PYG{n}{Y}\PYG{p}{)}\PYG{p}{;}
\end{sphinxVerbatim}

\noindent\sphinxincludegraphics{{aula-17-ajusteNaoLinear_8_0}.png}

\sphinxAtStartPar
O teste do alinhamento nos diz que a função de ajuste é adequada para a regressão linear.

\sphinxAtStartPar
\sphinxstylestrong{Regressão linear:} façamos a regressão linear para buscar os parâmetros \(b\) e \(Z\) do modelo linearizado.

\begin{sphinxVerbatim}[commandchars=\\\{\}]
\PYG{k+kn}{from} \PYG{n+nn}{scipy}\PYG{n+nn}{.}\PYG{n+nn}{stats} \PYG{k+kn}{import} \PYG{n}{linregress}

\PYG{n}{b}\PYG{p}{,}\PYG{n}{Z}\PYG{p}{,}\PYG{n}{R}\PYG{p}{,}\PYG{n}{p}\PYG{p}{,}\PYG{n}{e} \PYG{o}{=} \PYG{n}{linregress}\PYG{p}{(}\PYG{n}{x}\PYG{p}{,}\PYG{n}{Y}\PYG{p}{)}
\PYG{n+nb}{print}\PYG{p}{(}\PYG{l+s+sa}{f}\PYG{l+s+s1}{\PYGZsq{}}\PYG{l+s+s1}{Inclinação = }\PYG{l+s+si}{\PYGZob{}}\PYG{n}{b}\PYG{l+s+si}{:}\PYG{l+s+s1}{.3f}\PYG{l+s+si}{\PYGZcb{}}\PYG{l+s+s1}{; offset = }\PYG{l+s+si}{\PYGZob{}}\PYG{n}{Z}\PYG{l+s+si}{:}\PYG{l+s+s1}{.3f}\PYG{l+s+si}{\PYGZcb{}}\PYG{l+s+s1}{; R2 = }\PYG{l+s+si}{\PYGZob{}}\PYG{n}{R}\PYG{o}{*}\PYG{n}{R}\PYG{l+s+si}{:}\PYG{l+s+s1}{.3f}\PYG{l+s+si}{\PYGZcb{}}\PYG{l+s+s1}{.}\PYG{l+s+s1}{\PYGZsq{}}\PYG{p}{)}
\end{sphinxVerbatim}

\begin{sphinxVerbatim}[commandchars=\\\{\}]
Inclinação = 0.537; offset = 1.332; R2 = 0.999.
\end{sphinxVerbatim}

\sphinxAtStartPar
De fato, o coeficiente \(R^2 \approx 1.0\) mostra correlação quase máxima no modelo linearizado.

\sphinxAtStartPar
\sphinxstylestrong{Comparação entre dados experimentais e ajustados:} plotaremos agora a dispersão e o modelo ajustado.

\begin{sphinxVerbatim}[commandchars=\\\{\}]
\PYG{n}{a} \PYG{o}{=} \PYG{n}{np}\PYG{o}{.}\PYG{n}{exp}\PYG{p}{(}\PYG{n}{Z}\PYG{p}{)} \PYG{c+c1}{\PYGZsh{} recupera o valor de a }
\PYG{n}{modelo} \PYG{o}{=} \PYG{k}{lambda} \PYG{n}{x}\PYG{p}{:} \PYG{n}{a}\PYG{o}{*}\PYG{n}{np}\PYG{o}{.}\PYG{n}{exp}\PYG{p}{(}\PYG{n}{b}\PYG{o}{*}\PYG{n}{x}\PYG{p}{)}

\PYG{n}{plt}\PYG{o}{.}\PYG{n}{plot}\PYG{p}{(}\PYG{n}{x}\PYG{p}{,}\PYG{n}{y}\PYG{p}{,}\PYG{l+s+s1}{\PYGZsq{}}\PYG{l+s+s1}{o}\PYG{l+s+s1}{\PYGZsq{}}\PYG{p}{,}\PYG{n}{mfc}\PYG{o}{=}\PYG{l+s+s1}{\PYGZsq{}}\PYG{l+s+s1}{None}\PYG{l+s+s1}{\PYGZsq{}}\PYG{p}{)}
\PYG{n}{plt}\PYG{o}{.}\PYG{n}{plot}\PYG{p}{(}\PYG{n}{x}\PYG{p}{,}\PYG{n}{modelo}\PYG{p}{(}\PYG{n}{x}\PYG{p}{)}\PYG{p}{,}\PYG{l+s+s1}{\PYGZsq{}}\PYG{l+s+s1}{rs}\PYG{l+s+s1}{\PYGZsq{}}\PYG{p}{,}\PYG{n}{mfc}\PYG{o}{=}\PYG{l+s+s1}{\PYGZsq{}}\PYG{l+s+s1}{None}\PYG{l+s+s1}{\PYGZsq{}}\PYG{p}{)}\PYG{p}{;}
\PYG{n}{plt}\PYG{o}{.}\PYG{n}{legend}\PYG{p}{(}\PYG{p}{(}\PYG{l+s+s1}{\PYGZsq{}}\PYG{l+s+s1}{amostras}\PYG{l+s+s1}{\PYGZsq{}}\PYG{p}{,}\PYG{l+s+s1}{\PYGZsq{}}\PYG{l+s+s1}{ajuste}\PYG{l+s+s1}{\PYGZsq{}}\PYG{p}{)}\PYG{p}{)}\PYG{p}{;}
\end{sphinxVerbatim}

\noindent\sphinxincludegraphics{{aula-17-ajusteNaoLinear_14_0}.png}

\sphinxAtStartPar
\sphinxstylestrong{Estimando valores não tabelados:} visto que o modelo exponencial se acomoda bem aos dados experimentais, agora podemos estimar valores que são desconhecidos, tais como \(x = 3.2\) ou \(x = 7.5\).

\begin{sphinxVerbatim}[commandchars=\\\{\}]
\PYG{n}{xp} \PYG{o}{=} \PYG{n}{np}\PYG{o}{.}\PYG{n}{array}\PYG{p}{(}\PYG{p}{[}\PYG{l+m+mf}{3.2}\PYG{p}{,}\PYG{l+m+mf}{7.5}\PYG{p}{]}\PYG{p}{)} \PYG{c+c1}{\PYGZsh{} valores procurados}

\PYG{k}{for} \PYG{n}{p} \PYG{o+ow}{in} \PYG{n}{xp}\PYG{p}{:}
    \PYG{n+nb}{print}\PYG{p}{(}\PYG{l+s+sa}{f}\PYG{l+s+s1}{\PYGZsq{}}\PYG{l+s+s1}{Em x = }\PYG{l+s+si}{\PYGZob{}}\PYG{n}{p}\PYG{l+s+si}{:}\PYG{l+s+s1}{g}\PYG{l+s+si}{\PYGZcb{}}\PYG{l+s+s1}{, a estimativa é f(x) = }\PYG{l+s+si}{\PYGZob{}}\PYG{n}{modelo}\PYG{p}{(}\PYG{n}{p}\PYG{p}{)}\PYG{l+s+si}{:}\PYG{l+s+s1}{g}\PYG{l+s+si}{\PYGZcb{}}\PYG{l+s+s1}{\PYGZsq{}}\PYG{p}{)}
\end{sphinxVerbatim}

\begin{sphinxVerbatim}[commandchars=\\\{\}]
Em x = 3.2, a estimativa é f(x) = 21.097
Em x = 7.5, a estimativa é f(x) = 211.969
\end{sphinxVerbatim}


\subsection{Exercício}
\label{\detokenize{aula-17-ajusteNaoLinear:exercicio}}
\sphinxAtStartPar
A seguinte tabela mostra a variação de condutividade térmica relativa \(k\) de sódio com a temperatura \(T\) em graus Celsius. Busque um modelo não\sphinxhyphen{}linear que ajusta os dados no sentido de mínimos quadrados.


\begin{savenotes}\sphinxattablestart
\centering
\begin{tabulary}{\linewidth}[t]{|T|T|}
\hline
\sphinxstyletheadfamily 
\sphinxAtStartPar
k
&\sphinxstyletheadfamily 
\sphinxAtStartPar
T
\\
\hline
\sphinxAtStartPar
1.00
&
\sphinxAtStartPar
79
\\
\hline
\sphinxAtStartPar
0.932
&
\sphinxAtStartPar
190
\\
\hline
\sphinxAtStartPar
0.839
&
\sphinxAtStartPar
357
\\
\hline
\sphinxAtStartPar
0.759
&
\sphinxAtStartPar
524
\\
\hline
\sphinxAtStartPar
0.693
&
\sphinxAtStartPar
690
\\
\hline
\end{tabulary}
\par
\sphinxattableend\end{savenotes}


\chapter{Integração Numérica: Regras de Newton\sphinxhyphen{}Cotes}
\label{\detokenize{aula-18-integracao-newtonCotes:integracao-numerica-regras-de-newton-cotes}}\label{\detokenize{aula-18-integracao-newtonCotes::doc}}
\sphinxAtStartPar
As fórmulas de Newton\sphinxhyphen{}Cotes são os esquemas mais comuns de integração numérica. Elas são baseadas na estratégia de substituir uma função complicada ou dados tabulados por uma função aproximadora simples que seja fácil de integrar:
\begin{equation*}
\begin{split}
I = \int _a^b f(x)dx \cong \int _a^b f_n (x)dx \qquad (1)
\end{split}
\end{equation*}
\sphinxAtStartPar
em que \(f_n (x)\) é um polinômio da forma
\begin{equation*}
\begin{split}
f_n (x) = a_0 + a_1 x + \dots + a_{n-1} x^{n-1} + a_n x^n
\end{split}
\end{equation*}
\sphinxAtStartPar
em que \(n\) é o grau do polinômio.


\section{A Regra do Trapézio}
\label{\detokenize{aula-18-integracao-newtonCotes:a-regra-do-trapezio}}
\sphinxAtStartPar
A \sphinxstylestrong{regra do trapézio} corresponde ao caso no qual o polinômio na Equação (1) é de primeiro grau:
\begin{equation*}
\begin{split}
I = \int _a^b f(x)dx \cong \int _a^b f_1 (x)dx
\end{split}
\end{equation*}
\sphinxAtStartPar
onde
\begin{equation*}
\begin{split}
f_1 (x) = f(a) + \dfrac{f(b) - f(a)}{b - a} (x - a)
\end{split}
\end{equation*}
\sphinxAtStartPar
O resultado da integração é
\begin{equation*}
\begin{split}
I = (b - a) \dfrac{f(a) + f(b)}{2} \qquad (2)
\end{split}
\end{equation*}
\sphinxAtStartPar
Geometricamente, a regra dos trapézios é equivalente a aproximar a integral pela área do trapézio sob a reta ligando \(f(a)\) e \(f(b)\) (conforme figura abaixo). Portanto, a estimativa da integral pode ser representada por
\begin{equation*}
\begin{split}
I \cong \text{largura} \times \text{altura média} \\
I \cong (b - a) \times \text{altura média}
\end{split}
\end{equation*}
\sphinxAtStartPar
em que, para a regra dos trapézios, a altura média é a média dos valores da função nas extremidades, ou \([f (a) + f (b)]/2\), conforme Equação (2).

\begin{sphinxVerbatim}[commandchars=\\\{\}]
\PYG{k+kn}{import} \PYG{n+nn}{numpy} \PYG{k}{as} \PYG{n+nn}{np}
\PYG{k+kn}{import} \PYG{n+nn}{matplotlib}\PYG{n+nn}{.}\PYG{n+nn}{pyplot} \PYG{k}{as} \PYG{n+nn}{plt}
\PYG{k+kn}{import} \PYG{n+nn}{sympy} \PYG{k}{as} \PYG{n+nn}{sy}
\PYG{k+kn}{import} \PYG{n+nn}{scipy} \PYG{k}{as} \PYG{n+nn}{sp}
\PYG{k+kn}{from} \PYG{n+nn}{scipy} \PYG{k+kn}{import} \PYG{n}{interpolate}
\PYG{k+kn}{from} \PYG{n+nn}{scipy} \PYG{k+kn}{import} \PYG{n}{integrate}
\PYG{o}{\PYGZpc{}}\PYG{k}{matplotlib} inline
\end{sphinxVerbatim}

\begin{sphinxVerbatim}[commandchars=\\\{\}]
\PYG{c+c1}{\PYGZsh{} Representação geométrica da regra do trapézio}

\PYG{k}{def} \PYG{n+nf}{f}\PYG{p}{(}\PYG{n}{x}\PYG{p}{)}\PYG{p}{:}
    \PYG{k}{return} \PYG{o}{\PYGZhy{}}\PYG{n}{x}\PYG{o}{*}\PYG{o}{*}\PYG{l+m+mi}{2} \PYG{o}{+} \PYG{l+m+mi}{15}\PYG{o}{*}\PYG{n}{x} \PYG{o}{\PYGZhy{}} \PYG{l+m+mi}{10}

\PYG{n}{x} \PYG{o}{=} \PYG{n}{np}\PYG{o}{.}\PYG{n}{linspace}\PYG{p}{(}\PYG{l+m+mf}{0.5}\PYG{p}{,}\PYG{l+m+mi}{10}\PYG{p}{,}\PYG{l+m+mi}{100}\PYG{p}{)}
\PYG{n}{y} \PYG{o}{=} \PYG{n}{f}\PYG{p}{(}\PYG{n}{x}\PYG{p}{)}

\PYG{n}{plt}\PYG{o}{.}\PYG{n}{plot}\PYG{p}{(}\PYG{n}{x}\PYG{p}{,}\PYG{n}{y}\PYG{p}{,} \PYG{n}{label} \PYG{o}{=} \PYG{l+s+s1}{\PYGZsq{}}\PYG{l+s+s1}{f(x)}\PYG{l+s+s1}{\PYGZsq{}}\PYG{p}{)}
\PYG{n}{plt}\PYG{o}{.}\PYG{n}{plot}\PYG{p}{(}\PYG{p}{[}\PYG{l+m+mi}{2}\PYG{p}{,}\PYG{l+m+mi}{8}\PYG{p}{]}\PYG{p}{,} \PYG{p}{[}\PYG{n}{f}\PYG{p}{(}\PYG{l+m+mi}{2}\PYG{p}{)}\PYG{p}{,}\PYG{n}{f}\PYG{p}{(}\PYG{l+m+mi}{8}\PYG{p}{)}\PYG{p}{]}\PYG{p}{,} \PYG{l+s+s1}{\PYGZsq{}}\PYG{l+s+s1}{r}\PYG{l+s+s1}{\PYGZsq{}}\PYG{p}{,} \PYG{n}{label} \PYG{o}{=} \PYG{l+s+s1}{\PYGZsq{}}\PYG{l+s+s1}{f1(x)}\PYG{l+s+s1}{\PYGZsq{}}\PYG{p}{)}
\PYG{n}{plt}\PYG{o}{.}\PYG{n}{plot}\PYG{p}{(}\PYG{l+m+mi}{2}\PYG{p}{,}\PYG{n}{f}\PYG{p}{(}\PYG{l+m+mi}{2}\PYG{p}{)}\PYG{p}{,} \PYG{l+s+s1}{\PYGZsq{}}\PYG{l+s+s1}{o}\PYG{l+s+s1}{\PYGZsq{}}\PYG{p}{,} \PYG{n}{label} \PYG{o}{=} \PYG{l+s+s1}{\PYGZsq{}}\PYG{l+s+s1}{f(a)}\PYG{l+s+s1}{\PYGZsq{}}\PYG{p}{)}
\PYG{n}{plt}\PYG{o}{.}\PYG{n}{plot}\PYG{p}{(}\PYG{l+m+mi}{8}\PYG{p}{,}\PYG{n}{f}\PYG{p}{(}\PYG{l+m+mi}{8}\PYG{p}{)}\PYG{p}{,} \PYG{l+s+s1}{\PYGZsq{}}\PYG{l+s+s1}{o}\PYG{l+s+s1}{\PYGZsq{}}\PYG{p}{,} \PYG{n}{label} \PYG{o}{=} \PYG{l+s+s1}{\PYGZsq{}}\PYG{l+s+s1}{f(b)}\PYG{l+s+s1}{\PYGZsq{}}\PYG{p}{)}
\PYG{n}{plt}\PYG{o}{.}\PYG{n}{plot}\PYG{p}{(}\PYG{n}{np}\PYG{o}{.}\PYG{n}{mean}\PYG{p}{(}\PYG{p}{[}\PYG{l+m+mi}{2}\PYG{p}{,}\PYG{l+m+mi}{8}\PYG{p}{]}\PYG{p}{)}\PYG{p}{,} \PYG{n}{np}\PYG{o}{.}\PYG{n}{mean}\PYG{p}{(}\PYG{p}{[}\PYG{n}{f}\PYG{p}{(}\PYG{l+m+mi}{2}\PYG{p}{)}\PYG{p}{,}\PYG{n}{f}\PYG{p}{(}\PYG{l+m+mi}{8}\PYG{p}{)}\PYG{p}{]}\PYG{p}{)}\PYG{p}{,} \PYG{l+s+s1}{\PYGZsq{}}\PYG{l+s+s1}{o}\PYG{l+s+s1}{\PYGZsq{}}\PYG{p}{,} \PYG{n}{label} \PYG{o}{=} \PYG{l+s+s1}{\PYGZsq{}}\PYG{l+s+s1}{Altura Média}\PYG{l+s+s1}{\PYGZsq{}}\PYG{p}{)}
\PYG{n}{plt}\PYG{o}{.}\PYG{n}{fill}\PYG{p}{(}\PYG{p}{[}\PYG{l+m+mi}{2}\PYG{p}{,} \PYG{l+m+mi}{2}\PYG{p}{,} \PYG{l+m+mi}{8}\PYG{p}{,} \PYG{l+m+mi}{8}\PYG{p}{]}\PYG{p}{,}\PYG{p}{[}\PYG{l+m+mi}{0}\PYG{p}{,} \PYG{n}{f}\PYG{p}{(}\PYG{l+m+mi}{2}\PYG{p}{)}\PYG{p}{,} \PYG{n}{f}\PYG{p}{(}\PYG{l+m+mi}{8}\PYG{p}{)}\PYG{p}{,} \PYG{l+m+mi}{0}\PYG{p}{]}\PYG{p}{,}\PYG{n}{color}\PYG{o}{=}\PYG{l+s+s1}{\PYGZsq{}}\PYG{l+s+s1}{r}\PYG{l+s+s1}{\PYGZsq{}}\PYG{p}{,}\PYG{n}{alpha}\PYG{o}{=}\PYG{l+m+mf}{0.4}\PYG{p}{,} \PYG{n}{label} \PYG{o}{=} \PYG{l+s+s1}{\PYGZsq{}}\PYG{l+s+s1}{Área do Trapézio}\PYG{l+s+s1}{\PYGZsq{}}\PYG{p}{)}
\PYG{n}{plt}\PYG{o}{.}\PYG{n}{legend}\PYG{p}{(}\PYG{p}{)}
\PYG{n}{plt}\PYG{o}{.}\PYG{n}{axvline}\PYG{p}{(}\PYG{n}{x}\PYG{o}{=}\PYG{l+m+mi}{0}\PYG{p}{,}\PYG{n}{color}\PYG{o}{=}\PYG{l+s+s1}{\PYGZsq{}}\PYG{l+s+s1}{k}\PYG{l+s+s1}{\PYGZsq{}}\PYG{p}{,}\PYG{n}{linewidth}\PYG{o}{=}\PYG{l+m+mf}{0.6}\PYG{p}{,}\PYG{n}{linestyle}\PYG{o}{=}\PYG{l+s+s1}{\PYGZsq{}}\PYG{l+s+s1}{\PYGZhy{}\PYGZhy{}}\PYG{l+s+s1}{\PYGZsq{}}\PYG{p}{)}
\PYG{n}{plt}\PYG{o}{.}\PYG{n}{axhline}\PYG{p}{(}\PYG{n}{y}\PYG{o}{=}\PYG{l+m+mi}{0}\PYG{p}{,}\PYG{n}{color}\PYG{o}{=}\PYG{l+s+s1}{\PYGZsq{}}\PYG{l+s+s1}{k}\PYG{l+s+s1}{\PYGZsq{}}\PYG{p}{,}\PYG{n}{linewidth}\PYG{o}{=}\PYG{l+m+mf}{0.6}\PYG{p}{,}\PYG{n}{linestyle}\PYG{o}{=}\PYG{l+s+s1}{\PYGZsq{}}\PYG{l+s+s1}{\PYGZhy{}\PYGZhy{}}\PYG{l+s+s1}{\PYGZsq{}}\PYG{p}{)}\PYG{p}{;}
\end{sphinxVerbatim}

\noindent\sphinxincludegraphics{{aula-18-integracao-newtonCotes_2_0}.png}


\section{Aplicação Múltipla da Regra do Trapézio}
\label{\detokenize{aula-18-integracao-newtonCotes:aplicacao-multipla-da-regra-do-trapezio}}
\sphinxAtStartPar
Uma maneira de melhorar a acurácia da regra do trapézio é dividir o intervalo de integração de \(a\) a \(b\) em diversos segmentos e aplicar o método a cada segmento (conforme figura abaixo). As áreas correspondentes aos segmentos individuais podem então ser somadas parafornecer a integral para o intervalo inteiro. As equações resultantes são chamadas fórmulas de integração por aplicações múltiplas ou compostas.

\sphinxAtStartPar
Existem \(n + 1\) pontos base igualmente espaçados \((x_0, x_1, x_2, \dots , x_n)\). Consequentemente, existem \(n\) segmentos de largura igual:
\begin{equation*}
\begin{split}
h = \dfrac{b − a}{n} \qquad (3)
\end{split}
\end{equation*}
\sphinxAtStartPar
Se \(a\) e \(b\) forem designados por \(x_0\) e \(x_n\), respectivamente, a integral total pode ser representada como
\begin{equation*}
\begin{split}
I = \int _{x_0}^{x_1} f(x)dx + \int _{x_1}^{x_2} f(x)dx + \dots + \int _{x_{n - 1}}^{x_n} f(x)dx 
\end{split}
\end{equation*}
\sphinxAtStartPar
Substituindo cada integral pela regra do trapézio, obtém\sphinxhyphen{}se
\begin{equation*}
\begin{split}
I = h \dfrac{f(x_0) + f(x_1)}{2} + h \dfrac{f(x_1) + f(x_2)}{2} + \dots + h \dfrac{f(x_{n - 1}) + f(x_n)}{2}
\end{split}
\end{equation*}
\sphinxAtStartPar
ou, agrupando termos,
\begin{equation*}
\begin{split}
I = \dfrac{h}{2} \left[ f(x_0) + 2 \sum _{i = 1} ^{n - 1} f(x_i) + f(x_n) \right] \qquad (4)
\end{split}
\end{equation*}
\begin{sphinxVerbatim}[commandchars=\\\{\}]
\PYG{c+c1}{\PYGZsh{} Representação geométrica da aplicação múltipla da regra do trapézio}
\PYG{c+c1}{\PYGZsh{} É possível observar, visualmente, que houve uma redução do erro do resultado em comparação com o caso anterior}

\PYG{n}{xd} \PYG{o}{=} \PYG{n}{np}\PYG{o}{.}\PYG{n}{array}\PYG{p}{(}\PYG{p}{[}\PYG{l+m+mi}{2}\PYG{p}{,}\PYG{l+m+mi}{5}\PYG{p}{,}\PYG{l+m+mi}{8}\PYG{p}{]}\PYG{p}{)}
\PYG{n}{yd} \PYG{o}{=} \PYG{n}{f}\PYG{p}{(}\PYG{n}{xd}\PYG{p}{)}

\PYG{n}{plt}\PYG{o}{.}\PYG{n}{plot}\PYG{p}{(}\PYG{n}{x}\PYG{p}{,}\PYG{n}{y}\PYG{p}{,} \PYG{n}{label} \PYG{o}{=} \PYG{l+s+s1}{\PYGZsq{}}\PYG{l+s+s1}{f(x)}\PYG{l+s+s1}{\PYGZsq{}}\PYG{p}{)}
\PYG{n}{plt}\PYG{o}{.}\PYG{n}{plot}\PYG{p}{(}\PYG{n}{xd}\PYG{p}{,}\PYG{n}{yd}\PYG{p}{,} \PYG{l+s+s1}{\PYGZsq{}}\PYG{l+s+s1}{r}\PYG{l+s+s1}{\PYGZsq{}}\PYG{p}{)}
\PYG{n}{plt}\PYG{o}{.}\PYG{n}{plot}\PYG{p}{(}\PYG{l+m+mi}{2}\PYG{p}{,}\PYG{n}{f}\PYG{p}{(}\PYG{l+m+mi}{2}\PYG{p}{)}\PYG{p}{,} \PYG{l+s+s1}{\PYGZsq{}}\PYG{l+s+s1}{o}\PYG{l+s+s1}{\PYGZsq{}}\PYG{p}{,} \PYG{n}{label} \PYG{o}{=} \PYG{l+s+s1}{\PYGZsq{}}\PYG{l+s+s1}{f(x0)}\PYG{l+s+s1}{\PYGZsq{}}\PYG{p}{)}
\PYG{n}{plt}\PYG{o}{.}\PYG{n}{plot}\PYG{p}{(}\PYG{l+m+mi}{5}\PYG{p}{,}\PYG{n}{f}\PYG{p}{(}\PYG{l+m+mi}{5}\PYG{p}{)}\PYG{p}{,} \PYG{l+s+s1}{\PYGZsq{}}\PYG{l+s+s1}{o}\PYG{l+s+s1}{\PYGZsq{}}\PYG{p}{,} \PYG{n}{label} \PYG{o}{=} \PYG{l+s+s1}{\PYGZsq{}}\PYG{l+s+s1}{f(x1)}\PYG{l+s+s1}{\PYGZsq{}}\PYG{p}{)}
\PYG{n}{plt}\PYG{o}{.}\PYG{n}{plot}\PYG{p}{(}\PYG{l+m+mi}{8}\PYG{p}{,}\PYG{n}{f}\PYG{p}{(}\PYG{l+m+mi}{8}\PYG{p}{)}\PYG{p}{,} \PYG{l+s+s1}{\PYGZsq{}}\PYG{l+s+s1}{o}\PYG{l+s+s1}{\PYGZsq{}}\PYG{p}{,} \PYG{n}{label} \PYG{o}{=} \PYG{l+s+s1}{\PYGZsq{}}\PYG{l+s+s1}{f(x2)}\PYG{l+s+s1}{\PYGZsq{}}\PYG{p}{)}
\PYG{n}{plt}\PYG{o}{.}\PYG{n}{fill}\PYG{p}{(}\PYG{p}{[}\PYG{l+m+mi}{2}\PYG{p}{,} \PYG{l+m+mi}{2}\PYG{p}{,} \PYG{l+m+mi}{5}\PYG{p}{,} \PYG{l+m+mi}{5}\PYG{p}{]}\PYG{p}{,}\PYG{p}{[}\PYG{l+m+mi}{0}\PYG{p}{,} \PYG{n}{f}\PYG{p}{(}\PYG{l+m+mi}{2}\PYG{p}{)}\PYG{p}{,} \PYG{n}{f}\PYG{p}{(}\PYG{l+m+mi}{5}\PYG{p}{)}\PYG{p}{,} \PYG{l+m+mi}{0}\PYG{p}{]}\PYG{p}{,} \PYG{n}{color}\PYG{o}{=}\PYG{l+s+s1}{\PYGZsq{}}\PYG{l+s+s1}{r}\PYG{l+s+s1}{\PYGZsq{}}\PYG{p}{,} \PYG{n}{alpha}\PYG{o}{=}\PYG{l+m+mf}{0.6}\PYG{p}{,} \PYG{n}{label} \PYG{o}{=} \PYG{l+s+s1}{\PYGZsq{}}\PYG{l+s+s1}{Área do Trapézio 1}\PYG{l+s+s1}{\PYGZsq{}}\PYG{p}{)}
\PYG{n}{plt}\PYG{o}{.}\PYG{n}{fill}\PYG{p}{(}\PYG{p}{[}\PYG{l+m+mi}{5}\PYG{p}{,} \PYG{l+m+mi}{5}\PYG{p}{,} \PYG{l+m+mi}{8}\PYG{p}{,} \PYG{l+m+mi}{8}\PYG{p}{]}\PYG{p}{,}\PYG{p}{[}\PYG{l+m+mi}{0}\PYG{p}{,} \PYG{n}{f}\PYG{p}{(}\PYG{l+m+mi}{5}\PYG{p}{)}\PYG{p}{,} \PYG{n}{f}\PYG{p}{(}\PYG{l+m+mi}{8}\PYG{p}{)}\PYG{p}{,} \PYG{l+m+mi}{0}\PYG{p}{]}\PYG{p}{,} \PYG{n}{color}\PYG{o}{=}\PYG{l+s+s1}{\PYGZsq{}}\PYG{l+s+s1}{r}\PYG{l+s+s1}{\PYGZsq{}}\PYG{p}{,} \PYG{n}{alpha}\PYG{o}{=}\PYG{l+m+mf}{0.3}\PYG{p}{,} \PYG{n}{label} \PYG{o}{=} \PYG{l+s+s1}{\PYGZsq{}}\PYG{l+s+s1}{Área do Trapézio 2}\PYG{l+s+s1}{\PYGZsq{}}\PYG{p}{)}
\PYG{n}{plt}\PYG{o}{.}\PYG{n}{legend}\PYG{p}{(}\PYG{p}{)}
\PYG{n}{plt}\PYG{o}{.}\PYG{n}{axvline}\PYG{p}{(}\PYG{n}{x}\PYG{o}{=}\PYG{l+m+mi}{0}\PYG{p}{,}\PYG{n}{color}\PYG{o}{=}\PYG{l+s+s1}{\PYGZsq{}}\PYG{l+s+s1}{k}\PYG{l+s+s1}{\PYGZsq{}}\PYG{p}{,}\PYG{n}{linewidth}\PYG{o}{=}\PYG{l+m+mf}{0.6}\PYG{p}{,}\PYG{n}{linestyle}\PYG{o}{=}\PYG{l+s+s1}{\PYGZsq{}}\PYG{l+s+s1}{\PYGZhy{}\PYGZhy{}}\PYG{l+s+s1}{\PYGZsq{}}\PYG{p}{)}
\PYG{n}{plt}\PYG{o}{.}\PYG{n}{axhline}\PYG{p}{(}\PYG{n}{y}\PYG{o}{=}\PYG{l+m+mi}{0}\PYG{p}{,}\PYG{n}{color}\PYG{o}{=}\PYG{l+s+s1}{\PYGZsq{}}\PYG{l+s+s1}{k}\PYG{l+s+s1}{\PYGZsq{}}\PYG{p}{,}\PYG{n}{linewidth}\PYG{o}{=}\PYG{l+m+mf}{0.6}\PYG{p}{,}\PYG{n}{linestyle}\PYG{o}{=}\PYG{l+s+s1}{\PYGZsq{}}\PYG{l+s+s1}{\PYGZhy{}\PYGZhy{}}\PYG{l+s+s1}{\PYGZsq{}}\PYG{p}{)}\PYG{p}{;} 
\end{sphinxVerbatim}

\noindent\sphinxincludegraphics{{aula-18-integracao-newtonCotes_4_0}.png}


\section{A Regra 1/3 de Simpson}
\label{\detokenize{aula-18-integracao-newtonCotes:a-regra-1-3-de-simpson}}
\sphinxAtStartPar
A \sphinxstylestrong{regra 1/3 de Simpson} é obtida quando um polinômio interpolador de segundo grau é substituído na Equação (1):
\begin{equation*}
\begin{split}
I = \int _a^b f(x)dx \cong \int _a^b f_2 (x)dx
\end{split}
\end{equation*}
\sphinxAtStartPar
Se \(a\) e \(b\) forem designados por \(x_0\) e \(x_2\) e se \(f_2(x)\) for representado por um polinômio de Lagrange de segundo grau, a integral se torna
\begin{equation*}
\begin{split}
I = \int _{x_0}^{x_2} \left[ \dfrac{(x - x_1)(x - x_2)}{(x_0 - x_1)(x_0 - x_2)} f(x_0) + \dfrac{(x - x_1)(x - x_2)}{(x_0 - x_1)(x_0 - x_2)} f(x_1) + \dfrac{(x - x_1)(x - x_2)}{(x_0 - x_1)(x_0 - x_2)} f(x_2) \right] dx
\end{split}
\end{equation*}
\sphinxAtStartPar
Depois da integração e de manipulações algébricas, obtém\sphinxhyphen{}se a seguinte fórmula:
\begin{equation*}
\begin{split}
I \cong \dfrac{h}{3} [f(x_0) + 4 f(x_1) + f(x_2)] \qquad (5)
\end{split}
\end{equation*}
\sphinxAtStartPar
em que, para esse caso
\begin{equation*}
\begin{split}
h = \dfrac{b − a}{2} \qquad (6)
\end{split}
\end{equation*}
\begin{sphinxVerbatim}[commandchars=\\\{\}]
\PYG{c+c1}{\PYGZsh{} Descrição gráfica da regra 1/3 de Simpson: ela consiste em tomar a área sob uma parábola ligando três pontos}

\PYG{k}{def} \PYG{n+nf}{f}\PYG{p}{(}\PYG{n}{x}\PYG{p}{)}\PYG{p}{:}
    \PYG{k}{return} \PYG{l+m+mf}{0.6}\PYG{o}{*}\PYG{n}{np}\PYG{o}{.}\PYG{n}{sin}\PYG{p}{(}\PYG{n}{x}\PYG{o}{+}\PYG{n}{np}\PYG{o}{.}\PYG{n}{pi}\PYG{o}{/}\PYG{l+m+mi}{2}\PYG{p}{)} \PYG{o}{+} \PYG{l+m+mf}{0.2}\PYG{o}{*}\PYG{n}{x} \PYG{o}{+} \PYG{l+m+mi}{2}

\PYG{k}{def} \PYG{n+nf}{f2}\PYG{p}{(}\PYG{n}{x}\PYG{p}{)}\PYG{p}{:}
    \PYG{k}{return} \PYG{o}{\PYGZhy{}}\PYG{l+m+mf}{0.127382111059529}\PYG{o}{*}\PYG{n}{x}\PYG{o}{*}\PYG{o}{*}\PYG{l+m+mi}{2} \PYG{o}{+} \PYG{l+m+mf}{1.73647155618593}\PYG{o}{*}\PYG{n}{x} \PYG{o}{\PYGZhy{}} \PYG{l+m+mf}{2.0569711669823} \PYG{c+c1}{\PYGZsh{} Obtido pelo MMQ}

\PYG{n}{x} \PYG{o}{=} \PYG{n}{np}\PYG{o}{.}\PYG{n}{linspace}\PYG{p}{(}\PYG{l+m+mi}{2}\PYG{p}{,}\PYG{l+m+mi}{10}\PYG{p}{,}\PYG{l+m+mi}{100}\PYG{p}{)}
\PYG{n}{y} \PYG{o}{=} \PYG{n}{f}\PYG{p}{(}\PYG{n}{x}\PYG{p}{)}

\PYG{n}{xc} \PYG{o}{=} \PYG{n}{np}\PYG{o}{.}\PYG{n}{linspace}\PYG{p}{(}\PYG{l+m+mi}{2}\PYG{p}{,}\PYG{l+m+mi}{10}\PYG{p}{,}\PYG{l+m+mi}{100}\PYG{p}{)}
\PYG{n}{yc} \PYG{o}{=} \PYG{n}{f2}\PYG{p}{(}\PYG{n}{xc}\PYG{p}{)}

\PYG{n}{xf} \PYG{o}{=} \PYG{n}{np}\PYG{o}{.}\PYG{n}{linspace}\PYG{p}{(}\PYG{l+m+mi}{3}\PYG{p}{,}\PYG{l+m+mi}{9}\PYG{p}{,}\PYG{l+m+mi}{100}\PYG{p}{)}
\PYG{n}{yf} \PYG{o}{=} \PYG{n}{f2}\PYG{p}{(}\PYG{n}{xf}\PYG{p}{)}

\PYG{n}{xf} \PYG{o}{=} \PYG{n}{np}\PYG{o}{.}\PYG{n}{concatenate}\PYG{p}{(}\PYG{p}{(}\PYG{n}{np}\PYG{o}{.}\PYG{n}{array}\PYG{p}{(}\PYG{p}{[}\PYG{l+m+mi}{3}\PYG{p}{]}\PYG{p}{)}\PYG{p}{,} \PYG{n}{xf}\PYG{p}{,} \PYG{n}{np}\PYG{o}{.}\PYG{n}{array}\PYG{p}{(}\PYG{p}{[}\PYG{l+m+mi}{9}\PYG{p}{]}\PYG{p}{)}\PYG{p}{)}\PYG{p}{)}
\PYG{n}{yf} \PYG{o}{=} \PYG{n}{np}\PYG{o}{.}\PYG{n}{concatenate}\PYG{p}{(}\PYG{p}{(}\PYG{n}{np}\PYG{o}{.}\PYG{n}{array}\PYG{p}{(}\PYG{p}{[}\PYG{l+m+mi}{0}\PYG{p}{]}\PYG{p}{)}\PYG{p}{,} \PYG{n}{yf}\PYG{p}{,} \PYG{n}{np}\PYG{o}{.}\PYG{n}{array}\PYG{p}{(}\PYG{p}{[}\PYG{l+m+mi}{0}\PYG{p}{]}\PYG{p}{)}\PYG{p}{)}\PYG{p}{)}

\PYG{n}{plt}\PYG{o}{.}\PYG{n}{plot}\PYG{p}{(}\PYG{n}{x}\PYG{p}{,}\PYG{n}{y}\PYG{p}{,} \PYG{n}{label} \PYG{o}{=} \PYG{l+s+s1}{\PYGZsq{}}\PYG{l+s+s1}{f(x)}\PYG{l+s+s1}{\PYGZsq{}}\PYG{p}{)}
\PYG{n}{plt}\PYG{o}{.}\PYG{n}{plot}\PYG{p}{(}\PYG{n}{xc}\PYG{p}{,}\PYG{n}{yc}\PYG{p}{,} \PYG{l+s+s1}{\PYGZsq{}}\PYG{l+s+s1}{:r}\PYG{l+s+s1}{\PYGZsq{}}\PYG{p}{,} \PYG{n}{label} \PYG{o}{=} \PYG{l+s+s1}{\PYGZsq{}}\PYG{l+s+s1}{f2(x)}\PYG{l+s+s1}{\PYGZsq{}}\PYG{p}{)}
\PYG{n}{plt}\PYG{o}{.}\PYG{n}{plot}\PYG{p}{(}\PYG{l+m+mi}{3}\PYG{p}{,}\PYG{n}{f}\PYG{p}{(}\PYG{l+m+mi}{3}\PYG{p}{)}\PYG{p}{,} \PYG{l+s+s1}{\PYGZsq{}}\PYG{l+s+s1}{o}\PYG{l+s+s1}{\PYGZsq{}}\PYG{p}{,} \PYG{n}{label} \PYG{o}{=} \PYG{l+s+s1}{\PYGZsq{}}\PYG{l+s+s1}{f(x0)}\PYG{l+s+s1}{\PYGZsq{}}\PYG{p}{)}
\PYG{n}{plt}\PYG{o}{.}\PYG{n}{plot}\PYG{p}{(}\PYG{l+m+mi}{6}\PYG{p}{,}\PYG{n}{f}\PYG{p}{(}\PYG{l+m+mi}{6}\PYG{p}{)}\PYG{p}{,} \PYG{l+s+s1}{\PYGZsq{}}\PYG{l+s+s1}{o}\PYG{l+s+s1}{\PYGZsq{}}\PYG{p}{,} \PYG{n}{label} \PYG{o}{=} \PYG{l+s+s1}{\PYGZsq{}}\PYG{l+s+s1}{f(x1)}\PYG{l+s+s1}{\PYGZsq{}}\PYG{p}{)}
\PYG{n}{plt}\PYG{o}{.}\PYG{n}{plot}\PYG{p}{(}\PYG{l+m+mi}{9}\PYG{p}{,}\PYG{n}{f}\PYG{p}{(}\PYG{l+m+mi}{9}\PYG{p}{)}\PYG{p}{,} \PYG{l+s+s1}{\PYGZsq{}}\PYG{l+s+s1}{o}\PYG{l+s+s1}{\PYGZsq{}}\PYG{p}{,} \PYG{n}{label} \PYG{o}{=} \PYG{l+s+s1}{\PYGZsq{}}\PYG{l+s+s1}{f(x2)}\PYG{l+s+s1}{\PYGZsq{}}\PYG{p}{)}
\PYG{n}{plt}\PYG{o}{.}\PYG{n}{fill}\PYG{p}{(}\PYG{n}{xf}\PYG{p}{,}\PYG{n}{yf}\PYG{p}{,} \PYG{n}{color}\PYG{o}{=}\PYG{l+s+s1}{\PYGZsq{}}\PYG{l+s+s1}{r}\PYG{l+s+s1}{\PYGZsq{}}\PYG{p}{,} \PYG{n}{alpha}\PYG{o}{=}\PYG{l+m+mf}{0.4}\PYG{p}{)}
\PYG{n}{plt}\PYG{o}{.}\PYG{n}{legend}\PYG{p}{(}\PYG{p}{)}
\PYG{n}{plt}\PYG{o}{.}\PYG{n}{axvline}\PYG{p}{(}\PYG{n}{x}\PYG{o}{=}\PYG{l+m+mi}{0}\PYG{p}{,}\PYG{n}{color}\PYG{o}{=}\PYG{l+s+s1}{\PYGZsq{}}\PYG{l+s+s1}{k}\PYG{l+s+s1}{\PYGZsq{}}\PYG{p}{,}\PYG{n}{linewidth}\PYG{o}{=}\PYG{l+m+mf}{0.6}\PYG{p}{,}\PYG{n}{linestyle}\PYG{o}{=}\PYG{l+s+s1}{\PYGZsq{}}\PYG{l+s+s1}{\PYGZhy{}\PYGZhy{}}\PYG{l+s+s1}{\PYGZsq{}}\PYG{p}{)}
\PYG{n}{plt}\PYG{o}{.}\PYG{n}{axhline}\PYG{p}{(}\PYG{n}{y}\PYG{o}{=}\PYG{l+m+mi}{0}\PYG{p}{,}\PYG{n}{color}\PYG{o}{=}\PYG{l+s+s1}{\PYGZsq{}}\PYG{l+s+s1}{k}\PYG{l+s+s1}{\PYGZsq{}}\PYG{p}{,}\PYG{n}{linewidth}\PYG{o}{=}\PYG{l+m+mf}{0.6}\PYG{p}{,}\PYG{n}{linestyle}\PYG{o}{=}\PYG{l+s+s1}{\PYGZsq{}}\PYG{l+s+s1}{\PYGZhy{}\PYGZhy{}}\PYG{l+s+s1}{\PYGZsq{}}\PYG{p}{)}\PYG{p}{;}
\end{sphinxVerbatim}

\noindent\sphinxincludegraphics{{aula-18-integracao-newtonCotes_6_0}.png}


\section{Aplicações Múltiplas da Regra 1/3 de Simpson}
\label{\detokenize{aula-18-integracao-newtonCotes:aplicacoes-multiplas-da-regra-1-3-de-simpson}}
\sphinxAtStartPar
Do mesmo modo como no caso da regra do trapézio, a regra de Simpson pode ser melhorada dividindo\sphinxhyphen{}se o intervalo de integração em diversos segmentos de mesmo comprimento
\begin{equation*}
\begin{split}
h = \dfrac{b − a}{2}
\end{split}
\end{equation*}
\sphinxAtStartPar
A integral total pode ser representada como
\begin{equation*}
\begin{split}
I = \int _{x_0}^{x_2} f(x)dx + \int _{x_2}^{x_4} f(x)dx + \dots + \int _{x_{n - 2}}^{x_n} f(x)dx 
\end{split}
\end{equation*}
\sphinxAtStartPar
Substituindo cada integral individual pela regra 1/3 de Simpson, obtemos
\begin{equation*}
\begin{split}
I \cong 2 h \dfrac{f(x_0) + 4 f(x_1) + f(x_2)}{6} + 2 h \dfrac{f(x_2) + 4 f(x_3) + f(x_4)}{6} + \dots + 2 h \dfrac{f(x_{n - 2}) + 4 f(x_{n - 1}) + f(x_n)}{6}
\end{split}
\end{equation*}
\sphinxAtStartPar
ou
\begin{equation*}
\begin{split}
I \cong (b - a) \dfrac{ f(x_0) + 4 \sum _{i = 1, 3, 5}^{n-1} f(x_i) + 2 \sum _{i = 2, 4, 6}^{n-2} f(x_j) + f(x_n) }{3 n} \qquad (7)
\end{split}
\end{equation*}

\section{Regra 3/8 de Simpson}
\label{\detokenize{aula-18-integracao-newtonCotes:regra-3-8-de-simpson}}
\sphinxAtStartPar
De uma maneira parecida com a dedução da regra do trapézio e da regra 1/3 de Simpson, um polinômio de Lagrange de ordem três pode ser ajustado a quatro pontos e integrado
\begin{equation*}
\begin{split}
I = \int _a^b f(x)dx \cong \int _a^b f_3 (x)dx
\end{split}
\end{equation*}
\sphinxAtStartPar
para fornecer
\begin{equation*}
\begin{split}
I \cong \dfrac{3 h}{8} [f(x_0) + 3 f(x_1) + 3 f(x_2) + f(x_3)] \qquad (8)
\end{split}
\end{equation*}
\sphinxAtStartPar
em que
\begin{equation*}
\begin{split}
h = \dfrac{b − a}{3} \qquad (9)
\end{split}
\end{equation*}
\sphinxAtStartPar
Essa equação é chamada regra 3/8 de Simpson porque \(h\) é multiplicada por 3/8.

\begin{sphinxVerbatim}[commandchars=\\\{\}]
\PYG{c+c1}{\PYGZsh{} Descrição gráfica da regra 3/8 de Simpson: ela consiste em tomar a área sob uma equação cúbica ligando quatro pontos}

\PYG{k}{def} \PYG{n+nf}{f3}\PYG{p}{(}\PYG{n}{x}\PYG{p}{)}\PYG{p}{:}
    \PYG{k}{return} \PYG{o}{\PYGZhy{}}\PYG{l+m+mf}{0.0166482246490701}\PYG{o}{*}\PYG{n}{x}\PYG{o}{*}\PYG{o}{*}\PYG{l+m+mi}{3} \PYG{o}{+} \PYG{l+m+mf}{0.189467273747325}\PYG{o}{*}\PYG{n}{x}\PYG{o}{*}\PYG{o}{*}\PYG{l+m+mi}{2} \PYG{o}{\PYGZhy{}} \PYG{l+m+mf}{0.117878777555138}\PYG{o}{*}\PYG{n}{x} \PYG{o}{+} \PYG{l+m+mf}{1.10393743650415} \PYG{c+c1}{\PYGZsh{} Obtido pelo MMQ}

\PYG{n}{xc} \PYG{o}{=} \PYG{n}{np}\PYG{o}{.}\PYG{n}{linspace}\PYG{p}{(}\PYG{l+m+mi}{2}\PYG{p}{,}\PYG{l+m+mi}{10}\PYG{p}{,}\PYG{l+m+mi}{100}\PYG{p}{)}
\PYG{n}{yc} \PYG{o}{=} \PYG{n}{f3}\PYG{p}{(}\PYG{n}{xc}\PYG{p}{)}

\PYG{n}{xf} \PYG{o}{=} \PYG{n}{np}\PYG{o}{.}\PYG{n}{linspace}\PYG{p}{(}\PYG{l+m+mi}{3}\PYG{p}{,}\PYG{l+m+mi}{9}\PYG{p}{,}\PYG{l+m+mi}{100}\PYG{p}{)}
\PYG{n}{yf} \PYG{o}{=} \PYG{n}{f3}\PYG{p}{(}\PYG{n}{xf}\PYG{p}{)}

\PYG{n}{xf} \PYG{o}{=} \PYG{n}{np}\PYG{o}{.}\PYG{n}{concatenate}\PYG{p}{(}\PYG{p}{(}\PYG{n}{np}\PYG{o}{.}\PYG{n}{array}\PYG{p}{(}\PYG{p}{[}\PYG{l+m+mi}{3}\PYG{p}{]}\PYG{p}{)}\PYG{p}{,} \PYG{n}{xf}\PYG{p}{,} \PYG{n}{np}\PYG{o}{.}\PYG{n}{array}\PYG{p}{(}\PYG{p}{[}\PYG{l+m+mi}{9}\PYG{p}{]}\PYG{p}{)}\PYG{p}{)}\PYG{p}{)}
\PYG{n}{yf} \PYG{o}{=} \PYG{n}{np}\PYG{o}{.}\PYG{n}{concatenate}\PYG{p}{(}\PYG{p}{(}\PYG{n}{np}\PYG{o}{.}\PYG{n}{array}\PYG{p}{(}\PYG{p}{[}\PYG{l+m+mi}{0}\PYG{p}{]}\PYG{p}{)}\PYG{p}{,} \PYG{n}{yf}\PYG{p}{,} \PYG{n}{np}\PYG{o}{.}\PYG{n}{array}\PYG{p}{(}\PYG{p}{[}\PYG{l+m+mi}{0}\PYG{p}{]}\PYG{p}{)}\PYG{p}{)}\PYG{p}{)}

\PYG{n}{plt}\PYG{o}{.}\PYG{n}{plot}\PYG{p}{(}\PYG{n}{x}\PYG{p}{,}\PYG{n}{y}\PYG{p}{,} \PYG{n}{label} \PYG{o}{=} \PYG{l+s+s1}{\PYGZsq{}}\PYG{l+s+s1}{f(x)}\PYG{l+s+s1}{\PYGZsq{}}\PYG{p}{)}
\PYG{n}{plt}\PYG{o}{.}\PYG{n}{plot}\PYG{p}{(}\PYG{n}{xc}\PYG{p}{,}\PYG{n}{yc}\PYG{p}{,} \PYG{l+s+s1}{\PYGZsq{}}\PYG{l+s+s1}{:r}\PYG{l+s+s1}{\PYGZsq{}}\PYG{p}{,} \PYG{n}{label} \PYG{o}{=} \PYG{l+s+s1}{\PYGZsq{}}\PYG{l+s+s1}{f3(x)}\PYG{l+s+s1}{\PYGZsq{}}\PYG{p}{)}
\PYG{n}{plt}\PYG{o}{.}\PYG{n}{plot}\PYG{p}{(}\PYG{l+m+mi}{3}\PYG{p}{,} \PYG{n}{f}\PYG{p}{(}\PYG{l+m+mi}{3}\PYG{p}{)}\PYG{p}{,} \PYG{l+s+s1}{\PYGZsq{}}\PYG{l+s+s1}{o}\PYG{l+s+s1}{\PYGZsq{}}\PYG{p}{,} \PYG{n}{label} \PYG{o}{=} \PYG{l+s+s1}{\PYGZsq{}}\PYG{l+s+s1}{f(x0)}\PYG{l+s+s1}{\PYGZsq{}}\PYG{p}{)}
\PYG{n}{plt}\PYG{o}{.}\PYG{n}{plot}\PYG{p}{(}\PYG{l+m+mi}{5}\PYG{p}{,} \PYG{n}{f}\PYG{p}{(}\PYG{l+m+mi}{5}\PYG{p}{)}\PYG{p}{,} \PYG{l+s+s1}{\PYGZsq{}}\PYG{l+s+s1}{o}\PYG{l+s+s1}{\PYGZsq{}}\PYG{p}{,} \PYG{n}{label} \PYG{o}{=} \PYG{l+s+s1}{\PYGZsq{}}\PYG{l+s+s1}{f(x1)}\PYG{l+s+s1}{\PYGZsq{}}\PYG{p}{)}
\PYG{n}{plt}\PYG{o}{.}\PYG{n}{plot}\PYG{p}{(}\PYG{l+m+mi}{7}\PYG{p}{,} \PYG{n}{f}\PYG{p}{(}\PYG{l+m+mi}{7}\PYG{p}{)}\PYG{p}{,} \PYG{l+s+s1}{\PYGZsq{}}\PYG{l+s+s1}{o}\PYG{l+s+s1}{\PYGZsq{}}\PYG{p}{,} \PYG{n}{label} \PYG{o}{=} \PYG{l+s+s1}{\PYGZsq{}}\PYG{l+s+s1}{f(x2)}\PYG{l+s+s1}{\PYGZsq{}}\PYG{p}{)}
\PYG{n}{plt}\PYG{o}{.}\PYG{n}{plot}\PYG{p}{(}\PYG{l+m+mi}{9}\PYG{p}{,} \PYG{n}{f}\PYG{p}{(}\PYG{l+m+mi}{9}\PYG{p}{)}\PYG{p}{,} \PYG{l+s+s1}{\PYGZsq{}}\PYG{l+s+s1}{o}\PYG{l+s+s1}{\PYGZsq{}}\PYG{p}{,} \PYG{n}{label} \PYG{o}{=} \PYG{l+s+s1}{\PYGZsq{}}\PYG{l+s+s1}{f(x3)}\PYG{l+s+s1}{\PYGZsq{}}\PYG{p}{)}
\PYG{n}{plt}\PYG{o}{.}\PYG{n}{fill}\PYG{p}{(}\PYG{n}{xf}\PYG{p}{,}\PYG{n}{yf}\PYG{p}{,} \PYG{n}{color}\PYG{o}{=}\PYG{l+s+s1}{\PYGZsq{}}\PYG{l+s+s1}{r}\PYG{l+s+s1}{\PYGZsq{}}\PYG{p}{,} \PYG{n}{alpha}\PYG{o}{=}\PYG{l+m+mf}{0.4}\PYG{p}{)}
\PYG{n}{plt}\PYG{o}{.}\PYG{n}{legend}\PYG{p}{(}\PYG{p}{)}
\PYG{n}{plt}\PYG{o}{.}\PYG{n}{axvline}\PYG{p}{(}\PYG{n}{x}\PYG{o}{=}\PYG{l+m+mi}{0}\PYG{p}{,}\PYG{n}{color}\PYG{o}{=}\PYG{l+s+s1}{\PYGZsq{}}\PYG{l+s+s1}{k}\PYG{l+s+s1}{\PYGZsq{}}\PYG{p}{,}\PYG{n}{linewidth}\PYG{o}{=}\PYG{l+m+mf}{0.6}\PYG{p}{,}\PYG{n}{linestyle}\PYG{o}{=}\PYG{l+s+s1}{\PYGZsq{}}\PYG{l+s+s1}{\PYGZhy{}\PYGZhy{}}\PYG{l+s+s1}{\PYGZsq{}}\PYG{p}{)}
\PYG{n}{plt}\PYG{o}{.}\PYG{n}{axhline}\PYG{p}{(}\PYG{n}{y}\PYG{o}{=}\PYG{l+m+mi}{0}\PYG{p}{,}\PYG{n}{color}\PYG{o}{=}\PYG{l+s+s1}{\PYGZsq{}}\PYG{l+s+s1}{k}\PYG{l+s+s1}{\PYGZsq{}}\PYG{p}{,}\PYG{n}{linewidth}\PYG{o}{=}\PYG{l+m+mf}{0.6}\PYG{p}{,}\PYG{n}{linestyle}\PYG{o}{=}\PYG{l+s+s1}{\PYGZsq{}}\PYG{l+s+s1}{\PYGZhy{}\PYGZhy{}}\PYG{l+s+s1}{\PYGZsq{}}\PYG{p}{)}\PYG{p}{;}
\end{sphinxVerbatim}

\noindent\sphinxincludegraphics{{aula-18-integracao-newtonCotes_8_0}.png}


\section{Integração com Segmentos Desiguais}
\label{\detokenize{aula-18-integracao-newtonCotes:integracao-com-segmentos-desiguais}}
\sphinxAtStartPar
Até esse ponto, todas as fórmulas para integração numérica foram baseadas em dados igualmente espaçados. Na prática, existem muitas situações nas quais essa hipótese não é válida e precisamos lidar com segmentos de tamanhos distintos. Por exemplo, dados obtidos experimentalmente, muitas vezes, são desse tipo. Para tais casos, um método é aplicar a regra do trapézio para cada segmento e somar os resultados:
\begin{equation*}
\begin{split}
I = h_1 \dfrac{f(x_0) + f(x_1)}{2} + h_2 \dfrac{f(x_1) + f(x_2)}{2} + \dots + h_n \dfrac{f(x_{n - 1}) + f(x_n)}{2}
\end{split}
\end{equation*}
\sphinxAtStartPar
em que \(h_i\) é a largura do segmento \(i\).


\section{Implementações de Newton\sphinxhyphen{}Cotes: Regra do Trapézio e 1/3 Simpson Generalizadas}
\label{\detokenize{aula-18-integracao-newtonCotes:implementacoes-de-newton-cotes-regra-do-trapezio-e-1-3-simpson-generalizadas}}
\begin{sphinxVerbatim}[commandchars=\\\{\}]
\PYG{l+s+sd}{\PYGZsq{}\PYGZsq{}\PYGZsq{} Newton\PYGZhy{}Cotes: Regra do Trapezio}
\PYG{l+s+sd}{    assume Y igualmente espaçado e }
\PYG{l+s+sd}{    com pelo menos 2 pontos }
\PYG{l+s+sd}{\PYGZsq{}\PYGZsq{}\PYGZsq{}}
\PYG{k}{def} \PYG{n+nf}{integral\PYGZus{}trapezio}\PYG{p}{(}\PYG{n}{h}\PYG{p}{,}\PYG{n}{Y}\PYG{p}{)}\PYG{p}{:}
    \PYG{n}{val} \PYG{o}{=} \PYG{l+m+mf}{0.0}
    \PYG{k}{for} \PYG{n}{i} \PYG{o+ow}{in} \PYG{n+nb}{range}\PYG{p}{(}\PYG{l+m+mi}{1}\PYG{p}{,}\PYG{n}{Y}\PYG{o}{.}\PYG{n}{size}\PYG{o}{\PYGZhy{}}\PYG{l+m+mi}{1}\PYG{p}{)}\PYG{p}{:}
        \PYG{n}{val} \PYG{o}{+}\PYG{o}{=} \PYG{l+m+mi}{2}\PYG{o}{*}\PYG{n}{Y}\PYG{p}{[}\PYG{n}{i}\PYG{p}{]}
    \PYG{n}{val} \PYG{o}{=} \PYG{l+m+mf}{0.5}\PYG{o}{*}\PYG{n}{h}\PYG{o}{*}\PYG{p}{(} \PYG{n}{val} \PYG{o}{+} \PYG{n}{Y}\PYG{p}{[}\PYG{l+m+mi}{0}\PYG{p}{]} \PYG{o}{+} \PYG{n}{Y}\PYG{p}{[}\PYG{o}{\PYGZhy{}}\PYG{l+m+mi}{1}\PYG{p}{]} \PYG{p}{)}
    \PYG{k}{return} \PYG{n}{val}

\PYG{l+s+sd}{\PYGZsq{}\PYGZsq{}\PYGZsq{} Newton\PYGZhy{}Cotes: Regra 1/3 de Simpson}
\PYG{l+s+sd}{    assume Y igualmente espaçado e }
\PYG{l+s+sd}{    com pelo menos 3 pontos }
\PYG{l+s+sd}{\PYGZsq{}\PYGZsq{}\PYGZsq{}}
\PYG{k}{def} \PYG{n+nf}{integral\PYGZus{}onethird\PYGZus{}simpson}\PYG{p}{(}\PYG{n}{h}\PYG{p}{,}\PYG{n}{Y}\PYG{p}{)}\PYG{p}{:}
    \PYG{n}{val} \PYG{o}{=} \PYG{l+m+mf}{0.0}
    \PYG{k}{for} \PYG{n}{i} \PYG{o+ow}{in} \PYG{n+nb}{range}\PYG{p}{(}\PYG{l+m+mi}{1}\PYG{p}{,}\PYG{n}{Y}\PYG{o}{.}\PYG{n}{size}\PYG{o}{\PYGZhy{}}\PYG{l+m+mi}{1}\PYG{p}{,}\PYG{l+m+mi}{2}\PYG{p}{)}\PYG{p}{:}
        \PYG{n}{val} \PYG{o}{+}\PYG{o}{=} \PYG{l+m+mi}{4}\PYG{o}{*}\PYG{n}{Y}\PYG{p}{[}\PYG{n}{i}\PYG{p}{]}
    
    \PYG{k}{for} \PYG{n}{i} \PYG{o+ow}{in} \PYG{n+nb}{range}\PYG{p}{(}\PYG{l+m+mi}{2}\PYG{p}{,}\PYG{n}{Y}\PYG{o}{.}\PYG{n}{size}\PYG{o}{\PYGZhy{}}\PYG{l+m+mi}{2}\PYG{p}{,}\PYG{l+m+mi}{2}\PYG{p}{)}\PYG{p}{:}
        \PYG{n}{val} \PYG{o}{+}\PYG{o}{=} \PYG{l+m+mi}{2}\PYG{o}{*}\PYG{n}{Y}\PYG{p}{[}\PYG{n}{i}\PYG{p}{]}        
    
    \PYG{n}{val} \PYG{o}{=} \PYG{n}{h}\PYG{o}{/}\PYG{l+m+mf}{3.0}\PYG{o}{*}\PYG{p}{(} \PYG{n}{val} \PYG{o}{+} \PYG{n}{Y}\PYG{p}{[}\PYG{l+m+mi}{0}\PYG{p}{]} \PYG{o}{+} \PYG{n}{Y}\PYG{p}{[}\PYG{o}{\PYGZhy{}}\PYG{l+m+mi}{1}\PYG{p}{]} \PYG{p}{)}
    \PYG{k}{return} \PYG{n}{val}

\PYG{l+s+sd}{\PYGZsq{}\PYGZsq{}\PYGZsq{} HELPER\PYGZsq{}\PYGZsq{}\PYGZsq{}} 
\PYG{k}{def} \PYG{n+nf}{print\PYGZus{}metodo}\PYG{p}{(}\PYG{n}{flag}\PYG{p}{)}\PYG{p}{:}            
    \PYG{n+nb}{print}\PYG{p}{(}\PYG{l+s+s1}{\PYGZsq{}}\PYG{l+s+s1}{*** Método de integração: }\PYG{l+s+s1}{\PYGZsq{}} \PYG{o}{+} \PYG{n+nb}{str}\PYG{o}{.}\PYG{n}{upper}\PYG{p}{(}\PYG{n}{flag}\PYG{p}{)} \PYG{o}{+} \PYG{l+s+s1}{\PYGZsq{}}\PYG{l+s+s1}{ ***}\PYG{l+s+s1}{\PYGZsq{}}\PYG{p}{)}

\PYG{c+c1}{\PYGZsh{} limites de integração}
\PYG{n}{a} \PYG{o}{=} \PYG{l+m+mi}{0}
\PYG{n}{b} \PYG{o}{=} \PYG{l+m+mi}{93}

\PYG{c+c1}{\PYGZsh{} integrando }
\PYG{n}{f} \PYG{o}{=} \PYG{k}{lambda} \PYG{n}{v}\PYG{p}{:} \PYG{l+m+mi}{97000}\PYG{o}{*}\PYG{n}{v}\PYG{o}{/}\PYG{p}{(}\PYG{l+m+mi}{5}\PYG{o}{*}\PYG{n}{v}\PYG{o}{*}\PYG{o}{*}\PYG{l+m+mi}{2} \PYG{o}{+} \PYG{l+m+mi}{570000}\PYG{p}{)}

\PYG{c+c1}{\PYGZsh{} pontos de integração (ímpares para testar com 1/3 simpson)}
\PYG{n}{npi} \PYG{o}{=} \PYG{p}{[}\PYG{l+m+mi}{11}\PYG{p}{,}\PYG{l+m+mi}{101}\PYG{p}{,}\PYG{l+m+mi}{1001}\PYG{p}{,}\PYG{l+m+mi}{10001}\PYG{p}{,}\PYG{l+m+mi}{100001}\PYG{p}{]}

\PYG{c+c1}{\PYGZsh{} integral exata}
\PYG{n}{x} \PYG{o}{=} \PYG{n}{sy}\PYG{o}{.}\PYG{n}{Symbol}\PYG{p}{(}\PYG{l+s+s1}{\PYGZsq{}}\PYG{l+s+s1}{x}\PYG{l+s+s1}{\PYGZsq{}}\PYG{p}{)}
\PYG{n}{vex} \PYG{o}{=} \PYG{n}{sy}\PYG{o}{.}\PYG{n}{integrate}\PYG{p}{(}\PYG{n}{f}\PYG{p}{(}\PYG{n}{x}\PYG{p}{)}\PYG{p}{,}\PYG{p}{(}\PYG{n}{x}\PYG{p}{,}\PYG{n}{a}\PYG{p}{,}\PYG{n}{b}\PYG{p}{)}\PYG{p}{)}
\PYG{n+nb}{print}\PYG{p}{(}\PYG{l+s+s1}{\PYGZsq{}}\PYG{l+s+s1}{Ingegral exata (simbólica): }\PYG{l+s+s1}{\PYGZsq{}} \PYG{o}{+} \PYG{n+nb}{str}\PYG{p}{(}\PYG{n}{vex}\PYG{p}{)}\PYG{p}{)}
\PYG{n}{vex} \PYG{o}{=} \PYG{n+nb}{float}\PYG{p}{(}\PYG{n}{vex}\PYG{p}{)}
\PYG{n+nb}{print}\PYG{p}{(}\PYG{l+s+s1}{\PYGZsq{}}\PYG{l+s+s1}{Integral exata (numérica): }\PYG{l+s+s1}{\PYGZsq{}} \PYG{o}{+} \PYG{n+nb}{str}\PYG{p}{(}\PYG{n}{vex}\PYG{p}{)}\PYG{p}{)}

\PYG{n}{metodos} \PYG{o}{=} \PYG{p}{[}\PYG{l+s+s1}{\PYGZsq{}}\PYG{l+s+s1}{trapezio}\PYG{l+s+s1}{\PYGZsq{}}\PYG{p}{,}\PYG{l+s+s1}{\PYGZsq{}}\PYG{l+s+s1}{13simpson}\PYG{l+s+s1}{\PYGZsq{}}\PYG{p}{]}
\PYG{n}{no} \PYG{o}{=} \PYG{k+kc}{False}

\PYG{c+c1}{\PYGZsh{} integração numérica}
\PYG{k}{for} \PYG{n}{metodo} \PYG{o+ow}{in} \PYG{n}{metodos}\PYG{p}{:}
    \PYG{n}{I} \PYG{o}{=} \PYG{p}{[}\PYG{p}{]}
    \PYG{n}{EREL} \PYG{o}{=} \PYG{p}{[}\PYG{p}{]}
    \PYG{k}{for} \PYG{n}{n} \PYG{o+ow}{in} \PYG{n}{npi}\PYG{p}{:}
        \PYG{n}{v} \PYG{o}{=} \PYG{n}{np}\PYG{o}{.}\PYG{n}{linspace}\PYG{p}{(}\PYG{n}{a}\PYG{p}{,}\PYG{n}{b}\PYG{p}{,}\PYG{n}{num}\PYG{o}{=}\PYG{n}{n}\PYG{p}{,}\PYG{n}{endpoint}\PYG{o}{=}\PYG{k+kc}{True}\PYG{p}{)}
        \PYG{n}{fv} \PYG{o}{=} \PYG{n}{f}\PYG{p}{(}\PYG{n}{v}\PYG{p}{)}
        \PYG{n}{h} \PYG{o}{=} \PYG{p}{(}\PYG{n}{b}\PYG{o}{\PYGZhy{}}\PYG{n}{a}\PYG{p}{)}\PYG{o}{/}\PYG{n}{n}    
                
        \PYG{k}{if} \PYG{n}{metodo} \PYG{o}{==} \PYG{l+s+s1}{\PYGZsq{}}\PYG{l+s+s1}{trapezio}\PYG{l+s+s1}{\PYGZsq{}}\PYG{p}{:}
            \PYG{k}{if} \PYG{n}{no} \PYG{o}{==} \PYG{k+kc}{False}\PYG{p}{:}
                \PYG{n}{print\PYGZus{}metodo}\PYG{p}{(}\PYG{n}{metodo}\PYG{p}{)}
                \PYG{n}{no} \PYG{o}{=} \PYG{k+kc}{True}            
            \PYG{n}{val} \PYG{o}{=} \PYG{n}{integral\PYGZus{}trapezio}\PYG{p}{(}\PYG{n}{h}\PYG{p}{,}\PYG{n}{fv}\PYG{p}{)}
            \PYG{n}{erel} \PYG{o}{=} \PYG{n+nb}{abs}\PYG{p}{(}\PYG{p}{(}\PYG{n}{vex} \PYG{o}{\PYGZhy{}} \PYG{n}{val}\PYG{p}{)}\PYG{o}{/}\PYG{n}{vex}\PYG{p}{)}\PYG{o}{*}\PYG{l+m+mi}{100}
            \PYG{n}{I}\PYG{o}{.}\PYG{n}{append}\PYG{p}{(}\PYG{n}{val}\PYG{p}{)}
            \PYG{n}{EREL}\PYG{o}{.}\PYG{n}{append}\PYG{p}{(}\PYG{n}{erel}\PYG{p}{)}
            \PYG{n+nb}{print}\PYG{p}{(}\PYG{l+s+s2}{\PYGZdq{}}\PYG{l+s+s2}{no. de pontos de integração = }\PYG{l+s+si}{\PYGZob{}0:d\PYGZcb{}}\PYG{l+s+s2}{ }\PYG{l+s+se}{\PYGZbs{}t}\PYG{l+s+s2}{ I = }\PYG{l+s+si}{\PYGZob{}1:.10f\PYGZcb{}}\PYG{l+s+s2}{ }\PYG{l+s+se}{\PYGZbs{}t}\PYG{l+s+s2}{ EREL = }\PYG{l+s+si}{\PYGZob{}2:.10f\PYGZcb{}}\PYG{l+s+s2}{\PYGZpc{}}\PYG{l+s+s2}{\PYGZdq{}}\PYG{o}{.}\PYG{n}{format}\PYG{p}{(}\PYG{n}{n}\PYG{p}{,}\PYG{n}{val}\PYG{p}{,}\PYG{n}{erel}\PYG{p}{)} \PYG{p}{)}

        \PYG{k}{elif} \PYG{n}{metodo} \PYG{o}{==} \PYG{l+s+s1}{\PYGZsq{}}\PYG{l+s+s1}{13simpson}\PYG{l+s+s1}{\PYGZsq{}}\PYG{p}{:}
            \PYG{k}{if} \PYG{n}{no} \PYG{o}{==} \PYG{k+kc}{False}\PYG{p}{:}
                \PYG{n}{print\PYGZus{}metodo}\PYG{p}{(}\PYG{n}{metodo}\PYG{p}{)}
                \PYG{n}{no} \PYG{o}{=} \PYG{k+kc}{True}
            \PYG{n}{val} \PYG{o}{=} \PYG{n}{integral\PYGZus{}onethird\PYGZus{}simpson}\PYG{p}{(}\PYG{n}{h}\PYG{p}{,}\PYG{n}{fv}\PYG{p}{)}
            \PYG{n}{erel} \PYG{o}{=} \PYG{n+nb}{abs}\PYG{p}{(}\PYG{p}{(}\PYG{n}{vex} \PYG{o}{\PYGZhy{}} \PYG{n}{val}\PYG{p}{)}\PYG{o}{/}\PYG{n}{vex}\PYG{p}{)}\PYG{o}{*}\PYG{l+m+mi}{100}
            \PYG{n}{I}\PYG{o}{.}\PYG{n}{append}\PYG{p}{(}\PYG{n}{val}\PYG{p}{)}
            \PYG{n}{EREL}\PYG{o}{.}\PYG{n}{append}\PYG{p}{(}\PYG{n}{erel}\PYG{p}{)}        
            \PYG{n+nb}{print}\PYG{p}{(}\PYG{l+s+s2}{\PYGZdq{}}\PYG{l+s+s2}{no. de pontos de integração = }\PYG{l+s+si}{\PYGZob{}0:d\PYGZcb{}}\PYG{l+s+s2}{ }\PYG{l+s+se}{\PYGZbs{}t}\PYG{l+s+s2}{ I = }\PYG{l+s+si}{\PYGZob{}1:.10f\PYGZcb{}}\PYG{l+s+s2}{ }\PYG{l+s+se}{\PYGZbs{}t}\PYG{l+s+s2}{ EREL = }\PYG{l+s+si}{\PYGZob{}2:.10f\PYGZcb{}}\PYG{l+s+s2}{\PYGZpc{}}\PYG{l+s+s2}{\PYGZdq{}}\PYG{o}{.}\PYG{n}{format}\PYG{p}{(}\PYG{n}{n}\PYG{p}{,}\PYG{n}{val}\PYG{p}{,}\PYG{n}{erel}\PYG{p}{)} \PYG{p}{)}
    
    \PYG{n}{no} \PYG{o}{=} \PYG{k+kc}{False}
\end{sphinxVerbatim}

\begin{sphinxVerbatim}[commandchars=\\\{\}]
Ingegral exata (simbólica): \PYGZhy{}9700*log(114000) + 9700*log(122649)
Integral exata (numérica): 709.3432392521672
*** Método de integração: TRAPEZIO ***
no. de pontos de integração = 11 	 I = 644.6326388025 	 EREL = 9.1226076276\PYGZpc{}
no. de pontos de integração = 101 	 I = 702.3175905243 	 EREL = 0.9904441657\PYGZpc{}
no. de pontos de integração = 1001 	 I = 708.6345799441 	 EREL = 0.0999035825\PYGZpc{}
no. de pontos de integração = 10001 	 I = 709.2723117737 	 EREL = 0.0099990350\PYGZpc{}
no. de pontos de integração = 100001 	 I = 709.3361458882 	 EREL = 0.0009999903\PYGZpc{}
*** Método de integração: 13SIMPSON ***
no. de pontos de integração = 11 	 I = 644.8576901994 	 EREL = 9.0908809000\PYGZpc{}
no. de pontos de integração = 101 	 I = 702.3200388852 	 EREL = 0.9900990068\PYGZpc{}
no. de pontos de integração = 1001 	 I = 708.6346046475 	 EREL = 0.0999000999\PYGZpc{}
no. de pontos de integração = 10001 	 I = 709.2723120210 	 EREL = 0.0099990001\PYGZpc{}
no. de pontos de integração = 100001 	 I = 709.3361458907 	 EREL = 0.0009999900\PYGZpc{}
\end{sphinxVerbatim}


\subsection{Tarefa}
\label{\detokenize{aula-18-integracao-newtonCotes:tarefa}}
\sphinxAtStartPar
Implemente uma função para realizar a integração numérica pela regra 3/8 de Simpson, use\sphinxhyphen{}a para calcular o valor de
\begin{equation*}
\begin{split}\int_{a=0}^{b=93} \frac{97000v}{(5v^2 + 570000)} \, dx\end{split}
\end{equation*}
\sphinxAtStartPar
e compare o resultado com os obtidos pelas anteriores.


\section{Integração simbólica}
\label{\detokenize{aula-18-integracao-newtonCotes:integracao-simbolica}}
\sphinxAtStartPar
Vamos ver alguns exemplos de integração simbólica. Para termos uma impressão mais elegante de expressões, antes usamos a seguinte instrução:

\begin{sphinxVerbatim}[commandchars=\\\{\}]
\PYG{n}{sy}\PYG{o}{.}\PYG{n}{init\PYGZus{}printing}\PYG{p}{(}\PYG{p}{)}
\end{sphinxVerbatim}


\subsection{Regra quadratura de Simpson}
\label{\detokenize{aula-18-integracao-newtonCotes:regra-quadratura-de-simpson}}
\sphinxAtStartPar
Vamos usar símbolos para chegar à expressão da regra de quadratura de Simpson.

\begin{sphinxVerbatim}[commandchars=\\\{\}]
\PYG{n}{a}\PYG{p}{,}\PYG{n}{b}\PYG{p}{,}\PYG{n}{X} \PYG{o}{=} \PYG{n}{sy}\PYG{o}{.}\PYG{n}{symbols}\PYG{p}{(}\PYG{l+s+s2}{\PYGZdq{}}\PYG{l+s+s2}{a, b, x}\PYG{l+s+s2}{\PYGZdq{}}\PYG{p}{)}
\PYG{n}{f} \PYG{o}{=} \PYG{n}{sy}\PYG{o}{.}\PYG{n}{Function}\PYG{p}{(}\PYG{l+s+s2}{\PYGZdq{}}\PYG{l+s+s2}{f}\PYG{l+s+s2}{\PYGZdq{}}\PYG{p}{)}
\end{sphinxVerbatim}

\sphinxAtStartPar
Definimos tuplas para os pontos de amostra e pesos.

\begin{sphinxVerbatim}[commandchars=\\\{\}]
\PYG{n}{x} \PYG{o}{=} \PYG{n}{a}\PYG{p}{,} \PYG{p}{(}\PYG{n}{a} \PYG{o}{+} \PYG{n}{b}\PYG{p}{)}\PYG{o}{/}\PYG{l+m+mi}{2}\PYG{p}{,} \PYG{n}{b} \PYG{c+c1}{\PYGZsh{} ponto médio }
\PYG{n}{w} \PYG{o}{=} \PYG{p}{[}\PYG{n}{sy}\PYG{o}{.}\PYG{n}{symbols}\PYG{p}{(}\PYG{l+s+s2}{\PYGZdq{}}\PYG{l+s+s2}{w\PYGZus{}}\PYG{l+s+si}{\PYGZpc{}d}\PYG{l+s+s2}{\PYGZdq{}} \PYG{o}{\PYGZpc{}} \PYG{n}{i}\PYG{p}{)} \PYG{k}{for} \PYG{n}{i} \PYG{o+ow}{in} \PYG{n+nb}{range}\PYG{p}{(}\PYG{n+nb}{len}\PYG{p}{(}\PYG{n}{x}\PYG{p}{)}\PYG{p}{)}\PYG{p}{]} \PYG{c+c1}{\PYGZsh{} pesos }
\end{sphinxVerbatim}

\begin{sphinxVerbatim}[commandchars=\\\{\}]
\PYG{n}{q} \PYG{o}{=} \PYG{n+nb}{sum}\PYG{p}{(}\PYG{p}{[}\PYG{n}{w}\PYG{p}{[}\PYG{n}{i}\PYG{p}{]} \PYG{o}{*} \PYG{n}{f}\PYG{p}{(}\PYG{n}{x}\PYG{p}{[}\PYG{n}{i}\PYG{p}{]}\PYG{p}{)} \PYG{k}{for} \PYG{n}{i} \PYG{o+ow}{in} \PYG{n+nb}{range}\PYG{p}{(}\PYG{n+nb}{len}\PYG{p}{(}\PYG{n}{x}\PYG{p}{)}\PYG{p}{)}\PYG{p}{]}\PYG{p}{)}
\PYG{n}{q}
\end{sphinxVerbatim}

\noindent\sphinxincludegraphics{{aula-18-integracao-newtonCotes_19_0}.png}

\sphinxAtStartPar
Para calcular valores aproximados dos pesos \(w_i\), escolhemos a base polinomial
\begin{equation*}
\begin{split}\{ \phi_n(x) = x^n \}_{n=0}^2\end{split}
\end{equation*}
\sphinxAtStartPar
para a interpolação de \(f(x)\) e um objeto simbólico para representar cada uma dessas funções.

\begin{sphinxVerbatim}[commandchars=\\\{\}]
\PYG{n}{phi} \PYG{o}{=} \PYG{p}{[}\PYG{n}{sy}\PYG{o}{.}\PYG{n}{Lambda}\PYG{p}{(}\PYG{n}{X}\PYG{p}{,} \PYG{n}{X}\PYG{o}{*}\PYG{o}{*}\PYG{n}{n}\PYG{p}{)} \PYG{k}{for} \PYG{n}{n} \PYG{o+ow}{in} \PYG{n+nb}{range}\PYG{p}{(}\PYG{n+nb}{len}\PYG{p}{(}\PYG{n}{x}\PYG{p}{)}\PYG{p}{)}\PYG{p}{]}
\PYG{n}{phi}
\end{sphinxVerbatim}

\noindent\sphinxincludegraphics{{aula-18-integracao-newtonCotes_21_0}.png}

\sphinxAtStartPar
Agora temos que descobrir os valores dos pesos. A integral \(\int_a^b \phi_n(x) \, dx\) pode ser calculada analiticamente para cada função de base. Isto nos ajuda a resolver o seguinte sistema:
\begin{equation*}
\begin{split}\sum\limits_{i=0}^2 w_i \phi_n(x_i) = \int_a^b \phi_n(x) \, dx\end{split}
\end{equation*}
\sphinxAtStartPar
O sistema pode ser construído no \sphinxcode{\sphinxupquote{sympy}} da seguinte forma:

\begin{sphinxVerbatim}[commandchars=\\\{\}]
\PYG{n}{eqs} \PYG{o}{=} \PYG{p}{[}\PYG{n}{q}\PYG{o}{.}\PYG{n}{subs}\PYG{p}{(}\PYG{n}{f}\PYG{p}{,} \PYG{n}{phi}\PYG{p}{[}\PYG{n}{n}\PYG{p}{]}\PYG{p}{)} \PYG{o}{\PYGZhy{}} \PYG{n}{sy}\PYG{o}{.}\PYG{n}{integrate}\PYG{p}{(}\PYG{n}{phi}\PYG{p}{[}\PYG{n}{n}\PYG{p}{]}\PYG{p}{(}\PYG{n}{X}\PYG{p}{)}\PYG{p}{,} \PYG{p}{(}\PYG{n}{X}\PYG{p}{,} \PYG{n}{a}\PYG{p}{,} \PYG{n}{b}\PYG{p}{)}\PYG{p}{)} \PYG{k}{for} \PYG{n}{n} \PYG{o+ow}{in} \PYG{n+nb}{range}\PYG{p}{(}\PYG{n+nb}{len}\PYG{p}{(}\PYG{n}{phi}\PYG{p}{)}\PYG{p}{)}\PYG{p}{]}
\PYG{n}{eqs}
\end{sphinxVerbatim}

\noindent\sphinxincludegraphics{{aula-18-integracao-newtonCotes_24_0}.png}

\sphinxAtStartPar
Em seguida, resolvemos o sistema para obter as expressões analíticas para os pesos:

\begin{sphinxVerbatim}[commandchars=\\\{\}]
\PYG{n}{w\PYGZus{}sol} \PYG{o}{=} \PYG{n}{sy}\PYG{o}{.}\PYG{n}{solve}\PYG{p}{(}\PYG{n}{eqs}\PYG{p}{,}\PYG{n}{w}\PYG{p}{)}
\PYG{n}{w\PYGZus{}sol}
\end{sphinxVerbatim}

\noindent\sphinxincludegraphics{{aula-18-integracao-newtonCotes_26_0}.png}

\sphinxAtStartPar
Substituindo a solução na expressão simbólica para a regra de quadratura:

\begin{sphinxVerbatim}[commandchars=\\\{\}]
\PYG{n}{q}\PYG{o}{.}\PYG{n}{subs}\PYG{p}{(}\PYG{n}{w\PYGZus{}sol}\PYG{p}{)}\PYG{o}{.}\PYG{n}{simplify}\PYG{p}{(}\PYG{p}{)}
\end{sphinxVerbatim}

\noindent\sphinxincludegraphics{{aula-18-integracao-newtonCotes_28_0}.png}

\sphinxAtStartPar
Podemos verificar no material que esta é, de fato, a expressão para a regra de quadratura de Simpson.


\section{Integração múltipla}
\label{\detokenize{aula-18-integracao-newtonCotes:integracao-multipla}}
\sphinxAtStartPar
A integração em 2 ou mais variáveis pode ser feita usando as funções \sphinxcode{\sphinxupquote{dblquad}}, \sphinxcode{\sphinxupquote{tplquad}} e \sphinxcode{\sphinxupquote{nquad}}, onde o número de funções e de limites de integração deve se adequar ao tipo de integral.

\sphinxAtStartPar
Abaixo, temos alguns exemplos:


\subsection{Integração dupla}
\label{\detokenize{aula-18-integracao-newtonCotes:integracao-dupla}}
\sphinxAtStartPar
Neste exemplo, integramos
\$\(\int_0^1\int_0^1 e^{-x^2 - y^2} \, dx dy\)\$

\begin{sphinxVerbatim}[commandchars=\\\{\}]
\PYG{n}{integrate}\PYG{o}{.}\PYG{n}{dblquad}\PYG{p}{(}\PYG{k}{lambda} \PYG{n}{x}\PYG{p}{,} \PYG{n}{y}\PYG{p}{:} \PYG{n}{np}\PYG{o}{.}\PYG{n}{exp}\PYG{p}{(}\PYG{o}{\PYGZhy{}}\PYG{n}{x}\PYG{o}{*}\PYG{o}{*}\PYG{l+m+mi}{2}\PYG{o}{\PYGZhy{}}\PYG{n}{y}\PYG{o}{*}\PYG{o}{*}\PYG{l+m+mi}{2}\PYG{p}{)}\PYG{p}{,} \PYG{l+m+mi}{0}\PYG{p}{,} \PYG{l+m+mi}{1}\PYG{p}{,} \PYG{k}{lambda} \PYG{n}{x}\PYG{p}{:} \PYG{l+m+mi}{0}\PYG{p}{,} \PYG{k}{lambda} \PYG{n}{x}\PYG{p}{:} \PYG{l+m+mi}{1}\PYG{p}{)}
\end{sphinxVerbatim}

\noindent\sphinxincludegraphics{{aula-18-integracao-newtonCotes_32_0}.png}


\subsection{Integração tripla}
\label{\detokenize{aula-18-integracao-newtonCotes:integracao-tripla}}
\sphinxAtStartPar
Neste exemplo, integramos
\$\(\int_0^1\int_0^1\int_0^1 e^{-x^2 - y^2 - z^2} \, dx dy dz\)\$

\begin{sphinxVerbatim}[commandchars=\\\{\}]
\PYG{k}{def} \PYG{n+nf}{f}\PYG{p}{(}\PYG{n}{x}\PYG{p}{,} \PYG{n}{y}\PYG{p}{,} \PYG{n}{z}\PYG{p}{)}\PYG{p}{:} 
    \PYG{k}{return} \PYG{n}{np}\PYG{o}{.}\PYG{n}{exp}\PYG{p}{(}\PYG{o}{\PYGZhy{}}\PYG{n}{x}\PYG{o}{*}\PYG{o}{*}\PYG{l+m+mi}{2}\PYG{o}{\PYGZhy{}}\PYG{n}{y}\PYG{o}{*}\PYG{o}{*}\PYG{l+m+mi}{2}\PYG{o}{\PYGZhy{}}\PYG{n}{z}\PYG{o}{*}\PYG{o}{*}\PYG{l+m+mi}{2}\PYG{p}{)} 

\PYG{n}{a}\PYG{p}{,} \PYG{n}{b} \PYG{o}{=} \PYG{l+m+mi}{0}\PYG{p}{,} \PYG{l+m+mi}{1}
\PYG{n}{g}\PYG{p}{,} \PYG{n}{h} \PYG{o}{=} \PYG{k}{lambda} \PYG{n}{x}\PYG{p}{:} \PYG{l+m+mi}{0}\PYG{p}{,} \PYG{k}{lambda} \PYG{n}{x}\PYG{p}{:} \PYG{l+m+mi}{1}
\PYG{n}{q}\PYG{p}{,} \PYG{n}{r} \PYG{o}{=} \PYG{k}{lambda} \PYG{n}{x}\PYG{p}{,} \PYG{n}{y}\PYG{p}{:} \PYG{l+m+mi}{0}\PYG{p}{,} \PYG{k}{lambda} \PYG{n}{x}\PYG{p}{,} \PYG{n}{y}\PYG{p}{:} \PYG{l+m+mi}{1}
\PYG{n}{integrate}\PYG{o}{.}\PYG{n}{tplquad}\PYG{p}{(}\PYG{n}{f}\PYG{p}{,} \PYG{l+m+mi}{0}\PYG{p}{,} \PYG{l+m+mi}{1}\PYG{p}{,} \PYG{n}{g}\PYG{p}{,} \PYG{n}{h}\PYG{p}{,} \PYG{n}{q}\PYG{p}{,} \PYG{n}{r}\PYG{p}{)} 
\end{sphinxVerbatim}

\noindent\sphinxincludegraphics{{aula-18-integracao-newtonCotes_34_0}.png}


\chapter{Integração Numérica: Quadratura Gaussiana}
\label{\detokenize{aula-19-quadratura-gaussiana:integracao-numerica-quadratura-gaussiana}}\label{\detokenize{aula-19-quadratura-gaussiana::doc}}
\sphinxAtStartPar
Uma característica das fórmulas de Newton\sphinxhyphen{}Cotes é estimar a integral com base em valores da função tomados em nós igualmente espaçados. Consequentemente, a posição desses nós é predeterminada, isto é, fixada.

\sphinxAtStartPar
Agora, suponhamos que a restrição de ter nós fixados seja removida e que pudéssemos escolher arbitrariamente as posições por eles ocupadas. Com uma escolha adequada, será que conseguiríamos reduzir o erro de integração?

\sphinxAtStartPar
De fato, o método da Quadratura Gaussiana permite\sphinxhyphen{}nos estabelecer um conjunto mínimo de \sphinxstyleemphasis{nós} e \sphinxstyleemphasis{pesos} que realizam uma espécie de “balanceamento” do erro (subestimação vs. superestimação) e determinam aproximações numéricas de integrais com alta ordem.

\sphinxAtStartPar
As figuras abaixo mostram, por exemplo, como a integral de uma função \(f(x)\) é melhor calculada através da colocação de pontos em posições “especiais”. A regra do trapézio é posta em contraste com uma outra regra, que aprenderemos, cujo par de pontos, embora defina também um trapézio, ajuda a estimar melhor a integral de \(f(x)\) do que o observado no primeiro caso.

\sphinxAtStartPar
Em particular, daremos enfoque à chamada \sphinxstyleemphasis{Quadratura de Gauss\sphinxhyphen{}Legendre}.

\begin{sphinxVerbatim}[commandchars=\\\{\}]
\PYG{k+kn}{import} \PYG{n+nn}{matplotlib}\PYG{n+nn}{.}\PYG{n+nn}{pyplot} \PYG{k}{as} \PYG{n+nn}{plt}
\PYG{k+kn}{import} \PYG{n+nn}{numpy}\PYG{n+nn}{.}\PYG{n+nn}{polynomial}\PYG{n+nn}{.}\PYG{n+nn}{legendre} \PYG{k}{as} \PYG{n+nn}{leg}
\PYG{k+kn}{import} \PYG{n+nn}{numpy} \PYG{k}{as} \PYG{n+nn}{np}
\PYG{o}{\PYGZpc{}}\PYG{k}{matplotlib} inline 
\end{sphinxVerbatim}

\begin{sphinxVerbatim}[commandchars=\\\{\}]
\PYG{k}{def} \PYG{n+nf}{f}\PYG{p}{(}\PYG{n}{x}\PYG{p}{)}\PYG{p}{:}
    \PYG{k}{return} \PYG{o}{\PYGZhy{}}\PYG{n}{x}\PYG{o}{*}\PYG{o}{*}\PYG{l+m+mi}{2} \PYG{o}{+} \PYG{l+m+mi}{15}\PYG{o}{*}\PYG{n}{x} \PYG{o}{\PYGZhy{}} \PYG{l+m+mi}{10}

\PYG{n}{x} \PYG{o}{=} \PYG{n}{np}\PYG{o}{.}\PYG{n}{linspace}\PYG{p}{(}\PYG{l+m+mf}{0.5}\PYG{p}{,}\PYG{l+m+mi}{10}\PYG{p}{,}\PYG{l+m+mi}{100}\PYG{p}{)}
\PYG{n}{y} \PYG{o}{=} \PYG{n}{f}\PYG{p}{(}\PYG{n}{x}\PYG{p}{)}

\PYG{c+c1}{\PYGZsh{} Representação geométrica da regra do trapézio}

\PYG{n}{plt}\PYG{o}{.}\PYG{n}{figure}\PYG{p}{(}\PYG{p}{)}
\PYG{n}{plt}\PYG{o}{.}\PYG{n}{plot}\PYG{p}{(}\PYG{n}{x}\PYG{p}{,}\PYG{n}{y}\PYG{p}{,} \PYG{n}{label} \PYG{o}{=} \PYG{l+s+s1}{\PYGZsq{}}\PYG{l+s+s1}{f(x)}\PYG{l+s+s1}{\PYGZsq{}}\PYG{p}{)}
\PYG{n}{plt}\PYG{o}{.}\PYG{n}{plot}\PYG{p}{(}\PYG{p}{[}\PYG{l+m+mi}{2}\PYG{p}{,}\PYG{l+m+mi}{8}\PYG{p}{]}\PYG{p}{,} \PYG{p}{[}\PYG{n}{f}\PYG{p}{(}\PYG{l+m+mi}{2}\PYG{p}{)}\PYG{p}{,}\PYG{n}{f}\PYG{p}{(}\PYG{l+m+mi}{8}\PYG{p}{)}\PYG{p}{]}\PYG{p}{,} \PYG{l+s+s1}{\PYGZsq{}}\PYG{l+s+s1}{r}\PYG{l+s+s1}{\PYGZsq{}}\PYG{p}{,} \PYG{n}{label} \PYG{o}{=} \PYG{l+s+s1}{\PYGZsq{}}\PYG{l+s+s1}{f1(x)}\PYG{l+s+s1}{\PYGZsq{}}\PYG{p}{)}
\PYG{n}{plt}\PYG{o}{.}\PYG{n}{plot}\PYG{p}{(}\PYG{l+m+mi}{2}\PYG{p}{,}\PYG{n}{f}\PYG{p}{(}\PYG{l+m+mi}{2}\PYG{p}{)}\PYG{p}{,} \PYG{l+s+s1}{\PYGZsq{}}\PYG{l+s+s1}{o}\PYG{l+s+s1}{\PYGZsq{}}\PYG{p}{,} \PYG{n}{label} \PYG{o}{=} \PYG{l+s+s1}{\PYGZsq{}}\PYG{l+s+s1}{f(a)}\PYG{l+s+s1}{\PYGZsq{}}\PYG{p}{)}
\PYG{n}{plt}\PYG{o}{.}\PYG{n}{plot}\PYG{p}{(}\PYG{l+m+mi}{8}\PYG{p}{,}\PYG{n}{f}\PYG{p}{(}\PYG{l+m+mi}{8}\PYG{p}{)}\PYG{p}{,} \PYG{l+s+s1}{\PYGZsq{}}\PYG{l+s+s1}{o}\PYG{l+s+s1}{\PYGZsq{}}\PYG{p}{,} \PYG{n}{label} \PYG{o}{=} \PYG{l+s+s1}{\PYGZsq{}}\PYG{l+s+s1}{f(b)}\PYG{l+s+s1}{\PYGZsq{}}\PYG{p}{)}
\PYG{n}{plt}\PYG{o}{.}\PYG{n}{fill}\PYG{p}{(}\PYG{p}{[}\PYG{l+m+mi}{2}\PYG{p}{,} \PYG{l+m+mi}{2}\PYG{p}{,} \PYG{l+m+mi}{8}\PYG{p}{,} \PYG{l+m+mi}{8}\PYG{p}{]}\PYG{p}{,}\PYG{p}{[}\PYG{l+m+mi}{0}\PYG{p}{,} \PYG{n}{f}\PYG{p}{(}\PYG{l+m+mi}{2}\PYG{p}{)}\PYG{p}{,} \PYG{n}{f}\PYG{p}{(}\PYG{l+m+mi}{8}\PYG{p}{)}\PYG{p}{,} \PYG{l+m+mi}{0}\PYG{p}{]}\PYG{p}{,}\PYG{n}{color}\PYG{o}{=}\PYG{l+s+s1}{\PYGZsq{}}\PYG{l+s+s1}{r}\PYG{l+s+s1}{\PYGZsq{}}\PYG{p}{,}\PYG{n}{alpha}\PYG{o}{=}\PYG{l+m+mf}{0.4}\PYG{p}{,} \PYG{n}{label} \PYG{o}{=} \PYG{l+s+s1}{\PYGZsq{}}\PYG{l+s+s1}{I}\PYG{l+s+s1}{\PYGZsq{}}\PYG{p}{)}
\PYG{n}{plt}\PYG{o}{.}\PYG{n}{title}\PYG{p}{(}\PYG{l+s+s1}{\PYGZsq{}}\PYG{l+s+s1}{Regra do Trapézio}\PYG{l+s+s1}{\PYGZsq{}}\PYG{p}{)}
\PYG{n}{plt}\PYG{o}{.}\PYG{n}{legend}\PYG{p}{(}\PYG{p}{)}
\PYG{n}{plt}\PYG{o}{.}\PYG{n}{axvline}\PYG{p}{(}\PYG{n}{x}\PYG{o}{=}\PYG{l+m+mi}{0}\PYG{p}{,}\PYG{n}{color}\PYG{o}{=}\PYG{l+s+s1}{\PYGZsq{}}\PYG{l+s+s1}{k}\PYG{l+s+s1}{\PYGZsq{}}\PYG{p}{,}\PYG{n}{linewidth}\PYG{o}{=}\PYG{l+m+mf}{0.6}\PYG{p}{,}\PYG{n}{linestyle}\PYG{o}{=}\PYG{l+s+s1}{\PYGZsq{}}\PYG{l+s+s1}{\PYGZhy{}\PYGZhy{}}\PYG{l+s+s1}{\PYGZsq{}}\PYG{p}{)}
\PYG{n}{plt}\PYG{o}{.}\PYG{n}{axhline}\PYG{p}{(}\PYG{n}{y}\PYG{o}{=}\PYG{l+m+mi}{0}\PYG{p}{,}\PYG{n}{color}\PYG{o}{=}\PYG{l+s+s1}{\PYGZsq{}}\PYG{l+s+s1}{k}\PYG{l+s+s1}{\PYGZsq{}}\PYG{p}{,}\PYG{n}{linewidth}\PYG{o}{=}\PYG{l+m+mf}{0.6}\PYG{p}{,}\PYG{n}{linestyle}\PYG{o}{=}\PYG{l+s+s1}{\PYGZsq{}}\PYG{l+s+s1}{\PYGZhy{}\PYGZhy{}}\PYG{l+s+s1}{\PYGZsq{}}\PYG{p}{)}

\PYG{c+c1}{\PYGZsh{} Representação geométrica da quadratura de Gauss}

\PYG{k}{def} \PYG{n+nf}{f1}\PYG{p}{(}\PYG{n}{x}\PYG{p}{)}\PYG{p}{:}
    \PYG{k}{return} \PYG{l+m+mf}{5.0}\PYG{o}{*}\PYG{n}{x} \PYG{o}{+} \PYG{l+m+mf}{11.0}

\PYG{n}{plt}\PYG{o}{.}\PYG{n}{figure}\PYG{p}{(}\PYG{p}{)}
\PYG{n}{plt}\PYG{o}{.}\PYG{n}{plot}\PYG{p}{(}\PYG{n}{x}\PYG{p}{,}\PYG{n}{y}\PYG{p}{,} \PYG{n}{label} \PYG{o}{=} \PYG{l+s+s1}{\PYGZsq{}}\PYG{l+s+s1}{f(x)}\PYG{l+s+s1}{\PYGZsq{}}\PYG{p}{)}
\PYG{n}{plt}\PYG{o}{.}\PYG{n}{plot}\PYG{p}{(}\PYG{p}{[}\PYG{l+m+mi}{2}\PYG{p}{,}\PYG{l+m+mi}{8}\PYG{p}{]}\PYG{p}{,} \PYG{p}{[}\PYG{n}{f1}\PYG{p}{(}\PYG{l+m+mi}{2}\PYG{p}{)}\PYG{p}{,}\PYG{n}{f1}\PYG{p}{(}\PYG{l+m+mi}{8}\PYG{p}{)}\PYG{p}{]}\PYG{p}{,} \PYG{l+s+s1}{\PYGZsq{}}\PYG{l+s+s1}{r}\PYG{l+s+s1}{\PYGZsq{}}\PYG{p}{,} \PYG{n}{label} \PYG{o}{=} \PYG{l+s+s1}{\PYGZsq{}}\PYG{l+s+s1}{f1(x)}\PYG{l+s+s1}{\PYGZsq{}}\PYG{p}{)}
\PYG{n}{plt}\PYG{o}{.}\PYG{n}{plot}\PYG{p}{(}\PYG{l+m+mi}{3}\PYG{p}{,}\PYG{n}{f}\PYG{p}{(}\PYG{l+m+mi}{3}\PYG{p}{)}\PYG{p}{,} \PYG{l+s+s1}{\PYGZsq{}}\PYG{l+s+s1}{o}\PYG{l+s+s1}{\PYGZsq{}}\PYG{p}{,} \PYG{n}{label} \PYG{o}{=} \PYG{l+s+s1}{\PYGZsq{}}\PYG{l+s+s1}{f(a)}\PYG{l+s+s1}{\PYGZsq{}}\PYG{p}{)}
\PYG{n}{plt}\PYG{o}{.}\PYG{n}{plot}\PYG{p}{(}\PYG{l+m+mi}{7}\PYG{p}{,}\PYG{n}{f}\PYG{p}{(}\PYG{l+m+mi}{7}\PYG{p}{)}\PYG{p}{,} \PYG{l+s+s1}{\PYGZsq{}}\PYG{l+s+s1}{o}\PYG{l+s+s1}{\PYGZsq{}}\PYG{p}{,} \PYG{n}{label} \PYG{o}{=} \PYG{l+s+s1}{\PYGZsq{}}\PYG{l+s+s1}{f(b)}\PYG{l+s+s1}{\PYGZsq{}}\PYG{p}{)}
\PYG{n}{plt}\PYG{o}{.}\PYG{n}{fill}\PYG{p}{(}\PYG{p}{[}\PYG{l+m+mi}{2}\PYG{p}{,} \PYG{l+m+mi}{2}\PYG{p}{,} \PYG{l+m+mi}{8}\PYG{p}{,} \PYG{l+m+mi}{8}\PYG{p}{]}\PYG{p}{,}\PYG{p}{[}\PYG{l+m+mi}{0}\PYG{p}{,} \PYG{n}{f1}\PYG{p}{(}\PYG{l+m+mi}{2}\PYG{p}{)}\PYG{p}{,} \PYG{n}{f1}\PYG{p}{(}\PYG{l+m+mi}{8}\PYG{p}{)}\PYG{p}{,} \PYG{l+m+mi}{0}\PYG{p}{]}\PYG{p}{,}\PYG{n}{color}\PYG{o}{=}\PYG{l+s+s1}{\PYGZsq{}}\PYG{l+s+s1}{r}\PYG{l+s+s1}{\PYGZsq{}}\PYG{p}{,}\PYG{n}{alpha}\PYG{o}{=}\PYG{l+m+mf}{0.4}\PYG{p}{,} \PYG{n}{label} \PYG{o}{=} \PYG{l+s+s1}{\PYGZsq{}}\PYG{l+s+s1}{I}\PYG{l+s+s1}{\PYGZsq{}}\PYG{p}{)}
\PYG{n}{plt}\PYG{o}{.}\PYG{n}{title}\PYG{p}{(}\PYG{l+s+s1}{\PYGZsq{}}\PYG{l+s+s1}{Quadratura de Gauss}\PYG{l+s+s1}{\PYGZsq{}}\PYG{p}{)}
\PYG{n}{plt}\PYG{o}{.}\PYG{n}{legend}\PYG{p}{(}\PYG{p}{)}
\PYG{n}{plt}\PYG{o}{.}\PYG{n}{axvline}\PYG{p}{(}\PYG{n}{x}\PYG{o}{=}\PYG{l+m+mi}{0}\PYG{p}{,}\PYG{n}{color}\PYG{o}{=}\PYG{l+s+s1}{\PYGZsq{}}\PYG{l+s+s1}{k}\PYG{l+s+s1}{\PYGZsq{}}\PYG{p}{,}\PYG{n}{linewidth}\PYG{o}{=}\PYG{l+m+mf}{0.6}\PYG{p}{,}\PYG{n}{linestyle}\PYG{o}{=}\PYG{l+s+s1}{\PYGZsq{}}\PYG{l+s+s1}{\PYGZhy{}\PYGZhy{}}\PYG{l+s+s1}{\PYGZsq{}}\PYG{p}{)}
\PYG{n}{plt}\PYG{o}{.}\PYG{n}{axhline}\PYG{p}{(}\PYG{n}{y}\PYG{o}{=}\PYG{l+m+mi}{0}\PYG{p}{,}\PYG{n}{color}\PYG{o}{=}\PYG{l+s+s1}{\PYGZsq{}}\PYG{l+s+s1}{k}\PYG{l+s+s1}{\PYGZsq{}}\PYG{p}{,}\PYG{n}{linewidth}\PYG{o}{=}\PYG{l+m+mf}{0.6}\PYG{p}{,}\PYG{n}{linestyle}\PYG{o}{=}\PYG{l+s+s1}{\PYGZsq{}}\PYG{l+s+s1}{\PYGZhy{}\PYGZhy{}}\PYG{l+s+s1}{\PYGZsq{}}\PYG{p}{)}\PYG{p}{;}
\end{sphinxVerbatim}

\noindent\sphinxincludegraphics{{aula-19-quadratura-gaussiana_2_0}.png}

\noindent\sphinxincludegraphics{{aula-19-quadratura-gaussiana_2_1}.png}


\section{Dedução da Fórmula de Gauss\sphinxhyphen{}Legendre para Dois Pontos}
\label{\detokenize{aula-19-quadratura-gaussiana:deducao-da-formula-de-gauss-legendre-para-dois-pontos}}
\sphinxAtStartPar
O objetivo da quadratura de Gauss\sphinxhyphen{}Legendre é determinar os coeficientes de uma equação da forma
\begin{equation*}
\begin{split}
I \cong c_0 f(x_0) + c_1 f(x_1) \qquad (1)
\end{split}
\end{equation*}
\sphinxAtStartPar
em que os \(c\)’s são coeficientes desconhecidos. Entretanto, em contraste com a regra do trapézio que usa extremidades fixas \(a\) e \(b\), os argumentos da função \(x_0\) e \(x_1\) não são fixados nas extremidades, mas são incógnitas. Assim, agora temos um total de quatro incógnitas que devem ser calculadas e necessitamos de quatro condições para determiná\sphinxhyphen{}las exatamente.

\sphinxAtStartPar
Podemos obter duas dessas condições, supondo que a Equação (1) calcula a integral de uma função constante e de uma função linear exatamente. Então, para chegar a duas outras condições, simplesmente estendemos esse raciocínio supondo que ele também calculará a integral de uma função parabólica (\(y = x^2\)) e de uma cúbica (\(y = x^3\)) exatamente. Fazendo isso, determinamos todas as quatro incógnitas e, de quebra, deduzimos uma fórmula de integração linear de dois pontos que é exata para funções cúbicas. As quatro equações a serem resolvidas são:
\begin{equation*}
\begin{split}
c_0 f(x_0) + c_1 f(x_1) = \int _{-1}^{1} 1 dx = 2 \qquad (2) \\
c_0 f(x_0) + c_1 f(x_1) = \int _{-1}^{1} x dx = 0 \qquad (3) \\
c_0 f(x_0) + c_1 f(x_1) = \int _{-1}^{1} x^2 dx = \dfrac{2}{3} \qquad (4) \\
c_0 f(x_0) + c_1 f(x_1) = \int _{-1}^{1} x^3 dx = 0 \qquad (5)
\end{split}
\end{equation*}
\sphinxAtStartPar
As Equações (2) até (5) podem ser resolvidas simultaneamente por
\begin{equation*}
\begin{split}
c_0 = c_1 = 1 \\
x_0 = -\dfrac{1}{\sqrt{3}} \\
x_1 = \dfrac{1}{\sqrt{3}}
\end{split}
\end{equation*}
\sphinxAtStartPar
o que pode ser substituído na Equação (1) para fornecer a fórmula de Gauss\sphinxhyphen{}Legendre de dois pontos
\begin{equation*}
\begin{split}
I \cong f \left( \dfrac{-1}{\sqrt{3}} \right) + f \left( \dfrac{1}{\sqrt{3}} \right) \qquad (6)
\end{split}
\end{equation*}
\sphinxAtStartPar
Logo, chegamos ao interessante resultado que uma simples adição de valores da função em \(x = 1/\sqrt{3}\) e \(−1/\sqrt{3}\) fornece uma estimativa da integral que tem acurácia de terceira ordem.

\sphinxAtStartPar
Observe que os extremos de integração nas Equações (2) até (5) são de \(−1\) a \(1\). Isso foi feito para simplificar a matemática e tornar a formulação tão geral quanto possível. Uma simples mudança de variável pode ser usada para transformar outros extremos de integração para essa fórmula, o que se faz supondo que uma nova variável \(x_d\) está relacionada à variável original \(x\) de uma forma linear como em
\begin{equation*}
\begin{split}
x = a_0 + a_1 x_d \qquad (7)
\end{split}
\end{equation*}
\sphinxAtStartPar
Se o extremo inferior, \(x = a\), corresponder a \(x_d = −1\), esses valores podem ser substituídos na Equação (7) para fornecer
\begin{equation*}
\begin{split}
a = a_0 + a_1 (- 1) \qquad (8)
\end{split}
\end{equation*}
\sphinxAtStartPar
Analogamente, o extremo superior, \(x = b\), corresponde a \(x_d = 1\), e fornece
\begin{equation*}
\begin{split}
b = a_0 + a_1 (1) \qquad (9)
\end{split}
\end{equation*}
\sphinxAtStartPar
As Equações (8) e (9) podem ser resolvidas simultaneamente por
\begin{equation*}
\begin{split}
a_0 = \dfrac{b + a}{2}
\end{split}
\end{equation*}
\sphinxAtStartPar
e
\begin{equation*}
\begin{split}
a_1 = \dfrac{b - a}{2}
\end{split}
\end{equation*}
\sphinxAtStartPar
o que pode ser substituído na Equação (7) para fornecer
\begin{equation*}
\begin{split}
x = \dfrac{(b + a) + (b - a) x_d}{2} \qquad (10)
\end{split}
\end{equation*}
\sphinxAtStartPar
Essa equação pode ser derivada para dar
\begin{equation*}
\begin{split}
dx = \dfrac{b - a}{2} dx_d \qquad (11)
\end{split}
\end{equation*}
\sphinxAtStartPar
Os valores de \(x\) e \(dx\), nas Equações (10) e (11), respectivamente, podem ser substituídos na equação a ser integrada. Essas substituições transformam efetivamente o intervalo de integração sem mudar o valor da integral.


\section{Fórmulas com Mais Pontos}
\label{\detokenize{aula-19-quadratura-gaussiana:formulas-com-mais-pontos}}
\sphinxAtStartPar
Além da fórmula de dois pontos descrita acima, versões com mais pontos podem ser desenvolvidas na forma geral
\begin{equation*}
\begin{split}
I \cong c_0 f(x_0) + c_1 f(x_1) + \dots + c_{n−1} f(x_{n−1}) \qquad (12)
\end{split}
\end{equation*}
\sphinxAtStartPar
em que \(n\) é o número de pontos. Os valores para os \(c\)’s e os \(x\)’s para até (e incluindo) a fórmula de quatro pontos estão resumidos na tabela abaixo.


\begin{savenotes}\sphinxattablestart
\centering
\begin{tabulary}{\linewidth}[t]{|T|T|T|}
\hline
\sphinxstyletheadfamily 
\sphinxAtStartPar
Pontos
&\sphinxstyletheadfamily 
\sphinxAtStartPar
Fatores de Peso
&\sphinxstyletheadfamily 
\sphinxAtStartPar
Argumentos da Função
\\
\hline
\sphinxAtStartPar
3
&
\sphinxAtStartPar
\(c_0\) = 0.5555556
&
\sphinxAtStartPar
\(x_0\) = \sphinxhyphen{} 0.774596669
\\
\hline
\sphinxAtStartPar
\(c_1\) = 0.8888889
&
\sphinxAtStartPar
\(x_1\) = 0.0
&
\sphinxAtStartPar

\\
\hline
\sphinxAtStartPar
\(c_2\) = 0.5555556
&
\sphinxAtStartPar
\(x_2\) = 0.774596669
&
\sphinxAtStartPar

\\
\hline
\sphinxAtStartPar
4
&
\sphinxAtStartPar
\(c_0\) = 0.3478548
&
\sphinxAtStartPar
\(x_0\) = \sphinxhyphen{} 0.861136312
\\
\hline
\sphinxAtStartPar
\(c_1\) = 0.6521452
&
\sphinxAtStartPar
\(x_1\) = \sphinxhyphen{} 0.339981044
&
\sphinxAtStartPar

\\
\hline
\sphinxAtStartPar
\(c_2\) = 0.6521452
&
\sphinxAtStartPar
\(x_2\) = 0.339981044
&
\sphinxAtStartPar

\\
\hline
\sphinxAtStartPar
\(c_3\) = 0.3478548
&
\sphinxAtStartPar
\(x_3\) = 0.861136312
&
\sphinxAtStartPar

\\
\hline
\end{tabulary}
\par
\sphinxattableend\end{savenotes}


\section{Implementação}
\label{\detokenize{aula-19-quadratura-gaussiana:implementacao}}
\sphinxAtStartPar
Vamos implementar abaixo o código para gerar a tabela de pontos (nós) e pesos para integração numérica consoante as fórmulas de quadratura de \sphinxstylestrong{Gauss\sphinxhyphen{}Legendre}.

\begin{sphinxVerbatim}[commandchars=\\\{\}]
\PYG{c+c1}{\PYGZsh{} número de pontos de quadratura}
\PYG{n}{n} \PYG{o}{=} \PYG{l+m+mi}{8}

\PYG{c+c1}{\PYGZsh{} pontos e pesos}
\PYG{p}{(}\PYG{n}{pontos}\PYG{p}{,}\PYG{n}{pesos}\PYG{p}{)} \PYG{o}{=} \PYG{n}{leg}\PYG{o}{.}\PYG{n}{leggauss}\PYG{p}{(}\PYG{n}{n}\PYG{p}{)}        
\end{sphinxVerbatim}

\sphinxAtStartPar
Isto é, para a regra de 2 pontos, os nós de Gauss\sphinxhyphen{}Legendre são

\begin{sphinxVerbatim}[commandchars=\\\{\}]
\PYG{n+nb}{print}\PYG{p}{(}\PYG{n}{pontos}\PYG{p}{)}
\end{sphinxVerbatim}

\begin{sphinxVerbatim}[commandchars=\\\{\}]
[\PYGZhy{}0.96028986 \PYGZhy{}0.79666648 \PYGZhy{}0.52553241 \PYGZhy{}0.18343464  0.18343464  0.52553241
  0.79666648  0.96028986]
\end{sphinxVerbatim}

\sphinxAtStartPar
e os pesos correspondentes são:

\begin{sphinxVerbatim}[commandchars=\\\{\}]
\PYG{n+nb}{print}\PYG{p}{(}\PYG{n}{pesos}\PYG{p}{)}
\end{sphinxVerbatim}

\begin{sphinxVerbatim}[commandchars=\\\{\}]
[0.10122854 0.22238103 0.31370665 0.36268378 0.36268378 0.31370665
 0.22238103 0.10122854]
\end{sphinxVerbatim}

\begin{sphinxVerbatim}[commandchars=\\\{\}]
\PYG{n}{plt}\PYG{o}{.}\PYG{n}{plot}\PYG{p}{(}\PYG{n}{pontos}\PYG{p}{,}\PYG{n}{pesos}\PYG{p}{)}
\PYG{n}{plt}\PYG{o}{.}\PYG{n}{autoscale}\PYG{p}{;}
\end{sphinxVerbatim}

\noindent\sphinxincludegraphics{{aula-19-quadratura-gaussiana_10_0}.png}


\subsection{Tabela de pesos/pontos \sphinxhyphen{} Quadratura de Gauss\sphinxhyphen{}Legendre}
\label{\detokenize{aula-19-quadratura-gaussiana:tabela-de-pesos-pontos-quadratura-de-gauss-legendre}}
\sphinxAtStartPar
Para gerarmos uma tabela de pontos e pesos, basta fazer:

\begin{sphinxVerbatim}[commandchars=\\\{\}]
\PYG{c+c1}{\PYGZsh{} número máximo de pontos}
\PYG{n}{N} \PYG{o}{=} \PYG{l+m+mi}{16}

\PYG{k}{for} \PYG{n}{i} \PYG{o+ow}{in} \PYG{n+nb}{range}\PYG{p}{(}\PYG{l+m+mi}{1}\PYG{p}{,}\PYG{n}{N}\PYG{o}{+}\PYG{l+m+mi}{1}\PYG{p}{)}\PYG{p}{:}
    \PYG{p}{(}\PYG{n}{pontos}\PYG{p}{,}\PYG{n}{pesos}\PYG{p}{)} \PYG{o}{=} \PYG{n}{leg}\PYG{o}{.}\PYG{n}{leggauss}\PYG{p}{(}\PYG{n}{i}\PYG{p}{)}       
    \PYG{n+nb}{print}\PYG{p}{(}\PYG{l+s+s1}{\PYGZsq{}}\PYG{l+s+s1}{REGRA DE }\PYG{l+s+si}{\PYGZob{}0\PYGZcb{}}\PYG{l+s+s1}{ PONTO(S):}\PYG{l+s+se}{\PYGZbs{}n}\PYG{l+s+s1}{\PYGZhy{}\PYGZgt{} Pontos:}\PYG{l+s+s1}{\PYGZsq{}}\PYG{o}{.}\PYG{n}{format}\PYG{p}{(}\PYG{n}{i}\PYG{p}{)}\PYG{p}{)}
    \PYG{n+nb}{print}\PYG{p}{(}\PYG{n}{pontos}\PYG{p}{)}
    \PYG{n+nb}{print}\PYG{p}{(}\PYG{l+s+s1}{\PYGZsq{}}\PYG{l+s+s1}{\PYGZhy{}\PYGZgt{} Pesos:}\PYG{l+s+s1}{\PYGZsq{}}\PYG{p}{)}
    \PYG{n+nb}{print}\PYG{p}{(}\PYG{n}{pesos}\PYG{p}{)}
    \PYG{n+nb}{print}\PYG{p}{(}\PYG{l+s+s1}{\PYGZsq{}}\PYG{l+s+se}{\PYGZbs{}n}\PYG{l+s+s1}{\PYGZsq{}}\PYG{p}{)}
\end{sphinxVerbatim}

\begin{sphinxVerbatim}[commandchars=\\\{\}]
REGRA DE 1 PONTO(S):
\PYGZhy{}\PYGZgt{} Pontos:
[0.]
\PYGZhy{}\PYGZgt{} Pesos:
[2.]


REGRA DE 2 PONTO(S):
\PYGZhy{}\PYGZgt{} Pontos:
[\PYGZhy{}0.57735027  0.57735027]
\PYGZhy{}\PYGZgt{} Pesos:
[1. 1.]


REGRA DE 3 PONTO(S):
\PYGZhy{}\PYGZgt{} Pontos:
[\PYGZhy{}0.77459667  0.          0.77459667]
\PYGZhy{}\PYGZgt{} Pesos:
[0.55555556 0.88888889 0.55555556]


REGRA DE 4 PONTO(S):
\PYGZhy{}\PYGZgt{} Pontos:
[\PYGZhy{}0.86113631 \PYGZhy{}0.33998104  0.33998104  0.86113631]
\PYGZhy{}\PYGZgt{} Pesos:
[0.34785485 0.65214515 0.65214515 0.34785485]


REGRA DE 5 PONTO(S):
\PYGZhy{}\PYGZgt{} Pontos:
[\PYGZhy{}0.90617985 \PYGZhy{}0.53846931  0.          0.53846931  0.90617985]
\PYGZhy{}\PYGZgt{} Pesos:
[0.23692689 0.47862867 0.56888889 0.47862867 0.23692689]


REGRA DE 6 PONTO(S):
\PYGZhy{}\PYGZgt{} Pontos:
[\PYGZhy{}0.93246951 \PYGZhy{}0.66120939 \PYGZhy{}0.23861919  0.23861919  0.66120939  0.93246951]
\PYGZhy{}\PYGZgt{} Pesos:
[0.17132449 0.36076157 0.46791393 0.46791393 0.36076157 0.17132449]


REGRA DE 7 PONTO(S):
\PYGZhy{}\PYGZgt{} Pontos:
[\PYGZhy{}0.94910791 \PYGZhy{}0.74153119 \PYGZhy{}0.40584515  0.          0.40584515  0.74153119
  0.94910791]
\PYGZhy{}\PYGZgt{} Pesos:
[0.12948497 0.27970539 0.38183005 0.41795918 0.38183005 0.27970539
 0.12948497]


REGRA DE 8 PONTO(S):
\PYGZhy{}\PYGZgt{} Pontos:
[\PYGZhy{}0.96028986 \PYGZhy{}0.79666648 \PYGZhy{}0.52553241 \PYGZhy{}0.18343464  0.18343464  0.52553241
  0.79666648  0.96028986]
\PYGZhy{}\PYGZgt{} Pesos:
[0.10122854 0.22238103 0.31370665 0.36268378 0.36268378 0.31370665
 0.22238103 0.10122854]


REGRA DE 9 PONTO(S):
\PYGZhy{}\PYGZgt{} Pontos:
[\PYGZhy{}0.96816024 \PYGZhy{}0.83603111 \PYGZhy{}0.61337143 \PYGZhy{}0.32425342  0.          0.32425342
  0.61337143  0.83603111  0.96816024]
\PYGZhy{}\PYGZgt{} Pesos:
[0.08127439 0.18064816 0.2606107  0.31234708 0.33023936 0.31234708
 0.2606107  0.18064816 0.08127439]


REGRA DE 10 PONTO(S):
\PYGZhy{}\PYGZgt{} Pontos:
[\PYGZhy{}0.97390653 \PYGZhy{}0.86506337 \PYGZhy{}0.67940957 \PYGZhy{}0.43339539 \PYGZhy{}0.14887434  0.14887434
  0.43339539  0.67940957  0.86506337  0.97390653]
\PYGZhy{}\PYGZgt{} Pesos:
[0.06667134 0.14945135 0.21908636 0.26926672 0.29552422 0.29552422
 0.26926672 0.21908636 0.14945135 0.06667134]


REGRA DE 11 PONTO(S):
\PYGZhy{}\PYGZgt{} Pontos:
[\PYGZhy{}0.97822866 \PYGZhy{}0.8870626  \PYGZhy{}0.73015201 \PYGZhy{}0.51909613 \PYGZhy{}0.26954316  0.
  0.26954316  0.51909613  0.73015201  0.8870626   0.97822866]
\PYGZhy{}\PYGZgt{} Pesos:
[0.05566857 0.12558037 0.18629021 0.23319376 0.26280454 0.27292509
 0.26280454 0.23319376 0.18629021 0.12558037 0.05566857]


REGRA DE 12 PONTO(S):
\PYGZhy{}\PYGZgt{} Pontos:
[\PYGZhy{}0.98156063 \PYGZhy{}0.90411726 \PYGZhy{}0.76990267 \PYGZhy{}0.58731795 \PYGZhy{}0.3678315  \PYGZhy{}0.12523341
  0.12523341  0.3678315   0.58731795  0.76990267  0.90411726  0.98156063]
\PYGZhy{}\PYGZgt{} Pesos:
[0.04717534 0.10693933 0.16007833 0.20316743 0.23349254 0.24914705
 0.24914705 0.23349254 0.20316743 0.16007833 0.10693933 0.04717534]


REGRA DE 13 PONTO(S):
\PYGZhy{}\PYGZgt{} Pontos:
[\PYGZhy{}0.98418305 \PYGZhy{}0.9175984  \PYGZhy{}0.80157809 \PYGZhy{}0.64234934 \PYGZhy{}0.44849275 \PYGZhy{}0.23045832
  0.          0.23045832  0.44849275  0.64234934  0.80157809  0.9175984
  0.98418305]
\PYGZhy{}\PYGZgt{} Pesos:
[0.040484   0.0921215  0.13887351 0.17814598 0.20781605 0.22628318
 0.23255155 0.22628318 0.20781605 0.17814598 0.13887351 0.0921215
 0.040484  ]


REGRA DE 14 PONTO(S):
\PYGZhy{}\PYGZgt{} Pontos:
[\PYGZhy{}0.98628381 \PYGZhy{}0.92843488 \PYGZhy{}0.82720132 \PYGZhy{}0.6872929  \PYGZhy{}0.51524864 \PYGZhy{}0.31911237
 \PYGZhy{}0.10805495  0.10805495  0.31911237  0.51524864  0.6872929   0.82720132
  0.92843488  0.98628381]
\PYGZhy{}\PYGZgt{} Pesos:
[0.03511946 0.08015809 0.12151857 0.15720317 0.1855384  0.20519846
 0.21526385 0.21526385 0.20519846 0.1855384  0.15720317 0.12151857
 0.08015809 0.03511946]


REGRA DE 15 PONTO(S):
\PYGZhy{}\PYGZgt{} Pontos:
[\PYGZhy{}0.98799252 \PYGZhy{}0.93727339 \PYGZhy{}0.84820658 \PYGZhy{}0.72441773 \PYGZhy{}0.57097217 \PYGZhy{}0.39415135
 \PYGZhy{}0.20119409  0.          0.20119409  0.39415135  0.57097217  0.72441773
  0.84820658  0.93727339  0.98799252]
\PYGZhy{}\PYGZgt{} Pesos:
[0.03075324 0.07036605 0.10715922 0.13957068 0.16626921 0.186161
 0.19843149 0.20257824 0.19843149 0.186161   0.16626921 0.13957068
 0.10715922 0.07036605 0.03075324]


REGRA DE 16 PONTO(S):
\PYGZhy{}\PYGZgt{} Pontos:
[\PYGZhy{}0.98940093 \PYGZhy{}0.94457502 \PYGZhy{}0.8656312  \PYGZhy{}0.75540441 \PYGZhy{}0.61787624 \PYGZhy{}0.45801678
 \PYGZhy{}0.28160355 \PYGZhy{}0.09501251  0.09501251  0.28160355  0.45801678  0.61787624
  0.75540441  0.8656312   0.94457502  0.98940093]
\PYGZhy{}\PYGZgt{} Pesos:
[0.02715246 0.06225352 0.09515851 0.12462897 0.14959599 0.16915652
 0.18260342 0.18945061 0.18945061 0.18260342 0.16915652 0.14959599
 0.12462897 0.09515851 0.06225352 0.02715246]
\end{sphinxVerbatim}

\sphinxAtStartPar
A partir daí, podemos organizar uma tabela para a regra de até 8 pontos/pesos como segue:

\begin{sphinxVerbatim}[commandchars=\\\{\}]
\PYG{c+c1}{\PYGZsh{} número máximo de pontos}
\PYG{n}{N} \PYG{o}{=} \PYG{l+m+mi}{8}

\PYG{n}{header}\PYG{o}{=}\PYG{l+s+s1}{\PYGZsq{}}\PYG{l+s+s1}{| Regra | nó(s) | peso(s) |}\PYG{l+s+se}{\PYGZbs{}n}\PYG{l+s+s1}{|\PYGZhy{}\PYGZhy{}\PYGZhy{}|\PYGZhy{}\PYGZhy{}\PYGZhy{}|\PYGZhy{}\PYGZhy{}\PYGZhy{}|}\PYG{l+s+s1}{\PYGZsq{}}
\PYG{n+nb}{print}\PYG{p}{(}\PYG{n}{header}\PYG{p}{)}
\PYG{k}{for} \PYG{n}{i} \PYG{o+ow}{in} \PYG{n+nb}{range}\PYG{p}{(}\PYG{l+m+mi}{1}\PYG{p}{,}\PYG{n}{N}\PYG{o}{+}\PYG{l+m+mi}{1}\PYG{p}{)}\PYG{p}{:}
    \PYG{p}{(}\PYG{n}{pontos}\PYG{p}{,}\PYG{n}{pesos}\PYG{p}{)} \PYG{o}{=} \PYG{n}{leg}\PYG{o}{.}\PYG{n}{leggauss}\PYG{p}{(}\PYG{n}{i}\PYG{p}{)}       
    \PYG{n}{p} \PYG{o}{=} \PYG{l+s+s1}{\PYGZsq{}}\PYG{l+s+s1}{, }\PYG{l+s+s1}{\PYGZsq{}}\PYG{o}{.}\PYG{n}{join}\PYG{p}{(}\PYG{p}{[}\PYG{n+nb}{str}\PYG{p}{(}\PYG{n}{p}\PYG{p}{)} \PYG{k}{for} \PYG{n}{p} \PYG{o+ow}{in} \PYG{n}{pontos}\PYG{p}{]}\PYG{p}{)}
    \PYG{n}{w} \PYG{o}{=} \PYG{l+s+s1}{\PYGZsq{}}\PYG{l+s+s1}{, }\PYG{l+s+s1}{\PYGZsq{}}\PYG{o}{.}\PYG{n}{join}\PYG{p}{(}\PYG{p}{[}\PYG{n+nb}{str}\PYG{p}{(}\PYG{n}{p}\PYG{p}{)} \PYG{k}{for} \PYG{n}{p} \PYG{o+ow}{in} \PYG{n}{pesos}\PYG{p}{]}\PYG{p}{)}    
    \PYG{n}{row} \PYG{o}{=} \PYG{l+s+s1}{\PYGZsq{}}\PYG{l+s+s1}{|}\PYG{l+s+s1}{\PYGZsq{}} \PYG{o}{+} \PYG{n+nb}{str}\PYG{p}{(}\PYG{n}{i}\PYG{p}{)} \PYG{o}{+} \PYG{l+s+s1}{\PYGZsq{}}\PYG{l+s+s1}{|}\PYG{l+s+s1}{\PYGZsq{}} \PYG{o}{+} \PYG{n}{p} \PYG{o}{+} \PYG{l+s+s1}{\PYGZsq{}}\PYG{l+s+s1}{|}\PYG{l+s+s1}{\PYGZsq{}} \PYG{o}{+} \PYG{n}{w} \PYG{o}{+} \PYG{l+s+s1}{\PYGZsq{}}\PYG{l+s+s1}{|}\PYG{l+s+s1}{\PYGZsq{}}
    \PYG{n+nb}{print}\PYG{p}{(}\PYG{n}{row}\PYG{p}{)}                   
\end{sphinxVerbatim}

\begin{sphinxVerbatim}[commandchars=\\\{\}]
| Regra | nó(s) | peso(s) |
|\PYGZhy{}\PYGZhy{}\PYGZhy{}|\PYGZhy{}\PYGZhy{}\PYGZhy{}|\PYGZhy{}\PYGZhy{}\PYGZhy{}|
|1|0.0|2.0|
|2|\PYGZhy{}0.5773502691896257, 0.5773502691896257|1.0, 1.0|
|3|\PYGZhy{}0.7745966692414834, 0.0, 0.7745966692414834|0.5555555555555557, 0.8888888888888888, 0.5555555555555557|
|4|\PYGZhy{}0.8611363115940526, \PYGZhy{}0.33998104358485626, 0.33998104358485626, 0.8611363115940526|0.3478548451374537, 0.6521451548625462, 0.6521451548625462, 0.3478548451374537|
|5|\PYGZhy{}0.906179845938664, \PYGZhy{}0.5384693101056831, 0.0, 0.5384693101056831, 0.906179845938664|0.23692688505618942, 0.4786286704993662, 0.568888888888889, 0.4786286704993662, 0.23692688505618942|
|6|\PYGZhy{}0.932469514203152, \PYGZhy{}0.6612093864662645, \PYGZhy{}0.23861918608319693, 0.23861918608319693, 0.6612093864662645, 0.932469514203152|0.17132449237916975, 0.36076157304813894, 0.46791393457269137, 0.46791393457269137, 0.36076157304813894, 0.17132449237916975|
|7|\PYGZhy{}0.9491079123427585, \PYGZhy{}0.7415311855993945, \PYGZhy{}0.4058451513773972, 0.0, 0.4058451513773972, 0.7415311855993945, 0.9491079123427585|0.12948496616887065, 0.2797053914892766, 0.3818300505051183, 0.41795918367346896, 0.3818300505051183, 0.2797053914892766, 0.12948496616887065|
|8|\PYGZhy{}0.9602898564975362, \PYGZhy{}0.7966664774136267, \PYGZhy{}0.525532409916329, \PYGZhy{}0.18343464249564978, 0.18343464249564978, 0.525532409916329, 0.7966664774136267, 0.9602898564975362|0.10122853629037669, 0.22238103445337434, 0.31370664587788705, 0.36268378337836177, 0.36268378337836177, 0.31370664587788705, 0.22238103445337434, 0.10122853629037669|
\end{sphinxVerbatim}


\subsubsection{Tabela de quadratura de Gauss\sphinxhyphen{}Legendre}
\label{\detokenize{aula-19-quadratura-gaussiana:tabela-de-quadratura-de-gauss-legendre}}

\begin{savenotes}\sphinxattablestart
\centering
\begin{tabulary}{\linewidth}[t]{|T|T|T|}
\hline
\sphinxstyletheadfamily 
\sphinxAtStartPar
Regra
&\sphinxstyletheadfamily 
\sphinxAtStartPar
nó(s)
&\sphinxstyletheadfamily 
\sphinxAtStartPar
peso(s)
\\
\hline
\sphinxAtStartPar
1
&
\sphinxAtStartPar
0.0
&
\sphinxAtStartPar
2.0
\\
\hline
\sphinxAtStartPar
2
&
\sphinxAtStartPar
\sphinxhyphen{}0.57735026919, 0.57735026919
&
\sphinxAtStartPar
1.0, 1.0
\\
\hline
\sphinxAtStartPar
3
&
\sphinxAtStartPar
\sphinxhyphen{}0.774596669241, 0.0, 0.774596669241
&
\sphinxAtStartPar
0.555555555556, 0.888888888889, 0.555555555556
\\
\hline
\sphinxAtStartPar
4
&
\sphinxAtStartPar
\sphinxhyphen{}0.861136311594, \sphinxhyphen{}0.339981043585, 0.339981043585, 0.861136311594
&
\sphinxAtStartPar
0.347854845137, 0.652145154863, 0.652145154863, 0.347854845137
\\
\hline
\sphinxAtStartPar
5
&
\sphinxAtStartPar
\sphinxhyphen{}0.906179845939, \sphinxhyphen{}0.538469310106, 0.0, 0.538469310106, 0.906179845939
&
\sphinxAtStartPar
0.236926885056, 0.478628670499, 0.568888888889, 0.478628670499, 0.236926885056
\\
\hline
\sphinxAtStartPar
6
&
\sphinxAtStartPar
\sphinxhyphen{}0.932469514203, \sphinxhyphen{}0.661209386466, \sphinxhyphen{}0.238619186083, 0.238619186083, 0.661209386466, 0.932469514203
&
\sphinxAtStartPar
0.171324492379, 0.360761573048, 0.467913934573, 0.467913934573, 0.360761573048, 0.171324492379
\\
\hline
\sphinxAtStartPar
7
&
\sphinxAtStartPar
\sphinxhyphen{}0.949107912343, \sphinxhyphen{}0.741531185599, \sphinxhyphen{}0.405845151377, 0.0, 0.405845151377, 0.741531185599, 0.949107912343
&
\sphinxAtStartPar
0.129484966169, 0.279705391489, 0.381830050505, 0.417959183673, 0.381830050505, 0.279705391489, 0.129484966169
\\
\hline
\sphinxAtStartPar
8
&
\sphinxAtStartPar
\sphinxhyphen{}0.960289856498, \sphinxhyphen{}0.796666477414, \sphinxhyphen{}0.525532409916, \sphinxhyphen{}0.183434642496, 0.183434642496, 0.525532409916, 0.796666477414, 0.960289856498
&
\sphinxAtStartPar
0.10122853629, 0.222381034453, 0.313706645878, 0.362683783378, 0.362683783378, 0.313706645878, 0.222381034453, 0.10122853629
\\
\hline
\end{tabulary}
\par
\sphinxattableend\end{savenotes}


\section{Transformação de variáveis}
\label{\detokenize{aula-19-quadratura-gaussiana:transformacao-de-variaveis}}
\sphinxAtStartPar
Uma integral \(\int_a^b f(x) \, dx\) sobre \([a,b]\) arbitrário ser transformada em uma integral em \([-1,1]\) utilizando a mudança de variáveis:
\begin{equation*}
\begin{split}t = \dfrac{2x - a - b}{b - a} \Rightarrow x = \dfrac{1}{2}[(b-a)t + a + b]\end{split}
\end{equation*}
\begin{sphinxVerbatim}[commandchars=\\\{\}]
\PYG{n}{x} \PYG{o}{=} \PYG{n}{np}\PYG{o}{.}\PYG{n}{linspace}\PYG{p}{(}\PYG{l+m+mi}{2}\PYG{p}{,}\PYG{l+m+mi}{4}\PYG{p}{)}
\PYG{n}{y} \PYG{o}{=} \PYG{n}{x} \PYG{o}{\PYGZhy{}} \PYG{l+m+mi}{3}
\PYG{n}{plt}\PYG{o}{.}\PYG{n}{plot}\PYG{p}{(}\PYG{n}{x}\PYG{p}{,}\PYG{n}{y}\PYG{p}{)}\PYG{p}{;}
\PYG{n}{plt}\PYG{o}{.}\PYG{n}{plot}\PYG{p}{(}\PYG{n}{x}\PYG{p}{,}\PYG{n}{x}\PYG{o}{*}\PYG{l+m+mi}{0}\PYG{p}{)}\PYG{p}{;}
\PYG{n}{plt}\PYG{o}{.}\PYG{n}{xticks}\PYG{p}{(}\PYG{p}{[}\PYG{l+m+mi}{2}\PYG{p}{,}\PYG{l+m+mi}{4}\PYG{p}{]}\PYG{p}{,}\PYG{p}{[}\PYG{l+s+s1}{\PYGZsq{}}\PYG{l+s+s1}{a}\PYG{l+s+s1}{\PYGZsq{}}\PYG{p}{,}\PYG{l+s+s1}{\PYGZsq{}}\PYG{l+s+s1}{b}\PYG{l+s+s1}{\PYGZsq{}}\PYG{p}{]}\PYG{p}{)}\PYG{p}{;}
\PYG{n}{plt}\PYG{o}{.}\PYG{n}{yticks}\PYG{p}{(}\PYG{p}{[}\PYG{o}{\PYGZhy{}}\PYG{l+m+mi}{1}\PYG{p}{,}\PYG{l+m+mi}{1}\PYG{p}{]}\PYG{p}{,}\PYG{p}{[}\PYG{l+s+s1}{\PYGZsq{}}\PYG{l+s+s1}{\PYGZhy{}1}\PYG{l+s+s1}{\PYGZsq{}}\PYG{p}{,}\PYG{l+s+s1}{\PYGZsq{}}\PYG{l+s+s1}{1}\PYG{l+s+s1}{\PYGZsq{}}\PYG{p}{]}\PYG{p}{)}\PYG{p}{;}
\PYG{n}{plt}\PYG{o}{.}\PYG{n}{annotate}\PYG{p}{(}\PYG{l+s+s1}{\PYGZsq{}}\PYG{l+s+s1}{\PYGZdl{}t = }\PYG{l+s+s1}{\PYGZbs{}}\PYG{l+s+s1}{dfrac}\PYG{l+s+s1}{\PYGZob{}}\PYG{l+s+s1}{2x \PYGZhy{} a \PYGZhy{} b\PYGZcb{}}\PYG{l+s+s1}{\PYGZob{}}\PYG{l+s+s1}{b \PYGZhy{} a\PYGZcb{}\PYGZdl{}}\PYG{l+s+s1}{\PYGZsq{}}\PYG{p}{,}\PYG{p}{(}\PYG{l+m+mf}{2.8}\PYG{p}{,}\PYG{l+m+mf}{0.5}\PYG{p}{)}\PYG{p}{)}\PYG{p}{;}
\end{sphinxVerbatim}

\noindent\sphinxincludegraphics{{aula-19-quadratura-gaussiana_18_0}.png}


\subsection{Tarefa}
\label{\detokenize{aula-19-quadratura-gaussiana:tarefa}}
\sphinxAtStartPar
Defina uma função como a seguinte que retorne \sphinxcode{\sphinxupquote{output}}, tal que \sphinxcode{\sphinxupquote{type(output)}} seja \sphinxcode{\sphinxupquote{str}}.

\begin{sphinxVerbatim}[commandchars=\\\{\}]
\PYG{k}{def} \PYG{n+nf}{print\PYGZus{}gauss\PYGZus{}legendre\PYGZus{}table}\PYG{p}{(}\PYG{n}{N}\PYG{p}{)}\PYG{p}{:}
    \PYG{n}{header}\PYG{o}{=}\PYG{l+s+s1}{\PYGZsq{}}\PYG{l+s+s1}{| Regra | nó(s) | peso(s) |}\PYG{l+s+se}{\PYGZbs{}n}\PYG{l+s+s1}{|\PYGZhy{}\PYGZhy{}\PYGZhy{}|\PYGZhy{}\PYGZhy{}\PYGZhy{}|\PYGZhy{}\PYGZhy{}\PYGZhy{}|}\PYG{l+s+s1}{\PYGZsq{}}
    \PYG{n+nb}{print}\PYG{p}{(}\PYG{n}{header}\PYG{p}{)}
    \PYG{k}{for} \PYG{n}{i} \PYG{o+ow}{in} \PYG{n+nb}{range}\PYG{p}{(}\PYG{l+m+mi}{1}\PYG{p}{,}\PYG{n}{N}\PYG{o}{+}\PYG{l+m+mi}{1}\PYG{p}{)}\PYG{p}{:}
        \PYG{p}{(}\PYG{n}{pontos}\PYG{p}{,}\PYG{n}{pesos}\PYG{p}{)} \PYG{o}{=} \PYG{n}{leg}\PYG{o}{.}\PYG{n}{leggauss}\PYG{p}{(}\PYG{n}{i}\PYG{p}{)}       
        \PYG{n}{p} \PYG{o}{=} \PYG{l+s+s1}{\PYGZsq{}}\PYG{l+s+s1}{, }\PYG{l+s+s1}{\PYGZsq{}}\PYG{o}{.}\PYG{n}{join}\PYG{p}{(}\PYG{p}{[}\PYG{n+nb}{str}\PYG{p}{(}\PYG{n}{p}\PYG{p}{)} \PYG{k}{for} \PYG{n}{p} \PYG{o+ow}{in} \PYG{n}{pontos}\PYG{p}{]}\PYG{p}{)}
        \PYG{n}{w} \PYG{o}{=} \PYG{l+s+s1}{\PYGZsq{}}\PYG{l+s+s1}{, }\PYG{l+s+s1}{\PYGZsq{}}\PYG{o}{.}\PYG{n}{join}\PYG{p}{(}\PYG{p}{[}\PYG{n+nb}{str}\PYG{p}{(}\PYG{n}{p}\PYG{p}{)} \PYG{k}{for} \PYG{n}{p} \PYG{o+ow}{in} \PYG{n}{pesos}\PYG{p}{]}\PYG{p}{)}    
        \PYG{n}{row} \PYG{o}{=} \PYG{l+s+s1}{\PYGZsq{}}\PYG{l+s+s1}{|}\PYG{l+s+s1}{\PYGZsq{}} \PYG{o}{+} \PYG{n+nb}{str}\PYG{p}{(}\PYG{n}{i}\PYG{p}{)} \PYG{o}{+} \PYG{l+s+s1}{\PYGZsq{}}\PYG{l+s+s1}{|}\PYG{l+s+s1}{\PYGZsq{}} \PYG{o}{+} \PYG{n}{p} \PYG{o}{+} \PYG{l+s+s1}{\PYGZsq{}}\PYG{l+s+s1}{|}\PYG{l+s+s1}{\PYGZsq{}} \PYG{o}{+} \PYG{n}{w} \PYG{o}{+} \PYG{l+s+s1}{\PYGZsq{}}\PYG{l+s+s1}{|}\PYG{l+s+s1}{\PYGZsq{}}
        \PYG{n+nb}{print}\PYG{p}{(}\PYG{n}{row}\PYG{p}{)}             
\end{sphinxVerbatim}

\sphinxAtStartPar
Então, reimprima a tabela para 8 pesos/pontos anterior com o código.

\begin{sphinxVerbatim}[commandchars=\\\{\}]
\PYG{n}{output} \PYG{o}{=} \PYG{n}{print\PYGZus{}gauss\PYGZus{}legendre\PYGZus{}table}\PYG{p}{(}\PYG{l+m+mi}{8}\PYG{p}{)}
\end{sphinxVerbatim}

\sphinxAtStartPar
Em seguida, use o código abaixo para converter a saída da célula de código do Jupyter diretamente para Markdown.

\begin{sphinxVerbatim}[commandchars=\\\{\}]
\PYG{k+kn}{from} \PYG{n+nn}{IPython}\PYG{n+nn}{.}\PYG{n+nn}{display} \PYG{k+kn}{import} \PYG{n}{display}\PYG{p}{,} \PYG{n}{Markdown}
\PYG{n}{display}\PYG{p}{(}\PYG{n}{Markdown}\PYG{p}{(}\PYG{n}{output}\PYG{p}{)}\PYG{p}{)}
\end{sphinxVerbatim}

\sphinxAtStartPar
Por último, incorpore esta funcionalidade em \sphinxcode{\sphinxupquote{print\_gauss\_legendre\_table(N)}}, para \sphinxcode{\sphinxupquote{N}} dado.


\begin{savenotes}\sphinxattablestart
\centering
\begin{tabulary}{\linewidth}[t]{|T|T|T|}
\hline
\sphinxstyletheadfamily 
\sphinxAtStartPar
Regra
&\sphinxstyletheadfamily 
\sphinxAtStartPar
nó(s)
&\sphinxstyletheadfamily 
\sphinxAtStartPar
peso(s)
\\
\hline
\sphinxAtStartPar
1
&
\sphinxAtStartPar
0.0
&
\sphinxAtStartPar
2.0
\\
\hline
\sphinxAtStartPar
2
&
\sphinxAtStartPar
\sphinxhyphen{}0.57735026919, 0.57735026919
&
\sphinxAtStartPar
1.0, 1.0
\\
\hline
\sphinxAtStartPar
3
&
\sphinxAtStartPar
\sphinxhyphen{}0.774596669241, 0.0, 0.774596669241
&
\sphinxAtStartPar
0.555555555556, 0.888888888889, 0.555555555556
\\
\hline
\sphinxAtStartPar
4
&
\sphinxAtStartPar
\sphinxhyphen{}0.861136311594, \sphinxhyphen{}0.339981043585, 0.339981043585, 0.861136311594
&
\sphinxAtStartPar
0.347854845137, 0.652145154863, 0.652145154863, 0.347854845137
\\
\hline
\sphinxAtStartPar
5
&
\sphinxAtStartPar
\sphinxhyphen{}0.906179845939, \sphinxhyphen{}0.538469310106, 0.0, 0.538469310106, 0.906179845939
&
\sphinxAtStartPar
0.236926885056, 0.478628670499, 0.568888888889, 0.478628670499, 0.236926885056
\\
\hline
\sphinxAtStartPar
6
&
\sphinxAtStartPar
\sphinxhyphen{}0.932469514203, \sphinxhyphen{}0.661209386466, \sphinxhyphen{}0.238619186083, 0.238619186083, 0.661209386466, 0.932469514203
&
\sphinxAtStartPar
0.171324492379, 0.360761573048, 0.467913934573, 0.467913934573, 0.360761573048, 0.171324492379
\\
\hline
\sphinxAtStartPar
7
&
\sphinxAtStartPar
\sphinxhyphen{}0.949107912343, \sphinxhyphen{}0.741531185599, \sphinxhyphen{}0.405845151377, 0.0, 0.405845151377, 0.741531185599, 0.949107912343
&
\sphinxAtStartPar
0.129484966169, 0.279705391489, 0.381830050505, 0.417959183673, 0.381830050505, 0.279705391489, 0.129484966169
\\
\hline
\sphinxAtStartPar
8
&
\sphinxAtStartPar
\sphinxhyphen{}0.960289856498, \sphinxhyphen{}0.796666477414, \sphinxhyphen{}0.525532409916, \sphinxhyphen{}0.183434642496, 0.183434642496, 0.525532409916, 0.796666477414, 0.960289856498
&
\sphinxAtStartPar
0.10122853629, 0.222381034453, 0.313706645878, 0.362683783378, 0.362683783378, 0.313706645878, 0.222381034453, 0.10122853629
\\
\hline
\sphinxAtStartPar
9
&
\sphinxAtStartPar
\sphinxhyphen{}0.968160239508, \sphinxhyphen{}0.836031107327, \sphinxhyphen{}0.613371432701, \sphinxhyphen{}0.324253423404, 0.0, 0.324253423404, 0.613371432701, 0.836031107327, 0.968160239508
&
\sphinxAtStartPar
0.0812743883616, 0.180648160695, 0.260610696403, 0.31234707704, 0.330239355001, 0.31234707704, 0.260610696403, 0.180648160695, 0.0812743883616
\\
\hline
\sphinxAtStartPar
10
&
\sphinxAtStartPar
\sphinxhyphen{}0.973906528517, \sphinxhyphen{}0.865063366689, \sphinxhyphen{}0.679409568299, \sphinxhyphen{}0.433395394129, \sphinxhyphen{}0.148874338982, 0.148874338982, 0.433395394129, 0.679409568299, 0.865063366689, 0.973906528517
&
\sphinxAtStartPar
0.0666713443087, 0.149451349151, 0.219086362516, 0.26926671931, 0.295524224715, 0.295524224715, 0.26926671931, 0.219086362516, 0.149451349151, 0.0666713443087
\\
\hline
\sphinxAtStartPar
11
&
\sphinxAtStartPar
\sphinxhyphen{}0.978228658146, \sphinxhyphen{}0.887062599768, \sphinxhyphen{}0.730152005574, \sphinxhyphen{}0.519096129207, \sphinxhyphen{}0.269543155952, 0.0, 0.269543155952, 0.519096129207, 0.730152005574, 0.887062599768, 0.978228658146
&
\sphinxAtStartPar
0.0556685671162, 0.125580369465, 0.186290210928, 0.233193764592, 0.26280454451, 0.272925086778, 0.26280454451, 0.233193764592, 0.186290210928, 0.125580369465, 0.0556685671162
\\
\hline
\sphinxAtStartPar
12
&
\sphinxAtStartPar
\sphinxhyphen{}0.981560634247, \sphinxhyphen{}0.90411725637, \sphinxhyphen{}0.769902674194, \sphinxhyphen{}0.587317954287, \sphinxhyphen{}0.367831498998, \sphinxhyphen{}0.125233408511, 0.125233408511, 0.367831498998, 0.587317954287, 0.769902674194, 0.90411725637, 0.981560634247
&
\sphinxAtStartPar
0.0471753363865, 0.106939325995, 0.160078328543, 0.203167426723, 0.233492536538, 0.249147045813, 0.249147045813, 0.233492536538, 0.203167426723, 0.160078328543, 0.106939325995, 0.0471753363865
\\
\hline
\end{tabulary}
\par
\sphinxattableend\end{savenotes}


\chapter{Diferenciação Numérica}
\label{\detokenize{aula-20-diferenciacao-numerica:diferenciacao-numerica}}\label{\detokenize{aula-20-diferenciacao-numerica::doc}}

\section{Fórmula de Derivação de Alta Acurácia}
\label{\detokenize{aula-20-diferenciacao-numerica:formula-de-derivacao-de-alta-acuracia}}
\sphinxAtStartPar
A expansão em série de Taylor progressiva pode ser escrita como
\begin{equation*}
\begin{split}
f(x_{i+1}) = f(x_i) + f'(x_i) h + \dfrac{f '' (x_i)}{2} h^2 + \dots
\end{split}
\end{equation*}
\sphinxAtStartPar
o que pode ser resolvido por
\begin{equation*}
\begin{split}
f'(x_i) = \dfrac{f(x_{i+1}) - f(x_i)}{h} - \dfrac{f''(x_i)}{2} h + O(h^2) \qquad (1)
\end{split}
\end{equation*}
\sphinxAtStartPar
Truncamos esse resultado excluindo os termos da segunda derivada e de derivadas de ordem superior e ficamos com o resultado final
\begin{equation*}
\begin{split}
f' (x_i) = \dfrac{f(x_{i+1}) - f(x_i)}{h} + O(h) \qquad (2)
\end{split}
\end{equation*}
\sphinxAtStartPar
Em contraste a essa abordagem, agora manteremos o termo da segunda derivada substituindo a seguinte aproximação da segunda derivada
\begin{equation*}
\begin{split}
f'' (x_i) = \dfrac{f(x_{i+2}) - 2 f(x_{i+1}) + f(x_i)}{h^2} + O(h) \qquad (3)
\end{split}
\end{equation*}
\sphinxAtStartPar
na Equação (1) para obter
\begin{equation*}
\begin{split}
f' (x_i) = \dfrac{f(x_{i+1}) - f(x_i)}{h} - \dfrac{f(x_{i+2}) - 2 f(x_{i+1}) + f(x_i)}{2 h^2} + O(h^2)
\end{split}
\end{equation*}
\sphinxAtStartPar
ou, agrupando os termos,
\begin{equation*}
\begin{split}
f' (x_i) = \dfrac{- f(x_{i+2}) + 4 f(x_{i+1}) - 3 f(x_i)}{2 h} + O(h^2) \qquad (4)
\end{split}
\end{equation*}
\sphinxAtStartPar
Observe que a inclusão do termo da segunda derivada melhorou a precisão para \(O(h^2)\).


\section{Derivada de Dados Não Uniformemente Espaçados}
\label{\detokenize{aula-20-diferenciacao-numerica:derivada-de-dados-nao-uniformemente-espacados}}
\sphinxAtStartPar
Uma forma de tratar dados não uniformemente espaçados é ajustar um polinômio interpolador de Lagrange aos dados. Para uma derivada de primeira ordem mais acurada, por exemplo, podemos deduzir um polinômio de segundo grau para cada conjunto de três pontos adjacentes analiticamente através das expressões das funções de base de Lagrange. Em seguida, derivamos as funções de base de modo a obter
\begin{equation*}
\begin{split}f'(x) = f(x_{i-1}) \dfrac{2 x - x_i - x_{i+1}}{(x_{i-1} - x_i)(x_{i-1} - x_{i+1})} + f(x_i) \dfrac{2 x - x_{i-1} - x_{i+1}}{(x_i - x_{i-1})(x_i - x_{i+1})} + f(x_{i+1}) \dfrac{2 x - x_{i-1} - x_i}{(x_{i+1} - x_{i-1})(x_{i+1} - x_i)}, \qquad (5)\end{split}
\end{equation*}
\sphinxAtStartPar
em que \(x\) é o valor no qual se quer estimar a derivada.

\begin{sphinxVerbatim}[commandchars=\\\{\}]
\PYG{k+kn}{import} \PYG{n+nn}{numpy} \PYG{k}{as} \PYG{n+nn}{np} 
\PYG{k+kn}{import} \PYG{n+nn}{sympy} \PYG{k}{as} \PYG{n+nn}{sy}
\PYG{k+kn}{import} \PYG{n+nn}{matplotlib}\PYG{n+nn}{.}\PYG{n+nn}{pyplot} \PYG{k}{as} \PYG{n+nn}{plt}
\PYG{o}{\PYGZpc{}}\PYG{k}{matplotlib} inline
\end{sphinxVerbatim}


\section{Motivação}
\label{\detokenize{aula-20-diferenciacao-numerica:motivacao}}
\sphinxAtStartPar
Gerando curvas de deslocamento, velocidade e aceleração para uma partícula.

\begin{sphinxVerbatim}[commandchars=\\\{\}]
\PYG{c+c1}{\PYGZsh{} eixo temporal}
\PYG{n}{x} \PYG{o}{=} \PYG{n}{np}\PYG{o}{.}\PYG{n}{linspace}\PYG{p}{(}\PYG{l+m+mi}{0}\PYG{p}{,}\PYG{l+m+mi}{1}\PYG{p}{,}\PYG{l+m+mi}{30}\PYG{p}{,}\PYG{n}{endpoint}\PYG{o}{=}\PYG{k+kc}{True}\PYG{p}{)}

\PYG{c+c1}{\PYGZsh{} lei de movimento}
\PYG{n}{y} \PYG{o}{=} \PYG{l+m+mf}{0.93}\PYG{o}{*}\PYG{n}{np}\PYG{o}{.}\PYG{n}{cos}\PYG{p}{(}\PYG{l+m+mf}{2.5}\PYG{o}{*}\PYG{p}{(}\PYG{n}{x}\PYG{o}{\PYGZhy{}}\PYG{l+m+mi}{1}\PYG{p}{)}\PYG{p}{)}\PYG{o}{*}\PYG{n}{np}\PYG{o}{.}\PYG{n}{sin}\PYG{p}{(}\PYG{l+m+mi}{6}\PYG{o}{*}\PYG{n}{x}\PYG{o}{*}\PYG{n}{x}\PYG{p}{)}

\PYG{c+c1}{\PYGZsh{} velocidade e aceleração (funções analíticas)}
\PYG{c+c1}{\PYGZsh{} derivação simbólica}
\PYG{n}{t} \PYG{o}{=} \PYG{n}{sy}\PYG{o}{.}\PYG{n}{Symbol}\PYG{p}{(}\PYG{l+s+s1}{\PYGZsq{}}\PYG{l+s+s1}{t}\PYG{l+s+s1}{\PYGZsq{}}\PYG{p}{)}
\PYG{n}{st} \PYG{o}{=} \PYG{l+m+mf}{0.93}\PYG{o}{*}\PYG{n}{sy}\PYG{o}{.}\PYG{n}{cos}\PYG{p}{(}\PYG{l+m+mf}{2.5}\PYG{o}{*}\PYG{p}{(}\PYG{n}{t}\PYG{o}{\PYGZhy{}}\PYG{l+m+mi}{1}\PYG{p}{)}\PYG{p}{)}\PYG{o}{*}\PYG{n}{sy}\PYG{o}{.}\PYG{n}{sin}\PYG{p}{(}\PYG{l+m+mi}{6}\PYG{o}{*}\PYG{n}{t}\PYG{o}{*}\PYG{n}{t}\PYG{p}{)}
\PYG{n}{dt} \PYG{o}{=} \PYG{n}{sy}\PYG{o}{.}\PYG{n}{diff}\PYG{p}{(}\PYG{n}{st}\PYG{p}{,}\PYG{n}{t}\PYG{p}{)} \PYG{c+c1}{\PYGZsh{} v(t)}
\PYG{n}{dt2} \PYG{o}{=} \PYG{n}{sy}\PYG{o}{.}\PYG{n}{diff}\PYG{p}{(}\PYG{n}{dt}\PYG{p}{,}\PYG{n}{t}\PYG{p}{)} \PYG{c+c1}{\PYGZsh{} a(t)}

\PYG{c+c1}{\PYGZsh{} converte para numérico}
\PYG{n}{dy} \PYG{o}{=} \PYG{n}{np}\PYG{o}{.}\PYG{n}{asarray}\PYG{p}{(}\PYG{p}{[}\PYG{n}{dt}\PYG{o}{.}\PYG{n}{subs}\PYG{p}{(}\PYG{n}{t}\PYG{p}{,}\PYG{n}{xn}\PYG{p}{)} \PYG{k}{for} \PYG{n}{xn} \PYG{o+ow}{in} \PYG{n}{x}\PYG{p}{]}\PYG{p}{)}
\PYG{n}{dy2} \PYG{o}{=} \PYG{n}{np}\PYG{o}{.}\PYG{n}{asarray}\PYG{p}{(}\PYG{p}{[}\PYG{n}{dt2}\PYG{o}{.}\PYG{n}{subs}\PYG{p}{(}\PYG{n}{t}\PYG{p}{,}\PYG{n}{xn}\PYG{p}{)} \PYG{k}{for} \PYG{n}{xn} \PYG{o+ow}{in} \PYG{n}{x}\PYG{p}{]}\PYG{p}{)}

\PYG{c+c1}{\PYGZsh{} função para plotagem}
\PYG{k}{def} \PYG{n+nf}{plotting}\PYG{p}{(}\PYG{n}{x}\PYG{p}{,}\PYG{n}{y}\PYG{p}{,}\PYG{n}{c}\PYG{p}{,}\PYG{n}{lab}\PYG{p}{)}\PYG{p}{:}
    \PYG{n}{plt}\PYG{o}{.}\PYG{n}{plot}\PYG{p}{(}\PYG{n}{x}\PYG{p}{,}\PYG{n}{y}\PYG{p}{,}\PYG{n}{c}\PYG{p}{,}\PYG{n}{label}\PYG{o}{=}\PYG{n}{lab}\PYG{p}{)}
    \PYG{n}{plt}\PYG{o}{.}\PYG{n}{grid}\PYG{p}{(}\PYG{p}{)}
    \PYG{n}{plt}\PYG{o}{.}\PYG{n}{xlabel}\PYG{p}{(}\PYG{l+s+s1}{\PYGZsq{}}\PYG{l+s+s1}{\PYGZdl{}t\PYGZdl{}}\PYG{l+s+s1}{\PYGZsq{}}\PYG{p}{)}
    \PYG{n}{plt}\PYG{o}{.}\PYG{n}{tick\PYGZus{}params}\PYG{p}{(}
        \PYG{n}{axis}\PYG{o}{=}\PYG{l+s+s1}{\PYGZsq{}}\PYG{l+s+s1}{both}\PYG{l+s+s1}{\PYGZsq{}}\PYG{p}{,}        
        \PYG{n}{which}\PYG{o}{=}\PYG{l+s+s1}{\PYGZsq{}}\PYG{l+s+s1}{both}\PYG{l+s+s1}{\PYGZsq{}}\PYG{p}{,}      
        \PYG{n}{bottom}\PYG{o}{=}\PYG{k+kc}{False}\PYG{p}{,}      
        \PYG{n}{left}\PYG{o}{=}\PYG{k+kc}{False}\PYG{p}{,}         
        \PYG{n}{labelbottom}\PYG{o}{=}\PYG{k+kc}{False}\PYG{p}{,} 
        \PYG{n}{labelleft}\PYG{o}{=}\PYG{k+kc}{False}\PYG{p}{)}
    \PYG{n}{plt}\PYG{o}{.}\PYG{n}{legend}\PYG{p}{(}\PYG{p}{)}
    \PYG{n}{ax} \PYG{o}{=} \PYG{n}{plt}\PYG{o}{.}\PYG{n}{gca}\PYG{p}{(}\PYG{p}{)}
    \PYG{n}{ax}\PYG{o}{.}\PYG{n}{set\PYGZus{}facecolor}\PYG{p}{(}\PYG{l+s+s1}{\PYGZsq{}}\PYG{l+s+s1}{\PYGZsh{}F9F9F9}\PYG{l+s+s1}{\PYGZsq{}}\PYG{p}{)}
\end{sphinxVerbatim}

\begin{sphinxVerbatim}[commandchars=\\\{\}]
\PYG{c+c1}{\PYGZsh{} curva: deslocamento}
\PYG{n}{plotting}\PYG{p}{(}\PYG{n}{x}\PYG{p}{,}\PYG{n}{y}\PYG{p}{,}\PYG{l+s+s1}{\PYGZsq{}}\PYG{l+s+s1}{or}\PYG{l+s+s1}{\PYGZsq{}}\PYG{p}{,}\PYG{l+s+s1}{\PYGZsq{}}\PYG{l+s+s1}{\PYGZdl{}x(t)\PYGZdl{}}\PYG{l+s+s1}{\PYGZsq{}}\PYG{p}{)}
\end{sphinxVerbatim}

\noindent\sphinxincludegraphics{{aula-20-diferenciacao-numerica_4_0}.png}

\begin{sphinxVerbatim}[commandchars=\\\{\}]
\PYG{c+c1}{\PYGZsh{} curva: velocidade}
\PYG{n}{plotting}\PYG{p}{(}\PYG{n}{x}\PYG{p}{,}\PYG{n}{dy}\PYG{p}{,}\PYG{l+s+s1}{\PYGZsq{}}\PYG{l+s+s1}{og}\PYG{l+s+s1}{\PYGZsq{}}\PYG{p}{,}\PYG{l+s+s1}{\PYGZsq{}}\PYG{l+s+s1}{\PYGZdl{}v(t)\PYGZdl{}}\PYG{l+s+s1}{\PYGZsq{}}\PYG{p}{)}
\end{sphinxVerbatim}

\noindent\sphinxincludegraphics{{aula-20-diferenciacao-numerica_5_0}.png}

\begin{sphinxVerbatim}[commandchars=\\\{\}]
\PYG{c+c1}{\PYGZsh{} curva: aceleração}
\PYG{n}{plotting}\PYG{p}{(}\PYG{n}{x}\PYG{p}{,}\PYG{n}{dy2}\PYG{p}{,}\PYG{l+s+s1}{\PYGZsq{}}\PYG{l+s+s1}{ob}\PYG{l+s+s1}{\PYGZsq{}}\PYG{p}{,}\PYG{l+s+s1}{\PYGZsq{}}\PYG{l+s+s1}{\PYGZdl{}a(t)\PYGZdl{}}\PYG{l+s+s1}{\PYGZsq{}}\PYG{p}{)}
\end{sphinxVerbatim}

\noindent\sphinxincludegraphics{{aula-20-diferenciacao-numerica_6_0}.png}


\section{Derivada numérica progressiva e regressiva (primeira ordem)}
\label{\detokenize{aula-20-diferenciacao-numerica:derivada-numerica-progressiva-e-regressiva-primeira-ordem}}
\begin{sphinxVerbatim}[commandchars=\\\{\}]
\PYG{c+c1}{\PYGZsh{} Exemplo: f(x) = sen(x); }
\PYG{c+c1}{\PYGZsh{} Derivada analítica (verdadeira): f\PYGZsq{}(x=x0) = cos(x0)}
\PYG{n}{x0} \PYG{o}{=} \PYG{l+m+mf}{1.2}
\PYG{n}{h} \PYG{o}{=} \PYG{l+m+mf}{.01}
\PYG{n}{n} \PYG{o}{=} \PYG{l+m+mi}{10}
\PYG{n}{x} \PYG{o}{=} \PYG{n}{np}\PYG{o}{.}\PYG{n}{linspace}\PYG{p}{(}\PYG{n}{x0}\PYG{o}{\PYGZhy{}}\PYG{l+m+mi}{2}\PYG{o}{*}\PYG{n}{h}\PYG{p}{,}\PYG{n}{x0}\PYG{o}{+}\PYG{l+m+mi}{2}\PYG{o}{*}\PYG{n}{h}\PYG{p}{,}\PYG{n}{n}\PYG{p}{)}

\PYG{c+c1}{\PYGZsh{} expressões analíticas de f(x) e f\PYGZsq{}(x)}
\PYG{n}{f} \PYG{o}{=} \PYG{n}{np}\PYG{o}{.}\PYG{n}{sin}\PYG{p}{(}\PYG{n}{x}\PYG{p}{)}
\PYG{n}{df} \PYG{o}{=} \PYG{n}{np}\PYG{o}{.}\PYG{n}{cos}\PYG{p}{(}\PYG{n}{x}\PYG{p}{)}

\PYG{c+c1}{\PYGZsh{} valor \PYGZdq{}exato\PYGZdq{} de f\PYGZsq{}(x) no ponto x=x0}
\PYG{n}{dfe} \PYG{o}{=} \PYG{n}{np}\PYG{o}{.}\PYG{n}{cos}\PYG{p}{(}\PYG{n}{x0}\PYG{p}{)}

\PYG{n}{dfp} \PYG{o}{=} \PYG{p}{(}\PYG{n}{np}\PYG{o}{.}\PYG{n}{sin}\PYG{p}{(}\PYG{n}{x0}\PYG{o}{+}\PYG{n}{h}\PYG{p}{)} \PYG{o}{\PYGZhy{}} \PYG{n}{np}\PYG{o}{.}\PYG{n}{sin}\PYG{p}{(}\PYG{n}{x0}\PYG{p}{)}\PYG{p}{)}\PYG{o}{/}\PYG{n}{h}       \PYG{c+c1}{\PYGZsh{} DF progressiva}
\PYG{n}{dfr} \PYG{o}{=} \PYG{p}{(}\PYG{n}{np}\PYG{o}{.}\PYG{n}{sin}\PYG{p}{(}\PYG{n}{x0}\PYG{p}{)} \PYG{o}{\PYGZhy{}} \PYG{n}{np}\PYG{o}{.}\PYG{n}{sin}\PYG{p}{(}\PYG{n}{x0}\PYG{o}{\PYGZhy{}}\PYG{n}{h}\PYG{p}{)}\PYG{p}{)}\PYG{o}{/}\PYG{n}{h}       \PYG{c+c1}{\PYGZsh{} DF regressiva}
\PYG{n}{dfc} \PYG{o}{=} \PYG{p}{(}\PYG{n}{np}\PYG{o}{.}\PYG{n}{sin}\PYG{p}{(}\PYG{n}{x0}\PYG{o}{+}\PYG{n}{h}\PYG{p}{)} \PYG{o}{\PYGZhy{}} \PYG{n}{np}\PYG{o}{.}\PYG{n}{sin}\PYG{p}{(}\PYG{n}{x0}\PYG{o}{\PYGZhy{}}\PYG{n}{h}\PYG{p}{)}\PYG{p}{)}\PYG{o}{/}\PYG{p}{(}\PYG{l+m+mi}{2}\PYG{o}{*}\PYG{n}{h}\PYG{p}{)} \PYG{c+c1}{\PYGZsh{} DF centrada}

\PYG{n}{plt}\PYG{o}{.}\PYG{n}{plot}\PYG{p}{(}\PYG{n}{x}\PYG{p}{,}\PYG{n}{df}\PYG{p}{,}\PYG{l+s+s1}{\PYGZsq{}}\PYG{l+s+s1}{ok}\PYG{l+s+s1}{\PYGZsq{}}\PYG{p}{)} \PYG{c+c1}{\PYGZsh{} derivada exata}
\PYG{n}{plt}\PYG{o}{.}\PYG{n}{plot}\PYG{p}{(}\PYG{n}{x0}\PYG{p}{,}\PYG{n}{dfe}\PYG{p}{,}\PYG{l+s+s1}{\PYGZsq{}}\PYG{l+s+s1}{sy}\PYG{l+s+s1}{\PYGZsq{}}\PYG{p}{)} \PYG{c+c1}{\PYGZsh{} derivada exata no ponto x0}
\PYG{n}{plt}\PYG{o}{.}\PYG{n}{plot}\PYG{p}{(}\PYG{n}{x0}\PYG{p}{,}\PYG{n}{dfp}\PYG{p}{,}\PYG{l+s+s1}{\PYGZsq{}}\PYG{l+s+s1}{or}\PYG{l+s+s1}{\PYGZsq{}}\PYG{p}{)} \PYG{c+c1}{\PYGZsh{} derivada numérica PROGRESSIVA no ponto x0 }
\PYG{n}{plt}\PYG{o}{.}\PYG{n}{plot}\PYG{p}{(}\PYG{n}{x0}\PYG{p}{,}\PYG{n}{dfr}\PYG{p}{,}\PYG{l+s+s1}{\PYGZsq{}}\PYG{l+s+s1}{ob}\PYG{l+s+s1}{\PYGZsq{}}\PYG{p}{)} \PYG{c+c1}{\PYGZsh{} derivada numérica REGRESSIVA no ponto x0 }
\PYG{n}{plt}\PYG{o}{.}\PYG{n}{plot}\PYG{p}{(}\PYG{n}{x0}\PYG{p}{,}\PYG{n}{dfc}\PYG{p}{,}\PYG{l+s+s1}{\PYGZsq{}}\PYG{l+s+s1}{og}\PYG{l+s+s1}{\PYGZsq{}}\PYG{p}{)} \PYG{c+c1}{\PYGZsh{} derivada numérica CENTRADA no ponto x0 }
\PYG{n}{plt}\PYG{o}{.}\PYG{n}{grid}\PYG{p}{(}\PYG{k+kc}{True}\PYG{p}{)}

\PYG{n+nb}{print}\PYG{p}{(}\PYG{l+s+s1}{\PYGZsq{}}\PYG{l+s+s1}{h=}\PYG{l+s+si}{\PYGZob{}0:.3f\PYGZcb{}}\PYG{l+s+s1}{\PYGZsq{}}\PYG{o}{.}\PYG{n}{format}\PYG{p}{(}\PYG{n}{h}\PYG{p}{)}\PYG{p}{)}
\PYG{n+nb}{print}\PYG{p}{(}\PYG{l+s+s1}{\PYGZsq{}}\PYG{l+s+s1}{f}\PYG{l+s+se}{\PYGZbs{}\PYGZsq{}}\PYG{l+s+s1}{e(}\PYG{l+s+si}{\PYGZob{}0:g\PYGZcb{}}\PYG{l+s+s1}{) = }\PYG{l+s+si}{\PYGZob{}1:.8f\PYGZcb{}}\PYG{l+s+s1}{\PYGZsq{}}\PYG{o}{.}\PYG{n}{format}\PYG{p}{(}\PYG{n}{x0}\PYG{p}{,}\PYG{n}{dfe}\PYG{p}{)}\PYG{p}{)}
\PYG{n+nb}{print}\PYG{p}{(}\PYG{l+s+s1}{\PYGZsq{}}\PYG{l+s+s1}{f}\PYG{l+s+se}{\PYGZbs{}\PYGZsq{}}\PYG{l+s+s1}{p(}\PYG{l+s+si}{\PYGZob{}0:g\PYGZcb{}}\PYG{l+s+s1}{) = }\PYG{l+s+si}{\PYGZob{}1:.8f\PYGZcb{}}\PYG{l+s+s1}{\PYGZsq{}}\PYG{o}{.}\PYG{n}{format}\PYG{p}{(}\PYG{n}{x0}\PYG{p}{,}\PYG{n}{dfp}\PYG{p}{)}\PYG{p}{)}
\PYG{n+nb}{print}\PYG{p}{(}\PYG{l+s+s1}{\PYGZsq{}}\PYG{l+s+s1}{f}\PYG{l+s+se}{\PYGZbs{}\PYGZsq{}}\PYG{l+s+s1}{r(}\PYG{l+s+si}{\PYGZob{}0:g\PYGZcb{}}\PYG{l+s+s1}{) = }\PYG{l+s+si}{\PYGZob{}1:.8f\PYGZcb{}}\PYG{l+s+s1}{\PYGZsq{}}\PYG{o}{.}\PYG{n}{format}\PYG{p}{(}\PYG{n}{x0}\PYG{p}{,}\PYG{n}{dfr}\PYG{p}{)}\PYG{p}{)}
\PYG{n+nb}{print}\PYG{p}{(}\PYG{l+s+s1}{\PYGZsq{}}\PYG{l+s+s1}{f}\PYG{l+s+se}{\PYGZbs{}\PYGZsq{}}\PYG{l+s+s1}{c(}\PYG{l+s+si}{\PYGZob{}0:g\PYGZcb{}}\PYG{l+s+s1}{) = }\PYG{l+s+si}{\PYGZob{}1:.8f\PYGZcb{}}\PYG{l+s+s1}{\PYGZsq{}}\PYG{o}{.}\PYG{n}{format}\PYG{p}{(}\PYG{n}{x0}\PYG{p}{,}\PYG{n}{dfc}\PYG{p}{)}\PYG{p}{)}
\end{sphinxVerbatim}

\begin{sphinxVerbatim}[commandchars=\\\{\}]
h=0.010
f\PYGZsq{}e(1.2) = 0.36235775
f\PYGZsq{}p(1.2) = 0.35769156
f\PYGZsq{}r(1.2) = 0.36701187
f\PYGZsq{}c(1.2) = 0.36235172
\end{sphinxVerbatim}

\noindent\sphinxincludegraphics{{aula-20-diferenciacao-numerica_8_1}.png}


\subsection{Comparando resultados}
\label{\detokenize{aula-20-diferenciacao-numerica:comparando-resultados}}
\sphinxAtStartPar
\sphinxstylestrong{Exemplo:} considere a função \(f(x) = \frac{2^x}{x}\). Calcule a segunda derivada em \(x=2\) numericamente com a fórmula de diferença centrada a 3 pontos usando:
\begin{itemize}
\item {} 
\sphinxAtStartPar
Os pontos \(x = 1.8\), \(x = 2\) e \(x = 2.2\).

\item {} 
\sphinxAtStartPar
Os pontos \(x = 1.9\), \(x = 2\) e \(x = 2.1\).

\end{itemize}

\sphinxAtStartPar
Compare os resultados com a derivada analítica.


\subsubsection{Solução}
\label{\detokenize{aula-20-diferenciacao-numerica:solucao}}
\sphinxAtStartPar
Vamos primeiro computar a seguda derivada analítica de \(f(x)\) por computação simbólica.

\begin{sphinxVerbatim}[commandchars=\\\{\}]
\PYG{k+kn}{from} \PYG{n+nn}{sympy}\PYG{n+nn}{.}\PYG{n+nn}{abc} \PYG{k+kn}{import} \PYG{n}{x} 
\PYG{n}{sy}\PYG{o}{.}\PYG{n}{init\PYGZus{}printing}\PYG{p}{(}\PYG{p}{)}

\PYG{c+c1}{\PYGZsh{} função}
\PYG{n}{f} \PYG{o}{=} \PYG{l+m+mi}{2}\PYG{o}{*}\PYG{o}{*}\PYG{n}{x}\PYG{o}{/}\PYG{n}{x}

\PYG{c+c1}{\PYGZsh{} derivada segunda}
\PYG{n}{d2fdx2} \PYG{o}{=} \PYG{n}{sy}\PYG{o}{.}\PYG{n}{diff}\PYG{p}{(}\PYG{n}{f}\PYG{p}{,}\PYG{n}{x}\PYG{p}{,}\PYG{l+m+mi}{2}\PYG{p}{)}
\PYG{n}{d2fdx2}
\end{sphinxVerbatim}

\noindent\sphinxincludegraphics{{aula-20-diferenciacao-numerica_11_0}.png}

\begin{sphinxVerbatim}[commandchars=\\\{\}]
\PYG{c+c1}{\PYGZsh{} derivada em x = 2 }
\PYG{n}{d2fdx2\PYGZus{}p2} \PYG{o}{=} \PYG{n}{d2fdx2}\PYG{o}{.}\PYG{n}{subs}\PYG{p}{(}\PYG{n}{x}\PYG{p}{,}\PYG{l+m+mi}{2}\PYG{p}{)}
\end{sphinxVerbatim}

\begin{sphinxVerbatim}[commandchars=\\\{\}]
\PYG{c+c1}{\PYGZsh{} valor numérico}
\PYG{n}{d2fdx2\PYGZus{}p2\PYGZus{}num} \PYG{o}{=} \PYG{n+nb}{float}\PYG{p}{(}\PYG{n}{d2fdx2\PYGZus{}p2}\PYG{p}{)}
\PYG{n}{d2fdx2\PYGZus{}p2\PYGZus{}num} 
\end{sphinxVerbatim}

\noindent\sphinxincludegraphics{{aula-20-diferenciacao-numerica_13_0}.png}

\sphinxAtStartPar
Agora, vamos montar uma função para a fórmula da segunda derivada a 3 pontos (supondo que os pontos dados sejam igualmente espaçados:

\begin{sphinxVerbatim}[commandchars=\\\{\}]
\PYG{c+c1}{\PYGZsh{} função }
\PYG{k}{def} \PYG{n+nf}{der\PYGZus{}num\PYGZus{}2}\PYG{p}{(}\PYG{n}{x1}\PYG{p}{,}\PYG{n}{x2}\PYG{p}{,}\PYG{n}{x3}\PYG{p}{)}\PYG{p}{:}
    \PYG{n}{d2fdx2\PYGZus{}num} \PYG{o}{=} \PYG{k}{lambda} \PYG{n}{x1}\PYG{p}{,}\PYG{n}{x2}\PYG{p}{,}\PYG{n}{x3}\PYG{p}{:} \PYG{p}{(}\PYG{n}{f}\PYG{o}{.}\PYG{n}{subs}\PYG{p}{(}\PYG{n}{x}\PYG{p}{,}\PYG{n}{x1}\PYG{p}{)} \PYG{o}{\PYGZhy{}} \PYG{l+m+mi}{2}\PYG{o}{*}\PYG{n}{f}\PYG{o}{.}\PYG{n}{subs}\PYG{p}{(}\PYG{n}{x}\PYG{p}{,}\PYG{n}{x2}\PYG{p}{)} \PYG{o}{+} \PYG{n}{f}\PYG{o}{.}\PYG{n}{subs}\PYG{p}{(}\PYG{n}{x}\PYG{p}{,}\PYG{n}{x3}\PYG{p}{)}\PYG{p}{)}\PYG{o}{/}\PYG{p}{(}\PYG{n}{x2} \PYG{o}{\PYGZhy{}} \PYG{n}{x1}\PYG{p}{)}\PYG{o}{*}\PYG{o}{*}\PYG{l+m+mi}{2}
    \PYG{k}{return} \PYG{n+nb}{float}\PYG{p}{(}\PYG{n}{d2fdx2\PYGZus{}num}\PYG{p}{(}\PYG{n}{x1}\PYG{p}{,}\PYG{n}{x2}\PYG{p}{,}\PYG{n}{x3}\PYG{p}{)}\PYG{p}{)}
\end{sphinxVerbatim}

\sphinxAtStartPar
Aplicamos a nossa aproximação numérica ao primeiro conjunto de pontos para estimar a derivada numérica no ponto \(x=2\) por diferença centrada.

\begin{sphinxVerbatim}[commandchars=\\\{\}]
\PYG{n}{df2\PYGZus{}1} \PYG{o}{=} \PYG{n}{der\PYGZus{}num\PYGZus{}2}\PYG{p}{(}\PYG{l+m+mf}{1.8}\PYG{p}{,}\PYG{l+m+mf}{2.0}\PYG{p}{,}\PYG{l+m+mf}{2.2}\PYG{p}{)}
\PYG{n}{df2\PYGZus{}1}
\end{sphinxVerbatim}

\noindent\sphinxincludegraphics{{aula-20-diferenciacao-numerica_17_0}.png}

\sphinxAtStartPar
Em seguida, aplicamos a nossa aproximação numérica ao segundo conjunto de pontos:

\begin{sphinxVerbatim}[commandchars=\\\{\}]
\PYG{n}{df2\PYGZus{}2} \PYG{o}{=} \PYG{n}{der\PYGZus{}num\PYGZus{}2}\PYG{p}{(}\PYG{l+m+mf}{1.9}\PYG{p}{,}\PYG{l+m+mf}{2.0}\PYG{p}{,}\PYG{l+m+mf}{2.1}\PYG{p}{)}
\PYG{n}{df2\PYGZus{}2}
\end{sphinxVerbatim}

\noindent\sphinxincludegraphics{{aula-20-diferenciacao-numerica_19_0}.png}

\sphinxAtStartPar
Esperamos que o segundo conjunto de pontos nos dê uma estimativa mais próxima para a derivada analítica já que os pontos estão mais próximos um do outro. Para verificar isto, vamos medir o erro relativo percentual entre as derivadas numéricas e a derivada analítica.

\begin{sphinxVerbatim}[commandchars=\\\{\}]
\PYG{c+c1}{\PYGZsh{} erro relativo percentual }
\PYG{n}{erp} \PYG{o}{=} \PYG{k}{lambda} \PYG{n}{v}\PYG{p}{,}\PYG{n}{va}\PYG{p}{:} \PYG{n+nb}{abs}\PYG{p}{(}\PYG{n}{v}\PYG{o}{\PYGZhy{}}\PYG{n}{va}\PYG{p}{)}\PYG{o}{/}\PYG{n+nb}{abs}\PYG{p}{(}\PYG{n}{v}\PYG{p}{)}\PYG{o}{*}\PYG{l+m+mi}{100}

\PYG{c+c1}{\PYGZsh{} erro para o grupo 1}
\PYG{n+nb}{print}\PYG{p}{(}\PYG{l+s+s1}{\PYGZsq{}}\PYG{l+s+s1}{erro relativo percentual 1: }\PYG{l+s+s1}{\PYGZsq{}} \PYG{o}{+} \PYG{n+nb}{str}\PYG{p}{(}\PYG{n}{erp}\PYG{p}{(}\PYG{n}{d2fdx2\PYGZus{}p2\PYGZus{}num}\PYG{p}{,}\PYG{n}{df2\PYGZus{}1}\PYG{p}{)}\PYG{p}{)} \PYG{o}{+} \PYG{l+s+s1}{\PYGZsq{}}\PYG{l+s+s1}{\PYGZpc{}}\PYG{l+s+s1}{\PYGZsq{}} \PYG{p}{)}

\PYG{c+c1}{\PYGZsh{} erro para o grupo 2}
\PYG{n+nb}{print}\PYG{p}{(}\PYG{l+s+s1}{\PYGZsq{}}\PYG{l+s+s1}{erro relativo percentual 2: }\PYG{l+s+s1}{\PYGZsq{}} \PYG{o}{+} \PYG{n+nb}{str}\PYG{p}{(}\PYG{n}{erp}\PYG{p}{(}\PYG{n}{d2fdx2\PYGZus{}p2\PYGZus{}num}\PYG{p}{,}\PYG{n}{df2\PYGZus{}2}\PYG{p}{)}\PYG{p}{)} \PYG{o}{+} \PYG{l+s+s1}{\PYGZsq{}}\PYG{l+s+s1}{\PYGZpc{}}\PYG{l+s+s1}{\PYGZsq{}} \PYG{p}{)}
\end{sphinxVerbatim}

\begin{sphinxVerbatim}[commandchars=\\\{\}]
erro relativo percentual 1: 0.49948640831095764\PYGZpc{}
erro relativo percentual 2: 0.12404011077131777\PYGZpc{}
\end{sphinxVerbatim}


\chapter{Introdução à solução numérica de EDOs}
\label{\detokenize{aula-21-solucoes-edo:introducao-a-solucao-numerica-de-edos}}\label{\detokenize{aula-21-solucoes-edo::doc}}
\sphinxAtStartPar
Equações diferenciais ordinárias (EDOs) surgem em diversos problemas aplicados. Alguns exemplos:
\begin{itemize}
\item {} 
\sphinxAtStartPar
\sphinxstyleemphasis{Química}: decréscimo radioativo de carbono 14;

\item {} 
\sphinxAtStartPar
\sphinxstyleemphasis{Engenharia}: queda da pressão atmosférica;

\item {} 
\sphinxAtStartPar
\sphinxstyleemphasis{Economia}: precificação de ativos financeiros.

\end{itemize}

\sphinxAtStartPar
Nem sempre é possível obter soluções analíticas (forma fechada) para EDOs. Então, precisamos obter soluções aproximadas por meio de métodos numéricos.

\sphinxAtStartPar
No passado, muito esforço era empregado para se desenvolver métodos computacionais ótimos, mas a insuficiência de poder computacional era um entrave. Hoje em dia, com a evolução tecnológica, a capacidade computacional de alto desempenho permite que soluções numéricas sejam obtidas com menor esforço de processamento e margem de erro satisfatória. A seguir, faremos uma breve introdução teórica sobre modelos clássicos descritos por EDOs e a resolubilidade das equações.


\section{Modelos clássicos}
\label{\detokenize{aula-21-solucoes-edo:modelos-classicos}}

\subsection{EDOs de primeira ordem}
\label{\detokenize{aula-21-solucoes-edo:edos-de-primeira-ordem}}\begin{itemize}
\item {} 
\sphinxAtStartPar
\sphinxstylestrong{Crescimento e decrescimento}: modelo (de Malthus) utilizado em crescimento populacional, mor\sphinxhyphen{} talidade de espécies biológicas.

\end{itemize}
\begin{equation*}
\begin{split}y'(t)=ky\Rightarrow y(t)=ce^{kt}, c,k \in \mathbb{R}\end{split}
\end{equation*}
\sphinxAtStartPar
\sphinxstylestrong{Interpretação:} taxa de mudança da quantidade \(y\) é proporcional à própria quantidade ao longo do
tempo. Se \(k > 0\), temos uma lei de crescimento; se \(k < 0\), temos uma lei de descrescimento (ou queda).
\begin{itemize}
\item {} 
\sphinxAtStartPar
\sphinxstylestrong{Lei do resfriamento de Newton:} modelo utilizado para determinar a troca de calor entre um corpo material e um meio externo.

\end{itemize}
\begin{equation*}
\begin{split}T'(t) = k(T − T_{\infty}) \Rightarrow T(t) = T_0 + (T_{\infty} − T_0)e^{kt},\end{split}
\end{equation*}
\sphinxAtStartPar
onde \(T\) é a \sphinxstyleemphasis{temperatura do corpo}, \(k > 0 \in \mathbb{R}\) a \sphinxstyleemphasis{condutividade térmica} (dependente do material do corpo e nem sempre constante), \(T_{\infty}\) a \sphinxstyleemphasis{temperatura do ambiente} e \(T_0\) a \sphinxstyleemphasis{temperatura inicial}.

\sphinxAtStartPar
\sphinxstylestrong{Interpretação}: taxa de mudança da temperatura é proporcional a diferença entre a temperatura do objeto e do ambiente com o qual troca calor.


\subsection{EDOs de segunda ordem}
\label{\detokenize{aula-21-solucoes-edo:edos-de-segunda-ordem}}\begin{itemize}
\item {} 
\sphinxAtStartPar
\sphinxstylestrong{Variação da quantidade de movimento em um sistema (2a. lei de Newton)}: modelo utilizado para descrever a perda de equilíbrio de sistemas mecânicos (cordas vibrantes, molas amortecidas, escoamentos de fluidos viscosos).

\end{itemize}
\begin{equation*}
\begin{split}my''(t) = −by'(t) − ky + f(t),\end{split}
\end{equation*}
\sphinxAtStartPar
onde \(m\) é a \sphinxstyleemphasis{massa}, \(y\) é um \sphinxstyleemphasis{deslocamento}, \(t\) o \sphinxstyleemphasis{tempo}, b > 0 uma \sphinxstyleemphasis{constante de amortecimento} (absorvedor de choque), \(k > 0\) um \sphinxstyleemphasis{parâmetro da mola/empuxo} e \(f(t)\) uma \sphinxstyleemphasis{força externa}.

\sphinxAtStartPar
\sphinxstylestrong{Interpretação}: taxa de mudança da quantidade de movimento do corpo é igual às forças aplicadas sobre o mesmo. A solução geral desta equação não homogênea será omitida aqui (cf. Weiglholfer and Lindsay, p.32).
\begin{itemize}
\item {} 
\sphinxAtStartPar
\sphinxstylestrong{2a. lei de Kirchhoff}: modelo usado em circuitos elétricos e eletromagnetismo.

\end{itemize}
\begin{equation*}
\begin{split}LQ''(t) + RQ'(t) + \frac{1}{C}Q(t) = U(t),\end{split}
\end{equation*}
\sphinxAtStartPar
onde \(Q(t)\) é a \sphinxstyleemphasis{carga elétrica}, \(L\) é a \sphinxstyleemphasis{indutância}, \(R\) a \sphinxstyleemphasis{resistência}, \(C\) a \sphinxstyleemphasis{capacitância} e \(U(t)\) a \sphinxstyleemphasis{força eletromotora} (tensão elétrica).

\sphinxAtStartPar
\sphinxstylestrong{Interpretação}: a força eletromotora (bateria, por exemplo) em qualquer circuito fechado equilibra todas as diferenças de potencial (d.d.p.) naquele circuito. Em outras palavras: em um circuito fechado, a soma de todas as d.d.p. é nula.


\section{Teoria geral de resolubilidade de EDOs}
\label{\detokenize{aula-21-solucoes-edo:teoria-geral-de-resolubilidade-de-edos}}

\subsection{Problema de Valor Inicial (PVI)}
\label{\detokenize{aula-21-solucoes-edo:problema-de-valor-inicial-pvi}}
\sphinxAtStartPar
Um \sphinxstyleemphasis{problema de valor inicial} (PVI) é formado por uma EDO e uma \sphinxstyleemphasis{condição inicial}.
\begin{equation*}
\begin{split}\begin{cases}
y'(t) = f(t,y(t)) \\
y(t_0) = y_0
\end{cases},\end{split}
\end{equation*}
\sphinxAtStartPar
Acima, \(t\) é a \sphinxstyleemphasis{variável independente}, \(y(t)\) é a \sphinxstyleemphasis{variável dependente} (solução da EDO) e \(t_0\) é a \sphinxstyleemphasis{condição inicial}.

\sphinxAtStartPar
\sphinxstylestrong{Exemplo:} A EDO \(y'(t) = -[y(t)]^2 + y(t)\) possui a chamada solução trivial \(y(t) \equiv 0\) e a solução geral:
\begin{equation*}
\begin{split}y(t) = \dfrac{1}{1+ce^{-t}}\end{split}
\end{equation*}
\sphinxAtStartPar
Observemos que \(y(t)\) é indefinida quando \(1+ce^{-t}=0\), ou \(t = \ln(-c)\). Se \(y(0) = y_0 \neq 0\) for uma condição inicial geral, \(c = \frac{1}{y_0} - 1\) e teremos os seguintes resultados adicionais:


\begin{savenotes}\sphinxattablestart
\centering
\begin{tabulary}{\linewidth}[t]{|T|T|T|}
\hline
\sphinxstyletheadfamily 
\sphinxAtStartPar
condição inicial
&\sphinxstyletheadfamily 
\sphinxAtStartPar
valores de \(c\)
&\sphinxstyletheadfamily 
\sphinxAtStartPar
existência de solução
\\
\hline
\sphinxAtStartPar
\(y_0 > 0\)
&
\sphinxAtStartPar
\(c > -1\)
&
\sphinxAtStartPar
\(0 \leq t < \infty\)
\\
\hline
\sphinxAtStartPar
\(y_0 < 0\)
&
\sphinxAtStartPar
\(c < -1\)
&
\sphinxAtStartPar
\(0 < t < \ln(1 - y_0^{-1})\)
\\
\hline
\end{tabulary}
\par
\sphinxattableend\end{savenotes}

\sphinxAtStartPar
Abaixo vemos os gráficos de \(Y(t)\) quando \(c = \{-0.8,-0.5,-0.4,0,0.5,1,0,2.5\}\) e para a solução trivial.

\begin{sphinxVerbatim}[commandchars=\\\{\}]
\PYG{k+kn}{import} \PYG{n+nn}{numpy} \PYG{k}{as} \PYG{n+nn}{np} 
\PYG{k+kn}{import} \PYG{n+nn}{matplotlib}\PYG{n+nn}{.}\PYG{n+nn}{pyplot} \PYG{k}{as} \PYG{n+nn}{plt} 

\PYG{c+c1}{\PYGZsh{} variavel independente}
\PYG{n}{t} \PYG{o}{=} \PYG{n}{np}\PYG{o}{.}\PYG{n}{linspace}\PYG{p}{(}\PYG{l+m+mi}{0}\PYG{p}{,}\PYG{l+m+mi}{6}\PYG{p}{,}\PYG{l+m+mi}{100}\PYG{p}{)}

\PYG{c+c1}{\PYGZsh{} condicao inicial}
\PYG{n}{C} \PYG{o}{=} \PYG{p}{[}\PYG{o}{\PYGZhy{}}\PYG{l+m+mf}{0.8}\PYG{p}{,}\PYG{o}{\PYGZhy{}}\PYG{l+m+mf}{0.5}\PYG{p}{,}\PYG{o}{\PYGZhy{}}\PYG{l+m+mf}{0.4}\PYG{p}{,}\PYG{l+m+mi}{0}\PYG{p}{,}\PYG{l+m+mf}{0.5}\PYG{p}{,}\PYG{l+m+mi}{1}\PYG{p}{,}\PYG{l+m+mi}{0}\PYG{p}{,}\PYG{l+m+mf}{2.5}\PYG{p}{]}

\PYG{k}{for} \PYG{n}{c} \PYG{o+ow}{in} \PYG{n}{C}\PYG{p}{:}    

    \PYG{c+c1}{\PYGZsh{} solucoes não triviais}
    \PYG{k}{if} \PYG{n}{c} \PYG{o}{!=} \PYG{l+m+mi}{0}\PYG{p}{:}
        \PYG{n}{plt}\PYG{o}{.}\PYG{n}{plot}\PYG{p}{(}\PYG{n}{t}\PYG{p}{,}\PYG{l+m+mf}{1.}\PYG{o}{/}\PYG{p}{(}\PYG{l+m+mi}{1} \PYG{o}{+} \PYG{n}{c}\PYG{o}{*}\PYG{n}{np}\PYG{o}{.}\PYG{n}{exp}\PYG{p}{(}\PYG{o}{\PYGZhy{}}\PYG{n}{t}\PYG{p}{)}\PYG{p}{)}\PYG{p}{)}
    
    \PYG{c+c1}{\PYGZsh{} solucao trivial}
    \PYG{k}{else}\PYG{p}{:}
        \PYG{n}{plt}\PYG{o}{.}\PYG{n}{plot}\PYG{p}{(}\PYG{n}{t}\PYG{p}{,}\PYG{l+m+mi}{0}\PYG{o}{*}\PYG{n}{t}\PYG{p}{,}\PYG{l+s+s1}{\PYGZsq{}}\PYG{l+s+s1}{k\PYGZhy{}\PYGZhy{}}\PYG{l+s+s1}{\PYGZsq{}}\PYG{p}{)}    
            
\end{sphinxVerbatim}

\noindent\sphinxincludegraphics{{aula-21-solucoes-edo_1_0}.png}

\begin{sphinxVerbatim}[commandchars=\\\{\}]
\PYG{k+kn}{import} \PYG{n+nn}{numpy} \PYG{k}{as} \PYG{n+nn}{np}
\PYG{k+kn}{import} \PYG{n+nn}{matplotlib}\PYG{n+nn}{.}\PYG{n+nn}{pyplot} \PYG{k}{as} \PYG{n+nn}{plt}
\end{sphinxVerbatim}


\chapter{Método de Euler}
\label{\detokenize{aula-22-metodo-euler:metodo-de-euler}}\label{\detokenize{aula-22-metodo-euler::doc}}
\sphinxAtStartPar
As considerações anteriores sobre malhas numéricas são fundamentais para sugerir a notação que utilizaremos para expressar os métodos numéricos que estudaremos. O primeiro deles, chamado \sphinxstyleemphasis{método de Euler}, é considerado o método numérico mais simples para resolver PVIs. Embora não seja muito eficiente, é o ponto de partida para a compreensão de uma enorme família de métodos.

\sphinxAtStartPar
Ao longo do texto, denotaremos por \(y(t_n) = y_h(t_n) = y_n, \ \ n = 0,1,2,\ldots,N\) a solução aproximada de um PVI.


\section{Definição do método de Euler}
\label{\detokenize{aula-22-metodo-euler:definicao-do-metodo-de-euler}}
\sphinxAtStartPar
A derivação do método de Euler inicia\sphinxhyphen{}se com a seguinte aproximação para a derivada:
\begin{equation*}
\begin{split}y'(t) \approx \dfrac{y(t+h) - y(t)}{h},\end{split}
\end{equation*}
\sphinxAtStartPar
conhecida como \sphinxstyleemphasis{aproximação por diferença finita avançada (ou progressiva)}. Se aplicarmos esta definição ao nosso PVI padrão, em \(t=t_n\) teremos \(y'(t_n) = f(t_n,y(t_n))\), donde segue que
\begin{equation*}
\begin{split}\dfrac{y(t_{n+1}) - y(t_n)}{h} \approx f(t_n,y(t_n))\end{split}
\end{equation*}\begin{equation*}
\begin{split}y(t_{n+1}) \approx y(t_n) + h \, f(t_n,y(t_n)).\end{split}
\end{equation*}
\sphinxAtStartPar
O método de Euler toma esta aproximação como exata, de modo que o esquema numérico resultante é
\begin{equation*}
\begin{split}y_{n+1} = y_n + h \, f(t_n,y_n), \ \ 0 \le n \le N-1.\end{split}
\end{equation*}
\sphinxAtStartPar
A estimativa inicial é \(y_0 = Y_0\) ou alguma aproximação de \(Y_0\). Quando \(Y_0\) é obtido empiricamente, seu valor é conhecido apenas aproximadamente. A fórmula anterior permite o cálculo sequencial das iteradas do método de Euler \(y_1,y_2,\ldots,y_n\), aproximações para os valores exatos de \(y\) nesses instantes.


\section{Interpretação geométrica}
\label{\detokenize{aula-22-metodo-euler:interpretacao-geometrica}}
\sphinxAtStartPar
A figura a seguir ajuda\sphinxhyphen{}nos a interpretar o método de Euler geometricamente. A aproximação numérica da curva exata (azul) é feita por meio de retas tangentes. O valor \(y(t_{n+1})\) é erroneamente computado (comprimento \(BD\)) e excede o valor exato (comprimento \(BC\)) por uma quantidade (comprimento \(CD\)). Isto é, a equação da reta tangente a \((t_n,y(t_n))\) é \(r(t) = y(t_n) + f(t_n,y(t_n))(t-t_n)\). Na verdade, \(r(t_{n+1})\) coincide com o ponto \(D\).

\begin{sphinxVerbatim}[commandchars=\\\{\}]
\PYG{n}{x} \PYG{o}{=} \PYG{n}{np}\PYG{o}{.}\PYG{n}{linspace}\PYG{p}{(}\PYG{l+m+mf}{0.5}\PYG{p}{,}\PYG{l+m+mi}{4}\PYG{p}{,}\PYG{l+m+mi}{50}\PYG{p}{)}

\PYG{n}{plt}\PYG{o}{.}\PYG{n}{plot}\PYG{p}{(}\PYG{n}{x}\PYG{p}{,}\PYG{n}{np}\PYG{o}{.}\PYG{n}{log}\PYG{p}{(}\PYG{n}{x}\PYG{p}{)}\PYG{p}{,}\PYG{l+s+s1}{\PYGZsq{}}\PYG{l+s+s1}{b}\PYG{l+s+s1}{\PYGZsq{}}\PYG{p}{)}
\PYG{n}{x0} \PYG{o}{=} \PYG{l+m+mf}{1.0}
\PYG{n}{x1} \PYG{o}{=} \PYG{l+m+mf}{3.0}
\PYG{n}{plt}\PYG{o}{.}\PYG{n}{plot}\PYG{p}{(}\PYG{n}{x}\PYG{p}{,}\PYG{n}{np}\PYG{o}{.}\PYG{n}{log}\PYG{p}{(}\PYG{n}{x0}\PYG{p}{)} \PYG{o}{+} \PYG{n}{x}\PYG{o}{/}\PYG{n}{x0} \PYG{o}{\PYGZhy{}}\PYG{l+m+mi}{1}\PYG{p}{,}\PYG{l+s+s1}{\PYGZsq{}}\PYG{l+s+s1}{r\PYGZhy{}\PYGZhy{}}\PYG{l+s+s1}{\PYGZsq{}}\PYG{p}{)}
\PYG{n}{plt}\PYG{o}{.}\PYG{n}{plot}\PYG{p}{(}\PYG{n}{x0}\PYG{p}{,}\PYG{n}{np}\PYG{o}{.}\PYG{n}{log}\PYG{p}{(}\PYG{n}{x0}\PYG{p}{)}\PYG{p}{,}\PYG{l+s+s1}{\PYGZsq{}}\PYG{l+s+s1}{ob}\PYG{l+s+s1}{\PYGZsq{}}\PYG{p}{)}
\PYG{n}{plt}\PYG{o}{.}\PYG{n}{plot}\PYG{p}{(}\PYG{n}{x1}\PYG{p}{,}\PYG{n}{np}\PYG{o}{.}\PYG{n}{log}\PYG{p}{(}\PYG{n}{x1}\PYG{p}{)}\PYG{p}{,}\PYG{l+s+s1}{\PYGZsq{}}\PYG{l+s+s1}{ob}\PYG{l+s+s1}{\PYGZsq{}}\PYG{p}{)}
\PYG{n}{plt}\PYG{o}{.}\PYG{n}{plot}\PYG{p}{(}\PYG{n}{x1}\PYG{p}{,}\PYG{l+m+mi}{0}\PYG{p}{,}\PYG{l+s+s1}{\PYGZsq{}}\PYG{l+s+s1}{ok}\PYG{l+s+s1}{\PYGZsq{}}\PYG{p}{)}
\PYG{n}{plt}\PYG{o}{.}\PYG{n}{plot}\PYG{p}{(}\PYG{n}{x}\PYG{p}{,}\PYG{l+m+mi}{0}\PYG{o}{*}\PYG{n}{x}\PYG{p}{,}\PYG{l+s+s1}{\PYGZsq{}}\PYG{l+s+s1}{k\PYGZhy{}\PYGZhy{}}\PYG{l+s+s1}{\PYGZsq{}}\PYG{p}{)}
\PYG{n}{plt}\PYG{o}{.}\PYG{n}{plot}\PYG{p}{(}\PYG{n}{x1}\PYG{p}{,}\PYG{n}{np}\PYG{o}{.}\PYG{n}{log}\PYG{p}{(}\PYG{n}{x0}\PYG{p}{)} \PYG{o}{+} \PYG{n}{x1}\PYG{o}{/}\PYG{n}{x0} \PYG{o}{\PYGZhy{}}\PYG{l+m+mi}{1}\PYG{p}{,}\PYG{l+s+s1}{\PYGZsq{}}\PYG{l+s+s1}{or}\PYG{l+s+s1}{\PYGZsq{}}\PYG{p}{)}
\PYG{n}{plt}\PYG{o}{.}\PYG{n}{vlines}\PYG{p}{(}\PYG{n}{x1}\PYG{o}{+}\PYG{l+m+mf}{0.1}\PYG{p}{,}\PYG{n}{ymin}\PYG{o}{=}\PYG{l+m+mi}{0}\PYG{p}{,}\PYG{n}{ymax}\PYG{o}{=}\PYG{n}{np}\PYG{o}{.}\PYG{n}{log}\PYG{p}{(}\PYG{n}{x0}\PYG{p}{)} \PYG{o}{+} \PYG{n}{x1}\PYG{o}{/}\PYG{n}{x0} \PYG{o}{\PYGZhy{}}\PYG{l+m+mi}{1}\PYG{p}{,}\PYG{n}{colors}\PYG{o}{=}\PYG{l+s+s1}{\PYGZsq{}}\PYG{l+s+s1}{r}\PYG{l+s+s1}{\PYGZsq{}}\PYG{p}{,}\PYG{n}{linestyles}\PYG{o}{=}\PYG{l+s+s1}{\PYGZsq{}}\PYG{l+s+s1}{dashed}\PYG{l+s+s1}{\PYGZsq{}}\PYG{p}{)}
\PYG{n}{plt}\PYG{o}{.}\PYG{n}{vlines}\PYG{p}{(}\PYG{n}{x1}\PYG{p}{,}\PYG{n}{ymin}\PYG{o}{=}\PYG{l+m+mi}{0}\PYG{p}{,}\PYG{n}{ymax}\PYG{o}{=}\PYG{n}{np}\PYG{o}{.}\PYG{n}{log}\PYG{p}{(}\PYG{n}{x1}\PYG{p}{)}\PYG{p}{,}\PYG{n}{colors}\PYG{o}{=}\PYG{l+s+s1}{\PYGZsq{}}\PYG{l+s+s1}{b}\PYG{l+s+s1}{\PYGZsq{}}\PYG{p}{,}\PYG{n}{linestyles}\PYG{o}{=}\PYG{l+s+s1}{\PYGZsq{}}\PYG{l+s+s1}{dashed}\PYG{l+s+s1}{\PYGZsq{}}\PYG{p}{)}


\PYG{n}{xt} \PYG{o}{=} \PYG{p}{[}\PYG{n}{x0}\PYG{p}{,}\PYG{n}{x1}\PYG{p}{]}
\PYG{n}{plt}\PYG{o}{.}\PYG{n}{box}\PYG{p}{(}\PYG{k+kc}{False}\PYG{p}{)}
\PYG{n}{locs}\PYG{p}{,} \PYG{n}{labels} \PYG{o}{=} \PYG{n}{plt}\PYG{o}{.}\PYG{n}{xticks}\PYG{p}{(}\PYG{p}{)}
\PYG{n}{plt}\PYG{o}{.}\PYG{n}{xticks}\PYG{p}{(}\PYG{n}{xt}\PYG{p}{,} \PYG{p}{(}\PYG{l+s+s1}{\PYGZsq{}}\PYG{l+s+s1}{\PYGZdl{}t\PYGZus{}n\PYGZdl{}}\PYG{l+s+s1}{\PYGZsq{}}\PYG{p}{,}\PYG{l+s+s1}{\PYGZsq{}}\PYG{l+s+s1}{\PYGZdl{}t\PYGZus{}}\PYG{l+s+s1}{\PYGZob{}}\PYG{l+s+s1}{n+1\PYGZcb{}\PYGZdl{}}\PYG{l+s+s1}{\PYGZsq{}}\PYG{p}{)}\PYG{p}{)}
\PYG{n}{plt}\PYG{o}{.}\PYG{n}{tick\PYGZus{}params}\PYG{p}{(}\PYG{n}{axis}\PYG{o}{=}\PYG{l+s+s1}{\PYGZsq{}}\PYG{l+s+s1}{both}\PYG{l+s+s1}{\PYGZsq{}}\PYG{p}{,}\PYG{n}{width}\PYG{o}{=}\PYG{l+m+mf}{0.0}\PYG{p}{,}\PYG{n}{labelleft}\PYG{o}{=}\PYG{k+kc}{False}\PYG{p}{)}

\PYG{n}{fs} \PYG{o}{=} \PYG{l+m+mi}{18}
\PYG{n}{plt}\PYG{o}{.}\PYG{n}{annotate}\PYG{p}{(}\PYG{l+s+s1}{\PYGZsq{}}\PYG{l+s+s1}{A}\PYG{l+s+s1}{\PYGZsq{}}\PYG{p}{,}\PYG{n}{xy}\PYG{o}{=}\PYG{p}{(}\PYG{n}{x0}\PYG{o}{\PYGZhy{}}\PYG{l+m+mf}{0.2}\PYG{p}{,}\PYG{n}{np}\PYG{o}{.}\PYG{n}{log}\PYG{p}{(}\PYG{n}{x0}\PYG{p}{)}\PYG{o}{+}\PYG{l+m+mf}{0.2}\PYG{p}{)}\PYG{p}{,}\PYG{n}{fontsize}\PYG{o}{=}\PYG{n}{fs}\PYG{p}{)}
\PYG{n}{plt}\PYG{o}{.}\PYG{n}{annotate}\PYG{p}{(}\PYG{l+s+s1}{\PYGZsq{}}\PYG{l+s+s1}{B}\PYG{l+s+s1}{\PYGZsq{}}\PYG{p}{,}\PYG{n}{xy}\PYG{o}{=}\PYG{p}{(}\PYG{n}{x1}\PYG{o}{\PYGZhy{}}\PYG{l+m+mf}{0.2}\PYG{p}{,}\PYG{l+m+mi}{0}\PYG{o}{+}\PYG{l+m+mf}{0.2}\PYG{p}{)}\PYG{p}{,}\PYG{n}{fontsize}\PYG{o}{=}\PYG{n}{fs}\PYG{p}{)}
\PYG{n}{plt}\PYG{o}{.}\PYG{n}{annotate}\PYG{p}{(}\PYG{l+s+s1}{\PYGZsq{}}\PYG{l+s+s1}{C}\PYG{l+s+s1}{\PYGZsq{}}\PYG{p}{,}\PYG{n}{xy}\PYG{o}{=}\PYG{p}{(}\PYG{n}{x1}\PYG{o}{\PYGZhy{}}\PYG{l+m+mf}{0.1}\PYG{p}{,}\PYG{n}{np}\PYG{o}{.}\PYG{n}{log}\PYG{p}{(}\PYG{n}{x1}\PYG{p}{)}\PYG{o}{+}\PYG{l+m+mf}{0.2}\PYG{p}{)}\PYG{p}{,}\PYG{n}{fontsize}\PYG{o}{=}\PYG{n}{fs}\PYG{p}{)}
\PYG{n}{plt}\PYG{o}{.}\PYG{n}{annotate}\PYG{p}{(}\PYG{l+s+s1}{\PYGZsq{}}\PYG{l+s+s1}{D}\PYG{l+s+s1}{\PYGZsq{}}\PYG{p}{,}\PYG{n}{xy}\PYG{o}{=}\PYG{p}{(}\PYG{n}{x1}\PYG{o}{\PYGZhy{}}\PYG{l+m+mf}{0.2}\PYG{p}{,}\PYG{n}{np}\PYG{o}{.}\PYG{n}{log}\PYG{p}{(}\PYG{n}{x0}\PYG{p}{)} \PYG{o}{+} \PYG{n}{x1}\PYG{o}{/}\PYG{n}{x0} \PYG{o}{\PYGZhy{}} \PYG{l+m+mi}{1} \PYG{o}{+}\PYG{l+m+mf}{0.2}\PYG{p}{)}\PYG{p}{,}\PYG{n}{fontsize}\PYG{o}{=}\PYG{n}{fs}\PYG{p}{)}
\end{sphinxVerbatim}

\begin{sphinxVerbatim}[commandchars=\\\{\}]
Text(2.8, 2.2, \PYGZsq{}D\PYGZsq{})
\end{sphinxVerbatim}

\noindent\sphinxincludegraphics{{aula-22-metodo-euler_2_1}.png}

\sphinxAtStartPar
\sphinxstylestrong{Exemplo:} A solução exata do PVI
\begin{equation*}
\begin{split}\begin{cases}
y'(t) = -y(t)\\
y(0) = 1
\end{cases}\end{split}
\end{equation*}
\sphinxAtStartPar
é \(y(t) = e^{-t}\). O método de Euler é dado por
\begin{equation*}
\begin{split}y_{n+1} = y_n - h y_n = (1-h)y_n, \ \ n \geq 0,\end{split}
\end{equation*}
\sphinxAtStartPar
com \(Y_0 = 1\) e \(t_n = nh\).

\sphinxAtStartPar
Para \(h=0.1\), temos, por exemplo
\begin{equation*}
\begin{split}y_1 = (1-h)y_0 = 0.9(1) = 0.9\end{split}
\end{equation*}\begin{equation*}
\begin{split}y_2 = (1-h)y_1 = 0.9(0.9) = 0.81\end{split}
\end{equation*}\begin{equation*}
\begin{split}y_2 = (1-h)y_1 = 0.9(0.9) = 0.81,\end{split}
\end{equation*}
\sphinxAtStartPar
cujos erros são
\begin{equation*}
\begin{split}y_{h,1} - y_1 = e^{-0.1} - y_1 = 0.004837\end{split}
\end{equation*}\begin{equation*}
\begin{split}y_{h,2} - y_2 = e^{-0.2} - y_2 = 0.008731\end{split}
\end{equation*}

\section{Implementação computacional}
\label{\detokenize{aula-22-metodo-euler:implementacao-computacional}}
\sphinxAtStartPar
O seguinte código implementa o método de Euler explícito.

\begin{sphinxVerbatim}[commandchars=\\\{\}]
\PYG{k+kn}{from} \PYG{n+nn}{numpy} \PYG{k+kn}{import} \PYG{o}{*}

\PYG{k}{def} \PYG{n+nf}{euler\PYGZus{}expl}\PYG{p}{(}\PYG{n}{t0}\PYG{p}{,}\PYG{n}{tf}\PYG{p}{,}\PYG{n}{y0}\PYG{p}{,}\PYG{n}{h}\PYG{p}{,}\PYG{n}{fun}\PYG{p}{)}\PYG{p}{:}
    \PYG{l+s+sd}{\PYGZdq{}\PYGZdq{}\PYGZdq{}}
\PYG{l+s+sd}{    Resolve o PVI y\PYGZsq{} = f(t,y), t0 \PYGZlt{}= t \PYGZlt{}= tf, y(t0) = y0}
\PYG{l+s+sd}{    com passo h usando o metodo de Euler explicito. }
\PYG{l+s+sd}{    }
\PYG{l+s+sd}{    Entrada: }
\PYG{l+s+sd}{        t0  \PYGZhy{} tempo inicial}
\PYG{l+s+sd}{        tf  \PYGZhy{} tempo final }
\PYG{l+s+sd}{        y0  \PYGZhy{} condicao inicial }
\PYG{l+s+sd}{        h   \PYGZhy{} passo }
\PYG{l+s+sd}{        fun \PYGZhy{} funcao f(t,y) (anonima)}
\PYG{l+s+sd}{        }
\PYG{l+s+sd}{    Saida:}
\PYG{l+s+sd}{        t   \PYGZhy{} nos da malha numerica }
\PYG{l+s+sd}{        y   \PYGZhy{} solucao aproximada}
\PYG{l+s+sd}{    \PYGZdq{}\PYGZdq{}\PYGZdq{}}
    
    \PYG{n}{n} \PYG{o}{=} \PYG{n+nb}{round}\PYG{p}{(}\PYG{p}{(}\PYG{n}{tf} \PYG{o}{\PYGZhy{}} \PYG{n}{t0}\PYG{p}{)}\PYG{o}{/}\PYG{n}{h}\PYG{p}{)} \PYG{o}{+} \PYG{l+m+mi}{1}
    \PYG{n}{t} \PYG{o}{=} \PYG{n}{linspace}\PYG{p}{(}\PYG{n}{t0}\PYG{p}{,}\PYG{n}{t0}\PYG{o}{+}\PYG{p}{(}\PYG{n}{n}\PYG{o}{\PYGZhy{}}\PYG{l+m+mi}{1}\PYG{p}{)}\PYG{o}{*}\PYG{n}{h}\PYG{p}{,}\PYG{n}{n}\PYG{p}{)}
    \PYG{n}{y} \PYG{o}{=} \PYG{n}{linspace}\PYG{p}{(}\PYG{n}{t0}\PYG{p}{,}\PYG{n}{t0}\PYG{o}{+}\PYG{p}{(}\PYG{n}{n}\PYG{o}{\PYGZhy{}}\PYG{l+m+mi}{1}\PYG{p}{)}\PYG{o}{*}\PYG{n}{h}\PYG{p}{,}\PYG{n}{n}\PYG{p}{)}
    \PYG{n}{y} \PYG{o}{=} \PYG{n}{zeros}\PYG{p}{(}\PYG{p}{(}\PYG{n}{n}\PYG{p}{,}\PYG{p}{)}\PYG{p}{)}
    
    \PYG{n}{y}\PYG{p}{[}\PYG{l+m+mi}{0}\PYG{p}{]} \PYG{o}{=} \PYG{n}{y0}

    \PYG{k}{for} \PYG{n}{i} \PYG{o+ow}{in} \PYG{n+nb}{range}\PYG{p}{(}\PYG{l+m+mi}{1}\PYG{p}{,}\PYG{n}{n}\PYG{p}{)}\PYG{p}{:}
        \PYG{n}{y}\PYG{p}{[}\PYG{n}{i}\PYG{p}{]} \PYG{o}{=} \PYG{n}{y}\PYG{p}{[}\PYG{n}{i}\PYG{o}{\PYGZhy{}}\PYG{l+m+mi}{1}\PYG{p}{]} \PYG{o}{+} \PYG{n}{h}\PYG{o}{*}\PYG{n}{f}\PYG{p}{(}\PYG{n}{t}\PYG{p}{[}\PYG{n}{i}\PYG{o}{\PYGZhy{}}\PYG{l+m+mi}{1}\PYG{p}{]}\PYG{p}{,}\PYG{n}{y}\PYG{p}{[}\PYG{n}{i}\PYG{o}{\PYGZhy{}}\PYG{l+m+mi}{1}\PYG{p}{]}\PYG{p}{)}

    \PYG{k}{return} \PYG{p}{(}\PYG{n}{t}\PYG{p}{,}\PYG{n}{y}\PYG{p}{)}
\end{sphinxVerbatim}

\sphinxAtStartPar
\sphinxstylestrong{Exemplo:} Resolva
\begin{equation*}
\begin{split}\begin{cases}
y'(t) = \frac{y(t) + t^2 - 2}{t+1}\\
y(0) = 2 \\
0 \le t \le 6 \\ 
h = 0.1
\end{cases}\end{split}
\end{equation*}
\sphinxAtStartPar
Defina \(y_h(t)\) como a solução numérica, calcule o erro relativo e plote o gráfico de \(y_h(t)\) juntamente com o da solução exata \(y(t) = t^2 + 2t + 2 - 2(t+1)\ln(t+1)\)

\sphinxAtStartPar
\sphinxstylestrong{Solução:} O processo iterativo de Euler será dado por
\begin{equation*}
\begin{split}y_{n+1} = y_n + \dfrac{h(y_n + t_n^2 - 2)}{t_n + 1}, \ \ n \geq 0, \ \ y_0 = 2, t_n = nh.\end{split}
\end{equation*}
\sphinxAtStartPar
Entretanto, vamos usar o nosso programa \sphinxcode{\sphinxupquote{euler\_expl}}.

\begin{sphinxVerbatim}[commandchars=\\\{\}]
\PYG{c+c1}{\PYGZsh{} define funcao}
\PYG{n}{f} \PYG{o}{=} \PYG{k}{lambda} \PYG{n}{t}\PYG{p}{,}\PYG{n}{y}\PYG{p}{:} \PYG{p}{(}\PYG{n}{y} \PYG{o}{+} \PYG{n}{t}\PYG{o}{*}\PYG{o}{*}\PYG{l+m+mi}{2} \PYG{o}{\PYGZhy{}} \PYG{l+m+mi}{2}\PYG{p}{)}\PYG{o}{/}\PYG{p}{(}\PYG{n}{t}\PYG{o}{+}\PYG{l+m+mi}{1}\PYG{p}{)}

\PYG{c+c1}{\PYGZsh{} invoca metodo}
\PYG{n}{t0} \PYG{o}{=} \PYG{l+m+mf}{0.0}
\PYG{n}{tf} \PYG{o}{=} \PYG{l+m+mf}{6.0}
\PYG{n}{y0} \PYG{o}{=} \PYG{l+m+mf}{2.0}
\PYG{n}{h} \PYG{o}{=} \PYG{l+m+mf}{0.1}
\PYG{n}{t}\PYG{p}{,}\PYG{n}{y} \PYG{o}{=} \PYG{n}{euler\PYGZus{}expl}\PYG{p}{(}\PYG{n}{t0}\PYG{p}{,}\PYG{n}{tf}\PYG{p}{,}\PYG{n}{y0}\PYG{p}{,}\PYG{n}{h}\PYG{p}{,}\PYG{n}{f}\PYG{p}{)}

\PYG{c+c1}{\PYGZsh{} plota funcoes }
\PYG{n}{yex} \PYG{o}{=} \PYG{n}{t}\PYG{o}{*}\PYG{o}{*}\PYG{l+m+mi}{2} \PYG{o}{+} \PYG{l+m+mi}{2}\PYG{o}{*}\PYG{n}{t} \PYG{o}{+} \PYG{l+m+mi}{2} \PYG{o}{\PYGZhy{}} \PYG{l+m+mi}{2}\PYG{o}{*}\PYG{p}{(}\PYG{n}{t}\PYG{o}{+}\PYG{l+m+mi}{1}\PYG{p}{)}\PYG{o}{*}\PYG{n}{log}\PYG{p}{(}\PYG{n}{t}\PYG{o}{+}\PYG{l+m+mi}{1}\PYG{p}{)}
\PYG{n}{plt}\PYG{o}{.}\PYG{n}{plot}\PYG{p}{(}\PYG{n}{t}\PYG{p}{,}\PYG{n}{y}\PYG{p}{,}\PYG{l+s+s1}{\PYGZsq{}}\PYG{l+s+s1}{b\PYGZhy{}\PYGZhy{}}\PYG{l+s+s1}{\PYGZsq{}}\PYG{p}{,}\PYG{n}{label}\PYG{o}{=}\PYG{l+s+s1}{\PYGZsq{}}\PYG{l+s+s1}{\PYGZdl{}y\PYGZus{}h(t)\PYGZdl{}}\PYG{l+s+s1}{\PYGZsq{}}\PYG{p}{)}
\PYG{n}{plt}\PYG{o}{.}\PYG{n}{plot}\PYG{p}{(}\PYG{n}{t}\PYG{p}{,}\PYG{n}{yex}\PYG{p}{,}\PYG{l+s+s1}{\PYGZsq{}}\PYG{l+s+s1}{k}\PYG{l+s+s1}{\PYGZsq{}}\PYG{p}{,}\PYG{n}{label}\PYG{o}{=}\PYG{l+s+s1}{\PYGZsq{}}\PYG{l+s+s1}{\PYGZdl{}y(t)\PYGZdl{}}\PYG{l+s+s1}{\PYGZsq{}}\PYG{p}{)}
\PYG{n}{plt}\PYG{o}{.}\PYG{n}{legend}\PYG{p}{(}\PYG{p}{)}
\end{sphinxVerbatim}

\begin{sphinxVerbatim}[commandchars=\\\{\}]
\PYGZlt{}matplotlib.legend.Legend at 0x7fd023af8b20\PYGZgt{}
\end{sphinxVerbatim}

\noindent\sphinxincludegraphics{{aula-22-metodo-euler_7_1}.png}

\sphinxAtStartPar
Agora, vamos computar a curva do erro relativo e plotá\sphinxhyphen{}la.

\begin{sphinxVerbatim}[commandchars=\\\{\}]
\PYG{k}{def} \PYG{n+nf}{erro\PYGZus{}relativo}\PYG{p}{(}\PYG{n}{y}\PYG{p}{,}\PYG{n}{yh}\PYG{p}{)}\PYG{p}{:}
    \PYG{k}{return} \PYG{n+nb}{abs}\PYG{p}{(}\PYG{n}{y}\PYG{o}{\PYGZhy{}}\PYG{n}{yh}\PYG{p}{)}\PYG{o}{/}\PYG{n+nb}{abs}\PYG{p}{(}\PYG{n}{yh}\PYG{p}{)}

\PYG{n}{e} \PYG{o}{=} \PYG{n}{erro\PYGZus{}relativo}\PYG{p}{(}\PYG{n}{yex}\PYG{p}{,}\PYG{n}{y}\PYG{p}{)}\PYG{o}{+}\PYG{l+m+mf}{0.1}

\PYG{n}{plt}\PYG{o}{.}\PYG{n}{plot}\PYG{p}{(}\PYG{n}{t}\PYG{p}{,}\PYG{n}{e}\PYG{p}{,}\PYG{l+s+s1}{\PYGZsq{}}\PYG{l+s+s1}{r}\PYG{l+s+s1}{\PYGZsq{}}\PYG{p}{,}\PYG{n}{label}\PYG{o}{=}\PYG{l+s+s1}{\PYGZsq{}}\PYG{l+s+s1}{\PYGZdl{}ER\PYGZdl{}}\PYG{l+s+s1}{\PYGZsq{}}\PYG{p}{)}
\PYG{n}{plt}\PYG{o}{.}\PYG{n}{legend}\PYG{p}{(}\PYG{p}{)}
\PYG{n}{emax} \PYG{o}{=} \PYG{n}{np}\PYG{o}{.}\PYG{n}{max}\PYG{p}{(}\PYG{n}{e}\PYG{p}{)}
\PYG{n}{emin} \PYG{o}{=} \PYG{n}{np}\PYG{o}{.}\PYG{n}{min}\PYG{p}{(}\PYG{n}{e}\PYG{p}{)}

\PYG{n}{plt}\PYG{o}{.}\PYG{n}{plot}\PYG{p}{(}\PYG{n}{t}\PYG{p}{,}\PYG{n}{np}\PYG{o}{.}\PYG{n}{ones}\PYG{p}{(}\PYG{n}{t}\PYG{o}{.}\PYG{n}{shape}\PYG{p}{)}\PYG{o}{*}\PYG{n}{emax}\PYG{p}{,}\PYG{l+s+s1}{\PYGZsq{}}\PYG{l+s+s1}{b\PYGZhy{}\PYGZhy{}}\PYG{l+s+s1}{\PYGZsq{}}\PYG{p}{)}
\PYG{n}{plt}\PYG{o}{.}\PYG{n}{plot}\PYG{p}{(}\PYG{n}{t}\PYG{p}{,}\PYG{n}{np}\PYG{o}{.}\PYG{n}{ones}\PYG{p}{(}\PYG{n}{t}\PYG{o}{.}\PYG{n}{shape}\PYG{p}{)}\PYG{o}{*}\PYG{n}{emin}\PYG{p}{,}\PYG{l+s+s1}{\PYGZsq{}}\PYG{l+s+s1}{g\PYGZhy{}\PYGZhy{}}\PYG{l+s+s1}{\PYGZsq{}}\PYG{p}{)}
\end{sphinxVerbatim}

\begin{sphinxVerbatim}[commandchars=\\\{\}]
[\PYGZlt{}matplotlib.lines.Line2D at 0x7fd023bee880\PYGZgt{}]
\end{sphinxVerbatim}

\noindent\sphinxincludegraphics{{aula-22-metodo-euler_9_1}.png}

\sphinxAtStartPar
\sphinxstylestrong{Exercício:} Resolva o PVI do exemplo anterior usando por computador para \(h = 0.2, 0.1, 0.05\). Produza um \sphinxcode{\sphinxupquote{script}} para imprimir, para cada \(h\), os dados da saída na forma da tabela a seguir:


\begin{savenotes}\sphinxattablestart
\centering
\begin{tabulary}{\linewidth}[t]{|T|T|T|T|}
\hline
\sphinxstyletheadfamily 
\sphinxAtStartPar
\(t\)
&\sphinxstyletheadfamily 
\sphinxAtStartPar
\(y_{h}\)\((t)\)
&\sphinxstyletheadfamily 
\sphinxAtStartPar
\(EA\)
&\sphinxstyletheadfamily 
\sphinxAtStartPar
\(ER\)
\\
\hline
\sphinxAtStartPar
\(\vdots\)
&
\sphinxAtStartPar
\(\vdots\)
&
\sphinxAtStartPar
\(\vdots\)
&
\sphinxAtStartPar
\(\vdots\)
\\
\hline
\end{tabulary}
\par
\sphinxattableend\end{savenotes}


\section{Análise de erro}
\label{\detokenize{aula-22-metodo-euler:analise-de-erro}}
\sphinxAtStartPar
A análise de erro para o método de Euler tem o propósito de entender como o método funciona para que se possa estimar o erro ao usá\sphinxhyphen{}lo e, possivelmente, acelerar sua convergência. Procedimentos similares são aplicáveis a métodos numéricos mais eficientes.

\sphinxAtStartPar
Para a análise, assumiremos que o PVI padrão possui solução única \(y(t)\) em \([t_0,b]\) e que esta solução tem uma segunda derivada \(y''(t)\) limitada neste intervalo.

\sphinxAtStartPar
Consideremos a série de Taylor para aproximar \(y(t_{n+1})\):
\begin{equation*}
\begin{split}y(t_{n+1}) = y(t_{n}) + hy'(t_{n}) + \dfrac{h^2}{2}y''(\xi_{n}), \ \ t_n \leq \xi_n \leq t_{n+1}\end{split}
\end{equation*}
\sphinxAtStartPar
Uma vez que \(y(t)\) satisfaz a EDO, temos que
\begin{equation*}
\begin{split}y(t_{n+1}) = y(t_{n}) + hf(t_{n},y(t_n)) + \dfrac{h^2}{2}y''(\xi_{n}), \ \ t_n \leq \xi_n \leq t_{n+1}\end{split}
\end{equation*}
\sphinxAtStartPar
O termo \(T_{n+1} = \dfrac{h^2}{2}y''(\xi_{n})\), \sphinxstyleemphasis{erro de truncamento} para o método de Euler, é o erro na aproximação
\begin{equation*}
\begin{split}y(t_{n+1}) \approx y(t_{n}) + hy'(t_{n}).\end{split}
\end{equation*}
\sphinxAtStartPar
Assim, tendo em vista que
\begin{equation*}
\begin{split}y_{n+1} \approx y_n + hf(t_{n},y_n),\end{split}
\end{equation*}
\sphinxAtStartPar
subtraímos as equações para obter
\begin{equation*}
\begin{split}y(t_{n+1}) - y_{n+1} \approx y(t_{n}) - y_n + h[ f(t_{n},y(t_n)) - f(t_{n},y_n)] + \dfrac{h^2}{2}y''(\xi_{n}),\end{split}
\end{equation*}
\sphinxAtStartPar
mostrando que o error em \(y_{n+1}\) consiste de duas partes:
\begin{itemize}
\item {} 
\sphinxAtStartPar
\sphinxstylestrong{erro propagado:} \(T_{n+1} = y(t_{n}) - y_n + h[ f(t_{n},y(t_n)) - f(t_{n},y_n)]\).

\item {} 
\sphinxAtStartPar
\sphinxstylestrong{erro de truncamento:} \(T_{n+1} = \dfrac{h^2}{2}y''(\xi_{n})\).

\end{itemize}

\sphinxAtStartPar
O erro propagado pode ser simplificado pelo teorema do valor médio:
\begin{equation*}
\begin{split}f(t_{n},y(t_n)) - f(t_{n},y_n) = \dfrac{\partial f(t_n,\zeta_n)}{\partial y}[y(t_n) - y_n], \ \ y(t_n) \leq \zeta_n < y_n\end{split}
\end{equation*}
\sphinxAtStartPar
Se definirmos o erro \(e_k = y(t_k) - y_k, k \geq 0\), podemos reescrever a última equação como
\begin{equation*}
\begin{split}e_{n+1} = \left[ 1 + h \dfrac{\partial f(t_n,\zeta_n)}{\partial y} \right]e_n + T_{n+1},\end{split}
\end{equation*}
\sphinxAtStartPar
assim obtendo uma equação geral para análise de erro do método de Euler.


\subsection{Teoremas para limite de erro}
\label{\detokenize{aula-22-metodo-euler:teoremas-para-limite-de-erro}}
\sphinxAtStartPar
\sphinxstylestrong{Teorema:} Suponha que \(f(t,y)\) seja definida em um conjunto convexo \(D \subset \mathbb{R}^2\). Se existe uma constante \(L>0\) com
\begin{equation*}
\begin{split}\left| \dfrac{\partial f}{\partial y}(t,y)\right| \leq L, \ \ \forall(t,y) \in D\end{split}
\end{equation*}
\sphinxAtStartPar
\sphinxstylestrong{Teorema (limite de erro):} Suponha que f seja contínua e satisfaça a condição de Lipschitz com constante \(L\) em
\(D = \{ (t,y); \ a \leq t \leq b \}\) e que exista uma constante \(M\) com \(|y''(t)| \leq M, \ \ \forall t \in [a,b]\). Seja \(y(t)\) única solução do PVI
\begin{equation*}
\begin{split}y' = f(t,y), \quad y(a) = y_0, \quad t_0 \leq t \leq b\end{split}
\end{equation*}
\sphinxAtStartPar
e sejam \(w_0,w_1,\ldots,w_N\) aproximações geradas pelo método de Euler para algum inteiro positivo \(N\). Então, para cada \(i=0,1,2,\ldots,N\),
\begin{equation*}
\begin{split}|y(t_i)-w_i| \leq \frac{hM}{2L}[e^{L(t_i-a)} -1].\end{split}
\end{equation*}
\sphinxAtStartPar
O último teorema fornece um limitante de erro para o método de Euler. Ele evidencia a dependência linear do tamanho de passo \(h\). Logo, à medida que \(h->0\), uma maior precisão nas aproximações deve ser obtida.

\sphinxAtStartPar
\sphinxstylestrong{Exemplo:} Consideremos o PVI
\begin{equation*}
\begin{split}\begin{cases}
y'(t) = y - t^2 + 1 \\
y(0) = 0.5 \\
0 \le t \le 2 \\ 
h = 0.2
\end{cases}\end{split}
\end{equation*}
\sphinxAtStartPar
Uma vez que \(f(t,y) = y - t^2 + 1\), temos \(\dfrac{\partial f}{\partial y}(t,y) = 1, \forall y\). Assim, \(L=1\). A solução exata deste problema é \(y(t) = (t+1)^2 - 0.5e^t\), de modo que \(y''(t) = 2 - 0.5e^t\) e \(|y''(t)| \leq 0.5e^2 - 2 = M, \ \ \forall t \in [0,2]\). Pelo teorema do limite de erro, temos que
\begin{equation*}
\begin{split}|y_i - w_i| \leq 0.1(0.5e^2-2)(e^{t_i}-1)\end{split}
\end{equation*}
\begin{sphinxVerbatim}[commandchars=\\\{\}]
\PYG{k}{def} \PYG{n+nf}{f}\PYG{p}{(}\PYG{n}{t}\PYG{p}{,}\PYG{n}{y}\PYG{p}{)}\PYG{p}{:}
    \PYG{k}{return} \PYG{n}{y} \PYG{o}{\PYGZhy{}} \PYG{n}{t}\PYG{o}{*}\PYG{o}{*}\PYG{l+m+mi}{2} \PYG{o}{+} \PYG{l+m+mi}{1}

\PYG{c+c1}{\PYGZsh{} solucao numerica }
\PYG{n}{t}\PYG{p}{,}\PYG{n}{y} \PYG{o}{=} \PYG{n}{euler\PYGZus{}expl}\PYG{p}{(}\PYG{l+m+mi}{0}\PYG{p}{,}\PYG{l+m+mi}{2}\PYG{p}{,}\PYG{l+m+mf}{0.5}\PYG{p}{,}\PYG{l+m+mf}{0.2}\PYG{p}{,}\PYG{n}{f}\PYG{p}{)}

\PYG{c+c1}{\PYGZsh{} solucao exata}
\PYG{n}{yex} \PYG{o}{=} \PYG{p}{(}\PYG{n}{t}\PYG{o}{+}\PYG{l+m+mi}{1}\PYG{p}{)}\PYG{o}{*}\PYG{o}{*}\PYG{l+m+mi}{2} \PYG{o}{\PYGZhy{}} \PYG{l+m+mf}{0.5}\PYG{o}{*}\PYG{n}{np}\PYG{o}{.}\PYG{n}{exp}\PYG{p}{(}\PYG{n}{t}\PYG{p}{)}

\PYG{c+c1}{\PYGZsh{} erro}
\PYG{n}{erro} \PYG{o}{=} \PYG{n}{np}\PYG{o}{.}\PYG{n}{abs}\PYG{p}{(}\PYG{n}{y} \PYG{o}{\PYGZhy{}} \PYG{n}{yex}\PYG{p}{)}

\PYG{c+c1}{\PYGZsh{} limite de erro }
\PYG{n}{lim\PYGZus{}erro} \PYG{o}{=} \PYG{l+m+mf}{0.1}\PYG{o}{*}\PYG{p}{(}\PYG{l+m+mf}{0.5}\PYG{o}{*}\PYG{n}{np}\PYG{o}{.}\PYG{n}{exp}\PYG{p}{(}\PYG{l+m+mi}{1}\PYG{p}{)}\PYG{o}{*}\PYG{o}{*}\PYG{l+m+mi}{2} \PYG{o}{\PYGZhy{}} \PYG{l+m+mi}{2}\PYG{p}{)}\PYG{o}{*}\PYG{p}{(}\PYG{n}{np}\PYG{o}{.}\PYG{n}{exp}\PYG{p}{(}\PYG{n}{t}\PYG{p}{)} \PYG{o}{\PYGZhy{}} \PYG{l+m+mf}{1.}\PYG{p}{)}

\PYG{c+c1}{\PYGZsh{} tabela}

\PYG{n+nb}{print}\PYG{p}{(}\PYG{l+s+s2}{\PYGZdq{}}\PYG{l+s+s2}{Imprimindo comparação...}\PYG{l+s+se}{\PYGZbs{}n}\PYG{l+s+se}{\PYGZbs{}n}\PYG{l+s+s2}{ ti |    ei   |   Ei}\PYG{l+s+se}{\PYGZbs{}n}\PYG{l+s+s2}{\PYGZdq{}}\PYG{p}{)}
\PYG{k}{for} \PYG{n}{i} \PYG{o+ow}{in} \PYG{n+nb}{range}\PYG{p}{(}\PYG{n+nb}{len}\PYG{p}{(}\PYG{n}{y}\PYG{p}{)}\PYG{p}{)}\PYG{p}{:} 
    \PYG{n+nb}{print}\PYG{p}{(}\PYG{l+s+s2}{\PYGZdq{}}\PYG{l+s+si}{\PYGZob{}0:.1f\PYGZcb{}}\PYG{l+s+s2}{ | }\PYG{l+s+si}{\PYGZob{}1:0.5f\PYGZcb{}}\PYG{l+s+s2}{ | }\PYG{l+s+si}{\PYGZob{}2:0.5f\PYGZcb{}}\PYG{l+s+se}{\PYGZbs{}n}\PYG{l+s+s2}{\PYGZdq{}}\PYG{o}{.}\PYG{n}{format}\PYG{p}{(}\PYG{n}{t}\PYG{p}{[}\PYG{n}{i}\PYG{p}{]}\PYG{p}{,}\PYG{n}{erro}\PYG{p}{[}\PYG{n}{i}\PYG{p}{]}\PYG{p}{,}\PYG{n}{lim\PYGZus{}erro}\PYG{p}{[}\PYG{n}{i}\PYG{p}{]}\PYG{p}{)}\PYG{p}{)}
\end{sphinxVerbatim}

\begin{sphinxVerbatim}[commandchars=\\\{\}]
Imprimindo comparação...

 ti |    ei   |   Ei

0.0 | 0.00000 | 0.00000

0.2 | 0.02930 | 0.03752

0.4 | 0.06209 | 0.08334

0.6 | 0.09854 | 0.13931

0.8 | 0.13875 | 0.20767

1.0 | 0.18268 | 0.29117

1.2 | 0.23013 | 0.39315

1.4 | 0.28063 | 0.51771

1.6 | 0.33336 | 0.66985

1.8 | 0.38702 | 0.85568

2.0 | 0.43969 | 1.08264
\end{sphinxVerbatim}


\section{Problemas}
\label{\detokenize{aula-22-metodo-euler:problemas}}\begin{enumerate}
\sphinxsetlistlabels{\arabic}{enumi}{enumii}{}{.}%
\item {} 
\sphinxAtStartPar
Resolva os seguintes problemas usando o método de Euler com passos \(h = 0.2,0.1,0.05\). Compute o erro relativo usando a solução exata \(y(t)\). Para valores selecionados de \(t\), observe a razão com que o erro diminui à medida que \(h\) é reduzido pela metade.
a. \(y'(t) = [\cos(y(t))]^2, 0 \leq t \leq 10, \quad y(0) = 0; \quad y(t) = \tan^{-1}(t)\)

\end{enumerate}

\sphinxAtStartPar
b. \(y'(t) = \frac{1}{1+t^2} − 2[y(t)]^2, 0 \leq t \leq 10, \quad  y(0) = 0; \quad y(t) = \frac{t}{1+t^2}\)

\sphinxAtStartPar
c. \(y'(t) = \frac{1}{4}y(t)\left[1 - \frac{1}{20}y(t)􏰍\right], 0 \leq t \leq 20, \quad  y(0) = 1; \quad y(t) = \frac{20}{1 + 19e^{-t/4}}\)

\sphinxAtStartPar
d. \(y'(t)= -[y(t)]^2, y(t) = \frac{1}{t}, 1 \leq t \leq 10, \quad  y(1) = 1 \quad y(t) = \frac{1}{t}\)
\begin{enumerate}
\sphinxsetlistlabels{\arabic}{enumi}{enumii}{}{.}%
\item {} 
\sphinxAtStartPar
Considere o problema linear

\end{enumerate}
\begin{equation*}
\begin{split}y'(t) = \lambda y(t) + (1 - \lambda) \cos(t) - (1 + \lambda) \textrm{sen}(t), \quad y(0) = 1.\end{split}
\end{equation*}
\sphinxAtStartPar
A solução exata é \(y(t) = \textrm{sen}(t) + \cos(t)\). Resolva este problema usando o método de Euler com vários valores de \(\lambda\) e \(h\), para \(0 \leq t \leq 10\). Comente sobre os resultados.

\sphinxAtStartPar
a. \(\lambda = −1; \ \ h = 0.5, 0.25, 0.125\).

\sphinxAtStartPar
b. \(\lambda =1; \ \ h=0.5,0.25,0.125.\)

\sphinxAtStartPar
c. \(\lambda = −5; h = 0.5, 0.25, 0.125, 0.0625.\)

\sphinxAtStartPar
d. \(\lambda = 5; h = 0.125, 0.0625.\)
\begin{enumerate}
\sphinxsetlistlabels{\arabic}{enumi}{enumii}{}{.}%
\item {} 
\sphinxAtStartPar
Faça uma análise do erro obtido pelo método de Euler ao ser resolver o caso a. do problema 2, \(h = 0.25\).

\end{enumerate}


\chapter{Métodos de Taylor de Ordem Superior}
\label{\detokenize{aula-23-taylor-rungeKutta:metodos-de-taylor-de-ordem-superior}}\label{\detokenize{aula-23-taylor-rungeKutta::doc}}
\sphinxAtStartPar
Métodos que usam o desenvolvimento em série de Taylor de \(y(t)\) teoricamente fornecem solução para qualquer ED. Sob o ponto de vista computacional, Métodos de Taylor de ordem mais elevada são inaceitáveis, pois, exceto uma classe restrita de funções, o cálculo das derivadas totais é complicado.

\sphinxAtStartPar
Estes métodos são obtidos retendo\sphinxhyphen{}se termos de ordem superior na série de Taylor. Por sua vez, o Método de Euler é um caso particular, como veremos a seguir, já que tem os termos de ordem \(\geq 2\) truncados na série.

\sphinxAtStartPar
Suponhamos que \(y(t)\) a solução para o PVI
\$\(\begin{cases}
y' = f(t,y) \\
a \leq t \leq b \\
y(a) = \alpha
\end{cases}\)\(
é de classe \)\textbackslash{}mathcal\{C\}\textasciicircum{}\{n+1\}\(. Haja vista que a série de Taylor de \)y(t)\( em relação ao ponto \)t\_i\( pode ser expandida até a ordem \)n\$ como
\begin{equation*}
y(t_{i+1}) = y(t_i) + {h} y'(t) + \frac{ {h}^2 }{2} y''(t_i) + \dots + \frac{ {h}^n }{n!} y^{(n)}(t_i) + \frac{ {h}^{n+1} }{(n+1)!} y^{(n+1)}(\xi_i), 
\end{equation*}
\sphinxAtStartPar
para \(\xi_i \in (t_i,t_{i+1})\).

\sphinxAtStartPar
A diferenciação sucessiva de \(y(t)\) fornece
\begin{align*}
y'(t) &= f(t,y(t)) \\
y''(t) &= f'(t,y(t)) \\
		& \vdots  \\
y^k(t) &= f^{(k-1)}(t,y(t))
\end{align*}
\sphinxAtStartPar
Substituindo\sphinxhyphen{}as na série de Taylor, temos:
\begin{equation*}
y(t_{i+1}) = y(t_i) + h f(t_i,y(t_i)) + \frac{ h^2 }{2} f'(t_i,y(t_i)) + \dots + 
\frac{ h^n }{n!} f^{(n-1)}(t_i,y(t_i)) + \frac{ h^{n+1} }{(n+1)!} f^{(n)}(\xi_i,y(\xi_i))
\end{equation*}
\sphinxAtStartPar
Excluindo o termo de resto envolvendo \(\xi\), obtemos o \sphinxstylestrong{Método de Taylor de ordem \(n\)} através do seguinte processo iterativo
\begin{align*}
w_0 &= \alpha \\
w_{i+1} &= w_i + T^{(n)}(t_i,w_i), \quad i = 0,1,2, \ldots, N
\end{align*}
\sphinxAtStartPar
em que
\begin{equation*}
\begin{split}T^{(n)}(t_i,w_i) = f(t_i,w_i) + \frac{h}{2} f'(t_i,y(t_i)) + \dots + \frac{ h^{n-1} }{n!}f^{(n-1)}(t_i,w_i).\end{split}
\end{equation*}
\sphinxAtStartPar
Logo, vemos que o método de Euler é o Método de Taylor de ordem 1. As derivadas sucessivas podem ser calculadas pela Regra da Cadeia. Por exemplo,
\begin{equation*}
\begin{split}y' = f(t,y) \Rightarrow  y'' = f_t(t,y) + f_y(t,y) y' = f_t(t,y) + f_y(t,y) f(t,y).\end{split}
\end{equation*}
\sphinxAtStartPar
Todavia, os métodos da família Runge\sphinxhyphen{}Kutta são alternativas numéricas melhores para métodos de Taylor pois simulam o efeito das derivadas a partir de cálculos médios que não necessitam de derivadas analíticas. Estudaremos métodos de Runge\sphinxhyphen{}Kutta em breve.

\sphinxAtStartPar
\sphinxstylestrong{Exemplo:} Aplique o Método de Taylor de ordem 2 ao PVI
\begin{equation*}
\begin{split}\begin{cases}
y'= y − t^2 + 1 \\
0 \leq t \leq 2 \\
y(0) = 0.5.
\end{cases}\end{split}
\end{equation*}
\sphinxAtStartPar
Para o método de ordem 2, precisamos da primeira derivada de \(f(t,y(t)) = y(t) − t^2 + 1\) em relação a \(t\). Então,
\begin{equation*}
\begin{split}f'(t,y(t)) = \frac{d}{dt} (y − t^2 + 1) = y' − 2t = (y − t^2 + 1) − 2t,\end{split}
\end{equation*}
\sphinxAtStartPar
de modo que
\begin{equation*}
\begin{split}T^{(2)}(t_i,w_i) = 
f(t_i,w_i)+ \frac{h}{2}f'(t_i,w_i) = w_i − t_i^2 + 1 + \frac{h}{2}(w_i − t_i^2 + 1 − 2t_i) = \\
= \left( 1 + \frac{h}{2} \right) (w_i − t_i^2 + 1) − ht_i\end{split}
\end{equation*}
\sphinxAtStartPar
Como \(N = 10\), temos \(h = 0.2\) e \(t_i = 0.2i, \forall i = 1,2,...,10\). Assim, o método de segunda ordem torna\sphinxhyphen{}se
\begin{equation*}
\begin{split}w_0 = 0.5 \\
w_{i+1} = w_i + h \left[ \left(1 + \frac{h}{2} \right)(w_i − t_i^2 + 1) − h t_i \right] = \\
= w_i + 0.2 \left[ \left(1 + \frac{0.2}{2} \right) (w_i − 0.04i^2 + 1) −0.04i \right] \\
= 1.22w_i − 0.0088i^2 − 0.008i + 0.22.\end{split}
\end{equation*}
\sphinxAtStartPar
Os dois primeiros passos dão a aproximação:
\begin{equation*}
\begin{split}y(0.2) \approx w_1 = 1.22(0.5) − 0.0088(0)^2 − 0.008(0) + 0.22 = 0.83\end{split}
\end{equation*}\begin{equation*}
\begin{split}y(0.4) \approx w_2 = 1.22(0.83) − 0.0088(0.2)^2 − 0.008(0.2) + 0.22 = 1.2158\end{split}
\end{equation*}\begin{equation*}
\begin{split}(\ldots)\end{split}
\end{equation*}
\sphinxAtStartPar
Os demais passos seguem da mesma forma


\section{Métodos de Runge\sphinxhyphen{}Kutta}
\label{\detokenize{aula-23-taylor-rungeKutta:metodos-de-runge-kutta}}
\sphinxAtStartPar
O objetivo principal dos métodos de Runge\sphinxhyphen{}Kutta (RK) é imitar o comportamento de \(f(t,y)\) avaliando\sphinxhyphen{}a em vários pontos “abstratos” dentro de um mesmo passo numérico.

\sphinxAtStartPar
Esquemas do tipo RK são usados para reter precisão e substituir aproximações de baixa ou alta ordem via séries de Taylor. São populares na resolução de PVIs e mais simples de programar do que os métodos de Taylor.


\subsection{Forma geral}
\label{\detokenize{aula-23-taylor-rungeKutta:forma-geral}}
\sphinxAtStartPar
A forma geral de um método RK é dada por:
\begin{equation*}
\begin{split}\begin{cases}
w_{n+1} &=& w_n + hF(t_n,w_n;h), \ \ n \geq 0 \\
w_0 &=& y_0
\end{cases}\end{split}
\end{equation*}
\sphinxAtStartPar
O termo \(F(t_n,w_n;h)\) representa uma \sphinxstyleemphasis{inclinação média}, de maneira que, informalmente, métodos RK seja refraseados como:

\sphinxAtStartPar
\sphinxstyleemphasis{valor futuro = valor atual + passo x inclinação média}.

\sphinxAtStartPar
Se fizéssemos uma analogia com o movimento uniforme da física, \(w\) seria uma posição do espaço, \(h\) o tempo e \(F\) a velocidade, resultando em

\sphinxAtStartPar
\sphinxstyleemphasis{posição final = posição inicial + tempo x velocidade}

\sphinxAtStartPar
ou, equivalentemente, \(s_f = s_0 + vt\).


\subsection{Métodos de Runge\sphinxhyphen{}Kutta de 2a. ordem}
\label{\detokenize{aula-23-taylor-rungeKutta:metodos-de-runge-kutta-de-2a-ordem}}
\sphinxAtStartPar
Para métodos RK2, a inclinação média é dada pela expressão
\begin{equation*}
\begin{split}F(t,w;h) = b_1 f(t,w) + b_2 f(t+\alpha h, w + \beta h f(t,w),\end{split}
\end{equation*}
\sphinxAtStartPar
em que \(\alpha, \beta, b_1, b_2 \in \mathbb{R}\) são constantes a serem determinadas de modo que atinjamos um erro de truncamento
\begin{equation*}
\begin{split}T_{n+1}(w) = w_{n+1} - [ w_n + h F(t_n,w_n;h) ] \equiv \mathcal{O}(h^3),\end{split}
\end{equation*}
\sphinxAtStartPar
i.e., seja de terceira ordem.

\sphinxAtStartPar
Abaixo, vamos resolver o PVI:

\sphinxAtStartPar
\textbackslash{}begin\{cases\}
y’ = \sphinxhyphen{}1.2y + 7e\textasciicircum{}(\sphinxhyphen{}0.3x) \textbackslash{}
y(0) = 3 \textbackslash{}
0 < x \textbackslash{}leq 2 \textbackslash{}
h = 1,0.5,0.25,0.1
\textbackslash{}end\{cases\}

\begin{sphinxVerbatim}[commandchars=\\\{\}]
\PYG{c+c1}{\PYGZsh{} definição}
\PYG{n}{e} \PYG{o}{=} \PYG{n}{np}\PYG{o}{.}\PYG{n}{exp}\PYG{p}{(}\PYG{l+m+mi}{1}\PYG{p}{)}

\PYG{c+c1}{\PYGZsh{} dados de entrada }
\PYG{n}{a} \PYG{o}{=} \PYG{l+m+mi}{0}
\PYG{n}{b} \PYG{o}{=} \PYG{l+m+mf}{3.0}
\PYG{n}{ns} \PYG{o}{=} \PYG{p}{[}\PYG{l+m+mi}{4}\PYG{p}{,}\PYG{l+m+mi}{7}\PYG{p}{,}\PYG{l+m+mi}{13}\PYG{p}{,}\PYG{l+m+mi}{31}\PYG{p}{]}
\PYG{n}{w0} \PYG{o}{=} \PYG{l+m+mi}{3}
\PYG{n}{ode} \PYG{o}{=} \PYG{l+s+s1}{\PYGZsq{}}\PYG{l+s+s1}{\PYGZhy{}1.2*y + 7*e**(\PYGZhy{}0.3*x)}\PYG{l+s+s1}{\PYGZsq{}}

\PYG{c+c1}{\PYGZsh{} soluções numéricas}

\PYG{k}{for} \PYG{n}{n} \PYG{o+ow}{in} \PYG{n}{ns}\PYG{p}{:}
    \PYG{c+c1}{\PYGZsh{} MEE}
    \PYG{n}{x}\PYG{p}{,}\PYG{n}{we} \PYG{o}{=} \PYG{n}{ode\PYGZus{}euler\PYGZus{}expl}\PYG{p}{(}\PYG{n}{ode}\PYG{p}{,}\PYG{n}{a}\PYG{p}{,}\PYG{n}{b}\PYG{p}{,}\PYG{n}{n}\PYG{p}{,}\PYG{n}{w0}\PYG{p}{)}

    \PYG{c+c1}{\PYGZsh{} MEM}
    \PYG{n}{x}\PYG{p}{,}\PYG{n}{wm} \PYG{o}{=} \PYG{n}{ode\PYGZus{}euler\PYGZus{}mod}\PYG{p}{(}\PYG{n}{ode}\PYG{p}{,}\PYG{n}{a}\PYG{p}{,}\PYG{n}{b}\PYG{p}{,}\PYG{n}{n}\PYG{p}{,}\PYG{n}{w0}\PYG{p}{)}

    \PYG{c+c1}{\PYGZsh{} conversao de dados}
    \PYG{n}{x} \PYG{o}{=} \PYG{n}{np}\PYG{o}{.}\PYG{n}{asarray}\PYG{p}{(}\PYG{n}{x}\PYG{p}{)}
    \PYG{n}{we} \PYG{o}{=} \PYG{n}{np}\PYG{o}{.}\PYG{n}{asarray}\PYG{p}{(}\PYG{n}{we}\PYG{p}{)}
    \PYG{n}{wm} \PYG{o}{=} \PYG{n}{np}\PYG{o}{.}\PYG{n}{asarray}\PYG{p}{(}\PYG{n}{wm}\PYG{p}{)}

    \PYG{c+c1}{\PYGZsh{} solução exata}
    \PYG{n}{y} \PYG{o}{=} \PYG{l+m+mi}{70}\PYG{o}{/}\PYG{l+m+mi}{9}\PYG{o}{*}\PYG{n}{e}\PYG{o}{*}\PYG{o}{*}\PYG{p}{(}\PYG{o}{\PYGZhy{}}\PYG{l+m+mf}{0.3}\PYG{o}{*}\PYG{n}{x}\PYG{p}{)} \PYG{o}{\PYGZhy{}} \PYG{l+m+mi}{43}\PYG{o}{/}\PYG{l+m+mi}{9}\PYG{o}{*}\PYG{n}{e}\PYG{o}{*}\PYG{o}{*}\PYG{p}{(}\PYG{o}{\PYGZhy{}}\PYG{l+m+mf}{1.2}\PYG{o}{*}\PYG{n}{x}\PYG{p}{)}

    \PYG{c+c1}{\PYGZsh{} curvas}
    \PYG{n}{plt}\PYG{o}{.}\PYG{n}{figure}\PYG{p}{(}\PYG{p}{)}
    \PYG{n}{plt}\PYG{o}{.}\PYG{n}{plot}\PYG{p}{(}\PYG{n}{x}\PYG{p}{,}\PYG{n}{y}\PYG{p}{,}\PYG{l+s+s1}{\PYGZsq{}}\PYG{l+s+s1}{r}\PYG{l+s+s1}{\PYGZsq{}}\PYG{p}{,}\PYG{n}{label}\PYG{o}{=}\PYG{l+s+s1}{\PYGZsq{}}\PYG{l+s+s1}{exata}\PYG{l+s+s1}{\PYGZsq{}}\PYG{p}{)}
    \PYG{n}{plt}\PYG{o}{.}\PYG{n}{plot}\PYG{p}{(}\PYG{n}{x}\PYG{p}{,}\PYG{n}{we}\PYG{p}{,}\PYG{l+s+s1}{\PYGZsq{}}\PYG{l+s+s1}{bo}\PYG{l+s+s1}{\PYGZsq{}}\PYG{p}{,}\PYG{n}{label}\PYG{o}{=}\PYG{l+s+s1}{\PYGZsq{}}\PYG{l+s+s1}{Eu expl}\PYG{l+s+s1}{\PYGZsq{}}\PYG{p}{)}
    \PYG{n}{plt}\PYG{o}{.}\PYG{n}{plot}\PYG{p}{(}\PYG{n}{x}\PYG{p}{,}\PYG{n}{wm}\PYG{p}{,}\PYG{l+s+s1}{\PYGZsq{}}\PYG{l+s+s1}{go}\PYG{l+s+s1}{\PYGZsq{}}\PYG{p}{,}\PYG{n}{label}\PYG{o}{=}\PYG{l+s+s1}{\PYGZsq{}}\PYG{l+s+s1}{Eu mod}\PYG{l+s+s1}{\PYGZsq{}}\PYG{p}{)}
    \PYG{n}{plt}\PYG{o}{.}\PYG{n}{legend}\PYG{p}{(}\PYG{p}{)}
    \PYG{n}{tit} \PYG{o}{=} \PYG{l+s+s1}{\PYGZsq{}}\PYG{l+s+s1}{\PYGZdl{}h = }\PYG{l+s+s1}{\PYGZsq{}} \PYG{o}{+} \PYG{n+nb}{str}\PYG{p}{(}\PYG{p}{(}\PYG{n}{b}\PYG{o}{\PYGZhy{}}\PYG{n}{a}\PYG{p}{)}\PYG{o}{/}\PYG{p}{(}\PYG{n}{n}\PYG{o}{\PYGZhy{}}\PYG{l+m+mi}{1}\PYG{p}{)}\PYG{p}{)} \PYG{o}{+} \PYG{l+s+s1}{\PYGZsq{}}\PYG{l+s+s1}{\PYGZdl{}}\PYG{l+s+s1}{\PYGZsq{}}
    \PYG{n}{plt}\PYG{o}{.}\PYG{n}{title}\PYG{p}{(}\PYG{l+s+s1}{\PYGZsq{}}\PYG{l+s+s1}{\PYGZdl{}h=\PYGZdl{}}\PYG{l+s+s1}{\PYGZsq{}}\PYG{p}{)}
    \PYG{n}{plt}\PYG{o}{.}\PYG{n}{grid}\PYG{p}{(}\PYG{p}{)}
\end{sphinxVerbatim}

\noindent\sphinxincludegraphics{{aula-23-taylor-rungeKutta_3_0}.png}

\noindent\sphinxincludegraphics{{aula-23-taylor-rungeKutta_3_1}.png}

\noindent\sphinxincludegraphics{{aula-23-taylor-rungeKutta_3_2}.png}

\noindent\sphinxincludegraphics{{aula-23-taylor-rungeKutta_3_3}.png}


\part{Code sessions}


\chapter{Code session 1}
\label{\detokenize{codeSession-1-bisect:code-session-1}}\label{\detokenize{codeSession-1-bisect::doc}}
\begin{sphinxVerbatim}[commandchars=\\\{\}]
\PYG{o}{\PYGZpc{}}\PYG{k}{matplotlib} inline
\PYG{k+kn}{import} \PYG{n+nn}{numpy} \PYG{k}{as} \PYG{n+nn}{np} 
\PYG{k+kn}{import} \PYG{n+nn}{matplotlib}\PYG{n+nn}{.}\PYG{n+nn}{pyplot} \PYG{k}{as} \PYG{n+nn}{plt} 
\end{sphinxVerbatim}


\section{Determinação de raízes}
\label{\detokenize{codeSession-1-bisect:determinacao-de-raizes}}
\sphinxAtStartPar
\sphinxcode{\sphinxupquote{bisect}}

\sphinxAtStartPar
A função \sphinxcode{\sphinxupquote{bisect}} localiza a raiz de uma função dentro de um intervalo dado usando o método da bisseção.
Os argumentos de entrada obrigatórios desta função são:
\begin{enumerate}
\sphinxsetlistlabels{\arabic}{enumi}{enumii}{}{.}%
\item {} 
\sphinxAtStartPar
a função\sphinxhyphen{}alvo \sphinxcode{\sphinxupquote{f}} (contínua)

\item {} 
\sphinxAtStartPar
o limite esquerdo \sphinxcode{\sphinxupquote{a}}

\item {} 
\sphinxAtStartPar
o limite direito \sphinxcode{\sphinxupquote{b}}

\end{enumerate}

\sphinxAtStartPar
Parâmetros opcionais relevantes são:
\begin{itemize}
\item {} 
\sphinxAtStartPar
\sphinxcode{\sphinxupquote{xtol}}: tolerância (padrão: 2e\sphinxhyphen{}12)

\item {} 
\sphinxAtStartPar
\sphinxcode{\sphinxupquote{maxiter}}: número máximo de iterações (padrão: 100)

\item {} 
\sphinxAtStartPar
\sphinxcode{\sphinxupquote{disp}}: mostra erro se algoritmo não convergir (padrão: True)

\end{itemize}

\sphinxAtStartPar
O argumento de saída é:
\begin{itemize}
\item {} 
\sphinxAtStartPar
\sphinxcode{\sphinxupquote{x0}}: a estimativa para a raiz de \sphinxcode{\sphinxupquote{f}}

\end{itemize}

\sphinxAtStartPar
Como importá\sphinxhyphen{}la?

\begin{sphinxVerbatim}[commandchars=\\\{\}]
\PYG{k+kn}{from} \PYG{n+nn}{scipy}\PYG{n+nn}{.}\PYG{n+nn}{optimize} \PYG{k+kn}{import} \PYG{n}{bisect}
\end{sphinxVerbatim}

\begin{sphinxVerbatim}[commandchars=\\\{\}]
\PYG{k+kn}{from} \PYG{n+nn}{scipy}\PYG{n+nn}{.}\PYG{n+nn}{optimize} \PYG{k+kn}{import} \PYG{n}{bisect}
\end{sphinxVerbatim}


\subsection{Problema 1}
\label{\detokenize{codeSession-1-bisect:problema-1}}
\sphinxAtStartPar
Encontre a menor raiz positiva (real) de \(x^{3} - 3.23x^{2} - 5.54x + 9.84 = 0\)  pelo método da bisseção.


\subsubsection{Resolução}
\label{\detokenize{codeSession-1-bisect:resolucao}}
\begin{sphinxVerbatim}[commandchars=\\\{\}]
\PYG{c+c1}{\PYGZsh{} função}
\PYG{k}{def} \PYG{n+nf}{f}\PYG{p}{(}\PYG{n}{x}\PYG{p}{)}\PYG{p}{:} 
    \PYG{k}{return} \PYG{n}{x}\PYG{o}{*}\PYG{o}{*}\PYG{l+m+mi}{3} \PYG{o}{\PYGZhy{}} \PYG{l+m+mf}{3.23}\PYG{o}{*}\PYG{n}{x}\PYG{o}{*}\PYG{o}{*}\PYG{l+m+mi}{2} \PYG{o}{\PYGZhy{}} \PYG{l+m+mf}{5.54}\PYG{o}{*}\PYG{n}{x} \PYG{o}{+} \PYG{l+m+mf}{9.84}
\end{sphinxVerbatim}

\begin{sphinxVerbatim}[commandchars=\\\{\}]
\PYG{c+c1}{\PYGZsh{} analise gráfica }
\PYG{n}{x} \PYG{o}{=} \PYG{n}{np}\PYG{o}{.}\PYG{n}{linspace}\PYG{p}{(}\PYG{o}{\PYGZhy{}}\PYG{l+m+mi}{4}\PYG{p}{,}\PYG{l+m+mi}{5}\PYG{p}{)}
\PYG{n}{plt}\PYG{o}{.}\PYG{n}{plot}\PYG{p}{(}\PYG{n}{x}\PYG{p}{,}\PYG{n}{f}\PYG{p}{(}\PYG{n}{x}\PYG{p}{)}\PYG{p}{)}\PYG{p}{;}
\PYG{n}{plt}\PYG{o}{.}\PYG{n}{axhline}\PYG{p}{(}\PYG{n}{y}\PYG{o}{=}\PYG{l+m+mi}{0}\PYG{p}{,}\PYG{n}{color}\PYG{o}{=}\PYG{l+s+s1}{\PYGZsq{}}\PYG{l+s+s1}{r}\PYG{l+s+s1}{\PYGZsq{}}\PYG{p}{)}\PYG{p}{;}
\PYG{n}{plt}\PYG{o}{.}\PYG{n}{axvline}\PYG{p}{(}\PYG{n}{x}\PYG{o}{=}\PYG{l+m+mi}{0}\PYG{p}{,}\PYG{n}{color}\PYG{o}{=}\PYG{l+s+s1}{\PYGZsq{}}\PYG{l+s+s1}{r}\PYG{l+s+s1}{\PYGZsq{}}\PYG{p}{)}\PYG{p}{;}
\end{sphinxVerbatim}

\noindent\sphinxincludegraphics{{codeSession-1-bisect_7_0}.png}

\sphinxAtStartPar
Pelo gráfico, vemos que a menor raiz positiva está localizada no intervalo \((0,2]\). Vamos determiná\sphinxhyphen{}la utilizando este intervalo de confinamento.

\begin{sphinxVerbatim}[commandchars=\\\{\}]
\PYG{c+c1}{\PYGZsh{} resolução com bisect }

\PYG{n}{x} \PYG{o}{=} \PYG{n}{bisect}\PYG{p}{(}\PYG{n}{f}\PYG{p}{,}\PYG{l+m+mi}{0}\PYG{p}{,}\PYG{l+m+mi}{2}\PYG{p}{)} \PYG{c+c1}{\PYGZsh{} raiz }

\PYG{n+nb}{print}\PYG{p}{(}\PYG{l+s+s1}{\PYGZsq{}}\PYG{l+s+s1}{Raiz: x = }\PYG{l+s+si}{\PYGZpc{}f}\PYG{l+s+s1}{\PYGZsq{}} \PYG{o}{\PYGZpc{}} \PYG{n}{x}\PYG{p}{)}
\end{sphinxVerbatim}

\begin{sphinxVerbatim}[commandchars=\\\{\}]
Raiz: x = 1.230000
\end{sphinxVerbatim}


\subsection{Problema 2}
\label{\detokenize{codeSession-1-bisect:problema-2}}
\sphinxAtStartPar
Determine a menor raiz não nula positiva de \(\cosh(x) \cos(x) - 1 = 0\) dentro do intervalo \((4,5)\).


\subsubsection{Resolução}
\label{\detokenize{codeSession-1-bisect:id1}}
\sphinxAtStartPar
Sigamos o procedimento aprendido com \sphinxcode{\sphinxupquote{bisect}}.

\begin{sphinxVerbatim}[commandchars=\\\{\}]
\PYG{c+c1}{\PYGZsh{} função}
\PYG{k}{def} \PYG{n+nf}{f}\PYG{p}{(}\PYG{n}{x}\PYG{p}{)}\PYG{p}{:} 
    \PYG{k}{return} \PYG{n}{np}\PYG{o}{.}\PYG{n}{cosh}\PYG{p}{(}\PYG{n}{x}\PYG{p}{)}\PYG{o}{*}\PYG{n}{np}\PYG{o}{.}\PYG{n}{cos}\PYG{p}{(}\PYG{n}{x}\PYG{p}{)} \PYG{o}{\PYGZhy{}} \PYG{l+m+mi}{1} 
\end{sphinxVerbatim}

\begin{sphinxVerbatim}[commandchars=\\\{\}]
\PYG{c+c1}{\PYGZsh{} analise gráfica }
\PYG{n}{x} \PYG{o}{=} \PYG{n}{np}\PYG{o}{.}\PYG{n}{linspace}\PYG{p}{(}\PYG{l+m+mi}{4}\PYG{p}{,}\PYG{l+m+mi}{5}\PYG{p}{)}
\PYG{n}{plt}\PYG{o}{.}\PYG{n}{plot}\PYG{p}{(}\PYG{n}{x}\PYG{p}{,}\PYG{n}{f}\PYG{p}{(}\PYG{n}{x}\PYG{p}{)}\PYG{p}{)}\PYG{p}{;}
\PYG{n}{plt}\PYG{o}{.}\PYG{n}{axhline}\PYG{p}{(}\PYG{n}{y}\PYG{o}{=}\PYG{l+m+mi}{0}\PYG{p}{,}\PYG{n}{color}\PYG{o}{=}\PYG{l+s+s1}{\PYGZsq{}}\PYG{l+s+s1}{r}\PYG{l+s+s1}{\PYGZsq{}}\PYG{p}{)}\PYG{p}{;}
\end{sphinxVerbatim}

\noindent\sphinxincludegraphics{{codeSession-1-bisect_13_0}.png}

\begin{sphinxVerbatim}[commandchars=\\\{\}]
\PYG{c+c1}{\PYGZsh{} resolução com bisect }

\PYG{n}{x} \PYG{o}{=} \PYG{n}{bisect}\PYG{p}{(}\PYG{n}{f}\PYG{p}{,}\PYG{l+m+mi}{4}\PYG{p}{,}\PYG{l+m+mi}{5}\PYG{p}{)} \PYG{c+c1}{\PYGZsh{} raiz }

\PYG{n+nb}{print}\PYG{p}{(}\PYG{l+s+s1}{\PYGZsq{}}\PYG{l+s+s1}{Raiz: x = }\PYG{l+s+si}{\PYGZpc{}f}\PYG{l+s+s1}{\PYGZsq{}} \PYG{o}{\PYGZpc{}} \PYG{n}{x}\PYG{p}{)}
\end{sphinxVerbatim}

\begin{sphinxVerbatim}[commandchars=\\\{\}]
Raiz: x = 4.730041
\end{sphinxVerbatim}


\subsection{Problema 3}
\label{\detokenize{codeSession-1-bisect:problema-3}}
\sphinxAtStartPar
Uma raiz da equação \(\tan(x) - \tanh(x) = 0\) encontra\sphinxhyphen{}se em \((7.0,7.4)\). Determine esta raiz com três casas decimais de precisão pelo método da bisseção.


\subsubsection{Resolução}
\label{\detokenize{codeSession-1-bisect:id2}}
\begin{sphinxVerbatim}[commandchars=\\\{\}]
\PYG{c+c1}{\PYGZsh{} função}
\PYG{k}{def} \PYG{n+nf}{f}\PYG{p}{(}\PYG{n}{x}\PYG{p}{)}\PYG{p}{:} 
    \PYG{k}{return} \PYG{n}{np}\PYG{o}{.}\PYG{n}{tan}\PYG{p}{(}\PYG{n}{x}\PYG{p}{)} \PYG{o}{\PYGZhy{}} \PYG{n}{np}\PYG{o}{.}\PYG{n}{tanh}\PYG{p}{(}\PYG{n}{x}\PYG{p}{)}
\end{sphinxVerbatim}

\begin{sphinxVerbatim}[commandchars=\\\{\}]
\PYG{c+c1}{\PYGZsh{} analise gráfica }
\PYG{n}{x} \PYG{o}{=} \PYG{n}{np}\PYG{o}{.}\PYG{n}{linspace}\PYG{p}{(}\PYG{l+m+mi}{7}\PYG{p}{,}\PYG{l+m+mf}{7.4}\PYG{p}{)}
\PYG{n}{plt}\PYG{o}{.}\PYG{n}{plot}\PYG{p}{(}\PYG{n}{x}\PYG{p}{,}\PYG{n}{f}\PYG{p}{(}\PYG{n}{x}\PYG{p}{)}\PYG{p}{)}\PYG{p}{;}
\PYG{n}{plt}\PYG{o}{.}\PYG{n}{axhline}\PYG{p}{(}\PYG{n}{y}\PYG{o}{=}\PYG{l+m+mi}{0}\PYG{p}{,}\PYG{n}{color}\PYG{o}{=}\PYG{l+s+s1}{\PYGZsq{}}\PYG{l+s+s1}{r}\PYG{l+s+s1}{\PYGZsq{}}\PYG{p}{)}\PYG{p}{;}
\end{sphinxVerbatim}

\noindent\sphinxincludegraphics{{codeSession-1-bisect_18_0}.png}

\sphinxAtStartPar
Para obter as 3 casas decimas, vamos imprimir o valor final com 3 casas decimais.

\begin{sphinxVerbatim}[commandchars=\\\{\}]
\PYG{n}{x} \PYG{o}{=} \PYG{n}{bisect}\PYG{p}{(}\PYG{n}{f}\PYG{p}{,}\PYG{l+m+mi}{7}\PYG{p}{,}\PYG{l+m+mf}{7.4}\PYG{p}{)} \PYG{c+c1}{\PYGZsh{} raiz }

\PYG{n+nb}{print}\PYG{p}{(}\PYG{l+s+s1}{\PYGZsq{}}\PYG{l+s+s1}{Raiz: x = }\PYG{l+s+si}{\PYGZpc{}.3f}\PYG{l+s+s1}{\PYGZsq{}} \PYG{o}{\PYGZpc{}} \PYG{n}{x}\PYG{p}{)}
\end{sphinxVerbatim}

\begin{sphinxVerbatim}[commandchars=\\\{\}]
Raiz: x = 7.069
\end{sphinxVerbatim}


\subsection{Problema 4}
\label{\detokenize{codeSession-1-bisect:problema-4}}
\sphinxAtStartPar
Determine as raízes de \(\text{sen}(x) + 3\cos(x) - 1 = 0\) no intervalo \((-2,2)\).


\subsubsection{Resolução}
\label{\detokenize{codeSession-1-bisect:id3}}
\begin{sphinxVerbatim}[commandchars=\\\{\}]
\PYG{c+c1}{\PYGZsh{} função}
\PYG{k}{def} \PYG{n+nf}{f}\PYG{p}{(}\PYG{n}{x}\PYG{p}{)}\PYG{p}{:} 
    \PYG{k}{return} \PYG{n}{np}\PYG{o}{.}\PYG{n}{sin}\PYG{p}{(}\PYG{n}{x}\PYG{p}{)} \PYG{o}{+} \PYG{n}{np}\PYG{o}{.}\PYG{n}{cos}\PYG{p}{(}\PYG{n}{x}\PYG{p}{)} \PYG{o}{\PYGZhy{}} \PYG{l+m+mi}{1}
\end{sphinxVerbatim}

\begin{sphinxVerbatim}[commandchars=\\\{\}]
\PYG{c+c1}{\PYGZsh{} analise gráfica }
\PYG{n}{x} \PYG{o}{=} \PYG{n}{np}\PYG{o}{.}\PYG{n}{linspace}\PYG{p}{(}\PYG{o}{\PYGZhy{}}\PYG{l+m+mi}{2}\PYG{p}{,}\PYG{l+m+mi}{2}\PYG{p}{)}
\PYG{n}{plt}\PYG{o}{.}\PYG{n}{plot}\PYG{p}{(}\PYG{n}{x}\PYG{p}{,}\PYG{n}{f}\PYG{p}{(}\PYG{n}{x}\PYG{p}{)}\PYG{p}{)}\PYG{p}{;}
\PYG{n}{plt}\PYG{o}{.}\PYG{n}{axhline}\PYG{p}{(}\PYG{n}{y}\PYG{o}{=}\PYG{l+m+mi}{0}\PYG{p}{)}\PYG{p}{;}
\end{sphinxVerbatim}

\noindent\sphinxincludegraphics{{codeSession-1-bisect_24_0}.png}

\sphinxAtStartPar
A análise gráfica mostra duas raízes. Vamos encontrar uma de cada vez.

\begin{sphinxVerbatim}[commandchars=\\\{\}]
\PYG{c+c1}{\PYGZsh{} resolução com bisect }

\PYG{n}{x1} \PYG{o}{=} \PYG{n}{bisect}\PYG{p}{(}\PYG{n}{f}\PYG{p}{,}\PYG{o}{\PYGZhy{}}\PYG{l+m+mi}{2}\PYG{p}{,}\PYG{l+m+mi}{1}\PYG{p}{)} \PYG{c+c1}{\PYGZsh{} raiz 1  }
\PYG{n}{x2} \PYG{o}{=} \PYG{n}{bisect}\PYG{p}{(}\PYG{n}{f}\PYG{p}{,}\PYG{l+m+mi}{1}\PYG{p}{,}\PYG{l+m+mi}{2}\PYG{p}{)} \PYG{c+c1}{\PYGZsh{} raiz 2 }

\PYG{n+nb}{print}\PYG{p}{(}\PYG{l+s+s1}{\PYGZsq{}}\PYG{l+s+s1}{Raízes: x1 = }\PYG{l+s+si}{\PYGZpc{}f}\PYG{l+s+s1}{; x2 = }\PYG{l+s+si}{\PYGZpc{}f}\PYG{l+s+s1}{\PYGZsq{}} \PYG{o}{\PYGZpc{}} \PYG{p}{(}\PYG{n}{x1}\PYG{p}{,}\PYG{n}{x2}\PYG{p}{)}\PYG{p}{)}
\end{sphinxVerbatim}

\begin{sphinxVerbatim}[commandchars=\\\{\}]
Raízes: x1 = \PYGZhy{}0.000000; x2 = 1.570796
\end{sphinxVerbatim}


\subsection{Problema 5}
\label{\detokenize{codeSession-1-bisect:problema-5}}
\sphinxAtStartPar
Determine todas as raízes reais de \(P(x) = x^4 + 0.9x^3 - 2.3x^2 + 3.6x - 25.2\)


\subsubsection{Resolução}
\label{\detokenize{codeSession-1-bisect:id4}}
\begin{sphinxVerbatim}[commandchars=\\\{\}]
\PYG{c+c1}{\PYGZsh{} função }
\PYG{k}{def} \PYG{n+nf}{P}\PYG{p}{(}\PYG{n}{x}\PYG{p}{)}\PYG{p}{:}
    \PYG{k}{return} \PYG{n}{x}\PYG{o}{*}\PYG{o}{*}\PYG{l+m+mi}{4} \PYG{o}{+} \PYG{l+m+mf}{0.9}\PYG{o}{*}\PYG{n}{x}\PYG{o}{*}\PYG{o}{*}\PYG{l+m+mi}{3} \PYG{o}{\PYGZhy{}} \PYG{l+m+mf}{2.3}\PYG{o}{*}\PYG{n}{x}\PYG{o}{*}\PYG{o}{*}\PYG{l+m+mi}{2} \PYG{o}{+} \PYG{l+m+mf}{3.6}\PYG{o}{*}\PYG{n}{x} \PYG{o}{\PYGZhy{}} \PYG{l+m+mf}{25.2}
\end{sphinxVerbatim}

\begin{sphinxVerbatim}[commandchars=\\\{\}]
\PYG{c+c1}{\PYGZsh{} define função para analise gráfica }
\PYG{k}{def} \PYG{n+nf}{analise\PYGZus{}grafica}\PYG{p}{(}\PYG{n}{xrange}\PYG{p}{,}\PYG{n}{f}\PYG{p}{)}\PYG{p}{:}
    \PYG{n}{plt}\PYG{o}{.}\PYG{n}{plot}\PYG{p}{(}\PYG{n}{xrange}\PYG{p}{,}\PYG{n}{f}\PYG{p}{(}\PYG{n}{xrange}\PYG{p}{)}\PYG{p}{)}\PYG{p}{;}
    \PYG{n}{plt}\PYG{o}{.}\PYG{n}{axhline}\PYG{p}{(}\PYG{n}{y}\PYG{o}{=}\PYG{l+m+mi}{0}\PYG{p}{)}\PYG{p}{;}
\end{sphinxVerbatim}

\begin{sphinxVerbatim}[commandchars=\\\{\}]
\PYG{c+c1}{\PYGZsh{} analise gráfica 1}
\PYG{n}{xrange} \PYG{o}{=} \PYG{n}{np}\PYG{o}{.}\PYG{n}{linspace}\PYG{p}{(}\PYG{o}{\PYGZhy{}}\PYG{l+m+mi}{10}\PYG{p}{,}\PYG{l+m+mi}{10}\PYG{p}{)}
\PYG{n}{analise\PYGZus{}grafica}\PYG{p}{(}\PYG{n}{xrange}\PYG{p}{,}\PYG{n}{P}\PYG{p}{)}
\end{sphinxVerbatim}

\noindent\sphinxincludegraphics{{codeSession-1-bisect_31_0}.png}

\begin{sphinxVerbatim}[commandchars=\\\{\}]
\PYG{c+c1}{\PYGZsh{} refinamento}
\PYG{n}{xrange} \PYG{o}{=} \PYG{n}{np}\PYG{o}{.}\PYG{n}{linspace}\PYG{p}{(}\PYG{o}{\PYGZhy{}}\PYG{l+m+mi}{5}\PYG{p}{,}\PYG{l+m+mi}{5}\PYG{p}{)}
\PYG{n}{analise\PYGZus{}grafica}\PYG{p}{(}\PYG{n}{xrange}\PYG{p}{,}\PYG{n}{P}\PYG{p}{)}
\end{sphinxVerbatim}

\noindent\sphinxincludegraphics{{codeSession-1-bisect_32_0}.png}

\begin{sphinxVerbatim}[commandchars=\\\{\}]
\PYG{c+c1}{\PYGZsh{} resolução com bisect }

\PYG{n}{x1} \PYG{o}{=} \PYG{n}{bisect}\PYG{p}{(}\PYG{n}{P}\PYG{p}{,}\PYG{o}{\PYGZhy{}}\PYG{l+m+mi}{4}\PYG{p}{,}\PYG{o}{\PYGZhy{}}\PYG{l+m+mi}{2}\PYG{p}{)} \PYG{c+c1}{\PYGZsh{} raiz 1  }
\PYG{n}{x2} \PYG{o}{=} \PYG{n}{bisect}\PYG{p}{(}\PYG{n}{P}\PYG{p}{,}\PYG{l+m+mi}{1}\PYG{p}{,}\PYG{l+m+mi}{3}\PYG{p}{)} \PYG{c+c1}{\PYGZsh{} raiz 2 }

\PYG{n+nb}{print}\PYG{p}{(}\PYG{l+s+s1}{\PYGZsq{}}\PYG{l+s+s1}{Raízes: x1 = }\PYG{l+s+si}{\PYGZpc{}f}\PYG{l+s+s1}{; x2 = }\PYG{l+s+si}{\PYGZpc{}f}\PYG{l+s+s1}{\PYGZsq{}} \PYG{o}{\PYGZpc{}} \PYG{p}{(}\PYG{n}{x1}\PYG{p}{,}\PYG{n}{x2}\PYG{p}{)}\PYG{p}{)}
\end{sphinxVerbatim}

\begin{sphinxVerbatim}[commandchars=\\\{\}]
Raízes: x1 = \PYGZhy{}3.000000; x2 = 2.100000
\end{sphinxVerbatim}


\subsection{Problema 6}
\label{\detokenize{codeSession-1-bisect:problema-6}}
\sphinxAtStartPar
Um jogador de futebol americano está prestes a fazer um lançamento para outro jogador de seu time. O lançador tem uma altura de 1,82 m e o outro jogador está afastado de 18,2 m. A expressão que descreve o movimento da bola é a familiar equação da física que descreve o movimento de um projétil:
\begin{equation*}
\begin{split}y = x\tan(\theta) - \dfrac{1}{2}\dfrac{x^2 g}{v_0^2}\dfrac{1}{\cos^2(\theta)} + h,\end{split}
\end{equation*}
\sphinxAtStartPar
onde \(x\) e \(y\) são as distâncias horizontal e verical, respectivamente, \(g=9,8 \, m/s^2\) é a aceleração da gravidade, \(v_0\) é a velocidade inicial da bola quando deixa a mão do lançador e \(\theta\) é o Ângulo que a bola faz com o eixo horizontal nesse mesmo instante. Para \(v_0 = 15,2 \, m/s\), \(x = 18,2 \, m\), \(h = 1,82 \, m\) e \(y = 2,1 \, m\), determine o ângulo \(\theta\) no qual o jogador deve lançar a bola.


\subsubsection{Resolução}
\label{\detokenize{codeSession-1-bisect:id5}}
\begin{sphinxVerbatim}[commandchars=\\\{\}]
\PYG{c+c1}{\PYGZsh{} parâmetros do problema}
\PYG{n}{v0} \PYG{o}{=} \PYG{l+m+mf}{15.2}
\PYG{n}{x} \PYG{o}{=} \PYG{l+m+mf}{18.2}
\PYG{n}{h} \PYG{o}{=} \PYG{l+m+mf}{1.82}
\PYG{n}{y} \PYG{o}{=} \PYG{l+m+mf}{2.1}
\PYG{n}{g} \PYG{o}{=} \PYG{l+m+mf}{9.8}

\PYG{c+c1}{\PYGZsh{} função f(theta) = 0}
\PYG{n}{f} \PYG{o}{=} \PYG{k}{lambda} \PYG{n}{theta}\PYG{p}{:} \PYG{n}{x}\PYG{o}{*}\PYG{n}{np}\PYG{o}{.}\PYG{n}{tan}\PYG{p}{(}\PYG{n}{theta}\PYG{p}{)} \PYG{o}{\PYGZhy{}} \PYG{l+m+mf}{0.5}\PYG{o}{*}\PYG{p}{(}\PYG{n}{x}\PYG{o}{*}\PYG{o}{*}\PYG{l+m+mi}{2}\PYG{o}{*}\PYG{n}{g}\PYG{o}{/}\PYG{n}{v0}\PYG{o}{*}\PYG{o}{*}\PYG{l+m+mi}{2}\PYG{p}{)}\PYG{o}{*}\PYG{p}{(}\PYG{l+m+mi}{1}\PYG{o}{/}\PYG{p}{(}\PYG{n}{np}\PYG{o}{.}\PYG{n}{cos}\PYG{p}{(}\PYG{n}{theta}\PYG{p}{)}\PYG{o}{*}\PYG{o}{*}\PYG{l+m+mi}{2}\PYG{p}{)}\PYG{p}{)} \PYG{o}{+} \PYG{n}{h} \PYG{o}{\PYGZhy{}} \PYG{n}{y}
\end{sphinxVerbatim}

\begin{sphinxVerbatim}[commandchars=\\\{\}]
\PYG{c+c1}{\PYGZsh{} análise gráfica}
\PYG{n}{th} \PYG{o}{=} \PYG{n}{np}\PYG{o}{.}\PYG{n}{linspace}\PYG{p}{(}\PYG{l+m+mi}{0}\PYG{p}{,}\PYG{l+m+mf}{0.95}\PYG{o}{*}\PYG{n}{np}\PYG{o}{.}\PYG{n}{pi}\PYG{o}{/}\PYG{l+m+mi}{2}\PYG{p}{,}\PYG{l+m+mi}{50}\PYG{p}{)}
\PYG{n}{plt}\PYG{o}{.}\PYG{n}{plot}\PYG{p}{(}\PYG{n}{th}\PYG{p}{,}\PYG{n}{f}\PYG{p}{(}\PYG{n}{th}\PYG{p}{)}\PYG{p}{)}\PYG{p}{;}
\PYG{n}{plt}\PYG{o}{.}\PYG{n}{axhline}\PYG{p}{(}\PYG{n}{y}\PYG{o}{=}\PYG{l+m+mi}{0}\PYG{p}{,}\PYG{n}{color}\PYG{o}{=}\PYG{l+s+s1}{\PYGZsq{}}\PYG{l+s+s1}{r}\PYG{l+s+s1}{\PYGZsq{}}\PYG{p}{)}\PYG{p}{;}
\end{sphinxVerbatim}

\noindent\sphinxincludegraphics{{codeSession-1-bisect_37_0}.png}

\begin{sphinxVerbatim}[commandchars=\\\{\}]
\PYG{c+c1}{\PYGZsh{} análise gráfica \PYGZhy{} 2}
\PYG{n}{th} \PYG{o}{=} \PYG{n}{np}\PYG{o}{.}\PYG{n}{linspace}\PYG{p}{(}\PYG{l+m+mi}{0}\PYG{p}{,}\PYG{n}{np}\PYG{o}{.}\PYG{n}{pi}\PYG{o}{/}\PYG{l+m+mi}{4}\PYG{p}{,}\PYG{l+m+mi}{50}\PYG{p}{)}
\PYG{n}{plt}\PYG{o}{.}\PYG{n}{plot}\PYG{p}{(}\PYG{n}{th}\PYG{p}{,}\PYG{n}{f}\PYG{p}{(}\PYG{n}{th}\PYG{p}{)}\PYG{p}{)}\PYG{p}{;}
\PYG{n}{plt}\PYG{o}{.}\PYG{n}{axhline}\PYG{p}{(}\PYG{n}{y}\PYG{o}{=}\PYG{l+m+mi}{0}\PYG{p}{,}\PYG{n}{color}\PYG{o}{=}\PYG{l+s+s1}{\PYGZsq{}}\PYG{l+s+s1}{r}\PYG{l+s+s1}{\PYGZsq{}}\PYG{p}{)}\PYG{p}{;}
\end{sphinxVerbatim}

\noindent\sphinxincludegraphics{{codeSession-1-bisect_38_0}.png}

\begin{sphinxVerbatim}[commandchars=\\\{\}]
\PYG{c+c1}{\PYGZsh{} resolução por bisseção}
\PYG{n}{xr} \PYG{o}{=} \PYG{n}{bisect}\PYG{p}{(}\PYG{n}{f}\PYG{p}{,}\PYG{l+m+mf}{0.1}\PYG{p}{,}\PYG{l+m+mf}{0.6}\PYG{p}{)}
\PYG{n+nb}{print}\PYG{p}{(}\PYG{l+s+s1}{\PYGZsq{}}\PYG{l+s+s1}{Ângulo de lançamento: }\PYG{l+s+si}{\PYGZpc{}.2f}\PYG{l+s+s1}{ graus}\PYG{l+s+s1}{\PYGZsq{}} \PYG{o}{\PYGZpc{}} \PYG{n}{np}\PYG{o}{.}\PYG{n}{rad2deg}\PYG{p}{(}\PYG{n}{xr}\PYG{p}{)}\PYG{p}{)}
\end{sphinxVerbatim}

\begin{sphinxVerbatim}[commandchars=\\\{\}]
Ângulo de lançamento: 26.41 graus
\end{sphinxVerbatim}


\subsection{Problema 7}
\label{\detokenize{codeSession-1-bisect:problema-7}}
\sphinxAtStartPar
A equação de Bernoulli para o escoamento de um fluido em um canal aberto com um pequeno ressalto é
\begin{equation*}
\begin{split}\dfrac{Q^2}{2gb^2h_0^2} + h_0 = \dfrac{Q^2}{2gb^2h^2} + h + H,\end{split}
\end{equation*}
\sphinxAtStartPar
onde \(Q = 1.2 \, m^3/s\) é a vazão volumétrica, \(g = 9.81 \, m/s^2\) é a aceleração gravitacional, \(b = 1.8 \, m\) a largura do canal, \(h_0 = 0.6 \, m\) o nível da água à montante, \(H = 0.075 \, m\) a altura do ressalto e \(h\) o nível da água acima do ressalto. Determine \(h\).


\subsubsection{Resolução}
\label{\detokenize{codeSession-1-bisect:id6}}
\sphinxAtStartPar
Para este problema, definiremos duas funções, uma auxiliar, que chamaremos \sphinxcode{\sphinxupquote{a}}, e a função \sphinxcode{\sphinxupquote{f(h)}} que reescreve a equação de Bernoulli acima em função de \(h\).

\begin{sphinxVerbatim}[commandchars=\\\{\}]
\PYG{c+c1}{\PYGZsh{} função para cálculo de parâmetros}
\PYG{k}{def} \PYG{n+nf}{a}\PYG{p}{(}\PYG{n}{Q}\PYG{p}{,}\PYG{n}{g}\PYG{p}{,}\PYG{n}{b}\PYG{p}{,}\PYG{n}{H}\PYG{p}{,}\PYG{n}{h0}\PYG{p}{)}\PYG{p}{:}
    \PYG{k}{return} \PYG{n}{Q}\PYG{p}{,}\PYG{n}{g}\PYG{p}{,}\PYG{n}{b}\PYG{p}{,}\PYG{n}{H}\PYG{p}{,}\PYG{n}{h0}

\PYG{c+c1}{\PYGZsh{} função do nível de água}
\PYG{k}{def} \PYG{n+nf}{f}\PYG{p}{(}\PYG{n}{h}\PYG{p}{)}\PYG{p}{:}
    \PYG{n}{frate}\PYG{p}{,}\PYG{n}{grav}\PYG{p}{,}\PYG{n}{width}\PYG{p}{,}\PYG{n}{bHeight}\PYG{p}{,}\PYG{n}{ups} \PYG{o}{=} \PYG{n}{a}\PYG{p}{(}\PYG{n}{Q}\PYG{p}{,}\PYG{n}{g}\PYG{p}{,}\PYG{n}{b}\PYG{p}{,}\PYG{n}{H}\PYG{p}{,}\PYG{n}{h0}\PYG{p}{)}
    \PYG{n}{c} \PYG{o}{=} \PYG{k}{lambda} \PYG{n}{arg}\PYG{p}{:} \PYG{n}{frate}\PYG{o}{*}\PYG{o}{*}\PYG{l+m+mi}{2}\PYG{o}{/}\PYG{p}{(}\PYG{l+m+mi}{2}\PYG{o}{*}\PYG{n}{grav}\PYG{o}{*}\PYG{n}{width}\PYG{o}{*}\PYG{o}{*}\PYG{l+m+mi}{2}\PYG{o}{*}\PYG{n}{arg}\PYG{o}{*}\PYG{o}{*}\PYG{l+m+mi}{2}\PYG{p}{)}
    \PYG{k}{return} \PYG{n}{c}\PYG{p}{(}\PYG{n}{h}\PYG{p}{)} \PYG{o}{\PYGZhy{}} \PYG{n}{c}\PYG{p}{(}\PYG{n}{h0}\PYG{p}{)} \PYG{o}{+} \PYG{n}{h} \PYG{o}{\PYGZhy{}} \PYG{n}{h0} \PYG{o}{+} \PYG{n}{H} 
\end{sphinxVerbatim}

\sphinxAtStartPar
Note que a função \sphinxcode{\sphinxupquote{a}} é apenas uma conveniência para o cálculo do termo comum envolvendo a vazão e para construírmos uma generalização para os dados de entrada. Em seguida, definiremos os parâmetros de entrada do problema.

\begin{sphinxVerbatim}[commandchars=\\\{\}]
\PYG{c+c1}{\PYGZsh{} parâmetros de entrada}
\PYG{n}{Q} \PYG{o}{=} \PYG{l+m+mf}{1.2} \PYG{c+c1}{\PYGZsh{} m3/s }
\PYG{n}{g} \PYG{o}{=} \PYG{l+m+mf}{9.81} \PYG{c+c1}{\PYGZsh{} m/s2}
\PYG{n}{b} \PYG{o}{=} \PYG{l+m+mf}{1.8} \PYG{c+c1}{\PYGZsh{} m}
\PYG{n}{h0} \PYG{o}{=} \PYG{l+m+mf}{0.6} \PYG{c+c1}{\PYGZsh{} m}
\PYG{n}{H} \PYG{o}{=} \PYG{l+m+mf}{0.075} \PYG{c+c1}{\PYGZsh{} m}
\end{sphinxVerbatim}

\sphinxAtStartPar
A partir daí, podemos realizar a análise gráfica para verificar o comportamento de \sphinxcode{\sphinxupquote{f(h)}}.

\begin{sphinxVerbatim}[commandchars=\\\{\}]
\PYG{c+c1}{\PYGZsh{} análise gráfica}
\PYG{n}{h} \PYG{o}{=} \PYG{n}{np}\PYG{o}{.}\PYG{n}{linspace}\PYG{p}{(}\PYG{l+m+mf}{0.1}\PYG{p}{,}\PYG{l+m+mi}{6}\PYG{p}{,}\PYG{n}{num}\PYG{o}{=}\PYG{l+m+mi}{100}\PYG{p}{)}
\PYG{n}{plt}\PYG{o}{.}\PYG{n}{plot}\PYG{p}{(}\PYG{n}{h}\PYG{p}{,}\PYG{n}{f}\PYG{p}{(}\PYG{n}{h}\PYG{p}{)}\PYG{p}{,}\PYG{n}{h}\PYG{p}{,}\PYG{n}{f}\PYG{p}{(}\PYG{n}{h}\PYG{p}{)}\PYG{o}{*}\PYG{l+m+mi}{0}\PYG{p}{)}\PYG{p}{;}
\end{sphinxVerbatim}

\noindent\sphinxincludegraphics{{codeSession-1-bisect_46_0}.png}

\sphinxAtStartPar
Ampliemos a localização.

\begin{sphinxVerbatim}[commandchars=\\\{\}]
\PYG{c+c1}{\PYGZsh{} análise gráfica}
\PYG{n}{h} \PYG{o}{=} \PYG{n}{np}\PYG{o}{.}\PYG{n}{linspace}\PYG{p}{(}\PYG{l+m+mf}{0.25}\PYG{p}{,}\PYG{l+m+mf}{0.6}\PYG{p}{,}\PYG{n}{num}\PYG{o}{=}\PYG{l+m+mi}{100}\PYG{p}{)}
\PYG{n}{plt}\PYG{o}{.}\PYG{n}{plot}\PYG{p}{(}\PYG{n}{h}\PYG{p}{,}\PYG{n}{f}\PYG{p}{(}\PYG{n}{h}\PYG{p}{)}\PYG{p}{,}\PYG{n}{h}\PYG{p}{,}\PYG{n}{f}\PYG{p}{(}\PYG{n}{h}\PYG{p}{)}\PYG{o}{*}\PYG{l+m+mi}{0}\PYG{p}{)}\PYG{p}{;}
\end{sphinxVerbatim}

\noindent\sphinxincludegraphics{{codeSession-1-bisect_48_0}.png}

\sphinxAtStartPar
Verificamos que \sphinxcode{\sphinxupquote{f(h)}} admite duas soluções. Vamos determinar cada uma delas.

\begin{sphinxVerbatim}[commandchars=\\\{\}]
\PYG{c+c1}{\PYGZsh{} solução  }
\PYG{n}{h1} \PYG{o}{=} \PYG{n}{bisect}\PYG{p}{(}\PYG{n}{f}\PYG{p}{,}\PYG{l+m+mf}{0.25}\PYG{p}{,}\PYG{l+m+mf}{0.32}\PYG{p}{)}
\PYG{n+nb}{print}\PYG{p}{(}\PYG{l+s+s1}{\PYGZsq{}}\PYG{l+s+s1}{Raiz: h1 = }\PYG{l+s+si}{\PYGZpc{}f}\PYG{l+s+s1}{\PYGZsq{}} \PYG{o}{\PYGZpc{}} \PYG{n}{h1}\PYG{p}{)}

\PYG{n}{h2} \PYG{o}{=} \PYG{n}{bisect}\PYG{p}{(}\PYG{n}{f}\PYG{p}{,}\PYG{l+m+mf}{0.4}\PYG{p}{,}\PYG{l+m+mf}{0.55}\PYG{p}{)}
\PYG{n+nb}{print}\PYG{p}{(}\PYG{l+s+s1}{\PYGZsq{}}\PYG{l+s+s1}{Raiz: h2 = }\PYG{l+s+si}{\PYGZpc{}f}\PYG{l+s+s1}{\PYGZsq{}} \PYG{o}{\PYGZpc{}} \PYG{n}{h2}\PYG{p}{)}
\end{sphinxVerbatim}

\begin{sphinxVerbatim}[commandchars=\\\{\}]
Raiz: h1 = 0.264755
Raiz: h2 = 0.495755
\end{sphinxVerbatim}

\sphinxAtStartPar
\sphinxstylestrong{Nota:} as duas soluções viáveis dizem respeito ao regime de escoamento no canal aberto. Enquanto \(h_1\) é um limite para escoamento supercrítico (rápido), \(h_2\) é um limite para escoamento subcrítico (lento).


\subsection{Problema 8}
\label{\detokenize{codeSession-1-bisect:problema-8}}
\sphinxAtStartPar
A velocidade \(v\) de um foguete Saturn V em voo vertical próximo à superfície da Terra pode ser aproximada por
\begin{equation*}
\begin{split}v = u\textrm{ln}\left(\dfrac{M_0}{M_0 - \dot{m}t} \right) - gt,\end{split}
\end{equation*}
\sphinxAtStartPar
onde \(u = 2510 \, m/s\) é a velocidade de escape relativa ao foguete, \(M_0 = 2.8 \times 10^6 \, kg\) é a massa do foguete no momento do lançamento, \(\dot{m} = 13.3 \times 10^3 \, kg/s\) é a taxa de consumo de combustível, \(g = 9.81 \, m/s^2\) a aceleração gravitacional e \(t\) o tempo medido a partir do lançamento.

\sphinxAtStartPar
Determine o instante de tempo \(t^*\) quando o foguete atinge a velocidade do som (\(335 \, m/s\)).


\subsubsection{Resolução}
\label{\detokenize{codeSession-1-bisect:id7}}
\sphinxAtStartPar
Seguiremos a mesma ideia utilizada no Problema 7. Primeiramente, construímos uma função auxiliar para calcular parâmetros e, em seguida, definimos uma função \sphinxcode{\sphinxupquote{f(t)}}.

\begin{sphinxVerbatim}[commandchars=\\\{\}]
\PYG{c+c1}{\PYGZsh{} função para cálculo de parâmetros}
\PYG{k}{def} \PYG{n+nf}{a}\PYG{p}{(}\PYG{n}{u}\PYG{p}{,}\PYG{n}{M0}\PYG{p}{,}\PYG{n}{m}\PYG{p}{,}\PYG{n}{g}\PYG{p}{,}\PYG{n}{v}\PYG{p}{)}\PYG{p}{:}
    \PYG{k}{return} \PYG{n}{u}\PYG{p}{,}\PYG{n}{M0}\PYG{p}{,}\PYG{n}{m}\PYG{p}{,}\PYG{n}{g}\PYG{p}{,}\PYG{n}{v}

\PYG{c+c1}{\PYGZsh{} função do tempo}
\PYG{k}{def} \PYG{n+nf}{f}\PYG{p}{(}\PYG{n}{t}\PYG{p}{)}\PYG{p}{:}
    \PYG{n}{escape}\PYG{p}{,}\PYG{n}{mass}\PYG{p}{,}\PYG{n}{fuel}\PYG{p}{,}\PYG{n}{grav}\PYG{p}{,}\PYG{n}{vel} \PYG{o}{=} \PYG{n}{a}\PYG{p}{(}\PYG{n}{u}\PYG{p}{,}\PYG{n}{M0}\PYG{p}{,}\PYG{n}{m}\PYG{p}{,}\PYG{n}{g}\PYG{p}{,}\PYG{n}{v}\PYG{p}{)}    
    \PYG{k}{return} \PYG{n}{escape}\PYG{o}{*}\PYG{n}{np}\PYG{o}{.}\PYG{n}{log}\PYG{p}{(}\PYG{n}{mass}\PYG{o}{/}\PYG{p}{(}\PYG{n}{mass} \PYG{o}{\PYGZhy{}} \PYG{n}{fuel}\PYG{o}{*}\PYG{n}{t}\PYG{p}{)}\PYG{p}{)} \PYG{o}{\PYGZhy{}} \PYG{n}{g}\PYG{o}{*}\PYG{n}{t} \PYG{o}{\PYGZhy{}} \PYG{n}{vel}
\end{sphinxVerbatim}

\sphinxAtStartPar
Definimos os parâmetros do problema.

\begin{sphinxVerbatim}[commandchars=\\\{\}]
\PYG{c+c1}{\PYGZsh{} parâmetros de entrada}
\PYG{n}{u} \PYG{o}{=} \PYG{l+m+mf}{2510.0} \PYG{c+c1}{\PYGZsh{} m/s }
\PYG{n}{M0} \PYG{o}{=} \PYG{l+m+mf}{2.8e6} \PYG{c+c1}{\PYGZsh{} kg}
\PYG{n}{m} \PYG{o}{=} \PYG{l+m+mf}{13.3e3} \PYG{c+c1}{\PYGZsh{} kg/s}
\PYG{n}{g} \PYG{o}{=} \PYG{l+m+mf}{9.81} \PYG{c+c1}{\PYGZsh{} m/s2}
\PYG{n}{v} \PYG{o}{=} \PYG{l+m+mf}{335.0} \PYG{c+c1}{\PYGZsh{} m/s}
\end{sphinxVerbatim}

\sphinxAtStartPar
Utilizaremos a análise gráfica para determinar o intervalo de refinamento da raiz.

\begin{sphinxVerbatim}[commandchars=\\\{\}]
\PYG{c+c1}{\PYGZsh{} análise gráfica}
\PYG{n}{t} \PYG{o}{=} \PYG{n}{np}\PYG{o}{.}\PYG{n}{linspace}\PYG{p}{(}\PYG{l+m+mf}{0.5}\PYG{p}{,}\PYG{l+m+mi}{100}\PYG{p}{,}\PYG{n}{num}\PYG{o}{=}\PYG{l+m+mi}{100}\PYG{p}{)}
\PYG{n}{plt}\PYG{o}{.}\PYG{n}{plot}\PYG{p}{(}\PYG{n}{t}\PYG{p}{,}\PYG{n}{f}\PYG{p}{(}\PYG{n}{t}\PYG{p}{)}\PYG{p}{,}\PYG{n}{t}\PYG{p}{,}\PYG{n}{f}\PYG{p}{(}\PYG{n}{t}\PYG{p}{)}\PYG{o}{*}\PYG{l+m+mi}{0}\PYG{p}{)}\PYG{p}{;}
\end{sphinxVerbatim}

\noindent\sphinxincludegraphics{{codeSession-1-bisect_58_0}.png}

\sphinxAtStartPar
Podemos verificar que a raiz está entre 60 e 80 segundos. Utilizaremos estes limitantes.

\begin{sphinxVerbatim}[commandchars=\\\{\}]
\PYG{c+c1}{\PYGZsh{} solução  }
\PYG{n}{tr} \PYG{o}{=} \PYG{n}{bisect}\PYG{p}{(}\PYG{n}{f}\PYG{p}{,}\PYG{l+m+mi}{60}\PYG{p}{,}\PYG{l+m+mi}{80}\PYG{p}{)}
\PYG{n+nb}{print}\PYG{p}{(}\PYG{l+s+s1}{\PYGZsq{}}\PYG{l+s+s1}{Raiz: tr = }\PYG{l+s+si}{\PYGZpc{}.2f}\PYG{l+s+s1}{ s = }\PYG{l+s+si}{\PYGZpc{}.2f}\PYG{l+s+s1}{ min}\PYG{l+s+s1}{\PYGZsq{}} \PYG{o}{\PYGZpc{}} \PYG{p}{(}\PYG{n}{tr}\PYG{p}{,}\PYG{n}{tr}\PYG{o}{/}\PYG{l+m+mi}{60}\PYG{p}{)} \PYG{p}{)}
\end{sphinxVerbatim}

\begin{sphinxVerbatim}[commandchars=\\\{\}]
Raiz: tr = 70.88 s = 1.18 min
\end{sphinxVerbatim}

\sphinxAtStartPar
O foguete rompe a barreira do som em 1 minuto e 18 segundos!


\chapter{Code session 2}
\label{\detokenize{codeSession-2-newton:code-session-2}}\label{\detokenize{codeSession-2-newton::doc}}
\begin{sphinxVerbatim}[commandchars=\\\{\}]
\PYG{o}{\PYGZpc{}}\PYG{k}{matplotlib} inline 
\PYG{k+kn}{import} \PYG{n+nn}{numpy} \PYG{k}{as} \PYG{n+nn}{np}
\PYG{k+kn}{import} \PYG{n+nn}{matplotlib}\PYG{n+nn}{.}\PYG{n+nn}{pyplot} \PYG{k}{as} \PYG{n+nn}{plt} 
\PYG{k+kn}{import} \PYG{n+nn}{sympy} \PYG{k}{as} \PYG{n+nn}{sy}
\end{sphinxVerbatim}


\section{Determinação de raízes}
\label{\detokenize{codeSession-2-newton:determinacao-de-raizes}}

\section{\sphinxstyleliteralintitle{\sphinxupquote{newton}}}
\label{\detokenize{codeSession-2-newton:newton}}
\sphinxAtStartPar
A função \sphinxcode{\sphinxupquote{newton}} localiza a raiz de uma função dentro de um intervalo dado usando o método de Newton.
Os argumentos de entrada obrigatórios desta função são:
\begin{enumerate}
\sphinxsetlistlabels{\arabic}{enumi}{enumii}{}{.}%
\item {} 
\sphinxAtStartPar
a função\sphinxhyphen{}alvo \sphinxcode{\sphinxupquote{f}} (contínua)

\item {} 
\sphinxAtStartPar
a estimativa inicial \sphinxcode{\sphinxupquote{x0}}

\end{enumerate}

\sphinxAtStartPar
Parâmetros opcionais relevantes são:
\begin{itemize}
\item {} 
\sphinxAtStartPar
\sphinxcode{\sphinxupquote{fprime}}: a derivada da função, quando disponível. Caso ela não seja especificada, o \sphinxstyleemphasis{método da secante} é usado.

\item {} 
\sphinxAtStartPar
\sphinxcode{\sphinxupquote{fprime2}}: a segunda derivada da função, quando disponível. Se ela for especificada, o \sphinxstyleemphasis{método de Halley} é usado.

\item {} 
\sphinxAtStartPar
\sphinxcode{\sphinxupquote{tol}}: tolerância (padrão: 1.48e\sphinxhyphen{}08)

\item {} 
\sphinxAtStartPar
\sphinxcode{\sphinxupquote{maxiter}}: número máximo de iterações (padrão: 50)

\item {} 
\sphinxAtStartPar
\sphinxcode{\sphinxupquote{disp}}: mostra erro se algoritmo não convergir (padrão: True)

\end{itemize}

\sphinxAtStartPar
O argumento de saída é:
\begin{itemize}
\item {} 
\sphinxAtStartPar
\sphinxcode{\sphinxupquote{x}}: a estimativa para a raiz de \sphinxcode{\sphinxupquote{f}}

\end{itemize}

\sphinxAtStartPar
Como importá\sphinxhyphen{}la?

\begin{sphinxVerbatim}[commandchars=\\\{\}]
\PYG{k+kn}{from} \PYG{n+nn}{scipy}\PYG{n+nn}{.}\PYG{n+nn}{optimize} \PYG{k+kn}{import} \PYG{n}{newton}
\end{sphinxVerbatim}

\begin{sphinxVerbatim}[commandchars=\\\{\}]
\PYG{k+kn}{from} \PYG{n+nn}{scipy}\PYG{n+nn}{.}\PYG{n+nn}{optimize} \PYG{k+kn}{import} \PYG{n}{newton} 
\end{sphinxVerbatim}


\subsection{Problema 1}
\label{\detokenize{codeSession-2-newton:problema-1}}
\sphinxAtStartPar
Encontre a menor raiz positiva (real) de \(x^{3} - 3.23x^{2} - 5.54x + 9.84 = 0\) pelo método de Newton.


\subsubsection{Resolução}
\label{\detokenize{codeSession-2-newton:resolucao}}
\sphinxAtStartPar
Definimos a função e sua primeira derivada.

\begin{sphinxVerbatim}[commandchars=\\\{\}]
\PYG{c+c1}{\PYGZsh{} função}
\PYG{k}{def} \PYG{n+nf}{f}\PYG{p}{(}\PYG{n}{x}\PYG{p}{)}\PYG{p}{:} 
    \PYG{k}{return} \PYG{n}{x}\PYG{o}{*}\PYG{o}{*}\PYG{l+m+mi}{3} \PYG{o}{\PYGZhy{}} \PYG{l+m+mf}{3.23}\PYG{o}{*}\PYG{n}{x}\PYG{o}{*}\PYG{o}{*}\PYG{l+m+mi}{2} \PYG{o}{\PYGZhy{}} \PYG{l+m+mf}{5.54}\PYG{o}{*}\PYG{n}{x} \PYG{o}{+} \PYG{l+m+mf}{9.84}

\PYG{c+c1}{\PYGZsh{} 1a. derivada}
\PYG{k}{def} \PYG{n+nf}{df}\PYG{p}{(}\PYG{n}{x}\PYG{p}{)}\PYG{p}{:}
    \PYG{k}{return} \PYG{l+m+mi}{3}\PYG{o}{*}\PYG{n}{x}\PYG{o}{*}\PYG{o}{*}\PYG{l+m+mi}{2} \PYG{o}{\PYGZhy{}} \PYG{l+m+mi}{2}\PYG{o}{*}\PYG{l+m+mf}{3.23}\PYG{o}{*}\PYG{n}{x} \PYG{o}{\PYGZhy{}} \PYG{l+m+mf}{5.54}
\end{sphinxVerbatim}

\sphinxAtStartPar
Realizamos a análise gráfica.

\begin{sphinxVerbatim}[commandchars=\\\{\}]
\PYG{c+c1}{\PYGZsh{} analise gráfica }
\PYG{n}{x} \PYG{o}{=} \PYG{n}{np}\PYG{o}{.}\PYG{n}{linspace}\PYG{p}{(}\PYG{o}{\PYGZhy{}}\PYG{l+m+mi}{4}\PYG{p}{,}\PYG{l+m+mi}{5}\PYG{p}{)}
\PYG{n}{plt}\PYG{o}{.}\PYG{n}{plot}\PYG{p}{(}\PYG{n}{x}\PYG{p}{,}\PYG{n}{f}\PYG{p}{(}\PYG{n}{x}\PYG{p}{)}\PYG{p}{)}\PYG{p}{;}
\PYG{n}{plt}\PYG{o}{.}\PYG{n}{plot}\PYG{p}{(}\PYG{n}{x}\PYG{p}{,}\PYG{n}{df}\PYG{p}{(}\PYG{n}{x}\PYG{p}{)}\PYG{p}{)}\PYG{p}{;}
\PYG{n}{plt}\PYG{o}{.}\PYG{n}{axhline}\PYG{p}{(}\PYG{n}{y}\PYG{o}{=}\PYG{l+m+mi}{0}\PYG{p}{,}\PYG{n}{color}\PYG{o}{=}\PYG{l+s+s1}{\PYGZsq{}}\PYG{l+s+s1}{k}\PYG{l+s+s1}{\PYGZsq{}}\PYG{p}{,}\PYG{n}{ls}\PYG{o}{=}\PYG{l+s+s1}{\PYGZsq{}}\PYG{l+s+s1}{\PYGZhy{}\PYGZhy{}}\PYG{l+s+s1}{\PYGZsq{}}\PYG{p}{)}\PYG{p}{;}
\PYG{n}{plt}\PYG{o}{.}\PYG{n}{axvline}\PYG{p}{(}\PYG{n}{x}\PYG{o}{=}\PYG{l+m+mi}{0}\PYG{p}{,}\PYG{n}{color}\PYG{o}{=}\PYG{l+s+s1}{\PYGZsq{}}\PYG{l+s+s1}{r}\PYG{l+s+s1}{\PYGZsq{}}\PYG{p}{,}\PYG{n}{ls}\PYG{o}{=}\PYG{l+s+s1}{\PYGZsq{}}\PYG{l+s+s1}{\PYGZhy{}\PYGZhy{}}\PYG{l+s+s1}{\PYGZsq{}}\PYG{p}{)}\PYG{p}{;}
\PYG{n}{plt}\PYG{o}{.}\PYG{n}{legend}\PYG{p}{(}\PYG{p}{[}\PYG{l+s+s1}{\PYGZsq{}}\PYG{l+s+s1}{\PYGZdl{}f(x)\PYGZdl{}}\PYG{l+s+s1}{\PYGZsq{}}\PYG{p}{,}\PYG{l+s+s1}{\PYGZsq{}}\PYG{l+s+s1}{\PYGZdl{}f}\PYG{l+s+se}{\PYGZbs{}\PYGZsq{}}\PYG{l+s+s1}{(x)\PYGZdl{}}\PYG{l+s+s1}{\PYGZsq{}}\PYG{p}{,}\PYG{l+s+s1}{\PYGZsq{}}\PYG{l+s+s1}{\PYGZdl{}y=0\PYGZdl{}}\PYG{l+s+s1}{\PYGZsq{}}\PYG{p}{,}\PYG{l+s+s1}{\PYGZsq{}}\PYG{l+s+s1}{x=0}\PYG{l+s+s1}{\PYGZsq{}}\PYG{p}{]}\PYG{p}{)}\PYG{p}{;}
\end{sphinxVerbatim}

\noindent\sphinxincludegraphics{{codeSession-2-newton_10_0}.png}

\sphinxAtStartPar
Vamos realizar um estudo de diferentes estimativas iniciais e ver o que acontece.


\paragraph{Estimativa inicial: \protect\(x_0 = -1\protect\)}
\label{\detokenize{codeSession-2-newton:estimativa-inicial-x-0-1}}
\begin{sphinxVerbatim}[commandchars=\\\{\}]
\PYG{c+c1}{\PYGZsh{} resolução com newton }
\PYG{n}{x0} \PYG{o}{=} \PYG{o}{\PYGZhy{}}\PYG{l+m+mf}{1.}
\PYG{n}{x} \PYG{o}{=} \PYG{n}{newton}\PYG{p}{(}\PYG{n}{f}\PYG{p}{,}\PYG{n}{x0}\PYG{p}{,}\PYG{n}{df}\PYG{p}{)} \PYG{c+c1}{\PYGZsh{} raiz }
\PYG{n+nb}{print}\PYG{p}{(}\PYG{l+s+s1}{\PYGZsq{}}\PYG{l+s+s1}{Raiz: x = }\PYG{l+s+si}{\PYGZpc{}f}\PYG{l+s+s1}{\PYGZsq{}} \PYG{o}{\PYGZpc{}} \PYG{n}{x}\PYG{p}{)}
\end{sphinxVerbatim}

\begin{sphinxVerbatim}[commandchars=\\\{\}]
Raiz: x = \PYGZhy{}2.000000
\end{sphinxVerbatim}


\paragraph{Estimativa inicial: \protect\(x_0 = 0\protect\)}
\label{\detokenize{codeSession-2-newton:estimativa-inicial-x-0-0}}
\begin{sphinxVerbatim}[commandchars=\\\{\}]
\PYG{c+c1}{\PYGZsh{} resolução com newton }
\PYG{n}{x0} \PYG{o}{=} \PYG{l+m+mf}{0.}
\PYG{n}{x} \PYG{o}{=} \PYG{n}{newton}\PYG{p}{(}\PYG{n}{f}\PYG{p}{,}\PYG{n}{x0}\PYG{p}{,}\PYG{n}{df}\PYG{p}{)} \PYG{c+c1}{\PYGZsh{} raiz }
\PYG{n+nb}{print}\PYG{p}{(}\PYG{l+s+s1}{\PYGZsq{}}\PYG{l+s+s1}{Raiz: x = }\PYG{l+s+si}{\PYGZpc{}f}\PYG{l+s+s1}{\PYGZsq{}} \PYG{o}{\PYGZpc{}} \PYG{n}{x}\PYG{p}{)}
\end{sphinxVerbatim}

\begin{sphinxVerbatim}[commandchars=\\\{\}]
Raiz: x = 1.230000
\end{sphinxVerbatim}


\paragraph{Estimativa inicial: \protect\(x_0 = 3\protect\)}
\label{\detokenize{codeSession-2-newton:estimativa-inicial-x-0-3}}
\begin{sphinxVerbatim}[commandchars=\\\{\}]
\PYG{c+c1}{\PYGZsh{} resolução com newton }
\PYG{n}{x0} \PYG{o}{=} \PYG{l+m+mf}{3.}
\PYG{n}{x} \PYG{o}{=} \PYG{n}{newton}\PYG{p}{(}\PYG{n}{f}\PYG{p}{,}\PYG{n}{x0}\PYG{p}{,}\PYG{n}{df}\PYG{p}{)} \PYG{c+c1}{\PYGZsh{} raiz }
\PYG{n+nb}{print}\PYG{p}{(}\PYG{l+s+s1}{\PYGZsq{}}\PYG{l+s+s1}{Raiz: x = }\PYG{l+s+si}{\PYGZpc{}f}\PYG{l+s+s1}{\PYGZsq{}} \PYG{o}{\PYGZpc{}} \PYG{n}{x}\PYG{p}{)}
\end{sphinxVerbatim}

\begin{sphinxVerbatim}[commandchars=\\\{\}]
Raiz: x = 4.000000
\end{sphinxVerbatim}


\subsection{Problema 2}
\label{\detokenize{codeSession-2-newton:problema-2}}
\sphinxAtStartPar
Determine a menor raiz não nula positiva de \(\cosh(x) \cos(x) - 1 = 0\) dentro do intervalo \((4,5)\).


\subsubsection{Resolução:}
\label{\detokenize{codeSession-2-newton:id1}}
\sphinxAtStartPar
Primeiramente, vamos escrever a função \sphinxcode{\sphinxupquote{f(x)}}.

\begin{sphinxVerbatim}[commandchars=\\\{\}]
\PYG{c+c1}{\PYGZsh{} função}
\PYG{n}{f} \PYG{o}{=} \PYG{k}{lambda} \PYG{n}{x}\PYG{p}{:} \PYG{n}{np}\PYG{o}{.}\PYG{n}{cosh}\PYG{p}{(}\PYG{n}{x}\PYG{p}{)}\PYG{o}{*}\PYG{n}{np}\PYG{o}{.}\PYG{n}{cos}\PYG{p}{(}\PYG{n}{x}\PYG{p}{)} \PYG{o}{\PYGZhy{}} \PYG{l+m+mi}{1} 
\end{sphinxVerbatim}

\sphinxAtStartPar
Para computar a primeira derivada, vamos utilizar computação simbólica. Veja no início deste notebook que inserimos a instrução

\begin{sphinxVerbatim}[commandchars=\\\{\}]
\PYG{k+kn}{import} \PYG{n+nn}{sympy} \PYG{k}{as} \PYG{n+nn}{sy}
\end{sphinxVerbatim}

\sphinxAtStartPar
a qual nos permitirá utilizar objetos do módulo \sphinxcode{\sphinxupquote{sympy}}.

\sphinxAtStartPar
Em primeiro lugar, devemos estabelecer uma variável simbólica \sphinxcode{\sphinxupquote{xs}}.

\begin{sphinxVerbatim}[commandchars=\\\{\}]
\PYG{n}{xs} \PYG{o}{=} \PYG{n}{sy}\PYG{o}{.}\PYG{n}{Symbol}\PYG{p}{(}\PYG{l+s+s1}{\PYGZsq{}}\PYG{l+s+s1}{x}\PYG{l+s+s1}{\PYGZsq{}}\PYG{p}{)}
\end{sphinxVerbatim}

\sphinxAtStartPar
Em seguida, devemos utilizar as funções \sphinxcode{\sphinxupquote{cosh}} e \sphinxcode{\sphinxupquote{cos}} \sphinxstylestrong{simbólicas} para derivar \sphinxcode{\sphinxupquote{f}}. Elas serão \sphinxstylestrong{chamadas de dentro do módulo sympy}.

\sphinxAtStartPar
Escrevemos a expressão simbólica para a derivada.

\begin{sphinxVerbatim}[commandchars=\\\{\}]
\PYG{n}{d} \PYG{o}{=} \PYG{n}{sy}\PYG{o}{.}\PYG{n}{diff}\PYG{p}{(}\PYG{n}{sy}\PYG{o}{.}\PYG{n}{cosh}\PYG{p}{(}\PYG{n}{xs}\PYG{p}{)}\PYG{o}{*}\PYG{n}{sy}\PYG{o}{.}\PYG{n}{cos}\PYG{p}{(}\PYG{n}{xs}\PYG{p}{)} \PYG{o}{\PYGZhy{}} \PYG{l+m+mi}{1}\PYG{p}{)}
\PYG{n}{d}
\end{sphinxVerbatim}
\begin{equation*}
\begin{split}\displaystyle - \sin{\left(x \right)} \cosh{\left(x \right)} + \cos{\left(x \right)} \sinh{\left(x \right)}\end{split}
\end{equation*}
\sphinxAtStartPar
Note que \sphinxcode{\sphinxupquote{d}} é um objeto do módulo \sphinxcode{\sphinxupquote{sympy}}

\begin{sphinxVerbatim}[commandchars=\\\{\}]
\PYG{n+nb}{type}\PYG{p}{(}\PYG{n}{d}\PYG{p}{)}
\end{sphinxVerbatim}

\begin{sphinxVerbatim}[commandchars=\\\{\}]
sympy.core.add.Add
\end{sphinxVerbatim}

\sphinxAtStartPar
Podemos agora aproveitar a expressão de \sphinxcode{\sphinxupquote{d}} para criar nossa derivada. Se imprimirmos \sphinxcode{\sphinxupquote{d}}, teremos:

\begin{sphinxVerbatim}[commandchars=\\\{\}]
\PYG{n+nb}{print}\PYG{p}{(}\PYG{n}{d}\PYG{p}{)}
\end{sphinxVerbatim}

\begin{sphinxVerbatim}[commandchars=\\\{\}]
\PYGZhy{}sin(x)*cosh(x) + cos(x)*sinh(x)
\end{sphinxVerbatim}

\sphinxAtStartPar
Porém, precisamos indicar que as funções serão objetos numpy. Logo, adicionamos \sphinxcode{\sphinxupquote{np}}, de modo que:

\begin{sphinxVerbatim}[commandchars=\\\{\}]
\PYG{n}{df} \PYG{o}{=} \PYG{k}{lambda} \PYG{n}{x}\PYG{p}{:} \PYG{o}{\PYGZhy{}} \PYG{n}{np}\PYG{o}{.}\PYG{n}{sin}\PYG{p}{(}\PYG{n}{x}\PYG{p}{)}\PYG{o}{*}\PYG{n}{np}\PYG{o}{.}\PYG{n}{cosh}\PYG{p}{(}\PYG{n}{x}\PYG{p}{)} \PYG{o}{+} \PYG{n}{np}\PYG{o}{.}\PYG{n}{cos}\PYG{p}{(}\PYG{n}{x}\PYG{p}{)}\PYG{o}{*}\PYG{n}{np}\PYG{o}{.}\PYG{n}{sinh}\PYG{p}{(}\PYG{n}{x}\PYG{p}{)}
\end{sphinxVerbatim}

\sphinxAtStartPar
Agora, realizamos a análise gráfica.

\begin{sphinxVerbatim}[commandchars=\\\{\}]
\PYG{c+c1}{\PYGZsh{} analise gráfica }
\PYG{n}{x} \PYG{o}{=} \PYG{n}{np}\PYG{o}{.}\PYG{n}{linspace}\PYG{p}{(}\PYG{l+m+mi}{4}\PYG{p}{,}\PYG{l+m+mi}{5}\PYG{p}{)}
\PYG{n}{plt}\PYG{o}{.}\PYG{n}{plot}\PYG{p}{(}\PYG{n}{x}\PYG{p}{,}\PYG{n}{f}\PYG{p}{(}\PYG{n}{x}\PYG{p}{)}\PYG{p}{)}\PYG{p}{;}
\PYG{n}{plt}\PYG{o}{.}\PYG{n}{plot}\PYG{p}{(}\PYG{n}{x}\PYG{p}{,}\PYG{n}{df}\PYG{p}{(}\PYG{n}{x}\PYG{p}{)}\PYG{p}{)}\PYG{p}{;}
\PYG{n}{plt}\PYG{o}{.}\PYG{n}{axhline}\PYG{p}{(}\PYG{n}{y}\PYG{o}{=}\PYG{l+m+mi}{0}\PYG{p}{,}\PYG{n}{color}\PYG{o}{=}\PYG{l+s+s1}{\PYGZsq{}}\PYG{l+s+s1}{r}\PYG{l+s+s1}{\PYGZsq{}}\PYG{p}{,}\PYG{n}{ls}\PYG{o}{=}\PYG{l+s+s1}{\PYGZsq{}}\PYG{l+s+s1}{\PYGZhy{}\PYGZhy{}}\PYG{l+s+s1}{\PYGZsq{}}\PYG{p}{)}\PYG{p}{;}
\PYG{n}{plt}\PYG{o}{.}\PYG{n}{legend}\PYG{p}{(}\PYG{p}{[}\PYG{l+s+s1}{\PYGZsq{}}\PYG{l+s+s1}{\PYGZdl{}f(x)\PYGZdl{}}\PYG{l+s+s1}{\PYGZsq{}}\PYG{p}{,}\PYG{l+s+s1}{\PYGZsq{}}\PYG{l+s+s1}{\PYGZdl{}f}\PYG{l+s+se}{\PYGZbs{}\PYGZsq{}}\PYG{l+s+s1}{(x)\PYGZdl{}}\PYG{l+s+s1}{\PYGZsq{}}\PYG{p}{,}\PYG{l+s+s1}{\PYGZsq{}}\PYG{l+s+s1}{\PYGZdl{}y=0\PYGZdl{}}\PYG{l+s+s1}{\PYGZsq{}}\PYG{p}{]}\PYG{p}{)}\PYG{p}{;}
\end{sphinxVerbatim}

\noindent\sphinxincludegraphics{{codeSession-2-newton_33_0}.png}

\sphinxAtStartPar
Agora, vamos resolver optando pela estimativa inicial \(x_0 = 4.9\).

\begin{sphinxVerbatim}[commandchars=\\\{\}]
\PYG{c+c1}{\PYGZsh{} resolução com newton }
\PYG{n}{x0} \PYG{o}{=} \PYG{l+m+mf}{4.9}
\PYG{n}{x} \PYG{o}{=} \PYG{n}{newton}\PYG{p}{(}\PYG{n}{f}\PYG{p}{,}\PYG{n}{x0}\PYG{p}{,}\PYG{n}{df}\PYG{p}{)} \PYG{c+c1}{\PYGZsh{} raiz }
\PYG{n+nb}{print}\PYG{p}{(}\PYG{l+s+s1}{\PYGZsq{}}\PYG{l+s+s1}{Raiz: x = }\PYG{l+s+si}{\PYGZpc{}f}\PYG{l+s+s1}{\PYGZsq{}} \PYG{o}{\PYGZpc{}} \PYG{n}{x}\PYG{p}{)}
\end{sphinxVerbatim}

\begin{sphinxVerbatim}[commandchars=\\\{\}]
Raiz: x = 4.730041
\end{sphinxVerbatim}


\section{\sphinxstyleemphasis{Homework}}
\label{\detokenize{codeSession-2-newton:homework}}\begin{enumerate}
\sphinxsetlistlabels{\arabic}{enumi}{enumii}{}{.}%
\item {} 
\sphinxAtStartPar
Reproduza os Problemas de 3 a 8 da \sphinxstyleemphasis{Code Session 1} resolvendo com o método \sphinxcode{\sphinxupquote{newton}}.

\item {} 
\sphinxAtStartPar
Para os casos possíveis, determine a derivada. Caso contrário, utilize como método da Secante.

\item {} 
\sphinxAtStartPar
Pesquise sobre o método de Halley e aplique\sphinxhyphen{}o aos problemas usando também a função \sphinxcode{\sphinxupquote{newton}}, mas avaliando\sphinxhyphen{}a também com a segunda derivada.

\end{enumerate}


\chapter{Code session 3}
\label{\detokenize{codeSession-3-polyval:code-session-3}}\label{\detokenize{codeSession-3-polyval::doc}}
\begin{sphinxVerbatim}[commandchars=\\\{\}]
\PYG{o}{\PYGZpc{}}\PYG{k}{matplotlib} inline
\PYG{k+kn}{import} \PYG{n+nn}{numpy} \PYG{k}{as} \PYG{n+nn}{np} 
\PYG{k+kn}{import} \PYG{n+nn}{matplotlib}\PYG{n+nn}{.}\PYG{n+nn}{pyplot} \PYG{k}{as} \PYG{n+nn}{plt} 
\end{sphinxVerbatim}


\section{Determinação de raízes de polinômios}
\label{\detokenize{codeSession-3-polyval:determinacao-de-raizes-de-polinomios}}

\subsection{\sphinxstyleliteralintitle{\sphinxupquote{roots}}}
\label{\detokenize{codeSession-3-polyval:roots}}
\sphinxAtStartPar
A função \sphinxcode{\sphinxupquote{roots}} computa as raízes de uma função dentro de um intervalo dado usando o método de Hörner.
O único argumento de entrada desta função é
\begin{enumerate}
\sphinxsetlistlabels{\arabic}{enumi}{enumii}{}{.}%
\item {} 
\sphinxAtStartPar
o \sphinxstyleemphasis{array} \sphinxcode{\sphinxupquote{p}} com os coeficientes dos termos do polinômio.

\end{enumerate}
\begin{equation*}
\begin{split}P(x) = p_n x^n + p_{n-1} x^{n-1} + \ldots + p_1x + p_0\end{split}
\end{equation*}
\sphinxAtStartPar
O argumento de saída é:
\begin{itemize}
\item {} 
\sphinxAtStartPar
\sphinxcode{\sphinxupquote{x}}: \sphinxstyleemphasis{array} com as raízes de \(P(x)\).

\end{itemize}

\sphinxAtStartPar
Como importá\sphinxhyphen{}la?

\begin{sphinxVerbatim}[commandchars=\\\{\}]
\PYG{k+kn}{from} \PYG{n+nn}{numpy} \PYG{k+kn}{import} \PYG{n}{roots}
\end{sphinxVerbatim}

\sphinxAtStartPar
Porém, como já fizemos uma importação do \sphinxcode{\sphinxupquote{numpy}} acima, basta utilizarmos

\begin{sphinxVerbatim}[commandchars=\\\{\}]
\PYG{n}{np}\PYG{o}{.}\PYG{n}{roots}\PYG{p}{(}\PYG{n}{p}\PYG{p}{)}
\end{sphinxVerbatim}


\subsection{Problema 1}
\label{\detokenize{codeSession-3-polyval:problema-1}}
\sphinxAtStartPar
Determine as raízes de \(P(x) = 3x^3 +7x^2 - 36x + 20\).


\subsubsection{Resolução}
\label{\detokenize{codeSession-3-polyval:resolucao}}
\sphinxAtStartPar
Para tornar claro, em primeiro lugar, vamos inserir os coeficientes de \(P(x)\) em um \sphinxstyleemphasis{array} chamado \sphinxcode{\sphinxupquote{p}}.

\begin{sphinxVerbatim}[commandchars=\\\{\}]
\PYG{n}{p} \PYG{o}{=} \PYG{n}{np}\PYG{o}{.}\PYG{n}{array}\PYG{p}{(}\PYG{p}{[}\PYG{l+m+mi}{3}\PYG{p}{,}\PYG{l+m+mi}{7}\PYG{p}{,}\PYG{o}{\PYGZhy{}}\PYG{l+m+mi}{36}\PYG{p}{,}\PYG{l+m+mi}{20}\PYG{p}{]}\PYG{p}{)}
\end{sphinxVerbatim}

\sphinxAtStartPar
Em seguida, fazemos:

\begin{sphinxVerbatim}[commandchars=\\\{\}]
\PYG{n}{x} \PYG{o}{=} \PYG{n}{np}\PYG{o}{.}\PYG{n}{roots}\PYG{p}{(}\PYG{n}{p}\PYG{p}{)}
\end{sphinxVerbatim}

\sphinxAtStartPar
Podemos imprimir as raízes da seguinte forma:

\begin{sphinxVerbatim}[commandchars=\\\{\}]
\PYG{k}{for} \PYG{n}{i}\PYG{p}{,} \PYG{n}{v} \PYG{o+ow}{in} \PYG{n+nb}{enumerate}\PYG{p}{(}\PYG{n}{x}\PYG{p}{)}\PYG{p}{:}
    \PYG{n+nb}{print}\PYG{p}{(}\PYG{l+s+sa}{f}\PYG{l+s+s1}{\PYGZsq{}}\PYG{l+s+s1}{Raiz }\PYG{l+s+si}{\PYGZob{}}\PYG{n}{i}\PYG{l+s+si}{\PYGZcb{}}\PYG{l+s+s1}{: }\PYG{l+s+si}{\PYGZob{}}\PYG{n}{v}\PYG{l+s+si}{\PYGZcb{}}\PYG{l+s+s1}{\PYGZsq{}}\PYG{p}{)}
\end{sphinxVerbatim}

\begin{sphinxVerbatim}[commandchars=\\\{\}]
Raiz 0: \PYGZhy{}5.0
Raiz 1: 1.9999999999999987
Raiz 2: 0.6666666666666669
\end{sphinxVerbatim}


\subsection{\sphinxstyleliteralintitle{\sphinxupquote{polyval}}}
\label{\detokenize{codeSession-3-polyval:polyval}}
\sphinxAtStartPar
Podemos usar a função \sphinxcode{\sphinxupquote{polyval}} do \sphinxcode{\sphinxupquote{numpy}} para avaliar \(P(x)\) em \(x = x_0\). Verifiquemos, analiticamente, se as raízes anteriores satisfazem realmente o polinômio dado.

\begin{sphinxVerbatim}[commandchars=\\\{\}]
\PYG{k}{for} \PYG{n}{i} \PYG{o+ow}{in} \PYG{n}{x}\PYG{p}{:}
    \PYG{n}{v} \PYG{o}{=} \PYG{n}{np}\PYG{o}{.}\PYG{n}{polyval}\PYG{p}{(}\PYG{n}{p}\PYG{p}{,}\PYG{n}{i}\PYG{p}{)}
    \PYG{n+nb}{print}\PYG{p}{(}\PYG{l+s+sa}{f}\PYG{l+s+s1}{\PYGZsq{}}\PYG{l+s+s1}{P(x) = }\PYG{l+s+si}{\PYGZob{}}\PYG{n}{v}\PYG{l+s+si}{\PYGZcb{}}\PYG{l+s+s1}{\PYGZsq{}}\PYG{p}{)}
\end{sphinxVerbatim}

\begin{sphinxVerbatim}[commandchars=\\\{\}]
P(x) = 0.0
P(x) = \PYGZhy{}3.552713678800501e\PYGZhy{}14
P(x) = \PYGZhy{}7.105427357601002e\PYGZhy{}15
\end{sphinxVerbatim}

\sphinxAtStartPar
Note que as duas últimas raízes são “muito próximas” de zero, mas não exatamente zero.

\sphinxAtStartPar
Podemos também fazer uma verificação geométrica plotando o polinômio e suas raízes.

\begin{sphinxVerbatim}[commandchars=\\\{\}]
\PYG{n}{xx} \PYG{o}{=} \PYG{n}{np}\PYG{o}{.}\PYG{n}{linspace}\PYG{p}{(}\PYG{n}{np}\PYG{o}{.}\PYG{n}{min}\PYG{p}{(}\PYG{n}{x}\PYG{p}{)}\PYG{o}{\PYGZhy{}}\PYG{l+m+mf}{0.5}\PYG{p}{,}\PYG{n}{np}\PYG{o}{.}\PYG{n}{max}\PYG{p}{(}\PYG{n}{x}\PYG{p}{)}\PYG{o}{+}\PYG{l+m+mf}{0.5}\PYG{p}{)}
\PYG{n}{plt}\PYG{o}{.}\PYG{n}{plot}\PYG{p}{(}\PYG{n}{xx}\PYG{p}{,}\PYG{n}{np}\PYG{o}{.}\PYG{n}{polyval}\PYG{p}{(}\PYG{n}{p}\PYG{p}{,}\PYG{n}{xx}\PYG{p}{)}\PYG{p}{)}\PYG{p}{;}
\PYG{k}{for} \PYG{n}{i} \PYG{o+ow}{in} \PYG{n}{x}\PYG{p}{:}
    \PYG{n}{plt}\PYG{o}{.}\PYG{n}{plot}\PYG{p}{(}\PYG{n}{i}\PYG{p}{,}\PYG{n}{np}\PYG{o}{.}\PYG{n}{polyval}\PYG{p}{(}\PYG{n}{p}\PYG{p}{,}\PYG{n}{i}\PYG{p}{)}\PYG{p}{,}\PYG{l+s+s1}{\PYGZsq{}}\PYG{l+s+s1}{or}\PYG{l+s+s1}{\PYGZsq{}}\PYG{p}{)}
\end{sphinxVerbatim}

\noindent\sphinxincludegraphics{{codeSession-3-polyval_16_0}.png}


\subsection{Problema 2}
\label{\detokenize{codeSession-3-polyval:problema-2}}
\sphinxAtStartPar
Determine as raízes de \(P(x) = x^4 - 3x^2 + 3x\).


\subsubsection{Resolução}
\label{\detokenize{codeSession-3-polyval:id1}}
\sphinxAtStartPar
Resolvendo diretamente com \sphinxcode{\sphinxupquote{roots}} e usando \sphinxcode{\sphinxupquote{polyval}} para verificação, temos:

\begin{sphinxVerbatim}[commandchars=\\\{\}]
\PYG{c+c1}{\PYGZsh{} coeficientes e raízes}
\PYG{n}{p} \PYG{o}{=} \PYG{n}{np}\PYG{o}{.}\PYG{n}{array}\PYG{p}{(}\PYG{p}{[}\PYG{l+m+mi}{1}\PYG{p}{,}\PYG{l+m+mi}{0}\PYG{p}{,}\PYG{o}{\PYGZhy{}}\PYG{l+m+mi}{3}\PYG{p}{,}\PYG{l+m+mi}{3}\PYG{p}{,}\PYG{l+m+mi}{0}\PYG{p}{]}\PYG{p}{)}
\PYG{n}{x} \PYG{o}{=} \PYG{n}{np}\PYG{o}{.}\PYG{n}{roots}\PYG{p}{(}\PYG{n}{p}\PYG{p}{)}
\end{sphinxVerbatim}

\begin{sphinxVerbatim}[commandchars=\\\{\}]
\PYG{c+c1}{\PYGZsh{} imprimindo as raizes}
\PYG{k}{for} \PYG{n}{i}\PYG{p}{,} \PYG{n}{v} \PYG{o+ow}{in} \PYG{n+nb}{enumerate}\PYG{p}{(}\PYG{n}{x}\PYG{p}{)}\PYG{p}{:}
    \PYG{n+nb}{print}\PYG{p}{(}\PYG{l+s+sa}{f}\PYG{l+s+s1}{\PYGZsq{}}\PYG{l+s+s1}{Raiz }\PYG{l+s+si}{\PYGZob{}}\PYG{n}{i}\PYG{l+s+si}{\PYGZcb{}}\PYG{l+s+s1}{: }\PYG{l+s+si}{\PYGZob{}}\PYG{n}{v}\PYG{l+s+si}{\PYGZcb{}}\PYG{l+s+s1}{\PYGZsq{}}\PYG{p}{)}
\end{sphinxVerbatim}

\begin{sphinxVerbatim}[commandchars=\\\{\}]
Raiz 0: (\PYGZhy{}2.1038034027355357+0j)
Raiz 1: (1.051901701367768+0.5652358516771712j)
Raiz 2: (1.051901701367768\PYGZhy{}0.5652358516771712j)
Raiz 3: 0j
\end{sphinxVerbatim}

\sphinxAtStartPar
Note que, neste caso, as raízes são complexas.


\subsection{Problema 3}
\label{\detokenize{codeSession-3-polyval:problema-3}}
\sphinxAtStartPar
Determine as raízes de \(P(x) = x^5 - 30x^4 + 361x^3 - 2178x^2 + 6588x - 7992\).


\subsubsection{Resolução}
\label{\detokenize{codeSession-3-polyval:id2}}
\begin{sphinxVerbatim}[commandchars=\\\{\}]
\PYG{c+c1}{\PYGZsh{} coeficientes e raízes}
\PYG{n}{p} \PYG{o}{=} \PYG{n}{np}\PYG{o}{.}\PYG{n}{array}\PYG{p}{(}\PYG{p}{[}\PYG{l+m+mi}{1}\PYG{p}{,}\PYG{o}{\PYGZhy{}}\PYG{l+m+mi}{30}\PYG{p}{,}\PYG{l+m+mi}{361}\PYG{p}{,}\PYG{o}{\PYGZhy{}}\PYG{l+m+mi}{2178}\PYG{p}{,}\PYG{l+m+mi}{6588}\PYG{p}{,}\PYG{o}{\PYGZhy{}}\PYG{l+m+mi}{7992}\PYG{p}{]}\PYG{p}{)}
\PYG{n}{x} \PYG{o}{=} \PYG{n}{np}\PYG{o}{.}\PYG{n}{roots}\PYG{p}{(}\PYG{n}{p}\PYG{p}{)}
\end{sphinxVerbatim}

\begin{sphinxVerbatim}[commandchars=\\\{\}]
\PYG{c+c1}{\PYGZsh{} imprimindo as raizes}
\PYG{k}{for} \PYG{n}{i}\PYG{p}{,} \PYG{n}{v} \PYG{o+ow}{in} \PYG{n+nb}{enumerate}\PYG{p}{(}\PYG{n}{x}\PYG{p}{)}\PYG{p}{:}
    \PYG{n+nb}{print}\PYG{p}{(}\PYG{l+s+sa}{f}\PYG{l+s+s1}{\PYGZsq{}}\PYG{l+s+s1}{Raiz }\PYG{l+s+si}{\PYGZob{}}\PYG{n}{i}\PYG{l+s+si}{\PYGZcb{}}\PYG{l+s+s1}{: }\PYG{l+s+si}{\PYGZob{}}\PYG{n}{v}\PYG{l+s+si}{\PYGZcb{}}\PYG{l+s+s1}{\PYGZsq{}}\PYG{p}{)}
\end{sphinxVerbatim}

\begin{sphinxVerbatim}[commandchars=\\\{\}]
Raiz 0: (6.000000000009944+0.9999999999996999j)
Raiz 1: (6.000000000009944\PYGZhy{}0.9999999999996999j)
Raiz 2: (6.00026575921113+0j)
Raiz 3: (5.999867120384507+0.0002301556526862668j)
Raiz 4: (5.999867120384507\PYGZhy{}0.0002301556526862668j)
\end{sphinxVerbatim}


\chapter{Code session 4}
\label{\detokenize{codeSession-4-fsolve:code-session-4}}\label{\detokenize{codeSession-4-fsolve::doc}}

\section{\sphinxstyleliteralintitle{\sphinxupquote{fsolve}}}
\label{\detokenize{codeSession-4-fsolve:fsolve}}
\sphinxAtStartPar
A função \sphinxcode{\sphinxupquote{fsolve}} do submódulo \sphinxcode{\sphinxupquote{scipy.optimize}} pode ser usada como método geral para busca de raízes de equações não\sphinxhyphen{}lineares escalares ou vetoriais.

\sphinxAtStartPar
Para usar \sphinxcode{\sphinxupquote{fsolve}} em uma equação escalar, precisamos de, no mínimo:
\begin{itemize}
\item {} 
\sphinxAtStartPar
uma função que possui pelo menos um argumento

\item {} 
\sphinxAtStartPar
estimativa inicial para a raiz

\end{itemize}

\sphinxAtStartPar
Para equações vetoriais (sistemas), precisamos de mais argumentos. Vejamos o exemplo do paraquedista:

\begin{sphinxVerbatim}[commandchars=\\\{\}]
\PYG{k+kn}{import} \PYG{n+nn}{numpy} \PYG{k}{as} \PYG{n+nn}{np}
\PYG{k+kn}{import} \PYG{n+nn}{matplotlib}\PYG{n+nn}{.}\PYG{n+nn}{pyplot} \PYG{k}{as} \PYG{n+nn}{plt} 
\PYG{k+kn}{from} \PYG{n+nn}{scipy}\PYG{n+nn}{.}\PYG{n+nn}{optimize} \PYG{k+kn}{import} \PYG{n}{fsolve} 

\PYG{n}{t} \PYG{o}{=} \PYG{l+m+mf}{12.0}
\PYG{n}{v} \PYG{o}{=} \PYG{l+m+mf}{42.0}
\PYG{n}{m} \PYG{o}{=} \PYG{l+m+mf}{70.0}
\PYG{n}{g} \PYG{o}{=} \PYG{l+m+mf}{9.81}

\PYG{k}{def} \PYG{n+nf}{param}\PYG{p}{(}\PYG{n}{t}\PYG{p}{,}\PYG{n}{v}\PYG{p}{,}\PYG{n}{m}\PYG{p}{,}\PYG{n}{g}\PYG{p}{)}\PYG{p}{:}    
    \PYG{k}{return} \PYG{p}{[}\PYG{n}{t}\PYG{p}{,}\PYG{n}{v}\PYG{p}{,}\PYG{n}{m}\PYG{p}{,}\PYG{n}{g}\PYG{p}{]}

\PYG{k}{def} \PYG{n+nf}{fun}\PYG{p}{(}\PYG{n}{c}\PYG{p}{)}\PYG{p}{:}
    \PYG{n}{p} \PYG{o}{=} \PYG{n}{param}\PYG{p}{(}\PYG{n}{t}\PYG{p}{,}\PYG{n}{v}\PYG{p}{,}\PYG{n}{m}\PYG{p}{,}\PYG{n}{g}\PYG{p}{)}
    \PYG{k}{return} \PYG{n}{p}\PYG{p}{[}\PYG{l+m+mi}{3}\PYG{p}{]}\PYG{o}{*}\PYG{n}{p}\PYG{p}{[}\PYG{l+m+mi}{2}\PYG{p}{]}\PYG{o}{/}\PYG{n}{c}\PYG{o}{*}\PYG{p}{(}\PYG{l+m+mi}{1} \PYG{o}{\PYGZhy{}} \PYG{n}{np}\PYG{o}{.}\PYG{n}{exp}\PYG{p}{(}\PYG{o}{\PYGZhy{}}\PYG{n}{c}\PYG{o}{/}\PYG{n}{p}\PYG{p}{[}\PYG{l+m+mi}{2}\PYG{p}{]}\PYG{o}{*}\PYG{n}{p}\PYG{p}{[}\PYG{l+m+mi}{0}\PYG{p}{]}\PYG{p}{)}\PYG{p}{)} \PYG{o}{\PYGZhy{}} \PYG{n}{p}\PYG{p}{[}\PYG{l+m+mi}{1}\PYG{p}{]}

\PYG{c+c1}{\PYGZsh{} estimativa inicial }
\PYG{n}{c0} \PYG{o}{=} \PYG{o}{\PYGZhy{}}\PYG{l+m+mf}{1000.0}

\PYG{c+c1}{\PYGZsh{} raiz }
\PYG{n}{c\PYGZus{}raiz} \PYG{o}{=} \PYG{n}{fsolve}\PYG{p}{(}\PYG{n}{fun}\PYG{p}{,}\PYG{n}{c0}\PYG{p}{)}

\PYG{c+c1}{\PYGZsh{} impressao (estilo Python 2)}
\PYG{n+nb}{print}\PYG{p}{(}\PYG{l+s+s1}{\PYGZsq{}}\PYG{l+s+s1}{Minha raiz é }\PYG{l+s+si}{\PYGZpc{}.6f}\PYG{l+s+s1}{\PYGZsq{}} \PYG{o}{\PYGZpc{}} \PYG{n}{c\PYGZus{}raiz}\PYG{p}{)}

\PYG{c+c1}{\PYGZsh{} impressao (estilo Python 3)}
\PYG{n+nb}{print}\PYG{p}{(}\PYG{l+s+s2}{\PYGZdq{}}\PYG{l+s+s2}{Minha raiz é }\PYG{l+s+si}{\PYGZob{}0:.6f\PYGZcb{}}\PYG{l+s+s2}{\PYGZdq{}}\PYG{o}{.}\PYG{n}{format}\PYG{p}{(}\PYG{n}{c\PYGZus{}raiz}\PYG{p}{[}\PYG{l+m+mi}{0}\PYG{p}{]}\PYG{p}{)}\PYG{p}{)}
\end{sphinxVerbatim}

\begin{sphinxVerbatim}[commandchars=\\\{\}]
Minha raiz é 15.127432
Minha raiz é 15.127432
\end{sphinxVerbatim}


\subsection{Como incorporar tudo em uma só função}
\label{\detokenize{codeSession-4-fsolve:como-incorporar-tudo-em-uma-so-funcao}}
\begin{sphinxVerbatim}[commandchars=\\\{\}]
\PYG{k}{def} \PYG{n+nf}{minha\PYGZus{}fun}\PYG{p}{(}\PYG{n}{t}\PYG{p}{,}\PYG{n}{v}\PYG{p}{,}\PYG{n}{m}\PYG{p}{,}\PYG{n}{g}\PYG{p}{,}\PYG{n}{c0}\PYG{p}{)}\PYG{p}{:}
    
    \PYG{n}{p} \PYG{o}{=} \PYG{p}{[}\PYG{n}{t}\PYG{p}{,}\PYG{n}{v}\PYG{p}{,}\PYG{n}{m}\PYG{p}{,}\PYG{n}{g}\PYG{p}{]} 
    \PYG{n}{f} \PYG{o}{=} \PYG{k}{lambda} \PYG{n}{c}\PYG{p}{:} \PYG{n}{p}\PYG{p}{[}\PYG{l+m+mi}{3}\PYG{p}{]}\PYG{o}{*}\PYG{n}{p}\PYG{p}{[}\PYG{l+m+mi}{2}\PYG{p}{]}\PYG{o}{/}\PYG{n}{c}\PYG{o}{*}\PYG{p}{(}\PYG{l+m+mi}{1} \PYG{o}{\PYGZhy{}} \PYG{n}{np}\PYG{o}{.}\PYG{n}{exp}\PYG{p}{(}\PYG{o}{\PYGZhy{}}\PYG{n}{c}\PYG{o}{/}\PYG{n}{p}\PYG{p}{[}\PYG{l+m+mi}{2}\PYG{p}{]}\PYG{o}{*}\PYG{n}{p}\PYG{p}{[}\PYG{l+m+mi}{0}\PYG{p}{]}\PYG{p}{)}\PYG{p}{)} \PYG{o}{\PYGZhy{}} \PYG{n}{p}\PYG{p}{[}\PYG{l+m+mi}{1}\PYG{p}{]}
    \PYG{n}{c\PYGZus{}raiz} \PYG{o}{=} \PYG{n}{fsolve}\PYG{p}{(}\PYG{n}{f}\PYG{p}{,}\PYG{n}{c0}\PYG{p}{)}
    \PYG{n+nb}{print}\PYG{p}{(}\PYG{l+s+s2}{\PYGZdq{}}\PYG{l+s+s2}{\PYGZhy{}\PYGZhy{}\PYGZhy{}\PYGZgt{} Minha raiz é }\PYG{l+s+si}{\PYGZob{}0:.6f\PYGZcb{}}\PYG{l+s+s2}{\PYGZdq{}}\PYG{o}{.}\PYG{n}{format}\PYG{p}{(}\PYG{n}{c\PYGZus{}raiz}\PYG{p}{[}\PYG{l+m+mi}{0}\PYG{p}{]}\PYG{p}{)}\PYG{p}{)}
    \PYG{k}{return} \PYG{n}{f}\PYG{p}{,}\PYG{n}{c\PYGZus{}raiz}
\end{sphinxVerbatim}

\begin{sphinxVerbatim}[commandchars=\\\{\}]
\PYG{n}{fc}\PYG{p}{,}\PYG{n}{c\PYGZus{}raiz} \PYG{o}{=} \PYG{n}{minha\PYGZus{}fun}\PYG{p}{(}\PYG{n}{t}\PYG{p}{,}\PYG{n}{v}\PYG{p}{,}\PYG{n}{m}\PYG{p}{,}\PYG{n}{g}\PYG{p}{,}\PYG{n}{c0}\PYG{p}{)}
\end{sphinxVerbatim}

\begin{sphinxVerbatim}[commandchars=\\\{\}]
\PYGZhy{}\PYGZhy{}\PYGZhy{}\PYGZgt{} Minha raiz é 15.127432
\end{sphinxVerbatim}

\begin{sphinxVerbatim}[commandchars=\\\{\}]
\PYG{n}{a}\PYG{p}{,}\PYG{n}{b} \PYG{o}{=} \PYG{l+m+mi}{14}\PYG{p}{,}\PYG{l+m+mi}{16}
\PYG{n}{c} \PYG{o}{=} \PYG{n}{np}\PYG{o}{.}\PYG{n}{linspace}\PYG{p}{(}\PYG{n}{a}\PYG{p}{,}\PYG{n}{b}\PYG{p}{,}\PYG{l+m+mi}{100}\PYG{p}{)}

\PYG{n}{delta} \PYG{o}{=} \PYG{l+m+mf}{0.1}
\PYG{n}{plt}\PYG{o}{.}\PYG{n}{plot}\PYG{p}{(}\PYG{n}{c}\PYG{p}{,}\PYG{n}{fc}\PYG{p}{(}\PYG{n}{c}\PYG{p}{)}\PYG{p}{,}\PYG{n}{c}\PYG{p}{,}\PYG{l+m+mi}{0}\PYG{o}{*}\PYG{n}{c}\PYG{p}{,}\PYG{l+s+s1}{\PYGZsq{}}\PYG{l+s+s1}{r\PYGZhy{}\PYGZhy{}}\PYG{l+s+s1}{\PYGZsq{}}\PYG{p}{,}\PYG{n}{c\PYGZus{}raiz}\PYG{p}{,}\PYG{n}{fc}\PYG{p}{(}\PYG{n}{c\PYGZus{}raiz}\PYG{p}{)}\PYG{p}{,}\PYG{l+s+s1}{\PYGZsq{}}\PYG{l+s+s1}{o}\PYG{l+s+s1}{\PYGZsq{}}\PYG{p}{)}\PYG{p}{;}
\PYG{n}{plt}\PYG{o}{.}\PYG{n}{axvline}\PYG{p}{(}\PYG{n}{c\PYGZus{}raiz} \PYG{o}{\PYGZhy{}} \PYG{n}{delta}\PYG{p}{,}\PYG{n}{c}\PYG{o}{=}\PYG{l+s+s1}{\PYGZsq{}}\PYG{l+s+s1}{k}\PYG{l+s+s1}{\PYGZsq{}}\PYG{p}{,}\PYG{n}{ls}\PYG{o}{=}\PYG{l+s+s1}{\PYGZsq{}}\PYG{l+s+s1}{\PYGZhy{}\PYGZhy{}}\PYG{l+s+s1}{\PYGZsq{}}\PYG{p}{)}\PYG{p}{;}
\PYG{n}{plt}\PYG{o}{.}\PYG{n}{axvline}\PYG{p}{(}\PYG{n}{c\PYGZus{}raiz} \PYG{o}{+} \PYG{n}{delta}\PYG{p}{,}\PYG{n}{c}\PYG{o}{=}\PYG{l+s+s1}{\PYGZsq{}}\PYG{l+s+s1}{k}\PYG{l+s+s1}{\PYGZsq{}}\PYG{p}{,}\PYG{n}{ls}\PYG{o}{=}\PYG{l+s+s1}{\PYGZsq{}}\PYG{l+s+s1}{\PYGZhy{}\PYGZhy{}}\PYG{l+s+s1}{\PYGZsq{}}\PYG{p}{)}\PYG{p}{;}
\end{sphinxVerbatim}

\noindent\sphinxincludegraphics{{codeSession-4-fsolve_6_0}.png}


\subsection{Problema 1}
\label{\detokenize{codeSession-4-fsolve:problema-1}}
\sphinxAtStartPar
Resolva o sistema não\sphinxhyphen{}linear abaixo:
\begin{equation*}
\begin{split}\begin{cases}
x^2 + y^2 = 2 \\
x^2 - y^2 = 1
\end{cases}\end{split}
\end{equation*}

\subsubsection{Resolução}
\label{\detokenize{codeSession-4-fsolve:resolucao}}
\sphinxAtStartPar
Primeiramente, vamos plotar as curvas de nível 0 das funções que compõem o sistema. Faremos isto usando uma \sphinxstyleemphasis{grade numérica} e a função \sphinxcode{\sphinxupquote{contour}}.


\paragraph{Criando uma grade numérica bidimensional uniforme}
\label{\detokenize{codeSession-4-fsolve:criando-uma-grade-numerica-bidimensional-uniforme}}
\sphinxAtStartPar
Uma \sphinxstyleemphasis{grade numérica} é um conjunto de pontos separados por uma distância uniforme (ou variável). Neste caso em particular, criaremos uma \sphinxstyleemphasis{grade numérica} bidimensional. Podemos fazer isto da seguinte forma:

\begin{sphinxVerbatim}[commandchars=\\\{\}]
\PYG{k+kn}{import} \PYG{n+nn}{numpy} \PYG{k}{as} \PYG{n+nn}{np}
\PYG{k+kn}{import} \PYG{n+nn}{matplotlib}\PYG{n+nn}{.}\PYG{n+nn}{pyplot} \PYG{k}{as} \PYG{n+nn}{plt}

\PYG{c+c1}{\PYGZsh{} limites do domínio:}
\PYG{c+c1}{\PYGZsh{} região do plano [a,b] x [c,d]}
\PYG{n}{a}\PYG{p}{,} \PYG{n}{b} \PYG{o}{=} \PYG{o}{\PYGZhy{}}\PYG{l+m+mf}{4.0}\PYG{p}{,} \PYG{l+m+mf}{5.0}
\PYG{n}{c}\PYG{p}{,} \PYG{n}{d} \PYG{o}{=} \PYG{o}{\PYGZhy{}}\PYG{l+m+mf}{2.0}\PYG{p}{,} \PYG{l+m+mf}{3.0}

\PYG{c+c1}{\PYGZsh{} no. de pontos em cada direção}
\PYG{n}{nx}\PYG{p}{,} \PYG{n}{ny} \PYG{o}{=} \PYG{l+m+mi}{10}\PYG{p}{,} \PYG{l+m+mi}{20} 

\PYG{c+c1}{\PYGZsh{} distribuição dos pontos}
\PYG{n}{x} \PYG{o}{=} \PYG{n}{np}\PYG{o}{.}\PYG{n}{linspace}\PYG{p}{(}\PYG{n}{a}\PYG{p}{,}\PYG{n}{b}\PYG{p}{,}\PYG{n}{nx}\PYG{p}{)}
\PYG{n}{y} \PYG{o}{=} \PYG{n}{np}\PYG{o}{.}\PYG{n}{linspace}\PYG{p}{(}\PYG{n}{c}\PYG{p}{,}\PYG{n}{d}\PYG{p}{,}\PYG{n}{ny}\PYG{p}{)}

\PYG{c+c1}{\PYGZsh{} grade numérica 2D}
\PYG{p}{[}\PYG{n}{X}\PYG{p}{,}\PYG{n}{Y}\PYG{p}{]} \PYG{o}{=} \PYG{n}{np}\PYG{o}{.}\PYG{n}{meshgrid}\PYG{p}{(}\PYG{n}{x}\PYG{p}{,}\PYG{n}{y}\PYG{p}{)}

\PYG{c+c1}{\PYGZsh{} plotando pontos da grade numérica}
\PYG{n}{plt}\PYG{o}{.}\PYG{n}{scatter}\PYG{p}{(}\PYG{n}{X}\PYG{p}{,}\PYG{n}{Y}\PYG{p}{,}\PYG{n}{s}\PYG{o}{=}\PYG{l+m+mi}{3}\PYG{p}{,}\PYG{n}{c}\PYG{o}{=}\PYG{l+s+s1}{\PYGZsq{}}\PYG{l+s+s1}{k}\PYG{l+s+s1}{\PYGZsq{}}\PYG{p}{)}\PYG{p}{;}
\PYG{n}{plt}\PYG{o}{.}\PYG{n}{title}\PYG{p}{(}\PYG{l+s+s1}{\PYGZsq{}}\PYG{l+s+s1}{Grade numérica 2D: pontos (x,y)}\PYG{l+s+s1}{\PYGZsq{}}\PYG{p}{)}
\PYG{n}{plt}\PYG{o}{.}\PYG{n}{xlabel}\PYG{p}{(}\PYG{l+s+s1}{\PYGZsq{}}\PYG{l+s+s1}{x}\PYG{l+s+s1}{\PYGZsq{}}\PYG{p}{)}\PYG{p}{;} \PYG{n}{plt}\PYG{o}{.}\PYG{n}{ylabel}\PYG{p}{(}\PYG{l+s+s1}{\PYGZsq{}}\PYG{l+s+s1}{y}\PYG{l+s+s1}{\PYGZsq{}}\PYG{p}{)}\PYG{p}{;}
\end{sphinxVerbatim}

\noindent\sphinxincludegraphics{{codeSession-4-fsolve_12_0}.png}


\paragraph{Plotando curvas de nível}
\label{\detokenize{codeSession-4-fsolve:plotando-curvas-de-nivel}}
\sphinxAtStartPar
Plotaremos as curvas de nível 0 das funções não\sphinxhyphen{}lineares para realizar a análise gráfica e localizar as raízes para então escolhermos um vetor de estimativa inicial.

\sphinxAtStartPar
Para plotar curvas de nível das funções sobre a grade numérica anterior, fazemos o seguinte:

\begin{sphinxVerbatim}[commandchars=\\\{\}]
\PYG{c+c1}{\PYGZsh{} funções definidas sobre a grade 2D}
\PYG{n}{F} \PYG{o}{=} \PYG{n}{X}\PYG{o}{*}\PYG{o}{*}\PYG{l+m+mi}{2} \PYG{o}{+} \PYG{n}{Y}\PYG{o}{*}\PYG{o}{*}\PYG{l+m+mi}{2} \PYG{o}{\PYGZhy{}} \PYG{l+m+mi}{2}
\PYG{n}{G} \PYG{o}{=} \PYG{n}{X}\PYG{o}{*}\PYG{o}{*}\PYG{l+m+mi}{2} \PYG{o}{\PYGZhy{}} \PYG{n}{Y}\PYG{o}{*}\PYG{o}{*}\PYG{l+m+mi}{2} \PYG{o}{\PYGZhy{}} \PYG{l+m+mi}{1}

\PYG{c+c1}{\PYGZsh{} contorno de nível 0}
\PYG{n}{plt}\PYG{o}{.}\PYG{n}{contour}\PYG{p}{(}\PYG{n}{X}\PYG{p}{,}\PYG{n}{Y}\PYG{p}{,}\PYG{n}{F}\PYG{p}{,}\PYG{n}{colors}\PYG{o}{=}\PYG{l+s+s1}{\PYGZsq{}}\PYG{l+s+s1}{red}\PYG{l+s+s1}{\PYGZsq{}}\PYG{p}{,}\PYG{n}{levels}\PYG{o}{=}\PYG{l+m+mi}{0}\PYG{p}{)}\PYG{p}{;}
\PYG{n}{plt}\PYG{o}{.}\PYG{n}{contour}\PYG{p}{(}\PYG{n}{X}\PYG{p}{,}\PYG{n}{Y}\PYG{p}{,}\PYG{n}{G}\PYG{p}{,}\PYG{n}{colors}\PYG{o}{=}\PYG{l+s+s1}{\PYGZsq{}}\PYG{l+s+s1}{blue}\PYG{l+s+s1}{\PYGZsq{}}\PYG{p}{,}\PYG{n}{levels}\PYG{o}{=}\PYG{l+m+mi}{0}\PYG{p}{)}\PYG{p}{;}
\PYG{n}{plt}\PYG{o}{.}\PYG{n}{grid}\PYG{p}{(}\PYG{p}{)}
\end{sphinxVerbatim}

\noindent\sphinxincludegraphics{{codeSession-4-fsolve_14_0}.png}

\sphinxAtStartPar
Por que a figura está meio “tosca”? Porque temos poucos pontos na grade. Vamos aumentar o número de pontos. Este processo é conhecido como \sphinxstyleemphasis{refinamento de malha}.

\begin{sphinxVerbatim}[commandchars=\\\{\}]
\PYG{c+c1}{\PYGZsh{} refinando a malha numérica}
\PYG{n}{nx2}\PYG{p}{,} \PYG{n}{ny2} \PYG{o}{=} \PYG{l+m+mi}{100}\PYG{p}{,} \PYG{l+m+mi}{200}

\PYG{c+c1}{\PYGZsh{} redistribuição dos pontos}
\PYG{n}{x2} \PYG{o}{=} \PYG{n}{np}\PYG{o}{.}\PYG{n}{linspace}\PYG{p}{(}\PYG{n}{a}\PYG{p}{,}\PYG{n}{b}\PYG{p}{,}\PYG{n}{nx2}\PYG{p}{)}
\PYG{n}{y2} \PYG{o}{=} \PYG{n}{np}\PYG{o}{.}\PYG{n}{linspace}\PYG{p}{(}\PYG{n}{c}\PYG{p}{,}\PYG{n}{d}\PYG{p}{,}\PYG{n}{ny2}\PYG{p}{)}

\PYG{c+c1}{\PYGZsh{} grade numérica 2D refinada}
\PYG{p}{[}\PYG{n}{X2}\PYG{p}{,}\PYG{n}{Y2}\PYG{p}{]} \PYG{o}{=} \PYG{n}{np}\PYG{o}{.}\PYG{n}{meshgrid}\PYG{p}{(}\PYG{n}{x2}\PYG{p}{,}\PYG{n}{y2}\PYG{p}{)}

\PYG{c+c1}{\PYGZsh{} plotando pontos da grade numérica}
\PYG{n}{plt}\PYG{o}{.}\PYG{n}{scatter}\PYG{p}{(}\PYG{n}{X2}\PYG{p}{,}\PYG{n}{Y2}\PYG{p}{,}\PYG{n}{s}\PYG{o}{=}\PYG{l+m+mf}{0.1}\PYG{p}{,}\PYG{n}{c}\PYG{o}{=}\PYG{l+s+s1}{\PYGZsq{}}\PYG{l+s+s1}{b}\PYG{l+s+s1}{\PYGZsq{}}\PYG{p}{)}\PYG{p}{;}
\PYG{n}{plt}\PYG{o}{.}\PYG{n}{title}\PYG{p}{(}\PYG{l+s+s1}{\PYGZsq{}}\PYG{l+s+s1}{Grade numérica 2D refinada: muitos pontos (x,y)}\PYG{l+s+s1}{\PYGZsq{}}\PYG{p}{)}
\PYG{n}{plt}\PYG{o}{.}\PYG{n}{xlabel}\PYG{p}{(}\PYG{l+s+s1}{\PYGZsq{}}\PYG{l+s+s1}{x}\PYG{l+s+s1}{\PYGZsq{}}\PYG{p}{)}\PYG{p}{;} \PYG{n}{plt}\PYG{o}{.}\PYG{n}{ylabel}\PYG{p}{(}\PYG{l+s+s1}{\PYGZsq{}}\PYG{l+s+s1}{y}\PYG{l+s+s1}{\PYGZsq{}}\PYG{p}{)}\PYG{p}{;}
\end{sphinxVerbatim}

\noindent\sphinxincludegraphics{{codeSession-4-fsolve_16_0}.png}

\sphinxAtStartPar
Vamos plotar novamente as curvas de nível das funções sobre a grade numérica refinada.

\begin{sphinxVerbatim}[commandchars=\\\{\}]
\PYG{c+c1}{\PYGZsh{} funções definidas sobre a grade 2D refinada}
\PYG{n}{F2} \PYG{o}{=} \PYG{n}{X2}\PYG{o}{*}\PYG{o}{*}\PYG{l+m+mi}{2} \PYG{o}{+} \PYG{n}{Y2}\PYG{o}{*}\PYG{o}{*}\PYG{l+m+mi}{2} \PYG{o}{\PYGZhy{}} \PYG{l+m+mi}{2}
\PYG{n}{G2} \PYG{o}{=} \PYG{n}{X2}\PYG{o}{*}\PYG{o}{*}\PYG{l+m+mi}{2} \PYG{o}{\PYGZhy{}} \PYG{n}{Y2}\PYG{o}{*}\PYG{o}{*}\PYG{l+m+mi}{2} \PYG{o}{\PYGZhy{}} \PYG{l+m+mi}{1}

\PYG{c+c1}{\PYGZsh{} contorno de nível 0 na malha refinada}
\PYG{n}{plt}\PYG{o}{.}\PYG{n}{contour}\PYG{p}{(}\PYG{n}{X2}\PYG{p}{,}\PYG{n}{Y2}\PYG{p}{,}\PYG{n}{F2}\PYG{p}{,}\PYG{n}{colors}\PYG{o}{=}\PYG{l+s+s1}{\PYGZsq{}}\PYG{l+s+s1}{red}\PYG{l+s+s1}{\PYGZsq{}}\PYG{p}{,}\PYG{n}{levels}\PYG{o}{=}\PYG{l+m+mi}{0}\PYG{p}{)}\PYG{p}{;}
\PYG{n}{plt}\PYG{o}{.}\PYG{n}{contour}\PYG{p}{(}\PYG{n}{X2}\PYG{p}{,}\PYG{n}{Y2}\PYG{p}{,}\PYG{n}{G2}\PYG{p}{,}\PYG{n}{colors}\PYG{o}{=}\PYG{l+s+s1}{\PYGZsq{}}\PYG{l+s+s1}{blue}\PYG{l+s+s1}{\PYGZsq{}}\PYG{p}{,}\PYG{n}{levels}\PYG{o}{=}\PYG{l+m+mi}{0}\PYG{p}{)}\PYG{p}{;}
\PYG{n}{plt}\PYG{o}{.}\PYG{n}{grid}\PYG{p}{(}\PYG{p}{)}
\end{sphinxVerbatim}

\noindent\sphinxincludegraphics{{codeSession-4-fsolve_18_0}.png}


\paragraph{Estimativa inicial}
\label{\detokenize{codeSession-4-fsolve:estimativa-inicial}}
\sphinxAtStartPar
A partir do gráfico anterior, vemos que há 4 raízes possíveis para o sistema não\sphinxhyphen{}linear. Vamos escolher uma delas para aproximar. Por conveniência, escolhamos a que se encontra no primeiro quadrante.

\sphinxAtStartPar
Vamos fazer uma plotagem localizada no primeiro quadrante:

\begin{sphinxVerbatim}[commandchars=\\\{\}]
\PYG{c+c1}{\PYGZsh{} contorno de nível 0 na malha refinada}
\PYG{n}{plt}\PYG{o}{.}\PYG{n}{contour}\PYG{p}{(}\PYG{n}{X2}\PYG{p}{,}\PYG{n}{Y2}\PYG{p}{,}\PYG{n}{F2}\PYG{p}{,}\PYG{n}{colors}\PYG{o}{=}\PYG{l+s+s1}{\PYGZsq{}}\PYG{l+s+s1}{red}\PYG{l+s+s1}{\PYGZsq{}}\PYG{p}{,}\PYG{n}{levels}\PYG{o}{=}\PYG{l+m+mi}{0}\PYG{p}{)}\PYG{p}{;}
\PYG{n}{plt}\PYG{o}{.}\PYG{n}{contour}\PYG{p}{(}\PYG{n}{X2}\PYG{p}{,}\PYG{n}{Y2}\PYG{p}{,}\PYG{n}{G2}\PYG{p}{,}\PYG{n}{colors}\PYG{o}{=}\PYG{l+s+s1}{\PYGZsq{}}\PYG{l+s+s1}{blue}\PYG{l+s+s1}{\PYGZsq{}}\PYG{p}{,}\PYG{n}{levels}\PYG{o}{=}\PYG{l+m+mi}{0}\PYG{p}{)}\PYG{p}{;}
\PYG{n}{plt}\PYG{o}{.}\PYG{n}{xlim}\PYG{p}{(}\PYG{p}{[}\PYG{l+m+mf}{1.0}\PYG{p}{,}\PYG{l+m+mf}{1.4}\PYG{p}{]}\PYG{p}{)}
\PYG{n}{plt}\PYG{o}{.}\PYG{n}{ylim}\PYG{p}{(}\PYG{p}{[}\PYG{l+m+mf}{0.5}\PYG{p}{,}\PYG{l+m+mf}{1.0}\PYG{p}{]}\PYG{p}{)}
\PYG{n}{plt}\PYG{o}{.}\PYG{n}{xlabel}\PYG{p}{(}\PYG{l+s+s1}{\PYGZsq{}}\PYG{l+s+s1}{x}\PYG{l+s+s1}{\PYGZsq{}}\PYG{p}{)}\PYG{p}{;} \PYG{n}{plt}\PYG{o}{.}\PYG{n}{ylabel}\PYG{p}{(}\PYG{l+s+s1}{\PYGZsq{}}\PYG{l+s+s1}{y}\PYG{l+s+s1}{\PYGZsq{}}\PYG{p}{)}\PYG{p}{;} \PYG{n}{plt}\PYG{o}{.}\PYG{n}{grid}\PYG{p}{(}\PYG{p}{)}
\PYG{n}{plt}\PYG{o}{.}\PYG{n}{title}\PYG{p}{(}\PYG{l+s+s1}{\PYGZsq{}}\PYG{l+s+s1}{Localização da raiz do 1o. quadrante}\PYG{l+s+s1}{\PYGZsq{}}\PYG{p}{)}\PYG{p}{;}
\end{sphinxVerbatim}

\noindent\sphinxincludegraphics{{codeSession-4-fsolve_20_0}.png}

\sphinxAtStartPar
Observando o gráfico, faremos a escolha do ponto \((x_0,y_0) = (1.2,0.7)\) como estimativa inicial.


\paragraph{Resolução do sistema não\sphinxhyphen{}linear}
\label{\detokenize{codeSession-4-fsolve:resolucao-do-sistema-nao-linear}}
\sphinxAtStartPar
Para resolver o sistema não\sphinxhyphen{}linear, primeiro definimos uma função que retornará uma tupla contendo cada função do sistema em cada uma de suas coordenadas.

\begin{sphinxVerbatim}[commandchars=\\\{\}]
\PYG{c+c1}{\PYGZsh{} função que returna uma lista com as funções do sistema}
\PYG{k}{def} \PYG{n+nf}{F}\PYG{p}{(}\PYG{n+nb}{vars}\PYG{p}{)}\PYG{p}{:}
    \PYG{n}{x}\PYG{p}{,}\PYG{n}{y} \PYG{o}{=} \PYG{n+nb}{vars} \PYG{c+c1}{\PYGZsh{} cria x,y como variáveis locais}
    \PYG{n}{f} \PYG{o}{=} \PYG{n}{x}\PYG{o}{*}\PYG{o}{*}\PYG{l+m+mi}{2} \PYG{o}{+} \PYG{n}{y}\PYG{o}{*}\PYG{o}{*}\PYG{l+m+mi}{2} \PYG{o}{\PYGZhy{}} \PYG{l+m+mi}{2} \PYG{c+c1}{\PYGZsh{} f(x,y) = 0}
    \PYG{n}{g} \PYG{o}{=} \PYG{n}{x}\PYG{o}{*}\PYG{o}{*}\PYG{l+m+mi}{2} \PYG{o}{\PYGZhy{}} \PYG{n}{y}\PYG{o}{*}\PYG{o}{*}\PYG{l+m+mi}{2} \PYG{o}{\PYGZhy{}} \PYG{l+m+mi}{1} \PYG{c+c1}{\PYGZsh{} g(x,y) = 0}
    \PYG{k}{return} \PYG{p}{[}\PYG{n}{f}\PYG{p}{,}\PYG{n}{g}\PYG{p}{]}
\end{sphinxVerbatim}

\sphinxAtStartPar
Em seguida, usamos a função \sphinxcode{\sphinxupquote{fsolve}} passando o vetor inicial escolhido, isto é, \((x_0,y_0) = (1.2,0.7)\), para determinar a solução aproximada \((x_1,y_1)\).

\begin{sphinxVerbatim}[commandchars=\\\{\}]
\PYG{k+kn}{from} \PYG{n+nn}{scipy}\PYG{n+nn}{.}\PYG{n+nn}{optimize} \PYG{k+kn}{import} \PYG{n}{fsolve}

\PYG{n}{xr}\PYG{p}{,} \PYG{n}{yr} \PYG{o}{=} \PYG{n}{fsolve}\PYG{p}{(}\PYG{n}{F}\PYG{p}{,}\PYG{p}{(}\PYG{l+m+mf}{1.2}\PYG{p}{,}\PYG{l+m+mf}{0.7}\PYG{p}{)}\PYG{p}{)}
\PYG{n+nb}{print}\PYG{p}{(}\PYG{l+s+sa}{f}\PYG{l+s+s1}{\PYGZsq{}}\PYG{l+s+s1}{A solução aproximada é o vetor (xr,yr) = (}\PYG{l+s+si}{\PYGZob{}}\PYG{n}{xr}\PYG{l+s+si}{:}\PYG{l+s+s1}{.3f}\PYG{l+s+si}{\PYGZcb{}}\PYG{l+s+s1}{,}\PYG{l+s+si}{\PYGZob{}}\PYG{n}{yr}\PYG{l+s+si}{:}\PYG{l+s+s1}{.3f}\PYG{l+s+si}{\PYGZcb{}}\PYG{l+s+s1}{)}\PYG{l+s+s1}{\PYGZsq{}}\PYG{p}{)}
\end{sphinxVerbatim}

\begin{sphinxVerbatim}[commandchars=\\\{\}]
A solução aproximada é o vetor (xr,yr) = (1.225,0.707)
\end{sphinxVerbatim}


\paragraph{Verificação e plotagem}
\label{\detokenize{codeSession-4-fsolve:verificacao-e-plotagem}}
\sphinxAtStartPar
Podemos verificar que \(x_r\) e \(y_r\) satisfazem às equações dentro de uma certa precisão:

\begin{sphinxVerbatim}[commandchars=\\\{\}]
\PYG{n}{xr}\PYG{o}{*}\PYG{o}{*}\PYG{l+m+mi}{2} \PYG{o}{+} \PYG{n}{yr}\PYG{o}{*}\PYG{o}{*}\PYG{l+m+mi}{2} \PYG{o}{\PYGZhy{}} \PYG{l+m+mi}{2}
\end{sphinxVerbatim}

\begin{sphinxVerbatim}[commandchars=\\\{\}]
3.956968086527013e\PYGZhy{}11
\end{sphinxVerbatim}

\begin{sphinxVerbatim}[commandchars=\\\{\}]
\PYG{n}{xr}\PYG{o}{*}\PYG{o}{*}\PYG{l+m+mi}{2} \PYG{o}{\PYGZhy{}} \PYG{n}{yr}\PYG{o}{*}\PYG{o}{*}\PYG{l+m+mi}{2} \PYG{o}{\PYGZhy{}} \PYG{l+m+mi}{1}
\end{sphinxVerbatim}

\begin{sphinxVerbatim}[commandchars=\\\{\}]
\PYGZhy{}6.642031369352708e\PYGZhy{}11
\end{sphinxVerbatim}

\sphinxAtStartPar
Isto mostra que os valores estão muito próximos de zero.

\sphinxAtStartPar
Finalmente, podemos plotar as curvas de nível destacando a solução obtida.

\begin{sphinxVerbatim}[commandchars=\\\{\}]
\PYG{c+c1}{\PYGZsh{} contorno de nível 0 na malha refinada}
\PYG{n}{plt}\PYG{o}{.}\PYG{n}{contour}\PYG{p}{(}\PYG{n}{X2}\PYG{p}{,}\PYG{n}{Y2}\PYG{p}{,}\PYG{n}{F2}\PYG{p}{,}\PYG{n}{colors}\PYG{o}{=}\PYG{l+s+s1}{\PYGZsq{}}\PYG{l+s+s1}{red}\PYG{l+s+s1}{\PYGZsq{}}\PYG{p}{,}\PYG{n}{levels}\PYG{o}{=}\PYG{l+m+mi}{0}\PYG{p}{)}\PYG{p}{;}
\PYG{n}{plt}\PYG{o}{.}\PYG{n}{contour}\PYG{p}{(}\PYG{n}{X2}\PYG{p}{,}\PYG{n}{Y2}\PYG{p}{,}\PYG{n}{G2}\PYG{p}{,}\PYG{n}{colors}\PYG{o}{=}\PYG{l+s+s1}{\PYGZsq{}}\PYG{l+s+s1}{blue}\PYG{l+s+s1}{\PYGZsq{}}\PYG{p}{,}\PYG{n}{levels}\PYG{o}{=}\PYG{l+m+mi}{0}\PYG{p}{)}\PYG{p}{;}
\PYG{n}{plt}\PYG{o}{.}\PYG{n}{xlabel}\PYG{p}{(}\PYG{l+s+s1}{\PYGZsq{}}\PYG{l+s+s1}{x}\PYG{l+s+s1}{\PYGZsq{}}\PYG{p}{)}\PYG{p}{;} \PYG{n}{plt}\PYG{o}{.}\PYG{n}{ylabel}\PYG{p}{(}\PYG{l+s+s1}{\PYGZsq{}}\PYG{l+s+s1}{y}\PYG{l+s+s1}{\PYGZsq{}}\PYG{p}{)}\PYG{p}{;} \PYG{n}{plt}\PYG{o}{.}\PYG{n}{grid}\PYG{p}{(}\PYG{p}{)}
\PYG{n}{plt}\PYG{o}{.}\PYG{n}{title}\PYG{p}{(}\PYG{l+s+s1}{\PYGZsq{}}\PYG{l+s+s1}{Aproximação da raiz do 1o. quadrante}\PYG{l+s+s1}{\PYGZsq{}}\PYG{p}{)}\PYG{p}{;}
\PYG{n}{plt}\PYG{o}{.}\PYG{n}{scatter}\PYG{p}{(}\PYG{n}{xr}\PYG{p}{,}\PYG{n}{yr}\PYG{p}{,}\PYG{n}{c}\PYG{o}{=}\PYG{l+s+s1}{\PYGZsq{}}\PYG{l+s+s1}{green}\PYG{l+s+s1}{\PYGZsq{}}\PYG{p}{,}\PYG{n}{s}\PYG{o}{=}\PYG{l+m+mi}{50}\PYG{p}{)}\PYG{p}{;}
\end{sphinxVerbatim}

\noindent\sphinxincludegraphics{{codeSession-4-fsolve_31_0}.png}


\chapter{Code session 5}
\label{\detokenize{codeSession-5-solve:code-session-5}}\label{\detokenize{codeSession-5-solve::doc}}
\begin{sphinxVerbatim}[commandchars=\\\{\}]
\PYG{k+kn}{import} \PYG{n+nn}{numpy} \PYG{k}{as} \PYG{n+nn}{np}
\end{sphinxVerbatim}


\section{\sphinxstyleliteralintitle{\sphinxupquote{linalg.solve}}}
\label{\detokenize{codeSession-5-solve:linalg-solve}}
\sphinxAtStartPar
A função \sphinxcode{\sphinxupquote{solve}} é o método mais simples disponibilizado pelos módulos \sphinxcode{\sphinxupquote{numpy}} e \sphinxcode{\sphinxupquote{scipy}} para resolver sistemas matriciais lineares. Como a função pertence ao escopo da Álgebra Linear, ela está localizada em submódulo chamado \sphinxcode{\sphinxupquote{linalg}}. \sphinxcode{\sphinxupquote{solve}} calculará a solução exata do sistema como um método direto se a matriz do sistema for determinada (quadrada e sem colunas linearmente dependentes). Se a matriz for singular, o método retorna um erro. Se for de posto deficiente, o método resolve o sistema linear usando um algoritmo de mínimos quadrados.

\sphinxAtStartPar
Os argumentos de entrada obrigatórios desta função são:
\begin{enumerate}
\sphinxsetlistlabels{\arabic}{enumi}{enumii}{}{.}%
\item {} 
\sphinxAtStartPar
a matriz \sphinxcode{\sphinxupquote{A}} dos coeficientes

\item {} 
\sphinxAtStartPar
o vetor independente \sphinxcode{\sphinxupquote{b}}

\end{enumerate}

\sphinxAtStartPar
O argumento de saída é:
\begin{itemize}
\item {} 
\sphinxAtStartPar
\sphinxcode{\sphinxupquote{x}}: o vetor\sphinxhyphen{}solução do sistema.

\end{itemize}

\sphinxAtStartPar
Como importá\sphinxhyphen{}la?

\begin{sphinxVerbatim}[commandchars=\\\{\}]
\PYG{k+kn}{from} \PYG{n+nn}{numpy}\PYG{n+nn}{.}\PYG{n+nn}{linalg} \PYG{k+kn}{import} \PYG{n}{solve}
\end{sphinxVerbatim}

\begin{sphinxVerbatim}[commandchars=\\\{\}]
\PYG{k+kn}{from} \PYG{n+nn}{numpy}\PYG{n+nn}{.}\PYG{n+nn}{linalg} \PYG{k+kn}{import} \PYG{n}{solve}
\end{sphinxVerbatim}


\subsection{Problema 1}
\label{\detokenize{codeSession-5-solve:problema-1}}
\sphinxAtStartPar
Uma rede elétrica contém 3 \sphinxstyleemphasis{loops}. Ao aplicar a lei de Kirchoff a cada \sphinxstyleemphasis{loop}, o cientista Hely Johnson obteve as seguintes equações para as correntes \(i_1\), \(i_2\) e \(i_3\) em cada \sphinxstyleemphasis{loop:}
\begin{equation*}
\begin{split}\begin{eqnarray}
(50 + R)i_1 - Ri_2 - 30i_3 &=& 0 \\
- Ri_1 + (65 + R)i_2 - 15i_3 &=& 0 \\
-30i_1 - 15i_2 + 45i_3 &=& 120 
\end{eqnarray}\end{split}
\end{equation*}
\sphinxAtStartPar
Ajude Hely Johnson em seus experimentos e calcule as correntes para os valores de resistência \(R = \{ 5 \Omega, 10 \Omega, 20 \Omega \}\).


\subsection{Resolução}
\label{\detokenize{codeSession-5-solve:resolucao}}
\sphinxAtStartPar
Podemos começar definindo uma lista para armazenar os valores das resistências de teste:

\begin{sphinxVerbatim}[commandchars=\\\{\}]
\PYG{n}{R} \PYG{o}{=} \PYG{p}{[}\PYG{l+m+mf}{5.}\PYG{p}{,}\PYG{l+m+mf}{10.}\PYG{p}{,}\PYG{l+m+mf}{15.}\PYG{p}{]}
\end{sphinxVerbatim}

\sphinxAtStartPar
Em seguida, escreveremos a matriz dos coeficientes e o vetor independente (lado direito). Note que precisamos de uma “lista de listas”, ou melhor, um \sphinxstyleemphasis{“array de arrays”}, onde cada \sphinxstyleemphasis{array} está associado à uma linha da matriz.

\sphinxAtStartPar
Todavia, vamos definir uma função para montar o sistema em função do valor de \(R\) e resolvê\sphinxhyphen{}lo.

\begin{sphinxVerbatim}[commandchars=\\\{\}]
\PYG{c+c1}{\PYGZsh{} montagem do sistema}
\PYG{k}{def} \PYG{n+nf}{resolve\PYGZus{}sistema}\PYG{p}{(}\PYG{n}{R}\PYG{p}{)}\PYG{p}{:}
    \PYG{n}{A} \PYG{o}{=} \PYG{n}{np}\PYG{o}{.}\PYG{n}{array}\PYG{p}{(}\PYG{p}{[} \PYG{p}{[}\PYG{l+m+mi}{50}\PYG{o}{+}\PYG{n}{R}\PYG{p}{,}\PYG{o}{\PYGZhy{}}\PYG{n}{R}\PYG{p}{,}\PYG{o}{\PYGZhy{}}\PYG{l+m+mi}{30}\PYG{p}{]}\PYG{p}{,}\PYG{p}{[}\PYG{o}{\PYGZhy{}}\PYG{n}{R}\PYG{p}{,}\PYG{l+m+mi}{65}\PYG{o}{+}\PYG{n}{R}\PYG{p}{,}\PYG{o}{\PYGZhy{}}\PYG{l+m+mi}{15}\PYG{p}{]}\PYG{p}{,}\PYG{p}{[}\PYG{o}{\PYGZhy{}}\PYG{l+m+mi}{30}\PYG{p}{,}\PYG{o}{\PYGZhy{}}\PYG{l+m+mi}{15}\PYG{p}{,}\PYG{l+m+mi}{45}\PYG{p}{]} \PYG{p}{]}\PYG{p}{)}
    \PYG{n}{b} \PYG{o}{=} \PYG{n}{np}\PYG{o}{.}\PYG{n}{array}\PYG{p}{(}\PYG{p}{[}\PYG{l+m+mi}{0}\PYG{p}{,}\PYG{l+m+mi}{0}\PYG{p}{,}\PYG{l+m+mi}{120}\PYG{p}{]}\PYG{p}{)}
    \PYG{n}{x} \PYG{o}{=} \PYG{n}{solve}\PYG{p}{(}\PYG{n}{A}\PYG{p}{,}\PYG{n}{b}\PYG{p}{)}
    \PYG{k}{return} \PYG{n}{x}
\end{sphinxVerbatim}

\sphinxAtStartPar
Além disso, usaremos um laço \sphinxcode{\sphinxupquote{for}} para calcularmos todas as respostas necessárias de uma só vez e organizaremos os resultados em dicionário (objeto \sphinxcode{\sphinxupquote{dict}}):

\begin{sphinxVerbatim}[commandchars=\\\{\}]
\PYG{c+c1}{\PYGZsh{} salva soluções agrupadas em um dicionário}
\PYG{n}{sols} \PYG{o}{=} \PYG{p}{\PYGZob{}}\PYG{p}{\PYGZcb{}}
\PYG{k}{for} \PYG{n}{r} \PYG{o+ow}{in} \PYG{n}{R}\PYG{p}{:}
    \PYG{n}{sols}\PYG{p}{[}\PYG{n}{r}\PYG{p}{]} \PYG{o}{=} \PYG{n}{resolve\PYGZus{}sistema}\PYG{p}{(}\PYG{n}{r}\PYG{p}{)}
    \PYG{n+nb}{print}\PYG{p}{(}\PYG{l+s+s1}{\PYGZsq{}}\PYG{l+s+s1}{Para o valor de resistência R = }\PYG{l+s+si}{\PYGZob{}0:g\PYGZcb{}}\PYG{l+s+s1}{ ohms: i1 = }\PYG{l+s+si}{\PYGZob{}1:g\PYGZcb{}}\PYG{l+s+s1}{ A, i2 = }\PYG{l+s+si}{\PYGZob{}2:g\PYGZcb{}}\PYG{l+s+s1}{ A, i3 = }\PYG{l+s+si}{\PYGZob{}3:g\PYGZcb{}}\PYG{l+s+s1}{ A}\PYG{l+s+s1}{\PYGZsq{}}\PYG{o}{.}\PYG{n}{format}\PYG{p}{(}\PYG{n}{r}\PYG{p}{,} \PYG{n}{sols}\PYG{p}{[}\PYG{n}{r}\PYG{p}{]}\PYG{p}{[}\PYG{l+m+mi}{0}\PYG{p}{]}\PYG{p}{,} \PYG{n}{sols}\PYG{p}{[}\PYG{n}{r}\PYG{p}{]}\PYG{p}{[}\PYG{l+m+mi}{1}\PYG{p}{]}\PYG{p}{,} \PYG{n}{sols}\PYG{p}{[}\PYG{n}{r}\PYG{p}{]}\PYG{p}{[}\PYG{l+m+mi}{2}\PYG{p}{]}\PYG{p}{)}\PYG{p}{)}
\end{sphinxVerbatim}

\begin{sphinxVerbatim}[commandchars=\\\{\}]
Para o valor de resistência R = 5 ohms: i1 = 2.82927 A, i2 = 1.26829 A, i3 = 4.97561 A
Para o valor de resistência R = 10 ohms: i1 = 2.66667 A, i2 = 1.33333 A, i3 = 4.88889 A
Para o valor de resistência R = 15 ohms: i1 = 2.54545 A, i2 = 1.38182 A, i3 = 4.82424 A
\end{sphinxVerbatim}


\subsection{Tarefa}
\label{\detokenize{codeSession-5-solve:tarefa}}
\sphinxAtStartPar
Retorne ao notebook da {\hyperref[\detokenize{aula-09-eliminacao-gauss::doc}]{\sphinxcrossref{\DUrole{doc,std,std-doc}{Aula 09}}}} e use o conteúdo da aula para fazer uma verificação das soluções encontradas para este problema em cada caso. Use a função \sphinxcode{\sphinxupquote{linalg.cond}} para calcular o \sphinxstyleemphasis{número de condicionamento} da matriz do sistema em cada caso. (\sphinxstylestrong{Sugestão:} vide \sphinxcode{\sphinxupquote{numpy.allclose}})


\section{\sphinxstyleliteralintitle{\sphinxupquote{linalg.cholesky}}}
\label{\detokenize{codeSession-5-solve:linalg-cholesky}}
\sphinxAtStartPar
Assim como \sphinxcode{\sphinxupquote{solve}}, a função \sphinxcode{\sphinxupquote{cholesky}} está disponível tanto via \sphinxcode{\sphinxupquote{numpy}} como \sphinxcode{\sphinxupquote{scipy}} para determinar a decomposição de Cholesky de uma matriz simétrica e positiva definida.

\sphinxAtStartPar
Na prática, não é recomendável verificar se uma matriz é positiva definida através dos critérios teóricos. A função \sphinxcode{\sphinxupquote{cholesky}} não realiza essa checagem. Portanto, é importante que, pelo menos, se saiba que a matriz é simétrica. Para testar se ela atende a propriedade de definição positiva, a abordagem mais direta é usar \sphinxcode{\sphinxupquote{cholesky}} e verificar se ela retorna uma decomposição de Cholesky. Se não for o caso, um erro será lançado.

\sphinxAtStartPar
O argumento de entrada desta função é:
\begin{itemize}
\item {} 
\sphinxAtStartPar
\sphinxcode{\sphinxupquote{A}}: matriz dos coeficientes

\end{itemize}

\sphinxAtStartPar
O argumento de saída é:
\begin{itemize}
\item {} 
\sphinxAtStartPar
\sphinxcode{\sphinxupquote{L}}: o fator de Cholesky

\end{itemize}

\sphinxAtStartPar
Como importá\sphinxhyphen{}la?

\begin{sphinxVerbatim}[commandchars=\\\{\}]
\PYG{k+kn}{from} \PYG{n+nn}{numpy}\PYG{n+nn}{.}\PYG{n+nn}{linalg} \PYG{k+kn}{import} \PYG{n}{cholesky}
\end{sphinxVerbatim}

\begin{sphinxVerbatim}[commandchars=\\\{\}]
\PYG{k+kn}{from} \PYG{n+nn}{numpy}\PYG{n+nn}{.}\PYG{n+nn}{linalg} \PYG{k+kn}{import} \PYG{n}{cholesky}
\end{sphinxVerbatim}


\subsection{Problema 2}
\label{\detokenize{codeSession-5-solve:problema-2}}
\sphinxAtStartPar
Calcule o fator de Cholesky para a matriz \(A\) do Problema 1 para \(R = 5\).

\begin{sphinxVerbatim}[commandchars=\\\{\}]
\PYG{n}{R} \PYG{o}{=} \PYG{l+m+mf}{5.}
\PYG{n}{B} \PYG{o}{=} \PYG{n}{np}\PYG{o}{.}\PYG{n}{array}\PYG{p}{(}\PYG{p}{[} \PYG{p}{[}\PYG{l+m+mi}{50}\PYG{o}{+}\PYG{n}{R}\PYG{p}{,}\PYG{o}{\PYGZhy{}}\PYG{n}{R}\PYG{p}{,}\PYG{o}{\PYGZhy{}}\PYG{l+m+mi}{30}\PYG{p}{]}\PYG{p}{,}\PYG{p}{[}\PYG{o}{\PYGZhy{}}\PYG{n}{R}\PYG{p}{,}\PYG{l+m+mi}{65}\PYG{o}{+}\PYG{n}{R}\PYG{p}{,}\PYG{o}{\PYGZhy{}}\PYG{l+m+mi}{15}\PYG{p}{]}\PYG{p}{,}\PYG{p}{[}\PYG{o}{\PYGZhy{}}\PYG{l+m+mi}{30}\PYG{p}{,}\PYG{o}{\PYGZhy{}}\PYG{l+m+mi}{15}\PYG{p}{,}\PYG{l+m+mi}{45}\PYG{p}{]} \PYG{p}{]}\PYG{p}{)}

\PYG{n}{L} \PYG{o}{=} \PYG{n}{cholesky}\PYG{p}{(}\PYG{n}{B}\PYG{p}{)}
\PYG{n}{L}
\end{sphinxVerbatim}

\begin{sphinxVerbatim}[commandchars=\\\{\}]
array([[ 7.41619849,  0.        ,  0.        ],
       [\PYGZhy{}0.67419986,  8.33939174,  0.        ],
       [\PYGZhy{}4.04519917, \PYGZhy{}2.12572731,  4.91097211]])
\end{sphinxVerbatim}

\sphinxAtStartPar
Podemos verificar a decomposição multiplicando a matriz triangular (fator de Cholesky) pela sua transposta.

\begin{sphinxVerbatim}[commandchars=\\\{\}]
\PYG{n}{np}\PYG{o}{.}\PYG{n}{matmul}\PYG{p}{(}\PYG{n}{L}\PYG{p}{,}\PYG{n}{L}\PYG{o}{.}\PYG{n}{T}\PYG{p}{)}
\end{sphinxVerbatim}

\begin{sphinxVerbatim}[commandchars=\\\{\}]
array([[ 55.,  \PYGZhy{}5., \PYGZhy{}30.],
       [ \PYGZhy{}5.,  70., \PYGZhy{}15.],
       [\PYGZhy{}30., \PYGZhy{}15.,  45.]])
\end{sphinxVerbatim}

\begin{sphinxVerbatim}[commandchars=\\\{\}]
\PYG{n}{B}
\end{sphinxVerbatim}

\begin{sphinxVerbatim}[commandchars=\\\{\}]
array([[ 55.,  \PYGZhy{}5., \PYGZhy{}30.],
       [ \PYGZhy{}5.,  70., \PYGZhy{}15.],
       [\PYGZhy{}30., \PYGZhy{}15.,  45.]])
\end{sphinxVerbatim}

\sphinxAtStartPar
Entretanto, não é verdade que

\begin{sphinxVerbatim}[commandchars=\\\{\}]
\PYG{c+c1}{\PYGZsh{} por que a igualdade falha }
\PYG{c+c1}{\PYGZsh{} para algumas entradas?}
\PYG{n}{B} \PYG{o}{==} \PYG{n}{np}\PYG{o}{.}\PYG{n}{matmul}\PYG{p}{(}\PYG{n}{L}\PYG{p}{,}\PYG{n}{L}\PYG{o}{.}\PYG{n}{T}\PYG{p}{)}
\end{sphinxVerbatim}

\begin{sphinxVerbatim}[commandchars=\\\{\}]
array([[ True, False,  True],
       [False,  True, False],
       [ True, False,  True]])
\end{sphinxVerbatim}


\section{Problema 3}
\label{\detokenize{codeSession-5-solve:problema-3}}
\sphinxAtStartPar
Verificar se uma matriz simétrica é positiva definida.

\begin{sphinxVerbatim}[commandchars=\\\{\}]
\PYG{c+c1}{\PYGZsh{} cria matriz simétrica }
\PYG{n}{C} \PYG{o}{=} \PYG{n}{np}\PYG{o}{.}\PYG{n}{tril}\PYG{p}{(}\PYG{n}{B}\PYG{p}{)} \PYG{o}{\PYGZhy{}} \PYG{l+m+mi}{60}
\PYG{n}{C} \PYG{o}{=} \PYG{n}{np}\PYG{o}{.}\PYG{n}{tril}\PYG{p}{(}\PYG{n}{C}\PYG{p}{)} \PYG{o}{+} \PYG{n}{np}\PYG{o}{.}\PYG{n}{tril}\PYG{p}{(}\PYG{n}{C}\PYG{p}{,}\PYG{o}{\PYGZhy{}}\PYG{l+m+mi}{1}\PYG{p}{)}\PYG{o}{.}\PYG{n}{T}
\PYG{n}{C}
\end{sphinxVerbatim}

\begin{sphinxVerbatim}[commandchars=\\\{\}]
array([[ \PYGZhy{}5., \PYGZhy{}65., \PYGZhy{}90.],
       [\PYGZhy{}65.,  10., \PYGZhy{}75.],
       [\PYGZhy{}90., \PYGZhy{}75., \PYGZhy{}15.]])
\end{sphinxVerbatim}

\begin{sphinxVerbatim}[commandchars=\\\{\}]
\PYG{c+c1}{\PYGZsh{} erro! }
\PYG{c+c1}{\PYGZsh{} matriz não é PD}
\PYG{c+c1}{\PYGZsh{}cholesky(C)}
\end{sphinxVerbatim}

\begin{sphinxVerbatim}[commandchars=\\\{\}]
\PYG{c+c1}{\PYGZsh{} cria outra matriz simétrica}
\PYG{n}{D} \PYG{o}{=} \PYG{n}{np}\PYG{o}{.}\PYG{n}{tril}\PYG{p}{(}\PYG{n}{B}\PYG{p}{)} \PYG{o}{+} \PYG{l+m+mi}{1}
\PYG{n}{D} \PYG{o}{=} \PYG{n}{np}\PYG{o}{.}\PYG{n}{tril}\PYG{p}{(}\PYG{n}{D}\PYG{p}{)} \PYG{o}{+} \PYG{n}{np}\PYG{o}{.}\PYG{n}{tril}\PYG{p}{(}\PYG{n}{D}\PYG{p}{,}\PYG{o}{\PYGZhy{}}\PYG{l+m+mi}{1}\PYG{p}{)}\PYG{o}{.}\PYG{n}{T}
\PYG{n}{D}
\end{sphinxVerbatim}

\begin{sphinxVerbatim}[commandchars=\\\{\}]
array([[ 56.,  \PYGZhy{}4., \PYGZhy{}29.],
       [ \PYGZhy{}4.,  71., \PYGZhy{}14.],
       [\PYGZhy{}29., \PYGZhy{}14.,  46.]])
\end{sphinxVerbatim}

\begin{sphinxVerbatim}[commandchars=\\\{\}]
\PYG{c+c1}{\PYGZsh{} matriz é PD}
\PYG{n}{cholesky}\PYG{p}{(}\PYG{n}{D}\PYG{p}{)}
\end{sphinxVerbatim}

\begin{sphinxVerbatim}[commandchars=\\\{\}]
array([[ 7.48331477,  0.        ,  0.        ],
       [\PYGZhy{}0.53452248,  8.40917866,  0.        ],
       [\PYGZhy{}3.87528801, \PYGZhy{}1.91117697,  5.22776678]])
\end{sphinxVerbatim}


\section{Problema 4}
\label{\detokenize{codeSession-5-solve:problema-4}}
\sphinxAtStartPar
Resolva o Problema 1 para \(R = 10\) usando a fatoração de Cholesky.

\begin{sphinxVerbatim}[commandchars=\\\{\}]
\PYG{n}{R} \PYG{o}{=} \PYG{l+m+mf}{10.}
\PYG{n}{D} \PYG{o}{=} \PYG{n}{np}\PYG{o}{.}\PYG{n}{array}\PYG{p}{(}\PYG{p}{[} \PYG{p}{[}\PYG{l+m+mi}{50}\PYG{o}{+}\PYG{n}{R}\PYG{p}{,}\PYG{o}{\PYGZhy{}}\PYG{n}{R}\PYG{p}{,}\PYG{o}{\PYGZhy{}}\PYG{l+m+mi}{30}\PYG{p}{]}\PYG{p}{,}\PYG{p}{[}\PYG{o}{\PYGZhy{}}\PYG{n}{R}\PYG{p}{,}\PYG{l+m+mi}{65}\PYG{o}{+}\PYG{n}{R}\PYG{p}{,}\PYG{o}{\PYGZhy{}}\PYG{l+m+mi}{15}\PYG{p}{]}\PYG{p}{,}\PYG{p}{[}\PYG{o}{\PYGZhy{}}\PYG{l+m+mi}{30}\PYG{p}{,}\PYG{o}{\PYGZhy{}}\PYG{l+m+mi}{15}\PYG{p}{,}\PYG{l+m+mi}{45}\PYG{p}{]} \PYG{p}{]}\PYG{p}{)}
\PYG{n}{b} \PYG{o}{=} \PYG{n}{np}\PYG{o}{.}\PYG{n}{array}\PYG{p}{(}\PYG{p}{[}\PYG{l+m+mi}{0}\PYG{p}{,}\PYG{l+m+mi}{0}\PYG{p}{,}\PYG{l+m+mi}{120}\PYG{p}{]}\PYG{p}{)}

\PYG{c+c1}{\PYGZsh{} fator }
\PYG{n}{L} \PYG{o}{=} \PYG{n}{cholesky}\PYG{p}{(}\PYG{n}{D}\PYG{p}{)}

\PYG{c+c1}{\PYGZsh{} passo 1}
\PYG{c+c1}{\PYGZsh{} Ly = b}
\PYG{n}{y} \PYG{o}{=} \PYG{n}{solve}\PYG{p}{(}\PYG{n}{L}\PYG{p}{,}\PYG{n}{b}\PYG{p}{)}

\PYG{c+c1}{\PYGZsh{} passo 2}
\PYG{c+c1}{\PYGZsh{} L\PYGZca{}T x = y}
\PYG{n}{x} \PYG{o}{=} \PYG{n}{solve}\PYG{p}{(}\PYG{n}{L}\PYG{o}{.}\PYG{n}{T}\PYG{p}{,}\PYG{n}{y}\PYG{p}{)}

\PYG{c+c1}{\PYGZsh{} solução}
\PYG{n}{x}
\end{sphinxVerbatim}

\begin{sphinxVerbatim}[commandchars=\\\{\}]
array([2.66666667, 1.33333333, 4.88888889])
\end{sphinxVerbatim}

\sphinxAtStartPar
Compare a solução via Cholesky com a do Problema 1:

\begin{sphinxVerbatim}[commandchars=\\\{\}]
\PYG{n}{x}\PYG{p}{,} \PYG{n}{sols}\PYG{p}{[}\PYG{l+m+mi}{10}\PYG{p}{]}
\end{sphinxVerbatim}

\begin{sphinxVerbatim}[commandchars=\\\{\}]
(array([2.66666667, 1.33333333, 4.88888889]),
 array([2.66666667, 1.33333333, 4.88888889]))
\end{sphinxVerbatim}

\sphinxAtStartPar
Mais uma vez, note que:

\begin{sphinxVerbatim}[commandchars=\\\{\}]
\PYG{c+c1}{\PYGZsh{} A comparação falha.}
\PYG{c+c1}{\PYGZsh{} Por quê?}
\PYG{n}{x} \PYG{o}{==} \PYG{n}{sols}\PYG{p}{[}\PYG{l+m+mi}{10}\PYG{p}{]}
\end{sphinxVerbatim}

\begin{sphinxVerbatim}[commandchars=\\\{\}]
array([False,  True,  True])
\end{sphinxVerbatim}


\chapter{Code session 6}
\label{\detokenize{codeSession-6-interp:code-session-6}}\label{\detokenize{codeSession-6-interp::doc}}
\begin{sphinxVerbatim}[commandchars=\\\{\}]
\PYG{o}{\PYGZpc{}}\PYG{k}{matplotlib} inline
\end{sphinxVerbatim}


\section{\sphinxstyleliteralintitle{\sphinxupquote{interp1d}}}
\label{\detokenize{codeSession-6-interp:interp1d}}
\sphinxAtStartPar
A classe \sphinxcode{\sphinxupquote{interp1d}} do submódulo \sphinxcode{\sphinxupquote{scipy.interpolate}} pode ser usada como uma estrutura genérica para o cálculo de interpolação unidimensional do tipo \(y_k = f(x_k)\).

\sphinxAtStartPar
Para usar \sphinxcode{\sphinxupquote{interp1d}}, precisamos de, no mínimo, uma tabela de dados fornecida por dois parâmetros:
\begin{enumerate}
\sphinxsetlistlabels{\arabic}{enumi}{enumii}{}{.}%
\item {} 
\sphinxAtStartPar
\sphinxcode{\sphinxupquote{x}}: array de valores independentes

\item {} 
\sphinxAtStartPar
\sphinxcode{\sphinxupquote{y}}: array de valores dependentes

\end{enumerate}

\sphinxAtStartPar
Um dos argumentos opcionais relevantes de \sphinxcode{\sphinxupquote{interp1d}} é:
\begin{itemize}
\item {} 
\sphinxAtStartPar
\sphinxcode{\sphinxupquote{kind}}: tipo de dado \sphinxcode{\sphinxupquote{str}} ou \sphinxcode{\sphinxupquote{int}} que especifica o tipo de interpolação.

\end{itemize}

\sphinxAtStartPar
O valor padrão de \sphinxcode{\sphinxupquote{kind}} é \sphinxcode{\sphinxupquote{'linear'}}, o qual equivale à configuração de uma interpolação linear. Outras opções relevantes, bem como o que elas realizam estão dispostas na tabela a seguir:


\begin{savenotes}\sphinxattablestart
\centering
\begin{tabulary}{\linewidth}[t]{|T|T|}
\hline
\sphinxstyletheadfamily 
\sphinxAtStartPar
opção
&\sphinxstyletheadfamily 
\sphinxAtStartPar
interpolação
\\
\hline
\sphinxAtStartPar
\sphinxcode{\sphinxupquote{'nearest'}}
&
\sphinxAtStartPar
vizinho mais próximo
\\
\hline
\sphinxAtStartPar
\sphinxcode{\sphinxupquote{'zero'}}
&
\sphinxAtStartPar
interpolação por spline de ordem 0
\\
\hline
\sphinxAtStartPar
\sphinxcode{\sphinxupquote{'slinear'}}
&
\sphinxAtStartPar
interpolação por spline de ordem 1
\\
\hline
\sphinxAtStartPar
\sphinxcode{\sphinxupquote{'quadratic'}}
&
\sphinxAtStartPar
interpolação por spline de ordem 2
\\
\hline
\sphinxAtStartPar
\sphinxcode{\sphinxupquote{'cubic'}}
&
\sphinxAtStartPar
interpolação por spline de ordem 3
\\
\hline
\end{tabulary}
\par
\sphinxattableend\end{savenotes}

\sphinxAtStartPar
Se um valor inteiro for passado para \sphinxcode{\sphinxupquote{'kind'}}, ele indicará a ordem da spline interpolatória. Por exemplo, \sphinxcode{\sphinxupquote{'kind' = 4}} indica uma interpolação por spline de ordem 4.

\sphinxAtStartPar
Em Python, a classe \sphinxcode{\sphinxupquote{interp1d}} é chamada da seguinte forma:

\begin{sphinxVerbatim}[commandchars=\\\{\}]
\PYG{k+kn}{from} \PYG{n+nn}{scipy}\PYG{n+nn}{.}\PYG{n+nn}{interpolate} \PYG{k+kn}{import} \PYG{n}{interp1d}
\end{sphinxVerbatim}

\sphinxAtStartPar
Podemos, agora, resolver alguns problemas de interpolação unidimensional por meio desta classe.

\sphinxAtStartPar
Em primeiro lugar, vamos importar alguns módulos necessários para nossos cálculos.

\begin{sphinxVerbatim}[commandchars=\\\{\}]
\PYG{k+kn}{import} \PYG{n+nn}{numpy} \PYG{k}{as} \PYG{n+nn}{np}
\PYG{k+kn}{import} \PYG{n+nn}{matplotlib}\PYG{n+nn}{.}\PYG{n+nn}{pyplot} \PYG{k}{as} \PYG{n+nn}{plt} 
\PYG{k+kn}{from} \PYG{n+nn}{scipy}\PYG{n+nn}{.}\PYG{n+nn}{interpolate} \PYG{k+kn}{import} \PYG{n}{interp1d}
\end{sphinxVerbatim}


\subsection{Problema 1}
\label{\detokenize{codeSession-6-interp:problema-1}}
\sphinxAtStartPar
Valores de entalpia por unidade de massa, \(h\), de um plasma de Argônio em equilíbrio \sphinxstyleemphasis{versus} temperatura estão tabelados no arquivo \sphinxcode{\sphinxupquote{file\sphinxhyphen{}cs6\sphinxhyphen{}entalpia.csv}}. Usando esses dados:
\begin{itemize}
\item {} 
\sphinxAtStartPar
Escreva um programa para interpolar valores de \(h\) para temperaturas no intervalo \(5000 - 30000 \, ^{\circ}K\), com incrementos de \(500 \, ^{\circ}K\).

\item {} 
\sphinxAtStartPar
Plote o gráfico de dispersão marcando com asteriscos os valores de entalpia tabelados. Em seguida, plote gráficos de linha para as seguintes interpolações: \sphinxcode{\sphinxupquote{'nearest'}}, \sphinxcode{\sphinxupquote{'zero'}}, \sphinxcode{\sphinxupquote{'slinear'}} e \sphinxcode{\sphinxupquote{'quadratic'}}.

\item {} 
\sphinxAtStartPar
Compare os valores interpolados de \(h\) para cada um dos métodos de interpolação
\sphinxcode{\sphinxupquote{'zero'}}, \sphinxcode{\sphinxupquote{'slinear'}} e \sphinxcode{\sphinxupquote{'quadratic'}} do item anterior para \(T= 15150 \, ^{\circ}K\).

\end{itemize}

\sphinxAtStartPar
\sphinxstylestrong{Observação:} note que a temperatura da tabela deve ser multiplicada por 1000.


\subsubsection{Resolução}
\label{\detokenize{codeSession-6-interp:resolucao}}
\sphinxAtStartPar
Em primeiro lugar, vamos ler a tabela de dados, atribuir os valores tabelados em arrays e corrigir os valores de temperatura pelo fator 1000.

\begin{sphinxVerbatim}[commandchars=\\\{\}]
\PYG{c+c1}{\PYGZsh{} atribuindo colunas da matriz de dados em h e T}
\PYG{n}{h}\PYG{p}{,} \PYG{n}{T} \PYG{o}{=} \PYG{n}{np}\PYG{o}{.}\PYG{n}{loadtxt}\PYG{p}{(}\PYG{l+s+s1}{\PYGZsq{}}\PYG{l+s+s1}{file\PYGZhy{}cs6\PYGZhy{}entalpia.csv}\PYG{l+s+s1}{\PYGZsq{}}\PYG{p}{,} \PYG{n}{delimiter}\PYG{o}{=}\PYG{l+s+s1}{\PYGZsq{}}\PYG{l+s+s1}{,}\PYG{l+s+s1}{\PYGZsq{}}\PYG{p}{,} \PYG{n}{skiprows}\PYG{o}{=}\PYG{l+m+mi}{1}\PYG{p}{,} \PYG{n}{unpack}\PYG{o}{=}\PYG{k+kc}{True}\PYG{p}{)}

\PYG{c+c1}{\PYGZsh{} temperatura em milhares de Kelvin}
\PYG{n}{T} \PYG{o}{=} \PYG{l+m+mf}{1e3}\PYG{o}{*}\PYG{n}{T} 
\end{sphinxVerbatim}

\sphinxAtStartPar
Criamos um array para o intervalo de temperaturas desejado para interpolação usando \sphinxcode{\sphinxupquote{arange}}. Notemos que esta função exige um deslocamento do valor do incremento no último elemento do array, isto é, 30000 + 500 = 30500.

\begin{sphinxVerbatim}[commandchars=\\\{\}]
\PYG{c+c1}{\PYGZsh{} array de temperaturas com incremento de 500 K}
\PYG{n}{t} \PYG{o}{=} \PYG{n}{np}\PYG{o}{.}\PYG{n}{arange}\PYG{p}{(}\PYG{l+m+mf}{5000.0}\PYG{p}{,}\PYG{l+m+mf}{30500.0}\PYG{p}{,}\PYG{l+m+mi}{500}\PYG{p}{)}
\end{sphinxVerbatim}

\sphinxAtStartPar
Em seguida, usamos os valores tabelados para posterior aplicação de \sphinxcode{\sphinxupquote{interp1d}} sobre \sphinxcode{\sphinxupquote{t}} como uma função e imprimimos os valores interpolados de entalpia:

\begin{sphinxVerbatim}[commandchars=\\\{\}]
\PYG{c+c1}{\PYGZsh{} montagem da interpolação}
\PYG{n}{f} \PYG{o}{=} \PYG{n}{interp1d}\PYG{p}{(}\PYG{n}{T}\PYG{p}{,}\PYG{n}{h}\PYG{p}{)}

\PYG{c+c1}{\PYGZsh{} valores interpolados}
\PYG{n}{hi} \PYG{o}{=} \PYG{n}{f}\PYG{p}{(}\PYG{n}{t}\PYG{p}{)} 

\PYG{n}{hi} 
\end{sphinxVerbatim}

\begin{sphinxVerbatim}[commandchars=\\\{\}]
array([  3.3 ,   4.14,   4.98,   5.82,   6.66,   7.5 ,  14.36,  21.22,
        28.08,  34.94,  41.8 ,  43.8 ,  45.8 ,  47.8 ,  49.8 ,  51.8 ,
        53.64,  55.48,  57.32,  59.16,  61.  ,  69.02,  77.04,  85.06,
        93.08, 101.1 , 107.46, 113.82, 120.18, 126.54, 132.9 , 135.42,
       137.94, 140.46, 142.98, 145.5 , 150.68, 155.86, 161.04, 166.22,
       171.4 , 182.28, 193.16, 204.04, 214.92, 225.8 , 232.82, 239.84,
       246.86, 253.88, 260.9 ])
\end{sphinxVerbatim}

\sphinxAtStartPar
Vamos determinar os valores interpolados para cada método de interpolação e plotá\sphinxhyphen{}los juntamente com o gráfico de dispersão dos valores tabelados.

\begin{sphinxVerbatim}[commandchars=\\\{\}]
\PYG{c+c1}{\PYGZsh{} métodos de interpolação}
\PYG{n}{m} \PYG{o}{=} \PYG{p}{[}\PYG{l+s+s1}{\PYGZsq{}}\PYG{l+s+s1}{nearest}\PYG{l+s+s1}{\PYGZsq{}}\PYG{p}{,} \PYG{l+s+s1}{\PYGZsq{}}\PYG{l+s+s1}{zero}\PYG{l+s+s1}{\PYGZsq{}}\PYG{p}{,} \PYG{l+s+s1}{\PYGZsq{}}\PYG{l+s+s1}{slinear}\PYG{l+s+s1}{\PYGZsq{}}\PYG{p}{,} \PYG{l+s+s1}{\PYGZsq{}}\PYG{l+s+s1}{quadratic}\PYG{l+s+s1}{\PYGZsq{}}\PYG{p}{]}

\PYG{c+c1}{\PYGZsh{} objetos de interpolação para cada método }
\PYG{n}{F} \PYG{o}{=} \PYG{p}{[}\PYG{n}{interp1d}\PYG{p}{(}\PYG{n}{T}\PYG{p}{,}\PYG{n}{h}\PYG{p}{,}\PYG{n}{kind}\PYG{o}{=}\PYG{n}{k}\PYG{p}{)} \PYG{k}{for} \PYG{n}{k} \PYG{o+ow}{in} \PYG{n}{m}\PYG{p}{]}

\PYG{c+c1}{\PYGZsh{} valores interpolados}
\PYG{n}{him} \PYG{o}{=} \PYG{p}{[}\PYG{n}{f}\PYG{p}{(}\PYG{n}{t}\PYG{p}{)} \PYG{k}{for} \PYG{n}{f} \PYG{o+ow}{in} \PYG{n}{F}\PYG{p}{]}

\PYG{c+c1}{\PYGZsh{} plotagem dos valores tabelados}
\PYG{n}{plt}\PYG{o}{.}\PYG{n}{plot}\PYG{p}{(}\PYG{n}{T}\PYG{p}{,}\PYG{n}{h}\PYG{p}{,}\PYG{l+s+s1}{\PYGZsq{}}\PYG{l+s+s1}{*}\PYG{l+s+s1}{\PYGZsq{}}\PYG{p}{,}\PYG{n}{label}\PYG{o}{=}\PYG{l+s+s1}{\PYGZsq{}}\PYG{l+s+s1}{tabelado}\PYG{l+s+s1}{\PYGZsq{}}\PYG{p}{)}\PYG{p}{;}

\PYG{c+c1}{\PYGZsh{} plotagem dos métodos}
\PYG{k}{for} \PYG{n}{i} \PYG{o+ow}{in} \PYG{n+nb}{range}\PYG{p}{(}\PYG{l+m+mi}{4}\PYG{p}{)}\PYG{p}{:}
    \PYG{n}{plt}\PYG{o}{.}\PYG{n}{plot}\PYG{p}{(}\PYG{n}{t}\PYG{p}{,}\PYG{n}{him}\PYG{p}{[}\PYG{n}{i}\PYG{p}{]}\PYG{p}{,}\PYG{n}{label}\PYG{o}{=}\PYG{n}{m}\PYG{p}{[}\PYG{n}{i}\PYG{p}{]}\PYG{p}{)}

\PYG{c+c1}{\PYGZsh{} legenda}
\PYG{n}{plt}\PYG{o}{.}\PYG{n}{legend}\PYG{p}{(}\PYG{p}{)}\PYG{p}{;}    
\end{sphinxVerbatim}

\noindent\sphinxincludegraphics{{codeSession-6-interp_13_0}.png}

\sphinxAtStartPar
Até aqui, já cumprimos os dois primeiros requisitos do problema. Para o terceiro, usaremos as informações pré\sphinxhyphen{}computadas na lista \sphinxcode{\sphinxupquote{F}} para estimar os valores de entalpia quando \(T = 15150 \, ^{\circ} K\). Teremos os seguintes três valores:

\begin{sphinxVerbatim}[commandchars=\\\{\}]
\PYG{c+c1}{\PYGZsh{} calcula h(15150) para os métodos \PYGZsq{}zero\PYGZsq{}, \PYGZsq{}slinear\PYGZsq{} e \PYGZsq{}quadratic\PYGZsq{}}
\PYG{n}{h\PYGZus{}15150} \PYG{o}{=} \PYG{p}{[}\PYG{n}{f}\PYG{p}{(}\PYG{l+m+mi}{15150}\PYG{p}{)} \PYG{k}{for} \PYG{n}{f} \PYG{o+ow}{in} \PYG{n}{F}\PYG{p}{[}\PYG{l+m+mi}{1}\PYG{p}{:}\PYG{p}{]}\PYG{p}{]}

\PYG{n}{h\PYGZus{}15150}
\end{sphinxVerbatim}

\begin{sphinxVerbatim}[commandchars=\\\{\}]
[array(61.), array(63.406), array(62.60353522)]
\end{sphinxVerbatim}

\sphinxAtStartPar
Isto é, os valores de entalpia em \(T = 15150 \, ^{\circ} K\) podem ser organizdos na tabela a seguir:


\begin{savenotes}\sphinxattablestart
\centering
\begin{tabulary}{\linewidth}[t]{|T|T|}
\hline
\sphinxstyletheadfamily 
\sphinxAtStartPar
método
&\sphinxstyletheadfamily 
\sphinxAtStartPar
valor
\\
\hline
\sphinxAtStartPar
\sphinxcode{\sphinxupquote{'zero'}}
&
\sphinxAtStartPar
61.0 MJ/kg
\\
\hline
\sphinxAtStartPar
\sphinxcode{\sphinxupquote{'slinear'}}
&
\sphinxAtStartPar
63.406 MJ/kg
\\
\hline
\sphinxAtStartPar
\sphinxcode{\sphinxupquote{'quadratic'}}
&
\sphinxAtStartPar
62.604 MJ/kg
\\
\hline
\end{tabulary}
\par
\sphinxattableend\end{savenotes}

\sphinxAtStartPar
Levando em conta que quanto mais alta é a ordem de interpolação, melhor é a interpolação, podemos inferir que desses três valores, \(62.604 \, MJ/kg\) é o mais confiável para usar.


\subsection{Problema 2}
\label{\detokenize{codeSession-6-interp:problema-2}}
\sphinxAtStartPar
O arquivo \sphinxcode{\sphinxupquote{file\sphinxhyphen{}cs6\sphinxhyphen{}salinidade.csv}} tabela valores de salinidade da água (em ppt) em função da profundidade oceânica (em metros). Use interpolação por spline cúbica para gerar uma tabela de salinidades para profundidades de 0 a 3000 m com espaçamento de 10 m e estime os valores nas profundidades de 250 m, 750 m e 1800 m.


\subsection{Problema 3}
\label{\detokenize{codeSession-6-interp:problema-3}}
\sphinxAtStartPar
A tabela a seguir apresenta a potência de um motor a Diesel (em hp) em diferentes rotações (em rpm). Gere uma tabela de valores interpolados com espaçamento de 10 rpm e destaque as potências em 2300 rpm e 3650 rpm.


\begin{savenotes}\sphinxattablestart
\centering
\begin{tabulary}{\linewidth}[t]{|T|T|}
\hline
\sphinxstyletheadfamily 
\sphinxAtStartPar
velocidade (rpm)
&\sphinxstyletheadfamily 
\sphinxAtStartPar
potência (hp)
\\
\hline
\sphinxAtStartPar
1200
&
\sphinxAtStartPar
65
\\
\hline
\sphinxAtStartPar
1500
&
\sphinxAtStartPar
130
\\
\hline
\sphinxAtStartPar
2000
&
\sphinxAtStartPar
185
\\
\hline
\sphinxAtStartPar
2500
&
\sphinxAtStartPar
225
\\
\hline
\sphinxAtStartPar
3000
&
\sphinxAtStartPar
255
\\
\hline
\sphinxAtStartPar
3250
&
\sphinxAtStartPar
266
\\
\hline
\sphinxAtStartPar
3500
&
\sphinxAtStartPar
275
\\
\hline
\sphinxAtStartPar
3750
&
\sphinxAtStartPar
272
\\
\hline
\sphinxAtStartPar
4000
&
\sphinxAtStartPar
260
\\
\hline
\sphinxAtStartPar
4400
&
\sphinxAtStartPar
230
\\
\hline
\end{tabulary}
\par
\sphinxattableend\end{savenotes}


\chapter{Code session 7}
\label{\detokenize{codeSession-7-fit:code-session-7}}\label{\detokenize{codeSession-7-fit::doc}}
\begin{sphinxVerbatim}[commandchars=\\\{\}]
\PYG{o}{\PYGZpc{}}\PYG{k}{matplotlib} inline
\PYG{k+kn}{import} \PYG{n+nn}{numpy} \PYG{k}{as} \PYG{n+nn}{np} 
\PYG{k+kn}{import} \PYG{n+nn}{matplotlib}\PYG{n+nn}{.}\PYG{n+nn}{pyplot} \PYG{k}{as} \PYG{n+nn}{plt} 
\end{sphinxVerbatim}


\section{Regressão Linear}
\label{\detokenize{codeSession-7-fit:regressao-linear}}

\subsection{\sphinxstyleliteralintitle{\sphinxupquote{linregress}}}
\label{\detokenize{codeSession-7-fit:linregress}}
\sphinxAtStartPar
A regressão linear é o modelo mais básico para realizar ajuste de dados e frequentemente aplicado em estudos estatísticos. Em Python, a regressão linear pode ser realizada com a função \sphinxcode{\sphinxupquote{linregress}}. Esta função calcula a regressão linear por mínimos quadrados (a rigor, o termo deveria ser traduzido como \sphinxstyleemphasis{quadrados mínimos}) para dois conjuntos de medição.

\sphinxAtStartPar
Os argumentos de entrada obrigatórios desta função são:
\begin{enumerate}
\sphinxsetlistlabels{\arabic}{enumi}{enumii}{}{.}%
\item {} 
\sphinxAtStartPar
o primeiro conjunto de dados \sphinxcode{\sphinxupquote{x}} (lista ou objeto tipo \sphinxstyleemphasis{array})

\item {} 
\sphinxAtStartPar
o segundo conjunto de dados \sphinxcode{\sphinxupquote{y}} (lista ou objeto tipo \sphinxstyleemphasis{array})

\end{enumerate}

\sphinxAtStartPar
Os argumentos de saída são:
\begin{itemize}
\item {} 
\sphinxAtStartPar
\sphinxcode{\sphinxupquote{slope}}: coeficiente angular da reta obtida pela regressão linear

\item {} 
\sphinxAtStartPar
\sphinxcode{\sphinxupquote{intercept}}: coeficiente linear da reta obtida pela regressão linear

\item {} 
\sphinxAtStartPar
\sphinxcode{\sphinxupquote{rvalue}}: valor do coeficiente de correlação

\item {} 
\sphinxAtStartPar
\sphinxcode{\sphinxupquote{pvalue}}: valor\sphinxhyphen{} p de teste de hipótese

\item {} 
\sphinxAtStartPar
\sphinxcode{\sphinxupquote{stderror}}: medida de erro

\end{itemize}

\sphinxAtStartPar
Não utilizaremos os 2 últimos aqui.

\sphinxAtStartPar
\sphinxstylestrong{Nota:} para obter o valor do \sphinxstyleemphasis{coeficiente de determinação} \(R^2\), o valor de \sphinxcode{\sphinxupquote{rvalue}} deve ser elevado ao quadrado, i.e. \sphinxcode{\sphinxupquote{R2 = rvalue**2.}}

\sphinxAtStartPar
Como importá\sphinxhyphen{}la?

\begin{sphinxVerbatim}[commandchars=\\\{\}]
\PYG{k+kn}{from} \PYG{n+nn}{scipy}\PYG{n+nn}{.}\PYG{n+nn}{stats} \PYG{k+kn}{import} \PYG{n}{linregress}
\end{sphinxVerbatim}

\begin{sphinxVerbatim}[commandchars=\\\{\}]
\PYG{k+kn}{from} \PYG{n+nn}{scipy}\PYG{n+nn}{.}\PYG{n+nn}{stats} \PYG{k+kn}{import} \PYG{n}{linregress}
\end{sphinxVerbatim}


\subsection{Problema 1}
\label{\detokenize{codeSession-7-fit:problema-1}}
\sphinxAtStartPar
A tabela a seguir lista a massa \(M\) e o consumo médio \(C\) de automóveis fabricados pela Ford e Honda em 2008. Faça um ajuste linear \(C = b + aM\) aos dados e calcule o desvio padrão.


\begin{savenotes}\sphinxattablestart
\centering
\begin{tabulary}{\linewidth}[t]{|T|T|T|}
\hline
\sphinxstyletheadfamily 
\sphinxAtStartPar
modelo
&\sphinxstyletheadfamily 
\sphinxAtStartPar
massa (kg)
&\sphinxstyletheadfamily 
\sphinxAtStartPar
C (km/litro)
\\
\hline
\sphinxAtStartPar
Focus
&
\sphinxAtStartPar
1198
&
\sphinxAtStartPar
11.90
\\
\hline
\sphinxAtStartPar
Crown Victoria
&
\sphinxAtStartPar
1715
&
\sphinxAtStartPar
6.80
\\
\hline
\sphinxAtStartPar
Expedition
&
\sphinxAtStartPar
2530
&
\sphinxAtStartPar
5.53
\\
\hline
\sphinxAtStartPar
Explorer
&
\sphinxAtStartPar
2014
&
\sphinxAtStartPar
6.38
\\
\hline
\sphinxAtStartPar
F\sphinxhyphen{}150
&
\sphinxAtStartPar
2136
&
\sphinxAtStartPar
5.53
\\
\hline
\sphinxAtStartPar
Fusion
&
\sphinxAtStartPar
1492
&
\sphinxAtStartPar
8.50
\\
\hline
\sphinxAtStartPar
Taurus
&
\sphinxAtStartPar
1652
&
\sphinxAtStartPar
7.65
\\
\hline
\sphinxAtStartPar
Fit
&
\sphinxAtStartPar
1168
&
\sphinxAtStartPar
13.60
\\
\hline
\sphinxAtStartPar
Accord
&
\sphinxAtStartPar
1492
&
\sphinxAtStartPar
9.78
\\
\hline
\sphinxAtStartPar
CR\sphinxhyphen{}V
&
\sphinxAtStartPar
1602
&
\sphinxAtStartPar
8.93
\\
\hline
\sphinxAtStartPar
Civic
&
\sphinxAtStartPar
1192
&
\sphinxAtStartPar
11.90
\\
\hline
\sphinxAtStartPar
Ridgeline
&
\sphinxAtStartPar
2045
&
\sphinxAtStartPar
6.38
\\
\hline
\end{tabulary}
\par
\sphinxattableend\end{savenotes}

\sphinxAtStartPar
\sphinxstylestrong{Nota}: esta tabela está disponível em formato .csv no arquivo \sphinxcode{\sphinxupquote{file\sphinxhyphen{}cs7\sphinxhyphen{}autos.csv}}.


\subsection{Resolução}
\label{\detokenize{codeSession-7-fit:resolucao}}
\sphinxAtStartPar
Vamos ler o arquivo de dados e convertê\sphinxhyphen{}lo em uma matriz.

\begin{sphinxVerbatim}[commandchars=\\\{\}]
\PYG{c+c1}{\PYGZsh{} fname: nome do arquivo}
\PYG{c+c1}{\PYGZsh{} delimiter: separador dos dados }
\PYG{c+c1}{\PYGZsh{} skiprows: ignora linhas do arquivo (aqui, estamos removendo a primeira)}
\PYG{c+c1}{\PYGZsh{} usecols: colunas a serem lidas (aqui, estamos lendo a 2a. e 3a. colunas)}
\PYG{n}{dados} \PYG{o}{=} \PYG{n}{np}\PYG{o}{.}\PYG{n}{loadtxt}\PYG{p}{(}\PYG{n}{fname}\PYG{o}{=}\PYG{l+s+s1}{\PYGZsq{}}\PYG{l+s+s1}{file\PYGZhy{}cs7\PYGZhy{}autos.csv}\PYG{l+s+s1}{\PYGZsq{}}\PYG{p}{,}\PYG{n}{delimiter}\PYG{o}{=}\PYG{l+s+s1}{\PYGZsq{}}\PYG{l+s+s1}{,}\PYG{l+s+s1}{\PYGZsq{}}\PYG{p}{,}\PYG{n}{skiprows}\PYG{o}{=}\PYG{l+m+mi}{1}\PYG{p}{,}\PYG{n}{usecols}\PYG{o}{=}\PYG{p}{(}\PYG{l+m+mi}{1}\PYG{p}{,}\PYG{l+m+mi}{2}\PYG{p}{)}\PYG{p}{)}
\PYG{n+nb}{print}\PYG{p}{(}\PYG{n}{dados}\PYG{p}{)}
\end{sphinxVerbatim}

\begin{sphinxVerbatim}[commandchars=\\\{\}]
[[1198.     11.9 ]
 [1715.      6.8 ]
 [2530.      5.53]
 [2014.      6.38]
 [2136.      5.53]
 [1492.      8.5 ]
 [1652.      7.65]
 [1168.     13.6 ]
 [1492.      9.78]
 [1602.      8.93]
 [1192.     11.9 ]
 [2045.      6.38]]
\end{sphinxVerbatim}

\sphinxAtStartPar
Armazenamos os \sphinxstyleemphasis{arrays} devidamente:

\begin{sphinxVerbatim}[commandchars=\\\{\}]
\PYG{n}{M} \PYG{o}{=} \PYG{n}{dados}\PYG{p}{[}\PYG{p}{:}\PYG{p}{,}\PYG{l+m+mi}{0}\PYG{p}{]} \PYG{c+c1}{\PYGZsh{} massa}
\PYG{n}{C} \PYG{o}{=} \PYG{n}{dados}\PYG{p}{[}\PYG{p}{:}\PYG{p}{,}\PYG{l+m+mi}{1}\PYG{p}{]} \PYG{c+c1}{\PYGZsh{} consumo}
\end{sphinxVerbatim}

\sphinxAtStartPar
Fazemos a regressão linear:

\begin{sphinxVerbatim}[commandchars=\\\{\}]
\PYG{n}{a}\PYG{p}{,}\PYG{n}{b}\PYG{p}{,}\PYG{n}{R}\PYG{p}{,} \PYG{n}{p\PYGZus{}value}\PYG{p}{,} \PYG{n}{std\PYGZus{}err} \PYG{o}{=} \PYG{n}{linregress}\PYG{p}{(}\PYG{n}{M}\PYG{p}{,}\PYG{n}{C}\PYG{p}{)}
\PYG{n+nb}{print}\PYG{p}{(}\PYG{l+s+sa}{f}\PYG{l+s+s1}{\PYGZsq{}}\PYG{l+s+s1}{Regressão linear executada com a = }\PYG{l+s+si}{\PYGZob{}}\PYG{n}{a}\PYG{l+s+si}{:}\PYG{l+s+s1}{.3f}\PYG{l+s+si}{\PYGZcb{}}\PYG{l+s+s1}{, b = }\PYG{l+s+si}{\PYGZob{}}\PYG{n}{b}\PYG{l+s+si}{:}\PYG{l+s+s1}{.3f}\PYG{l+s+si}{\PYGZcb{}}\PYG{l+s+s1}{ e R2 = }\PYG{l+s+si}{\PYGZob{}}\PYG{n}{R}\PYG{o}{*}\PYG{n}{R}\PYG{l+s+si}{:}\PYG{l+s+s1}{.2f}\PYG{l+s+si}{\PYGZcb{}}\PYG{l+s+s1}{\PYGZsq{}}\PYG{p}{)}
\end{sphinxVerbatim}

\begin{sphinxVerbatim}[commandchars=\\\{\}]
Regressão linear executada com a = \PYGZhy{}0.006, b = 18.410 e R2 = 0.83
\end{sphinxVerbatim}

\sphinxAtStartPar
Enfim, podemos visualizar o resultado:

\begin{sphinxVerbatim}[commandchars=\\\{\}]
\PYG{n}{C2} \PYG{o}{=} \PYG{n}{b} \PYG{o}{+} \PYG{n}{a}\PYG{o}{*}\PYG{n}{M} \PYG{c+c1}{\PYGZsh{} ajuste}
\PYG{n}{mod} \PYG{o}{=} \PYG{n}{plt}\PYG{o}{.}\PYG{n}{plot}\PYG{p}{(}\PYG{n}{M}\PYG{p}{,}\PYG{n}{C2}\PYG{p}{,}\PYG{l+s+s1}{\PYGZsq{}}\PYG{l+s+s1}{r:}\PYG{l+s+s1}{\PYGZsq{}}\PYG{p}{)}\PYG{p}{;} \PYG{c+c1}{\PYGZsh{} modelo}
\PYG{n}{med} \PYG{o}{=} \PYG{n}{plt}\PYG{o}{.}\PYG{n}{scatter}\PYG{p}{(}\PYG{n}{M}\PYG{p}{,}\PYG{n}{C}\PYG{p}{)}\PYG{p}{;} \PYG{c+c1}{\PYGZsh{} medição}
\PYG{n}{plt}\PYG{o}{.}\PYG{n}{legend}\PYG{p}{(}\PYG{p}{\PYGZob{}}\PYG{l+s+s1}{\PYGZsq{}}\PYG{l+s+s1}{modelo de ajuste}\PYG{l+s+s1}{\PYGZsq{}}\PYG{p}{:}\PYG{n}{mod}\PYG{p}{,} \PYG{l+s+s1}{\PYGZsq{}}\PYG{l+s+s1}{medição}\PYG{l+s+s1}{\PYGZsq{}}\PYG{p}{:}\PYG{n}{med}\PYG{p}{\PYGZcb{}}\PYG{p}{)}\PYG{p}{;} \PYG{c+c1}{\PYGZsh{} legenda}

\PYG{n}{plt}\PYG{o}{.}\PYG{n}{annotate}\PYG{p}{(}\PYG{l+s+s1}{\PYGZsq{}}\PYG{l+s+s1}{y= }\PYG{l+s+si}{\PYGZob{}0:.2f\PYGZcb{}}\PYG{l+s+s1}{ + }\PYG{l+s+si}{\PYGZob{}1:.2f\PYGZcb{}}\PYG{l+s+s1}{x}\PYG{l+s+s1}{\PYGZsq{}}\PYG{o}{.}\PYG{n}{format}\PYG{p}{(}\PYG{n}{b}\PYG{p}{,}\PYG{n}{a}\PYG{p}{)}\PYG{p}{,}\PYG{p}{(}\PYG{l+m+mi}{2080}\PYG{p}{,}\PYG{l+m+mi}{11}\PYG{p}{)}\PYG{p}{,}\PYG{n}{fontsize}\PYG{o}{=}\PYG{l+m+mi}{12}\PYG{p}{,}\PYG{n}{c}\PYG{o}{=}\PYG{l+s+s1}{\PYGZsq{}}\PYG{l+s+s1}{r}\PYG{l+s+s1}{\PYGZsq{}}\PYG{p}{)}\PYG{p}{;}
\end{sphinxVerbatim}

\noindent\sphinxincludegraphics{{codeSession-7-fit_13_0}.png}


\section{Medindo o desvio padrão do ajuste por mínimos quadrados}
\label{\detokenize{codeSession-7-fit:medindo-o-desvio-padrao-do-ajuste-por-minimos-quadrados}}
\sphinxAtStartPar
Para calcular o desvio padrão do ajuste, precisamos reconhecer o número de amostras \(n\), o número de parâmetros do modelo de ajuste \(m\) e calcular a soma \(S\) dos quadrados. A fórmula utilizada é a seguinte:
\begin{equation*}
\begin{split}\sigma = \sqrt{ \dfrac{S}{n-m} },\end{split}
\end{equation*}
\sphinxAtStartPar
onde \(S = \sum\limits_{k=0}^n [y_i - \phi(x_i)]^2\).

\sphinxAtStartPar
Notemos que se \(n = m\) (caso da interpolação), \(\sigma = \infty\), i.e. seria indefinido, já que o denominador anular\sphinxhyphen{}se\sphinxhyphen{}ia.

\sphinxAtStartPar
O modelo de ajuste \(\phi(x)\) é considerado polinomial. Então, no caso da regressão linear, temos apenas 2 parâmetros: o coeficiente linear e o angular.

\sphinxAtStartPar
Sabemos que \(m=2\). Agora, resta usar \(n\) e calcular \(S\). Isto é tudo de que precisamos para calcular \(\sigma\) para o nosso problema.

\begin{sphinxVerbatim}[commandchars=\\\{\}]
\PYG{n}{n} \PYG{o}{=} \PYG{n}{M}\PYG{o}{.}\PYG{n}{size} \PYG{c+c1}{\PYGZsh{} número de amostras}
\PYG{n}{m} \PYG{o}{=} \PYG{l+m+mi}{2} \PYG{c+c1}{\PYGZsh{} número de parâmetros. }
\end{sphinxVerbatim}

\sphinxAtStartPar
O cálculo de \(S\) pode ser feito da seguinte maneira:

\begin{sphinxVerbatim}[commandchars=\\\{\}]
\PYG{n}{S} \PYG{o}{=} \PYG{n}{np}\PYG{o}{.}\PYG{n}{sum}\PYG{p}{(} \PYG{p}{(}\PYG{n}{C} \PYG{o}{\PYGZhy{}} \PYG{n}{C2}\PYG{p}{)}\PYG{o}{*}\PYG{p}{(}\PYG{n}{C} \PYG{o}{\PYGZhy{}} \PYG{n}{C2}\PYG{p}{)} \PYG{p}{)} \PYG{c+c1}{\PYGZsh{} soma dos quadrados (resíduos)}
\end{sphinxVerbatim}

\sphinxAtStartPar
Enfim, \(\sigma\) será dado por:

\begin{sphinxVerbatim}[commandchars=\\\{\}]
\PYG{n}{sigma} \PYG{o}{=} \PYG{n}{np}\PYG{o}{.}\PYG{n}{sqrt}\PYG{p}{(}\PYG{n}{S}\PYG{o}{/}\PYG{p}{(}\PYG{n}{n}\PYG{o}{\PYGZhy{}}\PYG{n}{m}\PYG{p}{)}\PYG{p}{)} \PYG{c+c1}{\PYGZsh{} desvio padrão}
\PYG{n+nb}{print}\PYG{p}{(}\PYG{l+s+sa}{f}\PYG{l+s+s1}{\PYGZsq{}}\PYG{l+s+s1}{σ = }\PYG{l+s+si}{\PYGZob{}}\PYG{n}{sigma}\PYG{l+s+si}{:}\PYG{l+s+s1}{.3f}\PYG{l+s+si}{\PYGZcb{}}\PYG{l+s+s1}{\PYGZsq{}}\PYG{p}{)}
\end{sphinxVerbatim}

\begin{sphinxVerbatim}[commandchars=\\\{\}]
σ = 1.164
\end{sphinxVerbatim}


\section{Ajuste polinomial (linear)}
\label{\detokenize{codeSession-7-fit:ajuste-polinomial-linear}}
\sphinxAtStartPar
O ajuste de formas lineares de ordem superior (polinomial) pode ser realizado por meio da função \sphinxcode{\sphinxupquote{polyfit}}.


\subsection{\sphinxstyleliteralintitle{\sphinxupquote{polyfit}}}
\label{\detokenize{codeSession-7-fit:polyfit}}
\sphinxAtStartPar
Esta função ajusta um polinômio de grau \(g\) à tabela de dados.

\sphinxAtStartPar
Os argumentos de entrada obrigatórios desta função são:
\begin{enumerate}
\sphinxsetlistlabels{\arabic}{enumi}{enumii}{}{.}%
\item {} 
\sphinxAtStartPar
o primeiro conjunto de dados \sphinxcode{\sphinxupquote{x}} (lista ou objeto tipo \sphinxstyleemphasis{array})

\item {} 
\sphinxAtStartPar
o segundo conjunto de dados \sphinxcode{\sphinxupquote{y}} (lista ou objeto tipo \sphinxstyleemphasis{array})

\item {} 
\sphinxAtStartPar
o grau do polinômio \sphinxcode{\sphinxupquote{g}}

\end{enumerate}

\sphinxAtStartPar
O principal argumento de saída é:
\begin{itemize}
\item {} 
\sphinxAtStartPar
\sphinxcode{\sphinxupquote{p}}: lista dos g+1 coeficientes do modelo (ordenados do maior para o menor grau)

\end{itemize}

\sphinxAtStartPar
Como importá\sphinxhyphen{}la?

\begin{sphinxVerbatim}[commandchars=\\\{\}]
\PYG{k+kn}{from} \PYG{n+nn}{numpy} \PYG{k+kn}{import} \PYG{n}{polyfit}
\end{sphinxVerbatim}

\sphinxAtStartPar
Como já importamos o \sphinxcode{\sphinxupquote{numpy}}, basta chamar a função com:

\begin{sphinxVerbatim}[commandchars=\\\{\}]
\PYG{n}{np}\PYG{o}{.}\PYG{n}{polyfit}\PYG{p}{(}\PYG{n}{x}\PYG{p}{,}\PYG{n}{y}\PYG{p}{,}\PYG{n}{deg}\PYG{p}{)}
\end{sphinxVerbatim}


\subsection{Problema 2}
\label{\detokenize{codeSession-7-fit:problema-2}}
\sphinxAtStartPar
Refaça o Problema 1 ajustando os dados com polinômios de grau 2, 3, 4 e 5 e plote os gráficos dos modelos ajustados aos dados em apenas uma figura.


\subsubsection{Resolução}
\label{\detokenize{codeSession-7-fit:id1}}
\sphinxAtStartPar
Uma vez que já temos as variáveis armazenadas na memória, basta criarmos os ajustes.

\begin{sphinxVerbatim}[commandchars=\\\{\}]
\PYG{n}{p2} \PYG{o}{=} \PYG{n}{np}\PYG{o}{.}\PYG{n}{polyfit}\PYG{p}{(}\PYG{n}{M}\PYG{p}{,}\PYG{n}{C}\PYG{p}{,}\PYG{l+m+mi}{2}\PYG{p}{)}
\PYG{n}{p3} \PYG{o}{=} \PYG{n}{np}\PYG{o}{.}\PYG{n}{polyfit}\PYG{p}{(}\PYG{n}{M}\PYG{p}{,}\PYG{n}{C}\PYG{p}{,}\PYG{l+m+mi}{3}\PYG{p}{)} 
\PYG{n}{p4} \PYG{o}{=} \PYG{n}{np}\PYG{o}{.}\PYG{n}{polyfit}\PYG{p}{(}\PYG{n}{M}\PYG{p}{,}\PYG{n}{C}\PYG{p}{,}\PYG{l+m+mi}{4}\PYG{p}{)} 
\PYG{n}{p5} \PYG{o}{=} \PYG{n}{np}\PYG{o}{.}\PYG{n}{polyfit}\PYG{p}{(}\PYG{n}{M}\PYG{p}{,}\PYG{n}{C}\PYG{p}{,}\PYG{l+m+mi}{5}\PYG{p}{)} 
\end{sphinxVerbatim}

\sphinxAtStartPar
Para imprimir a lista dos coeficientes, basta fazer:

\begin{sphinxVerbatim}[commandchars=\\\{\}]
\PYG{n+nb}{print}\PYG{p}{(}\PYG{n}{p2}\PYG{p}{)}
\PYG{n+nb}{print}\PYG{p}{(}\PYG{n}{p3}\PYG{p}{)}
\PYG{n+nb}{print}\PYG{p}{(}\PYG{n}{p4}\PYG{p}{)}
\PYG{n+nb}{print}\PYG{p}{(}\PYG{n}{p5}\PYG{p}{)}
\end{sphinxVerbatim}

\begin{sphinxVerbatim}[commandchars=\\\{\}]
[ 5.26020025e\PYGZhy{}06 \PYGZhy{}2.45829326e\PYGZhy{}02  3.41991952e+01]
[\PYGZhy{}1.89135665e\PYGZhy{}09  1.56050998e\PYGZhy{}05 \PYGZhy{}4.27556580e\PYGZhy{}02  4.44279113e+01]
[ 3.37176332e\PYGZhy{}12 \PYGZhy{}2.66523211e\PYGZhy{}08  8.21041425e\PYGZhy{}05 \PYGZhy{}1.20066752e\PYGZhy{}01
  7.72222694e+01]
[\PYGZhy{}3.12853412e\PYGZhy{}14  2.86132736e\PYGZhy{}10 \PYGZhy{}1.03205187e\PYGZhy{}06  1.83990614e\PYGZhy{}03
 \PYGZhy{}1.63096318e+00  5.87812297e+02]
\end{sphinxVerbatim}

\sphinxAtStartPar
Para plotarmos as curvas, devemos nos atentar para o grau dos modelos. Podemos criá\sphinxhyphen{}las da seguinte forma:

\begin{sphinxVerbatim}[commandchars=\\\{\}]
\PYG{n}{C22} \PYG{o}{=} \PYG{n}{p2}\PYG{p}{[}\PYG{l+m+mi}{0}\PYG{p}{]}\PYG{o}{*}\PYG{n}{M}\PYG{o}{*}\PYG{o}{*}\PYG{l+m+mi}{2} \PYG{o}{+} \PYG{n}{p2}\PYG{p}{[}\PYG{l+m+mi}{1}\PYG{p}{]}\PYG{o}{*}\PYG{n}{M}    \PYG{o}{+} \PYG{n}{p2}\PYG{p}{[}\PYG{l+m+mi}{2}\PYG{p}{]} \PYG{c+c1}{\PYGZsh{} modelo quadrático}
\PYG{n}{C23} \PYG{o}{=} \PYG{n}{p3}\PYG{p}{[}\PYG{l+m+mi}{0}\PYG{p}{]}\PYG{o}{*}\PYG{n}{M}\PYG{o}{*}\PYG{o}{*}\PYG{l+m+mi}{3} \PYG{o}{+} \PYG{n}{p3}\PYG{p}{[}\PYG{l+m+mi}{1}\PYG{p}{]}\PYG{o}{*}\PYG{n}{M}\PYG{o}{*}\PYG{o}{*}\PYG{l+m+mi}{2} \PYG{o}{+} \PYG{n}{p3}\PYG{p}{[}\PYG{l+m+mi}{2}\PYG{p}{]}\PYG{o}{*}\PYG{n}{M}    \PYG{o}{+} \PYG{n}{p3}\PYG{p}{[}\PYG{l+m+mi}{3}\PYG{p}{]} \PYG{c+c1}{\PYGZsh{} modelo cúbico}
\PYG{n}{C24} \PYG{o}{=} \PYG{n}{p4}\PYG{p}{[}\PYG{l+m+mi}{0}\PYG{p}{]}\PYG{o}{*}\PYG{n}{M}\PYG{o}{*}\PYG{o}{*}\PYG{l+m+mi}{4} \PYG{o}{+} \PYG{n}{p4}\PYG{p}{[}\PYG{l+m+mi}{1}\PYG{p}{]}\PYG{o}{*}\PYG{n}{M}\PYG{o}{*}\PYG{o}{*}\PYG{l+m+mi}{3} \PYG{o}{+} \PYG{n}{p4}\PYG{p}{[}\PYG{l+m+mi}{2}\PYG{p}{]}\PYG{o}{*}\PYG{n}{M}\PYG{o}{*}\PYG{o}{*}\PYG{l+m+mi}{2} \PYG{o}{+} \PYG{n}{p4}\PYG{p}{[}\PYG{l+m+mi}{3}\PYG{p}{]}\PYG{o}{*}\PYG{n}{M}    \PYG{o}{+} \PYG{n}{p4}\PYG{p}{[}\PYG{l+m+mi}{4}\PYG{p}{]} \PYG{c+c1}{\PYGZsh{} modelo de quarta ordem}
\PYG{n}{C25} \PYG{o}{=} \PYG{n}{p5}\PYG{p}{[}\PYG{l+m+mi}{0}\PYG{p}{]}\PYG{o}{*}\PYG{n}{M}\PYG{o}{*}\PYG{o}{*}\PYG{l+m+mi}{5} \PYG{o}{+} \PYG{n}{p5}\PYG{p}{[}\PYG{l+m+mi}{1}\PYG{p}{]}\PYG{o}{*}\PYG{n}{M}\PYG{o}{*}\PYG{o}{*}\PYG{l+m+mi}{4} \PYG{o}{+} \PYG{n}{p5}\PYG{p}{[}\PYG{l+m+mi}{2}\PYG{p}{]}\PYG{o}{*}\PYG{n}{M}\PYG{o}{*}\PYG{o}{*}\PYG{l+m+mi}{3} \PYG{o}{+} \PYG{n}{p5}\PYG{p}{[}\PYG{l+m+mi}{3}\PYG{p}{]}\PYG{o}{*}\PYG{n}{M}\PYG{o}{*}\PYG{o}{*}\PYG{l+m+mi}{2} \PYG{o}{+} \PYG{n}{p5}\PYG{p}{[}\PYG{l+m+mi}{4}\PYG{p}{]}\PYG{o}{*}\PYG{n}{M} \PYG{o}{+} \PYG{n}{p5}\PYG{p}{[}\PYG{l+m+mi}{5}\PYG{p}{]} \PYG{c+c1}{\PYGZsh{} modelo de quinta ordem}

\PYG{n}{plt}\PYG{o}{.}\PYG{n}{figure}\PYG{p}{(}\PYG{n}{figsize}\PYG{o}{=}\PYG{p}{(}\PYG{l+m+mi}{8}\PYG{p}{,}\PYG{l+m+mi}{5}\PYG{p}{)}\PYG{p}{)}
\PYG{n}{plt}\PYG{o}{.}\PYG{n}{grid}\PYG{p}{(}\PYG{k+kc}{True}\PYG{p}{)}
\PYG{n}{med} \PYG{o}{=} \PYG{n}{plt}\PYG{o}{.}\PYG{n}{plot}\PYG{p}{(}\PYG{n}{M}\PYG{p}{,}\PYG{n}{C}\PYG{p}{,}\PYG{l+s+s1}{\PYGZsq{}}\PYG{l+s+s1}{o}\PYG{l+s+s1}{\PYGZsq{}}\PYG{p}{,} \PYG{n}{ms}\PYG{o}{=}\PYG{l+m+mi}{6}\PYG{p}{)}\PYG{p}{;} \PYG{c+c1}{\PYGZsh{} medição}
\PYG{n}{mod2} \PYG{o}{=} \PYG{n}{plt}\PYG{o}{.}\PYG{n}{plot}\PYG{p}{(}\PYG{n}{M}\PYG{p}{,}\PYG{n}{C22}\PYG{p}{,}\PYG{l+s+s1}{\PYGZsq{}}\PYG{l+s+s1}{rs}\PYG{l+s+s1}{\PYGZsq{}}\PYG{p}{,}\PYG{n}{ms}\PYG{o}{=}\PYG{l+m+mi}{8}\PYG{p}{,} \PYG{n}{markerfacecolor}\PYG{o}{=}\PYG{l+s+s1}{\PYGZsq{}}\PYG{l+s+s1}{None}\PYG{l+s+s1}{\PYGZsq{}}\PYG{p}{)}\PYG{p}{;} \PYG{c+c1}{\PYGZsh{} modelo 2}
\PYG{n}{mod3} \PYG{o}{=} \PYG{n}{plt}\PYG{o}{.}\PYG{n}{plot}\PYG{p}{(}\PYG{n}{M}\PYG{p}{,}\PYG{n}{C23}\PYG{p}{,}\PYG{l+s+s1}{\PYGZsq{}}\PYG{l+s+s1}{gd}\PYG{l+s+s1}{\PYGZsq{}}\PYG{p}{,}\PYG{n}{ms}\PYG{o}{=}\PYG{l+m+mi}{8}\PYG{p}{,} \PYG{n}{markerfacecolor}\PYG{o}{=}\PYG{l+s+s1}{\PYGZsq{}}\PYG{l+s+s1}{None}\PYG{l+s+s1}{\PYGZsq{}}\PYG{p}{)}\PYG{p}{;} \PYG{c+c1}{\PYGZsh{} modelo 3}
\PYG{n}{mod4} \PYG{o}{=} \PYG{n}{plt}\PYG{o}{.}\PYG{n}{plot}\PYG{p}{(}\PYG{n}{M}\PYG{p}{,}\PYG{n}{C24}\PYG{p}{,}\PYG{l+s+s1}{\PYGZsq{}}\PYG{l+s+s1}{cv}\PYG{l+s+s1}{\PYGZsq{}}\PYG{p}{,}\PYG{n}{ms}\PYG{o}{=}\PYG{l+m+mi}{8}\PYG{p}{,} \PYG{n}{markerfacecolor}\PYG{o}{=}\PYG{l+s+s1}{\PYGZsq{}}\PYG{l+s+s1}{None}\PYG{l+s+s1}{\PYGZsq{}}\PYG{p}{)}\PYG{p}{;} \PYG{c+c1}{\PYGZsh{} modelo 4}
\PYG{n}{mod5} \PYG{o}{=} \PYG{n}{plt}\PYG{o}{.}\PYG{n}{plot}\PYG{p}{(}\PYG{n}{M}\PYG{p}{,}\PYG{n}{C25}\PYG{p}{,}\PYG{l+s+s1}{\PYGZsq{}}\PYG{l+s+s1}{m*}\PYG{l+s+s1}{\PYGZsq{}}\PYG{p}{,}\PYG{n}{ms}\PYG{o}{=}\PYG{l+m+mi}{8}\PYG{p}{,} \PYG{n}{markerfacecolor}\PYG{o}{=}\PYG{l+s+s1}{\PYGZsq{}}\PYG{l+s+s1}{None}\PYG{l+s+s1}{\PYGZsq{}}\PYG{p}{)}\PYG{p}{;} \PYG{c+c1}{\PYGZsh{} modelo 5}

\PYG{n}{plt}\PYG{o}{.}\PYG{n}{legend}\PYG{p}{(}\PYG{p}{(}\PYG{l+s+s1}{\PYGZsq{}}\PYG{l+s+s1}{medição}\PYG{l+s+s1}{\PYGZsq{}}\PYG{p}{,}
           \PYG{l+s+s1}{\PYGZsq{}}\PYG{l+s+s1}{ajuste 2a. ordem}\PYG{l+s+s1}{\PYGZsq{}}\PYG{p}{,}
           \PYG{l+s+s1}{\PYGZsq{}}\PYG{l+s+s1}{ajuste 3a. ordem}\PYG{l+s+s1}{\PYGZsq{}}\PYG{p}{,}
           \PYG{l+s+s1}{\PYGZsq{}}\PYG{l+s+s1}{ajuste 4a. ordem}\PYG{l+s+s1}{\PYGZsq{}}\PYG{p}{,}
           \PYG{l+s+s1}{\PYGZsq{}}\PYG{l+s+s1}{ajuste 5a. ordem}\PYG{l+s+s1}{\PYGZsq{}}\PYG{p}{)}\PYG{p}{)}\PYG{p}{;}
\end{sphinxVerbatim}

\noindent\sphinxincludegraphics{{codeSession-7-fit_27_0}.png}


\subsubsection{Exercício complementar}
\label{\detokenize{codeSession-7-fit:exercicio-complementar}}
\sphinxAtStartPar
Escreva uma função genérica que recebe a tabela de dados e o grau do modelo polinomial de ajuste e retorna os coeficientes do modelo, o desvio padrão do ajuste e os gráficos de dispersão conjuntamente com os dos modelos de ajuste.


\subsection{Problema 3}
\label{\detokenize{codeSession-7-fit:problema-3}}
\sphinxAtStartPar
A intensidade de radiação de uma substância radioativa foi medida em intervalos semestrais. A tabela de valores está disponível no arquivo \sphinxcode{\sphinxupquote{file\sphinxhyphen{}cs7\sphinxhyphen{}radiacao.csv}}, onte \(t\) é o tempo e \(\gamma\) é a intensidade relativa de radiação. Sabendo que a radioatividade decai exponencialmente com o tempo, \(\gamma(t) = ate^{-bt}\), estime a meia\sphinxhyphen{}vida radioativa (tempo no qual \(\gamma\) atinge metade de seu valor) da substância.


\subsubsection{Resolução}
\label{\detokenize{codeSession-7-fit:id2}}
\sphinxAtStartPar
Primeiramente, vamos ler o arquivo.

\begin{sphinxVerbatim}[commandchars=\\\{\}]
\PYG{n}{dados} \PYG{o}{=} \PYG{n}{np}\PYG{o}{.}\PYG{n}{loadtxt}\PYG{p}{(}\PYG{n}{fname}\PYG{o}{=}\PYG{l+s+s1}{\PYGZsq{}}\PYG{l+s+s1}{file\PYGZhy{}cs7\PYGZhy{}radiacao.csv}\PYG{l+s+s1}{\PYGZsq{}}\PYG{p}{,}\PYG{n}{delimiter}\PYG{o}{=}\PYG{l+s+s1}{\PYGZsq{}}\PYG{l+s+s1}{,}\PYG{l+s+s1}{\PYGZsq{}}\PYG{p}{,}\PYG{n}{skiprows}\PYG{o}{=}\PYG{l+m+mi}{1}\PYG{p}{)}
\end{sphinxVerbatim}

\sphinxAtStartPar
Agora, vamos coletar os dados e plotá\sphinxhyphen{}los apenas para verificar o comportamento da dispersão.

\begin{sphinxVerbatim}[commandchars=\\\{\}]
\PYG{n}{t} \PYG{o}{=} \PYG{n}{dados}\PYG{p}{[}\PYG{p}{:}\PYG{p}{,}\PYG{l+m+mi}{0}\PYG{p}{]}
\PYG{n}{g} \PYG{o}{=} \PYG{n}{dados}\PYG{p}{[}\PYG{p}{:}\PYG{p}{,}\PYG{l+m+mi}{1}\PYG{p}{]}
\PYG{n}{plt}\PYG{o}{.}\PYG{n}{plot}\PYG{p}{(}\PYG{n}{t}\PYG{p}{,}\PYG{n}{g}\PYG{p}{,}\PYG{l+s+s1}{\PYGZsq{}}\PYG{l+s+s1}{o}\PYG{l+s+s1}{\PYGZsq{}}\PYG{p}{)}\PYG{p}{;}
\end{sphinxVerbatim}

\noindent\sphinxincludegraphics{{codeSession-7-fit_34_0}.png}


\subsubsection{Teste de alinhamento}
\label{\detokenize{codeSession-7-fit:teste-de-alinhamento}}
\sphinxAtStartPar
Para ajustarmos um modelo não\sphinxhyphen{}linear à exponencial, antes precisamos convertê\sphinxhyphen{}la a uma forma linear, ou seja, linearizá\sphinxhyphen{}la. Para isso, aplicamos \(\log\) (nome da função \sphinxstyleemphasis{logaritmo natural} em Python) em ambos os lados da função. Ao deslocar o termo \(t\) fora da exponencial para o lado esquerdo da equação, teremos:
\begin{equation*}
\begin{split}\log\left(\frac{\gamma}{t}\right) = \log(a) - bt\end{split}
\end{equation*}
\sphinxAtStartPar
Definindo \(z = \log(\frac{\gamma}{t})\), \(x = \log(a)\), podemos agora fazer uma regressão linear nas variáveis \(t\) e \(z\) para o modelo
\begin{equation*}
\begin{split}z = x - bt\end{split}
\end{equation*}
\sphinxAtStartPar
Agora, plotando a dispersão no plano \((t,z)\), verificamos se a curva é aproximadamente uma reta.

\begin{sphinxVerbatim}[commandchars=\\\{\}]
\PYG{n}{z} \PYG{o}{=} \PYG{n}{np}\PYG{o}{.}\PYG{n}{log}\PYG{p}{(}\PYG{n}{g}\PYG{o}{/}\PYG{n}{t}\PYG{p}{)}
\PYG{n}{plt}\PYG{o}{.}\PYG{n}{plot}\PYG{p}{(}\PYG{n}{t}\PYG{p}{,}\PYG{n}{z}\PYG{p}{)}\PYG{p}{;}
\end{sphinxVerbatim}

\noindent\sphinxincludegraphics{{codeSession-7-fit_36_0}.png}

\sphinxAtStartPar
Como se vê, o teste de alinhamento mostra que a função exponencial é um modelo não\sphinxhyphen{}linear satisfatório para modelar o comportamento físico em questão.

\sphinxAtStartPar
Computando a regressão linear, temos:

\begin{sphinxVerbatim}[commandchars=\\\{\}]
\PYG{n}{b}\PYG{p}{,}\PYG{n}{x}\PYG{p}{,}\PYG{n}{R}\PYG{p}{,} \PYG{n}{p\PYGZus{}value}\PYG{p}{,} \PYG{n}{std\PYGZus{}err} \PYG{o}{=} \PYG{n}{linregress}\PYG{p}{(}\PYG{n}{t}\PYG{p}{,}\PYG{n}{z}\PYG{p}{)}
\PYG{n+nb}{print}\PYG{p}{(}\PYG{l+s+sa}{f}\PYG{l+s+s1}{\PYGZsq{}}\PYG{l+s+s1}{Regressão linear executada com inclinação = }\PYG{l+s+si}{\PYGZob{}}\PYG{n}{b}\PYG{l+s+si}{:}\PYG{l+s+s1}{.3f}\PYG{l+s+si}{\PYGZcb{}}\PYG{l+s+s1}{, interceptação = }\PYG{l+s+si}{\PYGZob{}}\PYG{n}{x}\PYG{l+s+si}{:}\PYG{l+s+s1}{.3f}\PYG{l+s+si}{\PYGZcb{}}\PYG{l+s+s1}{ e R2 = }\PYG{l+s+si}{\PYGZob{}}\PYG{n}{R}\PYG{o}{*}\PYG{n}{R}\PYG{l+s+si}{:}\PYG{l+s+s1}{.2f}\PYG{l+s+si}{\PYGZcb{}}\PYG{l+s+s1}{\PYGZsq{}}\PYG{p}{)}
\end{sphinxVerbatim}

\begin{sphinxVerbatim}[commandchars=\\\{\}]
Regressão linear executada com inclinação = \PYGZhy{}1.984, interceptação = 1.072 e R2 = 1.00
\end{sphinxVerbatim}

\sphinxAtStartPar
Vemos que, de fato, as variáveis têm uma altíssima correlação, visto que \(R^2 \approx 1\). Agora, para plotar o modelo de ajuste, recuperamos o valor de \(a\) operando inversamente e o usamos na curva do modelo para comparar com os dados experimentais.

\begin{sphinxVerbatim}[commandchars=\\\{\}]
\PYG{n}{a} \PYG{o}{=} \PYG{n}{np}\PYG{o}{.}\PYG{n}{exp}\PYG{p}{(}\PYG{n}{x}\PYG{p}{)}\PYG{p}{;} \PYG{c+c1}{\PYGZsh{} recuperando parâmetro de ajuste}
\PYG{n}{mod} \PYG{o}{=} \PYG{k}{lambda} \PYG{n}{t}\PYG{p}{:} \PYG{n}{a}\PYG{o}{*}\PYG{n}{t}\PYG{o}{*}\PYG{n}{np}\PYG{o}{.}\PYG{n}{exp}\PYG{p}{(}\PYG{n}{b}\PYG{o}{*}\PYG{n}{t}\PYG{p}{)}

\PYG{n}{plt}\PYG{o}{.}\PYG{n}{plot}\PYG{p}{(}\PYG{n}{t}\PYG{p}{,}\PYG{n}{g}\PYG{p}{,}\PYG{l+s+s1}{\PYGZsq{}}\PYG{l+s+s1}{o}\PYG{l+s+s1}{\PYGZsq{}}\PYG{p}{,}\PYG{n}{mfc}\PYG{o}{=}\PYG{l+s+s2}{\PYGZdq{}}\PYG{l+s+s2}{None}\PYG{l+s+s2}{\PYGZdq{}}\PYG{p}{)}\PYG{p}{;}
\PYG{n}{plt}\PYG{o}{.}\PYG{n}{plot}\PYG{p}{(}\PYG{n}{t}\PYG{p}{,}\PYG{n}{mod}\PYG{p}{(}\PYG{n}{t}\PYG{p}{)}\PYG{p}{,}\PYG{l+s+s1}{\PYGZsq{}}\PYG{l+s+s1}{o}\PYG{l+s+s1}{\PYGZsq{}}\PYG{p}{,}\PYG{n}{mfc}\PYG{o}{=}\PYG{l+s+s2}{\PYGZdq{}}\PYG{l+s+s2}{None}\PYG{l+s+s2}{\PYGZdq{}}\PYG{p}{)}\PYG{p}{;}
\PYG{n}{plt}\PYG{o}{.}\PYG{n}{legend}\PYG{p}{(}\PYG{p}{(}\PYG{l+s+s1}{\PYGZsq{}}\PYG{l+s+s1}{experimento}\PYG{l+s+s1}{\PYGZsq{}}\PYG{p}{,}\PYG{l+s+s1}{\PYGZsq{}}\PYG{l+s+s1}{ajuste}\PYG{l+s+s1}{\PYGZsq{}}\PYG{p}{)}\PYG{p}{)}\PYG{p}{;}
\end{sphinxVerbatim}

\noindent\sphinxincludegraphics{{codeSession-7-fit_41_0}.png}

\sphinxAtStartPar
O modelo está bem ajustado. Para estimar a meia\sphinxhyphen{}vida da substância, devemos encontrar o instante de tempo \(t_m\)  tal que \(\gamma(t_m) = 0.5\gamma_0\). Então, notemos que:
\begin{equation*}
\begin{split}\gamma(t_m) = at_me^{-bt_m} \Rightarrow 0.5\gamma_0 = at_me^{-bt_m}\end{split}
\end{equation*}
\sphinxAtStartPar
Todavia, não conseguimos uma relação explícita para \(t_m\), fato que nos leva a resolver um segundo problema de determinação de raízes do tipo \(f(t_m) = 0\) com
\begin{equation*}
\begin{split}f(t_m) = at_me^{-bt_m} - 0.5\gamma_0\end{split}
\end{equation*}
\sphinxAtStartPar
Vamos resolver este problema usando a função \sphinxcode{\sphinxupquote{fsolve}} do módulo \sphinxcode{\sphinxupquote{scipy.optimize}}, mas antes precisamos passar a ela uma estimativa inicial. Rapidamente, façamos uma análise gráfica da curva \(f(t_m)\) para \(t_m = [0,2]\) (este intervalo é obtido após algumas plotagens prévias).

\begin{sphinxVerbatim}[commandchars=\\\{\}]
\PYG{n}{f} \PYG{o}{=} \PYG{k}{lambda} \PYG{n}{tm}\PYG{p}{:} \PYG{n}{mod}\PYG{p}{(}\PYG{n}{tm}\PYG{p}{)} \PYG{o}{\PYGZhy{}} \PYG{l+m+mf}{0.5}\PYG{o}{*}\PYG{n}{g}\PYG{p}{[}\PYG{l+m+mi}{0}\PYG{p}{]}
\PYG{n}{ttm} \PYG{o}{=} \PYG{n}{np}\PYG{o}{.}\PYG{n}{linspace}\PYG{p}{(}\PYG{l+m+mi}{0}\PYG{p}{,}\PYG{l+m+mi}{2}\PYG{p}{)}
\PYG{n}{plt}\PYG{o}{.}\PYG{n}{plot}\PYG{p}{(}\PYG{n}{ttm}\PYG{p}{,}\PYG{n}{f}\PYG{p}{(}\PYG{n}{ttm}\PYG{p}{)}\PYG{p}{,}\PYG{n}{ttm}\PYG{p}{,}\PYG{l+m+mi}{0}\PYG{o}{*}\PYG{n}{f}\PYG{p}{(}\PYG{n}{ttm}\PYG{p}{)}\PYG{p}{)}\PYG{p}{;}
\end{sphinxVerbatim}

\noindent\sphinxincludegraphics{{codeSession-7-fit_44_0}.png}

\sphinxAtStartPar
Existem duas raízes no intervalo. Porém, observando os valores tabelados de \(t\), é fácil ver que o valor para a condição inicial deve ser maior do que \(t_0 = 0.5\) e, portanto, mais próximo da segunda raiz no gráfico. Então, escolhamos para \sphinxcode{\sphinxupquote{fsolve}} o valor inicial de \(t_m^0 = 1.25\).

\begin{sphinxVerbatim}[commandchars=\\\{\}]
\PYG{k+kn}{from} \PYG{n+nn}{scipy}\PYG{n+nn}{.}\PYG{n+nn}{optimize} \PYG{k+kn}{import} \PYG{n}{fsolve} 

\PYG{n}{tm} \PYG{o}{=} \PYG{n}{fsolve}\PYG{p}{(}\PYG{n}{f}\PYG{p}{,}\PYG{l+m+mf}{1.25}\PYG{p}{)}
\PYG{n+nb}{print}\PYG{p}{(}\PYG{l+s+sa}{f}\PYG{l+s+s1}{\PYGZsq{}}\PYG{l+s+s1}{Meia\PYGZhy{}vida localizada em tm = }\PYG{l+s+si}{\PYGZob{}}\PYG{n}{tm}\PYG{p}{[}\PYG{l+m+mi}{0}\PYG{p}{]}\PYG{l+s+si}{:}\PYG{l+s+s1}{.3f}\PYG{l+s+si}{\PYGZcb{}}\PYG{l+s+s1}{.}\PYG{l+s+s1}{\PYGZsq{}}\PYG{p}{)}
\end{sphinxVerbatim}

\begin{sphinxVerbatim}[commandchars=\\\{\}]
Meia\PYGZhy{}vida localizada em tm = 1.351.
\end{sphinxVerbatim}

\sphinxAtStartPar
Uma última verificação mostra que este valor de \(t_m\) é condizente com os dados experimentais, pois o seguinte erro é pequeno.

\begin{sphinxVerbatim}[commandchars=\\\{\}]
\PYG{n}{mod}\PYG{p}{(}\PYG{n}{tm}\PYG{p}{)} \PYG{o}{\PYGZhy{}} \PYG{l+m+mf}{0.5}\PYG{o}{*}\PYG{n}{g}\PYG{p}{[}\PYG{l+m+mi}{0}\PYG{p}{]}
\end{sphinxVerbatim}

\begin{sphinxVerbatim}[commandchars=\\\{\}]
array([1.11022302e\PYGZhy{}16])
\end{sphinxVerbatim}


\chapter{Code session 8}
\label{\detokenize{codeSession-8-integrate:code-session-8}}\label{\detokenize{codeSession-8-integrate::doc}}
\begin{sphinxVerbatim}[commandchars=\\\{\}]
\PYG{o}{\PYGZpc{}}\PYG{k}{matplotlib} inline
\PYG{k+kn}{import} \PYG{n+nn}{numpy} \PYG{k}{as} \PYG{n+nn}{np} 
\PYG{k+kn}{import} \PYG{n+nn}{matplotlib}\PYG{n+nn}{.}\PYG{n+nn}{pyplot} \PYG{k}{as} \PYG{n+nn}{plt} 
\end{sphinxVerbatim}

\sphinxAtStartPar
A integração numérica em uma variável pode ser realizada em Python utilizando a função \sphinxcode{\sphinxupquote{quad}} do módulo \sphinxcode{\sphinxupquote{scipy.integrate}}.


\section{\sphinxstyleliteralintitle{\sphinxupquote{quad}}}
\label{\detokenize{codeSession-8-integrate:quad}}
\sphinxAtStartPar
Esta função calcula a integral definida \(\int_a^b f(x) \, dx\) numericamente através de regras de quadratura.

\sphinxAtStartPar
Os argumentos de entrada obrigatórios desta função são:
\begin{enumerate}
\sphinxsetlistlabels{\arabic}{enumi}{enumii}{}{.}%
\item {} 
\sphinxAtStartPar
a função \sphinxcode{\sphinxupquote{f}} a ser integrada

\item {} 
\sphinxAtStartPar
o limite inferior \sphinxcode{\sphinxupquote{a}}

\item {} 
\sphinxAtStartPar
o limite superior \sphinxcode{\sphinxupquote{b}}

\end{enumerate}

\sphinxAtStartPar
Os principais argumentos de saída são:
\begin{itemize}
\item {} 
\sphinxAtStartPar
\sphinxcode{\sphinxupquote{y}}: valor numérico da integral

\item {} 
\sphinxAtStartPar
\sphinxcode{\sphinxupquote{abserr}}: estimativa do erro absoluto

\end{itemize}

\sphinxAtStartPar
Como importá\sphinxhyphen{}la?

\begin{sphinxVerbatim}[commandchars=\\\{\}]
\PYG{k+kn}{from} \PYG{n+nn}{scipy}\PYG{n+nn}{.}\PYG{n+nn}{integrate} \PYG{k+kn}{import} \PYG{n}{quad}
\end{sphinxVerbatim}

\begin{sphinxVerbatim}[commandchars=\\\{\}]
\PYG{k+kn}{from} \PYG{n+nn}{scipy}\PYG{n+nn}{.}\PYG{n+nn}{integrate} \PYG{k+kn}{import} \PYG{n}{quad}
\end{sphinxVerbatim}


\subsection{Problema 1}
\label{\detokenize{codeSession-8-integrate:problema-1}}
\sphinxAtStartPar
O período de um pêndulo simples de comprimento \(L\) é \(\tau = 4\sqrt{ \dfrac{L}{g} } h(\theta_0)\), onde \(g\) é a aceleração gravitacional, \(\theta_0\) representa a amplitude angular, e
\begin{equation*}
\begin{split}h(\theta_0) = \int_0^{\pi/2} \dfrac{1}{ \sqrt{ 1 - \textrm{sen}^2(\theta_0/2)\textrm{sen}^2 \theta } } \, d\theta\end{split}
\end{equation*}
\sphinxAtStartPar
Calcule \(h(\tau)\), para \(\tau = 15^{\circ}, \, 30^{\circ}, \, 45^{\circ}\) e compare estes valores com \(h(0^{\circ}) = \pi/2\) (a aproximação usada para pequenas amplitudes).


\subsubsection{Resolução}
\label{\detokenize{codeSession-8-integrate:resolucao}}
\sphinxAtStartPar
Em primeiro lugar, fazemos os cálculos diretos da integral para os distintos valores de \(\tau\).

\begin{sphinxVerbatim}[commandchars=\\\{\}]
\PYG{c+c1}{\PYGZsh{} cálculo direto das integrais caso a caso}

\PYG{n}{theta0} \PYG{o}{=} \PYG{n}{np}\PYG{o}{.}\PYG{n}{array}\PYG{p}{(}\PYG{p}{[}\PYG{l+m+mi}{0}\PYG{p}{,}\PYG{l+m+mi}{15}\PYG{p}{,}\PYG{l+m+mi}{30}\PYG{p}{,}\PYG{l+m+mi}{45}\PYG{p}{]}\PYG{p}{)} \PYG{c+c1}{\PYGZsh{} ângulos}

\PYG{n}{vals}\PYG{p}{,}\PYG{n}{errs} \PYG{o}{=} \PYG{p}{[}\PYG{p}{]}\PYG{p}{,}\PYG{p}{[}\PYG{p}{]} \PYG{c+c1}{\PYGZsh{} integrais, erros}

\PYG{k}{for} \PYG{n}{t} \PYG{o+ow}{in} \PYG{n}{theta0}\PYG{p}{:}
    \PYG{n}{f} \PYG{o}{=} \PYG{k}{lambda} \PYG{n}{theta}\PYG{p}{:} \PYG{l+m+mi}{1}\PYG{o}{/}\PYG{p}{(}\PYG{n}{np}\PYG{o}{.}\PYG{n}{sqrt}\PYG{p}{(}\PYG{l+m+mf}{1.} \PYG{o}{\PYGZhy{}} \PYG{n}{np}\PYG{o}{.}\PYG{n}{sin}\PYG{p}{(}\PYG{n}{t}\PYG{o}{/}\PYG{l+m+mi}{2}\PYG{p}{)}\PYG{o}{*}\PYG{o}{*}\PYG{l+m+mi}{2} \PYG{o}{*} \PYG{n}{np}\PYG{o}{.}\PYG{n}{sin}\PYG{p}{(}\PYG{n}{theta}\PYG{p}{)}\PYG{o}{*}\PYG{o}{*}\PYG{l+m+mi}{2}\PYG{p}{)}\PYG{p}{)}
    \PYG{n}{v}\PYG{p}{,}\PYG{n}{e} \PYG{o}{=} \PYG{n}{quad}\PYG{p}{(}\PYG{n}{f}\PYG{p}{,}\PYG{l+m+mi}{0}\PYG{p}{,}\PYG{n}{np}\PYG{o}{.}\PYG{n}{pi}\PYG{o}{/}\PYG{l+m+mi}{2}\PYG{p}{)}
    \PYG{n+nb}{print}\PYG{p}{(}\PYG{l+s+sa}{f}\PYG{l+s+s1}{\PYGZsq{}}\PYG{l+s+s1}{Integral h(}\PYG{l+s+si}{\PYGZob{}}\PYG{n}{t}\PYG{l+s+si}{\PYGZcb{}}\PYG{l+s+s1}{) = }\PYG{l+s+si}{\PYGZob{}}\PYG{n}{v}\PYG{l+s+si}{:}\PYG{l+s+s1}{g}\PYG{l+s+si}{\PYGZcb{}}\PYG{l+s+s1}{\PYGZsq{}}\PYG{p}{)}
    \PYG{n}{vals}\PYG{o}{.}\PYG{n}{append}\PYG{p}{(}\PYG{n}{v}\PYG{p}{)}
    \PYG{n}{errs}\PYG{o}{.}\PYG{n}{append}\PYG{p}{(}\PYG{n}{e}\PYG{p}{)}

\PYG{c+c1}{\PYGZsh{} converte listas para arrays}
\PYG{n}{vals} \PYG{o}{=} \PYG{n}{np}\PYG{o}{.}\PYG{n}{asarray}\PYG{p}{(}\PYG{n}{vals}\PYG{p}{)}
\PYG{n}{errs} \PYG{o}{=} \PYG{n}{np}\PYG{o}{.}\PYG{n}{asarray}\PYG{p}{(}\PYG{n}{errs}\PYG{p}{)}
\end{sphinxVerbatim}

\begin{sphinxVerbatim}[commandchars=\\\{\}]
Integral h(0) = 1.5708
Integral h(15) = 2.49203
Integral h(30) = 1.79372
Integral h(45) = 1.67896
\end{sphinxVerbatim}

\sphinxAtStartPar
Vemos que o valor das integrais é muito sensível. Para realizar uma comparação mais interessante, utilizaremos um cálculo relativo tomando o valor em \(h(15)\) como referência.

\begin{sphinxVerbatim}[commandchars=\\\{\}]
\PYG{n}{plt}\PYG{o}{.}\PYG{n}{stem}\PYG{p}{(}\PYG{n}{theta0}\PYG{p}{,} \PYG{p}{(}\PYG{n}{vals} \PYG{o}{\PYGZhy{}} \PYG{n}{vals}\PYG{p}{[}\PYG{l+m+mi}{0}\PYG{p}{]}\PYG{p}{)}\PYG{o}{/}\PYG{n}{vals}\PYG{p}{[}\PYG{l+m+mi}{0}\PYG{p}{]}\PYG{p}{,} \PYG{n}{use\PYGZus{}line\PYGZus{}collection}\PYG{o}{=}\PYG{k+kc}{True}\PYG{p}{)}\PYG{p}{;}
\PYG{n}{plt}\PYG{o}{.}\PYG{n}{xlabel}\PYG{p}{(}\PYG{l+s+s1}{\PYGZsq{}}\PYG{l+s+s1}{\PYGZdl{}}\PYG{l+s+se}{\PYGZbs{}\PYGZbs{}}\PYG{l+s+s1}{theta\PYGZus{}0\PYGZdl{}}\PYG{l+s+s1}{\PYGZsq{}}\PYG{p}{)}
\PYG{n}{plt}\PYG{o}{.}\PYG{n}{ylabel}\PYG{p}{(}\PYG{l+s+s1}{\PYGZsq{}}\PYG{l+s+s1}{erro relativo}\PYG{l+s+s1}{\PYGZsq{}}\PYG{p}{)}\PYG{p}{;}
\end{sphinxVerbatim}

\noindent\sphinxincludegraphics{{codeSession-8-integrate_8_0}.png}

\sphinxAtStartPar
Podemos verificar que a mudança de valor da integral entre \(15^{\circ}\) e \(30^{\circ}\) é da ordem de 60\%, enquanto que nos demais casos, ela se limita a 20\%.


\subsection{Problema 2}
\label{\detokenize{codeSession-8-integrate:problema-2}}
\sphinxAtStartPar
Uma corrente elétrica alternada é descrita por
\begin{equation*}
\begin{split}i(t) = i_0 \left( \textrm{sen}\left( \frac{\pi t}{t_0} \right) - \beta \, \textrm{sen}\left( \frac{2 \pi t}{t_0} \right)  \right),\end{split}
\end{equation*}
\sphinxAtStartPar
onde \(i_0 = 1 \, A\), \(t_0 = 0.05 \, s\) e \(\beta =0.2\). Calcule a corrente RMS definida por
\begin{equation*}
\begin{split}i_{rms} = \sqrt{ \dfrac{1}{t_0} \int_0^{t_0} i^2(t) \, dt}\end{split}
\end{equation*}

\subsubsection{Resolução}
\label{\detokenize{codeSession-8-integrate:id1}}
\sphinxAtStartPar
Neste caso, basta passarmos os valores iniciais e finais para computar a integral.

\begin{sphinxVerbatim}[commandchars=\\\{\}]
\PYG{n}{i0}\PYG{p}{,} \PYG{n}{t0}\PYG{p}{,} \PYG{n}{beta} \PYG{o}{=} \PYG{l+m+mf}{1.0}\PYG{p}{,} \PYG{l+m+mf}{0.05}\PYG{p}{,} \PYG{l+m+mf}{0.2} \PYG{c+c1}{\PYGZsh{} parâmetros iniciais}

\PYG{n}{i2} \PYG{o}{=} \PYG{k}{lambda} \PYG{n}{t}\PYG{p}{:} \PYG{p}{(} \PYG{n}{i0} \PYG{o}{*} \PYG{p}{(} \PYG{n}{np}\PYG{o}{.}\PYG{n}{sin}\PYG{p}{(} \PYG{p}{(}\PYG{n}{np}\PYG{o}{.}\PYG{n}{pi}\PYG{o}{*}\PYG{n}{t}\PYG{p}{)} \PYG{o}{/} \PYG{n}{t0} \PYG{p}{)} \PYG{o}{\PYGZhy{}} \PYG{n}{beta} \PYG{o}{*} \PYG{n}{np}\PYG{o}{.}\PYG{n}{sin}\PYG{p}{(} \PYG{p}{(}\PYG{l+m+mi}{2}\PYG{o}{*}\PYG{n}{np}\PYG{o}{.}\PYG{n}{pi}\PYG{o}{*}\PYG{n}{t}\PYG{p}{)} \PYG{o}{/} \PYG{n}{t0} \PYG{p}{)} \PYG{p}{)} \PYG{p}{)}\PYG{o}{*}\PYG{o}{*}\PYG{l+m+mi}{2} \PYG{c+c1}{\PYGZsh{} função}

\PYG{n}{i\PYGZus{}rms} \PYG{o}{=} \PYG{n}{np}\PYG{o}{.}\PYG{n}{sqrt}\PYG{p}{(} \PYG{l+m+mf}{1.0}\PYG{o}{/}\PYG{n}{t0} \PYG{o}{*} \PYG{n}{quad}\PYG{p}{(}\PYG{n}{i2}\PYG{p}{,} \PYG{l+m+mi}{0}\PYG{p}{,} \PYG{n}{t0}\PYG{p}{)}\PYG{p}{[}\PYG{l+m+mi}{0}\PYG{p}{]} \PYG{p}{)}

\PYG{n+nb}{print}\PYG{p}{(}\PYG{l+s+sa}{f}\PYG{l+s+s1}{\PYGZsq{}}\PYG{l+s+s1}{Corrente RMS = }\PYG{l+s+si}{\PYGZob{}}\PYG{n}{i\PYGZus{}rms}\PYG{l+s+si}{:}\PYG{l+s+s1}{g}\PYG{l+s+si}{\PYGZcb{}}\PYG{l+s+s1}{ A}\PYG{l+s+s1}{\PYGZsq{}}\PYG{p}{)}
\end{sphinxVerbatim}

\begin{sphinxVerbatim}[commandchars=\\\{\}]
Corrente RMS = 0.72111 A
\end{sphinxVerbatim}


\section{Regra do Trapézio Generalizada}
\label{\detokenize{codeSession-8-integrate:regra-do-trapezio-generalizada}}
\sphinxAtStartPar
A regra do trapézio generalizada (composta) pode ser calculada usando

\begin{sphinxVerbatim}[commandchars=\\\{\}]
\PYG{n}{scipy}\PYG{o}{.}\PYG{n}{integrate}\PYG{o}{.}\PYG{n}{cumtrapz}
\end{sphinxVerbatim}

\sphinxAtStartPar
Vamos utilizar a função \(i(t)\) do Problema 2 e estimar sua integral no intervalo \(t = [0,5]\) utilizando a regra do trapézio generalizada.

\begin{sphinxVerbatim}[commandchars=\\\{\}]
\PYG{c+c1}{\PYGZsh{} Visualização}
\PYG{n}{t} \PYG{o}{=} \PYG{n}{np}\PYG{o}{.}\PYG{n}{linspace}\PYG{p}{(}\PYG{l+m+mi}{0}\PYG{p}{,}\PYG{l+m+mi}{5}\PYG{p}{)}
\PYG{n}{i} \PYG{o}{=} \PYG{k}{lambda} \PYG{n}{t}\PYG{p}{:} \PYG{n}{i0} \PYG{o}{*} \PYG{p}{(} \PYG{n}{np}\PYG{o}{.}\PYG{n}{sin}\PYG{p}{(} \PYG{p}{(}\PYG{n}{np}\PYG{o}{.}\PYG{n}{pi}\PYG{o}{*}\PYG{n}{t}\PYG{p}{)} \PYG{o}{/} \PYG{n}{t0} \PYG{p}{)} \PYG{o}{\PYGZhy{}} \PYG{n}{beta} \PYG{o}{*} \PYG{n}{np}\PYG{o}{.}\PYG{n}{sin}\PYG{p}{(} \PYG{p}{(}\PYG{l+m+mi}{2}\PYG{o}{*}\PYG{n}{np}\PYG{o}{.}\PYG{n}{pi}\PYG{o}{*}\PYG{n}{t}\PYG{p}{)} \PYG{o}{/} \PYG{n}{t0} \PYG{p}{)} \PYG{p}{)}  \PYG{c+c1}{\PYGZsh{} função}

\PYG{n}{plt}\PYG{o}{.}\PYG{n}{plot}\PYG{p}{(}\PYG{n}{t}\PYG{p}{,}\PYG{n}{i}\PYG{p}{(}\PYG{n}{t}\PYG{p}{)}\PYG{p}{,}\PYG{l+s+s1}{\PYGZsq{}}\PYG{l+s+s1}{o}\PYG{l+s+s1}{\PYGZsq{}}\PYG{p}{)}\PYG{p}{;}
\PYG{n}{plt}\PYG{o}{.}\PYG{n}{xlabel}\PYG{p}{(}\PYG{l+s+s1}{\PYGZsq{}}\PYG{l+s+s1}{\PYGZdl{}t\PYGZdl{}}\PYG{l+s+s1}{\PYGZsq{}}\PYG{p}{)}
\PYG{n}{plt}\PYG{o}{.}\PYG{n}{ylabel}\PYG{p}{(}\PYG{l+s+s1}{\PYGZsq{}}\PYG{l+s+s1}{\PYGZdl{}i(t)\PYGZdl{}}\PYG{l+s+s1}{\PYGZsq{}}\PYG{p}{)}\PYG{p}{;}
\end{sphinxVerbatim}

\noindent\sphinxincludegraphics{{codeSession-8-integrate_14_0}.png}

\begin{sphinxVerbatim}[commandchars=\\\{\}]
\PYG{k+kn}{from} \PYG{n+nn}{scipy}\PYG{n+nn}{.}\PYG{n+nn}{integrate} \PYG{k+kn}{import} \PYG{n}{cumtrapz}

\PYG{n}{i} \PYG{o}{=} \PYG{k}{lambda} \PYG{n}{t}\PYG{p}{:} \PYG{n}{i0} \PYG{o}{*} \PYG{p}{(} \PYG{n}{np}\PYG{o}{.}\PYG{n}{sin}\PYG{p}{(} \PYG{p}{(}\PYG{n}{np}\PYG{o}{.}\PYG{n}{pi}\PYG{o}{*}\PYG{n}{t}\PYG{p}{)} \PYG{o}{/} \PYG{n}{t0} \PYG{p}{)} \PYG{o}{\PYGZhy{}} \PYG{n}{beta} \PYG{o}{*} \PYG{n}{np}\PYG{o}{.}\PYG{n}{sin}\PYG{p}{(} \PYG{p}{(}\PYG{l+m+mi}{2}\PYG{o}{*}\PYG{n}{np}\PYG{o}{.}\PYG{n}{pi}\PYG{o}{*}\PYG{n}{t}\PYG{p}{)} \PYG{o}{/} \PYG{n}{t0} \PYG{p}{)} \PYG{p}{)} \PYG{c+c1}{\PYGZsh{} função}

\PYG{n}{T} \PYG{o}{=} \PYG{n}{cumtrapz}\PYG{p}{(}\PYG{n}{i}\PYG{p}{(}\PYG{n}{t}\PYG{p}{)}\PYG{p}{,} \PYG{n}{t}\PYG{p}{)}\PYG{p}{[}\PYG{o}{\PYGZhy{}}\PYG{l+m+mi}{1}\PYG{p}{]} \PYG{c+c1}{\PYGZsh{} pega último valor, já que é cumulativa}

\PYG{n+nb}{print}\PYG{p}{(}\PYG{l+s+sa}{f}\PYG{l+s+s1}{\PYGZsq{}}\PYG{l+s+s1}{Integral por Trapézio = }\PYG{l+s+si}{\PYGZob{}}\PYG{n}{T}\PYG{l+s+si}{:}\PYG{l+s+s1}{g}\PYG{l+s+si}{\PYGZcb{}}\PYG{l+s+s1}{\PYGZsq{}}\PYG{p}{)}
\end{sphinxVerbatim}

\begin{sphinxVerbatim}[commandchars=\\\{\}]
Integral por Trapézio = \PYGZhy{}4.97605e\PYGZhy{}15
\end{sphinxVerbatim}


\section{Quadratura Gaussiana (QG)}
\label{\detokenize{codeSession-8-integrate:quadratura-gaussiana-qg}}
\sphinxAtStartPar
O cálculo de uma integral por quadratura Gaussiana pode ser calculado usando

\begin{sphinxVerbatim}[commandchars=\\\{\}]
\PYG{n}{scipy}\PYG{o}{.}\PYG{n}{integrate}\PYG{o}{.}\PYG{n}{quadrature}
\end{sphinxVerbatim}

\sphinxAtStartPar
Vamos utilizar a função \(i(t)\) do Problema 2 e estimar sua integral no intervalo \(t = [0,5]\) utilizando a QG para várias ordens (controladas pelo argumento \sphinxcode{\sphinxupquote{miniter}}).

\begin{sphinxVerbatim}[commandchars=\\\{\}]
\PYG{k+kn}{from} \PYG{n+nn}{scipy}\PYG{n+nn}{.}\PYG{n+nn}{integrate} \PYG{k+kn}{import} \PYG{n}{quadrature}

\PYG{k}{for} \PYG{n}{ordem} \PYG{o+ow}{in} \PYG{n+nb}{range}\PYG{p}{(}\PYG{l+m+mi}{1}\PYG{p}{,}\PYG{l+m+mi}{11}\PYG{p}{)}\PYG{p}{:}
    \PYG{n}{I\PYGZus{}QG}\PYG{p}{,}\PYG{n}{err\PYGZus{}QG} \PYG{o}{=} \PYG{n}{quadrature}\PYG{p}{(}\PYG{n}{i}\PYG{p}{,}\PYG{l+m+mi}{0}\PYG{p}{,}\PYG{l+m+mi}{5}\PYG{p}{,}\PYG{n}{miniter}\PYG{o}{=}\PYG{n}{ordem}\PYG{p}{)}
    \PYG{n+nb}{print}\PYG{p}{(}\PYG{l+s+sa}{f}\PYG{l+s+s1}{\PYGZsq{}}\PYG{l+s+s1}{Integral por QG (ordem }\PYG{l+s+si}{\PYGZob{}}\PYG{n}{ordem}\PYG{l+s+si}{\PYGZcb{}}\PYG{l+s+s1}{) = }\PYG{l+s+si}{\PYGZob{}}\PYG{n}{I\PYGZus{}QG}\PYG{l+s+si}{:}\PYG{l+s+s1}{g}\PYG{l+s+si}{\PYGZcb{}}\PYG{l+s+s1}{\PYGZsq{}}\PYG{p}{)}
\end{sphinxVerbatim}

\begin{sphinxVerbatim}[commandchars=\\\{\}]
Integral por QG (ordem 1) = 3.60822e\PYGZhy{}14
Integral por QG (ordem 2) = 9.71445e\PYGZhy{}15
Integral por QG (ordem 3) = 4.35416e\PYGZhy{}14
Integral por QG (ordem 4) = 6.92502e\PYGZhy{}14
Integral por QG (ordem 5) = 6.58501e\PYGZhy{}14
Integral por QG (ordem 6) = \PYGZhy{}5.7801e\PYGZhy{}14
Integral por QG (ordem 7) = \PYGZhy{}1.39472e\PYGZhy{}14
Integral por QG (ordem 8) = \PYGZhy{}1.60288e\PYGZhy{}14
Integral por QG (ordem 9) = \PYGZhy{}5.89806e\PYGZhy{}16
Integral por QG (ordem 10) = \PYGZhy{}4.02976e\PYGZhy{}14
\end{sphinxVerbatim}


\chapter{Code session 9}
\label{\detokenize{codeSession-9-solve_ivp:code-session-9}}\label{\detokenize{codeSession-9-solve_ivp::doc}}
\begin{sphinxVerbatim}[commandchars=\\\{\}]
\PYG{k+kn}{import} \PYG{n+nn}{numpy} \PYG{k}{as} \PYG{n+nn}{np} 
\PYG{k+kn}{import} \PYG{n+nn}{matplotlib}\PYG{n+nn}{.}\PYG{n+nn}{pyplot} \PYG{k}{as} \PYG{n+nn}{plt}
\end{sphinxVerbatim}

\sphinxAtStartPar
A integração numérica de uma EDO pode ser realizada em Python utilizando a função \sphinxcode{\sphinxupquote{solve\_ivp}} do módulo \sphinxcode{\sphinxupquote{scipy.integrate}}.


\section{\sphinxstyleliteralintitle{\sphinxupquote{solve\_ivp}}}
\label{\detokenize{codeSession-9-solve_ivp:solve-ivp}}
\sphinxAtStartPar
Esta função resolve um problema de valor inicial (PVI) para uma EDO ou um sistema de EDOs. Por padrão, o método de resolução é um algoritmo de Runge\sphinxhyphen{}Kutta com precisão de 4a. ordem.

\sphinxAtStartPar
Os argumentos de entrada obrigatórios desta função são:
\begin{enumerate}
\sphinxsetlistlabels{\arabic}{enumi}{enumii}{}{.}%
\item {} 
\sphinxAtStartPar
a função \sphinxcode{\sphinxupquote{f(t,y)}} a ser integrada

\item {} 
\sphinxAtStartPar
o domínio de integração, definido como uma tupla \sphinxcode{\sphinxupquote{(t0,tf)}}

\item {} 
\sphinxAtStartPar
a condição inicial \sphinxcode{\sphinxupquote{y0}}

\end{enumerate}

\sphinxAtStartPar
O principal argumento de saída é um objeto \sphinxcode{\sphinxupquote{sol}}, em que:
\begin{itemize}
\item {} 
\sphinxAtStartPar
\sphinxcode{\sphinxupquote{sol.t}}: retorna os valores do domínio

\item {} 
\sphinxAtStartPar
\sphinxcode{\sphinxupquote{sol.y}}: retorna os valores da solução numérica

\end{itemize}

\sphinxAtStartPar
Como importá\sphinxhyphen{}la?

\begin{sphinxVerbatim}[commandchars=\\\{\}]
\PYG{k+kn}{from} \PYG{n+nn}{scipy}\PYG{n+nn}{.}\PYG{n+nn}{integrate} \PYG{k+kn}{import} \PYG{n}{solve\PYGZus{}ivp}
\end{sphinxVerbatim}

\begin{sphinxVerbatim}[commandchars=\\\{\}]
\PYG{k+kn}{from} \PYG{n+nn}{scipy}\PYG{n+nn}{.}\PYG{n+nn}{integrate} \PYG{k+kn}{import} \PYG{n}{solve\PYGZus{}ivp}
\end{sphinxVerbatim}

\sphinxAtStartPar
\sphinxstylestrong{Exemplo:} Resolver numericamente o PVI
\begin{equation*}
\begin{split}\begin{cases}
y'(t) = -2y(t)\\
y(0) = 1 \\
0 < t \le 3
\end{cases}\end{split}
\end{equation*}
\sphinxAtStartPar
para \(h=1.0,0.1,0.001\).

\sphinxAtStartPar
Compare a solução numérica com a analítica: \(y_{an}(t) = e^{-t^2}\).

\begin{sphinxVerbatim}[commandchars=\\\{\}]
\PYG{n}{f} \PYG{o}{=} \PYG{k}{lambda} \PYG{n}{t}\PYG{p}{,}\PYG{n}{y}\PYG{p}{:} \PYG{o}{\PYGZhy{}}\PYG{l+m+mi}{2}\PYG{o}{*}\PYG{n}{y}\PYG{o}{*}\PYG{n}{t}
\PYG{n}{yan} \PYG{o}{=} \PYG{k}{lambda} \PYG{n}{t}\PYG{p}{:} \PYG{n}{np}\PYG{o}{.}\PYG{n}{exp}\PYG{p}{(}\PYG{o}{\PYGZhy{}}\PYG{n}{t}\PYG{o}{*}\PYG{o}{*}\PYG{l+m+mi}{2}\PYG{p}{)}

\PYG{n}{y0} \PYG{o}{=} \PYG{l+m+mi}{1}
\PYG{n}{a} \PYG{o}{=} \PYG{l+m+mf}{0.0}
\PYG{n}{b} \PYG{o}{=} \PYG{l+m+mf}{3.0}
\PYG{n}{T} \PYG{o}{=} \PYG{p}{[}\PYG{l+m+mf}{1.0}\PYG{p}{,}\PYG{l+m+mf}{0.1}\PYG{p}{,}\PYG{l+m+mf}{0.001}\PYG{p}{]}

\PYG{k}{for} \PYG{n}{k} \PYG{o+ow}{in} \PYG{n}{T}\PYG{p}{:}
    
    \PYG{n}{t} \PYG{o}{=} \PYG{n}{np}\PYG{o}{.}\PYG{n}{arange}\PYG{p}{(}\PYG{n}{a}\PYG{p}{,} \PYG{n}{b}\PYG{o}{+}\PYG{n}{k}\PYG{p}{,} \PYG{n}{k}\PYG{p}{)}
    \PYG{n}{y} \PYG{o}{=} \PYG{n}{solve\PYGZus{}ivp}\PYG{p}{(}\PYG{n}{f}\PYG{p}{,} \PYG{p}{(}\PYG{n}{a}\PYG{p}{,}\PYG{n}{b}\PYG{p}{)}\PYG{p}{,} \PYG{p}{[}\PYG{n}{y0}\PYG{p}{]}\PYG{p}{)}

    \PYG{n}{y2} \PYG{o}{=} \PYG{n}{t}\PYG{o}{*}\PYG{l+m+mi}{0}
    \PYG{n}{y2}\PYG{p}{[}\PYG{l+m+mi}{0}\PYG{p}{]} \PYG{o}{=} \PYG{n}{y0}
    \PYG{n}{dt} \PYG{o}{=} \PYG{p}{(}\PYG{n}{b}\PYG{o}{\PYGZhy{}}\PYG{n}{a}\PYG{p}{)}\PYG{o}{/} \PYG{p}{(}\PYG{n+nb}{len}\PYG{p}{(}\PYG{n}{t}\PYG{p}{)}\PYG{o}{\PYGZhy{}}\PYG{l+m+mi}{1}\PYG{p}{)}
    \PYG{k}{for} \PYG{n}{i} \PYG{o+ow}{in} \PYG{n+nb}{range}\PYG{p}{(}\PYG{l+m+mi}{0}\PYG{p}{,}\PYG{n+nb}{len}\PYG{p}{(}\PYG{n}{t}\PYG{p}{)}\PYG{o}{\PYGZhy{}}\PYG{l+m+mi}{1}\PYG{p}{)}\PYG{p}{:}
        \PYG{n}{y2}\PYG{p}{[}\PYG{n}{i}\PYG{o}{+}\PYG{l+m+mi}{1}\PYG{p}{]} \PYG{o}{=} \PYG{n}{y2}\PYG{p}{[}\PYG{n}{i}\PYG{p}{]} \PYG{o}{+} \PYG{n}{f}\PYG{p}{(}\PYG{n}{t}\PYG{p}{[}\PYG{n}{i}\PYG{p}{]}\PYG{p}{,}\PYG{n}{y2}\PYG{p}{[}\PYG{n}{i}\PYG{p}{]}\PYG{p}{)}\PYG{o}{*}\PYG{n}{dt}
            
    \PYG{n}{fig} \PYG{o}{=} \PYG{n}{plt}\PYG{o}{.}\PYG{n}{figure}\PYG{p}{(}\PYG{p}{)}
    \PYG{n}{ax} \PYG{o}{=} \PYG{n}{fig}\PYG{o}{.}\PYG{n}{add\PYGZus{}subplot}\PYG{p}{(}\PYG{l+m+mi}{111}\PYG{p}{)}
    \PYG{c+c1}{\PYGZsh{}plt.plot(t,y,\PYGZsq{}k\PYGZsq{},label=\PYGZsq{}odeint\PYGZsq{})}
    \PYG{n}{plt}\PYG{o}{.}\PYG{n}{plot}\PYG{p}{(}\PYG{n}{t}\PYG{p}{,}\PYG{n}{y2}\PYG{p}{,}\PYG{l+s+s1}{\PYGZsq{}}\PYG{l+s+s1}{r}\PYG{l+s+s1}{\PYGZsq{}}\PYG{p}{,}\PYG{n}{label}\PYG{o}{=}\PYG{l+s+s1}{\PYGZsq{}}\PYG{l+s+s1}{Euler}\PYG{l+s+s1}{\PYGZsq{}}\PYG{p}{)}
    \PYG{n}{plt}\PYG{o}{.}\PYG{n}{plot}\PYG{p}{(}\PYG{n}{t}\PYG{p}{,}\PYG{n}{yan}\PYG{p}{(}\PYG{n}{t}\PYG{p}{)}\PYG{p}{,}\PYG{l+s+s1}{\PYGZsq{}}\PYG{l+s+s1}{b}\PYG{l+s+s1}{\PYGZsq{}}\PYG{p}{,}\PYG{n}{label}\PYG{o}{=}\PYG{l+s+s1}{\PYGZsq{}}\PYG{l+s+s1}{sol. analítica}\PYG{l+s+s1}{\PYGZsq{}}\PYG{p}{)}
    \PYG{n}{plt}\PYG{o}{.}\PYG{n}{legend}\PYG{p}{(}\PYG{p}{)}
    \PYG{n}{s} \PYG{o}{=} \PYG{l+s+s1}{\PYGZsq{}}\PYG{l+s+s1}{Resolução de EDO: sol. analítica x mét. Euler (dt = }\PYG{l+s+s1}{\PYGZsq{}} \PYG{o}{+} \PYG{n+nb}{str}\PYG{p}{(}\PYG{n}{k}\PYG{p}{)} \PYG{o}{+} \PYG{l+s+s1}{\PYGZsq{}}\PYG{l+s+s1}{)}\PYG{l+s+s1}{\PYGZsq{}} 
    \PYG{n}{plt}\PYG{o}{.}\PYG{n}{title}\PYG{p}{(}\PYG{n}{s}\PYG{p}{)}
    \PYG{n}{plt}\PYG{o}{.}\PYG{n}{xlabel}\PYG{p}{(}\PYG{l+s+s1}{\PYGZsq{}}\PYG{l+s+s1}{\PYGZdl{}t\PYGZdl{}}\PYG{l+s+s1}{\PYGZsq{}}\PYG{p}{)} 
    \PYG{n}{plt}\PYG{o}{.}\PYG{n}{ylabel}\PYG{p}{(}\PYG{l+s+s1}{\PYGZsq{}}\PYG{l+s+s1}{\PYGZdl{}y(t)\PYGZdl{}}\PYG{l+s+s1}{\PYGZsq{}}\PYG{p}{)}
\end{sphinxVerbatim}

\noindent\sphinxincludegraphics{{codeSession-9-solve_ivp_5_0}.png}

\noindent\sphinxincludegraphics{{codeSession-9-solve_ivp_5_1}.png}

\noindent\sphinxincludegraphics{{codeSession-9-solve_ivp_5_2}.png}

\sphinxAtStartPar
\sphinxstylestrong{Exemplo:} Resolver numericamente o PVI
\begin{equation*}
\begin{split}\begin{cases}
y'(t) = y+t\\
y(0) = 1 \\
0 < t \le 5
\end{cases}\end{split}
\end{equation*}
\sphinxAtStartPar
para \(h=1.0,0.1,0.001\).

\sphinxAtStartPar
Compare a solução numérica com a analítica: \(y_{an}(t) = 2e^{t} - t - 1\).

\begin{sphinxVerbatim}[commandchars=\\\{\}]
\PYG{n}{f} \PYG{o}{=} \PYG{k}{lambda} \PYG{n}{t}\PYG{p}{,}\PYG{n}{y}\PYG{p}{:} \PYG{n}{y} \PYG{o}{+} \PYG{n}{t}
\PYG{n}{yan} \PYG{o}{=} \PYG{k}{lambda} \PYG{n}{t}\PYG{p}{:} \PYG{l+m+mi}{2}\PYG{o}{*}\PYG{n}{np}\PYG{o}{.}\PYG{n}{exp}\PYG{p}{(}\PYG{n}{t}\PYG{p}{)} \PYG{o}{\PYGZhy{}} \PYG{n}{t} \PYG{o}{\PYGZhy{}} \PYG{l+m+mi}{1}

\PYG{n}{y0} \PYG{o}{=} \PYG{l+m+mi}{1}
\PYG{n}{a} \PYG{o}{=} \PYG{l+m+mf}{0.0}
\PYG{n}{b} \PYG{o}{=} \PYG{l+m+mf}{3.0}
\PYG{n}{T} \PYG{o}{=} \PYG{p}{[}\PYG{l+m+mf}{1.0}\PYG{p}{,}\PYG{l+m+mf}{0.1}\PYG{p}{,}\PYG{l+m+mf}{0.001}\PYG{p}{]}

\PYG{k}{for} \PYG{n}{k} \PYG{o+ow}{in} \PYG{n}{T}\PYG{p}{:}
    
    \PYG{n}{t} \PYG{o}{=} \PYG{n}{np}\PYG{o}{.}\PYG{n}{arange}\PYG{p}{(}\PYG{n}{a}\PYG{p}{,} \PYG{n}{b}\PYG{o}{+}\PYG{n}{k}\PYG{p}{,} \PYG{n}{k}\PYG{p}{)}
    \PYG{n}{sol} \PYG{o}{=} \PYG{n}{solve\PYGZus{}ivp}\PYG{p}{(}\PYG{n}{f}\PYG{p}{,} \PYG{p}{(}\PYG{n}{a}\PYG{p}{,}\PYG{n}{b}\PYG{p}{)}\PYG{p}{,} \PYG{p}{[}\PYG{n}{y0}\PYG{p}{]}\PYG{p}{)}

    \PYG{n}{y2} \PYG{o}{=} \PYG{n}{t}\PYG{o}{*}\PYG{l+m+mi}{0}
    \PYG{n}{y2}\PYG{p}{[}\PYG{l+m+mi}{0}\PYG{p}{]} \PYG{o}{=} \PYG{n}{y0}
    \PYG{n}{dt} \PYG{o}{=} \PYG{p}{(}\PYG{n}{b}\PYG{o}{\PYGZhy{}}\PYG{n}{a}\PYG{p}{)}\PYG{o}{/} \PYG{p}{(}\PYG{n+nb}{len}\PYG{p}{(}\PYG{n}{t}\PYG{p}{)}\PYG{o}{\PYGZhy{}}\PYG{l+m+mi}{1}\PYG{p}{)}
    \PYG{k}{for} \PYG{n}{i} \PYG{o+ow}{in} \PYG{n+nb}{range}\PYG{p}{(}\PYG{l+m+mi}{0}\PYG{p}{,}\PYG{n+nb}{len}\PYG{p}{(}\PYG{n}{t}\PYG{p}{)}\PYG{o}{\PYGZhy{}}\PYG{l+m+mi}{1}\PYG{p}{)}\PYG{p}{:}
        \PYG{n}{y2}\PYG{p}{[}\PYG{n}{i}\PYG{o}{+}\PYG{l+m+mi}{1}\PYG{p}{]} \PYG{o}{=} \PYG{n}{y2}\PYG{p}{[}\PYG{n}{i}\PYG{p}{]} \PYG{o}{+} \PYG{n}{f}\PYG{p}{(}\PYG{n}{t}\PYG{p}{[}\PYG{n}{i}\PYG{p}{]}\PYG{p}{,}\PYG{n}{y2}\PYG{p}{[}\PYG{n}{i}\PYG{p}{]}\PYG{p}{)}\PYG{o}{*}\PYG{n}{dt}
            
    \PYG{n}{fig} \PYG{o}{=} \PYG{n}{plt}\PYG{o}{.}\PYG{n}{figure}\PYG{p}{(}\PYG{p}{)}
    \PYG{n}{ax} \PYG{o}{=} \PYG{n}{fig}\PYG{o}{.}\PYG{n}{add\PYGZus{}subplot}\PYG{p}{(}\PYG{l+m+mi}{111}\PYG{p}{)}
    \PYG{n}{plt}\PYG{o}{.}\PYG{n}{plot}\PYG{p}{(}\PYG{n}{sol}\PYG{o}{.}\PYG{n}{t}\PYG{p}{,}\PYG{n}{sol}\PYG{o}{.}\PYG{n}{y}\PYG{p}{[}\PYG{l+m+mi}{0}\PYG{p}{]}\PYG{p}{,}\PYG{l+s+s1}{\PYGZsq{}}\PYG{l+s+s1}{o\PYGZhy{}}\PYG{l+s+s1}{\PYGZsq{}}\PYG{p}{,}\PYG{n}{label}\PYG{o}{=}\PYG{l+s+s1}{\PYGZsq{}}\PYG{l+s+s1}{solve\PYGZus{}ivp}\PYG{l+s+s1}{\PYGZsq{}}\PYG{p}{)}
    \PYG{n}{plt}\PYG{o}{.}\PYG{n}{plot}\PYG{p}{(}\PYG{n}{t}\PYG{p}{,}\PYG{n}{y2}\PYG{p}{,}\PYG{l+s+s1}{\PYGZsq{}}\PYG{l+s+s1}{r}\PYG{l+s+s1}{\PYGZsq{}}\PYG{p}{,}\PYG{n}{label}\PYG{o}{=}\PYG{l+s+s1}{\PYGZsq{}}\PYG{l+s+s1}{Euler}\PYG{l+s+s1}{\PYGZsq{}}\PYG{p}{)}
    \PYG{n}{plt}\PYG{o}{.}\PYG{n}{plot}\PYG{p}{(}\PYG{n}{t}\PYG{p}{,}\PYG{n}{yan}\PYG{p}{(}\PYG{n}{t}\PYG{p}{)}\PYG{p}{,}\PYG{l+s+s1}{\PYGZsq{}}\PYG{l+s+s1}{b}\PYG{l+s+s1}{\PYGZsq{}}\PYG{p}{,}\PYG{n}{label}\PYG{o}{=}\PYG{l+s+s1}{\PYGZsq{}}\PYG{l+s+s1}{sol. analítica}\PYG{l+s+s1}{\PYGZsq{}}\PYG{p}{)}
    \PYG{n}{plt}\PYG{o}{.}\PYG{n}{legend}\PYG{p}{(}\PYG{p}{)}
    \PYG{n}{s} \PYG{o}{=} \PYG{l+s+s1}{\PYGZsq{}}\PYG{l+s+s1}{Resolução de EDO: sol. analítica x mét. Euler (dt = }\PYG{l+s+s1}{\PYGZsq{}} \PYG{o}{+} \PYG{n+nb}{str}\PYG{p}{(}\PYG{n}{k}\PYG{p}{)} \PYG{o}{+} \PYG{l+s+s1}{\PYGZsq{}}\PYG{l+s+s1}{)}\PYG{l+s+s1}{\PYGZsq{}} 
    \PYG{n}{plt}\PYG{o}{.}\PYG{n}{title}\PYG{p}{(}\PYG{n}{s}\PYG{p}{)}
    \PYG{n}{plt}\PYG{o}{.}\PYG{n}{xlabel}\PYG{p}{(}\PYG{l+s+s1}{\PYGZsq{}}\PYG{l+s+s1}{\PYGZdl{}t\PYGZdl{}}\PYG{l+s+s1}{\PYGZsq{}}\PYG{p}{)} 
    \PYG{n}{plt}\PYG{o}{.}\PYG{n}{ylabel}\PYG{p}{(}\PYG{l+s+s1}{\PYGZsq{}}\PYG{l+s+s1}{\PYGZdl{}y(t)\PYGZdl{}}\PYG{l+s+s1}{\PYGZsq{}}\PYG{p}{)}
\end{sphinxVerbatim}

\noindent\sphinxincludegraphics{{codeSession-9-solve_ivp_7_0}.png}

\noindent\sphinxincludegraphics{{codeSession-9-solve_ivp_7_1}.png}

\noindent\sphinxincludegraphics{{codeSession-9-solve_ivp_7_2}.png}

\begin{sphinxVerbatim}[commandchars=\\\{\}]
\PYG{l+s+sd}{\PYGZdq{}\PYGZdq{}\PYGZdq{}}
\PYG{l+s+sd}{\PYGZpc{}matplotlib inline }
\PYG{l+s+sd}{from scipy.integrate import odeint}
\PYG{l+s+sd}{import numpy as np }
\PYG{l+s+sd}{import matplotlib.pyplot as plt}
\PYG{l+s+sd}{from sympy import Symbol, integrate, sin, cos}

\PYG{l+s+sd}{\PYGZsh{} PVI }
\PYG{l+s+sd}{\PYGZsh{} y\PYGZsq{} = f(T,y)}
\PYG{l+s+sd}{\PYGZsh{} y(t0) = 1, T = (t0,tn]}

\PYG{l+s+sd}{\PYGZsh{} parametros}
\PYG{l+s+sd}{t0 = 0}
\PYG{l+s+sd}{tn = 10}
\PYG{l+s+sd}{nt = 35}
\PYG{l+s+sd}{y0\PYGZus{}ex2 = 1}

\PYG{l+s+sd}{\PYGZsh{} var. independente}
\PYG{l+s+sd}{T = np.linspace(t0,tn,nt)}

\PYG{l+s+sd}{\PYGZsh{} EDO}
\PYG{l+s+sd}{f\PYGZus{}ex2 = lambda t,y: y**2*t*(np.sin(t))}

\PYG{l+s+sd}{\PYGZsh{} solucao analitica simbolica }
\PYG{l+s+sd}{fun = \PYGZsq{}y**2*t*(sin(t))\PYGZsq{}}
\PYG{l+s+sd}{tt = Symbol(\PYGZsq{}t\PYGZsq{})}
\PYG{l+s+sd}{yan\PYGZus{}ex2 = integrate(fun,(tt,t0,tn))}
\PYG{l+s+sd}{print(yan\PYGZus{}ex2)}

\PYG{l+s+sd}{yan = yan\PYGZus{}ex2.subs(t,T)}

\PYG{l+s+sd}{y\PYGZus{}ex2 = solve\PYGZus{}ivp(f\PYGZus{}ex2, (t0,tn),[y0\PYGZus{}ex2])}

\PYG{l+s+sd}{fig = plt.figure()}
\PYG{l+s+sd}{ax = fig.add\PYGZus{}subplot(111)}
\PYG{l+s+sd}{plt.plot(y\PYGZus{}ex2.t,y\PYGZus{}ex2.y[0],\PYGZsq{}k\PYGZsq{},label=\PYGZsq{}solve\PYGZus{}ivp\PYGZsq{})}
\PYG{l+s+sd}{plt.legend()}
\PYG{l+s+sd}{s = \PYGZsq{}Resolução de EDO: sol. analítica x mét. Euler (dt = \PYGZsq{} + str((tn\PYGZhy{}t0)/nt) + \PYGZsq{})\PYGZsq{} }
\PYG{l+s+sd}{plt.title(s)}
\PYG{l+s+sd}{plt.xlabel(\PYGZsq{}\PYGZdl{}t\PYGZdl{}\PYGZsq{}); }
\PYG{l+s+sd}{plt.ylabel(\PYGZsq{}\PYGZdl{}y(t)\PYGZdl{}\PYGZsq{})}
\PYG{l+s+sd}{\PYGZdq{}\PYGZdq{}\PYGZdq{}}\PYG{p}{;}
\end{sphinxVerbatim}


\part{Exercícios resolvidos}


\chapter{Lista de Exercícios 1}
\label{\detokenize{lista-1-solucoes:lista-de-exercicios-1}}\label{\detokenize{lista-1-solucoes::doc}}
\sphinxAtStartPar
Solucionário matemático e computacional de exercícios selecionados da Lista de Exercícios 1.


\section{Exercícios computacionais}
\label{\detokenize{lista-1-solucoes:exercicios-computacionais}}

\subsection{Q}
\label{\detokenize{lista-1-solucoes:q}}
\sphinxAtStartPar
Considere a função quadrática \(f(x) = x^2 - 100.0001 + 0.01\):

\sphinxAtStartPar
a) Implemente a fórmula de Bhaskara para calcular suas raízes

\sphinxAtStartPar
b) Racionalize a fórmula de Bhaskara para a solução \(x_2\) (multiplique por \(\frac{-b + \sqrt{\Delta}}{-b + \sqrt{\Delta}}\)) e refaça o cálculo para \(x_2\).

\sphinxAtStartPar
c) Compare os resultados para \(x_2\).


\subsection{S}
\label{\detokenize{lista-1-solucoes:s}}
\begin{sphinxVerbatim}[commandchars=\\\{\}]
\PYG{k+kn}{from} \PYG{n+nn}{math} \PYG{k+kn}{import} \PYG{n}{sqrt}

\PYG{c+c1}{\PYGZsh{} a)}

\PYG{c+c1}{\PYGZsh{} fórmula de Bhaskara }
\PYG{n}{a}\PYG{p}{,} \PYG{n}{b}\PYG{p}{,} \PYG{n}{c} \PYG{o}{=} \PYG{l+m+mi}{1}\PYG{p}{,} \PYG{o}{\PYGZhy{}}\PYG{l+m+mf}{100.0001}\PYG{p}{,} \PYG{l+m+mf}{0.01}
\PYG{n}{Delta} \PYG{o}{=} \PYG{n}{b}\PYG{o}{*}\PYG{o}{*}\PYG{l+m+mi}{2} \PYG{o}{\PYGZhy{}} \PYG{l+m+mi}{4}\PYG{o}{*}\PYG{n}{a}\PYG{o}{*}\PYG{n}{c}

\PYG{c+c1}{\PYGZsh{} raízes}
\PYG{n}{x1} \PYG{o}{=} \PYG{p}{(}\PYG{o}{\PYGZhy{}} \PYG{n}{b} \PYG{o}{+} \PYG{n}{sqrt}\PYG{p}{(}\PYG{n}{Delta}\PYG{p}{)}\PYG{p}{)}\PYG{o}{/}\PYG{l+m+mi}{2}\PYG{o}{*}\PYG{n}{a}
\PYG{n}{x2} \PYG{o}{=} \PYG{p}{(}\PYG{o}{\PYGZhy{}} \PYG{n}{b} \PYG{o}{\PYGZhy{}} \PYG{n}{sqrt}\PYG{p}{(}\PYG{n}{Delta}\PYG{p}{)}\PYG{p}{)}\PYG{o}{/}\PYG{l+m+mi}{2}\PYG{o}{*}\PYG{n}{a}

\PYG{n+nb}{print}\PYG{p}{(}\PYG{l+s+s1}{\PYGZsq{}}\PYG{l+s+s1}{x1 = }\PYG{l+s+si}{\PYGZob{}0\PYGZcb{}}\PYG{l+s+s1}{\PYGZsq{}}\PYG{o}{.}\PYG{n}{format}\PYG{p}{(}\PYG{n}{x1}\PYG{p}{)}\PYG{p}{)}
\PYG{n+nb}{print}\PYG{p}{(}\PYG{l+s+s1}{\PYGZsq{}}\PYG{l+s+s1}{x2 = }\PYG{l+s+si}{\PYGZob{}0\PYGZcb{}}\PYG{l+s+s1}{\PYGZsq{}}\PYG{o}{.}\PYG{n}{format}\PYG{p}{(}\PYG{n}{x2}\PYG{p}{)}\PYG{p}{)}

\PYG{c+c1}{\PYGZsh{} b)}

\PYG{c+c1}{\PYGZsh{} Racionalizando, obtemos }

\PYG{n}{x2b} \PYG{o}{=} \PYG{l+m+mi}{2}\PYG{o}{*}\PYG{n}{c}\PYG{o}{/}\PYG{p}{(}\PYG{o}{\PYGZhy{}}\PYG{n}{b} \PYG{o}{+} \PYG{n}{sqrt}\PYG{p}{(}\PYG{n}{Delta}\PYG{p}{)}\PYG{p}{)}
\PYG{n+nb}{print}\PYG{p}{(}\PYG{l+s+s1}{\PYGZsq{}}\PYG{l+s+s1}{x2b = }\PYG{l+s+si}{\PYGZob{}0\PYGZcb{}}\PYG{l+s+s1}{\PYGZsq{}}\PYG{o}{.}\PYG{n}{format}\PYG{p}{(}\PYG{n}{x2b}\PYG{p}{)}\PYG{p}{)}


\PYG{c+c1}{\PYGZsh{} c) }

\PYG{l+s+sd}{\PYGZdq{}\PYGZdq{}\PYGZdq{}}
\PYG{l+s+sd}{Discussão: o valor calculado em b) possui um erro de arredondamento. }
\PYG{l+s+sd}{           Como b \PYGZlt{} 0, o numerador envolve a subtração de}
\PYG{l+s+sd}{           dois números quase iguais (erro de cancelamento)}
\PYG{l+s+sd}{           e isto afeta o resultado. }
\PYG{l+s+sd}{           }
\PYG{l+s+sd}{           Com a racionalização da expressão, obtemos uma fórmula}
\PYG{l+s+sd}{           menos propensa a erros. }
\PYG{l+s+sd}{\PYGZdq{}\PYGZdq{}\PYGZdq{}}
\PYG{n}{b2} \PYG{o}{=} \PYG{n}{sqrt}\PYG{p}{(}\PYG{n}{Delta}\PYG{p}{)}
\PYG{n+nb}{print}\PYG{p}{(}\PYG{n}{b2}\PYG{p}{)}
\PYG{n+nb}{print}\PYG{p}{(}\PYG{l+s+s1}{\PYGZsq{}}\PYG{l+s+s1}{Diferença no numerador: }\PYG{l+s+s1}{\PYGZsq{}} \PYG{o}{+} \PYG{n+nb}{str}\PYG{p}{(}\PYG{o}{\PYGZhy{}}\PYG{n}{b} \PYG{o}{\PYGZhy{}} \PYG{n}{b2}\PYG{p}{)}\PYG{p}{)}
\PYG{n+nb}{print}\PYG{p}{(}\PYG{l+s+s1}{\PYGZsq{}}\PYG{l+s+s1}{Erro absoluto na solução: }\PYG{l+s+s1}{\PYGZsq{}} \PYG{o}{+} \PYG{n+nb}{str}\PYG{p}{(}\PYG{n+nb}{abs}\PYG{p}{(}\PYG{n}{x2} \PYG{o}{\PYGZhy{}} \PYG{n}{x2b}\PYG{p}{)}\PYG{p}{)}\PYG{p}{)}
\end{sphinxVerbatim}

\begin{sphinxVerbatim}[commandchars=\\\{\}]
x1 = 100.0
x2 = 0.00010000000000331966
x2b = 0.0001
99.9999
Diferença no numerador: 0.0002000000000066393
Erro absoluto na solução: 3.3196508692975857e\PYGZhy{}15
\end{sphinxVerbatim}


\chapter{Lista de Exercícios 2}
\label{\detokenize{lista-2-solucoes:lista-de-exercicios-2}}\label{\detokenize{lista-2-solucoes::doc}}
\sphinxAtStartPar
Solucionário matemático e computacional de exercícios selecionados da Lista de Exercícios 2.

\begin{sphinxVerbatim}[commandchars=\\\{\}]
\PYG{o}{\PYGZpc{}}\PYG{k}{matplotlib} inline
\end{sphinxVerbatim}

\begin{sphinxVerbatim}[commandchars=\\\{\}]
\PYG{k+kn}{import} \PYG{n+nn}{metodosRaizes} \PYG{k}{as} \PYG{n+nn}{rz}
\PYG{k+kn}{import} \PYG{n+nn}{numpy} \PYG{k}{as} \PYG{n+nn}{np}
\PYG{k+kn}{import} \PYG{n+nn}{sympy} \PYG{k}{as} \PYG{n+nn}{sy}
\PYG{k+kn}{import} \PYG{n+nn}{matplotlib}\PYG{n+nn}{.}\PYG{n+nn}{pyplot} \PYG{k}{as} \PYG{n+nn}{plt}
\end{sphinxVerbatim}


\section{Q12 \sphinxhyphen{} L2}
\label{\detokenize{lista-2-solucoes:q12-l2}}
\sphinxAtStartPar
Primeiramente, recorramos à análise gráfica da função.

\begin{sphinxVerbatim}[commandchars=\\\{\}]
\PYG{k+kn}{from} \PYG{n+nn}{numpy} \PYG{k+kn}{import} \PYG{n}{cos}\PYG{p}{,} \PYG{n}{exp}

\PYG{c+c1}{\PYGZsh{} funcao }
\PYG{n}{p} \PYG{o}{=} \PYG{k}{lambda} \PYG{n}{x}\PYG{p}{:} \PYG{l+m+mi}{2}\PYG{o}{*}\PYG{n}{cos}\PYG{p}{(}\PYG{n}{x}\PYG{p}{)} \PYG{o}{\PYGZhy{}} \PYG{n}{exp}\PYG{p}{(}\PYG{n}{x}\PYG{p}{)}

\PYG{c+c1}{\PYGZsh{} plotagem}
\PYG{n}{x} \PYG{o}{=} \PYG{n}{np}\PYG{o}{.}\PYG{n}{linspace}\PYG{p}{(}\PYG{o}{\PYGZhy{}}\PYG{l+m+mi}{1}\PYG{p}{,}\PYG{l+m+mi}{1}\PYG{p}{,}\PYG{l+m+mi}{30}\PYG{p}{)}
\PYG{n}{plt}\PYG{o}{.}\PYG{n}{plot}\PYG{p}{(}\PYG{n}{x}\PYG{p}{,}\PYG{n}{p}\PYG{p}{(}\PYG{n}{x}\PYG{p}{)}\PYG{p}{,}\PYG{n}{x}\PYG{p}{,}\PYG{l+m+mi}{0}\PYG{o}{*}\PYG{n}{p}\PYG{p}{(}\PYG{n}{x}\PYG{p}{)}\PYG{p}{)}\PYG{p}{;}
\end{sphinxVerbatim}

\noindent\sphinxincludegraphics{{lista-2-solucoes_5_0}.png}

\sphinxAtStartPar
Vemos que a raiz está próxima de \(x= 0.5\). Assim, qualquer valor próximo a este é razoável como aproximação inicial.

\sphinxAtStartPar
Agora, antes de aplicar o método de Newton, vamos calcular a derivada da função simbolicamente.

\begin{sphinxVerbatim}[commandchars=\\\{\}]
\PYG{c+c1}{\PYGZsh{} derivada simbolica }
\PYG{n}{xsym} \PYG{o}{=} \PYG{n}{sy}\PYG{o}{.}\PYG{n}{Symbol}\PYG{p}{(}\PYG{l+s+s1}{\PYGZsq{}}\PYG{l+s+s1}{x}\PYG{l+s+s1}{\PYGZsq{}}\PYG{p}{)}
\PYG{n}{sy}\PYG{o}{.}\PYG{n}{diff}\PYG{p}{(}\PYG{l+m+mi}{2}\PYG{o}{*}\PYG{n}{sy}\PYG{o}{.}\PYG{n}{cos}\PYG{p}{(}\PYG{n}{xsym}\PYG{p}{)} \PYG{o}{\PYGZhy{}} \PYG{n}{sy}\PYG{o}{.}\PYG{n}{exp}\PYG{p}{(}\PYG{n}{xsym}\PYG{p}{)}\PYG{p}{,}\PYG{n}{xsym}\PYG{p}{)}
\end{sphinxVerbatim}
\begin{equation*}
\begin{split}\displaystyle - e^{x} - 2 \sin{\left(x \right)}\end{split}
\end{equation*}
\sphinxAtStartPar
Resolvamos com a nossa função programada.

\begin{sphinxVerbatim}[commandchars=\\\{\}]
\PYG{n}{x0} \PYG{o}{=} \PYG{l+m+mf}{0.2}
\PYG{n}{tol} \PYG{o}{=} \PYG{l+m+mf}{1e\PYGZhy{}3}
\PYG{n}{f} \PYG{o}{=} \PYG{l+s+s1}{\PYGZsq{}}\PYG{l+s+s1}{2*cos(x) \PYGZhy{} exp(x)}\PYG{l+s+s1}{\PYGZsq{}}
\PYG{n}{df} \PYG{o}{=} \PYG{l+s+s1}{\PYGZsq{}}\PYG{l+s+s1}{\PYGZhy{}2*sin(x) \PYGZhy{} exp(x)}\PYG{l+s+s1}{\PYGZsq{}}
\PYG{n}{xr} \PYG{o}{=} \PYG{n}{rz}\PYG{o}{.}\PYG{n}{newton}\PYG{p}{(}\PYG{n}{x0}\PYG{p}{,}\PYG{n}{f}\PYG{p}{,}\PYG{n}{df}\PYG{p}{,}\PYG{n}{tol}\PYG{p}{,}\PYG{l+m+mi}{10}\PYG{p}{,}\PYG{k+kc}{True}\PYG{p}{)}
\end{sphinxVerbatim}

\begin{sphinxVerbatim}[commandchars=\\\{\}]
Estimativa inicial: x0 = 0.2

i	 x		 f(x)		 df(x)		 ER
0	 0.200000	 0.738730	 \PYGZhy{}1.618741	 \PYGZhy{}
1	 0.656361	 \PYGZhy{}0.343328	 \PYGZhy{}3.148240	 6.952896e\PYGZhy{}01
2	 0.547307	 \PYGZhy{}0.020734	 \PYGZhy{}2.769371	 1.992555e\PYGZhy{}01
3	 0.539820	 \PYGZhy{}0.000096	 \PYGZhy{}2.743662	 1.386900e\PYGZhy{}02
4	 0.539785	 \PYGZhy{}0.000000	 \PYGZhy{}2.743542	 6.499799e\PYGZhy{}05
Solução obtida: x = 0.5397851616
\end{sphinxVerbatim}

\noindent\sphinxincludegraphics{{lista-2-solucoes_9_1}.png}

\sphinxAtStartPar
Verificação:

\begin{sphinxVerbatim}[commandchars=\\\{\}]
\PYG{l+m+mi}{2}\PYG{o}{*}\PYG{n}{cos}\PYG{p}{(}\PYG{n}{xr}\PYG{p}{)} \PYG{o}{\PYGZhy{}} \PYG{n}{exp}\PYG{p}{(}\PYG{n}{xr}\PYG{p}{)}
\end{sphinxVerbatim}

\begin{sphinxVerbatim}[commandchars=\\\{\}]
\PYGZhy{}2.1118811144305027e\PYGZhy{}09
\end{sphinxVerbatim}


\section{Q16 \sphinxhyphen{} L2}
\label{\detokenize{lista-2-solucoes:q16-l2}}
\begin{sphinxVerbatim}[commandchars=\\\{\}]
\PYG{n}{p} \PYG{o}{=} \PYG{k}{lambda} \PYG{n}{x}\PYG{p}{:} \PYG{n}{x}\PYG{o}{*}\PYG{o}{*}\PYG{l+m+mi}{5} \PYG{o}{\PYGZhy{}} \PYG{l+m+mi}{10}\PYG{o}{/}\PYG{l+m+mi}{9}\PYG{o}{*}\PYG{n}{x}\PYG{o}{*}\PYG{o}{*}\PYG{l+m+mi}{3} \PYG{o}{+} \PYG{l+m+mi}{5}\PYG{o}{/}\PYG{l+m+mi}{21}\PYG{o}{*}\PYG{n}{x}
\PYG{n}{x} \PYG{o}{=} \PYG{n}{np}\PYG{o}{.}\PYG{n}{linspace}\PYG{p}{(}\PYG{o}{\PYGZhy{}}\PYG{l+m+mi}{1}\PYG{p}{,}\PYG{l+m+mi}{1}\PYG{p}{,}\PYG{l+m+mi}{50}\PYG{p}{,}\PYG{n}{endpoint}\PYG{o}{=}\PYG{k+kc}{True}\PYG{p}{)}
\PYG{n}{plt}\PYG{o}{.}\PYG{n}{plot}\PYG{p}{(}\PYG{n}{x}\PYG{p}{,}\PYG{n}{p}\PYG{p}{(}\PYG{n}{x}\PYG{p}{)}\PYG{p}{,}\PYG{n}{x}\PYG{p}{,}\PYG{l+m+mi}{0}\PYG{o}{*}\PYG{n}{p}\PYG{p}{(}\PYG{n}{x}\PYG{p}{)}\PYG{p}{)}

\PYG{n}{xr} \PYG{o}{=} \PYG{n}{np}\PYG{o}{.}\PYG{n}{roots}\PYG{p}{(}\PYG{p}{[}\PYG{l+m+mi}{1}\PYG{p}{,}\PYG{l+m+mi}{0}\PYG{p}{,}\PYG{o}{\PYGZhy{}}\PYG{l+m+mi}{10}\PYG{o}{/}\PYG{l+m+mi}{9}\PYG{p}{,}\PYG{l+m+mi}{0}\PYG{p}{,}\PYG{l+m+mi}{5}\PYG{o}{/}\PYG{l+m+mi}{21}\PYG{p}{,}\PYG{l+m+mi}{0}\PYG{p}{]}\PYG{p}{)}
\PYG{n+nb}{print}\PYG{p}{(}\PYG{n}{xr}\PYG{p}{)}

\PYG{n}{plt}\PYG{o}{.}\PYG{n}{scatter}\PYG{p}{(}\PYG{n}{xr}\PYG{p}{,}\PYG{l+m+mi}{0}\PYG{o}{*}\PYG{n}{xr}\PYG{p}{,}\PYG{n}{c}\PYG{o}{=}\PYG{l+s+s1}{\PYGZsq{}}\PYG{l+s+s1}{r}\PYG{l+s+s1}{\PYGZsq{}}\PYG{p}{)}\PYG{p}{;}
\PYG{n}{plt}\PYG{o}{.}\PYG{n}{grid}\PYG{p}{(}\PYG{k+kc}{True}\PYG{p}{)}\PYG{p}{;}
\end{sphinxVerbatim}

\begin{sphinxVerbatim}[commandchars=\\\{\}]
[\PYGZhy{}0.90617985 \PYGZhy{}0.53846931  0.90617985  0.53846931  0.        ]
\end{sphinxVerbatim}

\noindent\sphinxincludegraphics{{lista-2-solucoes_13_1}.png}

\begin{sphinxVerbatim}[commandchars=\\\{\}]
\PYG{n}{tol} \PYG{o}{=} \PYG{l+m+mf}{1e\PYGZhy{}3}
\PYG{n}{nmax} \PYG{o}{=} \PYG{l+m+mi}{100}
\end{sphinxVerbatim}


\subsection{item b.1)}
\label{\detokenize{lista-2-solucoes:item-b-1}}
\begin{sphinxVerbatim}[commandchars=\\\{\}]
\PYG{c+c1}{\PYGZsh{} Aproximacao de xi1 pelo MNR}
\PYG{n}{x0} \PYG{o}{=} \PYG{o}{\PYGZhy{}}\PYG{l+m+mf}{0.8}
\PYG{n}{p} \PYG{o}{=} \PYG{l+s+s1}{\PYGZsq{}}\PYG{l+s+s1}{x**5 \PYGZhy{} 10/9*x**3 + 5/21*x}\PYG{l+s+s1}{\PYGZsq{}}
\PYG{n}{dp} \PYG{o}{=} \PYG{l+s+s1}{\PYGZsq{}}\PYG{l+s+s1}{5*x**4 \PYGZhy{}10/3*x**2 + 5/21}\PYG{l+s+s1}{\PYGZsq{}}
\PYG{n}{plotar} \PYG{o}{=} \PYG{k+kc}{True}
\PYG{n}{rz}\PYG{o}{.}\PYG{n}{newton}\PYG{p}{(}\PYG{n}{x0}\PYG{p}{,}\PYG{n}{p}\PYG{p}{,}\PYG{n}{dp}\PYG{p}{,}\PYG{n}{tol}\PYG{p}{,}\PYG{n}{nmax}\PYG{p}{,}\PYG{k+kc}{True}\PYG{p}{)}
\end{sphinxVerbatim}

\begin{sphinxVerbatim}[commandchars=\\\{\}]
Estimativa inicial: x0 = \PYGZhy{}0.8

i	 x		 f(x)		 df(x)		 ER
0	 \PYGZhy{}0.800000	 0.050733	 0.152762	 \PYGZhy{}
1	 \PYGZhy{}1.132103	 \PYGZhy{}0.517006	 4.179132	 2.933506e\PYGZhy{}01
2	 \PYGZhy{}1.008392	 \PYGZhy{}0.143443	 2.018543	 1.226817e\PYGZhy{}01
3	 \PYGZhy{}0.937329	 \PYGZhy{}0.031683	 1.169040	 7.581417e\PYGZhy{}02
4	 \PYGZhy{}0.910227	 \PYGZhy{}0.003604	 0.908555	 2.977501e\PYGZhy{}02
5	 \PYGZhy{}0.906261	 \PYGZhy{}0.000070	 0.873137	 4.376698e\PYGZhy{}03
6	 \PYGZhy{}0.906180	 \PYGZhy{}0.000000	 0.872424	 8.905774e\PYGZhy{}05
Solução obtida: x = \PYGZhy{}0.9061798789
\end{sphinxVerbatim}

\begin{sphinxVerbatim}[commandchars=\\\{\}]
\PYGZhy{}0.9061798789481786
\end{sphinxVerbatim}

\noindent\sphinxincludegraphics{{lista-2-solucoes_16_2}.png}

\begin{sphinxVerbatim}[commandchars=\\\{\}]
\PYG{n}{ff} \PYG{o}{=} \PYG{k}{lambda} \PYG{n}{x}\PYG{p}{:} \PYG{n}{x}\PYG{o}{*}\PYG{o}{*}\PYG{l+m+mi}{5} \PYG{o}{\PYGZhy{}} \PYG{l+m+mi}{10}\PYG{o}{/}\PYG{l+m+mi}{9}\PYG{o}{*}\PYG{n}{x}\PYG{o}{*}\PYG{o}{*}\PYG{l+m+mi}{3} \PYG{o}{+} \PYG{l+m+mi}{5}\PYG{o}{/}\PYG{l+m+mi}{21}\PYG{o}{*}\PYG{n}{x}
\PYG{n}{dff} \PYG{o}{=} \PYG{k}{lambda} \PYG{n}{x}\PYG{p}{:} \PYG{l+m+mi}{5}\PYG{o}{*}\PYG{n}{x}\PYG{o}{*}\PYG{o}{*}\PYG{l+m+mi}{4} \PYG{o}{\PYGZhy{}} \PYG{l+m+mi}{30}\PYG{o}{/}\PYG{l+m+mi}{9}\PYG{o}{*}\PYG{n}{x}\PYG{o}{*}\PYG{o}{*}\PYG{l+m+mi}{2} \PYG{o}{+} \PYG{l+m+mi}{5}\PYG{o}{/}\PYG{l+m+mi}{21}

\PYG{n}{xx} \PYG{o}{=} \PYG{l+m+mf}{0.857}
\PYG{n}{ff}\PYG{p}{(}\PYG{n}{xx}\PYG{p}{)}\PYG{p}{,}\PYG{n}{dff}\PYG{p}{(}\PYG{n}{xx}\PYG{p}{)}
\end{sphinxVerbatim}

\begin{sphinxVerbatim}[commandchars=\\\{\}]
(\PYGZhy{}0.03303209894521281, 0.4870085727669044)
\end{sphinxVerbatim}


\subsection{item b.2)}
\label{\detokenize{lista-2-solucoes:item-b-2}}
\begin{sphinxVerbatim}[commandchars=\\\{\}]
\PYG{c+c1}{\PYGZsh{} Aproximacao de xi2 pelo MB}

\PYG{n}{var} \PYG{o}{=} \PYG{l+s+s1}{\PYGZsq{}}\PYG{l+s+s1}{x}\PYG{l+s+s1}{\PYGZsq{}}
\PYG{n}{a}\PYG{p}{,}\PYG{n}{b} \PYG{o}{=} \PYG{o}{\PYGZhy{}}\PYG{l+m+mf}{0.75}\PYG{p}{,} \PYG{o}{\PYGZhy{}}\PYG{l+m+mf}{0.25} 
\PYG{n}{rz}\PYG{o}{.}\PYG{n}{bissecao}\PYG{p}{(}\PYG{n}{p}\PYG{p}{,}\PYG{n}{a}\PYG{p}{,}\PYG{n}{b}\PYG{p}{,}\PYG{n}{tol}\PYG{p}{,}\PYG{l+m+mi}{100}\PYG{p}{,}\PYG{n}{var}\PYG{p}{)}
\end{sphinxVerbatim}

\begin{sphinxVerbatim}[commandchars=\\\{\}]
i=1 a=\PYGZhy{}0.750000 b=\PYGZhy{}0.250000 xm=\PYGZhy{}0.500000 f(a)=0.052874 f(b)=\PYGZhy{}0.043139 f(xm)=\PYGZhy{}0.011409 b\PYGZhy{}a=0.500000
i=2 a=\PYGZhy{}0.750000 b=\PYGZhy{}0.500000 xm=\PYGZhy{}0.625000 f(a)=0.052874 f(b)=\PYGZhy{}0.011409 f(xm)=0.027090 b\PYGZhy{}a=0.250000
i=3 a=\PYGZhy{}0.625000 b=\PYGZhy{}0.500000 xm=\PYGZhy{}0.562500 f(a)=0.027090 f(b)=\PYGZhy{}0.011409 f(xm)=0.007512 b\PYGZhy{}a=0.125000
i=4 a=\PYGZhy{}0.562500 b=\PYGZhy{}0.500000 xm=\PYGZhy{}0.531250 f(a)=0.007512 f(b)=\PYGZhy{}0.011409 f(xm)=\PYGZhy{}0.002211 b\PYGZhy{}a=0.062500
i=5 a=\PYGZhy{}0.562500 b=\PYGZhy{}0.531250 xm=\PYGZhy{}0.546875 f(a)=0.007512 f(b)=\PYGZhy{}0.002211 f(xm)=0.002605 b\PYGZhy{}a=0.031250
i=6 a=\PYGZhy{}0.546875 b=\PYGZhy{}0.531250 xm=\PYGZhy{}0.539062 f(a)=0.002605 f(b)=\PYGZhy{}0.002211 f(xm)=0.000183 b\PYGZhy{}a=0.015625
Solução encontrada: \PYGZhy{}0.5390625
\end{sphinxVerbatim}

\begin{sphinxVerbatim}[commandchars=\\\{\}]
\PYGZhy{}0.5390625
\end{sphinxVerbatim}


\subsection{item b.3)}
\label{\detokenize{lista-2-solucoes:item-b-3}}
\begin{sphinxVerbatim}[commandchars=\\\{\}]
\PYG{c+c1}{\PYGZsh{} Aproximacao de xi3 pelo MFP}
\PYG{n}{a}\PYG{p}{,}\PYG{n}{b} \PYG{o}{=} \PYG{o}{\PYGZhy{}}\PYG{l+m+mf}{0.35}\PYG{p}{,} \PYG{l+m+mf}{0.25} 
\PYG{n}{rz}\PYG{o}{.}\PYG{n}{falsa\PYGZus{}posicao}\PYG{p}{(}\PYG{n}{p}\PYG{p}{,}\PYG{n}{a}\PYG{p}{,}\PYG{n}{b}\PYG{p}{,}\PYG{n}{tol}\PYG{p}{,}\PYG{l+m+mi}{100}\PYG{p}{,}\PYG{n}{var}\PYG{p}{)}
\end{sphinxVerbatim}

\begin{sphinxVerbatim}[commandchars=\\\{\}]
i=1 a=\PYGZhy{}0.350000 b=0.250000 xm=\PYGZhy{}0.057823 f(a)=\PYGZhy{}0.040947 f(b)=0.043139 f(xm)=\PYGZhy{}0.013553 b\PYGZhy{}a=0.600000
i=2 a=\PYGZhy{}0.057823 b=0.250000 xm=0.015767 f(a)=\PYGZhy{}0.013553 f(b)=0.043139 f(xm)=0.003750 b\PYGZhy{}a=0.307823
i=3 a=\PYGZhy{}0.057823 b=0.015767 xm=\PYGZhy{}0.000181 f(a)=\PYGZhy{}0.013553 f(b)=0.003750 f(xm)=\PYGZhy{}0.000043 b\PYGZhy{}a=0.073590
Solução encontrada: \PYGZhy{}0.00018060189341005293
\end{sphinxVerbatim}

\begin{sphinxVerbatim}[commandchars=\\\{\}]
\PYGZhy{}0.00018060189341005293
\end{sphinxVerbatim}


\subsection{item b.4)}
\label{\detokenize{lista-2-solucoes:item-b-4}}
\begin{sphinxVerbatim}[commandchars=\\\{\}]
\PYG{c+c1}{\PYGZsh{} Aproximacao de xi4 pelo MPF}

\PYG{n}{g} \PYG{o}{=} \PYG{l+s+s1}{\PYGZsq{}}\PYG{l+s+s1}{( 0.9*(5/21 + x**5))**1./3.}\PYG{l+s+s1}{\PYGZsq{}} 
\PYG{c+c1}{\PYGZsh{}g = eval(\PYGZsq{}lambda x:\PYGZsq{} + g)}

\PYG{c+c1}{\PYGZsh{}plt.plot(x,g(x))}
\PYG{n}{rz}\PYG{o}{.}\PYG{n}{ponto\PYGZus{}fixo}\PYG{p}{(}\PYG{l+m+mf}{0.4}\PYG{p}{,}\PYG{n}{p}\PYG{p}{,}\PYG{n}{g}\PYG{p}{,}\PYG{n}{tol}\PYG{p}{,}\PYG{n}{nmax}\PYG{p}{,}\PYG{k+kc}{True}\PYG{p}{)}
\end{sphinxVerbatim}

\begin{sphinxVerbatim}[commandchars=\\\{\}]
i	 x		 f(x)		 ER
0	 0.400000	 0.034367	 2.500000e+00
1	 0.074501	 0.017281	 4.369086e+00
2	 0.071429	 0.016604	 4.299795e\PYGZhy{}02
3	 0.071429	 0.016604	 1.829703e\PYGZhy{}06
\end{sphinxVerbatim}

\begin{sphinxVerbatim}[commandchars=\\\{\}]
0.07142912925875246
\end{sphinxVerbatim}

\noindent\sphinxincludegraphics{{lista-2-solucoes_23_2}.png}


\subsection{item b.5)}
\label{\detokenize{lista-2-solucoes:item-b-5}}
\begin{sphinxVerbatim}[commandchars=\\\{\}]
\PYG{c+c1}{\PYGZsh{} Aproximacao de xi5 pelo MS}

\PYG{n}{p} \PYG{o}{=} \PYG{l+s+s1}{\PYGZsq{}}\PYG{l+s+s1}{x**5 \PYGZhy{} 10/9*x**3 + 5/21*x}\PYG{l+s+s1}{\PYGZsq{}}
\PYG{n}{x0}\PYG{p}{,}\PYG{n}{x1}\PYG{o}{=}\PYG{l+m+mf}{0.8}\PYG{p}{,}\PYG{l+m+mf}{1.}
\PYG{n}{rz}\PYG{o}{.}\PYG{n}{secante}\PYG{p}{(}\PYG{n}{x0}\PYG{p}{,}\PYG{n}{x1}\PYG{p}{,}\PYG{n}{p}\PYG{p}{,}\PYG{n}{tol}\PYG{p}{,}\PYG{n}{nmax}\PYG{p}{,}\PYG{k+kc}{True}\PYG{p}{)}
\end{sphinxVerbatim}

\begin{sphinxVerbatim}[commandchars=\\\{\}]
Estimativas iniciais: xa = 0.8; xb = 1.0 

i	 x		 f(x)		 ER
1	 0.857094	 \PYGZhy{}0.032986	 1.667334e\PYGZhy{}01
2	 0.886562	 \PYGZhy{}0.015467	 3.323813e\PYGZhy{}02
3	 0.912577	 0.005764	 2.850796e\PYGZhy{}02
4	 0.905514	 \PYGZhy{}0.000579	 7.799970e\PYGZhy{}03
5	 0.906159	 \PYGZhy{}0.000018	 7.112008e\PYGZhy{}04
Solução obtida: x = 0.9061586909
\end{sphinxVerbatim}

\begin{sphinxVerbatim}[commandchars=\\\{\}]
0.9061586908755153
\end{sphinxVerbatim}

\noindent\sphinxincludegraphics{{lista-2-solucoes_25_2}.png}


\section{Exemplos}
\label{\detokenize{lista-2-solucoes:exemplos}}
\begin{sphinxVerbatim}[commandchars=\\\{\}]
\PYG{c+c1}{\PYGZsh{} exemplo bissecao}
\PYG{n}{rz}\PYG{o}{.}\PYG{n}{bissecao}\PYG{p}{(}\PYG{l+s+s1}{\PYGZsq{}}\PYG{l+s+s1}{x**3\PYGZhy{}9*x+3}\PYG{l+s+s1}{\PYGZsq{}}\PYG{p}{,}\PYG{l+m+mi}{0}\PYG{p}{,}\PYG{l+m+mi}{1}\PYG{p}{,}\PYG{l+m+mf}{1e\PYGZhy{}3}\PYG{p}{,}\PYG{l+m+mi}{100}\PYG{p}{,}\PYG{n}{var}\PYG{p}{)}
\end{sphinxVerbatim}

\begin{sphinxVerbatim}[commandchars=\\\{\}]
i=1 a=0.000000 b=1.000000 xm=0.500000 f(a)=3.000000 f(b)=\PYGZhy{}5.000000 f(xm)=\PYGZhy{}1.375000 b\PYGZhy{}a=1.000000
i=2 a=0.000000 b=0.500000 xm=0.250000 f(a)=3.000000 f(b)=\PYGZhy{}1.375000 f(xm)=0.765625 b\PYGZhy{}a=0.500000
i=3 a=0.250000 b=0.500000 xm=0.375000 f(a)=0.765625 f(b)=\PYGZhy{}1.375000 f(xm)=\PYGZhy{}0.322266 b\PYGZhy{}a=0.250000
i=4 a=0.250000 b=0.375000 xm=0.312500 f(a)=0.765625 f(b)=\PYGZhy{}0.322266 f(xm)=0.218018 b\PYGZhy{}a=0.125000
i=5 a=0.312500 b=0.375000 xm=0.343750 f(a)=0.218018 f(b)=\PYGZhy{}0.322266 f(xm)=\PYGZhy{}0.053131 b\PYGZhy{}a=0.062500
i=6 a=0.312500 b=0.343750 xm=0.328125 f(a)=0.218018 f(b)=\PYGZhy{}0.053131 f(xm)=0.082203 b\PYGZhy{}a=0.031250
i=7 a=0.328125 b=0.343750 xm=0.335938 f(a)=0.082203 f(b)=\PYGZhy{}0.053131 f(xm)=0.014474 b\PYGZhy{}a=0.015625
i=8 a=0.335938 b=0.343750 xm=0.339844 f(a)=0.014474 f(b)=\PYGZhy{}0.053131 f(xm)=\PYGZhy{}0.019344 b\PYGZhy{}a=0.007812
i=9 a=0.335938 b=0.339844 xm=0.337891 f(a)=0.014474 f(b)=\PYGZhy{}0.019344 f(xm)=\PYGZhy{}0.002439 b\PYGZhy{}a=0.003906
i=10 a=0.335938 b=0.337891 xm=0.336914 f(a)=0.014474 f(b)=\PYGZhy{}0.002439 f(xm)=0.006017 b\PYGZhy{}a=0.001953
i=11 a=0.336914 b=0.337891 xm=0.337402 f(a)=0.006017 f(b)=\PYGZhy{}0.002439 f(xm)=0.001789 b\PYGZhy{}a=0.000977
Solução encontrada: 0.33740234375
\end{sphinxVerbatim}

\begin{sphinxVerbatim}[commandchars=\\\{\}]
0.33740234375
\end{sphinxVerbatim}

\begin{sphinxVerbatim}[commandchars=\\\{\}]
\PYG{n}{rz}\PYG{o}{.}\PYG{n}{falsa\PYGZus{}posicao}\PYG{p}{(}\PYG{l+s+s1}{\PYGZsq{}}\PYG{l+s+s1}{x**3\PYGZhy{}9*x+3}\PYG{l+s+s1}{\PYGZsq{}}\PYG{p}{,}\PYG{l+m+mi}{0}\PYG{p}{,}\PYG{l+m+mi}{1}\PYG{p}{,}\PYG{l+m+mf}{1e\PYGZhy{}3}\PYG{p}{,}\PYG{l+m+mi}{100}\PYG{p}{,}\PYG{n}{var}\PYG{p}{)}
\end{sphinxVerbatim}

\begin{sphinxVerbatim}[commandchars=\\\{\}]
i=1 a=0.000000 b=1.000000 xm=0.375000 f(a)=3.000000 f(b)=\PYGZhy{}5.000000 f(xm)=\PYGZhy{}0.322266 b\PYGZhy{}a=1.000000
i=2 a=0.000000 b=0.375000 xm=0.338624 f(a)=3.000000 f(b)=\PYGZhy{}0.322266 f(xm)=\PYGZhy{}0.008790 b\PYGZhy{}a=0.375000
i=3 a=0.000000 b=0.338624 xm=0.337635 f(a)=3.000000 f(b)=\PYGZhy{}0.008790 f(xm)=\PYGZhy{}0.000226 b\PYGZhy{}a=0.338624
Solução encontrada: 0.33763504551140067
\end{sphinxVerbatim}

\begin{sphinxVerbatim}[commandchars=\\\{\}]
0.33763504551140067
\end{sphinxVerbatim}


\section{Problemas aplicados}
\label{\detokenize{lista-2-solucoes:problemas-aplicados}}
\sphinxAtStartPar
A temperatura \(T\) (em graus Kelvin) de um semicondutor pode ser calculada como uma função da resistência \(R\) (em ohms) pela equação de Steinhart\sphinxhyphen{}Hart:
\begin{equation*}
\begin{split}T = \dfrac{1}{A + B\ln(R) + C\ln(R)^3},\end{split}
\end{equation*}
\sphinxAtStartPar
para constantes \(A\), \(B\) e \(C\) dependentes do material. Supondo que \(A = 1.4056 \times 10^{-3}\), \(B = 240.7620 \times 10^{-6} K^{-1}\) e \(C = 7.48 \times 10^{-7}\), determine a resistência do semicondutor para uma temperatura de 135 graus Celsius.

\sphinxAtStartPar
\sphinxstylestrong{Sugestão:} use o método da bisseção e tolerância de \(10^{-6}\).


\subsection{Solução computacional}
\label{\detokenize{lista-2-solucoes:solucao-computacional}}
\begin{sphinxVerbatim}[commandchars=\\\{\}]
\PYG{k+kn}{from} \PYG{n+nn}{scipy}\PYG{n+nn}{.}\PYG{n+nn}{optimize} \PYG{k+kn}{import} \PYG{n}{bisect}

\PYG{c+c1}{\PYGZsh{} constantes}
\PYG{n}{A} \PYG{o}{=} \PYG{l+m+mf}{1.4056e\PYGZhy{}3}
\PYG{n}{B} \PYG{o}{=} \PYG{l+m+mf}{240.7620e\PYGZhy{}6}
\PYG{n}{C} \PYG{o}{=} \PYG{l+m+mf}{7.48e\PYGZhy{}7}
\PYG{n}{Tc} \PYG{o}{=} \PYG{l+m+mf}{135.0} \PYG{c+c1}{\PYGZsh{} temperatura em Celsius}
\PYG{n}{Tk} \PYG{o}{=} \PYG{n}{Tc} \PYG{o}{+} \PYG{l+m+mf}{273.15} \PYG{c+c1}{\PYGZsh{} temperatura em Kelvin}

\PYG{c+c1}{\PYGZsh{} f(R) = 0}
\PYG{n}{f} \PYG{o}{=} \PYG{k}{lambda} \PYG{n}{R}\PYG{p}{:} \PYG{l+m+mi}{1}\PYG{o}{/}\PYG{p}{(}\PYG{n}{A} \PYG{o}{+} \PYG{n}{B}\PYG{o}{*}\PYG{n}{np}\PYG{o}{.}\PYG{n}{log}\PYG{p}{(}\PYG{n}{R}\PYG{p}{)} \PYG{o}{+} \PYG{n}{C}\PYG{o}{*}\PYG{n}{np}\PYG{o}{.}\PYG{n}{log}\PYG{p}{(}\PYG{n}{R}\PYG{p}{)}\PYG{o}{*}\PYG{o}{*}\PYG{l+m+mi}{3}\PYG{p}{)} \PYG{o}{\PYGZhy{}} \PYG{n}{Tk}

\PYG{c+c1}{\PYGZsh{} localização }
\PYG{n}{R} \PYG{o}{=} \PYG{n}{np}\PYG{o}{.}\PYG{n}{linspace}\PYG{p}{(}\PYG{l+m+mi}{1}\PYG{p}{,}\PYG{l+m+mi}{100}\PYG{p}{,}\PYG{l+m+mi}{50}\PYG{p}{,}\PYG{k+kc}{True}\PYG{p}{)}
\PYG{n}{plt}\PYG{o}{.}\PYG{n}{plot}\PYG{p}{(}\PYG{n}{R}\PYG{p}{,}\PYG{n}{f}\PYG{p}{(}\PYG{n}{R}\PYG{p}{)}\PYG{p}{)}
\PYG{n}{plt}\PYG{o}{.}\PYG{n}{xlabel}\PYG{p}{(}\PYG{l+s+s1}{\PYGZsq{}}\PYG{l+s+s1}{\PYGZdl{}R\PYGZdl{}}\PYG{l+s+s1}{\PYGZsq{}}\PYG{p}{)}
\PYG{n}{plt}\PYG{o}{.}\PYG{n}{ylabel}\PYG{p}{(}\PYG{l+s+s1}{\PYGZsq{}}\PYG{l+s+s1}{\PYGZdl{}T\PYGZdl{}}\PYG{l+s+s1}{\PYGZsq{}}\PYG{p}{)}

\PYG{c+c1}{\PYGZsh{} refinamento a=1; b=100}
\PYG{n}{x} \PYG{o}{=} \PYG{n}{bisect}\PYG{p}{(}\PYG{n}{f}\PYG{p}{,}\PYG{l+m+mi}{1}\PYG{p}{,}\PYG{l+m+mi}{100}\PYG{p}{,}\PYG{n}{xtol}\PYG{o}{=}\PYG{l+m+mf}{1e\PYGZhy{}6}\PYG{p}{)}
\PYG{n+nb}{print}\PYG{p}{(}\PYG{l+s+s2}{\PYGZdq{}}\PYG{l+s+s2}{Para }\PYG{l+s+si}{\PYGZob{}0:.2f\PYGZcb{}}\PYG{l+s+s2}{ graus Celsius, a resistência é de }\PYG{l+s+si}{\PYGZob{}1:.2f\PYGZcb{}}\PYG{l+s+s2}{ ohms}\PYG{l+s+s2}{\PYGZdq{}}\PYG{o}{.}\PYG{n}{format}\PYG{p}{(}\PYG{n}{Tc}\PYG{p}{,}\PYG{n}{x}\PYG{p}{)}\PYG{p}{)}
\end{sphinxVerbatim}

\begin{sphinxVerbatim}[commandchars=\\\{\}]
Para 135.00 graus Celsius, a resistência é de 61.61 ohms
\end{sphinxVerbatim}

\noindent\sphinxincludegraphics{{lista-2-solucoes_32_1}.png}




\section{Problema aplicado (revisar!)}
\label{\detokenize{lista-2-solucoes:problema-aplicado-revisar}}
\sphinxAtStartPar
Quando se calcula o pagamento de uma hipoteca, a relação entre o montante do empréstimo \(E\), o pagamento mensal \(P\), a duração do empréstimo em anos \(a\), e a taxa de juros anual \(t\) é dada pela equação:
\begin{equation*}
\begin{split}P = \dfrac{Et}{12\left( 1 - \dfrac{1}{\left(1 + \frac{t}{12}\right)^{12a}} \right)}.\end{split}
\end{equation*}
\sphinxAtStartPar
Determine a taxa \(t\) de um empréstimo de R\$ 170.000 00 por 20 anos se o pagamento mensal for de R\$ 1.250,00.

\sphinxAtStartPar
\sphinxstylestrong{Sugestão:} usar método de Newton como função programada pelo estudante ou função residente.


\subsection{Solução computacional:}
\label{\detokenize{lista-2-solucoes:id1}}
\sphinxAtStartPar
Resolver o problema \(f(t) = 0\) com método de Newton.

\begin{sphinxVerbatim}[commandchars=\\\{\}]
\PYG{k+kn}{from} \PYG{n+nn}{scipy}\PYG{n+nn}{.}\PYG{n+nn}{optimize} \PYG{k+kn}{import} \PYG{n}{newton} 
\PYG{k+kn}{from} \PYG{n+nn}{sympy}\PYG{n+nn}{.}\PYG{n+nn}{utilities}\PYG{n+nn}{.}\PYG{n+nn}{lambdify} \PYG{k+kn}{import} \PYG{n}{lambdastr}

\PYG{n}{E} \PYG{o}{=} \PYG{l+m+mi}{170000} \PYG{c+c1}{\PYGZsh{} montante do empréstimo}
\PYG{n}{P} \PYG{o}{=} \PYG{l+m+mi}{1250} \PYG{c+c1}{\PYGZsh{} pagamento mensal}
\PYG{n}{a} \PYG{o}{=} \PYG{l+m+mi}{20} \PYG{c+c1}{\PYGZsh{} duração do empréstimo em anos}

\PYG{c+c1}{\PYGZsh{} define função para computar }
\PYG{c+c1}{\PYGZsh{} t: taxa de juros (incógnita)}
\PYG{k}{def} \PYG{n+nf}{taxa\PYGZus{}anuidade}\PYG{p}{(}\PYG{n}{E}\PYG{p}{,}\PYG{n}{P}\PYG{p}{,}\PYG{n}{a}\PYG{p}{)}\PYG{p}{:}

    \PYG{n}{Esym}\PYG{p}{,}\PYG{n}{t}\PYG{p}{,}\PYG{n}{asym}\PYG{p}{,}\PYG{n}{Psym} \PYG{o}{=} \PYG{n}{sy}\PYG{o}{.}\PYG{n}{symbols}\PYG{p}{(}\PYG{l+s+s1}{\PYGZsq{}}\PYG{l+s+s1}{Esym t asym Psym}\PYG{l+s+s1}{\PYGZsq{}}\PYG{p}{)}

    \PYG{n}{fsym} \PYG{o}{=} \PYG{n}{Esym}\PYG{o}{*}\PYG{n}{t}\PYG{o}{/}\PYG{p}{(} \PYG{l+m+mi}{12}\PYG{o}{*}\PYG{p}{(}\PYG{l+m+mi}{1} \PYG{o}{\PYGZhy{}} \PYG{l+m+mi}{1}\PYG{o}{/}\PYG{p}{(}\PYG{l+m+mi}{1} \PYG{o}{+} \PYG{n}{t}\PYG{o}{/}\PYG{l+m+mi}{12}\PYG{p}{)}\PYG{o}{*}\PYG{o}{*}\PYG{p}{(}\PYG{l+m+mi}{12}\PYG{o}{*}\PYG{n}{asym}\PYG{p}{)} \PYG{p}{)} \PYG{p}{)} \PYG{o}{\PYGZhy{}} \PYG{n}{Psym} 
    \PYG{n}{dfsym} \PYG{o}{=} \PYG{n}{sy}\PYG{o}{.}\PYG{n}{diff}\PYG{p}{(}\PYG{n}{fsym}\PYG{p}{,}\PYG{n}{t}\PYG{p}{)}

    \PYG{n}{f} \PYG{o}{=} \PYG{n}{fsym}\PYG{o}{.}\PYG{n}{subs}\PYG{p}{(}\PYG{p}{\PYGZob{}}\PYG{l+s+s1}{\PYGZsq{}}\PYG{l+s+s1}{Esym}\PYG{l+s+s1}{\PYGZsq{}}\PYG{p}{:}\PYG{n}{E}\PYG{p}{,}\PYG{l+s+s1}{\PYGZsq{}}\PYG{l+s+s1}{asym}\PYG{l+s+s1}{\PYGZsq{}}\PYG{p}{:}\PYG{n}{a}\PYG{p}{,}\PYG{l+s+s1}{\PYGZsq{}}\PYG{l+s+s1}{Psym}\PYG{l+s+s1}{\PYGZsq{}}\PYG{p}{:}\PYG{n}{P}\PYG{p}{\PYGZcb{}}\PYG{p}{)}
    \PYG{n}{df} \PYG{o}{=} \PYG{n}{dfsym}\PYG{o}{.}\PYG{n}{subs}\PYG{p}{(}\PYG{p}{\PYGZob{}}\PYG{l+s+s1}{\PYGZsq{}}\PYG{l+s+s1}{Esym}\PYG{l+s+s1}{\PYGZsq{}}\PYG{p}{:}\PYG{n}{E}\PYG{p}{,}\PYG{l+s+s1}{\PYGZsq{}}\PYG{l+s+s1}{asym}\PYG{l+s+s1}{\PYGZsq{}}\PYG{p}{:}\PYG{n}{a}\PYG{p}{,}\PYG{l+s+s1}{\PYGZsq{}}\PYG{l+s+s1}{Psym}\PYG{l+s+s1}{\PYGZsq{}}\PYG{p}{:}\PYG{n}{P}\PYG{p}{\PYGZcb{}}\PYG{p}{)}

    \PYG{n}{ft} \PYG{o}{=} \PYG{n+nb}{eval}\PYG{p}{(}\PYG{n}{lambdastr}\PYG{p}{(}\PYG{n}{t}\PYG{p}{,}\PYG{n}{f}\PYG{p}{)}\PYG{p}{)}
    \PYG{n}{dft} \PYG{o}{=} \PYG{n+nb}{eval}\PYG{p}{(}\PYG{n}{lambdastr}\PYG{p}{(}\PYG{n}{t}\PYG{p}{,}\PYG{n}{df}\PYG{p}{)}\PYG{p}{)}

    \PYG{c+c1}{\PYGZsh{} Aplicação do método de Newton }

    \PYG{c+c1}{\PYGZsh{} parâmetros }
    \PYG{n}{t0} \PYG{o}{=} \PYG{l+m+mf}{1.0} \PYG{c+c1}{\PYGZsh{} estimativa inicial}
    \PYG{n}{tole}\PYG{o}{=}\PYG{l+m+mf}{1e\PYGZhy{}3} \PYG{c+c1}{\PYGZsh{} tolerância de erro}
    \PYG{n}{nmax}\PYG{o}{=}\PYG{l+m+mi}{50} \PYG{c+c1}{\PYGZsh{} número máximo de iterações }

    \PYG{c+c1}{\PYGZsh{} método de Newton }
    \PYG{n}{t} \PYG{o}{=} \PYG{n}{newton}\PYG{p}{(}\PYG{n}{ft}\PYG{p}{,}\PYG{n}{t0}\PYG{p}{,}\PYG{n}{dft}\PYG{p}{,}\PYG{n}{tol}\PYG{o}{=}\PYG{n}{tole}\PYG{p}{,}\PYG{n}{maxiter}\PYG{o}{=}\PYG{n}{nmax}\PYG{p}{)}    
    \PYG{n}{msg} \PYG{o}{=} \PYG{l+s+s2}{\PYGZdq{}}\PYG{l+s+s2}{A taxa do empréstimo é de }\PYG{l+s+si}{\PYGZob{}0:.2f\PYGZcb{}}\PYG{l+s+si}{\PYGZpc{} a}\PYG{l+s+s2}{.a.}\PYG{l+s+s2}{\PYGZdq{}}\PYG{o}{.}\PYG{n}{format}\PYG{p}{(}\PYG{n}{t}\PYG{o}{*}\PYG{l+m+mi}{100}\PYG{p}{)}
    
    \PYG{k}{return} \PYG{p}{(}\PYG{n}{t}\PYG{p}{,}\PYG{n}{msg}\PYG{p}{)}

\PYG{c+c1}{\PYGZsh{} invoca função}
\PYG{n}{taxa\PYGZus{}anuidade}\PYG{p}{(}\PYG{n}{E}\PYG{p}{,}\PYG{n}{P}\PYG{p}{,}\PYG{n}{a}\PYG{p}{)}
\PYG{c+c1}{\PYGZsh{} não sei se a equação do livro está correta: negativo até 12 }
\PYG{c+c1}{\PYGZsh{} aa = np.arange(1,50)}
\PYG{c+c1}{\PYGZsh{} tt = []}
\PYG{c+c1}{\PYGZsh{} for i in aa:}
\PYG{c+c1}{\PYGZsh{}     t=taxa\PYGZus{}anuidade(E,P,i)}
\PYG{c+c1}{\PYGZsh{}     print(t[1])}
\PYG{c+c1}{\PYGZsh{}     tt.append(t[0])}
\PYG{c+c1}{\PYGZsh{} tt = np.asarray(tt)}
\PYG{c+c1}{\PYGZsh{} plt.plot(aa,tt)}
\end{sphinxVerbatim}

\begin{sphinxVerbatim}[commandchars=\\\{\}]
(0.0632479251521274, \PYGZsq{}A taxa do empréstimo é de 6.32\PYGZpc{} a.a.\PYGZsq{})
\end{sphinxVerbatim}


\chapter{Lista de Exercícios 3}
\label{\detokenize{lista-3-solucoes:lista-de-exercicios-3}}\label{\detokenize{lista-3-solucoes::doc}}
\sphinxAtStartPar
Solucionário matemático e computacional de exercícios selecionados da Lista de Exercícios 3.

\begin{sphinxVerbatim}[commandchars=\\\{\}]
\PYG{o}{\PYGZpc{}}\PYG{k}{matplotlib} inline
\end{sphinxVerbatim}

\begin{sphinxVerbatim}[commandchars=\\\{\}]
\PYG{k+kn}{import} \PYG{n+nn}{numpy} \PYG{k}{as} \PYG{n+nn}{np}
\PYG{k+kn}{import} \PYG{n+nn}{matplotlib}\PYG{n+nn}{.}\PYG{n+nn}{pyplot} \PYG{k}{as} \PYG{n+nn}{plt}
\end{sphinxVerbatim}

\begin{sphinxVerbatim}[commandchars=\\\{\}]
\PYG{c+c1}{\PYGZsh{} Funções Implementadas}

\PYG{k}{def} \PYG{n+nf}{gauss\PYGZus{}simples}\PYG{p}{(}\PYG{n}{AB}\PYG{p}{)}\PYG{p}{:}
    \PYG{l+s+sd}{\PYGZsq{}\PYGZsq{}\PYGZsq{}Realiza o cálculo de um sistema linear através do método de eliminação de Gauss sem pivotamento.}
\PYG{l+s+sd}{    }
\PYG{l+s+sd}{    Sinopse:}
\PYG{l+s+sd}{         X = gauss\PYGZus{}simples(AB)}
\PYG{l+s+sd}{    }
\PYG{l+s+sd}{    Entradas:}
\PYG{l+s+sd}{         AB \PYGZhy{} Matriz aumentada (np.array) do sistema linear}
\PYG{l+s+sd}{     }
\PYG{l+s+sd}{    Saídas:}
\PYG{l+s+sd}{         X \PYGZhy{} Vetor solução do sistema linear}
\PYG{l+s+sd}{         }
\PYG{l+s+sd}{         @ney}
\PYG{l+s+sd}{    \PYGZsq{}\PYGZsq{}\PYGZsq{}}
    
    \PYG{c+c1}{\PYGZsh{} Eliminação progressiva de variáveis}
    \PYG{k}{for} \PYG{n}{i} \PYG{o+ow}{in} \PYG{n+nb}{range}\PYG{p}{(}\PYG{n+nb}{len}\PYG{p}{(}\PYG{n}{AB}\PYG{p}{)}\PYG{p}{)}\PYG{p}{:}
        \PYG{k}{for} \PYG{n}{j} \PYG{o+ow}{in} \PYG{n+nb}{range}\PYG{p}{(}\PYG{n+nb}{len}\PYG{p}{(}\PYG{n}{AB}\PYG{p}{)}\PYG{p}{)}\PYG{p}{:}
            \PYG{k}{if} \PYG{p}{(}\PYG{n}{j} \PYG{o}{\PYGZgt{}} \PYG{n}{i}\PYG{p}{)}\PYG{p}{:}
                \PYG{n}{m} \PYG{o}{=} \PYG{n}{AB}\PYG{p}{[}\PYG{n}{j}\PYG{p}{,}\PYG{n}{i}\PYG{p}{]}\PYG{o}{/}\PYG{n}{AB}\PYG{p}{[}\PYG{n}{i}\PYG{p}{,}\PYG{n}{i}\PYG{p}{]} \PYG{c+c1}{\PYGZsh{} Fator multiplicador}
                \PYG{n}{AB}\PYG{p}{[}\PYG{n}{j}\PYG{p}{,}\PYG{p}{:}\PYG{p}{]} \PYG{o}{\PYGZhy{}}\PYG{o}{=} \PYG{n}{m}\PYG{o}{*}\PYG{n}{AB}\PYG{p}{[}\PYG{n}{i}\PYG{p}{,}\PYG{p}{:}\PYG{p}{]} \PYG{c+c1}{\PYGZsh{} Eliminação de variável}

    \PYG{c+c1}{\PYGZsh{} Substituição regressiva}
    \PYG{n}{A} \PYG{o}{=} \PYG{n}{AB}\PYG{p}{[}\PYG{p}{:}\PYG{p}{,}\PYG{l+m+mi}{0}\PYG{p}{:}\PYG{o}{\PYGZhy{}}\PYG{l+m+mi}{1}\PYG{p}{]}
    \PYG{n}{B} \PYG{o}{=} \PYG{n}{AB}\PYG{p}{[}\PYG{p}{:}\PYG{p}{,}\PYG{o}{\PYGZhy{}}\PYG{l+m+mi}{1}\PYG{p}{]}
    \PYG{n}{X} \PYG{o}{=} \PYG{n}{np}\PYG{o}{.}\PYG{n}{zeros}\PYG{p}{(}\PYG{p}{(}\PYG{n+nb}{len}\PYG{p}{(}\PYG{n}{AB}\PYG{p}{)}\PYG{p}{,}\PYG{l+m+mi}{1}\PYG{p}{)}\PYG{p}{)}

    \PYG{k}{for} \PYG{n}{i} \PYG{o+ow}{in} \PYG{n+nb}{range}\PYG{p}{(}\PYG{n+nb}{len}\PYG{p}{(}\PYG{n}{A}\PYG{p}{)}\PYG{o}{\PYGZhy{}}\PYG{l+m+mi}{1}\PYG{p}{,} \PYG{o}{\PYGZhy{}}\PYG{l+m+mi}{1}\PYG{p}{,} \PYG{o}{\PYGZhy{}}\PYG{l+m+mi}{1}\PYG{p}{)}\PYG{p}{:}
        \PYG{k}{for} \PYG{n}{j} \PYG{o+ow}{in} \PYG{n+nb}{range}\PYG{p}{(}\PYG{n+nb}{len}\PYG{p}{(}\PYG{n}{A}\PYG{p}{)}\PYG{p}{)}\PYG{p}{:}
            \PYG{n}{X}\PYG{p}{[}\PYG{n}{i}\PYG{p}{]} \PYG{o}{\PYGZhy{}}\PYG{o}{=} \PYG{n}{A}\PYG{p}{[}\PYG{n}{i}\PYG{p}{,}\PYG{n}{j}\PYG{p}{]}\PYG{o}{*}\PYG{n}{X}\PYG{p}{[}\PYG{n}{j}\PYG{p}{]}
        \PYG{n}{X}\PYG{p}{[}\PYG{n}{i}\PYG{p}{]} \PYG{o}{+}\PYG{o}{=} \PYG{n}{B}\PYG{p}{[}\PYG{n}{i}\PYG{p}{]}
        \PYG{n}{X}\PYG{p}{[}\PYG{n}{i}\PYG{p}{]} \PYG{o}{/}\PYG{o}{=} \PYG{n}{A}\PYG{p}{[}\PYG{n}{i}\PYG{p}{,}\PYG{n}{i}\PYG{p}{]}
    
    \PYG{n}{X} \PYG{o}{=} \PYG{n}{np}\PYG{o}{.}\PYG{n}{around}\PYG{p}{(}\PYG{n}{X}\PYG{p}{,} \PYG{n}{decimals}\PYG{o}{=}\PYG{l+m+mi}{3}\PYG{p}{)}
        
    \PYG{k}{return} \PYG{n}{X}

\PYG{k}{def} \PYG{n+nf}{gauss\PYGZus{}pivparc}\PYG{p}{(}\PYG{n}{AB}\PYG{p}{)}\PYG{p}{:}
    \PYG{l+s+sd}{\PYGZsq{}\PYGZsq{}\PYGZsq{}Realiza o cálculo de um sistema linear através do método de eliminação de Gauss com pivotamento parcial.}
\PYG{l+s+sd}{    }
\PYG{l+s+sd}{    Sinopse:}
\PYG{l+s+sd}{         X = gauss\PYGZus{}pivparc(AB)}
\PYG{l+s+sd}{    }
\PYG{l+s+sd}{    Entradas:}
\PYG{l+s+sd}{         AB \PYGZhy{} Matriz aumentada (np.array) do sistema linear}
\PYG{l+s+sd}{     }
\PYG{l+s+sd}{    Saídas:}
\PYG{l+s+sd}{         X \PYGZhy{} Vetor solução do sistema linear}
\PYG{l+s+sd}{         }
\PYG{l+s+sd}{         @ney}
\PYG{l+s+sd}{    \PYGZsq{}\PYGZsq{}\PYGZsq{}}
    
    \PYG{c+c1}{\PYGZsh{} Eliminação progressiva de variáveis}
    \PYG{k}{for} \PYG{n}{i} \PYG{o+ow}{in} \PYG{n+nb}{range}\PYG{p}{(}\PYG{n+nb}{len}\PYG{p}{(}\PYG{n}{AB}\PYG{p}{)}\PYG{p}{)}\PYG{p}{:}
        \PYG{k}{if} \PYG{p}{(}\PYG{n}{i} \PYG{o}{\PYGZlt{}} \PYG{n+nb}{len}\PYG{p}{(}\PYG{n}{AB}\PYG{p}{)}\PYG{p}{)}\PYG{p}{:} \PYG{c+c1}{\PYGZsh{} Pivotamento parcial}
            \PYG{n}{AB\PYGZus{}new} \PYG{o}{=} \PYG{n}{np}\PYG{o}{.}\PYG{n}{flip}\PYG{p}{(}\PYG{n}{AB}\PYG{p}{[}\PYG{n}{i}\PYG{p}{:}\PYG{p}{,} \PYG{n}{i}\PYG{p}{:}\PYG{p}{]}\PYG{p}{,} \PYG{l+m+mi}{0}\PYG{p}{)}
            \PYG{n}{AB}\PYG{p}{[}\PYG{n}{i}\PYG{p}{:}\PYG{p}{,} \PYG{n}{i}\PYG{p}{:}\PYG{p}{]} \PYG{o}{=} \PYG{n}{AB\PYGZus{}new}
        \PYG{k}{for} \PYG{n}{j} \PYG{o+ow}{in} \PYG{n+nb}{range}\PYG{p}{(}\PYG{n+nb}{len}\PYG{p}{(}\PYG{n}{AB}\PYG{p}{)}\PYG{p}{)}\PYG{p}{:}
            \PYG{k}{if} \PYG{p}{(}\PYG{n}{j} \PYG{o}{\PYGZgt{}} \PYG{n}{i}\PYG{p}{)}\PYG{p}{:}
                \PYG{n}{m} \PYG{o}{=} \PYG{n}{AB}\PYG{p}{[}\PYG{n}{j}\PYG{p}{,}\PYG{n}{i}\PYG{p}{]}\PYG{o}{/}\PYG{n}{AB}\PYG{p}{[}\PYG{n}{i}\PYG{p}{,}\PYG{n}{i}\PYG{p}{]} \PYG{c+c1}{\PYGZsh{} Fator multiplicador}
                \PYG{n}{AB}\PYG{p}{[}\PYG{n}{j}\PYG{p}{,}\PYG{p}{:}\PYG{p}{]} \PYG{o}{\PYGZhy{}}\PYG{o}{=} \PYG{n}{m}\PYG{o}{*}\PYG{n}{AB}\PYG{p}{[}\PYG{n}{i}\PYG{p}{,}\PYG{p}{:}\PYG{p}{]} \PYG{c+c1}{\PYGZsh{} Eliminação de variável}

    \PYG{c+c1}{\PYGZsh{} Substituição regressiva}
    \PYG{n}{A} \PYG{o}{=} \PYG{n}{AB}\PYG{p}{[}\PYG{p}{:}\PYG{p}{,}\PYG{l+m+mi}{0}\PYG{p}{:}\PYG{o}{\PYGZhy{}}\PYG{l+m+mi}{1}\PYG{p}{]}
    \PYG{n}{B} \PYG{o}{=} \PYG{n}{AB}\PYG{p}{[}\PYG{p}{:}\PYG{p}{,}\PYG{o}{\PYGZhy{}}\PYG{l+m+mi}{1}\PYG{p}{]}
    \PYG{n}{X} \PYG{o}{=} \PYG{n}{np}\PYG{o}{.}\PYG{n}{zeros}\PYG{p}{(}\PYG{p}{(}\PYG{n+nb}{len}\PYG{p}{(}\PYG{n}{AB}\PYG{p}{)}\PYG{p}{,}\PYG{l+m+mi}{1}\PYG{p}{)}\PYG{p}{)}

    \PYG{k}{for} \PYG{n}{i} \PYG{o+ow}{in} \PYG{n+nb}{range}\PYG{p}{(}\PYG{n+nb}{len}\PYG{p}{(}\PYG{n}{A}\PYG{p}{)}\PYG{o}{\PYGZhy{}}\PYG{l+m+mi}{1}\PYG{p}{,} \PYG{o}{\PYGZhy{}}\PYG{l+m+mi}{1}\PYG{p}{,} \PYG{o}{\PYGZhy{}}\PYG{l+m+mi}{1}\PYG{p}{)}\PYG{p}{:}
        \PYG{k}{for} \PYG{n}{j} \PYG{o+ow}{in} \PYG{n+nb}{range}\PYG{p}{(}\PYG{n+nb}{len}\PYG{p}{(}\PYG{n}{A}\PYG{p}{)}\PYG{p}{)}\PYG{p}{:}
            \PYG{n}{X}\PYG{p}{[}\PYG{n}{i}\PYG{p}{]} \PYG{o}{\PYGZhy{}}\PYG{o}{=} \PYG{n}{A}\PYG{p}{[}\PYG{n}{i}\PYG{p}{,}\PYG{n}{j}\PYG{p}{]}\PYG{o}{*}\PYG{n}{X}\PYG{p}{[}\PYG{n}{j}\PYG{p}{]}
        \PYG{n}{X}\PYG{p}{[}\PYG{n}{i}\PYG{p}{]} \PYG{o}{+}\PYG{o}{=} \PYG{n}{B}\PYG{p}{[}\PYG{n}{i}\PYG{p}{]}
        \PYG{n}{X}\PYG{p}{[}\PYG{n}{i}\PYG{p}{]} \PYG{o}{/}\PYG{o}{=} \PYG{n}{A}\PYG{p}{[}\PYG{n}{i}\PYG{p}{,}\PYG{n}{i}\PYG{p}{]}
    
    \PYG{n}{X} \PYG{o}{=} \PYG{n}{np}\PYG{o}{.}\PYG{n}{around}\PYG{p}{(}\PYG{n}{X}\PYG{p}{,} \PYG{n}{decimals}\PYG{o}{=}\PYG{l+m+mi}{3}\PYG{p}{)}
        
    \PYG{k}{return} \PYG{n}{X}

\PYG{k}{def} \PYG{n+nf}{gaussjordan}\PYG{p}{(}\PYG{n}{AB}\PYG{p}{)}\PYG{p}{:}
    \PYG{l+s+sd}{\PYGZsq{}\PYGZsq{}\PYGZsq{}Realiza o cálculo de um sistema linear através do método de eliminação de Gauss\PYGZhy{}Jordan simples.}
\PYG{l+s+sd}{    }
\PYG{l+s+sd}{    Sinopse:}
\PYG{l+s+sd}{         X = gaussjordan(AB)}
\PYG{l+s+sd}{    }
\PYG{l+s+sd}{    Entradas:}
\PYG{l+s+sd}{         AB \PYGZhy{} Matriz aumentada (np.array) do sistema linear}
\PYG{l+s+sd}{     }
\PYG{l+s+sd}{    Saídas:}
\PYG{l+s+sd}{         X \PYGZhy{} Vetor solução do sistema linear}
\PYG{l+s+sd}{         }
\PYG{l+s+sd}{         @ney}
\PYG{l+s+sd}{    \PYGZsq{}\PYGZsq{}\PYGZsq{}}
    
    \PYG{k}{for} \PYG{n}{i} \PYG{o+ow}{in} \PYG{n+nb}{range}\PYG{p}{(}\PYG{n+nb}{len}\PYG{p}{(}\PYG{n}{AB}\PYG{p}{)}\PYG{p}{)}\PYG{p}{:}
        \PYG{n}{AB}\PYG{p}{[}\PYG{n}{i}\PYG{p}{,}\PYG{p}{:}\PYG{p}{]} \PYG{o}{/}\PYG{o}{=} \PYG{n}{AB}\PYG{p}{[}\PYG{n}{i}\PYG{p}{,}\PYG{n}{i}\PYG{p}{]}
        \PYG{k}{for} \PYG{n}{j} \PYG{o+ow}{in} \PYG{n+nb}{range}\PYG{p}{(}\PYG{n+nb}{len}\PYG{p}{(}\PYG{n}{AB}\PYG{p}{)}\PYG{p}{)}\PYG{p}{:}
            \PYG{k}{if} \PYG{p}{(}\PYG{n}{j} \PYG{o}{!=} \PYG{n}{i}\PYG{p}{)}\PYG{p}{:}
                \PYG{n}{AB}\PYG{p}{[}\PYG{n}{j}\PYG{p}{,}\PYG{p}{:}\PYG{p}{]} \PYG{o}{\PYGZhy{}}\PYG{o}{=} \PYG{n}{AB}\PYG{p}{[}\PYG{n}{j}\PYG{p}{,}\PYG{n}{i}\PYG{p}{]}\PYG{o}{*}\PYG{n}{AB}\PYG{p}{[}\PYG{n}{i}\PYG{p}{,}\PYG{p}{:}\PYG{p}{]}
    
    \PYG{n}{AB} \PYG{o}{=} \PYG{n}{np}\PYG{o}{.}\PYG{n}{around}\PYG{p}{(}\PYG{n}{AB}\PYG{p}{,} \PYG{n}{decimals}\PYG{o}{=}\PYG{l+m+mi}{3}\PYG{p}{)}
                
    \PYG{k}{return} \PYG{n}{AB}\PYG{p}{[}\PYG{p}{:}\PYG{p}{,}\PYG{o}{\PYGZhy{}}\PYG{l+m+mi}{1}\PYG{p}{]}

\PYG{k}{def} \PYG{n+nf}{lu\PYGZus{}solver}\PYG{p}{(}\PYG{n}{AB}\PYG{p}{)}\PYG{p}{:}
    \PYG{l+s+sd}{\PYGZsq{}\PYGZsq{}\PYGZsq{}Realiza o cálculo de um sistema linear através do método de decomposição LU.}
\PYG{l+s+sd}{    }
\PYG{l+s+sd}{    Sinopse:}
\PYG{l+s+sd}{         L, U, X = lu\PYGZus{}solver(AB)}
\PYG{l+s+sd}{    }
\PYG{l+s+sd}{    Entradas:}
\PYG{l+s+sd}{         AB \PYGZhy{} Matriz aumentada (np.array) do sistema linear}
\PYG{l+s+sd}{     }
\PYG{l+s+sd}{    Saídas:}
\PYG{l+s+sd}{         L (lu[0]) \PYGZhy{} Matriz triangular inferior}
\PYG{l+s+sd}{         U (lu[1]) \PYGZhy{} Matriz triangular superior}
\PYG{l+s+sd}{         X (lu[2]) \PYGZhy{} Vetor solução do sistema linear}
\PYG{l+s+sd}{         }
\PYG{l+s+sd}{         @ney}
\PYG{l+s+sd}{    \PYGZsq{}\PYGZsq{}\PYGZsq{}}
    
    \PYG{c+c1}{\PYGZsh{} Decomposição LU}
    \PYG{n}{U} \PYG{o}{=} \PYG{n}{np}\PYG{o}{.}\PYG{n}{copy}\PYG{p}{(}\PYG{n}{AB}\PYG{p}{)}
    \PYG{n}{U} \PYG{o}{=} \PYG{n}{U}\PYG{p}{[}\PYG{p}{:}\PYG{p}{,}\PYG{l+m+mi}{0}\PYG{p}{:}\PYG{o}{\PYGZhy{}}\PYG{l+m+mi}{1}\PYG{p}{]}
    \PYG{n}{L} \PYG{o}{=} \PYG{n}{np}\PYG{o}{.}\PYG{n}{zeros}\PYG{p}{(}\PYG{p}{(}\PYG{n+nb}{len}\PYG{p}{(}\PYG{n}{AB}\PYG{p}{)}\PYG{p}{,}\PYG{n+nb}{len}\PYG{p}{(}\PYG{n}{AB}\PYG{p}{)}\PYG{p}{)}\PYG{p}{)}

    \PYG{k}{for} \PYG{n}{i} \PYG{o+ow}{in} \PYG{n+nb}{range}\PYG{p}{(}\PYG{n+nb}{len}\PYG{p}{(}\PYG{n}{AB}\PYG{p}{)}\PYG{p}{)}\PYG{p}{:}
        \PYG{k}{for} \PYG{n}{j} \PYG{o+ow}{in} \PYG{n+nb}{range}\PYG{p}{(}\PYG{n+nb}{len}\PYG{p}{(}\PYG{n}{AB}\PYG{p}{)}\PYG{p}{)}\PYG{p}{:}
            \PYG{k}{if} \PYG{p}{(}\PYG{n}{j} \PYG{o}{\PYGZgt{}} \PYG{n}{i}\PYG{p}{)}\PYG{p}{:}
                \PYG{n}{m} \PYG{o}{=} \PYG{n}{U}\PYG{p}{[}\PYG{n}{j}\PYG{p}{,}\PYG{n}{i}\PYG{p}{]}\PYG{o}{/}\PYG{n}{U}\PYG{p}{[}\PYG{n}{i}\PYG{p}{,}\PYG{n}{i}\PYG{p}{]}
                \PYG{n}{U}\PYG{p}{[}\PYG{n}{j}\PYG{p}{,}\PYG{p}{:}\PYG{p}{]} \PYG{o}{\PYGZhy{}}\PYG{o}{=} \PYG{n}{m}\PYG{o}{*}\PYG{n}{U}\PYG{p}{[}\PYG{n}{i}\PYG{p}{,}\PYG{p}{:}\PYG{p}{]}
                \PYG{n}{L}\PYG{p}{[}\PYG{n}{j}\PYG{p}{,}\PYG{n}{i}\PYG{p}{]} \PYG{o}{=} \PYG{n}{m}
            \PYG{k}{elif} \PYG{p}{(}\PYG{n}{j} \PYG{o}{==} \PYG{n}{i}\PYG{p}{)}\PYG{p}{:}
                \PYG{n}{L}\PYG{p}{[}\PYG{n}{i}\PYG{p}{,}\PYG{n}{i}\PYG{p}{]} \PYG{o}{=} \PYG{l+m+mi}{1}

    \PYG{c+c1}{\PYGZsh{} Substituição progressiva}
    \PYG{n}{B} \PYG{o}{=} \PYG{n}{np}\PYG{o}{.}\PYG{n}{copy}\PYG{p}{(}\PYG{n}{AB}\PYG{p}{)}
    \PYG{n}{B} \PYG{o}{=} \PYG{n}{B}\PYG{p}{[}\PYG{p}{:}\PYG{p}{,}\PYG{o}{\PYGZhy{}}\PYG{l+m+mi}{1}\PYG{p}{]}
    \PYG{n}{D} \PYG{o}{=} \PYG{n}{np}\PYG{o}{.}\PYG{n}{zeros}\PYG{p}{(}\PYG{p}{(}\PYG{n+nb}{len}\PYG{p}{(}\PYG{n}{AB}\PYG{p}{)}\PYG{p}{,}\PYG{l+m+mi}{1}\PYG{p}{)}\PYG{p}{)}

    \PYG{k}{for} \PYG{n}{i} \PYG{o+ow}{in} \PYG{n+nb}{range}\PYG{p}{(}\PYG{n+nb}{len}\PYG{p}{(}\PYG{n}{L}\PYG{p}{)}\PYG{p}{)}\PYG{p}{:}
        \PYG{k}{for} \PYG{n}{j} \PYG{o+ow}{in} \PYG{n+nb}{range}\PYG{p}{(}\PYG{n+nb}{len}\PYG{p}{(}\PYG{n}{L}\PYG{p}{)}\PYG{p}{)}\PYG{p}{:}
            \PYG{k}{if} \PYG{p}{(}\PYG{n}{i} \PYG{o}{\PYGZgt{}} \PYG{n}{j}\PYG{p}{)}\PYG{p}{:}
                \PYG{n}{D}\PYG{p}{[}\PYG{n}{i}\PYG{p}{]} \PYG{o}{\PYGZhy{}}\PYG{o}{=} \PYG{n}{L}\PYG{p}{[}\PYG{n}{i}\PYG{p}{,}\PYG{n}{j}\PYG{p}{]}\PYG{o}{*}\PYG{n}{D}\PYG{p}{[}\PYG{n}{j}\PYG{p}{]}
        \PYG{n}{D}\PYG{p}{[}\PYG{n}{i}\PYG{p}{]} \PYG{o}{+}\PYG{o}{=} \PYG{n}{B}\PYG{p}{[}\PYG{n}{i}\PYG{p}{]}

    \PYG{c+c1}{\PYGZsh{} Substituição regressiva}
    \PYG{n}{X} \PYG{o}{=} \PYG{n}{np}\PYG{o}{.}\PYG{n}{zeros}\PYG{p}{(}\PYG{p}{(}\PYG{n+nb}{len}\PYG{p}{(}\PYG{n}{AB}\PYG{p}{)}\PYG{p}{,}\PYG{l+m+mi}{1}\PYG{p}{)}\PYG{p}{)}

    \PYG{k}{for} \PYG{n}{i} \PYG{o+ow}{in} \PYG{n+nb}{range}\PYG{p}{(}\PYG{n+nb}{len}\PYG{p}{(}\PYG{n}{U}\PYG{p}{)}\PYG{o}{\PYGZhy{}}\PYG{l+m+mi}{1}\PYG{p}{,} \PYG{o}{\PYGZhy{}}\PYG{l+m+mi}{1}\PYG{p}{,} \PYG{o}{\PYGZhy{}}\PYG{l+m+mi}{1}\PYG{p}{)}\PYG{p}{:}
        \PYG{k}{for} \PYG{n}{j} \PYG{o+ow}{in} \PYG{n+nb}{range}\PYG{p}{(}\PYG{n+nb}{len}\PYG{p}{(}\PYG{n}{U}\PYG{p}{)}\PYG{p}{)}\PYG{p}{:}
            \PYG{n}{X}\PYG{p}{[}\PYG{n}{i}\PYG{p}{]} \PYG{o}{\PYGZhy{}}\PYG{o}{=} \PYG{n}{U}\PYG{p}{[}\PYG{n}{i}\PYG{p}{,}\PYG{n}{j}\PYG{p}{]}\PYG{o}{*}\PYG{n}{X}\PYG{p}{[}\PYG{n}{j}\PYG{p}{]}
        \PYG{n}{X}\PYG{p}{[}\PYG{n}{i}\PYG{p}{]} \PYG{o}{+}\PYG{o}{=} \PYG{n}{D}\PYG{p}{[}\PYG{n}{i}\PYG{p}{]}
        \PYG{n}{X}\PYG{p}{[}\PYG{n}{i}\PYG{p}{]} \PYG{o}{/}\PYG{o}{=} \PYG{n}{U}\PYG{p}{[}\PYG{n}{i}\PYG{p}{,}\PYG{n}{i}\PYG{p}{]}
    
    \PYG{n}{L} \PYG{o}{=} \PYG{n}{np}\PYG{o}{.}\PYG{n}{around}\PYG{p}{(}\PYG{n}{L}\PYG{p}{,} \PYG{n}{decimals}\PYG{o}{=}\PYG{l+m+mi}{3}\PYG{p}{)}
    \PYG{n}{U} \PYG{o}{=} \PYG{n}{np}\PYG{o}{.}\PYG{n}{around}\PYG{p}{(}\PYG{n}{U}\PYG{p}{,} \PYG{n}{decimals}\PYG{o}{=}\PYG{l+m+mi}{3}\PYG{p}{)}
    \PYG{n}{X} \PYG{o}{=} \PYG{n}{np}\PYG{o}{.}\PYG{n}{around}\PYG{p}{(}\PYG{n}{X}\PYG{p}{,} \PYG{n}{decimals}\PYG{o}{=}\PYG{l+m+mi}{3}\PYG{p}{)}
    
    \PYG{k}{return} \PYG{p}{(}\PYG{n}{L}\PYG{p}{,} \PYG{n}{U}\PYG{p}{,} \PYG{n}{X}\PYG{p}{)}

\PYG{k}{def} \PYG{n+nf}{cholesky}\PYG{p}{(}\PYG{n}{AB}\PYG{p}{)}\PYG{p}{:}
    \PYG{l+s+sd}{\PYGZsq{}\PYGZsq{}\PYGZsq{}Realiza o cálculo de um sistema linear através do método LU, por decomposição de Cholesky.}
\PYG{l+s+sd}{    }
\PYG{l+s+sd}{    Sinopse:}
\PYG{l+s+sd}{         C = cholesky(AB)}
\PYG{l+s+sd}{    }
\PYG{l+s+sd}{    Entradas:}
\PYG{l+s+sd}{         AB \PYGZhy{} Matriz aumentada (np.array) do sistema linear}
\PYG{l+s+sd}{     }
\PYG{l+s+sd}{    Saídas:}
\PYG{l+s+sd}{         G (C[0]) \PYGZhy{} Matriz triangular inferior}
\PYG{l+s+sd}{         X (C[1]) \PYGZhy{} Vetor solução do sistema linear}
\PYG{l+s+sd}{         }
\PYG{l+s+sd}{         @ney}
\PYG{l+s+sd}{    \PYGZsq{}\PYGZsq{}\PYGZsq{}}

    \PYG{c+c1}{\PYGZsh{} Decomposição de Cholesky}
    \PYG{n}{A} \PYG{o}{=} \PYG{n}{np}\PYG{o}{.}\PYG{n}{copy}\PYG{p}{(}\PYG{n}{AB}\PYG{p}{)}
    \PYG{n}{A} \PYG{o}{=} \PYG{n}{A}\PYG{p}{[}\PYG{p}{:}\PYG{p}{,}\PYG{l+m+mi}{0}\PYG{p}{:}\PYG{o}{\PYGZhy{}}\PYG{l+m+mi}{1}\PYG{p}{]}
    \PYG{n}{G} \PYG{o}{=} \PYG{n}{np}\PYG{o}{.}\PYG{n}{zeros}\PYG{p}{(}\PYG{p}{(}\PYG{n+nb}{len}\PYG{p}{(}\PYG{n}{A}\PYG{p}{)}\PYG{p}{,}\PYG{n+nb}{len}\PYG{p}{(}\PYG{n}{A}\PYG{p}{)}\PYG{p}{)}\PYG{p}{)}
    
    \PYG{k}{for} \PYG{n}{k} \PYG{o+ow}{in} \PYG{n+nb}{range}\PYG{p}{(}\PYG{n+nb}{len}\PYG{p}{(}\PYG{n}{A}\PYG{p}{)}\PYG{p}{)}\PYG{p}{:}
        \PYG{k}{for} \PYG{n}{i} \PYG{o+ow}{in} \PYG{n+nb}{range}\PYG{p}{(}\PYG{n}{k}\PYG{p}{)}\PYG{p}{:}
            \PYG{n}{s1} \PYG{o}{=} \PYG{l+m+mi}{0}
            \PYG{k}{for} \PYG{n}{j} \PYG{o+ow}{in} \PYG{n+nb}{range}\PYG{p}{(}\PYG{n}{i}\PYG{p}{)}\PYG{p}{:}
                \PYG{n}{s1} \PYG{o}{+}\PYG{o}{=} \PYG{n}{G}\PYG{p}{[}\PYG{n}{i}\PYG{p}{,}\PYG{n}{j}\PYG{p}{]}\PYG{o}{*}\PYG{n}{G}\PYG{p}{[}\PYG{n}{k}\PYG{p}{,}\PYG{n}{j}\PYG{p}{]}
            \PYG{n}{G}\PYG{p}{[}\PYG{n}{k}\PYG{p}{,}\PYG{n}{i}\PYG{p}{]} \PYG{o}{=} \PYG{p}{(}\PYG{n}{A}\PYG{p}{[}\PYG{n}{k}\PYG{p}{,}\PYG{n}{i}\PYG{p}{]} \PYG{o}{\PYGZhy{}} \PYG{n}{s1}\PYG{p}{)}\PYG{o}{/}\PYG{n}{G}\PYG{p}{[}\PYG{n}{i}\PYG{p}{,}\PYG{n}{i}\PYG{p}{]}
        \PYG{n}{s2} \PYG{o}{=} \PYG{l+m+mi}{0}
        \PYG{k}{for} \PYG{n}{j} \PYG{o+ow}{in} \PYG{n+nb}{range}\PYG{p}{(}\PYG{n}{k}\PYG{p}{)}\PYG{p}{:}
            \PYG{n}{s2} \PYG{o}{+}\PYG{o}{=} \PYG{p}{(}\PYG{n}{G}\PYG{p}{[}\PYG{n}{k}\PYG{p}{,}\PYG{n}{j}\PYG{p}{]}\PYG{p}{)}\PYG{o}{*}\PYG{o}{*}\PYG{l+m+mi}{2}
        \PYG{n}{G}\PYG{p}{[}\PYG{n}{k}\PYG{p}{,}\PYG{n}{k}\PYG{p}{]} \PYG{o}{=} \PYG{n}{np}\PYG{o}{.}\PYG{n}{sqrt}\PYG{p}{(}\PYG{n}{A}\PYG{p}{[}\PYG{n}{k}\PYG{p}{,}\PYG{n}{k}\PYG{p}{]} \PYG{o}{\PYGZhy{}} \PYG{n}{s2}\PYG{p}{)}
    
    \PYG{c+c1}{\PYGZsh{} Substituição progressiva}
    \PYG{n}{B} \PYG{o}{=} \PYG{n}{np}\PYG{o}{.}\PYG{n}{copy}\PYG{p}{(}\PYG{n}{AB}\PYG{p}{)}
    \PYG{n}{B} \PYG{o}{=} \PYG{n}{B}\PYG{p}{[}\PYG{p}{:}\PYG{p}{,}\PYG{o}{\PYGZhy{}}\PYG{l+m+mi}{1}\PYG{p}{]}
    \PYG{n}{D} \PYG{o}{=} \PYG{n}{np}\PYG{o}{.}\PYG{n}{zeros}\PYG{p}{(}\PYG{p}{(}\PYG{n+nb}{len}\PYG{p}{(}\PYG{n}{AB}\PYG{p}{)}\PYG{p}{,}\PYG{l+m+mi}{1}\PYG{p}{)}\PYG{p}{)}

    \PYG{k}{for} \PYG{n}{i} \PYG{o+ow}{in} \PYG{n+nb}{range}\PYG{p}{(}\PYG{n+nb}{len}\PYG{p}{(}\PYG{n}{G}\PYG{p}{)}\PYG{p}{)}\PYG{p}{:}
        \PYG{k}{for} \PYG{n}{j} \PYG{o+ow}{in} \PYG{n+nb}{range}\PYG{p}{(}\PYG{n+nb}{len}\PYG{p}{(}\PYG{n}{G}\PYG{p}{)}\PYG{p}{)}\PYG{p}{:}
            \PYG{k}{if} \PYG{p}{(}\PYG{n}{i} \PYG{o}{\PYGZgt{}} \PYG{n}{j}\PYG{p}{)}\PYG{p}{:}
                \PYG{n}{D}\PYG{p}{[}\PYG{n}{i}\PYG{p}{]} \PYG{o}{\PYGZhy{}}\PYG{o}{=} \PYG{n}{G}\PYG{p}{[}\PYG{n}{i}\PYG{p}{,}\PYG{n}{j}\PYG{p}{]}\PYG{o}{*}\PYG{n}{D}\PYG{p}{[}\PYG{n}{j}\PYG{p}{]}
        \PYG{n}{D}\PYG{p}{[}\PYG{n}{i}\PYG{p}{]} \PYG{o}{+}\PYG{o}{=} \PYG{n}{B}\PYG{p}{[}\PYG{n}{i}\PYG{p}{]}
        \PYG{n}{D}\PYG{p}{[}\PYG{n}{i}\PYG{p}{]} \PYG{o}{/}\PYG{o}{=} \PYG{n}{G}\PYG{p}{[}\PYG{n}{i}\PYG{p}{,}\PYG{n}{i}\PYG{p}{]}

    \PYG{c+c1}{\PYGZsh{} Substituição regressiva}
    \PYG{n}{GT} \PYG{o}{=} \PYG{n}{G}\PYG{o}{.}\PYG{n}{transpose}\PYG{p}{(}\PYG{p}{)}
    \PYG{n}{X} \PYG{o}{=} \PYG{n}{np}\PYG{o}{.}\PYG{n}{zeros}\PYG{p}{(}\PYG{p}{(}\PYG{n+nb}{len}\PYG{p}{(}\PYG{n}{AB}\PYG{p}{)}\PYG{p}{,}\PYG{l+m+mi}{1}\PYG{p}{)}\PYG{p}{)}

    \PYG{k}{for} \PYG{n}{i} \PYG{o+ow}{in} \PYG{n+nb}{range}\PYG{p}{(}\PYG{n+nb}{len}\PYG{p}{(}\PYG{n}{GT}\PYG{p}{)}\PYG{o}{\PYGZhy{}}\PYG{l+m+mi}{1}\PYG{p}{,} \PYG{o}{\PYGZhy{}}\PYG{l+m+mi}{1}\PYG{p}{,} \PYG{o}{\PYGZhy{}}\PYG{l+m+mi}{1}\PYG{p}{)}\PYG{p}{:}
        \PYG{k}{for} \PYG{n}{j} \PYG{o+ow}{in} \PYG{n+nb}{range}\PYG{p}{(}\PYG{n+nb}{len}\PYG{p}{(}\PYG{n}{GT}\PYG{p}{)}\PYG{p}{)}\PYG{p}{:}
            \PYG{n}{X}\PYG{p}{[}\PYG{n}{i}\PYG{p}{]} \PYG{o}{\PYGZhy{}}\PYG{o}{=} \PYG{n}{GT}\PYG{p}{[}\PYG{n}{i}\PYG{p}{,}\PYG{n}{j}\PYG{p}{]}\PYG{o}{*}\PYG{n}{X}\PYG{p}{[}\PYG{n}{j}\PYG{p}{]}
        \PYG{n}{X}\PYG{p}{[}\PYG{n}{i}\PYG{p}{]} \PYG{o}{+}\PYG{o}{=} \PYG{n}{D}\PYG{p}{[}\PYG{n}{i}\PYG{p}{]}
        \PYG{n}{X}\PYG{p}{[}\PYG{n}{i}\PYG{p}{]} \PYG{o}{/}\PYG{o}{=} \PYG{n}{GT}\PYG{p}{[}\PYG{n}{i}\PYG{p}{,}\PYG{n}{i}\PYG{p}{]}
    
    \PYG{n}{G} \PYG{o}{=} \PYG{n}{np}\PYG{o}{.}\PYG{n}{around}\PYG{p}{(}\PYG{n}{G}\PYG{p}{,} \PYG{n}{decimals}\PYG{o}{=}\PYG{l+m+mi}{3}\PYG{p}{)}
    \PYG{n}{X} \PYG{o}{=} \PYG{n}{np}\PYG{o}{.}\PYG{n}{around}\PYG{p}{(}\PYG{n}{X}\PYG{p}{,} \PYG{n}{decimals}\PYG{o}{=}\PYG{l+m+mi}{3}\PYG{p}{)}
    
    \PYG{k}{return} \PYG{n}{G}\PYG{p}{,}\PYG{n}{X}
\end{sphinxVerbatim}


\section{Sistemas lineares}
\label{\detokenize{lista-3-solucoes:sistemas-lineares}}

\subsection{Questão 1}
\label{\detokenize{lista-3-solucoes:questao-1}}
\sphinxAtStartPar
Escreva o seguinte conjunto de equações na forma matricial:
\label{equation:lista-3-solucoes:f6963643-92a7-41fa-b42b-4426a94878a6}\begin{align} 
8 &= 6x_3 + 2x_2 \\
2 − x_1 &= x_3 \\
5x_2 + x_1 &= 13
\end{align}
\sphinxAtStartPar
Mulitiplique a matriz dos coeficientes por sua transposta, i.e. \(AA^T\).

\begin{sphinxVerbatim}[commandchars=\\\{\}]
\PYG{c+c1}{\PYGZsh{} Solução}

\PYG{n}{A} \PYG{o}{=} \PYG{n}{np}\PYG{o}{.}\PYG{n}{array}\PYG{p}{(}\PYG{p}{[}\PYG{p}{[}\PYG{l+m+mi}{0}\PYG{p}{,} \PYG{l+m+mi}{2}\PYG{p}{,} \PYG{l+m+mi}{6}\PYG{p}{]}\PYG{p}{,} \PYG{p}{[}\PYG{l+m+mi}{1}\PYG{p}{,} \PYG{l+m+mi}{0}\PYG{p}{,} \PYG{l+m+mi}{1}\PYG{p}{]}\PYG{p}{,} \PYG{p}{[}\PYG{l+m+mi}{1}\PYG{p}{,} \PYG{l+m+mi}{5}\PYG{p}{,} \PYG{l+m+mi}{0}\PYG{p}{]}\PYG{p}{]}\PYG{p}{)}
\PYG{n}{AT} \PYG{o}{=} \PYG{n}{A}\PYG{o}{.}\PYG{n}{transpose}\PYG{p}{(}\PYG{p}{)}
\PYG{n}{B} \PYG{o}{=} \PYG{n}{np}\PYG{o}{.}\PYG{n}{dot}\PYG{p}{(}\PYG{n}{A}\PYG{p}{,}\PYG{n}{AT}\PYG{p}{)}

\PYG{n+nb}{print}\PYG{p}{(}\PYG{l+s+s2}{\PYGZdq{}}\PYG{l+s+s2}{A matriz dos coeficientes desse sistema é: }\PYG{l+s+se}{\PYGZbs{}n}\PYG{l+s+se}{\PYGZbs{}n}\PYG{l+s+s2}{\PYGZdq{}}\PYG{p}{,} \PYG{n}{A}\PYG{p}{)}
\PYG{n+nb}{print}\PYG{p}{(}\PYG{l+s+s2}{\PYGZdq{}}\PYG{l+s+se}{\PYGZbs{}n}\PYG{l+s+s2}{A sua transposta é: }\PYG{l+s+se}{\PYGZbs{}n}\PYG{l+s+se}{\PYGZbs{}n}\PYG{l+s+s2}{\PYGZdq{}}\PYG{p}{,} \PYG{n}{AT}\PYG{p}{)}
\PYG{n+nb}{print}\PYG{p}{(}\PYG{l+s+s2}{\PYGZdq{}}\PYG{l+s+se}{\PYGZbs{}n}\PYG{l+s+s2}{Logo, o produto entre A e sua transposta é: }\PYG{l+s+se}{\PYGZbs{}n}\PYG{l+s+se}{\PYGZbs{}n}\PYG{l+s+s2}{\PYGZdq{}}\PYG{p}{,} \PYG{n}{B}\PYG{p}{)}
\end{sphinxVerbatim}

\begin{sphinxVerbatim}[commandchars=\\\{\}]
A matriz dos coeficientes desse sistema é: 

 [[0 2 6]
 [1 0 1]
 [1 5 0]]

A sua transposta é: 

 [[0 1 1]
 [2 0 5]
 [6 1 0]]

Logo, o produto entre A e sua transposta é: 

 [[40  6 10]
 [ 6  2  1]
 [10  1 26]]
\end{sphinxVerbatim}


\subsection{Questão 2}
\label{\detokenize{lista-3-solucoes:questao-2}}
\sphinxAtStartPar
Use o metodo gráfico para resolver:
\label{equation:lista-3-solucoes:e202baee-0a38-480e-bbfe-8893b2fa583c}\begin{align} 
4x_1 − 8x_2 &= −24 \\
−x_1 + 6x_2 &= 34
\end{align}
\begin{sphinxVerbatim}[commandchars=\\\{\}]
\PYG{c+c1}{\PYGZsh{} Solução}

\PYG{k}{def} \PYG{n+nf}{f1}\PYG{p}{(}\PYG{n}{x}\PYG{p}{)}\PYG{p}{:}
    \PYG{k}{return} \PYG{p}{(}\PYG{l+m+mi}{4}\PYG{o}{*}\PYG{n}{x} \PYG{o}{+} \PYG{l+m+mi}{24}\PYG{p}{)}\PYG{o}{/}\PYG{l+m+mi}{8} \PYG{c+c1}{\PYGZsh{} x2 isolado na primeira equação}

\PYG{k}{def} \PYG{n+nf}{f2}\PYG{p}{(}\PYG{n}{x}\PYG{p}{)}\PYG{p}{:}
    \PYG{k}{return} \PYG{p}{(}\PYG{n}{x} \PYG{o}{+} \PYG{l+m+mi}{34}\PYG{p}{)}\PYG{o}{/}\PYG{l+m+mi}{6} \PYG{c+c1}{\PYGZsh{} x2 isolado na segunda equação}

\PYG{n}{x1} \PYG{o}{=} \PYG{n}{np}\PYG{o}{.}\PYG{n}{linspace}\PYG{p}{(}\PYG{l+m+mi}{0}\PYG{p}{,} \PYG{l+m+mi}{15}\PYG{p}{,} \PYG{l+m+mi}{100}\PYG{p}{)}

\PYG{n}{y1} \PYG{o}{=} \PYG{n}{f1}\PYG{p}{(}\PYG{n}{x1}\PYG{p}{)} \PYG{c+c1}{\PYGZsh{} Valores de x2 na primeira equação}
\PYG{n}{y2} \PYG{o}{=} \PYG{n}{f2}\PYG{p}{(}\PYG{n}{x1}\PYG{p}{)} \PYG{c+c1}{\PYGZsh{} Valores de x2 na segunda equação}

\PYG{n}{plt}\PYG{o}{.}\PYG{n}{plot}\PYG{p}{(}\PYG{n}{x1}\PYG{p}{,}\PYG{n}{y1}\PYG{p}{,} \PYG{n}{label}\PYG{o}{=}\PYG{l+s+s2}{\PYGZdq{}}\PYG{l+s+s2}{Reta da equação 1}\PYG{l+s+s2}{\PYGZdq{}}\PYG{p}{)}
\PYG{n}{plt}\PYG{o}{.}\PYG{n}{plot}\PYG{p}{(}\PYG{n}{x1}\PYG{p}{,}\PYG{n}{y2}\PYG{p}{,} \PYG{n}{label}\PYG{o}{=}\PYG{l+s+s2}{\PYGZdq{}}\PYG{l+s+s2}{Reta da equação 2}\PYG{l+s+s2}{\PYGZdq{}}\PYG{p}{)}
\PYG{n}{plt}\PYG{o}{.}\PYG{n}{grid}\PYG{p}{(}\PYG{p}{)}
\PYG{n}{plt}\PYG{o}{.}\PYG{n}{legend}\PYG{p}{(}\PYG{p}{)}
\PYG{n}{plt}\PYG{o}{.}\PYG{n}{plot}\PYG{p}{(}\PYG{l+m+mi}{8}\PYG{p}{,}\PYG{l+m+mi}{7}\PYG{p}{,}\PYG{l+s+s1}{\PYGZsq{}}\PYG{l+s+s1}{or}\PYG{l+s+s1}{\PYGZsq{}}\PYG{p}{)}
\end{sphinxVerbatim}

\begin{sphinxVerbatim}[commandchars=\\\{\}]
[\PYGZlt{}matplotlib.lines.Line2D at 0x7f84f958d8d0\PYGZgt{}]
\end{sphinxVerbatim}

\noindent\sphinxincludegraphics{{lista-3-solucoes_8_1}.png}


\subsection{Questão 3}
\label{\detokenize{lista-3-solucoes:questao-3}}
\sphinxAtStartPar
Para o conjunto de equações:
\label{equation:lista-3-solucoes:c7d22b46-79f3-4433-bcd0-6842317332db}\begin{align} 
2x_2 + 5x_3 &= 9 \\
2x_1 + x_2 + x_3 &= 9 \\
3x_1 + x_2 &= 10
\end{align}
\sphinxAtStartPar
(i) Calcule o determinante.

\sphinxAtStartPar
(ii) Use a regra de Cramer para determinar \(x_1\), \(x_2\), \(x_3\).

\begin{sphinxVerbatim}[commandchars=\\\{\}]
\PYG{c+c1}{\PYGZsh{} Solução}

\PYG{c+c1}{\PYGZsh{} (i)}
\PYG{n}{A} \PYG{o}{=} \PYG{n}{np}\PYG{o}{.}\PYG{n}{array}\PYG{p}{(}\PYG{p}{[}\PYG{p}{[}\PYG{l+m+mi}{0}\PYG{p}{,} \PYG{l+m+mi}{2}\PYG{p}{,} \PYG{l+m+mi}{5}\PYG{p}{]}\PYG{p}{,} \PYG{p}{[}\PYG{l+m+mi}{2}\PYG{p}{,} \PYG{l+m+mi}{1}\PYG{p}{,} \PYG{l+m+mi}{1}\PYG{p}{]}\PYG{p}{,} \PYG{p}{[}\PYG{l+m+mi}{3}\PYG{p}{,} \PYG{l+m+mi}{1}\PYG{p}{,} \PYG{l+m+mi}{0}\PYG{p}{]}\PYG{p}{]}\PYG{p}{)}
\PYG{n}{det} \PYG{o}{=} \PYG{n}{np}\PYG{o}{.}\PYG{n}{linalg}\PYG{o}{.}\PYG{n}{det}\PYG{p}{(}\PYG{n}{A}\PYG{p}{)}
\PYG{n+nb}{print}\PYG{p}{(}\PYG{l+s+s2}{\PYGZdq{}}\PYG{l+s+s2}{(i) det(A) = }\PYG{l+s+s2}{\PYGZdq{}}\PYG{p}{,} \PYG{n}{det}\PYG{p}{)}

\PYG{c+c1}{\PYGZsh{} (ii)}
\PYG{n}{A} \PYG{o}{=} \PYG{n+nb}{tuple}\PYG{p}{(}\PYG{n}{A}\PYG{p}{)}
\PYG{n}{B} \PYG{o}{=} \PYG{n}{np}\PYG{o}{.}\PYG{n}{array}\PYG{p}{(}\PYG{p}{[}\PYG{p}{[}\PYG{l+m+mi}{9}\PYG{p}{]}\PYG{p}{,} \PYG{p}{[}\PYG{l+m+mi}{9}\PYG{p}{]}\PYG{p}{,} \PYG{p}{[}\PYG{l+m+mi}{10}\PYG{p}{]}\PYG{p}{]}\PYG{p}{)}
\PYG{n}{X} \PYG{o}{=} \PYG{p}{[}\PYG{p}{]}

\PYG{k}{for} \PYG{n}{coluna} \PYG{o+ow}{in} \PYG{n+nb}{range}\PYG{p}{(}\PYG{n+nb}{len}\PYG{p}{(}\PYG{n}{A}\PYG{p}{)}\PYG{p}{)}\PYG{p}{:}
    \PYG{n}{C} \PYG{o}{=} \PYG{n}{np}\PYG{o}{.}\PYG{n}{asarray}\PYG{p}{(}\PYG{n}{A}\PYG{p}{)}
    \PYG{k}{for} \PYG{n}{linha} \PYG{o+ow}{in} \PYG{n+nb}{range}\PYG{p}{(}\PYG{n+nb}{len}\PYG{p}{(}\PYG{n}{A}\PYG{p}{)}\PYG{p}{)}\PYG{p}{:}
        \PYG{n}{C}\PYG{p}{[}\PYG{n}{linha}\PYG{p}{]}\PYG{p}{[}\PYG{n}{coluna}\PYG{p}{]} \PYG{o}{=} \PYG{n}{B}\PYG{p}{[}\PYG{n}{linha}\PYG{p}{]}
    \PYG{n}{X}\PYG{o}{.}\PYG{n}{append}\PYG{p}{(}\PYG{n}{np}\PYG{o}{.}\PYG{n}{linalg}\PYG{o}{.}\PYG{n}{det}\PYG{p}{(}\PYG{n}{C}\PYG{p}{)}\PYG{o}{/}\PYG{n}{np}\PYG{o}{.}\PYG{n}{linalg}\PYG{o}{.}\PYG{n}{det}\PYG{p}{(}\PYG{n}{np}\PYG{o}{.}\PYG{n}{asarray}\PYG{p}{(}\PYG{n}{A}\PYG{p}{)}\PYG{p}{)}\PYG{p}{)}

\PYG{n}{X} \PYG{o}{=} \PYG{p}{[}\PYG{n+nb}{round}\PYG{p}{(}\PYG{n}{i}\PYG{p}{,}\PYG{l+m+mi}{3}\PYG{p}{)} \PYG{k}{for} \PYG{n}{i} \PYG{o+ow}{in} \PYG{n}{X}\PYG{p}{]}
    
\PYG{n+nb}{print}\PYG{p}{(}\PYG{l+s+s2}{\PYGZdq{}}\PYG{l+s+se}{\PYGZbs{}n}\PYG{l+s+s2}{(ii) Os valores de [x1, x2, x3] são:}\PYG{l+s+s2}{\PYGZdq{}}\PYG{p}{,} \PYG{n}{X}\PYG{p}{)}
\end{sphinxVerbatim}

\begin{sphinxVerbatim}[commandchars=\\\{\}]
(i) det(A) =  0.9999999999999991

(ii) Os valores de [x1, x2, x3] são: [6.0, \PYGZhy{}8.0, 5.0]
\end{sphinxVerbatim}


\subsection{Questão 4}
\label{\detokenize{lista-3-solucoes:questao-4}}
\sphinxAtStartPar
Dadas as equações:
\label{equation:lista-3-solucoes:03ccafe5-69b9-4b4a-ac4f-a19e893582e7}\begin{align} 
10x_1 + 2x_2 − x_3 &= 27 \\
−3x_1 − 6x_2 + 2x_3 &= −61.5 \\
x_1 + 5x_2 + 5x_3 &= −21.5
\end{align}
\sphinxAtStartPar
Resolva por Eliminação de Gauss Simples. Mostre todos os passos de cálculo.

\begin{sphinxVerbatim}[commandchars=\\\{\}]
\PYG{c+c1}{\PYGZsh{} Solução}

\PYG{n}{AB} \PYG{o}{=} \PYG{n}{np}\PYG{o}{.}\PYG{n}{array}\PYG{p}{(}\PYG{p}{[}\PYG{p}{[}\PYG{l+m+mf}{10.}\PYG{p}{,} \PYG{l+m+mf}{2.}\PYG{p}{,} \PYG{o}{\PYGZhy{}}\PYG{l+m+mf}{1.}\PYG{p}{,} \PYG{l+m+mi}{27}\PYG{p}{]}\PYG{p}{,} \PYG{p}{[}\PYG{o}{\PYGZhy{}}\PYG{l+m+mf}{3.}\PYG{p}{,} \PYG{o}{\PYGZhy{}}\PYG{l+m+mf}{6.}\PYG{p}{,} \PYG{l+m+mf}{2.}\PYG{p}{,} \PYG{l+m+mf}{61.5}\PYG{p}{]}\PYG{p}{,} \PYG{p}{[}\PYG{l+m+mf}{1.}\PYG{p}{,} \PYG{l+m+mf}{5.}\PYG{p}{,} \PYG{l+m+mf}{5.}\PYG{p}{,} \PYG{o}{\PYGZhy{}}\PYG{l+m+mf}{21.5}\PYG{p}{]}\PYG{p}{]}\PYG{p}{)}

\PYG{c+c1}{\PYGZsh{} Eliminação progressiva de variáveis}

\PYG{n}{f} \PYG{o}{=} \PYG{l+m+mi}{0}\PYG{p}{;}

\PYG{k}{for} \PYG{n}{i} \PYG{o+ow}{in} \PYG{n+nb}{range}\PYG{p}{(}\PYG{n+nb}{len}\PYG{p}{(}\PYG{n}{AB}\PYG{p}{)}\PYG{p}{)}\PYG{p}{:}
    \PYG{k}{for} \PYG{n}{j} \PYG{o+ow}{in} \PYG{n+nb}{range}\PYG{p}{(}\PYG{n+nb}{len}\PYG{p}{(}\PYG{n}{AB}\PYG{p}{)}\PYG{p}{)}\PYG{p}{:}
        \PYG{k}{if} \PYG{p}{(}\PYG{n}{j} \PYG{o}{\PYGZgt{}} \PYG{n}{i}\PYG{p}{)}\PYG{p}{:}
            \PYG{n}{m} \PYG{o}{=} \PYG{n}{AB}\PYG{p}{[}\PYG{n}{j}\PYG{p}{,}\PYG{n}{i}\PYG{p}{]}\PYG{o}{/}\PYG{n}{AB}\PYG{p}{[}\PYG{n}{i}\PYG{p}{,}\PYG{n}{i}\PYG{p}{]} \PYG{c+c1}{\PYGZsh{} Fator multiplicador}
            \PYG{n}{f} \PYG{o}{+}\PYG{o}{=} \PYG{l+m+mi}{1}
            \PYG{n+nb}{print}\PYG{p}{(}\PYG{l+s+s2}{\PYGZdq{}}\PYG{l+s+se}{\PYGZbs{}n}\PYG{l+s+s2}{O fator multiplicativo}\PYG{l+s+s2}{\PYGZdq{}}\PYG{p}{,} \PYG{n}{f}\PYG{p}{,} \PYG{l+s+s2}{\PYGZdq{}}\PYG{l+s+s2}{é:}\PYG{l+s+s2}{\PYGZdq{}}\PYG{p}{,} \PYG{n}{m}\PYG{p}{)}
            \PYG{n}{AB}\PYG{p}{[}\PYG{n}{j}\PYG{p}{,}\PYG{p}{:}\PYG{p}{]} \PYG{o}{\PYGZhy{}}\PYG{o}{=} \PYG{n}{m}\PYG{o}{*}\PYG{n}{AB}\PYG{p}{[}\PYG{n}{i}\PYG{p}{,}\PYG{p}{:}\PYG{p}{]} \PYG{c+c1}{\PYGZsh{} Eliminação de variável}
            \PYG{n+nb}{print}\PYG{p}{(}\PYG{l+s+s2}{\PYGZdq{}}\PYG{l+s+s2}{Após a eliminação de variável, temos:}\PYG{l+s+se}{\PYGZbs{}n}\PYG{l+s+s2}{\PYGZdq{}}\PYG{p}{,} \PYG{n}{AB}\PYG{p}{)}

\PYG{c+c1}{\PYGZsh{} Substituição regressiva}

\PYG{n}{A} \PYG{o}{=} \PYG{n}{AB}\PYG{p}{[}\PYG{p}{:}\PYG{p}{,}\PYG{l+m+mi}{0}\PYG{p}{:}\PYG{o}{\PYGZhy{}}\PYG{l+m+mi}{1}\PYG{p}{]}
\PYG{n}{B} \PYG{o}{=} \PYG{n}{AB}\PYG{p}{[}\PYG{p}{:}\PYG{p}{,}\PYG{o}{\PYGZhy{}}\PYG{l+m+mi}{1}\PYG{p}{]}
\PYG{n}{X} \PYG{o}{=} \PYG{n}{np}\PYG{o}{.}\PYG{n}{zeros}\PYG{p}{(}\PYG{p}{(}\PYG{l+m+mi}{3}\PYG{p}{,}\PYG{l+m+mi}{1}\PYG{p}{)}\PYG{p}{)}

\PYG{k}{for} \PYG{n}{i} \PYG{o+ow}{in} \PYG{n+nb}{range}\PYG{p}{(}\PYG{n+nb}{len}\PYG{p}{(}\PYG{n}{A}\PYG{p}{)}\PYG{o}{\PYGZhy{}}\PYG{l+m+mi}{1}\PYG{p}{,} \PYG{o}{\PYGZhy{}}\PYG{l+m+mi}{1}\PYG{p}{,} \PYG{o}{\PYGZhy{}}\PYG{l+m+mi}{1}\PYG{p}{)}\PYG{p}{:}
    \PYG{k}{for} \PYG{n}{j} \PYG{o+ow}{in} \PYG{n+nb}{range}\PYG{p}{(}\PYG{n+nb}{len}\PYG{p}{(}\PYG{n}{A}\PYG{p}{)}\PYG{p}{)}\PYG{p}{:}
        \PYG{n}{X}\PYG{p}{[}\PYG{n}{i}\PYG{p}{]} \PYG{o}{\PYGZhy{}}\PYG{o}{=} \PYG{n}{A}\PYG{p}{[}\PYG{n}{i}\PYG{p}{,}\PYG{n}{j}\PYG{p}{]}\PYG{o}{*}\PYG{n}{X}\PYG{p}{[}\PYG{n}{j}\PYG{p}{]}
    \PYG{n}{X}\PYG{p}{[}\PYG{n}{i}\PYG{p}{]} \PYG{o}{+}\PYG{o}{=} \PYG{n}{B}\PYG{p}{[}\PYG{n}{i}\PYG{p}{]}
    \PYG{n}{X}\PYG{p}{[}\PYG{n}{i}\PYG{p}{]} \PYG{o}{/}\PYG{o}{=} \PYG{n}{A}\PYG{p}{[}\PYG{n}{i}\PYG{p}{,}\PYG{n}{i}\PYG{p}{]}

\PYG{n}{X} \PYG{o}{=} \PYG{n}{np}\PYG{o}{.}\PYG{n}{around}\PYG{p}{(}\PYG{n}{X}\PYG{p}{,} \PYG{n}{decimals}\PYG{o}{=}\PYG{l+m+mi}{3}\PYG{p}{)}

\PYG{n+nb}{print}\PYG{p}{(}\PYG{l+s+s2}{\PYGZdq{}}\PYG{l+s+se}{\PYGZbs{}n}\PYG{l+s+s2}{A solução do problema é: }\PYG{l+s+se}{\PYGZbs{}n}\PYG{l+s+s2}{\PYGZdq{}}\PYG{p}{,} \PYG{n}{X}\PYG{p}{)}
\end{sphinxVerbatim}

\begin{sphinxVerbatim}[commandchars=\\\{\}]
O fator multiplicativo 1 é: \PYGZhy{}0.3
Após a eliminação de variável, temos:
 [[ 10.    2.   \PYGZhy{}1.   27. ]
 [  0.   \PYGZhy{}5.4   1.7  69.6]
 [  1.    5.    5.  \PYGZhy{}21.5]]

O fator multiplicativo 2 é: 0.1
Após a eliminação de variável, temos:
 [[ 10.    2.   \PYGZhy{}1.   27. ]
 [  0.   \PYGZhy{}5.4   1.7  69.6]
 [  0.    4.8   5.1 \PYGZhy{}24.2]]

O fator multiplicativo 3 é: \PYGZhy{}0.8888888888888888
Após a eliminação de variável, temos:
 [[10.          2.         \PYGZhy{}1.         27.        ]
 [ 0.         \PYGZhy{}5.4         1.7        69.6       ]
 [ 0.          0.          6.61111111 37.66666667]]

A solução do problema é: 
 [[  5.489]
 [\PYGZhy{}11.095]
 [  5.697]]
\end{sphinxVerbatim}


\subsection{Questão 5}
\label{\detokenize{lista-3-solucoes:questao-5}}
\sphinxAtStartPar
Use Eliminação de Gauss com pivotamento parcial para resolver:
\label{equation:lista-3-solucoes:3966bd08-4408-4cba-8b3a-c712c87d3e2e}\begin{align} 
8x_1 + 2x_2 − 2x_3 &= −2 \\
10x_1 + 2x_2 − 4x_3 &= 4 \\
12x_1 + 2x_2 + 2x_3 &= 6
\end{align}
\begin{sphinxVerbatim}[commandchars=\\\{\}]
\PYG{c+c1}{\PYGZsh{} Solução}

\PYG{n}{AB} \PYG{o}{=} \PYG{n}{np}\PYG{o}{.}\PYG{n}{array}\PYG{p}{(}\PYG{p}{[}\PYG{p}{[}\PYG{l+m+mf}{8.}\PYG{p}{,} \PYG{l+m+mf}{2.}\PYG{p}{,} \PYG{o}{\PYGZhy{}}\PYG{l+m+mf}{2.}\PYG{p}{,} \PYG{o}{\PYGZhy{}}\PYG{l+m+mi}{2}\PYG{p}{]}\PYG{p}{,} \PYG{p}{[}\PYG{l+m+mf}{10.}\PYG{p}{,} \PYG{l+m+mf}{2.}\PYG{p}{,} \PYG{o}{\PYGZhy{}}\PYG{l+m+mf}{4.}\PYG{p}{,} \PYG{l+m+mi}{4}\PYG{p}{]}\PYG{p}{,} \PYG{p}{[}\PYG{l+m+mf}{12.}\PYG{p}{,} \PYG{l+m+mf}{2.}\PYG{p}{,} \PYG{l+m+mf}{2.}\PYG{p}{,} \PYG{l+m+mi}{6}\PYG{p}{]}\PYG{p}{]}\PYG{p}{)}
\PYG{n}{X} \PYG{o}{=} \PYG{n}{gauss\PYGZus{}pivparc}\PYG{p}{(}\PYG{n}{AB}\PYG{p}{)}

\PYG{n+nb}{print}\PYG{p}{(}\PYG{l+s+s2}{\PYGZdq{}}\PYG{l+s+s2}{A solução do problema é: }\PYG{l+s+se}{\PYGZbs{}n}\PYG{l+s+s2}{\PYGZdq{}}\PYG{p}{,} \PYG{n}{X}\PYG{p}{)}
\end{sphinxVerbatim}

\begin{sphinxVerbatim}[commandchars=\\\{\}]
A solução do problema é: 
 [[  2.5]
 [\PYGZhy{}11.5]
 [ \PYGZhy{}0.5]]
\end{sphinxVerbatim}


\subsection{Questão 6}
\label{\detokenize{lista-3-solucoes:questao-6}}
\sphinxAtStartPar
Dadas as equações:
\label{equation:lista-3-solucoes:d75e4085-75f2-4b14-9427-30e7f024ea24}\begin{align} 
2x_1 − 6x_2 − x_3 &= −38 \\
−3x_1 − x_2 + 7x_3 &= 34 \\
−8x_1 + x_2 − 2x_3 &= −20
\end{align}
\sphinxAtStartPar
Resolva por Eliminação de Gauss com pivotamento parcial.

\begin{sphinxVerbatim}[commandchars=\\\{\}]
\PYG{c+c1}{\PYGZsh{} Solução}

\PYG{n}{AB} \PYG{o}{=} \PYG{n}{np}\PYG{o}{.}\PYG{n}{array}\PYG{p}{(}\PYG{p}{[}\PYG{p}{[}\PYG{l+m+mf}{2.}\PYG{p}{,} \PYG{o}{\PYGZhy{}}\PYG{l+m+mf}{6.}\PYG{p}{,} \PYG{o}{\PYGZhy{}}\PYG{l+m+mf}{1.}\PYG{p}{,} \PYG{o}{\PYGZhy{}}\PYG{l+m+mi}{38}\PYG{p}{]}\PYG{p}{,} \PYG{p}{[}\PYG{o}{\PYGZhy{}}\PYG{l+m+mf}{3.}\PYG{p}{,} \PYG{o}{\PYGZhy{}}\PYG{l+m+mf}{1.}\PYG{p}{,} \PYG{l+m+mf}{7.}\PYG{p}{,} \PYG{l+m+mi}{34}\PYG{p}{]}\PYG{p}{,} \PYG{p}{[}\PYG{o}{\PYGZhy{}}\PYG{l+m+mf}{8.}\PYG{p}{,} \PYG{l+m+mf}{1.}\PYG{p}{,} \PYG{o}{\PYGZhy{}}\PYG{l+m+mf}{2.}\PYG{p}{,} \PYG{o}{\PYGZhy{}}\PYG{l+m+mi}{20}\PYG{p}{]}\PYG{p}{]}\PYG{p}{)}
\PYG{n}{X} \PYG{o}{=} \PYG{n}{gauss\PYGZus{}pivparc}\PYG{p}{(}\PYG{n}{AB}\PYG{p}{)}
\PYG{n+nb}{print}\PYG{p}{(}\PYG{l+s+s2}{\PYGZdq{}}\PYG{l+s+s2}{A solução do problema é: }\PYG{l+s+se}{\PYGZbs{}n}\PYG{l+s+s2}{\PYGZdq{}}\PYG{p}{,} \PYG{n}{X}\PYG{p}{)}
\end{sphinxVerbatim}

\begin{sphinxVerbatim}[commandchars=\\\{\}]
A solução do problema é: 
 [[1.63 ]
 [5.812]
 [6.386]]
\end{sphinxVerbatim}


\subsection{Questão 7}
\label{\detokenize{lista-3-solucoes:questao-7}}
\sphinxAtStartPar
Use Eliminação de Gauss\sphinxhyphen{}Jordan para resolver:
\label{equation:lista-3-solucoes:94aff013-5070-44fc-836c-0e283ff3aa40}\begin{align} 
2x_1 + x_2 − x_3 &= 1 \\
5x_1 + 2x_2 + 2x_3 &= −4 \\
3x_1 + x_2 + x_3 &= 5
\end{align}
\sphinxAtStartPar
Não utilize pivotamento. Substitua seus resultados nas equações originais para verificá\sphinxhyphen{}los.

\begin{sphinxVerbatim}[commandchars=\\\{\}]
\PYG{c+c1}{\PYGZsh{} Solução}

\PYG{n}{AB} \PYG{o}{=} \PYG{n}{np}\PYG{o}{.}\PYG{n}{array}\PYG{p}{(}\PYG{p}{[}\PYG{p}{[}\PYG{l+m+mf}{2.}\PYG{p}{,} \PYG{l+m+mf}{1.}\PYG{p}{,} \PYG{o}{\PYGZhy{}}\PYG{l+m+mf}{1.}\PYG{p}{,} \PYG{l+m+mi}{1}\PYG{p}{]}\PYG{p}{,} \PYG{p}{[}\PYG{l+m+mf}{5.}\PYG{p}{,} \PYG{l+m+mf}{2.}\PYG{p}{,} \PYG{l+m+mf}{2.}\PYG{p}{,} \PYG{o}{\PYGZhy{}}\PYG{l+m+mi}{4}\PYG{p}{]}\PYG{p}{,} \PYG{p}{[}\PYG{l+m+mf}{3.}\PYG{p}{,} \PYG{l+m+mf}{1.}\PYG{p}{,} \PYG{l+m+mf}{1.}\PYG{p}{,} \PYG{l+m+mi}{5}\PYG{p}{]}\PYG{p}{]}\PYG{p}{)}
\PYG{n}{X} \PYG{o}{=} \PYG{n}{gaussjordan}\PYG{p}{(}\PYG{n}{AB}\PYG{p}{)}
\PYG{n+nb}{print}\PYG{p}{(}\PYG{l+s+s2}{\PYGZdq{}}\PYG{l+s+s2}{A solução do problema é}\PYG{l+s+s2}{\PYGZdq{}}\PYG{p}{,} \PYG{n}{X}\PYG{p}{)}
\end{sphinxVerbatim}

\begin{sphinxVerbatim}[commandchars=\\\{\}]
A solução do problema é [ 14. \PYGZhy{}32.  \PYGZhy{}5.]
\end{sphinxVerbatim}


\subsection{Questão 8}
\label{\detokenize{lista-3-solucoes:questao-8}}
\sphinxAtStartPar
Resolva:
\label{equation:lista-3-solucoes:5fda1d52-5b2e-4ddd-8943-b5b69a6471be}\begin{align}
x_1 + x_2 − x_3 &= −3 \\
6x_1 + 2x_2 + 2x_3 &= 2 \\
−3x_1 + 4x_2 + x_3 &= 1
\end{align}
\sphinxAtStartPar
(i) Por Eliminação de Gauss simples.

\sphinxAtStartPar
(ii) Por Eliminação de Gauss com pivotamento parcial.

\sphinxAtStartPar
(iii) Por Eliminação de Gauss\sphinxhyphen{}Jordan sem pivotamento parcial.

\begin{sphinxVerbatim}[commandchars=\\\{\}]
\PYG{c+c1}{\PYGZsh{} Solução}

\PYG{c+c1}{\PYGZsh{} (i) Eliminação de Gauss simples}
\PYG{n}{AB} \PYG{o}{=} \PYG{n}{np}\PYG{o}{.}\PYG{n}{array}\PYG{p}{(}\PYG{p}{[}\PYG{p}{[}\PYG{l+m+mf}{1.}\PYG{p}{,} \PYG{l+m+mf}{1.}\PYG{p}{,} \PYG{o}{\PYGZhy{}}\PYG{l+m+mf}{1.}\PYG{p}{,} \PYG{o}{\PYGZhy{}}\PYG{l+m+mi}{3}\PYG{p}{]}\PYG{p}{,} \PYG{p}{[}\PYG{l+m+mf}{6.}\PYG{p}{,} \PYG{l+m+mf}{2.}\PYG{p}{,} \PYG{l+m+mf}{2.}\PYG{p}{,} \PYG{l+m+mi}{2}\PYG{p}{]}\PYG{p}{,} \PYG{p}{[}\PYG{o}{\PYGZhy{}}\PYG{l+m+mf}{3.}\PYG{p}{,} \PYG{l+m+mf}{4.}\PYG{p}{,} \PYG{l+m+mf}{1.}\PYG{p}{,} \PYG{l+m+mi}{1}\PYG{p}{]}\PYG{p}{]}\PYG{p}{)}
\PYG{n}{X} \PYG{o}{=} \PYG{n}{gauss\PYGZus{}simples}\PYG{p}{(}\PYG{n}{AB}\PYG{p}{)}
\PYG{n+nb}{print}\PYG{p}{(}\PYG{l+s+s2}{\PYGZdq{}}\PYG{l+s+s2}{(i) A solução do problema, pelo método da eliminação de Gauss simples, é: }\PYG{l+s+se}{\PYGZbs{}n}\PYG{l+s+s2}{\PYGZdq{}}\PYG{p}{,} \PYG{n}{X}\PYG{p}{)}

\PYG{c+c1}{\PYGZsh{} (ii) Eliminação de Gauss com pivotamento parcial}
\PYG{n}{AB} \PYG{o}{=} \PYG{n}{np}\PYG{o}{.}\PYG{n}{array}\PYG{p}{(}\PYG{p}{[}\PYG{p}{[}\PYG{l+m+mf}{1.}\PYG{p}{,} \PYG{l+m+mf}{1.}\PYG{p}{,} \PYG{o}{\PYGZhy{}}\PYG{l+m+mf}{1.}\PYG{p}{,} \PYG{o}{\PYGZhy{}}\PYG{l+m+mi}{3}\PYG{p}{]}\PYG{p}{,} \PYG{p}{[}\PYG{l+m+mf}{6.}\PYG{p}{,} \PYG{l+m+mf}{2.}\PYG{p}{,} \PYG{l+m+mf}{2.}\PYG{p}{,} \PYG{l+m+mi}{2}\PYG{p}{]}\PYG{p}{,} \PYG{p}{[}\PYG{o}{\PYGZhy{}}\PYG{l+m+mf}{3.}\PYG{p}{,} \PYG{l+m+mf}{4.}\PYG{p}{,} \PYG{l+m+mf}{1.}\PYG{p}{,} \PYG{l+m+mi}{1}\PYG{p}{]}\PYG{p}{]}\PYG{p}{)}
\PYG{n}{X} \PYG{o}{=} \PYG{n}{gauss\PYGZus{}pivparc}\PYG{p}{(}\PYG{n}{AB}\PYG{p}{)}
\PYG{n+nb}{print}\PYG{p}{(}\PYG{l+s+s2}{\PYGZdq{}}\PYG{l+s+se}{\PYGZbs{}n}\PYG{l+s+s2}{(ii) A solução do problema, pelo método da eliminação de Gauss com pivotamento parcial, é: }\PYG{l+s+se}{\PYGZbs{}n}\PYG{l+s+s2}{\PYGZdq{}}\PYG{p}{,} \PYG{n}{X}\PYG{p}{)}

\PYG{c+c1}{\PYGZsh{} (iii) Eliminação de Gauss\PYGZhy{}Jordan sem pivotamento parcial}
\PYG{n}{AB} \PYG{o}{=} \PYG{n}{np}\PYG{o}{.}\PYG{n}{array}\PYG{p}{(}\PYG{p}{[}\PYG{p}{[}\PYG{l+m+mf}{1.}\PYG{p}{,} \PYG{l+m+mf}{1.}\PYG{p}{,} \PYG{o}{\PYGZhy{}}\PYG{l+m+mf}{1.}\PYG{p}{,} \PYG{o}{\PYGZhy{}}\PYG{l+m+mi}{3}\PYG{p}{]}\PYG{p}{,} \PYG{p}{[}\PYG{l+m+mf}{6.}\PYG{p}{,} \PYG{l+m+mf}{2.}\PYG{p}{,} \PYG{l+m+mf}{2.}\PYG{p}{,} \PYG{l+m+mi}{2}\PYG{p}{]}\PYG{p}{,} \PYG{p}{[}\PYG{o}{\PYGZhy{}}\PYG{l+m+mf}{3.}\PYG{p}{,} \PYG{l+m+mf}{4.}\PYG{p}{,} \PYG{l+m+mf}{1.}\PYG{p}{,} \PYG{l+m+mi}{1}\PYG{p}{]}\PYG{p}{]}\PYG{p}{)}
\PYG{n}{X} \PYG{o}{=} \PYG{n}{gaussjordan}\PYG{p}{(}\PYG{n}{AB}\PYG{p}{)}
\PYG{n+nb}{print}\PYG{p}{(}\PYG{l+s+s2}{\PYGZdq{}}\PYG{l+s+se}{\PYGZbs{}n}\PYG{l+s+s2}{(ii) A solução do problema, pelo método da eliminação de Gauss\PYGZhy{}Jordan sem pivotamento parcial, é: }\PYG{l+s+se}{\PYGZbs{}n}\PYG{l+s+s2}{\PYGZdq{}}\PYG{p}{,} \PYG{n}{AB}\PYG{p}{[}\PYG{p}{:}\PYG{p}{,}\PYG{o}{\PYGZhy{}}\PYG{l+m+mi}{1}\PYG{p}{]}\PYG{p}{)}
\end{sphinxVerbatim}

\begin{sphinxVerbatim}[commandchars=\\\{\}]
(i) A solução do problema, pelo método da eliminação de Gauss simples, é: 
 [[\PYGZhy{}0.25]
 [\PYGZhy{}0.5 ]
 [ 2.25]]

(ii) A solução do problema, pelo método da eliminação de Gauss com pivotamento parcial, é: 
 [[\PYGZhy{}0.25]
 [\PYGZhy{}0.5 ]
 [ 2.25]]

(ii) A solução do problema, pelo método da eliminação de Gauss\PYGZhy{}Jordan sem pivotamento parcial, é: 
 [\PYGZhy{}0.25 \PYGZhy{}0.5   2.25]
\end{sphinxVerbatim}


\section{Fatoração LU}
\label{\detokenize{lista-3-solucoes:fatoracao-lu}}

\subsection{Questão 9}
\label{\detokenize{lista-3-solucoes:questao-9}}
\sphinxAtStartPar
Resolva o seguinte sistema de equações por decomposição LU sem pivotamento (usando a definição A = LU):
\label{equation:lista-3-solucoes:73ac95ba-a489-49a9-90c6-a3513daf66e8}\begin{align} 
8x_1 + 4x_2 − x_3 &= 11 \\
−2x_1 + 5x_2 + x_3 &= 4 \\
2x_1 − x_2 + 6x_3 &= 7
\end{align}
\sphinxAtStartPar
Em seguida, determine a matriz inversa \(A^{−1} = U^{−1} L^{−1}\). Verifique seus resultados comprovando que \(AA^{−1} = I\).

\begin{sphinxVerbatim}[commandchars=\\\{\}]
\PYG{c+c1}{\PYGZsh{}Solução}
\PYG{n}{AB} \PYG{o}{=} \PYG{n}{np}\PYG{o}{.}\PYG{n}{array}\PYG{p}{(}\PYG{p}{[}\PYG{p}{[}\PYG{l+m+mf}{8.}\PYG{p}{,} \PYG{l+m+mf}{4.}\PYG{p}{,} \PYG{o}{\PYGZhy{}}\PYG{l+m+mf}{1.}\PYG{p}{,} \PYG{l+m+mf}{11.}\PYG{p}{]}\PYG{p}{,} \PYG{p}{[}\PYG{o}{\PYGZhy{}}\PYG{l+m+mf}{2.}\PYG{p}{,} \PYG{l+m+mf}{5.}\PYG{p}{,} \PYG{l+m+mf}{1.}\PYG{p}{,} \PYG{l+m+mf}{4.}\PYG{p}{]}\PYG{p}{,} \PYG{p}{[}\PYG{l+m+mf}{2.}\PYG{p}{,} \PYG{o}{\PYGZhy{}}\PYG{l+m+mf}{1.}\PYG{p}{,} \PYG{l+m+mf}{6.}\PYG{p}{,} \PYG{l+m+mf}{7.}\PYG{p}{]}\PYG{p}{]}\PYG{p}{)}

\PYG{n}{L}\PYG{p}{,} \PYG{n}{U}\PYG{p}{,} \PYG{n}{X} \PYG{o}{=} \PYG{n}{lu\PYGZus{}solver}\PYG{p}{(}\PYG{n}{AB}\PYG{p}{)}

\PYG{n}{A} \PYG{o}{=} \PYG{n}{np}\PYG{o}{.}\PYG{n}{copy}\PYG{p}{(}\PYG{n}{AB}\PYG{p}{)}
\PYG{n}{A} \PYG{o}{=} \PYG{n}{A}\PYG{p}{[}\PYG{p}{:}\PYG{p}{,}\PYG{l+m+mi}{0}\PYG{p}{:}\PYG{o}{\PYGZhy{}}\PYG{l+m+mi}{1}\PYG{p}{]}

\PYG{n+nb}{print}\PYG{p}{(}\PYG{l+s+s2}{\PYGZdq{}}\PYG{l+s+s2}{A solução do sistema é: }\PYG{l+s+se}{\PYGZbs{}n}\PYG{l+s+s2}{\PYGZdq{}}\PYG{p}{,} \PYG{n}{X}\PYG{p}{)}
\PYG{n+nb}{print}\PYG{p}{(}\PYG{l+s+s2}{\PYGZdq{}}\PYG{l+s+se}{\PYGZbs{}n}\PYG{l+s+s2}{A inversa de A é: }\PYG{l+s+se}{\PYGZbs{}n}\PYG{l+s+s2}{\PYGZdq{}}\PYG{p}{,} \PYG{n}{np}\PYG{o}{.}\PYG{n}{around}\PYG{p}{(}\PYG{n}{np}\PYG{o}{.}\PYG{n}{linalg}\PYG{o}{.}\PYG{n}{inv}\PYG{p}{(}\PYG{n}{A}\PYG{p}{)}\PYG{p}{,} \PYG{n}{decimals}\PYG{o}{=}\PYG{l+m+mi}{3}\PYG{p}{)}\PYG{p}{)}
\PYG{n+nb}{print}\PYG{p}{(}\PYG{l+s+s2}{\PYGZdq{}}\PYG{l+s+se}{\PYGZbs{}n}\PYG{l+s+s2}{O produto entre as inversas de U e L é: }\PYG{l+s+se}{\PYGZbs{}n}\PYG{l+s+s2}{\PYGZdq{}}\PYG{p}{,} \PYG{n}{np}\PYG{o}{.}\PYG{n}{around}\PYG{p}{(}\PYG{n}{np}\PYG{o}{.}\PYG{n}{dot}\PYG{p}{(}\PYG{n}{np}\PYG{o}{.}\PYG{n}{linalg}\PYG{o}{.}\PYG{n}{inv}\PYG{p}{(}\PYG{n}{U}\PYG{p}{)}\PYG{p}{,}\PYG{n}{np}\PYG{o}{.}\PYG{n}{linalg}\PYG{o}{.}\PYG{n}{inv}\PYG{p}{(}\PYG{n}{L}\PYG{p}{)}\PYG{p}{)}\PYG{p}{,} \PYG{n}{decimals}\PYG{o}{=}\PYG{l+m+mi}{3}\PYG{p}{)}\PYG{p}{)}
\PYG{n+nb}{print}\PYG{p}{(}\PYG{l+s+s2}{\PYGZdq{}}\PYG{l+s+se}{\PYGZbs{}n}\PYG{l+s+s2}{O produto entre A e sua inversa é: }\PYG{l+s+se}{\PYGZbs{}n}\PYG{l+s+s2}{\PYGZdq{}}\PYG{p}{,} \PYG{n}{np}\PYG{o}{.}\PYG{n}{around}\PYG{p}{(}\PYG{n}{np}\PYG{o}{.}\PYG{n}{dot}\PYG{p}{(}\PYG{n}{A}\PYG{p}{,}\PYG{n}{np}\PYG{o}{.}\PYG{n}{linalg}\PYG{o}{.}\PYG{n}{inv}\PYG{p}{(}\PYG{n}{A}\PYG{p}{)}\PYG{p}{)}\PYG{p}{)}\PYG{p}{)}
\end{sphinxVerbatim}

\begin{sphinxVerbatim}[commandchars=\\\{\}]
A solução do sistema é: 
 [[1.]
 [1.]
 [1.]]

A inversa de A é: 
 [[ 0.099 \PYGZhy{}0.074  0.029]
 [ 0.045  0.16  \PYGZhy{}0.019]
 [\PYGZhy{}0.026  0.051  0.154]]

O produto entre as inversas de U e L é: 
 [[ 0.099 \PYGZhy{}0.074  0.029]
 [ 0.045  0.16  \PYGZhy{}0.019]
 [\PYGZhy{}0.026  0.051  0.154]]

O produto entre A e sua inversa é: 
 [[ 1.  0.  0.]
 [\PYGZhy{}0.  1.  0.]
 [\PYGZhy{}0.  0.  1.]]
\end{sphinxVerbatim}


\subsection{Questão 10}
\label{\detokenize{lista-3-solucoes:questao-10}}
\sphinxAtStartPar
Resolva o seguinte sistema de equações por decomposição LU com pivotamento parcial.
\label{equation:lista-3-solucoes:a9b61201-dde3-47cd-80bf-f5b0dac6c48a}\begin{align} 
2x_1 − 6x_2 − x_3 &= −38 \\
−3x_1 − x_2 + 7x_3 &= −34 \\
−8x_1 + x_2 − 2x_3 &= −20
\end{align}
\sphinxAtStartPar
Em seguida, determine a matriz inversa \(A^{−1} = U^{−1} L^{−1}\). Verifique seus resultados comprovando que \(AA^{−1} = I\).

\begin{sphinxVerbatim}[commandchars=\\\{\}]
\PYG{c+c1}{\PYGZsh{}Solução}
\PYG{n}{AB} \PYG{o}{=} \PYG{n}{np}\PYG{o}{.}\PYG{n}{array}\PYG{p}{(}\PYG{p}{[}\PYG{p}{[}\PYG{l+m+mf}{2.}\PYG{p}{,} \PYG{o}{\PYGZhy{}}\PYG{l+m+mf}{6.}\PYG{p}{,} \PYG{o}{\PYGZhy{}}\PYG{l+m+mf}{1.}\PYG{p}{,} \PYG{o}{\PYGZhy{}}\PYG{l+m+mf}{38.}\PYG{p}{]}\PYG{p}{,} \PYG{p}{[}\PYG{o}{\PYGZhy{}}\PYG{l+m+mf}{3.}\PYG{p}{,} \PYG{o}{\PYGZhy{}}\PYG{l+m+mf}{1.}\PYG{p}{,} \PYG{l+m+mf}{7.}\PYG{p}{,} \PYG{o}{\PYGZhy{}}\PYG{l+m+mf}{34.}\PYG{p}{]}\PYG{p}{,} \PYG{p}{[}\PYG{o}{\PYGZhy{}}\PYG{l+m+mf}{8.}\PYG{p}{,} \PYG{l+m+mf}{1.}\PYG{p}{,} \PYG{o}{\PYGZhy{}}\PYG{l+m+mf}{2.}\PYG{p}{,} \PYG{o}{\PYGZhy{}}\PYG{l+m+mf}{20.}\PYG{p}{]}\PYG{p}{]}\PYG{p}{)}

\PYG{n}{L}\PYG{p}{,} \PYG{n}{U}\PYG{p}{,} \PYG{n}{X} \PYG{o}{=} \PYG{n}{lu\PYGZus{}solver}\PYG{p}{(}\PYG{n}{AB}\PYG{p}{)}

\PYG{n}{A} \PYG{o}{=} \PYG{n}{np}\PYG{o}{.}\PYG{n}{copy}\PYG{p}{(}\PYG{n}{AB}\PYG{p}{)}
\PYG{n}{A} \PYG{o}{=} \PYG{n}{A}\PYG{p}{[}\PYG{p}{:}\PYG{p}{,}\PYG{l+m+mi}{0}\PYG{p}{:}\PYG{o}{\PYGZhy{}}\PYG{l+m+mi}{1}\PYG{p}{]}

\PYG{n+nb}{print}\PYG{p}{(}\PYG{l+s+s2}{\PYGZdq{}}\PYG{l+s+s2}{A matriz L é: }\PYG{l+s+se}{\PYGZbs{}n}\PYG{l+s+s2}{\PYGZdq{}}\PYG{p}{,} \PYG{n}{L}\PYG{p}{)}
\PYG{n+nb}{print}\PYG{p}{(}\PYG{l+s+s2}{\PYGZdq{}}\PYG{l+s+se}{\PYGZbs{}n}\PYG{l+s+s2}{A matriz U é: }\PYG{l+s+se}{\PYGZbs{}n}\PYG{l+s+s2}{\PYGZdq{}}\PYG{p}{,} \PYG{n}{U}\PYG{p}{)}
\PYG{n+nb}{print}\PYG{p}{(}\PYG{l+s+s2}{\PYGZdq{}}\PYG{l+s+se}{\PYGZbs{}n}\PYG{l+s+s2}{A solução do sistema é: }\PYG{l+s+se}{\PYGZbs{}n}\PYG{l+s+s2}{\PYGZdq{}}\PYG{p}{,} \PYG{n}{X}\PYG{p}{)}
\PYG{n+nb}{print}\PYG{p}{(}\PYG{l+s+s2}{\PYGZdq{}}\PYG{l+s+se}{\PYGZbs{}n}\PYG{l+s+s2}{A inversa de A é: }\PYG{l+s+se}{\PYGZbs{}n}\PYG{l+s+s2}{\PYGZdq{}}\PYG{p}{,} \PYG{n}{np}\PYG{o}{.}\PYG{n}{around}\PYG{p}{(}\PYG{n}{np}\PYG{o}{.}\PYG{n}{linalg}\PYG{o}{.}\PYG{n}{inv}\PYG{p}{(}\PYG{n}{A}\PYG{p}{)}\PYG{p}{,} \PYG{n}{decimals}\PYG{o}{=}\PYG{l+m+mi}{3}\PYG{p}{)}\PYG{p}{)}
\PYG{n+nb}{print}\PYG{p}{(}\PYG{l+s+s2}{\PYGZdq{}}\PYG{l+s+se}{\PYGZbs{}n}\PYG{l+s+s2}{O produto entre as inversas de U e L é: }\PYG{l+s+se}{\PYGZbs{}n}\PYG{l+s+s2}{\PYGZdq{}}\PYG{p}{,} \PYG{n}{np}\PYG{o}{.}\PYG{n}{around}\PYG{p}{(}\PYG{n}{np}\PYG{o}{.}\PYG{n}{dot}\PYG{p}{(}\PYG{n}{np}\PYG{o}{.}\PYG{n}{linalg}\PYG{o}{.}\PYG{n}{inv}\PYG{p}{(}\PYG{n}{U}\PYG{p}{)}\PYG{p}{,}\PYG{n}{np}\PYG{o}{.}\PYG{n}{linalg}\PYG{o}{.}\PYG{n}{inv}\PYG{p}{(}\PYG{n}{L}\PYG{p}{)}\PYG{p}{)}\PYG{p}{,} \PYG{n}{decimals}\PYG{o}{=}\PYG{l+m+mi}{3}\PYG{p}{)}\PYG{p}{)}
\PYG{n+nb}{print}\PYG{p}{(}\PYG{l+s+s2}{\PYGZdq{}}\PYG{l+s+se}{\PYGZbs{}n}\PYG{l+s+s2}{O produto entre A e sua inversa é: }\PYG{l+s+se}{\PYGZbs{}n}\PYG{l+s+s2}{\PYGZdq{}}\PYG{p}{,} \PYG{n}{np}\PYG{o}{.}\PYG{n}{around}\PYG{p}{(}\PYG{n}{np}\PYG{o}{.}\PYG{n}{dot}\PYG{p}{(}\PYG{n}{A}\PYG{p}{,}\PYG{n}{np}\PYG{o}{.}\PYG{n}{linalg}\PYG{o}{.}\PYG{n}{inv}\PYG{p}{(}\PYG{n}{A}\PYG{p}{)}\PYG{p}{)}\PYG{p}{)}\PYG{p}{)}
\end{sphinxVerbatim}

\begin{sphinxVerbatim}[commandchars=\\\{\}]
A matriz L é: 
 [[ 1.   0.   0. ]
 [\PYGZhy{}1.5  1.   0. ]
 [\PYGZhy{}4.   2.3  1. ]]

A matriz U é: 
 [[  2.    \PYGZhy{}6.    \PYGZhy{}1.  ]
 [  0.   \PYGZhy{}10.     5.5 ]
 [  0.     0.   \PYGZhy{}18.65]]

A solução do sistema é: 
 [[ 4.]
 [ 8.]
 [\PYGZhy{}2.]]

A inversa de A é: 
 [[\PYGZhy{}0.013 \PYGZhy{}0.035 \PYGZhy{}0.115]
 [\PYGZhy{}0.166 \PYGZhy{}0.032 \PYGZhy{}0.029]
 [\PYGZhy{}0.029  0.123 \PYGZhy{}0.054]]

O produto entre as inversas de U e L é: 
 [[\PYGZhy{}0.013 \PYGZhy{}0.035 \PYGZhy{}0.115]
 [\PYGZhy{}0.166 \PYGZhy{}0.032 \PYGZhy{}0.029]
 [\PYGZhy{}0.029  0.123 \PYGZhy{}0.054]]

O produto entre A e sua inversa é: 
 [[ 1. \PYGZhy{}0. \PYGZhy{}0.]
 [ 0.  1.  0.]
 [ 0.  0.  1.]]
\end{sphinxVerbatim}


\subsection{Questão 11}
\label{\detokenize{lista-3-solucoes:questao-11}}
\sphinxAtStartPar
Resolva o seguinte sistema de equações por decomposição LU.
\label{equation:lista-3-solucoes:3657173d-581f-48bf-8b12-2dcbf9cad0a6}\begin{align} 
10x_1 + 2x_2 − x_3 &= 27 \\
−3x_1 − 6x_2 + 2x_3 &= −61.5 \\
x_1 + x_2 + 5x_3 &= −21.5
\end{align}
\sphinxAtStartPar
Em seguida, determine a matriz inversa \(A^{−1} = U^{−1} L^{−1}\). Verifique seus resultados comprovando que \(AA^{−1} = I\).

\begin{sphinxVerbatim}[commandchars=\\\{\}]
\PYG{c+c1}{\PYGZsh{}Solução}
\PYG{n}{AB} \PYG{o}{=} \PYG{n}{np}\PYG{o}{.}\PYG{n}{array}\PYG{p}{(}\PYG{p}{[}\PYG{p}{[}\PYG{l+m+mf}{10.}\PYG{p}{,} \PYG{l+m+mf}{2.}\PYG{p}{,} \PYG{o}{\PYGZhy{}}\PYG{l+m+mf}{1.}\PYG{p}{,} \PYG{l+m+mf}{27.}\PYG{p}{]}\PYG{p}{,} \PYG{p}{[}\PYG{o}{\PYGZhy{}}\PYG{l+m+mf}{3.}\PYG{p}{,} \PYG{o}{\PYGZhy{}}\PYG{l+m+mf}{6.}\PYG{p}{,} \PYG{l+m+mf}{2.}\PYG{p}{,} \PYG{o}{\PYGZhy{}}\PYG{l+m+mf}{61.5}\PYG{p}{]}\PYG{p}{,} \PYG{p}{[}\PYG{l+m+mf}{1.}\PYG{p}{,} \PYG{l+m+mf}{1.}\PYG{p}{,} \PYG{l+m+mf}{5.}\PYG{p}{,} \PYG{o}{\PYGZhy{}}\PYG{l+m+mf}{21.5}\PYG{p}{]}\PYG{p}{]}\PYG{p}{)}

\PYG{n}{L}\PYG{p}{,} \PYG{n}{U}\PYG{p}{,} \PYG{n}{X} \PYG{o}{=} \PYG{n}{lu\PYGZus{}solver}\PYG{p}{(}\PYG{n}{AB}\PYG{p}{)}

\PYG{n}{A} \PYG{o}{=} \PYG{n}{np}\PYG{o}{.}\PYG{n}{copy}\PYG{p}{(}\PYG{n}{AB}\PYG{p}{)}
\PYG{n}{A} \PYG{o}{=} \PYG{n}{A}\PYG{p}{[}\PYG{p}{:}\PYG{p}{,}\PYG{l+m+mi}{0}\PYG{p}{:}\PYG{o}{\PYGZhy{}}\PYG{l+m+mi}{1}\PYG{p}{]}

\PYG{n+nb}{print}\PYG{p}{(}\PYG{l+s+s2}{\PYGZdq{}}\PYG{l+s+s2}{A matriz L é: }\PYG{l+s+se}{\PYGZbs{}n}\PYG{l+s+s2}{\PYGZdq{}}\PYG{p}{,} \PYG{n}{L}\PYG{p}{)}
\PYG{n+nb}{print}\PYG{p}{(}\PYG{l+s+s2}{\PYGZdq{}}\PYG{l+s+se}{\PYGZbs{}n}\PYG{l+s+s2}{A matriz U é: }\PYG{l+s+se}{\PYGZbs{}n}\PYG{l+s+s2}{\PYGZdq{}}\PYG{p}{,} \PYG{n}{U}\PYG{p}{)}
\PYG{n+nb}{print}\PYG{p}{(}\PYG{l+s+s2}{\PYGZdq{}}\PYG{l+s+se}{\PYGZbs{}n}\PYG{l+s+s2}{A solução do sistema é: }\PYG{l+s+se}{\PYGZbs{}n}\PYG{l+s+s2}{\PYGZdq{}}\PYG{p}{,} \PYG{n}{X}\PYG{p}{)}
\PYG{n+nb}{print}\PYG{p}{(}\PYG{l+s+s2}{\PYGZdq{}}\PYG{l+s+se}{\PYGZbs{}n}\PYG{l+s+s2}{A inversa de A é: }\PYG{l+s+se}{\PYGZbs{}n}\PYG{l+s+s2}{\PYGZdq{}}\PYG{p}{,} \PYG{n}{np}\PYG{o}{.}\PYG{n}{around}\PYG{p}{(}\PYG{n}{np}\PYG{o}{.}\PYG{n}{linalg}\PYG{o}{.}\PYG{n}{inv}\PYG{p}{(}\PYG{n}{A}\PYG{p}{)}\PYG{p}{,} \PYG{n}{decimals}\PYG{o}{=}\PYG{l+m+mi}{3}\PYG{p}{)}\PYG{p}{)}
\PYG{n+nb}{print}\PYG{p}{(}\PYG{l+s+s2}{\PYGZdq{}}\PYG{l+s+se}{\PYGZbs{}n}\PYG{l+s+s2}{O produto entre as inversas de U e L é: }\PYG{l+s+se}{\PYGZbs{}n}\PYG{l+s+s2}{\PYGZdq{}}\PYG{p}{,} \PYG{n}{np}\PYG{o}{.}\PYG{n}{around}\PYG{p}{(}\PYG{n}{np}\PYG{o}{.}\PYG{n}{dot}\PYG{p}{(}\PYG{n}{np}\PYG{o}{.}\PYG{n}{linalg}\PYG{o}{.}\PYG{n}{inv}\PYG{p}{(}\PYG{n}{U}\PYG{p}{)}\PYG{p}{,}\PYG{n}{np}\PYG{o}{.}\PYG{n}{linalg}\PYG{o}{.}\PYG{n}{inv}\PYG{p}{(}\PYG{n}{L}\PYG{p}{)}\PYG{p}{)}\PYG{p}{,} \PYG{n}{decimals}\PYG{o}{=}\PYG{l+m+mi}{3}\PYG{p}{)}\PYG{p}{)}
\PYG{n+nb}{print}\PYG{p}{(}\PYG{l+s+s2}{\PYGZdq{}}\PYG{l+s+se}{\PYGZbs{}n}\PYG{l+s+s2}{O produto entre A e sua inversa é: }\PYG{l+s+se}{\PYGZbs{}n}\PYG{l+s+s2}{\PYGZdq{}}\PYG{p}{,} \PYG{n}{np}\PYG{o}{.}\PYG{n}{around}\PYG{p}{(}\PYG{n}{np}\PYG{o}{.}\PYG{n}{dot}\PYG{p}{(}\PYG{n}{A}\PYG{p}{,}\PYG{n}{np}\PYG{o}{.}\PYG{n}{linalg}\PYG{o}{.}\PYG{n}{inv}\PYG{p}{(}\PYG{n}{A}\PYG{p}{)}\PYG{p}{)}\PYG{p}{)}\PYG{p}{)}
\end{sphinxVerbatim}

\begin{sphinxVerbatim}[commandchars=\\\{\}]
A matriz L é: 
 [[ 1.     0.     0.   ]
 [\PYGZhy{}0.3    1.     0.   ]
 [ 0.1   \PYGZhy{}0.148  1.   ]]

A matriz U é: 
 [[10.     2.    \PYGZhy{}1.   ]
 [ 0.    \PYGZhy{}5.4    1.7  ]
 [ 0.     0.     5.352]]

A solução do sistema é: 
 [[ 0.5]
 [ 8. ]
 [\PYGZhy{}6. ]]

A inversa de A é: 
 [[ 0.111  0.038  0.007]
 [\PYGZhy{}0.059 \PYGZhy{}0.176  0.059]
 [\PYGZhy{}0.01   0.028  0.187]]

O produto entre as inversas de U e L é: 
 [[ 0.111  0.038  0.007]
 [\PYGZhy{}0.059 \PYGZhy{}0.176  0.059]
 [\PYGZhy{}0.01   0.028  0.187]]

O produto entre A e sua inversa é: 
 [[ 1. \PYGZhy{}0.  0.]
 [ 0.  1.  0.]
 [\PYGZhy{}0. \PYGZhy{}0.  1.]]
\end{sphinxVerbatim}


\section{Fatoração de Cholesky}
\label{\detokenize{lista-3-solucoes:fatoracao-de-cholesky}}

\subsection{Questão 12}
\label{\detokenize{lista-3-solucoes:questao-12}}
\sphinxAtStartPar
Determine quais matrizes a seguir são (i) simétricas, (ii) singulares, (iii) diagonalmente dominantes, (iv) positivas definidas.

\sphinxAtStartPar
(a) \textbackslash{}begin\{pmatrix\}
2 \& 1 \textbackslash{}
1 \& 3
\textbackslash{}end\{pmatrix\}

\sphinxAtStartPar
(b) \textbackslash{}begin\{pmatrix\}
2 \& 1 \& 0 \textbackslash{}
0 \& 3 \& 0 \textbackslash{}
1 \& 0 \& 4
\textbackslash{}end\{pmatrix\}

\sphinxAtStartPar
© \textbackslash{}begin\{pmatrix\}
4 \& 2 \& 6 \textbackslash{}
3 \& 0 \& 7 \textbackslash{}
\sphinxhyphen{}2 \& \sphinxhyphen{}1 \& \sphinxhyphen{}3
\textbackslash{}end\{pmatrix\}

\sphinxAtStartPar
(d) \textbackslash{}begin\{pmatrix\}
4 \& 0 \& 0 \& 0 \textbackslash{}
6 \& 7 \& 0 \& 0 \textbackslash{}
9 \& 11 \& 1 \& 0 \textbackslash{}
5 \& 4 \& 1 \& 1
\textbackslash{}end\{pmatrix\}

\begin{sphinxVerbatim}[commandchars=\\\{\}]
\PYG{c+c1}{\PYGZsh{} Solução}

\PYG{n}{A} \PYG{o}{=} \PYG{n}{np}\PYG{o}{.}\PYG{n}{array}\PYG{p}{(}\PYG{p}{[}\PYG{p}{[}\PYG{l+m+mf}{2.}\PYG{p}{,} \PYG{l+m+mf}{1.}\PYG{p}{]}\PYG{p}{,} \PYG{p}{[}\PYG{l+m+mf}{1.}\PYG{p}{,} \PYG{l+m+mf}{3.}\PYG{p}{]}\PYG{p}{]}\PYG{p}{)}
\PYG{n}{B} \PYG{o}{=} \PYG{n}{np}\PYG{o}{.}\PYG{n}{array}\PYG{p}{(}\PYG{p}{[}\PYG{p}{[}\PYG{l+m+mf}{2.}\PYG{p}{,} \PYG{l+m+mf}{1.}\PYG{p}{,} \PYG{l+m+mf}{0.}\PYG{p}{]}\PYG{p}{,} \PYG{p}{[}\PYG{l+m+mf}{0.}\PYG{p}{,} \PYG{l+m+mf}{3.}\PYG{p}{,} \PYG{l+m+mf}{0.}\PYG{p}{]}\PYG{p}{,} \PYG{p}{[}\PYG{l+m+mf}{1.}\PYG{p}{,} \PYG{l+m+mf}{0.}\PYG{p}{,} \PYG{l+m+mf}{4.}\PYG{p}{]}\PYG{p}{]}\PYG{p}{)}
\PYG{n}{C} \PYG{o}{=} \PYG{n}{np}\PYG{o}{.}\PYG{n}{array}\PYG{p}{(}\PYG{p}{[}\PYG{p}{[}\PYG{l+m+mf}{4.}\PYG{p}{,} \PYG{l+m+mf}{2.}\PYG{p}{,} \PYG{l+m+mf}{6.}\PYG{p}{]}\PYG{p}{,} \PYG{p}{[}\PYG{l+m+mf}{3.}\PYG{p}{,} \PYG{l+m+mf}{0.}\PYG{p}{,} \PYG{l+m+mf}{7.}\PYG{p}{]}\PYG{p}{,} \PYG{p}{[}\PYG{o}{\PYGZhy{}}\PYG{l+m+mf}{2.}\PYG{p}{,} \PYG{o}{\PYGZhy{}}\PYG{l+m+mf}{1.}\PYG{p}{,} \PYG{o}{\PYGZhy{}}\PYG{l+m+mf}{3.}\PYG{p}{]}\PYG{p}{]}\PYG{p}{)}
\PYG{n}{D} \PYG{o}{=} \PYG{n}{np}\PYG{o}{.}\PYG{n}{array}\PYG{p}{(}\PYG{p}{[}\PYG{p}{[}\PYG{l+m+mf}{4.}\PYG{p}{,} \PYG{l+m+mf}{0.}\PYG{p}{,} \PYG{l+m+mf}{0.}\PYG{p}{,} \PYG{l+m+mf}{0.}\PYG{p}{]}\PYG{p}{,} \PYG{p}{[}\PYG{l+m+mf}{6.}\PYG{p}{,} \PYG{l+m+mf}{7.}\PYG{p}{,} \PYG{l+m+mf}{0.}\PYG{p}{,} \PYG{l+m+mf}{0.}\PYG{p}{]}\PYG{p}{,} \PYG{p}{[}\PYG{l+m+mf}{9.}\PYG{p}{,} \PYG{l+m+mf}{11.}\PYG{p}{,} \PYG{l+m+mf}{1.}\PYG{p}{,} \PYG{l+m+mf}{0.}\PYG{p}{]}\PYG{p}{,} \PYG{p}{[}\PYG{l+m+mf}{5.}\PYG{p}{,} \PYG{l+m+mf}{4.}\PYG{p}{,} \PYG{l+m+mf}{1.}\PYG{p}{,} \PYG{l+m+mf}{1.}\PYG{p}{]}\PYG{p}{]}\PYG{p}{)}

\PYG{n}{menA} \PYG{o}{=} \PYG{p}{[}\PYG{p}{]}
\PYG{k}{for} \PYG{n}{i} \PYG{o+ow}{in} \PYG{n+nb}{range}\PYG{p}{(}\PYG{l+m+mi}{1}\PYG{p}{,}\PYG{n+nb}{len}\PYG{p}{(}\PYG{n}{A}\PYG{p}{)}\PYG{o}{+}\PYG{l+m+mi}{1}\PYG{p}{)}\PYG{p}{:}
    \PYG{n}{menA}\PYG{o}{.}\PYG{n}{append}\PYG{p}{(}\PYG{n}{np}\PYG{o}{.}\PYG{n}{linalg}\PYG{o}{.}\PYG{n}{det}\PYG{p}{(}\PYG{n}{A}\PYG{p}{[}\PYG{l+m+mi}{0}\PYG{p}{:}\PYG{n}{i}\PYG{p}{,} \PYG{l+m+mi}{0}\PYG{p}{:}\PYG{n}{i}\PYG{p}{]}\PYG{p}{)}\PYG{p}{)}
\PYG{n+nb}{print}\PYG{p}{(}\PYG{l+s+s2}{\PYGZdq{}}\PYG{l+s+s2}{Os determinantes dos menores principais de (a) são:}\PYG{l+s+s2}{\PYGZdq{}}\PYG{p}{,} \PYG{n}{np}\PYG{o}{.}\PYG{n}{around}\PYG{p}{(}\PYG{n}{menA}\PYG{p}{,} \PYG{n}{decimals}\PYG{o}{=}\PYG{l+m+mi}{3}\PYG{p}{)}\PYG{p}{)}

\PYG{n}{menB} \PYG{o}{=} \PYG{p}{[}\PYG{p}{]}
\PYG{k}{for} \PYG{n}{i} \PYG{o+ow}{in} \PYG{n+nb}{range}\PYG{p}{(}\PYG{l+m+mi}{1}\PYG{p}{,}\PYG{n+nb}{len}\PYG{p}{(}\PYG{n}{B}\PYG{p}{)}\PYG{o}{+}\PYG{l+m+mi}{1}\PYG{p}{)}\PYG{p}{:}
    \PYG{n}{menB}\PYG{o}{.}\PYG{n}{append}\PYG{p}{(}\PYG{n}{np}\PYG{o}{.}\PYG{n}{linalg}\PYG{o}{.}\PYG{n}{det}\PYG{p}{(}\PYG{n}{B}\PYG{p}{[}\PYG{l+m+mi}{0}\PYG{p}{:}\PYG{n}{i}\PYG{p}{,} \PYG{l+m+mi}{0}\PYG{p}{:}\PYG{n}{i}\PYG{p}{]}\PYG{p}{)}\PYG{p}{)}
\PYG{n+nb}{print}\PYG{p}{(}\PYG{l+s+s2}{\PYGZdq{}}\PYG{l+s+se}{\PYGZbs{}n}\PYG{l+s+s2}{Os determinantes dos menores principais de (b) são:}\PYG{l+s+s2}{\PYGZdq{}}\PYG{p}{,} \PYG{n}{np}\PYG{o}{.}\PYG{n}{around}\PYG{p}{(}\PYG{n}{menB}\PYG{p}{,} \PYG{n}{decimals}\PYG{o}{=}\PYG{l+m+mi}{3}\PYG{p}{)}\PYG{p}{)}

\PYG{n}{menC} \PYG{o}{=} \PYG{p}{[}\PYG{p}{]}
\PYG{k}{for} \PYG{n}{i} \PYG{o+ow}{in} \PYG{n+nb}{range}\PYG{p}{(}\PYG{l+m+mi}{1}\PYG{p}{,}\PYG{n+nb}{len}\PYG{p}{(}\PYG{n}{C}\PYG{p}{)}\PYG{o}{+}\PYG{l+m+mi}{1}\PYG{p}{)}\PYG{p}{:}
    \PYG{n}{menC}\PYG{o}{.}\PYG{n}{append}\PYG{p}{(}\PYG{n}{np}\PYG{o}{.}\PYG{n}{linalg}\PYG{o}{.}\PYG{n}{det}\PYG{p}{(}\PYG{n}{C}\PYG{p}{[}\PYG{l+m+mi}{0}\PYG{p}{:}\PYG{n}{i}\PYG{p}{,} \PYG{l+m+mi}{0}\PYG{p}{:}\PYG{n}{i}\PYG{p}{]}\PYG{p}{)}\PYG{p}{)}
\PYG{n+nb}{print}\PYG{p}{(}\PYG{l+s+s2}{\PYGZdq{}}\PYG{l+s+se}{\PYGZbs{}n}\PYG{l+s+s2}{Os determinantes dos menores principais de (c) são:}\PYG{l+s+s2}{\PYGZdq{}}\PYG{p}{,} \PYG{n}{np}\PYG{o}{.}\PYG{n}{around}\PYG{p}{(}\PYG{n}{menC}\PYG{p}{,} \PYG{n}{decimals}\PYG{o}{=}\PYG{l+m+mi}{3}\PYG{p}{)}\PYG{p}{)}

\PYG{n}{menD} \PYG{o}{=} \PYG{p}{[}\PYG{p}{]}
\PYG{k}{for} \PYG{n}{i} \PYG{o+ow}{in} \PYG{n+nb}{range}\PYG{p}{(}\PYG{l+m+mi}{1}\PYG{p}{,}\PYG{n+nb}{len}\PYG{p}{(}\PYG{n}{D}\PYG{p}{)}\PYG{o}{+}\PYG{l+m+mi}{1}\PYG{p}{)}\PYG{p}{:}
    \PYG{n}{menD}\PYG{o}{.}\PYG{n}{append}\PYG{p}{(}\PYG{n}{np}\PYG{o}{.}\PYG{n}{linalg}\PYG{o}{.}\PYG{n}{det}\PYG{p}{(}\PYG{n}{D}\PYG{p}{[}\PYG{l+m+mi}{0}\PYG{p}{:}\PYG{n}{i}\PYG{p}{,} \PYG{l+m+mi}{0}\PYG{p}{:}\PYG{n}{i}\PYG{p}{]}\PYG{p}{)}\PYG{p}{)}
\PYG{n+nb}{print}\PYG{p}{(}\PYG{l+s+s2}{\PYGZdq{}}\PYG{l+s+se}{\PYGZbs{}n}\PYG{l+s+s2}{Os determinantes dos menores principais de (d) são:}\PYG{l+s+s2}{\PYGZdq{}}\PYG{p}{,} \PYG{n}{np}\PYG{o}{.}\PYG{n}{around}\PYG{p}{(}\PYG{n}{menD}\PYG{p}{,} \PYG{n}{decimals}\PYG{o}{=}\PYG{l+m+mi}{3}\PYG{p}{)}\PYG{p}{)}
\end{sphinxVerbatim}

\begin{sphinxVerbatim}[commandchars=\\\{\}]
Os determinantes dos menores principais de (a) são: [2. 5.]

Os determinantes dos menores principais de (b) são: [ 2.  6. 24.]

Os determinantes dos menores principais de (c) são: [ 4. \PYGZhy{}6.  0.]

Os determinantes dos menores principais de (d) são: [ 4. 28. 28. 28.]
\end{sphinxVerbatim}


\subsubsection{Solução:}
\label{\detokenize{lista-3-solucoes:solucao}}
\sphinxAtStartPar
(i) Apenas a matriz em (a) e simétrica.

\sphinxAtStartPar
(ii) Apenas a matriz em (d) é singular.

\sphinxAtStartPar
(iii) As matrizes em (a) e (b) são diagonalmente dominantes.

\sphinxAtStartPar
(iv) As matrizes em (a), (b) e (d) são positivas\sphinxhyphen{}definidas.


\subsection{Questão 13}
\label{\detokenize{lista-3-solucoes:questao-13}}
\sphinxAtStartPar
Determine a fatoração de Cholesky \(A = GGT\) das matrizes a seguir.

\sphinxAtStartPar
(a) \textbackslash{}begin\{pmatrix\}
2 \& −1 \& 0 \textbackslash{}
−1 \& 2 \& −1 \textbackslash{}
0 \& −1 \& 2
\textbackslash{}end\{pmatrix\}

\sphinxAtStartPar
(b) \textbackslash{}begin\{pmatrix\}
4 \& 1 \& 1 \& 1 \textbackslash{}
1 \& 3 \& −1 \& 1 \textbackslash{}
1 \& −1 \& 2 \& 0 \textbackslash{}
1 \& 1 \& 0 \& 2
\textbackslash{}end\{pmatrix\}

\sphinxAtStartPar
© \textbackslash{}begin\{pmatrix\}
4 \& 1 \& −1 \& 0 \textbackslash{}
1 \& 3 \& −1 \& 0 \textbackslash{}
−1 \& −1 \& 5 \& 2 \textbackslash{}
0 \& 0 \& 2 \& 4
\textbackslash{}end\{pmatrix\}

\begin{sphinxVerbatim}[commandchars=\\\{\}]
\PYG{c+c1}{\PYGZsh{} Solução}

\PYG{c+c1}{\PYGZsh{} (a)}
\PYG{n+nb}{print}\PYG{p}{(}\PYG{l+s+s2}{\PYGZdq{}}\PYG{l+s+s2}{(a)}\PYG{l+s+s2}{\PYGZdq{}}\PYG{p}{)}
\PYG{n}{A} \PYG{o}{=} \PYG{n}{np}\PYG{o}{.}\PYG{n}{array}\PYG{p}{(}\PYG{p}{[}\PYG{p}{[}\PYG{l+m+mf}{2.}\PYG{p}{,} \PYG{o}{\PYGZhy{}}\PYG{l+m+mf}{1.}\PYG{p}{,} \PYG{l+m+mf}{0.}\PYG{p}{]}\PYG{p}{,} \PYG{p}{[}\PYG{o}{\PYGZhy{}}\PYG{l+m+mf}{1.}\PYG{p}{,} \PYG{l+m+mf}{2.}\PYG{p}{,} \PYG{o}{\PYGZhy{}}\PYG{l+m+mf}{1.}\PYG{p}{]}\PYG{p}{,} \PYG{p}{[}\PYG{l+m+mf}{0.}\PYG{p}{,} \PYG{o}{\PYGZhy{}}\PYG{l+m+mf}{1.}\PYG{p}{,} \PYG{l+m+mf}{2.}\PYG{p}{]}\PYG{p}{]}\PYG{p}{)}
\PYG{n}{G} \PYG{o}{=} \PYG{n}{np}\PYG{o}{.}\PYG{n}{zeros}\PYG{p}{(}\PYG{p}{(}\PYG{n+nb}{len}\PYG{p}{(}\PYG{n}{A}\PYG{p}{)}\PYG{p}{,}\PYG{n+nb}{len}\PYG{p}{(}\PYG{n}{A}\PYG{p}{)}\PYG{p}{)}\PYG{p}{)}

\PYG{k}{for} \PYG{n}{k} \PYG{o+ow}{in} \PYG{n+nb}{range}\PYG{p}{(}\PYG{n+nb}{len}\PYG{p}{(}\PYG{n}{A}\PYG{p}{)}\PYG{p}{)}\PYG{p}{:}
    \PYG{k}{for} \PYG{n}{i} \PYG{o+ow}{in} \PYG{n+nb}{range}\PYG{p}{(}\PYG{n}{k}\PYG{p}{)}\PYG{p}{:}
        \PYG{n}{s1} \PYG{o}{=} \PYG{l+m+mi}{0}
        \PYG{k}{for} \PYG{n}{j} \PYG{o+ow}{in} \PYG{n+nb}{range}\PYG{p}{(}\PYG{n}{i}\PYG{p}{)}\PYG{p}{:}
            \PYG{n}{s1} \PYG{o}{+}\PYG{o}{=} \PYG{n}{G}\PYG{p}{[}\PYG{n}{i}\PYG{p}{,}\PYG{n}{j}\PYG{p}{]}\PYG{o}{*}\PYG{n}{G}\PYG{p}{[}\PYG{n}{k}\PYG{p}{,}\PYG{n}{j}\PYG{p}{]}
        \PYG{n}{G}\PYG{p}{[}\PYG{n}{k}\PYG{p}{,}\PYG{n}{i}\PYG{p}{]} \PYG{o}{=} \PYG{p}{(}\PYG{n}{A}\PYG{p}{[}\PYG{n}{k}\PYG{p}{,}\PYG{n}{i}\PYG{p}{]} \PYG{o}{\PYGZhy{}} \PYG{n}{s1}\PYG{p}{)}\PYG{o}{/}\PYG{n}{G}\PYG{p}{[}\PYG{n}{i}\PYG{p}{,}\PYG{n}{i}\PYG{p}{]}
    \PYG{n}{s2} \PYG{o}{=} \PYG{l+m+mi}{0}
    \PYG{k}{for} \PYG{n}{j} \PYG{o+ow}{in} \PYG{n+nb}{range}\PYG{p}{(}\PYG{n}{k}\PYG{p}{)}\PYG{p}{:}
        \PYG{n}{s2} \PYG{o}{+}\PYG{o}{=} \PYG{p}{(}\PYG{n}{G}\PYG{p}{[}\PYG{n}{k}\PYG{p}{,}\PYG{n}{j}\PYG{p}{]}\PYG{p}{)}\PYG{o}{*}\PYG{o}{*}\PYG{l+m+mi}{2}
    \PYG{n}{G}\PYG{p}{[}\PYG{n}{k}\PYG{p}{,}\PYG{n}{k}\PYG{p}{]} \PYG{o}{=} \PYG{n}{np}\PYG{o}{.}\PYG{n}{sqrt}\PYG{p}{(}\PYG{n}{A}\PYG{p}{[}\PYG{n}{k}\PYG{p}{,}\PYG{n}{k}\PYG{p}{]} \PYG{o}{\PYGZhy{}} \PYG{n}{s2}\PYG{p}{)}

\PYG{n}{G} \PYG{o}{=} \PYG{n}{np}\PYG{o}{.}\PYG{n}{around}\PYG{p}{(}\PYG{n}{G}\PYG{p}{,} \PYG{n}{decimals}\PYG{o}{=}\PYG{l+m+mi}{3}\PYG{p}{)}

\PYG{n+nb}{print}\PYG{p}{(}\PYG{l+s+s2}{\PYGZdq{}}\PYG{l+s+s2}{A matriz G é:}\PYG{l+s+se}{\PYGZbs{}n}\PYG{l+s+s2}{\PYGZdq{}}\PYG{p}{,} \PYG{n}{G}\PYG{p}{)}

\PYG{c+c1}{\PYGZsh{} (b)}
\PYG{n+nb}{print}\PYG{p}{(}\PYG{l+s+s2}{\PYGZdq{}}\PYG{l+s+se}{\PYGZbs{}n}\PYG{l+s+s2}{(b)}\PYG{l+s+s2}{\PYGZdq{}}\PYG{p}{)}
\PYG{n}{B} \PYG{o}{=} \PYG{n}{np}\PYG{o}{.}\PYG{n}{array}\PYG{p}{(}\PYG{p}{[}\PYG{p}{[}\PYG{l+m+mf}{4.}\PYG{p}{,} \PYG{l+m+mf}{1.}\PYG{p}{,} \PYG{l+m+mf}{1.}\PYG{p}{,} \PYG{l+m+mf}{1.}\PYG{p}{]}\PYG{p}{,} \PYG{p}{[}\PYG{l+m+mf}{1.}\PYG{p}{,} \PYG{l+m+mf}{3.}\PYG{p}{,} \PYG{o}{\PYGZhy{}}\PYG{l+m+mf}{1.}\PYG{p}{,} \PYG{l+m+mf}{1.}\PYG{p}{]}\PYG{p}{,} \PYG{p}{[}\PYG{l+m+mf}{1.}\PYG{p}{,} \PYG{o}{\PYGZhy{}}\PYG{l+m+mf}{1.}\PYG{p}{,} \PYG{l+m+mf}{2.}\PYG{p}{,} \PYG{l+m+mf}{0.}\PYG{p}{]}\PYG{p}{,} \PYG{p}{[}\PYG{l+m+mf}{1.}\PYG{p}{,} \PYG{l+m+mf}{1.}\PYG{p}{,} \PYG{l+m+mf}{0.}\PYG{p}{,} \PYG{l+m+mf}{2.}\PYG{p}{]}\PYG{p}{]}\PYG{p}{)}
\PYG{n}{G} \PYG{o}{=} \PYG{n}{np}\PYG{o}{.}\PYG{n}{zeros}\PYG{p}{(}\PYG{p}{(}\PYG{n+nb}{len}\PYG{p}{(}\PYG{n}{B}\PYG{p}{)}\PYG{p}{,}\PYG{n+nb}{len}\PYG{p}{(}\PYG{n}{B}\PYG{p}{)}\PYG{p}{)}\PYG{p}{)}

\PYG{k}{for} \PYG{n}{k} \PYG{o+ow}{in} \PYG{n+nb}{range}\PYG{p}{(}\PYG{n+nb}{len}\PYG{p}{(}\PYG{n}{B}\PYG{p}{)}\PYG{p}{)}\PYG{p}{:}
    \PYG{k}{for} \PYG{n}{i} \PYG{o+ow}{in} \PYG{n+nb}{range}\PYG{p}{(}\PYG{n}{k}\PYG{p}{)}\PYG{p}{:}
        \PYG{n}{s1} \PYG{o}{=} \PYG{l+m+mi}{0}
        \PYG{k}{for} \PYG{n}{j} \PYG{o+ow}{in} \PYG{n+nb}{range}\PYG{p}{(}\PYG{n}{i}\PYG{p}{)}\PYG{p}{:}
            \PYG{n}{s1} \PYG{o}{+}\PYG{o}{=} \PYG{n}{G}\PYG{p}{[}\PYG{n}{i}\PYG{p}{,}\PYG{n}{j}\PYG{p}{]}\PYG{o}{*}\PYG{n}{G}\PYG{p}{[}\PYG{n}{k}\PYG{p}{,}\PYG{n}{j}\PYG{p}{]}
        \PYG{n}{G}\PYG{p}{[}\PYG{n}{k}\PYG{p}{,}\PYG{n}{i}\PYG{p}{]} \PYG{o}{=} \PYG{p}{(}\PYG{n}{B}\PYG{p}{[}\PYG{n}{k}\PYG{p}{,}\PYG{n}{i}\PYG{p}{]} \PYG{o}{\PYGZhy{}} \PYG{n}{s1}\PYG{p}{)}\PYG{o}{/}\PYG{n}{G}\PYG{p}{[}\PYG{n}{i}\PYG{p}{,}\PYG{n}{i}\PYG{p}{]}
    \PYG{n}{s2} \PYG{o}{=} \PYG{l+m+mi}{0}
    \PYG{k}{for} \PYG{n}{j} \PYG{o+ow}{in} \PYG{n+nb}{range}\PYG{p}{(}\PYG{n}{k}\PYG{p}{)}\PYG{p}{:}
        \PYG{n}{s2} \PYG{o}{+}\PYG{o}{=} \PYG{p}{(}\PYG{n}{G}\PYG{p}{[}\PYG{n}{k}\PYG{p}{,}\PYG{n}{j}\PYG{p}{]}\PYG{p}{)}\PYG{o}{*}\PYG{o}{*}\PYG{l+m+mi}{2}
    \PYG{n}{G}\PYG{p}{[}\PYG{n}{k}\PYG{p}{,}\PYG{n}{k}\PYG{p}{]} \PYG{o}{=} \PYG{n}{np}\PYG{o}{.}\PYG{n}{sqrt}\PYG{p}{(}\PYG{n}{B}\PYG{p}{[}\PYG{n}{k}\PYG{p}{,}\PYG{n}{k}\PYG{p}{]} \PYG{o}{\PYGZhy{}} \PYG{n}{s2}\PYG{p}{)}

\PYG{n}{G} \PYG{o}{=} \PYG{n}{np}\PYG{o}{.}\PYG{n}{around}\PYG{p}{(}\PYG{n}{G}\PYG{p}{,} \PYG{n}{decimals}\PYG{o}{=}\PYG{l+m+mi}{3}\PYG{p}{)}

\PYG{n+nb}{print}\PYG{p}{(}\PYG{l+s+s2}{\PYGZdq{}}\PYG{l+s+s2}{A matriz G é:}\PYG{l+s+se}{\PYGZbs{}n}\PYG{l+s+s2}{\PYGZdq{}}\PYG{p}{,} \PYG{n}{G}\PYG{p}{)}

\PYG{c+c1}{\PYGZsh{} (c)}
\PYG{n+nb}{print}\PYG{p}{(}\PYG{l+s+s2}{\PYGZdq{}}\PYG{l+s+se}{\PYGZbs{}n}\PYG{l+s+s2}{(c)}\PYG{l+s+s2}{\PYGZdq{}}\PYG{p}{)}
\PYG{n}{C} \PYG{o}{=} \PYG{n}{np}\PYG{o}{.}\PYG{n}{array}\PYG{p}{(}\PYG{p}{[}\PYG{p}{[}\PYG{l+m+mf}{4.}\PYG{p}{,} \PYG{l+m+mf}{1.}\PYG{p}{,} \PYG{o}{\PYGZhy{}}\PYG{l+m+mf}{1.}\PYG{p}{,} \PYG{l+m+mf}{0.}\PYG{p}{]}\PYG{p}{,} \PYG{p}{[}\PYG{l+m+mf}{1.}\PYG{p}{,} \PYG{l+m+mf}{3.}\PYG{p}{,} \PYG{o}{\PYGZhy{}}\PYG{l+m+mf}{1.}\PYG{p}{,} \PYG{l+m+mf}{0.}\PYG{p}{]}\PYG{p}{,} \PYG{p}{[}\PYG{o}{\PYGZhy{}}\PYG{l+m+mf}{1.}\PYG{p}{,} \PYG{o}{\PYGZhy{}}\PYG{l+m+mf}{1.}\PYG{p}{,} \PYG{l+m+mf}{5.}\PYG{p}{,} \PYG{l+m+mf}{2.}\PYG{p}{]}\PYG{p}{,} \PYG{p}{[}\PYG{l+m+mf}{0.}\PYG{p}{,} \PYG{l+m+mf}{0.}\PYG{p}{,} \PYG{l+m+mf}{2.}\PYG{p}{,} \PYG{l+m+mf}{4.}\PYG{p}{]}\PYG{p}{]}\PYG{p}{)}
\PYG{n}{G} \PYG{o}{=} \PYG{n}{np}\PYG{o}{.}\PYG{n}{zeros}\PYG{p}{(}\PYG{p}{(}\PYG{n+nb}{len}\PYG{p}{(}\PYG{n}{C}\PYG{p}{)}\PYG{p}{,}\PYG{n+nb}{len}\PYG{p}{(}\PYG{n}{C}\PYG{p}{)}\PYG{p}{)}\PYG{p}{)}

\PYG{k}{for} \PYG{n}{k} \PYG{o+ow}{in} \PYG{n+nb}{range}\PYG{p}{(}\PYG{n+nb}{len}\PYG{p}{(}\PYG{n}{C}\PYG{p}{)}\PYG{p}{)}\PYG{p}{:}
    \PYG{k}{for} \PYG{n}{i} \PYG{o+ow}{in} \PYG{n+nb}{range}\PYG{p}{(}\PYG{n}{k}\PYG{p}{)}\PYG{p}{:}
        \PYG{n}{s1} \PYG{o}{=} \PYG{l+m+mi}{0}
        \PYG{k}{for} \PYG{n}{j} \PYG{o+ow}{in} \PYG{n+nb}{range}\PYG{p}{(}\PYG{n}{i}\PYG{p}{)}\PYG{p}{:}
            \PYG{n}{s1} \PYG{o}{+}\PYG{o}{=} \PYG{n}{G}\PYG{p}{[}\PYG{n}{i}\PYG{p}{,}\PYG{n}{j}\PYG{p}{]}\PYG{o}{*}\PYG{n}{G}\PYG{p}{[}\PYG{n}{k}\PYG{p}{,}\PYG{n}{j}\PYG{p}{]}
        \PYG{n}{G}\PYG{p}{[}\PYG{n}{k}\PYG{p}{,}\PYG{n}{i}\PYG{p}{]} \PYG{o}{=} \PYG{p}{(}\PYG{n}{C}\PYG{p}{[}\PYG{n}{k}\PYG{p}{,}\PYG{n}{i}\PYG{p}{]} \PYG{o}{\PYGZhy{}} \PYG{n}{s1}\PYG{p}{)}\PYG{o}{/}\PYG{n}{G}\PYG{p}{[}\PYG{n}{i}\PYG{p}{,}\PYG{n}{i}\PYG{p}{]}
    \PYG{n}{s2} \PYG{o}{=} \PYG{l+m+mi}{0}
    \PYG{k}{for} \PYG{n}{j} \PYG{o+ow}{in} \PYG{n+nb}{range}\PYG{p}{(}\PYG{n}{k}\PYG{p}{)}\PYG{p}{:}
        \PYG{n}{s2} \PYG{o}{+}\PYG{o}{=} \PYG{p}{(}\PYG{n}{G}\PYG{p}{[}\PYG{n}{k}\PYG{p}{,}\PYG{n}{j}\PYG{p}{]}\PYG{p}{)}\PYG{o}{*}\PYG{o}{*}\PYG{l+m+mi}{2}
    \PYG{n}{G}\PYG{p}{[}\PYG{n}{k}\PYG{p}{,}\PYG{n}{k}\PYG{p}{]} \PYG{o}{=} \PYG{n}{np}\PYG{o}{.}\PYG{n}{sqrt}\PYG{p}{(}\PYG{n}{C}\PYG{p}{[}\PYG{n}{k}\PYG{p}{,}\PYG{n}{k}\PYG{p}{]} \PYG{o}{\PYGZhy{}} \PYG{n}{s2}\PYG{p}{)}

\PYG{n}{G} \PYG{o}{=} \PYG{n}{np}\PYG{o}{.}\PYG{n}{around}\PYG{p}{(}\PYG{n}{G}\PYG{p}{,} \PYG{n}{decimals}\PYG{o}{=}\PYG{l+m+mi}{3}\PYG{p}{)}

\PYG{n+nb}{print}\PYG{p}{(}\PYG{l+s+s2}{\PYGZdq{}}\PYG{l+s+s2}{A matriz G é:}\PYG{l+s+se}{\PYGZbs{}n}\PYG{l+s+s2}{\PYGZdq{}}\PYG{p}{,} \PYG{n}{G}\PYG{p}{)}
\end{sphinxVerbatim}

\begin{sphinxVerbatim}[commandchars=\\\{\}]
(a)
A matriz G é:
 [[ 1.414  0.     0.   ]
 [\PYGZhy{}0.707  1.225  0.   ]
 [ 0.    \PYGZhy{}0.816  1.155]]

(b)
A matriz G é:
 [[ 2.     0.     0.     0.   ]
 [ 0.5    1.658  0.     0.   ]
 [ 0.5   \PYGZhy{}0.754  1.087  0.   ]
 [ 0.5    0.452  0.084  1.24 ]]

(c)
A matriz G é:
 [[ 2.     0.     0.     0.   ]
 [ 0.5    1.658  0.     0.   ]
 [\PYGZhy{}0.5   \PYGZhy{}0.452  2.132  0.   ]
 [ 0.     0.     0.938  1.766]]
\end{sphinxVerbatim}


\subsection{Questão 14}
\label{\detokenize{lista-3-solucoes:questao-14}}
\sphinxAtStartPar
Resolva os seguintes sistemas de equações por decomposição de Cholesky.

\sphinxAtStartPar
(a) \textbackslash{}begin\{cases\}
8x\_1 + 20x\_2 + 15x\_3 = 50 \textbackslash{}
20x\_1 + 80x\_2 + 50x\_3 = 250 \textbackslash{}
15x\_1 + 50x\_2 + 60x\_3 = 100
\textbackslash{}end\{cases\}

\sphinxAtStartPar
(b) \textbackslash{}begin\{cases\}
6x\_1 + 15x\_2 + 55x\_3 = 152.6 \textbackslash{}
15x\_1 + 55x\_2 + 225x\_3 = 585.6 \textbackslash{}
55x\_1 + 225x\_2 + 979x\_3 = 2488.8
\textbackslash{}end\{cases\}

\sphinxAtStartPar
©
\textbackslash{}begin\{cases\}
2x\_1 − x\_2 = 3 \textbackslash{}
−x\_1 + 2x\_2 − x\_3 = − 3 \textbackslash{}
−x\_2 + 2x\_3 = 1
\textbackslash{}end\{cases\}

\sphinxAtStartPar
(d)
\textbackslash{}begin\{cases\}
4x\_1 + x\_2 + x\_3 + x\_4 = 0.65 \textbackslash{}
x\_1 + 3x\_2 − x\_3 + x\_4 = 0.05 \textbackslash{}
x\_1 − x\_2 + 2x\_3 = 0 \textbackslash{}
x\_1 + x\_2 + 2x\_4 = 0.5
\textbackslash{}end\{cases\}

\begin{sphinxVerbatim}[commandchars=\\\{\}]
\PYG{c+c1}{\PYGZsh{} Solução}

\PYG{c+c1}{\PYGZsh{} (a)}
\PYG{n+nb}{print}\PYG{p}{(}\PYG{l+s+s2}{\PYGZdq{}}\PYG{l+s+s2}{(a)}\PYG{l+s+s2}{\PYGZdq{}}\PYG{p}{)}
\PYG{n}{AB} \PYG{o}{=} \PYG{n}{np}\PYG{o}{.}\PYG{n}{array}\PYG{p}{(}\PYG{p}{[}\PYG{p}{[}\PYG{l+m+mf}{8.}\PYG{p}{,} \PYG{l+m+mf}{20.}\PYG{p}{,} \PYG{l+m+mf}{15.}\PYG{p}{,} \PYG{l+m+mf}{50.}\PYG{p}{]}\PYG{p}{,} \PYG{p}{[}\PYG{l+m+mf}{20.}\PYG{p}{,} \PYG{l+m+mf}{80.}\PYG{p}{,} \PYG{l+m+mf}{50.}\PYG{p}{,} \PYG{l+m+mf}{250.}\PYG{p}{]}\PYG{p}{,} \PYG{p}{[}\PYG{l+m+mf}{15.}\PYG{p}{,} \PYG{l+m+mf}{50.}\PYG{p}{,} \PYG{l+m+mf}{60.}\PYG{p}{,} \PYG{l+m+mf}{100.}\PYG{p}{]}\PYG{p}{]}\PYG{p}{)}
\PYG{n}{G}\PYG{p}{,} \PYG{n}{X} \PYG{o}{=} \PYG{n}{cholesky}\PYG{p}{(}\PYG{n}{AB}\PYG{p}{)}
\PYG{n+nb}{print}\PYG{p}{(}\PYG{l+s+s2}{\PYGZdq{}}\PYG{l+s+s2}{A solução do sistema é: }\PYG{l+s+se}{\PYGZbs{}n}\PYG{l+s+s2}{\PYGZdq{}}\PYG{p}{,} \PYG{n}{G}\PYG{p}{)}

\PYG{c+c1}{\PYGZsh{} (b)}
\PYG{n+nb}{print}\PYG{p}{(}\PYG{l+s+s2}{\PYGZdq{}}\PYG{l+s+se}{\PYGZbs{}n}\PYG{l+s+s2}{(b)}\PYG{l+s+s2}{\PYGZdq{}}\PYG{p}{)}
\PYG{n}{AB} \PYG{o}{=} \PYG{n}{np}\PYG{o}{.}\PYG{n}{array}\PYG{p}{(}\PYG{p}{[}\PYG{p}{[}\PYG{l+m+mf}{6.}\PYG{p}{,} \PYG{l+m+mf}{15.}\PYG{p}{,} \PYG{l+m+mf}{55.}\PYG{p}{,} \PYG{l+m+mf}{152.6}\PYG{p}{]}\PYG{p}{,} \PYG{p}{[}\PYG{l+m+mf}{15.}\PYG{p}{,} \PYG{l+m+mf}{55.}\PYG{p}{,} \PYG{l+m+mf}{225.}\PYG{p}{,} \PYG{l+m+mf}{585.6}\PYG{p}{]}\PYG{p}{,} \PYG{p}{[}\PYG{l+m+mf}{55.}\PYG{p}{,} \PYG{l+m+mf}{225.}\PYG{p}{,} \PYG{l+m+mf}{979.}\PYG{p}{,} \PYG{l+m+mf}{2488.8}\PYG{p}{]}\PYG{p}{]}\PYG{p}{)}
\PYG{n}{G}\PYG{p}{,} \PYG{n}{X} \PYG{o}{=} \PYG{n}{cholesky}\PYG{p}{(}\PYG{n}{AB}\PYG{p}{)}
\PYG{n+nb}{print}\PYG{p}{(}\PYG{l+s+s2}{\PYGZdq{}}\PYG{l+s+s2}{A solução do sistema é: }\PYG{l+s+se}{\PYGZbs{}n}\PYG{l+s+s2}{\PYGZdq{}}\PYG{p}{,} \PYG{n}{G}\PYG{p}{)}

\PYG{c+c1}{\PYGZsh{} (c)}
\PYG{n+nb}{print}\PYG{p}{(}\PYG{l+s+s2}{\PYGZdq{}}\PYG{l+s+se}{\PYGZbs{}n}\PYG{l+s+s2}{(c)}\PYG{l+s+s2}{\PYGZdq{}}\PYG{p}{)}
\PYG{n}{AB} \PYG{o}{=} \PYG{n}{np}\PYG{o}{.}\PYG{n}{array}\PYG{p}{(}\PYG{p}{[}\PYG{p}{[}\PYG{l+m+mf}{2.}\PYG{p}{,} \PYG{o}{\PYGZhy{}}\PYG{l+m+mf}{1.}\PYG{p}{,} \PYG{l+m+mf}{0.}\PYG{p}{,} \PYG{l+m+mf}{3.}\PYG{p}{]}\PYG{p}{,} \PYG{p}{[}\PYG{o}{\PYGZhy{}}\PYG{l+m+mf}{1.}\PYG{p}{,} \PYG{l+m+mf}{2.}\PYG{p}{,} \PYG{o}{\PYGZhy{}}\PYG{l+m+mf}{1.}\PYG{p}{,} \PYG{o}{\PYGZhy{}}\PYG{l+m+mf}{3.}\PYG{p}{]}\PYG{p}{,} \PYG{p}{[}\PYG{l+m+mf}{0.}\PYG{p}{,} \PYG{o}{\PYGZhy{}}\PYG{l+m+mf}{1.}\PYG{p}{,} \PYG{l+m+mf}{2.}\PYG{p}{,} \PYG{l+m+mf}{1.}\PYG{p}{]}\PYG{p}{]}\PYG{p}{)}
\PYG{n}{G}\PYG{p}{,} \PYG{n}{X} \PYG{o}{=} \PYG{n}{cholesky}\PYG{p}{(}\PYG{n}{AB}\PYG{p}{)}
\PYG{n+nb}{print}\PYG{p}{(}\PYG{l+s+s2}{\PYGZdq{}}\PYG{l+s+s2}{A solução do sistema é: }\PYG{l+s+se}{\PYGZbs{}n}\PYG{l+s+s2}{\PYGZdq{}}\PYG{p}{,} \PYG{n}{G}\PYG{p}{)}

\PYG{c+c1}{\PYGZsh{} (d)}
\PYG{n+nb}{print}\PYG{p}{(}\PYG{l+s+s2}{\PYGZdq{}}\PYG{l+s+se}{\PYGZbs{}n}\PYG{l+s+s2}{(d)}\PYG{l+s+s2}{\PYGZdq{}}\PYG{p}{)}
\PYG{n}{AB} \PYG{o}{=} \PYG{n}{np}\PYG{o}{.}\PYG{n}{array}\PYG{p}{(}\PYG{p}{[}\PYG{p}{[}\PYG{l+m+mf}{4.}\PYG{p}{,} \PYG{l+m+mf}{1.}\PYG{p}{,} \PYG{l+m+mf}{1.}\PYG{p}{,} \PYG{l+m+mf}{1.}\PYG{p}{,} \PYG{l+m+mf}{0.65}\PYG{p}{]}\PYG{p}{,} \PYG{p}{[}\PYG{l+m+mf}{1.}\PYG{p}{,} \PYG{l+m+mf}{3.}\PYG{p}{,} \PYG{o}{\PYGZhy{}}\PYG{l+m+mf}{1.}\PYG{p}{,} \PYG{l+m+mf}{1.}\PYG{p}{,} \PYG{l+m+mf}{0.05}\PYG{p}{]}\PYG{p}{,} \PYG{p}{[}\PYG{l+m+mf}{1.}\PYG{p}{,} \PYG{o}{\PYGZhy{}}\PYG{l+m+mf}{1.}\PYG{p}{,} \PYG{l+m+mf}{2.}\PYG{p}{,} \PYG{l+m+mf}{0.}\PYG{p}{,} \PYG{l+m+mf}{0.}\PYG{p}{]}\PYG{p}{,} \PYG{p}{[}\PYG{l+m+mf}{1.}\PYG{p}{,} \PYG{l+m+mf}{1.}\PYG{p}{,} \PYG{l+m+mf}{0.}\PYG{p}{,} \PYG{l+m+mf}{2.}\PYG{p}{,} \PYG{l+m+mf}{0.5}\PYG{p}{]}\PYG{p}{]}\PYG{p}{)}
\PYG{n}{G}\PYG{p}{,} \PYG{n}{X} \PYG{o}{=} \PYG{n}{cholesky}\PYG{p}{(}\PYG{n}{AB}\PYG{p}{)}
\PYG{n+nb}{print}\PYG{p}{(}\PYG{l+s+s2}{\PYGZdq{}}\PYG{l+s+s2}{A solução do sistema é: }\PYG{l+s+se}{\PYGZbs{}n}\PYG{l+s+s2}{\PYGZdq{}}\PYG{p}{,} \PYG{n}{G}\PYG{p}{)}
\end{sphinxVerbatim}

\begin{sphinxVerbatim}[commandchars=\\\{\}]
(a)
A solução do sistema é: 
 [[2.828 0.    0.   ]
 [7.071 5.477 0.   ]
 [5.303 2.282 5.164]]

(b)
A solução do sistema é: 
 [[ 2.449  0.     0.   ]
 [ 6.124  4.183  0.   ]
 [22.454 20.917  6.11 ]]

(c)
A solução do sistema é: 
 [[ 1.414  0.     0.   ]
 [\PYGZhy{}0.707  1.225  0.   ]
 [ 0.    \PYGZhy{}0.816  1.155]]

(d)
A solução do sistema é: 
 [[ 2.     0.     0.     0.   ]
 [ 0.5    1.658  0.     0.   ]
 [ 0.5   \PYGZhy{}0.754  1.087  0.   ]
 [ 0.5    0.452  0.084  1.24 ]]
\end{sphinxVerbatim}


\chapter{Lista de Exercícios 4}
\label{\detokenize{lista-4-solucoes:lista-de-exercicios-4}}\label{\detokenize{lista-4-solucoes::doc}}
\sphinxAtStartPar
Solucionário matemático e computacional de exercícios selecionados da Lista de Exercícios 4.

\begin{sphinxVerbatim}[commandchars=\\\{\}]
\PYG{o}{\PYGZpc{}}\PYG{k}{matplotlib} inline
\end{sphinxVerbatim}

\begin{sphinxVerbatim}[commandchars=\\\{\}]
\PYG{k+kn}{import} \PYG{n+nn}{numpy} \PYG{k}{as} \PYG{n+nn}{np}
\PYG{k+kn}{import} \PYG{n+nn}{sympy} \PYG{k}{as} \PYG{n+nn}{sy}
\PYG{k+kn}{import} \PYG{n+nn}{matplotlib}\PYG{n+nn}{.}\PYG{n+nn}{pyplot} \PYG{k}{as} \PYG{n+nn}{plt}
\end{sphinxVerbatim}

\begin{sphinxVerbatim}[commandchars=\\\{\}]
\PYG{c+c1}{\PYGZsh{} Funções Implementadas}

\PYG{c+c1}{\PYGZsh{} Solução de Sistemas}

\PYG{k}{def} \PYG{n+nf}{gaussjacobi}\PYG{p}{(}\PYG{n}{AB}\PYG{p}{,}\PYG{n}{ER}\PYG{p}{,}\PYG{n}{X0}\PYG{p}{)}\PYG{p}{:}
    \PYG{l+s+sd}{\PYGZsq{}\PYGZsq{}\PYGZsq{}Realiza o cálculo de um sistema linear através do método de iterativo de Gauss\PYGZhy{}Jacobi.}
\PYG{l+s+sd}{    }
\PYG{l+s+sd}{    Sinopse:}
\PYG{l+s+sd}{         X = gaussjacobi(AB,ER,X0)}
\PYG{l+s+sd}{    }
\PYG{l+s+sd}{    Entradas:}
\PYG{l+s+sd}{         AB \PYGZhy{} Matriz aumentada (np.array) do sistema linear}
\PYG{l+s+sd}{         ER \PYGZhy{} Erro relativo (forma decimal) entre a iteração i e a iteração i\PYGZhy{}1}
\PYG{l+s+sd}{         X0 \PYGZhy{} Vetor estimativa inicial da solução}
\PYG{l+s+sd}{     }
\PYG{l+s+sd}{    Saídas:}
\PYG{l+s+sd}{         X \PYGZhy{} Vetor solução do sistema linear}
\PYG{l+s+sd}{         }
\PYG{l+s+sd}{         @ney}
\PYG{l+s+sd}{    \PYGZsq{}\PYGZsq{}\PYGZsq{}}
    
    \PYG{n}{A} \PYG{o}{=} \PYG{n}{AB}\PYG{p}{[}\PYG{p}{:}\PYG{p}{,}\PYG{l+m+mi}{0}\PYG{p}{:}\PYG{o}{\PYGZhy{}}\PYG{l+m+mi}{1}\PYG{p}{]}
    \PYG{n}{B} \PYG{o}{=} \PYG{n}{AB}\PYG{p}{[}\PYG{p}{:}\PYG{p}{,}\PYG{o}{\PYGZhy{}}\PYG{l+m+mi}{1}\PYG{p}{]}
    \PYG{n}{X} \PYG{o}{=} \PYG{n}{X0}
    \PYG{n}{erro} \PYG{o}{=} \PYG{l+m+mi}{1}
    
    \PYG{k}{while} \PYG{p}{(}\PYG{n}{erro} \PYG{o}{\PYGZgt{}} \PYG{n}{ER}\PYG{p}{)}\PYG{p}{:}
        \PYG{n}{Xp} \PYG{o}{=} \PYG{n}{np}\PYG{o}{.}\PYG{n}{copy}\PYG{p}{(}\PYG{n}{X}\PYG{p}{)}
        \PYG{k}{for} \PYG{n}{i} \PYG{o+ow}{in} \PYG{n+nb}{range}\PYG{p}{(}\PYG{n+nb}{len}\PYG{p}{(}\PYG{n}{A}\PYG{p}{)}\PYG{p}{)}\PYG{p}{:}
            \PYG{n}{s} \PYG{o}{=} \PYG{l+m+mi}{0}
            \PYG{k}{for} \PYG{n}{j} \PYG{o+ow}{in} \PYG{n+nb}{range}\PYG{p}{(}\PYG{n+nb}{len}\PYG{p}{(}\PYG{n}{A}\PYG{p}{)}\PYG{p}{)}\PYG{p}{:}
                \PYG{k}{if} \PYG{p}{(}\PYG{n}{i} \PYG{o}{!=} \PYG{n}{j}\PYG{p}{)}\PYG{p}{:}
                    \PYG{n}{s} \PYG{o}{+}\PYG{o}{=} \PYG{n}{A}\PYG{p}{[}\PYG{n}{i}\PYG{p}{,}\PYG{n}{j}\PYG{p}{]}\PYG{o}{*}\PYG{n}{Xp}\PYG{p}{[}\PYG{n}{j}\PYG{p}{]}
            \PYG{n}{X}\PYG{p}{[}\PYG{n}{i}\PYG{p}{]} \PYG{o}{=} \PYG{p}{(}\PYG{n}{B}\PYG{p}{[}\PYG{n}{i}\PYG{p}{]} \PYG{o}{\PYGZhy{}} \PYG{n}{s}\PYG{p}{)}\PYG{o}{/}\PYG{n}{A}\PYG{p}{[}\PYG{n}{i}\PYG{p}{,}\PYG{n}{i}\PYG{p}{]}
        \PYG{n}{EA} \PYG{o}{=} \PYG{n}{np}\PYG{o}{.}\PYG{n}{absolute}\PYG{p}{(}\PYG{n}{X} \PYG{o}{\PYGZhy{}} \PYG{n}{Xp}\PYG{p}{)}
        \PYG{n}{Emax} \PYG{o}{=} \PYG{n}{np}\PYG{o}{.}\PYG{n}{amax}\PYG{p}{(}\PYG{n}{EA}\PYG{p}{)}
        \PYG{n}{idx} \PYG{o}{=} \PYG{n}{np}\PYG{o}{.}\PYG{n}{where}\PYG{p}{(}\PYG{n}{EA} \PYG{o}{==} \PYG{n}{Emax}\PYG{p}{)}
        
        \PYG{k}{if} \PYG{p}{(}\PYG{n+nb}{len}\PYG{p}{(}\PYG{n}{idx}\PYG{p}{[}\PYG{l+m+mi}{0}\PYG{p}{]}\PYG{p}{)} \PYG{o}{\PYGZgt{}} \PYG{l+m+mi}{1}\PYG{p}{)}\PYG{p}{:}
            \PYG{n}{erro} \PYG{o}{=} \PYG{n}{np}\PYG{o}{.}\PYG{n}{absolute}\PYG{p}{(}\PYG{n}{EA}\PYG{p}{[}\PYG{n}{idx}\PYG{p}{[}\PYG{l+m+mi}{0}\PYG{p}{]}\PYG{p}{[}\PYG{l+m+mi}{0}\PYG{p}{]}\PYG{p}{]}\PYG{o}{/}\PYG{n}{X}\PYG{p}{[}\PYG{n}{idx}\PYG{p}{[}\PYG{l+m+mi}{0}\PYG{p}{]}\PYG{p}{[}\PYG{l+m+mi}{0}\PYG{p}{]}\PYG{p}{]}\PYG{p}{)}
        \PYG{k}{else}\PYG{p}{:}
            \PYG{n}{erro} \PYG{o}{=} \PYG{n}{np}\PYG{o}{.}\PYG{n}{absolute}\PYG{p}{(}\PYG{n}{EA}\PYG{p}{[}\PYG{n}{idx}\PYG{p}{[}\PYG{l+m+mi}{0}\PYG{p}{]}\PYG{p}{]}\PYG{o}{/}\PYG{n}{X}\PYG{p}{[}\PYG{n}{idx}\PYG{p}{[}\PYG{l+m+mi}{0}\PYG{p}{]}\PYG{p}{]}\PYG{p}{)}
    
    \PYG{n}{X} \PYG{o}{=} \PYG{n}{np}\PYG{o}{.}\PYG{n}{around}\PYG{p}{(}\PYG{n}{X}\PYG{p}{,} \PYG{n}{decimals}\PYG{o}{=}\PYG{l+m+mi}{3}\PYG{p}{)}

    \PYG{k}{return} \PYG{n}{X}

\PYG{k}{def} \PYG{n+nf}{gaussseidel}\PYG{p}{(}\PYG{n}{AB}\PYG{p}{,}\PYG{n}{ER}\PYG{p}{,}\PYG{n}{X0}\PYG{p}{)}\PYG{p}{:}
    \PYG{l+s+sd}{\PYGZsq{}\PYGZsq{}\PYGZsq{} Função que realiza o cálculo de um sistema linear através do método de iterativo de Gauss\PYGZhy{}Seidel.}
\PYG{l+s+sd}{    }
\PYG{l+s+sd}{    Sinopse:}
\PYG{l+s+sd}{         X = gaussseidel(AB,ER,X0)}
\PYG{l+s+sd}{    }
\PYG{l+s+sd}{    Entradas:}
\PYG{l+s+sd}{         AB \PYGZhy{} Matriz aumentada (np.array) do sistema linear}
\PYG{l+s+sd}{         ER \PYGZhy{} Erro relativo (forma decimal) entre a iteração i e a iteração i\PYGZhy{}1}
\PYG{l+s+sd}{         X0 \PYGZhy{} Vetor estimativa inicial da solução}
\PYG{l+s+sd}{     }
\PYG{l+s+sd}{    Saídas:}
\PYG{l+s+sd}{         X \PYGZhy{} Vetor solução do sistema linear}
\PYG{l+s+sd}{         }
\PYG{l+s+sd}{         @ney}
\PYG{l+s+sd}{    \PYGZsq{}\PYGZsq{}\PYGZsq{}}
    
    \PYG{n}{A} \PYG{o}{=} \PYG{n}{AB}\PYG{p}{[}\PYG{p}{:}\PYG{p}{,}\PYG{l+m+mi}{0}\PYG{p}{:}\PYG{o}{\PYGZhy{}}\PYG{l+m+mi}{1}\PYG{p}{]}
    \PYG{n}{B} \PYG{o}{=} \PYG{n}{AB}\PYG{p}{[}\PYG{p}{:}\PYG{p}{,}\PYG{o}{\PYGZhy{}}\PYG{l+m+mi}{1}\PYG{p}{]}
    \PYG{n}{X} \PYG{o}{=} \PYG{n}{X0}
    \PYG{n}{erro} \PYG{o}{=} \PYG{l+m+mi}{1}
    
    \PYG{k}{while} \PYG{p}{(}\PYG{n}{erro} \PYG{o}{\PYGZgt{}} \PYG{n}{ER}\PYG{p}{)}\PYG{p}{:}
        \PYG{n}{Xp} \PYG{o}{=} \PYG{n}{np}\PYG{o}{.}\PYG{n}{copy}\PYG{p}{(}\PYG{n}{X}\PYG{p}{)}
        \PYG{k}{for} \PYG{n}{i} \PYG{o+ow}{in} \PYG{n+nb}{range}\PYG{p}{(}\PYG{n+nb}{len}\PYG{p}{(}\PYG{n}{A}\PYG{p}{)}\PYG{p}{)}\PYG{p}{:}
            \PYG{n}{s} \PYG{o}{=} \PYG{l+m+mi}{0}
            \PYG{k}{for} \PYG{n}{j} \PYG{o+ow}{in} \PYG{n+nb}{range}\PYG{p}{(}\PYG{n+nb}{len}\PYG{p}{(}\PYG{n}{A}\PYG{p}{)}\PYG{p}{)}\PYG{p}{:}
                \PYG{k}{if} \PYG{p}{(}\PYG{n}{i} \PYG{o}{!=} \PYG{n}{j}\PYG{p}{)}\PYG{p}{:}
                    \PYG{n}{s} \PYG{o}{+}\PYG{o}{=} \PYG{n}{A}\PYG{p}{[}\PYG{n}{i}\PYG{p}{,}\PYG{n}{j}\PYG{p}{]}\PYG{o}{*}\PYG{n}{X}\PYG{p}{[}\PYG{n}{j}\PYG{p}{]}
            \PYG{n}{X}\PYG{p}{[}\PYG{n}{i}\PYG{p}{]} \PYG{o}{=} \PYG{p}{(}\PYG{n}{B}\PYG{p}{[}\PYG{n}{i}\PYG{p}{]} \PYG{o}{\PYGZhy{}} \PYG{n}{s}\PYG{p}{)}\PYG{o}{/}\PYG{n}{A}\PYG{p}{[}\PYG{n}{i}\PYG{p}{,}\PYG{n}{i}\PYG{p}{]}
        \PYG{n}{EA} \PYG{o}{=} \PYG{n}{np}\PYG{o}{.}\PYG{n}{absolute}\PYG{p}{(}\PYG{n}{X} \PYG{o}{\PYGZhy{}} \PYG{n}{Xp}\PYG{p}{)}
        \PYG{n}{Emax} \PYG{o}{=} \PYG{n}{np}\PYG{o}{.}\PYG{n}{amax}\PYG{p}{(}\PYG{n}{EA}\PYG{p}{)}
        \PYG{n}{idx} \PYG{o}{=} \PYG{n}{np}\PYG{o}{.}\PYG{n}{where}\PYG{p}{(}\PYG{n}{EA} \PYG{o}{==} \PYG{n}{Emax}\PYG{p}{)}
        
        \PYG{k}{if} \PYG{p}{(}\PYG{n+nb}{len}\PYG{p}{(}\PYG{n}{idx}\PYG{p}{[}\PYG{l+m+mi}{0}\PYG{p}{]}\PYG{p}{)} \PYG{o}{\PYGZgt{}} \PYG{l+m+mi}{1}\PYG{p}{)}\PYG{p}{:}
            \PYG{n}{erro} \PYG{o}{=} \PYG{n}{np}\PYG{o}{.}\PYG{n}{absolute}\PYG{p}{(}\PYG{n}{EA}\PYG{p}{[}\PYG{n}{idx}\PYG{p}{[}\PYG{l+m+mi}{0}\PYG{p}{]}\PYG{p}{[}\PYG{l+m+mi}{0}\PYG{p}{]}\PYG{p}{]}\PYG{o}{/}\PYG{n}{X}\PYG{p}{[}\PYG{n}{idx}\PYG{p}{[}\PYG{l+m+mi}{0}\PYG{p}{]}\PYG{p}{[}\PYG{l+m+mi}{0}\PYG{p}{]}\PYG{p}{]}\PYG{p}{)}
        \PYG{k}{else}\PYG{p}{:}
            \PYG{n}{erro} \PYG{o}{=} \PYG{n}{np}\PYG{o}{.}\PYG{n}{absolute}\PYG{p}{(}\PYG{n}{EA}\PYG{p}{[}\PYG{n}{idx}\PYG{p}{[}\PYG{l+m+mi}{0}\PYG{p}{]}\PYG{p}{]}\PYG{o}{/}\PYG{n}{X}\PYG{p}{[}\PYG{n}{idx}\PYG{p}{[}\PYG{l+m+mi}{0}\PYG{p}{]}\PYG{p}{]}\PYG{p}{)}
    
    \PYG{n}{X} \PYG{o}{=} \PYG{n}{np}\PYG{o}{.}\PYG{n}{around}\PYG{p}{(}\PYG{n}{X}\PYG{p}{,} \PYG{n}{decimals}\PYG{o}{=}\PYG{l+m+mi}{3}\PYG{p}{)}
    
    \PYG{k}{return} \PYG{n}{X}

\PYG{k}{def} \PYG{n+nf}{jacobian}\PYG{p}{(}\PYG{n}{F}\PYG{p}{,}\PYG{n}{Xs}\PYG{p}{)}\PYG{p}{:}
    \PYG{l+s+sd}{\PYGZsq{}\PYGZsq{}\PYGZsq{}Função que realiza o cálculo da matriz jacobiana.}
\PYG{l+s+sd}{    }
\PYG{l+s+sd}{    Sinopse:}
\PYG{l+s+sd}{         J = jacobian(F,Xs)}
\PYG{l+s+sd}{    }
\PYG{l+s+sd}{    Entradas:}
\PYG{l+s+sd}{         F \PYGZhy{} Vetor contendo as equações (simbólicas) do sistema não\PYGZhy{}linear}
\PYG{l+s+sd}{         Xs \PYGZhy{} Vetor contendo as variáveis (simbólicas) do sistema não\PYGZhy{}linear}
\PYG{l+s+sd}{    }
\PYG{l+s+sd}{    Saídas:}
\PYG{l+s+sd}{         J \PYGZhy{} Matriz jacobiana (simbólica)}
\PYG{l+s+sd}{         }
\PYG{l+s+sd}{         @ney}
\PYG{l+s+sd}{    \PYGZsq{}\PYGZsq{}\PYGZsq{}}
    \PYG{n}{J} \PYG{o}{=} \PYG{n}{sy}\PYG{o}{.}\PYG{n}{zeros}\PYG{p}{(}\PYG{n+nb}{len}\PYG{p}{(}\PYG{n}{Xs}\PYG{p}{)}\PYG{p}{,} \PYG{n+nb}{len}\PYG{p}{(}\PYG{n}{Xs}\PYG{p}{)}\PYG{p}{)}
    
    \PYG{k}{for} \PYG{n}{i} \PYG{o+ow}{in} \PYG{n+nb}{range}\PYG{p}{(}\PYG{n+nb}{len}\PYG{p}{(}\PYG{n}{Xs}\PYG{p}{)}\PYG{p}{)}\PYG{p}{:}
        \PYG{k}{for} \PYG{n}{j} \PYG{o+ow}{in} \PYG{n+nb}{range}\PYG{p}{(}\PYG{n+nb}{len}\PYG{p}{(}\PYG{n}{Xs}\PYG{p}{)}\PYG{p}{)}\PYG{p}{:}
            \PYG{n}{J}\PYG{p}{[}\PYG{n}{i}\PYG{p}{,}\PYG{n}{j}\PYG{p}{]} \PYG{o}{=} \PYG{n}{sy}\PYG{o}{.}\PYG{n}{diff}\PYG{p}{(}\PYG{n}{F}\PYG{p}{[}\PYG{n}{i}\PYG{p}{]}\PYG{p}{,}\PYG{n}{Xs}\PYG{p}{[}\PYG{n}{j}\PYG{p}{]}\PYG{p}{)}
    
    \PYG{k}{return} \PYG{n}{J}

\PYG{k}{def} \PYG{n+nf}{newtonnaolin}\PYG{p}{(}\PYG{n}{F}\PYG{p}{,}\PYG{n}{Xs}\PYG{p}{,}\PYG{n}{ER}\PYG{p}{,}\PYG{n}{X0}\PYG{p}{)}\PYG{p}{:}
    \PYG{l+s+sd}{\PYGZsq{}\PYGZsq{}\PYGZsq{}Função que realiza o cálculo de um sistema não\PYGZhy{}linear através do método de iterativo de Newton.}
\PYG{l+s+sd}{    }
\PYG{l+s+sd}{    Sinopse:}
\PYG{l+s+sd}{         X = newtonnaolin(F,Xs,ER,X0)}
\PYG{l+s+sd}{    }
\PYG{l+s+sd}{    Entradas:}
\PYG{l+s+sd}{         F  \PYGZhy{} Vetor (sy.Matrix) contendo as equações (simbólicas) do sistema não\PYGZhy{}linear}
\PYG{l+s+sd}{         Xs \PYGZhy{} Vetor (sy.Matrix) contendo as variáveis (simbólico) do sistema não linear}
\PYG{l+s+sd}{         ER \PYGZhy{} Erro relativo (forma decimal) entre a iteração i e a iteração i\PYGZhy{}1}
\PYG{l+s+sd}{         X0 \PYGZhy{} Vetor estimativa inicial da solução}
\PYG{l+s+sd}{     }
\PYG{l+s+sd}{    Saídas:}
\PYG{l+s+sd}{         X \PYGZhy{} Vetor solução do sistema linear}
\PYG{l+s+sd}{         }
\PYG{l+s+sd}{         @ney}
\PYG{l+s+sd}{    \PYGZsq{}\PYGZsq{}\PYGZsq{}}
    
    \PYG{n}{J} \PYG{o}{=} \PYG{n}{jacobian}\PYG{p}{(}\PYG{n}{F}\PYG{p}{,}\PYG{n}{Xs}\PYG{p}{)}
    \PYG{n}{erro} \PYG{o}{=} \PYG{l+m+mi}{1}
    \PYG{n}{X} \PYG{o}{=} \PYG{n}{np}\PYG{o}{.}\PYG{n}{copy}\PYG{p}{(}\PYG{n}{X0}\PYG{p}{)}
    
    \PYG{k}{while} \PYG{p}{(}\PYG{n}{erro} \PYG{o}{\PYGZgt{}} \PYG{n}{ER}\PYG{p}{)}\PYG{p}{:}
        \PYG{n}{Xp} \PYG{o}{=} \PYG{n}{np}\PYG{o}{.}\PYG{n}{copy}\PYG{p}{(}\PYG{n}{X}\PYG{p}{)}
        \PYG{n}{A} \PYG{o}{=} \PYG{n}{J}\PYG{p}{[}\PYG{p}{:}\PYG{p}{,}\PYG{p}{:}\PYG{p}{]}
        \PYG{n}{B} \PYG{o}{=} \PYG{n}{F}\PYG{p}{[}\PYG{p}{:}\PYG{p}{,}\PYG{p}{:}\PYG{p}{]}
        \PYG{n}{subs} \PYG{o}{=} \PYG{p}{[}\PYG{p}{]}
        
        \PYG{k}{for} \PYG{n}{i} \PYG{o+ow}{in} \PYG{n+nb}{range}\PYG{p}{(}\PYG{n+nb}{len}\PYG{p}{(}\PYG{n}{Xs}\PYG{p}{)}\PYG{p}{)}\PYG{p}{:}
            \PYG{n}{subs}\PYG{o}{.}\PYG{n}{append}\PYG{p}{(}\PYG{p}{(}\PYG{n}{Xs}\PYG{p}{[}\PYG{n}{i}\PYG{p}{]}\PYG{p}{,}\PYG{n}{X}\PYG{p}{[}\PYG{n}{i}\PYG{p}{]}\PYG{p}{)}\PYG{p}{)}
            
        \PYG{n}{A} \PYG{o}{=} \PYG{n}{np}\PYG{o}{.}\PYG{n}{asarray}\PYG{p}{(}\PYG{n}{A}\PYG{o}{.}\PYG{n}{subs}\PYG{p}{(}\PYG{n}{subs}\PYG{p}{)}\PYG{p}{,} \PYG{n}{dtype}\PYG{o}{=}\PYG{n+nb}{float}\PYG{p}{)}
        \PYG{n}{B} \PYG{o}{=} \PYG{n}{np}\PYG{o}{.}\PYG{n}{asarray}\PYG{p}{(}\PYG{n}{B}\PYG{o}{.}\PYG{n}{subs}\PYG{p}{(}\PYG{n}{subs}\PYG{p}{)}\PYG{p}{,} \PYG{n}{dtype}\PYG{o}{=}\PYG{n+nb}{float}\PYG{p}{)}
        \PYG{n}{B} \PYG{o}{*}\PYG{o}{=} \PYG{o}{\PYGZhy{}}\PYG{l+m+mi}{1}
        
        \PYG{n}{S} \PYG{o}{=} \PYG{n}{np}\PYG{o}{.}\PYG{n}{linalg}\PYG{o}{.}\PYG{n}{solve}\PYG{p}{(}\PYG{n}{A}\PYG{p}{,}\PYG{n}{B}\PYG{p}{)}
        \PYG{n}{S} \PYG{o}{=} \PYG{n}{np}\PYG{o}{.}\PYG{n}{transpose}\PYG{p}{(}\PYG{n}{S}\PYG{p}{)}
        
        \PYG{n}{X} \PYG{o}{+}\PYG{o}{=} \PYG{n}{S}\PYG{p}{[}\PYG{l+m+mi}{0}\PYG{p}{]}
        
        \PYG{n}{EA} \PYG{o}{=} \PYG{n}{np}\PYG{o}{.}\PYG{n}{absolute}\PYG{p}{(}\PYG{n}{X} \PYG{o}{\PYGZhy{}} \PYG{n}{Xp}\PYG{p}{)}
        \PYG{n}{Emax} \PYG{o}{=} \PYG{n}{np}\PYG{o}{.}\PYG{n}{amax}\PYG{p}{(}\PYG{n}{EA}\PYG{p}{)}
        \PYG{n}{idx} \PYG{o}{=} \PYG{n}{np}\PYG{o}{.}\PYG{n}{where}\PYG{p}{(}\PYG{n}{EA} \PYG{o}{==} \PYG{n}{Emax}\PYG{p}{)}
        
        \PYG{k}{if} \PYG{p}{(}\PYG{n+nb}{len}\PYG{p}{(}\PYG{n}{idx}\PYG{p}{[}\PYG{l+m+mi}{0}\PYG{p}{]}\PYG{p}{)} \PYG{o}{\PYGZgt{}} \PYG{l+m+mi}{1}\PYG{p}{)}\PYG{p}{:}
            \PYG{n}{erro} \PYG{o}{=} \PYG{n}{np}\PYG{o}{.}\PYG{n}{absolute}\PYG{p}{(}\PYG{n}{EA}\PYG{p}{[}\PYG{n}{idx}\PYG{p}{[}\PYG{l+m+mi}{0}\PYG{p}{]}\PYG{p}{[}\PYG{l+m+mi}{0}\PYG{p}{]}\PYG{p}{]}\PYG{o}{/}\PYG{n}{X}\PYG{p}{[}\PYG{n}{idx}\PYG{p}{[}\PYG{l+m+mi}{0}\PYG{p}{]}\PYG{p}{[}\PYG{l+m+mi}{0}\PYG{p}{]}\PYG{p}{]}\PYG{p}{)}
        \PYG{k}{else}\PYG{p}{:}
            \PYG{n}{erro} \PYG{o}{=} \PYG{n}{np}\PYG{o}{.}\PYG{n}{absolute}\PYG{p}{(}\PYG{n}{EA}\PYG{p}{[}\PYG{n}{idx}\PYG{p}{[}\PYG{l+m+mi}{0}\PYG{p}{]}\PYG{p}{]}\PYG{o}{/}\PYG{n}{X}\PYG{p}{[}\PYG{n}{idx}\PYG{p}{[}\PYG{l+m+mi}{0}\PYG{p}{]}\PYG{p}{]}\PYG{p}{)}
    
    \PYG{n}{X} \PYG{o}{=} \PYG{n}{np}\PYG{o}{.}\PYG{n}{around}\PYG{p}{(}\PYG{n}{X}\PYG{p}{,} \PYG{n}{decimals}\PYG{o}{=}\PYG{l+m+mi}{3}\PYG{p}{)}
    
    \PYG{k}{return} \PYG{n}{X}

\PYG{c+c1}{\PYGZsh{} Interpolação}

\PYG{k}{def} \PYG{n+nf}{int\PYGZus{}newton}\PYG{p}{(}\PYG{n}{X}\PYG{p}{,} \PYG{n}{Y}\PYG{p}{,} \PYG{n}{x0}\PYG{p}{)}\PYG{p}{:}
    \PYG{l+s+sd}{\PYGZsq{}\PYGZsq{}\PYGZsq{}Função que realiza a interpolação de grau n de n+1 pontos dados, utilizando o método das diferenças divididas}
\PYG{l+s+sd}{    de Newton.}
\PYG{l+s+sd}{    }
\PYG{l+s+sd}{    Sinopse:}
\PYG{l+s+sd}{         y0 = int\PYGZus{}newton(X, Y, x0)}
\PYG{l+s+sd}{    }
\PYG{l+s+sd}{    Entradas:}
\PYG{l+s+sd}{         X \PYGZhy{} Array com as coordenadas em x para os pontos dados}
\PYG{l+s+sd}{         Y \PYGZhy{} Array com as coordenadas em y para os pontos dados}
\PYG{l+s+sd}{         x0 \PYGZhy{} Ponto ao qual se deseja a estimativa f(x0)}
\PYG{l+s+sd}{     }
\PYG{l+s+sd}{    Saídas:}
\PYG{l+s+sd}{         y0 \PYGZhy{} Estimativa desejada f(x0)}
\PYG{l+s+sd}{         }
\PYG{l+s+sd}{         @ney}
\PYG{l+s+sd}{    \PYGZsq{}\PYGZsq{}\PYGZsq{}}
    
    \PYG{n}{n} \PYG{o}{=} \PYG{n+nb}{len}\PYG{p}{(}\PYG{n}{X}\PYG{p}{)}
    \PYG{n}{x} \PYG{o}{=} \PYG{n}{sy}\PYG{o}{.}\PYG{n}{symbols}\PYG{p}{(}\PYG{l+s+s2}{\PYGZdq{}}\PYG{l+s+s2}{x}\PYG{l+s+s2}{\PYGZdq{}}\PYG{p}{)}
    
    \PYG{n}{dx} \PYG{o}{=} \PYG{p}{[}\PYG{p}{]}
    \PYG{n}{f} \PYG{o}{=} \PYG{p}{[}\PYG{p}{]}
    \PYG{n}{aux} \PYG{o}{=} \PYG{n}{np}\PYG{o}{.}\PYG{n}{diff}\PYG{p}{(}\PYG{n}{Y}\PYG{p}{)}

    \PYG{k}{for} \PYG{n}{i} \PYG{o+ow}{in} \PYG{n+nb}{range}\PYG{p}{(}\PYG{l+m+mi}{1}\PYG{p}{,}\PYG{n}{n}\PYG{p}{)}\PYG{p}{:}
        \PYG{n}{dx}\PYG{o}{.}\PYG{n}{append}\PYG{p}{(}\PYG{n}{X}\PYG{p}{[}\PYG{n}{i}\PYG{p}{:}\PYG{p}{]} \PYG{o}{\PYGZhy{}} \PYG{n}{X}\PYG{p}{[}\PYG{p}{:}\PYG{o}{\PYGZhy{}}\PYG{n}{i}\PYG{p}{]}\PYG{p}{)}
        \PYG{n}{f}\PYG{o}{.}\PYG{n}{append}\PYG{p}{(}\PYG{n}{aux}\PYG{o}{/}\PYG{n}{dx}\PYG{p}{[}\PYG{n}{i}\PYG{o}{\PYGZhy{}}\PYG{l+m+mi}{1}\PYG{p}{]}\PYG{p}{)} \PYG{c+c1}{\PYGZsh{} Diferenças Divididas}
        \PYG{n}{aux} \PYG{o}{=} \PYG{n}{np}\PYG{o}{.}\PYG{n}{diff}\PYG{p}{(}\PYG{n}{f}\PYG{p}{[}\PYG{n}{i}\PYG{o}{\PYGZhy{}}\PYG{l+m+mi}{1}\PYG{p}{]}\PYG{p}{)}
    
    \PYG{n}{b} \PYG{o}{=} \PYG{p}{[}\PYG{p}{]}
    
    \PYG{k}{for} \PYG{n}{i} \PYG{o+ow}{in} \PYG{n}{f}\PYG{p}{:}
        \PYG{n}{b}\PYG{o}{.}\PYG{n}{append}\PYG{p}{(}\PYG{n}{i}\PYG{p}{[}\PYG{l+m+mi}{0}\PYG{p}{]}\PYG{p}{)} \PYG{c+c1}{\PYGZsh{} Coeficientes do polinômio}
    
    \PYG{n}{f} \PYG{o}{=} \PYG{n}{Y}\PYG{p}{[}\PYG{l+m+mi}{0}\PYG{p}{]}
    \PYG{n}{aux} \PYG{o}{=} \PYG{l+m+mi}{1}

    \PYG{k}{for} \PYG{n}{i} \PYG{o+ow}{in} \PYG{n+nb}{range}\PYG{p}{(}\PYG{n}{n}\PYG{o}{\PYGZhy{}}\PYG{l+m+mi}{1}\PYG{p}{)}\PYG{p}{:}
        \PYG{n}{aux} \PYG{o}{*}\PYG{o}{=} \PYG{n}{x} \PYG{o}{\PYGZhy{}} \PYG{n}{X}\PYG{p}{[}\PYG{n}{i}\PYG{p}{]}
        \PYG{n}{f} \PYG{o}{+}\PYG{o}{=} \PYG{n}{b}\PYG{p}{[}\PYG{n}{i}\PYG{p}{]}\PYG{o}{*}\PYG{n}{aux} \PYG{c+c1}{\PYGZsh{} Polinômio}
        
    \PYG{n}{y0} \PYG{o}{=} \PYG{n}{f}\PYG{o}{.}\PYG{n}{subs}\PYG{p}{(}\PYG{n}{x}\PYG{p}{,}\PYG{n}{x0}\PYG{p}{)} \PYG{c+c1}{\PYGZsh{} Estimativa desejada}
    
    \PYG{k}{return} \PYG{n+nb}{float}\PYG{p}{(}\PYG{n}{y0}\PYG{p}{)}

\PYG{k}{def} \PYG{n+nf}{int\PYGZus{}lagrange}\PYG{p}{(}\PYG{n}{X}\PYG{p}{,} \PYG{n}{Y}\PYG{p}{,} \PYG{n}{x0}\PYG{p}{)}\PYG{p}{:}
    \PYG{l+s+sd}{\PYGZsq{}\PYGZsq{}\PYGZsq{}Função que realiza a interpolação de grau n de n+1 pontos dados, utilizando o método de Lagrange.}
\PYG{l+s+sd}{    }
\PYG{l+s+sd}{    Sinopse:}
\PYG{l+s+sd}{         y0 = int\PYGZus{}lagrange(X, Y, x0)}
\PYG{l+s+sd}{    }
\PYG{l+s+sd}{    Entradas:}
\PYG{l+s+sd}{         X \PYGZhy{} Vetor com as coordenadas em x para pontos dados}
\PYG{l+s+sd}{         Y \PYGZhy{} Vetor com as coordenadas em y para pontos dados}
\PYG{l+s+sd}{         x0 \PYGZhy{} Ponto ao qual se deseja a estimativa f(x0)}
\PYG{l+s+sd}{     }
\PYG{l+s+sd}{    Saídas:}
\PYG{l+s+sd}{         y0 \PYGZhy{} Estimativa desejada f(x0)}
\PYG{l+s+sd}{         }
\PYG{l+s+sd}{         @ney}
\PYG{l+s+sd}{    \PYGZsq{}\PYGZsq{}\PYGZsq{}}     
    
    \PYG{n}{n} \PYG{o}{=} \PYG{n+nb}{len}\PYG{p}{(}\PYG{n}{X}\PYG{p}{)}
    \PYG{n}{x} \PYG{o}{=} \PYG{n}{sy}\PYG{o}{.}\PYG{n}{symbols}\PYG{p}{(}\PYG{l+s+s2}{\PYGZdq{}}\PYG{l+s+s2}{x}\PYG{l+s+s2}{\PYGZdq{}}\PYG{p}{)}
    \PYG{n}{f} \PYG{o}{=} \PYG{l+m+mi}{0}
    
    \PYG{k}{for} \PYG{n}{i} \PYG{o+ow}{in} \PYG{n+nb}{range}\PYG{p}{(}\PYG{n}{n}\PYG{p}{)}\PYG{p}{:}
        \PYG{n}{L} \PYG{o}{=} \PYG{l+m+mi}{1}
        \PYG{k}{for} \PYG{n}{j} \PYG{o+ow}{in} \PYG{n+nb}{range}\PYG{p}{(}\PYG{n}{n}\PYG{p}{)}\PYG{p}{:}
            \PYG{k}{if} \PYG{p}{(}\PYG{n}{i} \PYG{o}{!=} \PYG{n}{j}\PYG{p}{)}\PYG{p}{:}
                \PYG{n}{L} \PYG{o}{*}\PYG{o}{=} \PYG{p}{(}\PYG{n}{x} \PYG{o}{\PYGZhy{}} \PYG{n}{X}\PYG{p}{[}\PYG{n}{j}\PYG{p}{]}\PYG{p}{)}\PYG{o}{/}\PYG{p}{(}\PYG{n}{X}\PYG{p}{[}\PYG{n}{i}\PYG{p}{]} \PYG{o}{\PYGZhy{}} \PYG{n}{X}\PYG{p}{[}\PYG{n}{j}\PYG{p}{]}\PYG{p}{)}
        \PYG{n}{f} \PYG{o}{+}\PYG{o}{=} \PYG{n}{L}\PYG{o}{*}\PYG{n}{Y}\PYG{p}{[}\PYG{n}{i}\PYG{p}{]} \PYG{c+c1}{\PYGZsh{} Polinômio}
    
    \PYG{n}{y0} \PYG{o}{=} \PYG{n}{f}\PYG{o}{.}\PYG{n}{subs}\PYG{p}{(}\PYG{n}{x}\PYG{p}{,}\PYG{n}{x0}\PYG{p}{)} \PYG{c+c1}{\PYGZsh{} Estimativa desejada}
    
    \PYG{k}{return} \PYG{n+nb}{float}\PYG{p}{(}\PYG{n}{y0}\PYG{p}{)}
\end{sphinxVerbatim}


\section{Métodos iterativos para sistemas lineares}
\label{\detokenize{lista-4-solucoes:metodos-iterativos-para-sistemas-lineares}}

\subsection{Questão 1}
\label{\detokenize{lista-4-solucoes:questao-1}}
\sphinxAtStartPar
Verifique o condicionamento e o critério das linhas para os sistemas abaixo e em seguida aplique o método de Gauss\sphinxhyphen{}Jacobi para determinar uma solução aproximada, com erro absoluto inferior a \(10^{-2}\), tomando a aproximação inicial \(x^{(0)} = 0\).

\sphinxAtStartPar
(a) \textbackslash{}begin\{cases\}
2x\_1 − x\_2 = 1 \textbackslash{}
x\_1 + 2x\_2 = 3
\textbackslash{}end\{cases\}

\sphinxAtStartPar
(b) \textbackslash{}begin\{cases\}
x\_1 − 0.25x\_2 − 0.25x\_3 = 0 \textbackslash{}
−0.25x\_1 + x\_2 − 0.25x\_4 = 0 \textbackslash{}
−0.25x\_1 + x\_3 − 0.25x\_4 = 0.25 \textbackslash{}
−0.25x\_2 + x\_4 = 0.25
\textbackslash{}end\{cases\}

\sphinxAtStartPar
\sphinxstylestrong{Obs.} Tome como número de condicionamento o valor \(C = det(Ã)\), em que \(Ã\) e obtida de \(A\) fazendo com que o maior elemento em valor absoluto em cada linha de \(Ã\) seja igual a \(1\).

\begin{sphinxVerbatim}[commandchars=\\\{\}]
\PYG{c+c1}{\PYGZsh{} Solução}

\PYG{c+c1}{\PYGZsh{} (a)}
\PYG{n}{A} \PYG{o}{=} \PYG{n}{np}\PYG{o}{.}\PYG{n}{array}\PYG{p}{(}\PYG{p}{[}\PYG{p}{[}\PYG{l+m+mf}{2.}\PYG{p}{,} \PYG{o}{\PYGZhy{}}\PYG{l+m+mf}{1.}\PYG{p}{]}\PYG{p}{,} \PYG{p}{[}\PYG{l+m+mf}{1.}\PYG{p}{,} \PYG{l+m+mf}{2.}\PYG{p}{]}\PYG{p}{]}\PYG{p}{)} \PYG{c+c1}{\PYGZsh{} Matriz de coeficientes constantes}
\PYG{n}{B} \PYG{o}{=} \PYG{n}{np}\PYG{o}{.}\PYG{n}{array}\PYG{p}{(}\PYG{p}{[}\PYG{p}{[}\PYG{l+m+mf}{1.}\PYG{p}{]}\PYG{p}{,} \PYG{p}{[}\PYG{l+m+mf}{3.}\PYG{p}{]}\PYG{p}{]}\PYG{p}{)} \PYG{c+c1}{\PYGZsh{} Vetor de constantes}
\PYG{n}{X} \PYG{o}{=} \PYG{n}{np}\PYG{o}{.}\PYG{n}{array}\PYG{p}{(}\PYG{p}{[}\PYG{p}{[}\PYG{l+m+mf}{0.}\PYG{p}{]}\PYG{p}{,} \PYG{p}{[}\PYG{l+m+mf}{0.}\PYG{p}{]}\PYG{p}{]}\PYG{p}{)} \PYG{c+c1}{\PYGZsh{} Estimativa inicial (X0)}
\PYG{n}{EA} \PYG{o}{=} \PYG{l+m+mi}{10}\PYG{o}{*}\PYG{o}{*}\PYG{p}{(}\PYG{o}{\PYGZhy{}}\PYG{l+m+mi}{2}\PYG{p}{)}
\PYG{n}{erro} \PYG{o}{=} \PYG{l+m+mi}{1}

\PYG{k}{while} \PYG{p}{(}\PYG{n}{erro} \PYG{o}{\PYGZgt{}} \PYG{n}{EA}\PYG{p}{)}\PYG{p}{:}
    \PYG{n}{Xp} \PYG{o}{=} \PYG{n}{np}\PYG{o}{.}\PYG{n}{copy}\PYG{p}{(}\PYG{n}{X}\PYG{p}{)}
    \PYG{k}{for} \PYG{n}{i} \PYG{o+ow}{in} \PYG{n+nb}{range}\PYG{p}{(}\PYG{n+nb}{len}\PYG{p}{(}\PYG{n}{A}\PYG{p}{)}\PYG{p}{)}\PYG{p}{:}
        \PYG{n}{s} \PYG{o}{=} \PYG{l+m+mi}{0}
        \PYG{k}{for} \PYG{n}{j} \PYG{o+ow}{in} \PYG{n+nb}{range}\PYG{p}{(}\PYG{n+nb}{len}\PYG{p}{(}\PYG{n}{A}\PYG{p}{)}\PYG{p}{)}\PYG{p}{:}
            \PYG{k}{if} \PYG{p}{(}\PYG{n}{i} \PYG{o}{!=} \PYG{n}{j}\PYG{p}{)}\PYG{p}{:}
                \PYG{n}{s} \PYG{o}{+}\PYG{o}{=} \PYG{n}{A}\PYG{p}{[}\PYG{n}{i}\PYG{p}{,}\PYG{n}{j}\PYG{p}{]}\PYG{o}{*}\PYG{n}{Xp}\PYG{p}{[}\PYG{n}{j}\PYG{p}{]}
        \PYG{n}{X}\PYG{p}{[}\PYG{n}{i}\PYG{p}{]} \PYG{o}{=} \PYG{p}{(}\PYG{n}{B}\PYG{p}{[}\PYG{n}{i}\PYG{p}{]} \PYG{o}{\PYGZhy{}} \PYG{n}{s}\PYG{p}{)}\PYG{o}{/}\PYG{n}{A}\PYG{p}{[}\PYG{n}{i}\PYG{p}{,}\PYG{n}{i}\PYG{p}{]}
    \PYG{n}{erro} \PYG{o}{=} \PYG{n}{np}\PYG{o}{.}\PYG{n}{amax}\PYG{p}{(}\PYG{n}{np}\PYG{o}{.}\PYG{n}{absolute}\PYG{p}{(}\PYG{n}{X} \PYG{o}{\PYGZhy{}} \PYG{n}{Xp}\PYG{p}{)}\PYG{p}{)}

\PYG{n}{X} \PYG{o}{=} \PYG{n}{np}\PYG{o}{.}\PYG{n}{around}\PYG{p}{(}\PYG{n}{X}\PYG{p}{,} \PYG{n}{decimals}\PYG{o}{=}\PYG{l+m+mi}{3}\PYG{p}{)}    
    
\PYG{n+nb}{print}\PYG{p}{(}\PYG{l+s+s2}{\PYGZdq{}}\PYG{l+s+s2}{(a)}\PYG{l+s+se}{\PYGZbs{}n}\PYG{l+s+s2}{O número C de condicionamento é:}\PYG{l+s+s2}{\PYGZdq{}}\PYG{p}{,} \PYG{n}{np}\PYG{o}{.}\PYG{n}{around}\PYG{p}{(}\PYG{n}{np}\PYG{o}{.}\PYG{n}{linalg}\PYG{o}{.}\PYG{n}{det}\PYG{p}{(}\PYG{n}{A}\PYG{p}{)}\PYG{p}{,} \PYG{n}{decimals}\PYG{o}{=}\PYG{l+m+mi}{3}\PYG{p}{)}\PYG{p}{)}
\PYG{n+nb}{print}\PYG{p}{(}\PYG{l+s+s2}{\PYGZdq{}}\PYG{l+s+se}{\PYGZbs{}n}\PYG{l+s+s2}{A solução do problema é:}\PYG{l+s+se}{\PYGZbs{}n}\PYG{l+s+s2}{\PYGZdq{}}\PYG{p}{,} \PYG{n}{X}\PYG{p}{)}

\PYG{c+c1}{\PYGZsh{} (b)}
\PYG{n}{A} \PYG{o}{=} \PYG{n}{np}\PYG{o}{.}\PYG{n}{array}\PYG{p}{(}\PYG{p}{[}\PYG{p}{[}\PYG{l+m+mf}{1.}\PYG{p}{,} \PYG{o}{\PYGZhy{}}\PYG{l+m+mf}{0.25}\PYG{p}{,} \PYG{o}{\PYGZhy{}}\PYG{l+m+mf}{0.25}\PYG{p}{,} \PYG{l+m+mf}{0.}\PYG{p}{]}\PYG{p}{,} \PYG{p}{[}\PYG{o}{\PYGZhy{}}\PYG{l+m+mf}{0.25}\PYG{p}{,} \PYG{l+m+mf}{1.}\PYG{p}{,} \PYG{l+m+mf}{0.}\PYG{p}{,} \PYG{o}{\PYGZhy{}}\PYG{l+m+mf}{0.25}\PYG{p}{]}\PYG{p}{,} \PYG{p}{[}\PYG{o}{\PYGZhy{}}\PYG{l+m+mf}{0.25}\PYG{p}{,} \PYG{l+m+mf}{0.}\PYG{p}{,} \PYG{l+m+mf}{1.}\PYG{p}{,} \PYG{o}{\PYGZhy{}}\PYG{l+m+mf}{0.25}\PYG{p}{]}\PYG{p}{,} \PYG{p}{[}\PYG{l+m+mf}{0.}\PYG{p}{,} \PYG{o}{\PYGZhy{}}\PYG{l+m+mf}{0.25}\PYG{p}{,} \PYG{l+m+mf}{0.}\PYG{p}{,} \PYG{l+m+mf}{1.}\PYG{p}{]}\PYG{p}{]}\PYG{p}{)} \PYG{c+c1}{\PYGZsh{} Coeficientes}
\PYG{n}{B} \PYG{o}{=} \PYG{n}{np}\PYG{o}{.}\PYG{n}{array}\PYG{p}{(}\PYG{p}{[}\PYG{p}{[}\PYG{l+m+mf}{0.}\PYG{p}{]}\PYG{p}{,} \PYG{p}{[}\PYG{l+m+mf}{0.}\PYG{p}{]}\PYG{p}{,} \PYG{p}{[}\PYG{l+m+mf}{0.25}\PYG{p}{]}\PYG{p}{,} \PYG{p}{[}\PYG{l+m+mf}{0.25}\PYG{p}{]}\PYG{p}{]}\PYG{p}{)} \PYG{c+c1}{\PYGZsh{} Vetor de constantes}
\PYG{n}{X} \PYG{o}{=} \PYG{n}{np}\PYG{o}{.}\PYG{n}{array}\PYG{p}{(}\PYG{p}{[}\PYG{p}{[}\PYG{l+m+mf}{0.}\PYG{p}{]}\PYG{p}{,} \PYG{p}{[}\PYG{l+m+mf}{0.}\PYG{p}{]}\PYG{p}{,} \PYG{p}{[}\PYG{l+m+mf}{0.}\PYG{p}{]}\PYG{p}{,} \PYG{p}{[}\PYG{l+m+mf}{0.}\PYG{p}{]}\PYG{p}{]}\PYG{p}{)} \PYG{c+c1}{\PYGZsh{} Estimativa inicial (X0)}
\PYG{n}{EA} \PYG{o}{=} \PYG{l+m+mi}{10}\PYG{o}{*}\PYG{o}{*}\PYG{p}{(}\PYG{o}{\PYGZhy{}}\PYG{l+m+mi}{2}\PYG{p}{)}
\PYG{n}{erro} \PYG{o}{=} \PYG{l+m+mi}{1}

\PYG{k}{while} \PYG{p}{(}\PYG{n}{erro} \PYG{o}{\PYGZgt{}} \PYG{n}{EA}\PYG{p}{)}\PYG{p}{:}
    \PYG{n}{Xp} \PYG{o}{=} \PYG{n}{np}\PYG{o}{.}\PYG{n}{copy}\PYG{p}{(}\PYG{n}{X}\PYG{p}{)}
    \PYG{k}{for} \PYG{n}{i} \PYG{o+ow}{in} \PYG{n+nb}{range}\PYG{p}{(}\PYG{n+nb}{len}\PYG{p}{(}\PYG{n}{A}\PYG{p}{)}\PYG{p}{)}\PYG{p}{:}
        \PYG{n}{s} \PYG{o}{=} \PYG{l+m+mi}{0}
        \PYG{k}{for} \PYG{n}{j} \PYG{o+ow}{in} \PYG{n+nb}{range}\PYG{p}{(}\PYG{n+nb}{len}\PYG{p}{(}\PYG{n}{A}\PYG{p}{)}\PYG{p}{)}\PYG{p}{:}
            \PYG{k}{if} \PYG{p}{(}\PYG{n}{i} \PYG{o}{!=} \PYG{n}{j}\PYG{p}{)}\PYG{p}{:}
                \PYG{n}{s} \PYG{o}{+}\PYG{o}{=} \PYG{n}{A}\PYG{p}{[}\PYG{n}{i}\PYG{p}{,}\PYG{n}{j}\PYG{p}{]}\PYG{o}{*}\PYG{n}{Xp}\PYG{p}{[}\PYG{n}{j}\PYG{p}{]}
        \PYG{n}{X}\PYG{p}{[}\PYG{n}{i}\PYG{p}{]} \PYG{o}{=} \PYG{p}{(}\PYG{n}{B}\PYG{p}{[}\PYG{n}{i}\PYG{p}{]} \PYG{o}{\PYGZhy{}} \PYG{n}{s}\PYG{p}{)}\PYG{o}{/}\PYG{n}{A}\PYG{p}{[}\PYG{n}{i}\PYG{p}{,}\PYG{n}{i}\PYG{p}{]}
    \PYG{n}{erro} \PYG{o}{=} \PYG{n}{np}\PYG{o}{.}\PYG{n}{amax}\PYG{p}{(}\PYG{n}{np}\PYG{o}{.}\PYG{n}{absolute}\PYG{p}{(}\PYG{n}{X} \PYG{o}{\PYGZhy{}} \PYG{n}{Xp}\PYG{p}{)}\PYG{p}{)}

\PYG{n}{X} \PYG{o}{=} \PYG{n}{np}\PYG{o}{.}\PYG{n}{around}\PYG{p}{(}\PYG{n}{X}\PYG{p}{,} \PYG{n}{decimals}\PYG{o}{=}\PYG{l+m+mi}{3}\PYG{p}{)}    
    
\PYG{n+nb}{print}\PYG{p}{(}\PYG{l+s+s2}{\PYGZdq{}}\PYG{l+s+se}{\PYGZbs{}n}\PYG{l+s+s2}{(b)}\PYG{l+s+se}{\PYGZbs{}n}\PYG{l+s+s2}{O número C de condicionamento é:}\PYG{l+s+s2}{\PYGZdq{}}\PYG{p}{,} \PYG{n}{np}\PYG{o}{.}\PYG{n}{around}\PYG{p}{(}\PYG{n}{np}\PYG{o}{.}\PYG{n}{linalg}\PYG{o}{.}\PYG{n}{det}\PYG{p}{(}\PYG{n}{A}\PYG{p}{)}\PYG{p}{,} \PYG{n}{decimals}\PYG{o}{=}\PYG{l+m+mi}{3}\PYG{p}{)}\PYG{p}{)}
\PYG{n+nb}{print}\PYG{p}{(}\PYG{l+s+s2}{\PYGZdq{}}\PYG{l+s+se}{\PYGZbs{}n}\PYG{l+s+s2}{A solução do problema é:}\PYG{l+s+se}{\PYGZbs{}n}\PYG{l+s+s2}{\PYGZdq{}}\PYG{p}{,} \PYG{n}{X}\PYG{p}{)}
\end{sphinxVerbatim}

\begin{sphinxVerbatim}[commandchars=\\\{\}]
(a)
O número C de condicionamento é: 5.0

A solução do problema é:
 [[0.998]
 [1.002]]

(b)
O número C de condicionamento é: 0.812

A solução do problema é:
 [[0.107]
 [0.093]
 [0.343]
 [0.272]]
\end{sphinxVerbatim}


\subsection{Questão 2}
\label{\detokenize{lista-4-solucoes:questao-2}}
\sphinxAtStartPar
Considere o sistema

\sphinxAtStartPar
\textbackslash{}begin\{cases\}
15c\_1 − 2c\_2 − c\_3 = 3800 \textbackslash{}
−3c\_1 + 18c\_2 − 6c\_3 = 1200 \textbackslash{}
−4c\_1 − c\_2 + 12c\_3 = 2350
\textbackslash{}end\{cases\}

\sphinxAtStartPar
(a) Verifique o condicionamento do sistema.

\sphinxAtStartPar
(b) Verifique o critério das linhas.

\sphinxAtStartPar
© Verifique o critério de Sassenfeld.

\sphinxAtStartPar
(d) Determine uma solução aproximada do sistema com erro relativo percentual abaixo de \(5\%\) usando o método iterativo de Gauss\sphinxhyphen{}Jacobi.

\sphinxAtStartPar
(e) Repita o item (d) usando o método de Gauss\sphinxhyphen{}Seidel.

\sphinxAtStartPar
(f) Compare o número de iterações nos itens (d) e (e).

\begin{sphinxVerbatim}[commandchars=\\\{\}]
\PYG{c+c1}{\PYGZsh{} Solução}

\PYG{n}{AB} \PYG{o}{=} \PYG{n}{np}\PYG{o}{.}\PYG{n}{array}\PYG{p}{(}\PYG{p}{[}\PYG{p}{[}\PYG{l+m+mf}{15.}\PYG{p}{,} \PYG{o}{\PYGZhy{}}\PYG{l+m+mf}{2.}\PYG{p}{,} \PYG{o}{\PYGZhy{}}\PYG{l+m+mf}{1.}\PYG{p}{,} \PYG{l+m+mf}{3800.}\PYG{p}{]}\PYG{p}{,} \PYG{p}{[}\PYG{o}{\PYGZhy{}}\PYG{l+m+mf}{3.}\PYG{p}{,} \PYG{l+m+mf}{18.}\PYG{p}{,} \PYG{o}{\PYGZhy{}}\PYG{l+m+mf}{6.}\PYG{p}{,} \PYG{l+m+mf}{1200.}\PYG{p}{]}\PYG{p}{,} \PYG{p}{[}\PYG{o}{\PYGZhy{}}\PYG{l+m+mf}{4.}\PYG{p}{,} \PYG{o}{\PYGZhy{}}\PYG{l+m+mf}{1.}\PYG{p}{,} \PYG{l+m+mf}{12.}\PYG{p}{,} \PYG{l+m+mf}{2350.}\PYG{p}{]}\PYG{p}{]}\PYG{p}{)}
\PYG{n}{ER} \PYG{o}{=} \PYG{l+m+mi}{5}\PYG{o}{/}\PYG{l+m+mi}{100}
\PYG{n}{X0} \PYG{o}{=} \PYG{n}{np}\PYG{o}{.}\PYG{n}{array}\PYG{p}{(}\PYG{p}{[}\PYG{p}{[}\PYG{l+m+mf}{0.}\PYG{p}{]}\PYG{p}{,} \PYG{p}{[}\PYG{l+m+mf}{0.}\PYG{p}{]}\PYG{p}{,} \PYG{p}{[}\PYG{l+m+mf}{0.}\PYG{p}{]}\PYG{p}{]}\PYG{p}{)}

\PYG{c+c1}{\PYGZsh{} (d) Gauss\PYGZhy{}Jacobi}
\PYG{n}{A} \PYG{o}{=} \PYG{n}{AB}\PYG{p}{[}\PYG{p}{:}\PYG{p}{,}\PYG{l+m+mi}{0}\PYG{p}{:}\PYG{o}{\PYGZhy{}}\PYG{l+m+mi}{1}\PYG{p}{]}
\PYG{n}{B} \PYG{o}{=} \PYG{n}{AB}\PYG{p}{[}\PYG{p}{:}\PYG{p}{,}\PYG{o}{\PYGZhy{}}\PYG{l+m+mi}{1}\PYG{p}{]}
\PYG{n}{X} \PYG{o}{=} \PYG{n}{np}\PYG{o}{.}\PYG{n}{copy}\PYG{p}{(}\PYG{n}{X0}\PYG{p}{)}
\PYG{n}{erro} \PYG{o}{=} \PYG{l+m+mi}{1}
\PYG{n}{k1} \PYG{o}{=} \PYG{l+m+mi}{0}
    
\PYG{k}{while} \PYG{p}{(}\PYG{n}{erro} \PYG{o}{\PYGZgt{}} \PYG{n}{ER}\PYG{p}{)}\PYG{p}{:}
    \PYG{n}{Xp} \PYG{o}{=} \PYG{n}{np}\PYG{o}{.}\PYG{n}{copy}\PYG{p}{(}\PYG{n}{X}\PYG{p}{)}
    \PYG{k}{for} \PYG{n}{i} \PYG{o+ow}{in} \PYG{n+nb}{range}\PYG{p}{(}\PYG{n+nb}{len}\PYG{p}{(}\PYG{n}{A}\PYG{p}{)}\PYG{p}{)}\PYG{p}{:}
        \PYG{n}{s} \PYG{o}{=} \PYG{l+m+mi}{0}
        \PYG{k}{for} \PYG{n}{j} \PYG{o+ow}{in} \PYG{n+nb}{range}\PYG{p}{(}\PYG{n+nb}{len}\PYG{p}{(}\PYG{n}{A}\PYG{p}{)}\PYG{p}{)}\PYG{p}{:}
            \PYG{k}{if} \PYG{p}{(}\PYG{n}{i} \PYG{o}{!=} \PYG{n}{j}\PYG{p}{)}\PYG{p}{:}
                \PYG{n}{s} \PYG{o}{+}\PYG{o}{=} \PYG{n}{A}\PYG{p}{[}\PYG{n}{i}\PYG{p}{,}\PYG{n}{j}\PYG{p}{]}\PYG{o}{*}\PYG{n}{Xp}\PYG{p}{[}\PYG{n}{j}\PYG{p}{]}
        \PYG{n}{X}\PYG{p}{[}\PYG{n}{i}\PYG{p}{]} \PYG{o}{=} \PYG{p}{(}\PYG{n}{B}\PYG{p}{[}\PYG{n}{i}\PYG{p}{]} \PYG{o}{\PYGZhy{}} \PYG{n}{s}\PYG{p}{)}\PYG{o}{/}\PYG{n}{A}\PYG{p}{[}\PYG{n}{i}\PYG{p}{,}\PYG{n}{i}\PYG{p}{]}
    \PYG{n}{EA} \PYG{o}{=} \PYG{n}{np}\PYG{o}{.}\PYG{n}{absolute}\PYG{p}{(}\PYG{n}{X} \PYG{o}{\PYGZhy{}} \PYG{n}{Xp}\PYG{p}{)}
    \PYG{n}{Emax} \PYG{o}{=} \PYG{n}{np}\PYG{o}{.}\PYG{n}{amax}\PYG{p}{(}\PYG{n}{EA}\PYG{p}{)}
    \PYG{n}{idx} \PYG{o}{=} \PYG{n}{np}\PYG{o}{.}\PYG{n}{where}\PYG{p}{(}\PYG{n}{EA} \PYG{o}{==} \PYG{n}{Emax}\PYG{p}{)}
    \PYG{n}{erro} \PYG{o}{=} \PYG{n}{np}\PYG{o}{.}\PYG{n}{absolute}\PYG{p}{(}\PYG{n}{EA}\PYG{p}{[}\PYG{n}{idx}\PYG{p}{[}\PYG{l+m+mi}{0}\PYG{p}{]}\PYG{p}{]}\PYG{o}{/}\PYG{n}{X}\PYG{p}{[}\PYG{n}{idx}\PYG{p}{[}\PYG{l+m+mi}{0}\PYG{p}{]}\PYG{p}{]}\PYG{p}{)}
    \PYG{n}{k1} \PYG{o}{+}\PYG{o}{=} \PYG{l+m+mi}{1}

\PYG{n}{X} \PYG{o}{=} \PYG{n}{np}\PYG{o}{.}\PYG{n}{around}\PYG{p}{(}\PYG{n}{X}\PYG{p}{,} \PYG{n}{decimals}\PYG{o}{=}\PYG{l+m+mi}{3}\PYG{p}{)}    
    
\PYG{n+nb}{print}\PYG{p}{(}\PYG{l+s+s2}{\PYGZdq{}}\PYG{l+s+s2}{(d) A solução de X é:}\PYG{l+s+se}{\PYGZbs{}n}\PYG{l+s+s2}{\PYGZdq{}}\PYG{p}{,} \PYG{n}{X}\PYG{p}{)}

\PYG{c+c1}{\PYGZsh{} (e) Gauss\PYGZhy{}Seidel}
\PYG{n}{A} \PYG{o}{=} \PYG{n}{AB}\PYG{p}{[}\PYG{p}{:}\PYG{p}{,}\PYG{l+m+mi}{0}\PYG{p}{:}\PYG{o}{\PYGZhy{}}\PYG{l+m+mi}{1}\PYG{p}{]}
\PYG{n}{B} \PYG{o}{=} \PYG{n}{AB}\PYG{p}{[}\PYG{p}{:}\PYG{p}{,}\PYG{o}{\PYGZhy{}}\PYG{l+m+mi}{1}\PYG{p}{]}
\PYG{n}{X} \PYG{o}{=} \PYG{n}{np}\PYG{o}{.}\PYG{n}{copy}\PYG{p}{(}\PYG{n}{X0}\PYG{p}{)}
\PYG{n}{erro} \PYG{o}{=} \PYG{l+m+mi}{1}
\PYG{n}{k2} \PYG{o}{=} \PYG{l+m+mi}{0}
    
\PYG{k}{while} \PYG{p}{(}\PYG{n}{erro} \PYG{o}{\PYGZgt{}} \PYG{n}{ER}\PYG{p}{)}\PYG{p}{:}
    \PYG{n}{Xp} \PYG{o}{=} \PYG{n}{np}\PYG{o}{.}\PYG{n}{copy}\PYG{p}{(}\PYG{n}{X}\PYG{p}{)}
    \PYG{k}{for} \PYG{n}{i} \PYG{o+ow}{in} \PYG{n+nb}{range}\PYG{p}{(}\PYG{n+nb}{len}\PYG{p}{(}\PYG{n}{A}\PYG{p}{)}\PYG{p}{)}\PYG{p}{:}
        \PYG{n}{s} \PYG{o}{=} \PYG{l+m+mi}{0}
        \PYG{k}{for} \PYG{n}{j} \PYG{o+ow}{in} \PYG{n+nb}{range}\PYG{p}{(}\PYG{n+nb}{len}\PYG{p}{(}\PYG{n}{A}\PYG{p}{)}\PYG{p}{)}\PYG{p}{:}
            \PYG{k}{if} \PYG{p}{(}\PYG{n}{i} \PYG{o}{!=} \PYG{n}{j}\PYG{p}{)}\PYG{p}{:}
                \PYG{n}{s} \PYG{o}{+}\PYG{o}{=} \PYG{n}{A}\PYG{p}{[}\PYG{n}{i}\PYG{p}{,}\PYG{n}{j}\PYG{p}{]}\PYG{o}{*}\PYG{n}{X}\PYG{p}{[}\PYG{n}{j}\PYG{p}{]}
        \PYG{n}{X}\PYG{p}{[}\PYG{n}{i}\PYG{p}{]} \PYG{o}{=} \PYG{p}{(}\PYG{n}{B}\PYG{p}{[}\PYG{n}{i}\PYG{p}{]} \PYG{o}{\PYGZhy{}} \PYG{n}{s}\PYG{p}{)}\PYG{o}{/}\PYG{n}{A}\PYG{p}{[}\PYG{n}{i}\PYG{p}{,}\PYG{n}{i}\PYG{p}{]}
    \PYG{n}{EA} \PYG{o}{=} \PYG{n}{np}\PYG{o}{.}\PYG{n}{absolute}\PYG{p}{(}\PYG{n}{X} \PYG{o}{\PYGZhy{}} \PYG{n}{Xp}\PYG{p}{)}
    \PYG{n}{Emax} \PYG{o}{=} \PYG{n}{np}\PYG{o}{.}\PYG{n}{amax}\PYG{p}{(}\PYG{n}{EA}\PYG{p}{)}
    \PYG{n}{idx} \PYG{o}{=} \PYG{n}{np}\PYG{o}{.}\PYG{n}{where}\PYG{p}{(}\PYG{n}{EA} \PYG{o}{==} \PYG{n}{Emax}\PYG{p}{)}
    \PYG{n}{erro} \PYG{o}{=} \PYG{n}{np}\PYG{o}{.}\PYG{n}{absolute}\PYG{p}{(}\PYG{n}{EA}\PYG{p}{[}\PYG{n}{idx}\PYG{p}{[}\PYG{l+m+mi}{0}\PYG{p}{]}\PYG{p}{]}\PYG{o}{/}\PYG{n}{X}\PYG{p}{[}\PYG{n}{idx}\PYG{p}{[}\PYG{l+m+mi}{0}\PYG{p}{]}\PYG{p}{]}\PYG{p}{)}
    \PYG{n}{k2} \PYG{o}{+}\PYG{o}{=} \PYG{l+m+mi}{1}

\PYG{n}{X} \PYG{o}{=} \PYG{n}{np}\PYG{o}{.}\PYG{n}{around}\PYG{p}{(}\PYG{n}{X}\PYG{p}{,} \PYG{n}{decimals}\PYG{o}{=}\PYG{l+m+mi}{3}\PYG{p}{)}

\PYG{n+nb}{print}\PYG{p}{(}\PYG{l+s+s2}{\PYGZdq{}}\PYG{l+s+se}{\PYGZbs{}n}\PYG{l+s+s2}{(e) A solução de X é:}\PYG{l+s+se}{\PYGZbs{}n}\PYG{l+s+s2}{\PYGZdq{}}\PYG{p}{,} \PYG{n}{X}\PYG{p}{)}

\PYG{c+c1}{\PYGZsh{} (f)}
\PYG{n+nb}{print}\PYG{p}{(}\PYG{l+s+s2}{\PYGZdq{}}\PYG{l+s+se}{\PYGZbs{}n}\PYG{l+s+s2}{(f) O número de iterações pelo método de Gauss\PYGZhy{}Jacobi foi:}\PYG{l+s+s2}{\PYGZdq{}}\PYG{p}{,} \PYG{n}{k1}\PYG{p}{)}
\PYG{n+nb}{print}\PYG{p}{(}\PYG{l+s+s2}{\PYGZdq{}}\PYG{l+s+s2}{O número de iterações pelo método de Gauss\PYGZhy{}Seidel foi:}\PYG{l+s+s2}{\PYGZdq{}}\PYG{p}{,} \PYG{n}{k2}\PYG{p}{)}
\end{sphinxVerbatim}

\begin{sphinxVerbatim}[commandchars=\\\{\}]
(d) A solução de X é:
 [[301.184]
 [216.637]
 [311.689]]

(e) A solução de X é:
 [[302.072]
 [220.055]
 [314.862]]

(f) O número de iterações pelo método de Gauss\PYGZhy{}Jacobi foi: 4
O número de iterações pelo método de Gauss\PYGZhy{}Seidel foi: 3
\end{sphinxVerbatim}


\subsection{Questão 3}
\label{\detokenize{lista-4-solucoes:questao-3}}
\sphinxAtStartPar
Use o método iterativo de Gauss\sphinxhyphen{}Seidel para determinar uma solução aproximada do sistema com erro relativo percentual abaixo de \(5\%\) dos seguintes sistemas. Antes verifique o critério de Sassenfeld e caso algum dos sitemas não o satisfaça, reorganize o sistema permutando linhas e/ou colunas para garantir a convergência.

\sphinxAtStartPar
(a) \textbackslash{}begin\{cases\}
10x\_1 + 2x\_2 − x\_3 = 27 \textbackslash{}
−3x\_1 − 6x\_2 + 2x\_3 = −61.5 \textbackslash{}
x\_1 + x\_2 + 5x\_3 = −21.5
\textbackslash{}end\{cases\}

\sphinxAtStartPar
(b) \textbackslash{}begin\{cases\}
−3x\_1 + x\_2 + 12x\_3 = 50 \textbackslash{}
6x\_1 − x\_2 − x\_3 = 3 \textbackslash{}
6x\_1 + 9x\_2 + x\_3 = 40
\textbackslash{}end\{cases\}

\sphinxAtStartPar
© \textbackslash{}begin\{cases\}
2x\_1 − 6x\_2 − x\_3 = −38 \textbackslash{}
−3x\_1 − x\_2 + 7x\_3 = −34 \textbackslash{}
−8x\_1 + x\_2 − 2x\_3 = −20
\textbackslash{}end\{cases\}

\begin{sphinxVerbatim}[commandchars=\\\{\}]
\PYG{c+c1}{\PYGZsh{} Solução}

\PYG{c+c1}{\PYGZsh{} (a)}
\PYG{n}{AB} \PYG{o}{=} \PYG{n}{np}\PYG{o}{.}\PYG{n}{array}\PYG{p}{(}\PYG{p}{[}\PYG{p}{[}\PYG{l+m+mf}{10.}\PYG{p}{,} \PYG{l+m+mf}{2.}\PYG{p}{,} \PYG{o}{\PYGZhy{}}\PYG{l+m+mf}{1.}\PYG{p}{,} \PYG{l+m+mf}{27.}\PYG{p}{]}\PYG{p}{,} \PYG{p}{[}\PYG{o}{\PYGZhy{}}\PYG{l+m+mf}{3.}\PYG{p}{,} \PYG{o}{\PYGZhy{}}\PYG{l+m+mf}{6.}\PYG{p}{,} \PYG{l+m+mf}{2.}\PYG{p}{,} \PYG{o}{\PYGZhy{}}\PYG{l+m+mf}{61.5}\PYG{p}{]}\PYG{p}{,} \PYG{p}{[}\PYG{l+m+mf}{1.}\PYG{p}{,} \PYG{l+m+mf}{1.}\PYG{p}{,} \PYG{l+m+mf}{5.}\PYG{p}{,} \PYG{o}{\PYGZhy{}}\PYG{l+m+mf}{21.5}\PYG{p}{]}\PYG{p}{]}\PYG{p}{)}
\PYG{n}{ER} \PYG{o}{=} \PYG{l+m+mi}{5}\PYG{o}{/}\PYG{l+m+mi}{100}
\PYG{n}{X0} \PYG{o}{=} \PYG{n}{np}\PYG{o}{.}\PYG{n}{array}\PYG{p}{(}\PYG{p}{[}\PYG{l+m+mf}{0.}\PYG{p}{,} \PYG{l+m+mf}{0.}\PYG{p}{,} \PYG{l+m+mf}{0.}\PYG{p}{]}\PYG{p}{)}
\PYG{n}{X} \PYG{o}{=} \PYG{n}{gaussseidel}\PYG{p}{(}\PYG{n}{AB}\PYG{p}{,}\PYG{n}{ER}\PYG{p}{,}\PYG{n}{X0}\PYG{p}{)}
\PYG{n+nb}{print}\PYG{p}{(}\PYG{l+s+s2}{\PYGZdq{}}\PYG{l+s+s2}{(a)}\PYG{l+s+se}{\PYGZbs{}n}\PYG{l+s+s2}{A solução do sistema linear é: }\PYG{l+s+se}{\PYGZbs{}n}\PYG{l+s+s2}{\PYGZdq{}}\PYG{p}{,} \PYG{n}{X}\PYG{p}{)}

\PYG{c+c1}{\PYGZsh{} (b)}
\PYG{n}{AB} \PYG{o}{=} \PYG{n}{np}\PYG{o}{.}\PYG{n}{array}\PYG{p}{(}\PYG{p}{[}\PYG{p}{[}\PYG{l+m+mf}{6.}\PYG{p}{,} \PYG{o}{\PYGZhy{}}\PYG{l+m+mf}{1.}\PYG{p}{,} \PYG{o}{\PYGZhy{}}\PYG{l+m+mf}{1.}\PYG{p}{,} \PYG{l+m+mf}{3.}\PYG{p}{]}\PYG{p}{,} \PYG{p}{[}\PYG{l+m+mf}{6.}\PYG{p}{,} \PYG{l+m+mf}{9.}\PYG{p}{,} \PYG{l+m+mf}{1.}\PYG{p}{,} \PYG{l+m+mf}{40.}\PYG{p}{]}\PYG{p}{,} \PYG{p}{[}\PYG{o}{\PYGZhy{}}\PYG{l+m+mf}{3.}\PYG{p}{,} \PYG{l+m+mf}{1.}\PYG{p}{,} \PYG{l+m+mf}{12.}\PYG{p}{,} \PYG{l+m+mf}{50.}\PYG{p}{]}\PYG{p}{]}\PYG{p}{)}
\PYG{n}{ER} \PYG{o}{=} \PYG{l+m+mi}{5}\PYG{o}{/}\PYG{l+m+mi}{100}
\PYG{n}{X0} \PYG{o}{=} \PYG{n}{np}\PYG{o}{.}\PYG{n}{array}\PYG{p}{(}\PYG{p}{[}\PYG{l+m+mf}{0.}\PYG{p}{,} \PYG{l+m+mf}{0.}\PYG{p}{,} \PYG{l+m+mf}{0.}\PYG{p}{]}\PYG{p}{)}
\PYG{n}{X} \PYG{o}{=} \PYG{n}{gaussseidel}\PYG{p}{(}\PYG{n}{AB}\PYG{p}{,}\PYG{n}{ER}\PYG{p}{,}\PYG{n}{X0}\PYG{p}{)}
\PYG{n+nb}{print}\PYG{p}{(}\PYG{l+s+s2}{\PYGZdq{}}\PYG{l+s+se}{\PYGZbs{}n}\PYG{l+s+s2}{(b)}\PYG{l+s+se}{\PYGZbs{}n}\PYG{l+s+s2}{A solução do sistema linear é: }\PYG{l+s+se}{\PYGZbs{}n}\PYG{l+s+s2}{\PYGZdq{}}\PYG{p}{,} \PYG{n}{X}\PYG{p}{)}

\PYG{c+c1}{\PYGZsh{} (c)}
\PYG{n}{AB} \PYG{o}{=} \PYG{n}{np}\PYG{o}{.}\PYG{n}{array}\PYG{p}{(}\PYG{p}{[}\PYG{p}{[}\PYG{o}{\PYGZhy{}}\PYG{l+m+mf}{8.}\PYG{p}{,} \PYG{l+m+mf}{1.}\PYG{p}{,} \PYG{o}{\PYGZhy{}}\PYG{l+m+mf}{2.}\PYG{p}{,} \PYG{o}{\PYGZhy{}}\PYG{l+m+mf}{20.}\PYG{p}{]}\PYG{p}{,} \PYG{p}{[}\PYG{l+m+mf}{2.}\PYG{p}{,} \PYG{o}{\PYGZhy{}}\PYG{l+m+mf}{6.}\PYG{p}{,} \PYG{o}{\PYGZhy{}}\PYG{l+m+mf}{1.}\PYG{p}{,} \PYG{o}{\PYGZhy{}}\PYG{l+m+mf}{38.}\PYG{p}{]}\PYG{p}{,} \PYG{p}{[}\PYG{o}{\PYGZhy{}}\PYG{l+m+mf}{3.}\PYG{p}{,} \PYG{o}{\PYGZhy{}}\PYG{l+m+mf}{1.}\PYG{p}{,} \PYG{l+m+mf}{7.}\PYG{p}{,} \PYG{o}{\PYGZhy{}}\PYG{l+m+mf}{34.}\PYG{p}{]}\PYG{p}{]}\PYG{p}{)}
\PYG{n}{ER} \PYG{o}{=} \PYG{l+m+mi}{5}\PYG{o}{/}\PYG{l+m+mi}{100}
\PYG{n}{X0} \PYG{o}{=} \PYG{n}{np}\PYG{o}{.}\PYG{n}{array}\PYG{p}{(}\PYG{p}{[}\PYG{l+m+mf}{0.}\PYG{p}{,} \PYG{l+m+mf}{0.}\PYG{p}{,} \PYG{l+m+mf}{0.}\PYG{p}{]}\PYG{p}{)}
\PYG{n}{X} \PYG{o}{=} \PYG{n}{gaussseidel}\PYG{p}{(}\PYG{n}{AB}\PYG{p}{,}\PYG{n}{ER}\PYG{p}{,}\PYG{n}{X0}\PYG{p}{)}
\PYG{n+nb}{print}\PYG{p}{(}\PYG{l+s+s2}{\PYGZdq{}}\PYG{l+s+se}{\PYGZbs{}n}\PYG{l+s+s2}{(c)}\PYG{l+s+se}{\PYGZbs{}n}\PYG{l+s+s2}{A solução do sistema linear é: }\PYG{l+s+se}{\PYGZbs{}n}\PYG{l+s+s2}{\PYGZdq{}}\PYG{p}{,} \PYG{n}{X}\PYG{p}{)}
\end{sphinxVerbatim}

\begin{sphinxVerbatim}[commandchars=\\\{\}]
(a)
A solução do sistema linear é: 
 [ 0.5  8.  \PYGZhy{}6. ]

(b)
A solução do sistema linear é: 
 [1.697 2.829 4.355]

(c)
A solução do sistema linear é: 
 [ 4.005  7.992 \PYGZhy{}1.999]
\end{sphinxVerbatim}


\subsection{Questão 4}
\label{\detokenize{lista-4-solucoes:questao-4}}
\sphinxAtStartPar
Dos seguintes sistemas abaixo identifique qual(is) deles nao pode(m) ser resolvido(s) usando o método iterativo de Gauss\sphinxhyphen{}Seidel.

\sphinxAtStartPar
(a) \textbackslash{}begin\{cases\}
9x + 3y + z = 13 \textbackslash{}
−6x + 8z = 2 \textbackslash{}
2x + 5y − z = 6
\textbackslash{}end\{cases\}

\sphinxAtStartPar
(b) \textbackslash{}begin\{cases\}
x + y + 6z = 8 \textbackslash{}
x + 5y − z = 5 \textbackslash{}
4x + 2y − 2z = 4
\textbackslash{}end\{cases\}

\sphinxAtStartPar
© \textbackslash{}begin\{cases\}
−3x + 4y + 5z = 6 \textbackslash{}
−2x + 2y − 3z = −3 \textbackslash{}
2y − z = 1
\textbackslash{}end\{cases\}


\subsubsection{Solução}
\label{\detokenize{lista-4-solucoes:solucao}}
\sphinxAtStartPar
(a) Convergente

\sphinxAtStartPar
(b) Divergente

\sphinxAtStartPar
© Divergente


\subsection{Questão 5}
\label{\detokenize{lista-4-solucoes:questao-5}}
\sphinxAtStartPar
Uma companhia de eletrônica produz transistores, resistores e chips de computador. Cada transistor usa quatro unidades de cobre, uma unidade de zinco e duas unidades de vidro. Cada resisitor usa três, três e uma unidades de cada material, respectivamente, e cada chip de computador usa duas, uma e três unidades desses materiais, respectivamente. Colocando essas
informações em uma tabela, tem\sphinxhyphen{}se:


\begin{savenotes}\sphinxattablestart
\centering
\begin{tabulary}{\linewidth}[t]{|T|T|T|T|}
\hline
\sphinxstyletheadfamily 
\sphinxAtStartPar
Componente
&\sphinxstyletheadfamily 
\sphinxAtStartPar
Cobre
&\sphinxstyletheadfamily 
\sphinxAtStartPar
Zinco
&\sphinxstyletheadfamily 
\sphinxAtStartPar
Vidro
\\
\hline
\sphinxAtStartPar
Transistor
&
\sphinxAtStartPar
4
&
\sphinxAtStartPar
1
&
\sphinxAtStartPar
2
\\
\hline
\sphinxAtStartPar
Resistor
&
\sphinxAtStartPar
3
&
\sphinxAtStartPar
3
&
\sphinxAtStartPar
1
\\
\hline
\sphinxAtStartPar
Chip de Computador
&
\sphinxAtStartPar
2
&
\sphinxAtStartPar
1
&
\sphinxAtStartPar
3
\\
\hline
\end{tabulary}
\par
\sphinxattableend\end{savenotes}

\sphinxAtStartPar
O fornecimento desses materiais varia de semana para semana. Assim, a companhia precisa determinar uma meta de produção diferente para cada semana. Por exemplo, em uma semana a quantidade total de materiais disponíveis era 960 unidades de cobre, 510 unidades de zinco e 610 unidades de vidro. Determine o sistema de equações que modela essa meta de produção e determine a sua solução, pelo método de Gauss\sphinxhyphen{}Seidel, para determinar o número de transistores, resistores e chips de computador fabricados nessa semana.

\begin{sphinxVerbatim}[commandchars=\\\{\}]
\PYG{c+c1}{\PYGZsh{} Solução}

\PYG{n}{AB} \PYG{o}{=} \PYG{n}{np}\PYG{o}{.}\PYG{n}{array}\PYG{p}{(}\PYG{p}{[}\PYG{p}{[}\PYG{l+m+mf}{4.}\PYG{p}{,} \PYG{l+m+mf}{3.}\PYG{p}{,} \PYG{l+m+mf}{2.}\PYG{p}{,} \PYG{l+m+mf}{960.}\PYG{p}{]}\PYG{p}{,} \PYG{p}{[}\PYG{l+m+mf}{1.}\PYG{p}{,} \PYG{l+m+mf}{3.}\PYG{p}{,} \PYG{l+m+mf}{1.}\PYG{p}{,} \PYG{l+m+mf}{510.}\PYG{p}{]}\PYG{p}{,} \PYG{p}{[}\PYG{l+m+mf}{2.}\PYG{p}{,} \PYG{l+m+mf}{1.}\PYG{p}{,} \PYG{l+m+mf}{3.}\PYG{p}{,} \PYG{l+m+mf}{610.}\PYG{p}{]}\PYG{p}{]}\PYG{p}{)}
\PYG{n}{ER} \PYG{o}{=} \PYG{l+m+mi}{1}\PYG{o}{/}\PYG{l+m+mi}{100}
\PYG{n}{X0} \PYG{o}{=} \PYG{n}{np}\PYG{o}{.}\PYG{n}{array}\PYG{p}{(}\PYG{p}{[}\PYG{p}{[}\PYG{l+m+mf}{0.}\PYG{p}{]}\PYG{p}{,} \PYG{p}{[}\PYG{l+m+mf}{0.}\PYG{p}{]}\PYG{p}{,} \PYG{p}{[}\PYG{l+m+mf}{0.}\PYG{p}{]}\PYG{p}{]}\PYG{p}{)}

\PYG{n}{X} \PYG{o}{=} \PYG{n}{gaussseidel}\PYG{p}{(}\PYG{n}{AB}\PYG{p}{,}\PYG{n}{ER}\PYG{p}{,}\PYG{n}{X0}\PYG{p}{)}
\PYG{n}{X} \PYG{o}{=} \PYG{n}{np}\PYG{o}{.}\PYG{n}{transpose}\PYG{p}{(}\PYG{n}{np}\PYG{o}{.}\PYG{n}{rint}\PYG{p}{(}\PYG{n}{X}\PYG{p}{)}\PYG{p}{)}
\PYG{n+nb}{print}\PYG{p}{(}\PYG{l+s+s2}{\PYGZdq{}}\PYG{l+s+s2}{[T,R,C] =}\PYG{l+s+s2}{\PYGZdq{}}\PYG{p}{,} \PYG{n}{X}\PYG{p}{[}\PYG{l+m+mi}{0}\PYG{p}{]}\PYG{p}{)}
\end{sphinxVerbatim}

\begin{sphinxVerbatim}[commandchars=\\\{\}]
[T,R,C] = [120. 100.  90.]
\end{sphinxVerbatim}


\section{Sistemas não\sphinxhyphen{}lineares}
\label{\detokenize{lista-4-solucoes:sistemas-nao-lineares}}

\subsection{Questão 6}
\label{\detokenize{lista-4-solucoes:questao-6}}
\sphinxAtStartPar
Usando o Método de Newton, determinar uma raiz para cada sistema não\sphinxhyphen{}linear abaixo com precisão \(\epsilon= 10^{-3}\):

\sphinxAtStartPar
(i) \$\(\begin{cases} x^2 + y^2 = 2 \\ x^2 - y^2 = 1 \end{cases} \quad 
x^{(0)} = \begin{bmatrix} 1.2 \\ 0.7 \end{bmatrix}\)\$

\sphinxAtStartPar
(ii) \$\(\begin{cases} 3x^2 y - y^3 = 4 \\ x^2 - x y^3 = 9 \end{cases} \quad
x^{(0)} = \begin{bmatrix} -1 \\ -2 \end{bmatrix}\)\$

\sphinxAtStartPar
(i) \$\(\begin{cases} (x-1)^2 + y^2 = 4 \\ x^2 + (y-1)^2 = 4 \end{cases} \quad
x^{(0)} = \begin{bmatrix} 2 \\ 1 \end{bmatrix}\)\$

\begin{sphinxVerbatim}[commandchars=\\\{\}]
\PYG{c+c1}{\PYGZsh{} Solução}

\PYG{n}{x}\PYG{p}{,} \PYG{n}{y} \PYG{o}{=} \PYG{n}{sy}\PYG{o}{.}\PYG{n}{symbols}\PYG{p}{(}\PYG{l+s+s2}{\PYGZdq{}}\PYG{l+s+s2}{x, y}\PYG{l+s+s2}{\PYGZdq{}}\PYG{p}{)}
\PYG{n}{Xs} \PYG{o}{=} \PYG{n}{sy}\PYG{o}{.}\PYG{n}{Matrix}\PYG{p}{(}\PYG{p}{[}\PYG{n}{x}\PYG{p}{,} \PYG{n}{y}\PYG{p}{]}\PYG{p}{)}
\PYG{n}{ER} \PYG{o}{=} \PYG{l+m+mi}{1}\PYG{o}{/}\PYG{l+m+mi}{100}

\PYG{c+c1}{\PYGZsh{} (i)}
\PYG{n}{F} \PYG{o}{=} \PYG{n}{sy}\PYG{o}{.}\PYG{n}{Matrix}\PYG{p}{(}\PYG{p}{[}\PYG{n}{x}\PYG{o}{*}\PYG{o}{*}\PYG{l+m+mi}{2} \PYG{o}{+} \PYG{n}{y}\PYG{o}{*}\PYG{o}{*}\PYG{l+m+mi}{2} \PYG{o}{\PYGZhy{}} \PYG{l+m+mi}{2}\PYG{p}{,} \PYG{n}{x}\PYG{o}{*}\PYG{o}{*}\PYG{l+m+mi}{2} \PYG{o}{\PYGZhy{}} \PYG{n}{y}\PYG{o}{*}\PYG{o}{*}\PYG{l+m+mi}{2} \PYG{o}{\PYGZhy{}} \PYG{l+m+mi}{1}\PYG{p}{]}\PYG{p}{)}
\PYG{n}{X0} \PYG{o}{=} \PYG{n}{np}\PYG{o}{.}\PYG{n}{array}\PYG{p}{(}\PYG{p}{[}\PYG{l+m+mf}{1.2}\PYG{p}{,} \PYG{l+m+mf}{0.7}\PYG{p}{]}\PYG{p}{)}
\PYG{n}{X} \PYG{o}{=} \PYG{n}{newtonnaolin}\PYG{p}{(}\PYG{n}{F}\PYG{p}{,}\PYG{n}{Xs}\PYG{p}{,}\PYG{n}{ER}\PYG{p}{,}\PYG{n}{X0}\PYG{p}{)}
\PYG{n+nb}{print}\PYG{p}{(}\PYG{l+s+s2}{\PYGZdq{}}\PYG{l+s+s2}{(i) A solução do problema é:}\PYG{l+s+s2}{\PYGZdq{}}\PYG{p}{,} \PYG{n}{X}\PYG{p}{)}

\PYG{c+c1}{\PYGZsh{} (ii)}
\PYG{n}{F} \PYG{o}{=} \PYG{n}{sy}\PYG{o}{.}\PYG{n}{Matrix}\PYG{p}{(}\PYG{p}{[}\PYG{l+m+mi}{3}\PYG{o}{*}\PYG{p}{(}\PYG{n}{x}\PYG{o}{*}\PYG{o}{*}\PYG{l+m+mi}{2}\PYG{p}{)}\PYG{o}{*}\PYG{n}{y} \PYG{o}{\PYGZhy{}} \PYG{n}{y}\PYG{o}{*}\PYG{o}{*}\PYG{l+m+mi}{3} \PYG{o}{\PYGZhy{}} \PYG{l+m+mi}{4}\PYG{p}{,} \PYG{n}{x}\PYG{o}{*}\PYG{o}{*}\PYG{l+m+mi}{2} \PYG{o}{\PYGZhy{}} \PYG{n}{x}\PYG{o}{*}\PYG{n}{y}\PYG{o}{*}\PYG{o}{*}\PYG{l+m+mi}{3} \PYG{o}{\PYGZhy{}} \PYG{l+m+mi}{9}\PYG{p}{]}\PYG{p}{)}
\PYG{n}{X0} \PYG{o}{=} \PYG{n}{np}\PYG{o}{.}\PYG{n}{array}\PYG{p}{(}\PYG{p}{[}\PYG{o}{\PYGZhy{}}\PYG{l+m+mf}{1.}\PYG{p}{,} \PYG{o}{\PYGZhy{}}\PYG{l+m+mf}{2.}\PYG{p}{]}\PYG{p}{)}
\PYG{n}{X} \PYG{o}{=} \PYG{n}{newtonnaolin}\PYG{p}{(}\PYG{n}{F}\PYG{p}{,}\PYG{n}{Xs}\PYG{p}{,}\PYG{n}{ER}\PYG{p}{,}\PYG{n}{X0}\PYG{p}{)}
\PYG{n+nb}{print}\PYG{p}{(}\PYG{l+s+s2}{\PYGZdq{}}\PYG{l+s+se}{\PYGZbs{}n}\PYG{l+s+s2}{(ii) A solução do problema é:}\PYG{l+s+s2}{\PYGZdq{}}\PYG{p}{,} \PYG{n}{X}\PYG{p}{)}

\PYG{c+c1}{\PYGZsh{} (iii)}
\PYG{n}{F} \PYG{o}{=} \PYG{n}{sy}\PYG{o}{.}\PYG{n}{Matrix}\PYG{p}{(}\PYG{p}{[}\PYG{p}{(}\PYG{n}{x}\PYG{o}{\PYGZhy{}}\PYG{l+m+mi}{1}\PYG{p}{)}\PYG{o}{*}\PYG{o}{*}\PYG{l+m+mi}{2} \PYG{o}{+} \PYG{n}{y}\PYG{o}{*}\PYG{o}{*}\PYG{l+m+mi}{2} \PYG{o}{\PYGZhy{}} \PYG{l+m+mi}{4}\PYG{p}{,} \PYG{n}{x}\PYG{o}{*}\PYG{o}{*}\PYG{l+m+mi}{2} \PYG{o}{+} \PYG{p}{(}\PYG{n}{y}\PYG{o}{\PYGZhy{}}\PYG{l+m+mi}{1}\PYG{p}{)}\PYG{o}{*}\PYG{o}{*}\PYG{l+m+mi}{2} \PYG{o}{\PYGZhy{}} \PYG{l+m+mi}{4}\PYG{p}{]}\PYG{p}{)}
\PYG{n}{X0} \PYG{o}{=} \PYG{n}{np}\PYG{o}{.}\PYG{n}{array}\PYG{p}{(}\PYG{p}{[}\PYG{l+m+mf}{2.}\PYG{p}{,} \PYG{l+m+mf}{1.}\PYG{p}{]}\PYG{p}{)}
\PYG{n}{X} \PYG{o}{=} \PYG{n}{newtonnaolin}\PYG{p}{(}\PYG{n}{F}\PYG{p}{,}\PYG{n}{Xs}\PYG{p}{,}\PYG{n}{ER}\PYG{p}{,}\PYG{n}{X0}\PYG{p}{)}
\PYG{n+nb}{print}\PYG{p}{(}\PYG{l+s+s2}{\PYGZdq{}}\PYG{l+s+se}{\PYGZbs{}n}\PYG{l+s+s2}{(iii) A solução do problema é:}\PYG{l+s+s2}{\PYGZdq{}}\PYG{p}{,} \PYG{n}{X}\PYG{p}{)}
\end{sphinxVerbatim}

\begin{sphinxVerbatim}[commandchars=\\\{\}]
(i) A solução do problema é: [1.225 0.707]

(ii) A solução do problema é: [3.002 0.148]

(iii) A solução do problema é: [1.823 1.823]
\end{sphinxVerbatim}


\section{Interpolação Polinomial}
\label{\detokenize{lista-4-solucoes:interpolacao-polinomial}}

\subsection{Questão 7}
\label{\detokenize{lista-4-solucoes:questao-7}}
\sphinxAtStartPar
Use um polinômio interpolador de Lagrange de primeiro e segundo graus para calcular \(ln(2)\), com base nos seguintes dados:


\begin{savenotes}\sphinxattablestart
\centering
\begin{tabulary}{\linewidth}[t]{|T|T|}
\hline
\sphinxstyletheadfamily 
\sphinxAtStartPar
\(x_i\)
&\sphinxstyletheadfamily 
\sphinxAtStartPar
\(f(x_i)\)
\\
\hline
\sphinxAtStartPar
1
&
\sphinxAtStartPar
0
\\
\hline
\sphinxAtStartPar
4
&
\sphinxAtStartPar
1.386294
\\
\hline
\sphinxAtStartPar
6
&
\sphinxAtStartPar
1.791759
\\
\hline
\end{tabulary}
\par
\sphinxattableend\end{savenotes}

\begin{sphinxVerbatim}[commandchars=\\\{\}]
\PYG{c+c1}{\PYGZsh{} Solução}

\PYG{n}{X} \PYG{o}{=} \PYG{p}{[}\PYG{l+m+mi}{1}\PYG{p}{,} \PYG{l+m+mi}{4}\PYG{p}{,} \PYG{l+m+mi}{6}\PYG{p}{]}
\PYG{n}{Y} \PYG{o}{=} \PYG{p}{[}\PYG{l+m+mi}{0}\PYG{p}{,} \PYG{l+m+mf}{1.386294}\PYG{p}{,} \PYG{l+m+mf}{1.791759}\PYG{p}{]}

\PYG{c+c1}{\PYGZsh{} (i) Polinômio de Primeiro Grau}
\PYG{n}{y1} \PYG{o}{=} \PYG{n}{int\PYGZus{}lagrange}\PYG{p}{(}\PYG{n}{X}\PYG{p}{[}\PYG{l+m+mi}{0}\PYG{p}{:}\PYG{l+m+mi}{2}\PYG{p}{]}\PYG{p}{,}\PYG{n}{Y}\PYG{p}{[}\PYG{l+m+mi}{0}\PYG{p}{:}\PYG{l+m+mi}{2}\PYG{p}{]}\PYG{p}{,}\PYG{l+m+mi}{2}\PYG{p}{)}
\PYG{n+nb}{print}\PYG{p}{(}\PYG{l+s+s2}{\PYGZdq{}}\PYG{l+s+s2}{(i)}\PYG{l+s+se}{\PYGZbs{}n}\PYG{l+s+s2}{A estimativa de ln(2) para o polinômio de primeiro grau é:}\PYG{l+s+s2}{\PYGZdq{}}\PYG{p}{,} \PYG{n}{y1}\PYG{p}{)}

\PYG{c+c1}{\PYGZsh{} (i) Polinômio de Segundo Grau}
\PYG{n}{y2} \PYG{o}{=} \PYG{n}{int\PYGZus{}lagrange}\PYG{p}{(}\PYG{n}{X}\PYG{p}{,}\PYG{n}{Y}\PYG{p}{,}\PYG{l+m+mi}{2}\PYG{p}{)}
\PYG{n+nb}{print}\PYG{p}{(}\PYG{l+s+s2}{\PYGZdq{}}\PYG{l+s+se}{\PYGZbs{}n}\PYG{l+s+s2}{(ii)}\PYG{l+s+se}{\PYGZbs{}n}\PYG{l+s+s2}{A estimativa de ln(2) para o polinômio de segundo grau é:}\PYG{l+s+s2}{\PYGZdq{}}\PYG{p}{,} \PYG{n}{y2}\PYG{p}{)}

\PYG{n}{plt}\PYG{o}{.}\PYG{n}{plot}\PYG{p}{(}\PYG{n}{X}\PYG{p}{,}\PYG{n}{Y}\PYG{p}{,} \PYG{l+s+s1}{\PYGZsq{}}\PYG{l+s+s1}{o\PYGZhy{}}\PYG{l+s+s1}{\PYGZsq{}}\PYG{p}{)}
\PYG{n}{plt}\PYG{o}{.}\PYG{n}{plot}\PYG{p}{(}\PYG{l+m+mi}{2}\PYG{p}{,}\PYG{n}{y1}\PYG{p}{,} \PYG{l+s+s1}{\PYGZsq{}}\PYG{l+s+s1}{or}\PYG{l+s+s1}{\PYGZsq{}}\PYG{p}{,} \PYG{l+m+mi}{2}\PYG{p}{,}\PYG{n}{y2}\PYG{p}{,} \PYG{l+s+s1}{\PYGZsq{}}\PYG{l+s+s1}{og}\PYG{l+s+s1}{\PYGZsq{}}\PYG{p}{)}
\PYG{n}{plt}\PYG{o}{.}\PYG{n}{legend}\PYG{p}{(}\PYG{p}{[}\PYG{l+s+s1}{\PYGZsq{}}\PYG{l+s+s1}{Spline Linear de y = ln(x)}\PYG{l+s+s1}{\PYGZsq{}}\PYG{p}{,} \PYG{l+s+s1}{\PYGZsq{}}\PYG{l+s+s1}{Interpolação de 1°}\PYG{l+s+s1}{\PYGZsq{}}\PYG{p}{,} \PYG{l+s+s1}{\PYGZsq{}}\PYG{l+s+s1}{Interpolação de 2°}\PYG{l+s+s1}{\PYGZsq{}}\PYG{p}{]}\PYG{p}{)}\PYG{p}{;}
\end{sphinxVerbatim}

\begin{sphinxVerbatim}[commandchars=\\\{\}]
(i)
A estimativa de ln(2) para o polinômio de primeiro grau é: 0.46209799999999995

(ii)
A estimativa de ln(2) para o polinômio de segundo grau é: 0.5658441999999999
\end{sphinxVerbatim}

\noindent\sphinxincludegraphics{{lista-4-solucoes_20_1}.png}


\subsection{Questão 8}
\label{\detokenize{lista-4-solucoes:questao-8}}
\sphinxAtStartPar
Faça uma estimativa de \(log(10)\) usando interpolação linear

\sphinxAtStartPar
(i) entre \(log(8) = 0.9030900\) e \(log(12) = 1.0791812\)

\sphinxAtStartPar
(ii) entre \(log(9) = 0.9542425\) e \(log(11) = 1.0413927\)

\begin{sphinxVerbatim}[commandchars=\\\{\}]
\PYG{c+c1}{\PYGZsh{} Solução}

\PYG{c+c1}{\PYGZsh{} (i)}
\PYG{n}{X} \PYG{o}{=} \PYG{p}{[}\PYG{l+m+mi}{8}\PYG{p}{,} \PYG{l+m+mi}{12}\PYG{p}{]}
\PYG{n}{Y} \PYG{o}{=} \PYG{p}{[}\PYG{l+m+mf}{0.9030900}\PYG{p}{,} \PYG{l+m+mf}{1.0791812}\PYG{p}{]}

\PYG{n}{y1} \PYG{o}{=} \PYG{n}{int\PYGZus{}lagrange}\PYG{p}{(}\PYG{n}{X}\PYG{p}{,}\PYG{n}{Y}\PYG{p}{,}\PYG{l+m+mi}{10}\PYG{p}{)}
\PYG{n+nb}{print}\PYG{p}{(}\PYG{l+s+s2}{\PYGZdq{}}\PYG{l+s+s2}{(i)}\PYG{l+s+se}{\PYGZbs{}n}\PYG{l+s+s2}{A estimativa de ln(10) é:}\PYG{l+s+s2}{\PYGZdq{}}\PYG{p}{,} \PYG{n}{y1}\PYG{p}{)}

\PYG{c+c1}{\PYGZsh{} (i)}
\PYG{n}{X} \PYG{o}{=} \PYG{p}{[}\PYG{l+m+mi}{9}\PYG{p}{,} \PYG{l+m+mi}{11}\PYG{p}{]}
\PYG{n}{Y} \PYG{o}{=} \PYG{p}{[}\PYG{l+m+mf}{0.9542425}\PYG{p}{,} \PYG{l+m+mf}{1.0413927}\PYG{p}{]}

\PYG{n}{y2} \PYG{o}{=} \PYG{n}{int\PYGZus{}lagrange}\PYG{p}{(}\PYG{n}{X}\PYG{p}{,}\PYG{n}{Y}\PYG{p}{,}\PYG{l+m+mi}{10}\PYG{p}{)}
\PYG{n+nb}{print}\PYG{p}{(}\PYG{l+s+s2}{\PYGZdq{}}\PYG{l+s+se}{\PYGZbs{}n}\PYG{l+s+s2}{(ii)}\PYG{l+s+se}{\PYGZbs{}n}\PYG{l+s+s2}{A estimativa de ln(10) é:}\PYG{l+s+s2}{\PYGZdq{}}\PYG{p}{,} \PYG{n}{y2}\PYG{p}{)}

\PYG{n}{X} \PYG{o}{=} \PYG{p}{[}\PYG{l+m+mi}{8}\PYG{p}{,} \PYG{l+m+mi}{9}\PYG{p}{,} \PYG{l+m+mi}{11}\PYG{p}{,} \PYG{l+m+mi}{12}\PYG{p}{]}
\PYG{n}{Y} \PYG{o}{=} \PYG{p}{[}\PYG{l+m+mf}{0.9030900}\PYG{p}{,} \PYG{l+m+mf}{0.9542425}\PYG{p}{,} \PYG{l+m+mf}{1.0413927}\PYG{p}{,} \PYG{l+m+mf}{1.0791812}\PYG{p}{]}

\PYG{n}{plt}\PYG{o}{.}\PYG{n}{plot}\PYG{p}{(}\PYG{n}{X}\PYG{p}{,}\PYG{n}{Y}\PYG{p}{,} \PYG{l+s+s1}{\PYGZsq{}}\PYG{l+s+s1}{o\PYGZhy{}}\PYG{l+s+s1}{\PYGZsq{}}\PYG{p}{)}
\PYG{n}{plt}\PYG{o}{.}\PYG{n}{plot}\PYG{p}{(}\PYG{l+m+mi}{10}\PYG{p}{,}\PYG{n}{y1}\PYG{p}{,} \PYG{l+s+s1}{\PYGZsq{}}\PYG{l+s+s1}{or}\PYG{l+s+s1}{\PYGZsq{}}\PYG{p}{,} \PYG{l+m+mi}{10}\PYG{p}{,}\PYG{n}{y2}\PYG{p}{,} \PYG{l+s+s1}{\PYGZsq{}}\PYG{l+s+s1}{og}\PYG{l+s+s1}{\PYGZsq{}}\PYG{p}{)}
\PYG{n}{plt}\PYG{o}{.}\PYG{n}{legend}\PYG{p}{(}\PYG{p}{[}\PYG{l+s+s1}{\PYGZsq{}}\PYG{l+s+s1}{Spline Linear de y = log(x)}\PYG{l+s+s1}{\PYGZsq{}}\PYG{p}{,} \PYG{l+s+s1}{\PYGZsq{}}\PYG{l+s+s1}{Interpolação (i)}\PYG{l+s+s1}{\PYGZsq{}}\PYG{p}{,} \PYG{l+s+s1}{\PYGZsq{}}\PYG{l+s+s1}{Interpolação (ii)}\PYG{l+s+s1}{\PYGZsq{}}\PYG{p}{]}\PYG{p}{)}\PYG{p}{;}
\end{sphinxVerbatim}

\begin{sphinxVerbatim}[commandchars=\\\{\}]
(i)
A estimativa de ln(10) é: 0.9911356000000002

(ii)
A estimativa de ln(10) é: 0.9978176
\end{sphinxVerbatim}

\noindent\sphinxincludegraphics{{lista-4-solucoes_22_1}.png}


\subsection{Questão 9}
\label{\detokenize{lista-4-solucoes:questao-9}}
\sphinxAtStartPar
Considere os dados e faça o que se pede:


\begin{savenotes}\sphinxattablestart
\centering
\begin{tabulary}{\linewidth}[t]{|T|T|}
\hline
\sphinxstyletheadfamily 
\sphinxAtStartPar
x
&\sphinxstyletheadfamily 
\sphinxAtStartPar
y
\\
\hline
\sphinxAtStartPar
1
&
\sphinxAtStartPar
3
\\
\hline
\sphinxAtStartPar
2
&
\sphinxAtStartPar
6
\\
\hline
\sphinxAtStartPar
3
&
\sphinxAtStartPar
19
\\
\hline
\sphinxAtStartPar
5
&
\sphinxAtStartPar
99
\\
\hline
\sphinxAtStartPar
7
&
\sphinxAtStartPar
291
\\
\hline
\sphinxAtStartPar
8
&
\sphinxAtStartPar
444
\\
\hline
\end{tabulary}
\par
\sphinxattableend\end{savenotes}

\sphinxAtStartPar
(i) Calcule f(4) usando polinomios interpoladores de Lagrange de primeiro a terceiro graus.

\sphinxAtStartPar
(ii) Calcule f(4) usando polinomios interpoladores de Newton de primeiro a quarto graus.

\begin{sphinxVerbatim}[commandchars=\\\{\}]
\PYG{c+c1}{\PYGZsh{} Solução}

\PYG{n}{X} \PYG{o}{=} \PYG{n}{np}\PYG{o}{.}\PYG{n}{array}\PYG{p}{(}\PYG{p}{[}\PYG{l+m+mi}{1}\PYG{p}{,} \PYG{l+m+mi}{2}\PYG{p}{,} \PYG{l+m+mi}{3}\PYG{p}{,} \PYG{l+m+mi}{5}\PYG{p}{,} \PYG{l+m+mi}{7}\PYG{p}{,} \PYG{l+m+mi}{8}\PYG{p}{]}\PYG{p}{)}
\PYG{n}{Y} \PYG{o}{=} \PYG{n}{np}\PYG{o}{.}\PYG{n}{array}\PYG{p}{(}\PYG{p}{[}\PYG{l+m+mi}{3}\PYG{p}{,} \PYG{l+m+mi}{6}\PYG{p}{,} \PYG{l+m+mi}{19}\PYG{p}{,} \PYG{l+m+mi}{99}\PYG{p}{,} \PYG{l+m+mi}{291}\PYG{p}{,} \PYG{l+m+mi}{444}\PYG{p}{]}\PYG{p}{)}

\PYG{c+c1}{\PYGZsh{} (i)}
\PYG{n}{y1} \PYG{o}{=} \PYG{n}{int\PYGZus{}lagrange}\PYG{p}{(}\PYG{n}{X}\PYG{p}{[}\PYG{l+m+mi}{2}\PYG{p}{:}\PYG{l+m+mi}{4}\PYG{p}{]}\PYG{p}{,}\PYG{n}{Y}\PYG{p}{[}\PYG{l+m+mi}{2}\PYG{p}{:}\PYG{l+m+mi}{4}\PYG{p}{]}\PYG{p}{,}\PYG{l+m+mi}{4}\PYG{p}{)}
\PYG{n+nb}{print}\PYG{p}{(}\PYG{l+s+s2}{\PYGZdq{}}\PYG{l+s+s2}{(i)}\PYG{l+s+se}{\PYGZbs{}n}\PYG{l+s+s2}{A estimativa de f(4) de primeiro grau é:}\PYG{l+s+s2}{\PYGZdq{}}\PYG{p}{,} \PYG{n}{y1}\PYG{p}{)}
\PYG{n}{y2} \PYG{o}{=} \PYG{n}{int\PYGZus{}lagrange}\PYG{p}{(}\PYG{n}{X}\PYG{p}{[}\PYG{l+m+mi}{1}\PYG{p}{:}\PYG{l+m+mi}{4}\PYG{p}{]}\PYG{p}{,}\PYG{n}{Y}\PYG{p}{[}\PYG{l+m+mi}{1}\PYG{p}{:}\PYG{l+m+mi}{4}\PYG{p}{]}\PYG{p}{,}\PYG{l+m+mi}{4}\PYG{p}{)}
\PYG{n+nb}{print}\PYG{p}{(}\PYG{l+s+s2}{\PYGZdq{}}\PYG{l+s+s2}{A estimativa de f(4) de segundo grau é:}\PYG{l+s+s2}{\PYGZdq{}}\PYG{p}{,} \PYG{n}{y2}\PYG{p}{)}
\PYG{n}{y3} \PYG{o}{=} \PYG{n}{int\PYGZus{}lagrange}\PYG{p}{(}\PYG{n}{X}\PYG{p}{[}\PYG{p}{:}\PYG{l+m+mi}{4}\PYG{p}{]}\PYG{p}{,}\PYG{n}{Y}\PYG{p}{[}\PYG{p}{:}\PYG{l+m+mi}{4}\PYG{p}{]}\PYG{p}{,}\PYG{l+m+mi}{4}\PYG{p}{)}
\PYG{n+nb}{print}\PYG{p}{(}\PYG{l+s+s2}{\PYGZdq{}}\PYG{l+s+s2}{A estimativa de f(4) de terceiro grau é:}\PYG{l+s+s2}{\PYGZdq{}}\PYG{p}{,} \PYG{n}{y3}\PYG{p}{)}

\PYG{c+c1}{\PYGZsh{} (i)}
\PYG{n}{y4} \PYG{o}{=} \PYG{n}{int\PYGZus{}newton}\PYG{p}{(}\PYG{n}{X}\PYG{p}{[}\PYG{l+m+mi}{2}\PYG{p}{:}\PYG{l+m+mi}{4}\PYG{p}{]}\PYG{p}{,}\PYG{n}{Y}\PYG{p}{[}\PYG{l+m+mi}{2}\PYG{p}{:}\PYG{l+m+mi}{4}\PYG{p}{]}\PYG{p}{,}\PYG{l+m+mi}{4}\PYG{p}{)}
\PYG{n+nb}{print}\PYG{p}{(}\PYG{l+s+s2}{\PYGZdq{}}\PYG{l+s+se}{\PYGZbs{}n}\PYG{l+s+s2}{(ii)}\PYG{l+s+se}{\PYGZbs{}n}\PYG{l+s+s2}{A estimativa de f(4) de primeiro grau é:}\PYG{l+s+s2}{\PYGZdq{}}\PYG{p}{,} \PYG{n}{y4}\PYG{p}{)}
\PYG{n}{y5} \PYG{o}{=} \PYG{n}{int\PYGZus{}newton}\PYG{p}{(}\PYG{n}{X}\PYG{p}{[}\PYG{l+m+mi}{1}\PYG{p}{:}\PYG{l+m+mi}{4}\PYG{p}{]}\PYG{p}{,}\PYG{n}{Y}\PYG{p}{[}\PYG{l+m+mi}{1}\PYG{p}{:}\PYG{l+m+mi}{4}\PYG{p}{]}\PYG{p}{,}\PYG{l+m+mi}{4}\PYG{p}{)}
\PYG{n+nb}{print}\PYG{p}{(}\PYG{l+s+s2}{\PYGZdq{}}\PYG{l+s+s2}{A estimativa de f(4) de segundo grau é:}\PYG{l+s+s2}{\PYGZdq{}}\PYG{p}{,} \PYG{n}{y5}\PYG{p}{)}
\PYG{n}{y6} \PYG{o}{=} \PYG{n}{int\PYGZus{}newton}\PYG{p}{(}\PYG{n}{X}\PYG{p}{[}\PYG{p}{:}\PYG{l+m+mi}{4}\PYG{p}{]}\PYG{p}{,}\PYG{n}{Y}\PYG{p}{[}\PYG{p}{:}\PYG{l+m+mi}{4}\PYG{p}{]}\PYG{p}{,}\PYG{l+m+mi}{4}\PYG{p}{)}
\PYG{n+nb}{print}\PYG{p}{(}\PYG{l+s+s2}{\PYGZdq{}}\PYG{l+s+s2}{A estimativa de f(4) de terceiro grau é:}\PYG{l+s+s2}{\PYGZdq{}}\PYG{p}{,} \PYG{n}{y6}\PYG{p}{)}
\PYG{n}{y7} \PYG{o}{=} \PYG{n}{int\PYGZus{}newton}\PYG{p}{(}\PYG{n}{X}\PYG{p}{[}\PYG{p}{:}\PYG{o}{\PYGZhy{}}\PYG{l+m+mi}{1}\PYG{p}{]}\PYG{p}{,}\PYG{n}{Y}\PYG{p}{[}\PYG{p}{:}\PYG{o}{\PYGZhy{}}\PYG{l+m+mi}{1}\PYG{p}{]}\PYG{p}{,}\PYG{l+m+mi}{4}\PYG{p}{)}
\PYG{n+nb}{print}\PYG{p}{(}\PYG{l+s+s2}{\PYGZdq{}}\PYG{l+s+s2}{A estimativa de f(4) de quarto grau é:}\PYG{l+s+s2}{\PYGZdq{}}\PYG{p}{,} \PYG{n}{y7}\PYG{p}{)}

\PYG{n}{plt}\PYG{o}{.}\PYG{n}{plot}\PYG{p}{(}\PYG{n}{X}\PYG{p}{,}\PYG{n}{Y}\PYG{p}{,} \PYG{l+s+s1}{\PYGZsq{}}\PYG{l+s+s1}{o\PYGZhy{}}\PYG{l+s+s1}{\PYGZsq{}}\PYG{p}{)}
\PYG{n}{plt}\PYG{o}{.}\PYG{n}{plot}\PYG{p}{(}\PYG{l+m+mi}{4}\PYG{p}{,}\PYG{n}{y1}\PYG{p}{,} \PYG{l+s+s1}{\PYGZsq{}}\PYG{l+s+s1}{o}\PYG{l+s+s1}{\PYGZsq{}}\PYG{p}{,} \PYG{l+m+mi}{4}\PYG{p}{,}\PYG{n}{y2}\PYG{p}{,} \PYG{l+s+s1}{\PYGZsq{}}\PYG{l+s+s1}{o}\PYG{l+s+s1}{\PYGZsq{}}\PYG{p}{,} \PYG{l+m+mi}{4}\PYG{p}{,}\PYG{n}{y3}\PYG{p}{,} \PYG{l+s+s1}{\PYGZsq{}}\PYG{l+s+s1}{o}\PYG{l+s+s1}{\PYGZsq{}}\PYG{p}{,} \PYG{l+m+mi}{4}\PYG{p}{,}\PYG{n}{y4}\PYG{p}{,} \PYG{l+s+s1}{\PYGZsq{}}\PYG{l+s+s1}{o}\PYG{l+s+s1}{\PYGZsq{}}\PYG{p}{,} \PYG{l+m+mi}{4}\PYG{p}{,}\PYG{n}{y5}\PYG{p}{,} \PYG{l+s+s1}{\PYGZsq{}}\PYG{l+s+s1}{o}\PYG{l+s+s1}{\PYGZsq{}}\PYG{p}{,} \PYG{l+m+mi}{4}\PYG{p}{,}\PYG{n}{y6}\PYG{p}{,} \PYG{l+s+s1}{\PYGZsq{}}\PYG{l+s+s1}{o}\PYG{l+s+s1}{\PYGZsq{}}\PYG{p}{,} \PYG{l+m+mi}{4}\PYG{p}{,}\PYG{n}{y7}\PYG{p}{,} \PYG{l+s+s1}{\PYGZsq{}}\PYG{l+s+s1}{o}\PYG{l+s+s1}{\PYGZsq{}}\PYG{p}{)}
\PYG{n}{plt}\PYG{o}{.}\PYG{n}{legend}\PYG{p}{(}\PYG{p}{[}\PYG{l+s+s1}{\PYGZsq{}}\PYG{l+s+s1}{Spline Linear de y = log(x)}\PYG{l+s+s1}{\PYGZsq{}}\PYG{p}{]}\PYG{p}{)}\PYG{p}{;}
\end{sphinxVerbatim}

\begin{sphinxVerbatim}[commandchars=\\\{\}]
(i)
A estimativa de f(4) de primeiro grau é: 59.0
A estimativa de f(4) de segundo grau é: 50.0
A estimativa de f(4) de terceiro grau é: 48.0

(ii)
A estimativa de f(4) de primeiro grau é: 59.0
A estimativa de f(4) de segundo grau é: 50.0
A estimativa de f(4) de terceiro grau é: 48.0
A estimativa de f(4) de quarto grau é: 48.0
\end{sphinxVerbatim}

\noindent\sphinxincludegraphics{{lista-4-solucoes_24_1}.png}


\subsection{Questão 10}
\label{\detokenize{lista-4-solucoes:questao-10}}
\sphinxAtStartPar
Conhecendo a seguinte tabela


\begin{savenotes}\sphinxattablestart
\centering
\begin{tabulary}{\linewidth}[t]{|T|T|}
\hline
\sphinxstyletheadfamily 
\sphinxAtStartPar
x
&\sphinxstyletheadfamily 
\sphinxAtStartPar
f(x)
\\
\hline
\sphinxAtStartPar
\sphinxhyphen{}1
&
\sphinxAtStartPar
15
\\
\hline
\sphinxAtStartPar
0
&
\sphinxAtStartPar
8
\\
\hline
\sphinxAtStartPar
3
&
\sphinxAtStartPar
\sphinxhyphen{}1
\\
\hline
\end{tabulary}
\par
\sphinxattableend\end{savenotes}

\sphinxAtStartPar
determine:

\sphinxAtStartPar
(i) O polinômio de interpolação para a função definida por este conjunto de pares de pontos. Ou seja, determine a solução do sistema
\begin{equation*}
\begin{split} Xa = f \end{split}
\end{equation*}
\sphinxAtStartPar
para o vetor de \(a\) dos coeficientes de \(P_2(x)\).

\sphinxAtStartPar
(ii) O polinomio de interpolação na forma de Newton.

\begin{sphinxVerbatim}[commandchars=\\\{\}]
\PYG{c+c1}{\PYGZsh{} Solução}

\PYG{n}{X} \PYG{o}{=} \PYG{n}{np}\PYG{o}{.}\PYG{n}{array}\PYG{p}{(}\PYG{p}{[}\PYG{o}{\PYGZhy{}}\PYG{l+m+mi}{1}\PYG{p}{,} \PYG{l+m+mi}{0}\PYG{p}{,} \PYG{l+m+mi}{3}\PYG{p}{]}\PYG{p}{)}
\PYG{n}{Y} \PYG{o}{=} \PYG{n}{np}\PYG{o}{.}\PYG{n}{array}\PYG{p}{(}\PYG{p}{[}\PYG{l+m+mi}{15}\PYG{p}{,} \PYG{l+m+mi}{8}\PYG{p}{,} \PYG{o}{\PYGZhy{}}\PYG{l+m+mi}{1}\PYG{p}{]}\PYG{p}{)}

\PYG{n}{n} \PYG{o}{=} \PYG{n+nb}{len}\PYG{p}{(}\PYG{n}{X}\PYG{p}{)}
\PYG{n}{x} \PYG{o}{=} \PYG{n}{sy}\PYG{o}{.}\PYG{n}{symbols}\PYG{p}{(}\PYG{l+s+s2}{\PYGZdq{}}\PYG{l+s+s2}{x}\PYG{l+s+s2}{\PYGZdq{}}\PYG{p}{)}
    
\PYG{n}{dx} \PYG{o}{=} \PYG{p}{[}\PYG{p}{]}
\PYG{n}{f} \PYG{o}{=} \PYG{p}{[}\PYG{p}{]}
\PYG{n}{aux} \PYG{o}{=} \PYG{n}{np}\PYG{o}{.}\PYG{n}{diff}\PYG{p}{(}\PYG{n}{Y}\PYG{p}{)}

\PYG{k}{for} \PYG{n}{i} \PYG{o+ow}{in} \PYG{n+nb}{range}\PYG{p}{(}\PYG{l+m+mi}{1}\PYG{p}{,}\PYG{n}{n}\PYG{p}{)}\PYG{p}{:}
    \PYG{n}{dx}\PYG{o}{.}\PYG{n}{append}\PYG{p}{(}\PYG{n}{X}\PYG{p}{[}\PYG{n}{i}\PYG{p}{:}\PYG{p}{]} \PYG{o}{\PYGZhy{}} \PYG{n}{X}\PYG{p}{[}\PYG{p}{:}\PYG{o}{\PYGZhy{}}\PYG{n}{i}\PYG{p}{]}\PYG{p}{)}
    \PYG{n}{f}\PYG{o}{.}\PYG{n}{append}\PYG{p}{(}\PYG{n}{aux}\PYG{o}{/}\PYG{n}{dx}\PYG{p}{[}\PYG{n}{i}\PYG{o}{\PYGZhy{}}\PYG{l+m+mi}{1}\PYG{p}{]}\PYG{p}{)}
    \PYG{n}{aux} \PYG{o}{=} \PYG{n}{np}\PYG{o}{.}\PYG{n}{diff}\PYG{p}{(}\PYG{n}{f}\PYG{p}{[}\PYG{n}{i}\PYG{o}{\PYGZhy{}}\PYG{l+m+mi}{1}\PYG{p}{]}\PYG{p}{)}
    
\PYG{n}{b} \PYG{o}{=} \PYG{p}{[}\PYG{p}{]}
    
\PYG{k}{for} \PYG{n}{i} \PYG{o+ow}{in} \PYG{n}{f}\PYG{p}{:}
    \PYG{n}{b}\PYG{o}{.}\PYG{n}{append}\PYG{p}{(}\PYG{n}{i}\PYG{p}{[}\PYG{l+m+mi}{0}\PYG{p}{]}\PYG{p}{)}
    
\PYG{n}{f} \PYG{o}{=} \PYG{n}{Y}\PYG{p}{[}\PYG{l+m+mi}{0}\PYG{p}{]}
\PYG{n}{aux} \PYG{o}{=} \PYG{l+m+mi}{1}

\PYG{k}{for} \PYG{n}{i} \PYG{o+ow}{in} \PYG{n+nb}{range}\PYG{p}{(}\PYG{n}{n}\PYG{o}{\PYGZhy{}}\PYG{l+m+mi}{1}\PYG{p}{)}\PYG{p}{:}
    \PYG{n}{aux} \PYG{o}{*}\PYG{o}{=} \PYG{n}{x} \PYG{o}{\PYGZhy{}} \PYG{n}{X}\PYG{p}{[}\PYG{n}{i}\PYG{p}{]}
    \PYG{n}{f} \PYG{o}{+}\PYG{o}{=} \PYG{n}{b}\PYG{p}{[}\PYG{n}{i}\PYG{p}{]}\PYG{o}{*}\PYG{n}{aux}

\PYG{n+nb}{print}\PYG{p}{(}\PYG{l+s+s2}{\PYGZdq{}}\PYG{l+s+s2}{O polinômio é: P(x) =}\PYG{l+s+s2}{\PYGZdq{}}\PYG{p}{,} \PYG{n}{sy}\PYG{o}{.}\PYG{n}{simplify}\PYG{p}{(}\PYG{n}{f}\PYG{p}{)}\PYG{p}{)}
\end{sphinxVerbatim}

\begin{sphinxVerbatim}[commandchars=\\\{\}]
O polinômio é: P(x) = 1.0*x**2 \PYGZhy{} 6.0*x + 8.0
\end{sphinxVerbatim}


\subsection{Questão 11}
\label{\detokenize{lista-4-solucoes:questao-11}}
\sphinxAtStartPar
Usando os dados da tabela do exercício anterior:

\sphinxAtStartPar
(i) Determine \(P_2(x)\) pela forma de Lagrange.

\sphinxAtStartPar
(ii) Calcule uma aproximação para \(f(1)\) usando o item (i).

\begin{sphinxVerbatim}[commandchars=\\\{\}]
\PYG{c+c1}{\PYGZsh{} Solução}

\PYG{n}{X} \PYG{o}{=} \PYG{p}{[}\PYG{o}{\PYGZhy{}}\PYG{l+m+mi}{1}\PYG{p}{,} \PYG{l+m+mi}{0}\PYG{p}{,} \PYG{l+m+mi}{3}\PYG{p}{]}
\PYG{n}{Y} \PYG{o}{=} \PYG{p}{[}\PYG{l+m+mi}{15}\PYG{p}{,} \PYG{l+m+mi}{8}\PYG{p}{,} \PYG{o}{\PYGZhy{}}\PYG{l+m+mi}{1}\PYG{p}{]}

\PYG{c+c1}{\PYGZsh{} (i)}
\PYG{n}{n} \PYG{o}{=} \PYG{n+nb}{len}\PYG{p}{(}\PYG{n}{X}\PYG{p}{)}
\PYG{n}{x} \PYG{o}{=} \PYG{n}{sy}\PYG{o}{.}\PYG{n}{symbols}\PYG{p}{(}\PYG{l+s+s2}{\PYGZdq{}}\PYG{l+s+s2}{x}\PYG{l+s+s2}{\PYGZdq{}}\PYG{p}{)}
\PYG{n}{f} \PYG{o}{=} \PYG{l+m+mi}{0}
    
\PYG{k}{for} \PYG{n}{i} \PYG{o+ow}{in} \PYG{n+nb}{range}\PYG{p}{(}\PYG{n}{n}\PYG{p}{)}\PYG{p}{:}
    \PYG{n}{L} \PYG{o}{=} \PYG{l+m+mi}{1}
    \PYG{k}{for} \PYG{n}{j} \PYG{o+ow}{in} \PYG{n+nb}{range}\PYG{p}{(}\PYG{n}{n}\PYG{p}{)}\PYG{p}{:}
        \PYG{k}{if} \PYG{p}{(}\PYG{n}{i} \PYG{o}{!=} \PYG{n}{j}\PYG{p}{)}\PYG{p}{:}
            \PYG{n}{L} \PYG{o}{*}\PYG{o}{=} \PYG{p}{(}\PYG{n}{x} \PYG{o}{\PYGZhy{}} \PYG{n}{X}\PYG{p}{[}\PYG{n}{j}\PYG{p}{]}\PYG{p}{)}\PYG{o}{/}\PYG{p}{(}\PYG{n}{X}\PYG{p}{[}\PYG{n}{i}\PYG{p}{]} \PYG{o}{\PYGZhy{}} \PYG{n}{X}\PYG{p}{[}\PYG{n}{j}\PYG{p}{]}\PYG{p}{)}
    \PYG{n}{f} \PYG{o}{+}\PYG{o}{=} \PYG{n}{L}\PYG{o}{*}\PYG{n}{Y}\PYG{p}{[}\PYG{n}{i}\PYG{p}{]}

\PYG{n+nb}{print}\PYG{p}{(}\PYG{l+s+s2}{\PYGZdq{}}\PYG{l+s+s2}{(i)}\PYG{l+s+se}{\PYGZbs{}n}\PYG{l+s+s2}{P(x) =}\PYG{l+s+s2}{\PYGZdq{}}\PYG{p}{,} \PYG{n}{sy}\PYG{o}{.}\PYG{n}{simplify}\PYG{p}{(}\PYG{n}{f}\PYG{p}{)}\PYG{p}{)}

\PYG{c+c1}{\PYGZsh{} (ii)}
\PYG{n}{y} \PYG{o}{=} \PYG{n}{f}\PYG{o}{.}\PYG{n}{subs}\PYG{p}{(}\PYG{n}{x}\PYG{p}{,}\PYG{l+m+mi}{1}\PYG{p}{)}

\PYG{n+nb}{print}\PYG{p}{(}\PYG{l+s+s2}{\PYGZdq{}}\PYG{l+s+se}{\PYGZbs{}n}\PYG{l+s+s2}{(ii)}\PYG{l+s+se}{\PYGZbs{}n}\PYG{l+s+s2}{A aproximação para f(1) é:}\PYG{l+s+s2}{\PYGZdq{}}\PYG{p}{,} \PYG{n+nb}{float}\PYG{p}{(}\PYG{n}{y}\PYG{p}{)}\PYG{p}{)}

\PYG{n}{plt}\PYG{o}{.}\PYG{n}{plot}\PYG{p}{(}\PYG{n}{X}\PYG{p}{,}\PYG{n}{Y}\PYG{p}{,} \PYG{l+s+s1}{\PYGZsq{}}\PYG{l+s+s1}{o\PYGZhy{}}\PYG{l+s+s1}{\PYGZsq{}}\PYG{p}{)}
\PYG{n}{plt}\PYG{o}{.}\PYG{n}{plot}\PYG{p}{(}\PYG{l+m+mi}{1}\PYG{p}{,}\PYG{n}{y}\PYG{p}{,} \PYG{l+s+s1}{\PYGZsq{}}\PYG{l+s+s1}{o}\PYG{l+s+s1}{\PYGZsq{}}\PYG{p}{)}
\PYG{n}{plt}\PYG{o}{.}\PYG{n}{legend}\PYG{p}{(}\PYG{p}{[}\PYG{l+s+s1}{\PYGZsq{}}\PYG{l+s+s1}{Spline Linear de y = f(x)}\PYG{l+s+s1}{\PYGZsq{}}\PYG{p}{,} \PYG{l+s+s1}{\PYGZsq{}}\PYG{l+s+s1}{Valor Interpolado}\PYG{l+s+s1}{\PYGZsq{}}\PYG{p}{]}\PYG{p}{)}\PYG{p}{;}
\end{sphinxVerbatim}

\begin{sphinxVerbatim}[commandchars=\\\{\}]
(i)
P(x) = x**2 \PYGZhy{} 6*x + 8

(ii)
A aproximação para f(1) é: 3.0
\end{sphinxVerbatim}

\noindent\sphinxincludegraphics{{lista-4-solucoes_28_1}.png}


\subsection{Questão 12}
\label{\detokenize{lista-4-solucoes:questao-12}}
\sphinxAtStartPar
Dada a tabela:


\begin{savenotes}\sphinxattablestart
\centering
\begin{tabulary}{\linewidth}[t]{|T|T|}
\hline
\sphinxstyletheadfamily 
\sphinxAtStartPar
\(x\)
&\sphinxstyletheadfamily 
\sphinxAtStartPar
\(x e^{3x}\)
\\
\hline
\sphinxAtStartPar
0
&
\sphinxAtStartPar
1
\\
\hline
\sphinxAtStartPar
0.1
&
\sphinxAtStartPar
1.3499
\\
\hline
\sphinxAtStartPar
0.2
&
\sphinxAtStartPar
1.8221
\\
\hline
\sphinxAtStartPar
0.3
&
\sphinxAtStartPar
2.4596
\\
\hline
\sphinxAtStartPar
0.4
&
\sphinxAtStartPar
3.3201
\\
\hline
\sphinxAtStartPar
0.5
&
\sphinxAtStartPar
4.4817
\\
\hline
\end{tabulary}
\par
\sphinxattableend\end{savenotes}

\sphinxAtStartPar
calcule \(f(0.25)\), onde \(f(x) = x e^{3x}\) usando polinômio de interpolação do 2° grau:

\sphinxAtStartPar
(i) Usando \(x_0 = 0.2, x_1 = 0.3, x_2 = 0.4\);

\sphinxAtStartPar
(ii) Usando \(x_0 = 0.1, x_1 = 0.2, x_2 = 0.3\);

\begin{sphinxVerbatim}[commandchars=\\\{\}]
\PYG{c+c1}{\PYGZsh{} Solução}

\PYG{c+c1}{\PYGZsh{} (i)}
\PYG{n}{X} \PYG{o}{=} \PYG{p}{[}\PYG{l+m+mf}{0.2}\PYG{p}{,} \PYG{l+m+mf}{0.3}\PYG{p}{,} \PYG{l+m+mf}{0.4}\PYG{p}{]}
\PYG{n}{Y} \PYG{o}{=} \PYG{p}{[}\PYG{l+m+mf}{1.8221}\PYG{p}{,} \PYG{l+m+mf}{2.4596}\PYG{p}{,} \PYG{l+m+mf}{3.3201}\PYG{p}{]}

\PYG{n}{y1} \PYG{o}{=} \PYG{n}{int\PYGZus{}lagrange}\PYG{p}{(}\PYG{n}{X}\PYG{p}{,}\PYG{n}{Y}\PYG{p}{,}\PYG{l+m+mf}{0.25}\PYG{p}{)}
\PYG{n+nb}{print}\PYG{p}{(}\PYG{l+s+s2}{\PYGZdq{}}\PYG{l+s+s2}{(i)}\PYG{l+s+se}{\PYGZbs{}n}\PYG{l+s+s2}{A estimativa de f(0.25) é:}\PYG{l+s+s2}{\PYGZdq{}}\PYG{p}{,} \PYG{n}{y1}\PYG{p}{)}

\PYG{c+c1}{\PYGZsh{} (i)}
\PYG{n}{X} \PYG{o}{=} \PYG{p}{[}\PYG{l+m+mf}{0.1}\PYG{p}{,} \PYG{l+m+mf}{0.2}\PYG{p}{,} \PYG{l+m+mf}{0.3}\PYG{p}{]}
\PYG{n}{Y} \PYG{o}{=} \PYG{p}{[}\PYG{l+m+mf}{1.3499}\PYG{p}{,} \PYG{l+m+mf}{1.8221}\PYG{p}{,} \PYG{l+m+mf}{2.4596}\PYG{p}{]}

\PYG{n}{y2} \PYG{o}{=} \PYG{n}{int\PYGZus{}lagrange}\PYG{p}{(}\PYG{n}{X}\PYG{p}{,}\PYG{n}{Y}\PYG{p}{,}\PYG{l+m+mf}{0.25}\PYG{p}{)}
\PYG{n+nb}{print}\PYG{p}{(}\PYG{l+s+s2}{\PYGZdq{}}\PYG{l+s+se}{\PYGZbs{}n}\PYG{l+s+s2}{(ii)}\PYG{l+s+se}{\PYGZbs{}n}\PYG{l+s+s2}{A estimativa de f(0.25) é:}\PYG{l+s+s2}{\PYGZdq{}}\PYG{p}{,} \PYG{n}{y2}\PYG{p}{)}

\PYG{n}{X} \PYG{o}{=} \PYG{p}{[}\PYG{l+m+mi}{0}\PYG{p}{,} \PYG{l+m+mf}{0.1}\PYG{p}{,} \PYG{l+m+mf}{0.2}\PYG{p}{,} \PYG{l+m+mf}{0.3}\PYG{p}{,} \PYG{l+m+mf}{0.4}\PYG{p}{,} \PYG{l+m+mf}{0.5}\PYG{p}{]}
\PYG{n}{Y} \PYG{o}{=} \PYG{p}{[}\PYG{l+m+mi}{1}\PYG{p}{,} \PYG{l+m+mf}{1.3499}\PYG{p}{,} \PYG{l+m+mf}{1.8221}\PYG{p}{,} \PYG{l+m+mf}{2.4596}\PYG{p}{,} \PYG{l+m+mf}{3.3201}\PYG{p}{,} \PYG{l+m+mf}{4.4817}\PYG{p}{]}

\PYG{n}{plt}\PYG{o}{.}\PYG{n}{plot}\PYG{p}{(}\PYG{n}{X}\PYG{p}{,}\PYG{n}{Y}\PYG{p}{,} \PYG{l+s+s1}{\PYGZsq{}}\PYG{l+s+s1}{o\PYGZhy{}}\PYG{l+s+s1}{\PYGZsq{}}\PYG{p}{)}
\PYG{n}{plt}\PYG{o}{.}\PYG{n}{plot}\PYG{p}{(}\PYG{l+m+mf}{0.25}\PYG{p}{,}\PYG{n}{y1}\PYG{p}{,} \PYG{l+s+s1}{\PYGZsq{}}\PYG{l+s+s1}{o}\PYG{l+s+s1}{\PYGZsq{}}\PYG{p}{,} \PYG{l+m+mf}{0.25}\PYG{p}{,}\PYG{n}{y2}\PYG{p}{,} \PYG{l+s+s1}{\PYGZsq{}}\PYG{l+s+s1}{o}\PYG{l+s+s1}{\PYGZsq{}}\PYG{p}{)}
\PYG{n}{plt}\PYG{o}{.}\PYG{n}{legend}\PYG{p}{(}\PYG{p}{[}\PYG{l+s+s1}{\PYGZsq{}}\PYG{l+s+s1}{Spline Linear de y = f(x)}\PYG{l+s+s1}{\PYGZsq{}}\PYG{p}{,} \PYG{l+s+s1}{\PYGZsq{}}\PYG{l+s+s1}{Interpolação (i)}\PYG{l+s+s1}{\PYGZsq{}}\PYG{p}{,} \PYG{l+s+s1}{\PYGZsq{}}\PYG{l+s+s1}{Interpolação (ii)}\PYG{l+s+s1}{\PYGZsq{}}\PYG{p}{]}\PYG{p}{)}\PYG{p}{;}
\end{sphinxVerbatim}

\begin{sphinxVerbatim}[commandchars=\\\{\}]
(i)
A estimativa de f(0.25) é: 2.112975

(ii)
A estimativa de f(0.25) é: 2.1201875
\end{sphinxVerbatim}

\noindent\sphinxincludegraphics{{lista-4-solucoes_30_1}.png}


\subsection{Questão 13}
\label{\detokenize{lista-4-solucoes:questao-13}}
\sphinxAtStartPar
Para a seguinte função tabelada


\begin{savenotes}\sphinxattablestart
\centering
\begin{tabulary}{\linewidth}[t]{|T|T|}
\hline
\sphinxstyletheadfamily 
\sphinxAtStartPar
\(x\)
&\sphinxstyletheadfamily 
\sphinxAtStartPar
\(x e^{3x}\)
\\
\hline
\sphinxAtStartPar
0
&
\sphinxAtStartPar
1
\\
\hline
\sphinxAtStartPar
0.1
&
\sphinxAtStartPar
1.3499
\\
\hline
\sphinxAtStartPar
0.2
&
\sphinxAtStartPar
1.8221
\\
\hline
\sphinxAtStartPar
0.3
&
\sphinxAtStartPar
2.4596
\\
\hline
\sphinxAtStartPar
0.4
&
\sphinxAtStartPar
3.3201
\\
\hline
\sphinxAtStartPar
0.5
&
\sphinxAtStartPar
4.4817
\\
\hline
\end{tabulary}
\par
\sphinxattableend\end{savenotes}

\sphinxAtStartPar
construir a tabela de diferenças divididas.

\begin{sphinxVerbatim}[commandchars=\\\{\}]
\PYG{c+c1}{\PYGZsh{} Solução}

\PYG{n}{X} \PYG{o}{=} \PYG{n}{np}\PYG{o}{.}\PYG{n}{array}\PYG{p}{(}\PYG{p}{[}\PYG{l+m+mi}{0}\PYG{p}{,} \PYG{l+m+mf}{0.1}\PYG{p}{,} \PYG{l+m+mf}{0.2}\PYG{p}{,} \PYG{l+m+mf}{0.3}\PYG{p}{,} \PYG{l+m+mf}{0.4}\PYG{p}{,} \PYG{l+m+mf}{0.5}\PYG{p}{]}\PYG{p}{)}
\PYG{n}{Y} \PYG{o}{=} \PYG{n}{np}\PYG{o}{.}\PYG{n}{array}\PYG{p}{(}\PYG{p}{[}\PYG{l+m+mi}{1}\PYG{p}{,} \PYG{l+m+mf}{1.3499}\PYG{p}{,} \PYG{l+m+mf}{1.8221}\PYG{p}{,} \PYG{l+m+mf}{2.4596}\PYG{p}{,} \PYG{l+m+mf}{3.3201}\PYG{p}{,} \PYG{l+m+mf}{4.4817}\PYG{p}{]}\PYG{p}{)}

\PYG{n}{n} \PYG{o}{=} \PYG{n+nb}{len}\PYG{p}{(}\PYG{n}{X}\PYG{p}{)}
\PYG{n}{x} \PYG{o}{=} \PYG{n}{sy}\PYG{o}{.}\PYG{n}{symbols}\PYG{p}{(}\PYG{l+s+s2}{\PYGZdq{}}\PYG{l+s+s2}{x}\PYG{l+s+s2}{\PYGZdq{}}\PYG{p}{)}
    
\PYG{n}{dx} \PYG{o}{=} \PYG{p}{[}\PYG{p}{]}
\PYG{n}{f} \PYG{o}{=} \PYG{p}{[}\PYG{p}{]}
\PYG{n}{aux} \PYG{o}{=} \PYG{n}{np}\PYG{o}{.}\PYG{n}{diff}\PYG{p}{(}\PYG{n}{Y}\PYG{p}{)}

\PYG{k}{for} \PYG{n}{i} \PYG{o+ow}{in} \PYG{n+nb}{range}\PYG{p}{(}\PYG{l+m+mi}{1}\PYG{p}{,}\PYG{n}{n}\PYG{p}{)}\PYG{p}{:}
    \PYG{n}{dx}\PYG{o}{.}\PYG{n}{append}\PYG{p}{(}\PYG{n}{X}\PYG{p}{[}\PYG{n}{i}\PYG{p}{:}\PYG{p}{]} \PYG{o}{\PYGZhy{}} \PYG{n}{X}\PYG{p}{[}\PYG{p}{:}\PYG{o}{\PYGZhy{}}\PYG{n}{i}\PYG{p}{]}\PYG{p}{)}
    \PYG{n}{f}\PYG{o}{.}\PYG{n}{append}\PYG{p}{(}\PYG{n}{aux}\PYG{o}{/}\PYG{n}{dx}\PYG{p}{[}\PYG{n}{i}\PYG{o}{\PYGZhy{}}\PYG{l+m+mi}{1}\PYG{p}{]}\PYG{p}{)}
    \PYG{n}{aux} \PYG{o}{=} \PYG{n}{np}\PYG{o}{.}\PYG{n}{diff}\PYG{p}{(}\PYG{n}{f}\PYG{p}{[}\PYG{n}{i}\PYG{o}{\PYGZhy{}}\PYG{l+m+mi}{1}\PYG{p}{]}\PYG{p}{)}

\PYG{k}{for} \PYG{n}{i} \PYG{o+ow}{in} \PYG{n+nb}{range}\PYG{p}{(}\PYG{n+nb}{len}\PYG{p}{(}\PYG{n}{f}\PYG{p}{)}\PYG{p}{)}\PYG{p}{:}
    \PYG{n+nb}{print}\PYG{p}{(}\PYG{l+s+s2}{\PYGZdq{}}\PYG{l+s+se}{\PYGZbs{}n}\PYG{l+s+s2}{ As}\PYG{l+s+s2}{\PYGZdq{}}\PYG{p}{,} \PYG{n}{i}\PYG{o}{+}\PYG{l+m+mi}{1}\PYG{p}{,} \PYG{l+s+s2}{\PYGZdq{}}\PYG{l+s+s2}{ªs diferenças divididas são:}\PYG{l+s+se}{\PYGZbs{}n}\PYG{l+s+s2}{\PYGZdq{}}\PYG{p}{,} \PYG{n}{f}\PYG{p}{[}\PYG{n}{i}\PYG{p}{]}\PYG{p}{)}
\end{sphinxVerbatim}

\begin{sphinxVerbatim}[commandchars=\\\{\}]
 As 1 ªs diferenças divididas são:
 [ 3.499  4.722  6.375  8.605 11.616]

 As 2 ªs diferenças divididas são:
 [ 6.115  8.265 11.15  15.055]

 As 3 ªs diferenças divididas são:
 [ 7.16666667  9.61666667 13.01666667]

 As 4 ªs diferenças divididas são:
 [6.125 8.5  ]

 As 5 ªs diferenças divididas são:
 [4.75]
\end{sphinxVerbatim}


\chapter{Lista de Exercícios 5}
\label{\detokenize{lista-5-solucoes:lista-de-exercicios-5}}\label{\detokenize{lista-5-solucoes::doc}}
\sphinxAtStartPar
Solucionário matemático e computacional de exercícios selecionados da Lista de Exercícios 5.

\begin{sphinxVerbatim}[commandchars=\\\{\}]
\PYG{o}{\PYGZpc{}}\PYG{k}{matplotlib} inline
\end{sphinxVerbatim}

\begin{sphinxVerbatim}[commandchars=\\\{\}]
\PYG{c+c1}{\PYGZsh{} importação de módulos}
\PYG{k+kn}{import} \PYG{n+nn}{numpy} \PYG{k}{as} \PYG{n+nn}{np}
\PYG{k+kn}{import} \PYG{n+nn}{matplotlib}\PYG{n+nn}{.}\PYG{n+nn}{pyplot} \PYG{k}{as} \PYG{n+nn}{plt}
\end{sphinxVerbatim}


\section{Ajuste de curvas}
\label{\detokenize{lista-5-solucoes:ajuste-de-curvas}}
\sphinxAtStartPar
\sphinxstylestrong{Obs.:}: função \sphinxcode{\sphinxupquote{polyfit}} retorna coeficientes da curva de ajuste na ordem:

\begin{sphinxVerbatim}[commandchars=\\\{\}]
\PYG{n}{P} \PYG{o}{=} \PYG{n}{p}\PYG{p}{[}\PYG{l+m+mi}{0}\PYG{p}{]} \PYG{o}{+} \PYG{n}{p}\PYG{p}{[}\PYG{l+m+mi}{1}\PYG{p}{]}\PYG{o}{*}\PYG{n}{x} \PYG{o}{+} \PYG{n}{p}\PYG{p}{[}\PYG{l+m+mi}{2}\PYG{p}{]}\PYG{o}{*}\PYG{n}{x}\PYG{o}{*}\PYG{o}{*}\PYG{l+m+mi}{2} \PYG{o}{+} \PYG{n}{p}\PYG{p}{[}\PYG{l+m+mi}{3}\PYG{p}{]}\PYG{o}{*}\PYG{n}{x}\PYG{o}{*}\PYG{o}{*}\PYG{l+m+mi}{3} \PYG{o}{+} \PYG{o}{.}\PYG{o}{.}\PYG{o}{.}
\end{sphinxVerbatim}


\subsection{Exemplo: solução fechada 2D}
\label{\detokenize{lista-5-solucoes:exemplo-solucao-fechada-2d}}
\begin{sphinxVerbatim}[commandchars=\\\{\}]
\PYG{c+c1}{\PYGZsh{} tabela de dados}
\PYG{n}{x} \PYG{o}{=} \PYG{n}{np}\PYG{o}{.}\PYG{n}{array}\PYG{p}{(}\PYG{p}{[}\PYG{l+m+mi}{1}\PYG{p}{,}\PYG{l+m+mi}{2}\PYG{p}{,}\PYG{l+m+mi}{3}\PYG{p}{,}\PYG{l+m+mi}{4}\PYG{p}{]}\PYG{p}{)}
\PYG{n}{y} \PYG{o}{=} \PYG{n}{np}\PYG{o}{.}\PYG{n}{array}\PYG{p}{(}\PYG{p}{[}\PYG{l+m+mf}{0.5}\PYG{p}{,}\PYG{l+m+mf}{2.5}\PYG{p}{,}\PYG{l+m+mf}{2.0}\PYG{p}{,}\PYG{l+m+mf}{4.0}\PYG{p}{]}\PYG{p}{)}

\PYG{c+c1}{\PYGZsh{} min quad}
\PYG{n}{m} \PYG{o}{=} \PYG{n}{np}\PYG{o}{.}\PYG{n}{size}\PYG{p}{(}\PYG{n}{x}\PYG{p}{)}
\PYG{n}{alpha1} \PYG{o}{=} \PYG{p}{(}\PYG{n}{m}\PYG{o}{*}\PYG{n}{np}\PYG{o}{.}\PYG{n}{dot}\PYG{p}{(}\PYG{n}{x}\PYG{p}{,}\PYG{n}{y}\PYG{p}{)} \PYG{o}{\PYGZhy{}} \PYG{n}{np}\PYG{o}{.}\PYG{n}{sum}\PYG{p}{(}\PYG{n}{x}\PYG{p}{)}\PYG{o}{*}\PYG{n}{np}\PYG{o}{.}\PYG{n}{sum}\PYG{p}{(}\PYG{n}{y}\PYG{p}{)}\PYG{p}{)}\PYG{o}{/}\PYG{p}{(}\PYG{n}{m}\PYG{o}{*}\PYG{n}{np}\PYG{o}{.}\PYG{n}{dot}\PYG{p}{(}\PYG{n}{x}\PYG{p}{,}\PYG{n}{x}\PYG{p}{)}\PYG{o}{\PYGZhy{}}\PYG{n}{np}\PYG{o}{.}\PYG{n}{sum}\PYG{p}{(}\PYG{n}{x}\PYG{p}{)}\PYG{o}{*}\PYG{o}{*}\PYG{l+m+mi}{2}\PYG{p}{)}
\PYG{n}{alpha0} \PYG{o}{=} \PYG{n}{np}\PYG{o}{.}\PYG{n}{mean}\PYG{p}{(}\PYG{n}{y}\PYG{p}{)} \PYG{o}{\PYGZhy{}} \PYG{n}{alpha1}\PYG{o}{*}\PYG{n}{np}\PYG{o}{.}\PYG{n}{mean}\PYG{p}{(}\PYG{n}{x}\PYG{p}{)}

\PYG{c+c1}{\PYGZsh{} plot}
\PYG{n}{plt}\PYG{o}{.}\PYG{n}{scatter}\PYG{p}{(}\PYG{n}{x}\PYG{p}{,}\PYG{n}{y}\PYG{p}{)}
\PYG{n}{y2} \PYG{o}{=} \PYG{n}{alpha0} \PYG{o}{+} \PYG{n}{alpha1}\PYG{o}{*}\PYG{n}{x}
\PYG{n}{plt}\PYG{o}{.}\PYG{n}{plot}\PYG{p}{(}\PYG{n}{x}\PYG{p}{,}\PYG{n}{y2}\PYG{p}{,}\PYG{l+s+s1}{\PYGZsq{}}\PYG{l+s+s1}{r:}\PYG{l+s+s1}{\PYGZsq{}}\PYG{p}{)}\PYG{p}{;}
\end{sphinxVerbatim}

\noindent\sphinxincludegraphics{{lista-5-solucoes_5_0}.png}


\section{Resolução da Lista 5}
\label{\detokenize{lista-5-solucoes:resolucao-da-lista-5}}

\subsection{Função\sphinxhyphen{}base para computar ajuste e plotar resultados}
\label{\detokenize{lista-5-solucoes:funcao-base-para-computar-ajuste-e-plotar-resultados}}
\begin{sphinxVerbatim}[commandchars=\\\{\}]
\PYG{l+s+sd}{\PYGZdq{}\PYGZdq{}\PYGZdq{}}
\PYG{l+s+sd}{Resolve problema de ajuste polinomial discreto e plota resultado}

\PYG{l+s+sd}{    entrada: }
\PYG{l+s+sd}{        x,y : tabela de valores (numpy arrays)}
\PYG{l+s+sd}{        g   : grau do polinômio de ajuste (int)}
\PYG{l+s+sd}{\PYGZdq{}\PYGZdq{}\PYGZdq{}}
\PYG{k}{def} \PYG{n+nf}{resolve\PYGZus{}ajuste}\PYG{p}{(}\PYG{n}{x}\PYG{p}{,}\PYG{n}{y}\PYG{p}{,}\PYG{n}{g}\PYG{p}{)}\PYG{p}{:}
    
    \PYG{n}{c} \PYG{o}{=} \PYG{n}{np}\PYG{o}{.}\PYG{n}{polyfit}\PYG{p}{(}\PYG{n}{x}\PYG{p}{,} \PYG{n}{y}\PYG{p}{,} \PYG{n}{g}\PYG{p}{)} \PYG{c+c1}{\PYGZsh{} ajuste}
    \PYG{n}{p} \PYG{o}{=} \PYG{n}{np}\PYG{o}{.}\PYG{n}{poly1d}\PYG{p}{(}\PYG{n}{c}\PYG{p}{)} 
    \PYG{n}{plt}\PYG{o}{.}\PYG{n}{plot}\PYG{p}{(}\PYG{n}{x}\PYG{p}{,}\PYG{n}{y}\PYG{p}{,}\PYG{l+s+s1}{\PYGZsq{}}\PYG{l+s+s1}{d}\PYG{l+s+s1}{\PYGZsq{}}\PYG{p}{,}\PYG{n}{x}\PYG{p}{,}\PYG{n}{p}\PYG{p}{(}\PYG{n}{x}\PYG{p}{)}\PYG{p}{,}\PYG{l+s+s1}{\PYGZsq{}}\PYG{l+s+s1}{o\PYGZhy{}}\PYG{l+s+s1}{\PYGZsq{}}\PYG{p}{)}
    \PYG{n}{plt}\PYG{o}{.}\PYG{n}{xlabel}\PYG{p}{(}\PYG{n+nb}{str}\PYG{p}{(}\PYG{n}{x}\PYG{p}{)}\PYG{p}{)}
    \PYG{k}{for} \PYG{n}{i} \PYG{o+ow}{in} \PYG{n+nb}{range}\PYG{p}{(}\PYG{n}{g}\PYG{o}{+}\PYG{l+m+mi}{1}\PYG{p}{)}\PYG{p}{:}
        \PYG{n+nb}{print}\PYG{p}{(}\PYG{l+s+s1}{\PYGZsq{}}\PYG{l+s+s1}{Coeficiente de ajuste p[}\PYG{l+s+s1}{\PYGZsq{}} \PYG{o}{+} \PYG{n+nb}{str}\PYG{p}{(}\PYG{n}{i}\PYG{p}{)} \PYG{o}{+} \PYG{l+s+s1}{\PYGZsq{}}\PYG{l+s+s1}{]: }\PYG{l+s+s1}{\PYGZsq{}} \PYG{o}{+} \PYG{n+nb}{str}\PYG{p}{(}\PYG{n}{c}\PYG{p}{[}\PYG{n}{i}\PYG{p}{]}\PYG{p}{)}\PYG{p}{)}

    \PYG{k}{return} \PYG{n}{p}
\end{sphinxVerbatim}


\subsection{L5\sphinxhyphen{}Q2}
\label{\detokenize{lista-5-solucoes:l5-q2}}
\sphinxAtStartPar
Ajuste os dados abaixo pelo método dos mínimos quadrados:

\sphinxAtStartPar
\textbackslash{}begin\{array\}\{c|cccccccc\}
x \& 1 \& 2 \& 3 \& 4 \& 5 \& 6 \& 7 \& 8 \textbackslash{}
\textbackslash{}hline
y \& 0.5 \& 0.6  \& 0.9 \& 0.8 \& 1.2 \& 1.5 \&  1.7 \& 2.0
\textbackslash{}end\{array\}

\sphinxAtStartPar
a) por reta

\sphinxAtStartPar
b) por parábola do tipo \(ax^2 + bx + c\);


\subsection{Solução}
\label{\detokenize{lista-5-solucoes:solucao}}

\subsubsection{Metodologia matemática}
\label{\detokenize{lista-5-solucoes:metodologia-matematica}}
\sphinxAtStartPar
a)
\begin{equation*}
\begin{split}g_1(x) = x$$$$ g_2(x)=1\end{split}
\end{equation*}
\sphinxAtStartPar
\textbackslash{}begin\{array\}\{c|cccccccc\}
g\_1 \& 1 \& 2 \& 3 \& 4 \& 5 \& 6 \& 7 \& 8 \textbackslash{}
\textbackslash{}hline
g\_2 \& 1 \& 1 \& 1\& 1\& 1 \& 1 \& 1 \& 1 \textbackslash{}
\textbackslash{}hline
f \& 0.5 \& 0.6  \& 0.9 \& 0.8 \& 1.2 \& 1.5 \&  1.7 \& 2.0
\textbackslash{}end\{array\}
\begin{eqnarray*}
a_{11} &=& <g_1,g_1> &= 204\\
a_{12} &=& <g_1,g_2> &= 36 \\
a_{21} &=& a_{12}  \\
a_{22} &=& <g_2,g_2> &= 8\\
b_{1} &=&  <g_1,f>   &= 50.5\\
b_{2} &=&  <g_2,f>  &= 9.2
\end{eqnarray*}\begin{equation*}
\begin{split}\begin{bmatrix}
204 & 36 \\
36 &  8
\end{bmatrix}
\begin{bmatrix}
\alpha_1 \\
\alpha_2 
\end{bmatrix} = 
\begin{bmatrix}
50.5 \\
9.2
\end{bmatrix}\end{split}
\end{equation*}\begin{equation*}
\begin{split}\begin{align*}
204\alpha_1+36\alpha_2 &= 50.5 \\
36\alpha_1+8\alpha_2 &= 9.2
\end{align*}\end{split}
\end{equation*}\begin{equation*}
\begin{split}\alpha_1 = 0.2166 ,\  \alpha_2 = 0.175\end{split}
\end{equation*}\begin{equation*}
\begin{split}f(x) = 0.1749x + 0.2167\end{split}
\end{equation*}
\sphinxAtStartPar
b)
\begin{equation*}
\begin{split}g_1(x) = x²$$$$ g_2(x)=x$$$$ g_3(x)=1\end{split}
\end{equation*}
\sphinxAtStartPar
\textbackslash{}begin\{array\}\{c|cccccccc\}
g\_1 \& 1 \& 4 \& 9 \& 16 \& 25 \& 36 \& 49 \& 64 \textbackslash{}
\textbackslash{}hline
g\_2 \& 1 \& 2 \& 3 \& 4 \& 5 \& 6 \& 7 \& 8 \textbackslash{}
\textbackslash{}hline
g\_3 \& 1 \& 1 \& 1\& 1\& 1 \& 1 \& 1 \& 1 \textbackslash{}
\textbackslash{}hline
f \& 0.5 \& 0.6  \& 0.9 \& 0.8 \& 1.2 \& 1.5 \&  1.7 \& 2.0
\textbackslash{}end\{array\}
\begin{align*}
a_{11}=\ <g_1,g_1> \ &= 8772\\
a_{12}=\ a_{21}   =\  <g_1,g_2> \ &= 1296 \\
a_{13}=\ a_{31}   =\  <g_1,g_3> \ &= 204 \\
a_{22}=\ <g_2,g_2> \ &= 204\\
a_{23}=\ a_{32}   =\  <g_2,g_3> \ &= 36 \\
a_{33}=\ <g_3,g_3> \ &= 8 \\
b_{1}=\  <g_1,f>  \ &= 319.1\\
b_{2}=\  <g_2,f>  \ &= 50.5\\
b_{3}=\  <g_3,f>  \ &= 9.2
\end{align*}\begin{equation*}
\begin{split}\begin{bmatrix}
8872 & 1296 & 204 \\
1296 &  204 & 36 \\
204 &  36 & 8
\end{bmatrix}
\begin{bmatrix}
\alpha_1 \\
\alpha_2 \\
\alpha_3 
\end{bmatrix} = 
\begin{bmatrix}
319.1\\
50.5 \\
9.2 
\end{bmatrix}\end{split}
\end{equation*}\begin{equation*}
\begin{split}\begin{align*}
8872\alpha_1+1296\alpha_2 + 204\alpha_3 &= 319.1 \\
1296\alpha_1+204\alpha_2 + 36\alpha_3 &= 50.5 \\
204\alpha_1+36\alpha_2 + 8\alpha_3 &= 9.2
\end{align*}\end{split}
\end{equation*}\begin{equation*}
\begin{split}\alpha_1 = 0.4071 ,\  \alpha_2 = 0.07738, \  \alpha_3 = 0.01547\end{split}
\end{equation*}\begin{equation*}
\begin{split}f(x) = 0.01547 + 0.07738x + 0.4071x^2\end{split}
\end{equation*}

\subsubsection{Metodologia computacional}
\label{\detokenize{lista-5-solucoes:metodologia-computacional}}
\sphinxAtStartPar
a)

\begin{sphinxVerbatim}[commandchars=\\\{\}]
\PYG{c+c1}{\PYGZsh{} tabela de dados}
\PYG{n}{x} \PYG{o}{=} \PYG{n}{np}\PYG{o}{.}\PYG{n}{arange}\PYG{p}{(}\PYG{l+m+mi}{1}\PYG{p}{,}\PYG{l+m+mi}{9}\PYG{p}{)}
\PYG{n}{y} \PYG{o}{=} \PYG{n}{np}\PYG{o}{.}\PYG{n}{array}\PYG{p}{(}\PYG{p}{[}\PYG{l+m+mf}{0.5}\PYG{p}{,}\PYG{l+m+mf}{.6}\PYG{p}{,}\PYG{l+m+mf}{.9}\PYG{p}{,}\PYG{l+m+mf}{.8}\PYG{p}{,}\PYG{l+m+mf}{1.2}\PYG{p}{,}\PYG{l+m+mf}{1.5}\PYG{p}{,}\PYG{l+m+mf}{1.7}\PYG{p}{,}\PYG{l+m+mf}{2.}\PYG{p}{]}\PYG{p}{)}

\PYG{c+c1}{\PYGZsh{} grau 1}
\PYG{n}{g} \PYG{o}{=} \PYG{l+m+mi}{1}
\PYG{n}{p} \PYG{o}{=} \PYG{n}{resolve\PYGZus{}ajuste}\PYG{p}{(}\PYG{n}{x}\PYG{p}{,}\PYG{n}{y}\PYG{p}{,}\PYG{n}{g}\PYG{p}{)}
\end{sphinxVerbatim}

\begin{sphinxVerbatim}[commandchars=\\\{\}]
Coeficiente de ajuste p[0]: 0.2166666666666667
Coeficiente de ajuste p[1]: 0.17499999999999996
\end{sphinxVerbatim}

\noindent\sphinxincludegraphics{{lista-5-solucoes_15_1}.png}

\sphinxAtStartPar
b)

\begin{sphinxVerbatim}[commandchars=\\\{\}]
\PYG{c+c1}{\PYGZsh{} grau 2}
\PYG{n}{g} \PYG{o}{=} \PYG{l+m+mi}{2}
\PYG{n}{p} \PYG{o}{=} \PYG{n}{resolve\PYGZus{}ajuste}\PYG{p}{(}\PYG{n}{x}\PYG{p}{,}\PYG{n}{y}\PYG{p}{,}\PYG{n}{g}\PYG{p}{)}
\end{sphinxVerbatim}

\begin{sphinxVerbatim}[commandchars=\\\{\}]
Coeficiente de ajuste p[0]: 0.015476190476190437
Coeficiente de ajuste p[1]: 0.07738095238095268
Coeficiente de ajuste p[2]: 0.407142857142857
\end{sphinxVerbatim}

\noindent\sphinxincludegraphics{{lista-5-solucoes_17_1}.png}


\subsection{L5\sphinxhyphen{}Q3}
\label{\detokenize{lista-5-solucoes:l5-q3}}
\sphinxAtStartPar
Dada a tabela abaixo, faça o gráfico de dispersão dos dados e ajuste uma curva da melhor maneira possı́vel:

\sphinxAtStartPar
\textbackslash{}begin\{array\}\{c|cccccc\}
x \& 0.5 \& 0.75 \& 1 \& 1.5 \& 2.0 \& 2.5 \& 3.0  \textbackslash{}
\textbackslash{}hline
y \& \sphinxhyphen{}2.8 \& \sphinxhyphen{}0.6 \& 1\& 3.2\& 4.8 \& 6.0 \& 7.0
\textbackslash{}end\{array\}


\subsection{Solução}
\label{\detokenize{lista-5-solucoes:id1}}

\subsubsection{Metodologia matemática}
\label{\detokenize{lista-5-solucoes:id2}}\begin{equation*}
\begin{split}g_1(x) = 1$$$$ g_2(x)=x$$$$ g_3(x)=x²$$ $$ g_4(x)=x³\end{split}
\end{equation*}
\sphinxAtStartPar
\textbackslash{}begin\{array\}\{c|cccccc\}
g\_1 \& 1 \& 1 \& 1\& 1\& 1 \& 1 \& 1 \textbackslash{}
g\_2 \& 0.5 \& 0.75 \& 1 \& 1.5 \& 2.0 \& 2.5 \& 3.0  \textbackslash{}
\textbackslash{}hline
g\_3 \& 0.25 \& 0.5625\& 1\& 2.25\& 4 \& 6.25 \& 9.0\textbackslash{}
\textbackslash{}hline
g\_4 \& 0.1250 \& 0.4219 \& 1\& 3.375\& 8 \&15. 625 \& 27 \textbackslash{}
f \& \sphinxhyphen{}2.8 \& \sphinxhyphen{}0.6 \& 1\& 3.2\& 4.8 \& 6.0 \& 7.0
\textbackslash{}end\{array\}
\begin{align*}
a_{11}=\ <g_1,g_1> \ &= 7\\
a_{12}=\ a_{21}   =\  <g_1,g_2> \ &= 11.25 \\
a_{13}=\ a_{31}   =\  <g_1,g_3> \ &= 23.3125 \\
a_{14}=\ a_{41}   =\  <g_1,g_4> \ &= 55.5468 \\
a_{22}=\ <g_2,g_2> \ &= 23.3125\\
a_{23}=\ a_{32}   =\  <g_2,g_3> \ &= 55.5468 \\
a_{24}=\ a_{42}   =\  <g_2,g_4> \ &= 142.5039 \\
a_{33}=\ <g_3,g_3> \ &= 142.5039 \\
a_{34}=\ <g_3,g_4> \ &= 381.5185 \\
a_{44}=\ <g_4,g_4> \ &= 1049.7248 \\
b_{1}=\  <g_1,f>  \ &= 18.6\\
b_{2}=\  <g_2,f>  \ &= 49.55\\
b_{3}=\  <g_3,f>  \ &= 126.86\\
b_{4}=\  <g_4,f>  \ &= 332.34
\end{align*}\begin{equation*}
\begin{split}\begin{bmatrix}
7 & 11.25 & 23.3125 & 55.5468 \\
11.25 &  23.3125 & 55.5468 & 142.5039 \\
23.3125 &  55.5468 & 142.5039 & 381.5185\\
55.5468 &  142.5039 & 381.5185 & 1049.7248
\end{bmatrix}
\begin{bmatrix}
\alpha_1 \\
\alpha_2 \\
\alpha_3 \\
\alpha_4 
\end{bmatrix} = 
\begin{bmatrix}
18.6\\
49.55 \\
126.8625\\
332.3468
\end{bmatrix}\end{split}
\end{equation*}\begin{equation*}
\begin{split}\alpha_1 = -8.1043 ,\  \alpha_2 = 12.7882 \  \alpha_3 = -4.3250 , \  \alpha_4 = 0.5813\end{split}
\end{equation*}\begin{equation*}
\begin{split}f(x) = 0.5813  - 4.3250x + 12.79x^2 - 8.1043x^3\end{split}
\end{equation*}

\subsubsection{Metodologia computacional}
\label{\detokenize{lista-5-solucoes:id3}}
\begin{sphinxVerbatim}[commandchars=\\\{\}]
\PYG{c+c1}{\PYGZsh{} tabela}
\PYG{n}{x} \PYG{o}{=} \PYG{n}{np}\PYG{o}{.}\PYG{n}{array}\PYG{p}{(}\PYG{p}{[}\PYG{l+m+mf}{0.5}\PYG{p}{,}\PYG{l+m+mf}{0.75}\PYG{p}{,}\PYG{l+m+mi}{1}\PYG{p}{,}\PYG{l+m+mf}{1.5}\PYG{p}{,}\PYG{l+m+mf}{2.0}\PYG{p}{,}\PYG{l+m+mf}{2.5}\PYG{p}{,}\PYG{l+m+mf}{3.0}\PYG{p}{]}\PYG{p}{)}
\PYG{n}{y} \PYG{o}{=} \PYG{n}{np}\PYG{o}{.}\PYG{n}{array}\PYG{p}{(}\PYG{p}{[}\PYG{o}{\PYGZhy{}}\PYG{l+m+mf}{2.8}\PYG{p}{,}\PYG{o}{\PYGZhy{}}\PYG{l+m+mf}{0.6}\PYG{p}{,}\PYG{l+m+mi}{1}\PYG{p}{,}\PYG{l+m+mf}{3.2}\PYG{p}{,}\PYG{l+m+mf}{4.8}\PYG{p}{,}\PYG{l+m+mf}{6.0}\PYG{p}{,}\PYG{l+m+mf}{7.0}\PYG{p}{]}\PYG{p}{)}

\PYG{c+c1}{\PYGZsh{} grau (teste para g = 2,3,4,5 e veja o erro)}
\PYG{n}{g} \PYG{o}{=} \PYG{l+m+mi}{3}
\PYG{n}{p} \PYG{o}{=} \PYG{n}{resolve\PYGZus{}ajuste}\PYG{p}{(}\PYG{n}{x}\PYG{p}{,}\PYG{n}{y}\PYG{p}{,}\PYG{n}{g}\PYG{p}{)}

\PYG{c+c1}{\PYGZsh{} erro }
\PYG{n}{np}\PYG{o}{.}\PYG{n}{sum}\PYG{p}{(}\PYG{p}{(}\PYG{n}{p}\PYG{p}{(}\PYG{n}{x}\PYG{p}{)}\PYG{o}{\PYGZhy{}}\PYG{n}{y}\PYG{p}{)}\PYG{o}{*}\PYG{o}{*}\PYG{l+m+mi}{2}\PYG{p}{)}
\end{sphinxVerbatim}

\begin{sphinxVerbatim}[commandchars=\\\{\}]
Coeficiente de ajuste p[0]: 0.588808229974697
Coeficiente de ajuste p[1]: \PYGZhy{}4.365386461549416
Coeficiente de ajuste p[2]: 12.85159200943416
Coeficiente de ajuste p[3]: \PYGZhy{}8.131256481519904
\end{sphinxVerbatim}

\begin{sphinxVerbatim}[commandchars=\\\{\}]
0.040251260748212274
\end{sphinxVerbatim}

\noindent\sphinxincludegraphics{{lista-5-solucoes_22_2}.png}


\subsection{L5\sphinxhyphen{}Q4}
\label{\detokenize{lista-5-solucoes:l5-q4}}
\sphinxAtStartPar
A tabela abaixo mostra as alturas e pesos de nove homens entre as idades de 25 a 29 anos extraı́da ao acaso entre funcionários de uma grande indústria:

\sphinxAtStartPar
\textbackslash{}begin\{array\}\{c|ccccccccc\}
\textbackslash{}text\{Altura/cm\} \& 183\& 173 \& 168 \& 188  \& 158 \& 163 \& 193 \&163 \&178 \textbackslash{}
\textbackslash{}hline
\textbackslash{}text\{Peso/kg\} \& 79 \& 69 \& 70\& 81  \& 61 \&63 \&79 \& 71 \& 73
\textbackslash{}end\{array\}


\subsubsection{Metodologia matemática}
\label{\detokenize{lista-5-solucoes:id4}}\begin{equation*}
\begin{split}g_1(x) = 1$$$$ g_2(x)=x\end{split}
\end{equation*}
\sphinxAtStartPar
\textbackslash{}begin\{array\}\{c|cccccccc\}
g\_1 \& 1 \& 1 \& 1\& 1\& 1 \& 1 \& 1 \& 1\&1\textbackslash{}
\textbackslash{}hline
g\_2 \& 183\& 173 \& 168 \& 188  \& 158 \& 163 \& 193 \&163 \&178 \textbackslash{}
\textbackslash{}hline
f \&  79 \& 69 \& 70\& 81  \& 61 \&63 \&79 \& 71 \& 73
\textbackslash{}end\{array\}
\begin{align*}
a_{11}=\ <g_1,g_1> \ &= 9\\
a_{12}=\ a_{21}   =  <g_1,g_2> \ &= 1567 \\
a_{22}=\ <g_2,g_2> \ &= 274.021\\
b_{1}=\ <g_1,f>  \ &= 646\\
b_{2}=\ <g_2,f>  \ &= 113.103
\end{align*}\begin{equation*}
\begin{split}\begin{bmatrix}
9 & 1567 \\
1567 &  274.021
\end{bmatrix}
\begin{bmatrix}
\alpha_1 \\
\alpha_2 
\end{bmatrix} = 
\begin{bmatrix}
646 \\
113.103
\end{bmatrix}\end{split}
\end{equation*}\begin{equation*}
\begin{split}\alpha_1 = -20.0832 ,\  \alpha_2 = 0.5276\end{split}
\end{equation*}\begin{equation*}
\begin{split}f(x) = 0.5276x - 20.0832\end{split}
\end{equation*}

\subsubsection{Metodologia computacional}
\label{\detokenize{lista-5-solucoes:id5}}
\begin{sphinxVerbatim}[commandchars=\\\{\}]
\PYG{n}{altura} \PYG{o}{=} \PYG{n}{np}\PYG{o}{.}\PYG{n}{array}\PYG{p}{(}\PYG{p}{[}\PYG{l+m+mi}{183}\PYG{p}{,}\PYG{l+m+mi}{173}\PYG{p}{,}\PYG{l+m+mi}{168}\PYG{p}{,}\PYG{l+m+mi}{188}\PYG{p}{,}\PYG{l+m+mi}{158}\PYG{p}{,}\PYG{l+m+mi}{163}\PYG{p}{,}\PYG{l+m+mi}{193}\PYG{p}{,}\PYG{l+m+mi}{163}\PYG{p}{,}\PYG{l+m+mi}{178}\PYG{p}{]}\PYG{p}{)}
\PYG{n}{peso} \PYG{o}{=} \PYG{n}{np}\PYG{o}{.}\PYG{n}{array}\PYG{p}{(}\PYG{p}{[}\PYG{l+m+mi}{79}\PYG{p}{,}\PYG{l+m+mi}{69}\PYG{p}{,}\PYG{l+m+mi}{70}\PYG{p}{,}\PYG{l+m+mi}{81}\PYG{p}{,}\PYG{l+m+mi}{61}\PYG{p}{,}\PYG{l+m+mi}{63}\PYG{p}{,}\PYG{l+m+mi}{79}\PYG{p}{,}\PYG{l+m+mi}{71}\PYG{p}{,}\PYG{l+m+mi}{73}\PYG{p}{]}\PYG{p}{)}

\PYG{c+c1}{\PYGZsh{} gráfico de dispersão }
\PYG{n}{plt}\PYG{o}{.}\PYG{n}{scatter}\PYG{p}{(}\PYG{n}{altura}\PYG{p}{,}\PYG{n}{peso}\PYG{p}{)}
\end{sphinxVerbatim}

\begin{sphinxVerbatim}[commandchars=\\\{\}]
\PYGZlt{}matplotlib.collections.PathCollection at 0x7fc4720ed610\PYGZgt{}
\end{sphinxVerbatim}

\noindent\sphinxincludegraphics{{lista-5-solucoes_26_1}.png}


\subsection{solucao\sphinxhyphen{}L5\sphinxhyphen{}Q4\sphinxhyphen{}b}
\label{\detokenize{lista-5-solucoes:solucao-l5-q4-b}}
\begin{sphinxVerbatim}[commandchars=\\\{\}]
\PYG{c+c1}{\PYGZsh{} ajuste da reta }
\PYG{n}{p} \PYG{o}{=} \PYG{n}{resolve\PYGZus{}ajuste}\PYG{p}{(}\PYG{n}{altura}\PYG{p}{,}\PYG{n}{peso}\PYG{p}{,}\PYG{l+m+mi}{1}\PYG{p}{)}
\end{sphinxVerbatim}

\begin{sphinxVerbatim}[commandchars=\\\{\}]
Coeficiente de ajuste p[0]: 0.527570093457944
Coeficiente de ajuste p[1]: \PYGZhy{}20.078037383177612
\end{sphinxVerbatim}

\noindent\sphinxincludegraphics{{lista-5-solucoes_28_1}.png}


\subsection{solucao\sphinxhyphen{}L5\sphinxhyphen{}Q4\sphinxhyphen{}c}
\label{\detokenize{lista-5-solucoes:solucao-l5-q4-c}}
\begin{sphinxVerbatim}[commandchars=\\\{\}]
\PYG{c+c1}{\PYGZsh{} estimativa de peso (kg)}
\PYG{n}{alt} \PYG{o}{=} \PYG{l+m+mi}{175} 
\PYG{n}{p\PYGZus{}c} \PYG{o}{=} \PYG{n}{p}\PYG{p}{(}\PYG{n}{alt}\PYG{p}{)}
\PYG{n+nb}{print}\PYG{p}{(}\PYG{n}{p\PYGZus{}c}\PYG{p}{)}

\PYG{c+c1}{\PYGZsh{} estimativa de altura (cm)}
\PYG{n}{alt} \PYG{o}{=} \PYG{l+m+mi}{80}
\PYG{n}{a\PYGZus{}c} \PYG{o}{=} \PYG{p}{(}\PYG{n}{alt} \PYG{o}{\PYGZhy{}} \PYG{n}{p}\PYG{p}{[}\PYG{l+m+mi}{0}\PYG{p}{]}\PYG{p}{)}\PYG{o}{/}\PYG{n}{p}\PYG{p}{[}\PYG{l+m+mi}{1}\PYG{p}{]}
\PYG{n+nb}{print}\PYG{p}{(}\PYG{n}{a\PYGZus{}c}\PYG{p}{)}
\end{sphinxVerbatim}

\begin{sphinxVerbatim}[commandchars=\\\{\}]
72.2467289719626
189.69619131975205
\end{sphinxVerbatim}


\subsection{solucao\sphinxhyphen{}L5\sphinxhyphen{}Q4\sphinxhyphen{}d}
\label{\detokenize{lista-5-solucoes:solucao-l5-q4-d}}
\begin{sphinxVerbatim}[commandchars=\\\{\}]
\PYG{n}{p2} \PYG{o}{=} \PYG{n}{resolve\PYGZus{}ajuste}\PYG{p}{(}\PYG{n}{peso}\PYG{p}{,}\PYG{n}{altura}\PYG{p}{,}\PYG{l+m+mi}{1}\PYG{p}{)}
\end{sphinxVerbatim}

\begin{sphinxVerbatim}[commandchars=\\\{\}]
Coeficiente de ajuste p[0]: 1.5856741573033717
Coeficiente de ajuste p[1]: 60.294943820224695
\end{sphinxVerbatim}

\noindent\sphinxincludegraphics{{lista-5-solucoes_32_1}.png}


\subsection{solucao\sphinxhyphen{}L5\sphinxhyphen{}Q4\sphinxhyphen{}e}
\label{\detokenize{lista-5-solucoes:solucao-l5-q4-e}}
\begin{sphinxVerbatim}[commandchars=\\\{\}]
\PYG{c+c1}{\PYGZsh{} estimativa de peso a partir de altura com novo ajuste }
\PYG{n}{alt} \PYG{o}{=} \PYG{l+m+mi}{175}
\PYG{n}{p\PYGZus{}e} \PYG{o}{=} \PYG{p}{(}\PYG{n}{alt} \PYG{o}{\PYGZhy{}} \PYG{n}{p2}\PYG{p}{[}\PYG{l+m+mi}{0}\PYG{p}{]}\PYG{p}{)}\PYG{o}{/}\PYG{n}{p2}\PYG{p}{[}\PYG{l+m+mi}{1}\PYG{p}{]}
\PYG{n+nb}{print}\PYG{p}{(}\PYG{n}{p\PYGZus{}e}\PYG{p}{)}

\PYG{c+c1}{\PYGZsh{} comparação com item (c): pesos }
\PYG{c+c1}{\PYGZsh{} pequena diferença entre os valores}
\PYG{n}{dif} \PYG{o}{=} \PYG{n+nb}{abs}\PYG{p}{(}\PYG{n}{p\PYGZus{}c}\PYG{o}{\PYGZhy{}}\PYG{n}{p\PYGZus{}e}\PYG{p}{)}
\PYG{n+nb}{print}\PYG{p}{(}\PYG{n}{dif}\PYG{p}{)}
\end{sphinxVerbatim}

\begin{sphinxVerbatim}[commandchars=\\\{\}]
72.33835252435782
0.09162355239521958
\end{sphinxVerbatim}


\subsection{solucao\sphinxhyphen{}L5\sphinxhyphen{}Q4\sphinxhyphen{}f}
\label{\detokenize{lista-5-solucoes:solucao-l5-q4-f}}
\begin{sphinxVerbatim}[commandchars=\\\{\}]
\PYG{c+c1}{\PYGZsh{} comparação das retas de ajuste}
\PYG{c+c1}{\PYGZsh{} influência do resíduo =\PYGZgt{} inclinações diferentes}

\PYG{n}{plt}\PYG{o}{.}\PYG{n}{plot}\PYG{p}{(}\PYG{n}{altura}\PYG{p}{,}\PYG{n}{p}\PYG{p}{(}\PYG{n}{altura}\PYG{p}{)}\PYG{p}{,}\PYG{n}{altura}\PYG{p}{,}\PYG{p}{(}\PYG{n}{altura} \PYG{o}{\PYGZhy{}} \PYG{n}{p2}\PYG{p}{[}\PYG{l+m+mi}{0}\PYG{p}{]}\PYG{p}{)}\PYG{o}{/}\PYG{n}{p2}\PYG{p}{[}\PYG{l+m+mi}{1}\PYG{p}{]}\PYG{p}{)}
\end{sphinxVerbatim}

\begin{sphinxVerbatim}[commandchars=\\\{\}]
[\PYGZlt{}matplotlib.lines.Line2D at 0x7fc4724d15d0\PYGZgt{},
 \PYGZlt{}matplotlib.lines.Line2D at 0x7fc4724d1810\PYGZgt{}]
\end{sphinxVerbatim}

\noindent\sphinxincludegraphics{{lista-5-solucoes_36_1}.png}


\section{solucao\sphinxhyphen{}L5\sphinxhyphen{}Q5\sphinxhyphen{}a}
\label{\detokenize{lista-5-solucoes:solucao-l5-q5-a}}

\subsection{AJUSTE POR RETA}
\label{\detokenize{lista-5-solucoes:ajuste-por-reta}}
\begin{sphinxVerbatim}[commandchars=\\\{\}]
\PYG{l+s+sd}{\PYGZdq{}\PYGZdq{}\PYGZdq{} }
\PYG{l+s+sd}{Nota: pesquisa no IBGE em maio de 2018 }
\PYG{l+s+sd}{      mostra que a população já rompeu 209 mi. }
\PYG{l+s+sd}{\PYGZdq{}\PYGZdq{}\PYGZdq{}}

\PYG{c+c1}{\PYGZsh{} tabela de dados}
\PYG{n}{ano} \PYG{o}{=} \PYG{n}{np}\PYG{o}{.}\PYG{n}{array}\PYG{p}{(}\PYG{p}{[}\PYG{l+m+mi}{1900}\PYG{p}{,} \PYG{l+m+mi}{1920}\PYG{p}{,} \PYG{l+m+mi}{1940}\PYG{p}{,} \PYG{l+m+mi}{1950}\PYG{p}{,} \PYG{l+m+mi}{1960}\PYG{p}{,} \PYG{l+m+mi}{1970}\PYG{p}{,} \PYG{l+m+mi}{1980}\PYG{p}{,} \PYG{l+m+mi}{1991}\PYG{p}{,} \PYG{l+m+mi}{2000}\PYG{p}{,} \PYG{l+m+mi}{2010}\PYG{p}{,} \PYG{l+m+mi}{2015}\PYG{p}{]}\PYG{p}{)}
\PYG{n}{hab} \PYG{o}{=} \PYG{n}{np}\PYG{o}{.}\PYG{n}{array}\PYG{p}{(}\PYG{p}{[}\PYG{l+m+mf}{17.4}\PYG{p}{,} \PYG{l+m+mf}{30.6}\PYG{p}{,} \PYG{l+m+mf}{41.2}\PYG{p}{,} \PYG{l+m+mf}{51.9}\PYG{p}{,} \PYG{l+m+mf}{70.2}\PYG{p}{,} \PYG{l+m+mf}{93.1}\PYG{p}{,} \PYG{l+m+mf}{119.0}\PYG{p}{,} \PYG{l+m+mf}{146.2}\PYG{p}{,} \PYG{l+m+mf}{175.8}\PYG{p}{,} \PYG{l+m+mf}{198.6}\PYG{p}{,} \PYG{l+m+mf}{207.8}\PYG{p}{]}\PYG{p}{)}

\PYG{c+c1}{\PYGZsh{} plotagem}
\PYG{n}{fig}\PYG{p}{,}\PYG{n}{ax} \PYG{o}{=} \PYG{n}{plt}\PYG{o}{.}\PYG{n}{subplots}\PYG{p}{(}\PYG{l+m+mi}{1}\PYG{p}{,}\PYG{l+m+mi}{1}\PYG{p}{)}
\PYG{n}{p} \PYG{o}{=} \PYG{n}{resolve\PYGZus{}ajuste}\PYG{p}{(}\PYG{n}{ano}\PYG{p}{,}\PYG{n}{hab}\PYG{p}{,}\PYG{l+m+mi}{1}\PYG{p}{)}
\PYG{n}{ax}\PYG{o}{.}\PYG{n}{grid}\PYG{p}{(}\PYG{n}{axis}\PYG{o}{=}\PYG{l+s+s1}{\PYGZsq{}}\PYG{l+s+s1}{x}\PYG{l+s+s1}{\PYGZsq{}}\PYG{p}{)}

\PYG{c+c1}{\PYGZsh{} plotagem do comportamento preditivo}

\PYG{c+c1}{\PYGZsh{} linha tracejada e area  }
\PYG{n}{ano\PYGZus{}m} \PYG{o}{=} \PYG{n}{np}\PYG{o}{.}\PYG{n}{linspace}\PYG{p}{(}\PYG{l+m+mi}{2016}\PYG{p}{,} \PYG{l+m+mi}{2025}\PYG{p}{,} \PYG{n}{num}\PYG{o}{=}\PYG{l+m+mi}{10}\PYG{p}{,} \PYG{n}{endpoint}\PYG{o}{=}\PYG{k+kc}{True}\PYG{p}{)}
\PYG{n}{v2} \PYG{o}{=} \PYG{n}{np}\PYG{o}{.}\PYG{n}{ones}\PYG{p}{(}\PYG{n}{np}\PYG{o}{.}\PYG{n}{shape}\PYG{p}{(}\PYG{n}{ano\PYGZus{}m}\PYG{p}{)}\PYG{p}{)}
\PYG{n}{ax}\PYG{o}{.}\PYG{n}{fill\PYGZus{}between}\PYG{p}{(}\PYG{n}{ano\PYGZus{}m}\PYG{p}{,}\PYG{l+m+mi}{200}\PYG{o}{*}\PYG{n}{v2}\PYG{p}{,}\PYG{l+m+mi}{220}\PYG{o}{*}\PYG{n}{v2}\PYG{p}{,}\PYG{n}{facecolor}\PYG{o}{=}\PYG{l+s+s1}{\PYGZsq{}}\PYG{l+s+s1}{g}\PYG{l+s+s1}{\PYGZsq{}}\PYG{p}{,}\PYG{n}{alpha}\PYG{o}{=}\PYG{l+m+mf}{0.2}\PYG{p}{)}
\PYG{n}{plt}\PYG{o}{.}\PYG{n}{plot}\PYG{p}{(}\PYG{n}{ano\PYGZus{}m}\PYG{p}{,}\PYG{n}{p}\PYG{p}{(}\PYG{n}{ano\PYGZus{}m}\PYG{p}{)}\PYG{p}{,}\PYG{l+s+s1}{\PYGZsq{}}\PYG{l+s+s1}{:}\PYG{l+s+s1}{\PYGZsq{}}\PYG{p}{)}
\PYG{n}{plt}\PYG{o}{.}\PYG{n}{plot}\PYG{p}{(}\PYG{l+m+mi}{2025}\PYG{p}{,}\PYG{n}{p}\PYG{p}{(}\PYG{l+m+mi}{2025}\PYG{p}{)}\PYG{p}{,}\PYG{l+s+s1}{\PYGZsq{}}\PYG{l+s+s1}{go}\PYG{l+s+s1}{\PYGZsq{}}\PYG{p}{)}

\PYG{c+c1}{\PYGZsh{} estimativa no ano 2025}
\PYG{n}{p}\PYG{p}{(}\PYG{l+m+mi}{2025}\PYG{p}{)}
\end{sphinxVerbatim}

\begin{sphinxVerbatim}[commandchars=\\\{\}]
Coeficiente de ajuste p[0]: 1.7924185769967504
Coeficiente de ajuste p[1]: \PYGZhy{}3420.8153029001523
\end{sphinxVerbatim}

\begin{sphinxVerbatim}[commandchars=\\\{\}]
208.83231551826748
\end{sphinxVerbatim}

\noindent\sphinxincludegraphics{{lista-5-solucoes_39_2}.png}

\begin{sphinxVerbatim}[commandchars=\\\{\}]
\PYG{c+c1}{\PYGZsh{} abrindo vetor de 1900 a 2024}
\PYG{n}{anos} \PYG{o}{=} \PYG{n}{np}\PYG{o}{.}\PYG{n}{arange}\PYG{p}{(}\PYG{n}{ano}\PYG{p}{[}\PYG{l+m+mi}{0}\PYG{p}{]}\PYG{p}{,}\PYG{n}{ano}\PYG{p}{[}\PYG{o}{\PYGZhy{}}\PYG{l+m+mi}{1}\PYG{p}{]}\PYG{o}{+}\PYG{l+m+mi}{10}\PYG{p}{)}
\PYG{n}{pops} \PYG{o}{=} \PYG{n}{p}\PYG{p}{(}\PYG{n}{anos}\PYG{p}{)} \PYG{c+c1}{\PYGZsh{} população}

\PYG{c+c1}{\PYGZsh{} encontrando ano em que a populacao superou 100 mi}
\PYG{n}{lim100} \PYG{o}{=} \PYG{n}{np}\PYG{o}{.}\PYG{n}{nonzero}\PYG{p}{(}\PYG{n}{pops}\PYG{o}{\PYGZgt{}}\PYG{l+m+mi}{100}\PYG{p}{)}
\PYG{n}{lim150} \PYG{o}{=} \PYG{n}{np}\PYG{o}{.}\PYG{n}{nonzero}\PYG{p}{(}\PYG{n}{pops}\PYG{o}{\PYGZgt{}}\PYG{l+m+mi}{150}\PYG{p}{)}
\PYG{n}{lim200} \PYG{o}{=} \PYG{n}{np}\PYG{o}{.}\PYG{n}{nonzero}\PYG{p}{(}\PYG{n}{pops}\PYG{o}{\PYGZgt{}}\PYG{l+m+mi}{200}\PYG{p}{)}
\PYG{n}{np}\PYG{o}{.}\PYG{n}{shape}\PYG{p}{(}\PYG{n}{lim100}\PYG{p}{)}
\PYG{n+nb}{print}\PYG{p}{(}\PYG{l+s+s1}{\PYGZsq{}}\PYG{l+s+s1}{Marca de 100 milhões de pessoas: }\PYG{l+s+s1}{\PYGZsq{}} \PYG{o}{+} \PYG{n+nb}{str}\PYG{p}{(}\PYG{n}{anos}\PYG{p}{[}\PYG{n}{lim100}\PYG{p}{[}\PYG{l+m+mi}{0}\PYG{p}{]}\PYG{p}{[}\PYG{l+m+mi}{0}\PYG{p}{]}\PYG{p}{]}\PYG{p}{)}\PYG{p}{)}
\PYG{n+nb}{print}\PYG{p}{(}\PYG{l+s+s1}{\PYGZsq{}}\PYG{l+s+s1}{Marca de 150 milhões de pessoas: }\PYG{l+s+s1}{\PYGZsq{}} \PYG{o}{+} \PYG{n+nb}{str}\PYG{p}{(}\PYG{n}{anos}\PYG{p}{[}\PYG{n}{lim150}\PYG{p}{[}\PYG{l+m+mi}{0}\PYG{p}{]}\PYG{p}{[}\PYG{l+m+mi}{0}\PYG{p}{]}\PYG{p}{]}\PYG{p}{)}\PYG{p}{)}
\PYG{n+nb}{print}\PYG{p}{(}\PYG{l+s+s1}{\PYGZsq{}}\PYG{l+s+s1}{Marca de 200 milhões de pessoas: }\PYG{l+s+s1}{\PYGZsq{}} \PYG{o}{+} \PYG{n+nb}{str}\PYG{p}{(}\PYG{n}{anos}\PYG{p}{[}\PYG{n}{lim200}\PYG{p}{[}\PYG{l+m+mi}{0}\PYG{p}{]}\PYG{p}{[}\PYG{l+m+mi}{0}\PYG{p}{]}\PYG{p}{]}\PYG{p}{)}\PYG{p}{)}
\end{sphinxVerbatim}

\begin{sphinxVerbatim}[commandchars=\\\{\}]
Marca de 100 milhões de pessoas: 1965
Marca de 150 milhões de pessoas: 1993
Marca de 200 milhões de pessoas: 2021
\end{sphinxVerbatim}

\begin{sphinxVerbatim}[commandchars=\\\{\}]
\PYG{c+c1}{\PYGZsh{} plotagem por faixas de valores }
\PYG{n}{fig}\PYG{p}{,}\PYG{n}{ax} \PYG{o}{=} \PYG{n}{plt}\PYG{o}{.}\PYG{n}{subplots}\PYG{p}{(}\PYG{l+m+mi}{1}\PYG{p}{,}\PYG{l+m+mi}{1}\PYG{p}{)}
\PYG{n}{p} \PYG{o}{=} \PYG{n}{resolve\PYGZus{}ajuste}\PYG{p}{(}\PYG{n}{ano}\PYG{p}{,}\PYG{n}{hab}\PYG{p}{,}\PYG{l+m+mi}{1}\PYG{p}{)}
\PYG{n}{ax}\PYG{o}{.}\PYG{n}{grid}\PYG{p}{(}\PYG{n}{axis}\PYG{o}{=}\PYG{l+s+s1}{\PYGZsq{}}\PYG{l+s+s1}{x}\PYG{l+s+s1}{\PYGZsq{}}\PYG{p}{)}
\PYG{n}{v1} \PYG{o}{=} \PYG{n}{np}\PYG{o}{.}\PYG{n}{ones}\PYG{p}{(}\PYG{n}{np}\PYG{o}{.}\PYG{n}{shape}\PYG{p}{(}\PYG{n}{ano}\PYG{p}{)}\PYG{p}{)}
\PYG{n}{ax}\PYG{o}{.}\PYG{n}{fill\PYGZus{}between}\PYG{p}{(}\PYG{n}{ano}\PYG{p}{,}\PYG{l+m+mi}{100}\PYG{o}{*}\PYG{n}{v1}\PYG{p}{,}\PYG{l+m+mi}{150}\PYG{o}{*}\PYG{n}{v1}\PYG{p}{,}\PYG{n}{alpha}\PYG{o}{=}\PYG{l+m+mf}{0.1}\PYG{p}{)}
\PYG{n}{ax}\PYG{o}{.}\PYG{n}{fill\PYGZus{}between}\PYG{p}{(}\PYG{n}{ano}\PYG{p}{,}\PYG{l+m+mi}{150}\PYG{o}{*}\PYG{n}{v1}\PYG{p}{,}\PYG{l+m+mi}{200}\PYG{o}{*}\PYG{n}{v1}\PYG{p}{,}\PYG{n}{facecolor}\PYG{o}{=}\PYG{l+s+s1}{\PYGZsq{}}\PYG{l+s+s1}{r}\PYG{l+s+s1}{\PYGZsq{}}\PYG{p}{,}\PYG{n}{alpha}\PYG{o}{=}\PYG{l+m+mf}{0.1}\PYG{p}{)}
\end{sphinxVerbatim}

\begin{sphinxVerbatim}[commandchars=\\\{\}]
Coeficiente de ajuste p[0]: 1.7924185769967504
Coeficiente de ajuste p[1]: \PYGZhy{}3420.8153029001523
\end{sphinxVerbatim}

\begin{sphinxVerbatim}[commandchars=\\\{\}]
\PYGZlt{}matplotlib.collections.PolyCollection at 0x7fc47246b410\PYGZgt{}
\end{sphinxVerbatim}

\noindent\sphinxincludegraphics{{lista-5-solucoes_41_2}.png}


\subsection{AJUSTE POR PARÁBOLA}
\label{\detokenize{lista-5-solucoes:ajuste-por-parabola}}
\begin{sphinxVerbatim}[commandchars=\\\{\}]
\PYG{c+c1}{\PYGZsh{} plotagem}
\PYG{n}{fig}\PYG{p}{,}\PYG{n}{ax} \PYG{o}{=} \PYG{n}{plt}\PYG{o}{.}\PYG{n}{subplots}\PYG{p}{(}\PYG{l+m+mi}{1}\PYG{p}{,}\PYG{l+m+mi}{1}\PYG{p}{)}
\PYG{n}{p} \PYG{o}{=} \PYG{n}{resolve\PYGZus{}ajuste}\PYG{p}{(}\PYG{n}{ano}\PYG{p}{,}\PYG{n}{hab}\PYG{p}{,}\PYG{l+m+mi}{2}\PYG{p}{)}
\PYG{n}{ax}\PYG{o}{.}\PYG{n}{grid}\PYG{p}{(}\PYG{n}{axis}\PYG{o}{=}\PYG{l+s+s1}{\PYGZsq{}}\PYG{l+s+s1}{x}\PYG{l+s+s1}{\PYGZsq{}}\PYG{p}{)}

\PYG{c+c1}{\PYGZsh{} plotagem do comportamento preditivo}

\PYG{c+c1}{\PYGZsh{} linha tracejada e area  }
\PYG{n}{ano\PYGZus{}m} \PYG{o}{=} \PYG{n}{np}\PYG{o}{.}\PYG{n}{linspace}\PYG{p}{(}\PYG{l+m+mi}{2016}\PYG{p}{,} \PYG{l+m+mi}{2025}\PYG{p}{,} \PYG{n}{num}\PYG{o}{=}\PYG{l+m+mi}{10}\PYG{p}{,} \PYG{n}{endpoint}\PYG{o}{=}\PYG{k+kc}{True}\PYG{p}{)}
\PYG{n}{v2} \PYG{o}{=} \PYG{n}{np}\PYG{o}{.}\PYG{n}{ones}\PYG{p}{(}\PYG{n}{np}\PYG{o}{.}\PYG{n}{shape}\PYG{p}{(}\PYG{n}{ano\PYGZus{}m}\PYG{p}{)}\PYG{p}{)}
\PYG{n}{ax}\PYG{o}{.}\PYG{n}{fill\PYGZus{}between}\PYG{p}{(}\PYG{n}{ano\PYGZus{}m}\PYG{p}{,}\PYG{l+m+mi}{200}\PYG{o}{*}\PYG{n}{v2}\PYG{p}{,}\PYG{l+m+mi}{260}\PYG{o}{*}\PYG{n}{v2}\PYG{p}{,}\PYG{n}{facecolor}\PYG{o}{=}\PYG{l+s+s1}{\PYGZsq{}}\PYG{l+s+s1}{g}\PYG{l+s+s1}{\PYGZsq{}}\PYG{p}{,}\PYG{n}{alpha}\PYG{o}{=}\PYG{l+m+mf}{0.2}\PYG{p}{)}
\PYG{n}{plt}\PYG{o}{.}\PYG{n}{plot}\PYG{p}{(}\PYG{n}{ano\PYGZus{}m}\PYG{p}{,}\PYG{n}{p}\PYG{p}{(}\PYG{n}{ano\PYGZus{}m}\PYG{p}{)}\PYG{p}{,}\PYG{l+s+s1}{\PYGZsq{}}\PYG{l+s+s1}{:}\PYG{l+s+s1}{\PYGZsq{}}\PYG{p}{)}
\PYG{n}{plt}\PYG{o}{.}\PYG{n}{plot}\PYG{p}{(}\PYG{l+m+mi}{2025}\PYG{p}{,}\PYG{n}{p}\PYG{p}{(}\PYG{l+m+mi}{2025}\PYG{p}{)}\PYG{p}{,}\PYG{l+s+s1}{\PYGZsq{}}\PYG{l+s+s1}{go}\PYG{l+s+s1}{\PYGZsq{}}\PYG{p}{)}

\PYG{c+c1}{\PYGZsh{} estimativa no ano 2025}
\PYG{n}{p}\PYG{p}{(}\PYG{l+m+mi}{2025}\PYG{p}{)}
\end{sphinxVerbatim}

\begin{sphinxVerbatim}[commandchars=\\\{\}]
Coeficiente de ajuste p[0]: 0.01418349430843006
Coeficiente de ajuste p[1]: \PYGZhy{}53.81354030390032
Coeficiente de ajuste p[2]: 51061.1779002923
\end{sphinxVerbatim}

\begin{sphinxVerbatim}[commandchars=\\\{\}]
249.95013340016885
\end{sphinxVerbatim}

\noindent\sphinxincludegraphics{{lista-5-solucoes_43_2}.png}

\begin{sphinxVerbatim}[commandchars=\\\{\}]
\PYG{c+c1}{\PYGZsh{} abrindo vetor de 1900 a 2024}
\PYG{n}{anos} \PYG{o}{=} \PYG{n}{np}\PYG{o}{.}\PYG{n}{arange}\PYG{p}{(}\PYG{n}{ano}\PYG{p}{[}\PYG{l+m+mi}{0}\PYG{p}{]}\PYG{p}{,}\PYG{n}{ano}\PYG{p}{[}\PYG{o}{\PYGZhy{}}\PYG{l+m+mi}{1}\PYG{p}{]}\PYG{o}{+}\PYG{l+m+mi}{10}\PYG{p}{)}
\PYG{n}{pops} \PYG{o}{=} \PYG{n}{p}\PYG{p}{(}\PYG{n}{anos}\PYG{p}{)} \PYG{c+c1}{\PYGZsh{} população}

\PYG{c+c1}{\PYGZsh{} encontrando ano em que a populacao superou 100 mi}
\PYG{n}{lim100} \PYG{o}{=} \PYG{n}{np}\PYG{o}{.}\PYG{n}{nonzero}\PYG{p}{(}\PYG{n}{pops}\PYG{o}{\PYGZgt{}}\PYG{l+m+mi}{100}\PYG{p}{)}
\PYG{n}{lim150} \PYG{o}{=} \PYG{n}{np}\PYG{o}{.}\PYG{n}{nonzero}\PYG{p}{(}\PYG{n}{pops}\PYG{o}{\PYGZgt{}}\PYG{l+m+mi}{150}\PYG{p}{)}
\PYG{n}{lim200} \PYG{o}{=} \PYG{n}{np}\PYG{o}{.}\PYG{n}{nonzero}\PYG{p}{(}\PYG{n}{pops}\PYG{o}{\PYGZgt{}}\PYG{l+m+mi}{200}\PYG{p}{)}
\PYG{n}{np}\PYG{o}{.}\PYG{n}{shape}\PYG{p}{(}\PYG{n}{lim100}\PYG{p}{)}
\PYG{n+nb}{print}\PYG{p}{(}\PYG{l+s+s1}{\PYGZsq{}}\PYG{l+s+s1}{Marca de 100 milhões de pessoas: }\PYG{l+s+s1}{\PYGZsq{}} \PYG{o}{+} \PYG{n+nb}{str}\PYG{p}{(}\PYG{n}{anos}\PYG{p}{[}\PYG{n}{lim100}\PYG{p}{[}\PYG{l+m+mi}{0}\PYG{p}{]}\PYG{p}{[}\PYG{l+m+mi}{0}\PYG{p}{]}\PYG{p}{]}\PYG{p}{)}\PYG{p}{)}
\PYG{n+nb}{print}\PYG{p}{(}\PYG{l+s+s1}{\PYGZsq{}}\PYG{l+s+s1}{Marca de 150 milhões de pessoas: }\PYG{l+s+s1}{\PYGZsq{}} \PYG{o}{+} \PYG{n+nb}{str}\PYG{p}{(}\PYG{n}{anos}\PYG{p}{[}\PYG{n}{lim150}\PYG{p}{[}\PYG{l+m+mi}{0}\PYG{p}{]}\PYG{p}{[}\PYG{l+m+mi}{0}\PYG{p}{]}\PYG{p}{]}\PYG{p}{)}\PYG{p}{)}
\PYG{n+nb}{print}\PYG{p}{(}\PYG{l+s+s1}{\PYGZsq{}}\PYG{l+s+s1}{Marca de 200 milhões de pessoas: }\PYG{l+s+s1}{\PYGZsq{}} \PYG{o}{+} \PYG{n+nb}{str}\PYG{p}{(}\PYG{n}{anos}\PYG{p}{[}\PYG{n}{lim200}\PYG{p}{[}\PYG{l+m+mi}{0}\PYG{p}{]}\PYG{p}{[}\PYG{l+m+mi}{0}\PYG{p}{]}\PYG{p}{]}\PYG{p}{)}\PYG{p}{)}
\end{sphinxVerbatim}

\begin{sphinxVerbatim}[commandchars=\\\{\}]
Marca de 100 milhões de pessoas: 1974
Marca de 150 milhões de pessoas: 1994
Marca de 200 milhões de pessoas: 2011
\end{sphinxVerbatim}

\begin{sphinxVerbatim}[commandchars=\\\{\}]
\PYG{c+c1}{\PYGZsh{} plotagem por faixas de valores }
\PYG{n}{fig}\PYG{p}{,}\PYG{n}{ax} \PYG{o}{=} \PYG{n}{plt}\PYG{o}{.}\PYG{n}{subplots}\PYG{p}{(}\PYG{l+m+mi}{1}\PYG{p}{,}\PYG{l+m+mi}{1}\PYG{p}{)}
\PYG{n}{p} \PYG{o}{=} \PYG{n}{resolve\PYGZus{}ajuste}\PYG{p}{(}\PYG{n}{ano}\PYG{p}{,}\PYG{n}{hab}\PYG{p}{,}\PYG{l+m+mi}{2}\PYG{p}{)}
\PYG{n}{ax}\PYG{o}{.}\PYG{n}{grid}\PYG{p}{(}\PYG{n}{axis}\PYG{o}{=}\PYG{l+s+s1}{\PYGZsq{}}\PYG{l+s+s1}{x}\PYG{l+s+s1}{\PYGZsq{}}\PYG{p}{)}
\PYG{n}{v1} \PYG{o}{=} \PYG{n}{np}\PYG{o}{.}\PYG{n}{ones}\PYG{p}{(}\PYG{n}{np}\PYG{o}{.}\PYG{n}{shape}\PYG{p}{(}\PYG{n}{ano}\PYG{p}{)}\PYG{p}{)}
\PYG{n}{ax}\PYG{o}{.}\PYG{n}{fill\PYGZus{}between}\PYG{p}{(}\PYG{n}{ano}\PYG{p}{,}\PYG{l+m+mi}{100}\PYG{o}{*}\PYG{n}{v1}\PYG{p}{,}\PYG{l+m+mi}{150}\PYG{o}{*}\PYG{n}{v1}\PYG{p}{,}\PYG{n}{alpha}\PYG{o}{=}\PYG{l+m+mf}{0.1}\PYG{p}{)}
\PYG{n}{ax}\PYG{o}{.}\PYG{n}{fill\PYGZus{}between}\PYG{p}{(}\PYG{n}{ano}\PYG{p}{,}\PYG{l+m+mi}{150}\PYG{o}{*}\PYG{n}{v1}\PYG{p}{,}\PYG{l+m+mi}{200}\PYG{o}{*}\PYG{n}{v1}\PYG{p}{,}\PYG{n}{facecolor}\PYG{o}{=}\PYG{l+s+s1}{\PYGZsq{}}\PYG{l+s+s1}{r}\PYG{l+s+s1}{\PYGZsq{}}\PYG{p}{,}\PYG{n}{alpha}\PYG{o}{=}\PYG{l+m+mf}{0.1}\PYG{p}{)}
\end{sphinxVerbatim}

\begin{sphinxVerbatim}[commandchars=\\\{\}]
Coeficiente de ajuste p[0]: 0.01418349430843006
Coeficiente de ajuste p[1]: \PYGZhy{}53.81354030390032
Coeficiente de ajuste p[2]: 51061.1779002923
\end{sphinxVerbatim}

\begin{sphinxVerbatim}[commandchars=\\\{\}]
\PYGZlt{}matplotlib.collections.PolyCollection at 0x7fc471ff9310\PYGZgt{}
\end{sphinxVerbatim}

\noindent\sphinxincludegraphics{{lista-5-solucoes_45_2}.png}


\section{Integração Numérica}
\label{\detokenize{lista-5-solucoes:integracao-numerica}}

\subsection{Função\sphinxhyphen{}base para integrais de Newton\sphinxhyphen{}Cotes}
\label{\detokenize{lista-5-solucoes:funcao-base-para-integrais-de-newton-cotes}}
\begin{sphinxVerbatim}[commandchars=\\\{\}]
\PYG{k}{def} \PYG{n+nf}{integral\PYGZus{}newton\PYGZus{}cotes}\PYG{p}{(}\PYG{n}{x}\PYG{p}{,}\PYG{n}{y}\PYG{p}{,}\PYG{n}{metodo}\PYG{p}{)}\PYG{p}{:}

    \PYG{c+c1}{\PYGZsh{} diff computa h = x[k+1] \PYGZhy{} x[k]; }
    \PYG{c+c1}{\PYGZsh{} duplo diff retorna 0 se igualmente espaçado}
    \PYG{n}{h} \PYG{o}{=} \PYG{n}{np}\PYG{o}{.}\PYG{n}{diff}\PYG{p}{(}\PYG{n}{x}\PYG{p}{)}
    \PYG{n}{hh} \PYG{o}{=} \PYG{n}{np}\PYG{o}{.}\PYG{n}{diff}\PYG{p}{(}\PYG{n}{h}\PYG{p}{)} 

    \PYG{c+c1}{\PYGZsh{} verifica se vetor é, de fato, de zeros, dentro de tolerância}
    \PYG{c+c1}{\PYGZsh{} se não, lança erro}
    \PYG{n}{np}\PYG{o}{.}\PYG{n}{testing}\PYG{o}{.}\PYG{n}{assert\PYGZus{}allclose}\PYG{p}{(}\PYG{n}{hh}\PYG{p}{,} \PYG{l+m+mi}{0}\PYG{o}{*}\PYG{n}{hh}\PYG{p}{,} \PYG{n}{atol}\PYG{o}{=}\PYG{l+m+mf}{1e\PYGZhy{}08}\PYG{p}{)}

    \PYG{c+c1}{\PYGZsh{} switch    }
    \PYG{k}{if} \PYG{n}{metodo} \PYG{o+ow}{is} \PYG{l+s+s1}{\PYGZsq{}}\PYG{l+s+s1}{trapezio}\PYG{l+s+s1}{\PYGZsq{}}\PYG{p}{:}
    
        \PYG{c+c1}{\PYGZsh{} montando vetor de somas y[k] + y[k+1]    }
        \PYG{n}{cs} \PYG{o}{=} \PYG{n}{np}\PYG{o}{.}\PYG{n}{cumsum}\PYG{p}{(}\PYG{n}{y}\PYG{p}{)}
        \PYG{n}{u} \PYG{o}{=} \PYG{n}{np}\PYG{o}{.}\PYG{n}{concatenate}\PYG{p}{(}\PYG{p}{(}\PYG{n}{np}\PYG{o}{.}\PYG{n}{array}\PYG{p}{(}\PYG{p}{[}\PYG{l+m+mi}{0}\PYG{p}{]}\PYG{p}{)}\PYG{p}{,}\PYG{n}{cs}\PYG{p}{[}\PYG{l+m+mi}{0}\PYG{p}{:}\PYG{o}{\PYGZhy{}}\PYG{l+m+mi}{2}\PYG{p}{]}\PYG{p}{)}\PYG{p}{)}
        \PYG{n}{ss} \PYG{o}{=} \PYG{n}{cs}\PYG{p}{[}\PYG{l+m+mi}{1}\PYG{p}{:}\PYG{p}{]}\PYG{o}{\PYGZhy{}}\PYG{n}{u} \PYG{c+c1}{\PYGZsh{} somas }

        \PYG{c+c1}{\PYGZsh{} integral (regra generalizada)}
        \PYG{n}{integral} \PYG{o}{=} \PYG{n}{h}\PYG{p}{[}\PYG{l+m+mi}{0}\PYG{p}{]}\PYG{o}{/}\PYG{l+m+mi}{2}\PYG{o}{*}\PYG{n}{np}\PYG{o}{.}\PYG{n}{sum}\PYG{p}{(}\PYG{n}{ss}\PYG{p}{)}
        
    \PYG{k}{elif} \PYG{n}{metodo} \PYG{o+ow}{is} \PYG{l+s+s1}{\PYGZsq{}}\PYG{l+s+s1}{simpson13}\PYG{l+s+s1}{\PYGZsq{}}\PYG{p}{:} 
        
        \PYG{k}{if} \PYG{n}{np}\PYG{o}{.}\PYG{n}{size}\PYG{p}{(}\PYG{n}{x}\PYG{p}{)} \PYG{o}{\PYGZpc{}} \PYG{l+m+mi}{2} \PYG{o+ow}{is} \PYG{l+m+mi}{0}\PYG{p}{:}
            \PYG{k}{raise} \PYG{n+ne}{ValueError}\PYG{p}{(}\PYG{l+s+s1}{\PYGZsq{}}\PYG{l+s+s1}{Regra de Simpson válida apenas para número ímpar de pontos.}\PYG{l+s+s1}{\PYGZsq{}}\PYG{p}{)}    
    
        \PYG{c+c1}{\PYGZsh{} constroi pesos }
        
        \PYG{c+c1}{\PYGZsh{} ignora primeiro e ultimo}
        \PYG{n}{ie} \PYG{o}{=} \PYG{n}{np}\PYG{o}{.}\PYG{n}{array}\PYG{p}{(}\PYG{n+nb}{range}\PYG{p}{(}\PYG{l+m+mi}{0}\PYG{p}{,}\PYG{n}{np}\PYG{o}{.}\PYG{n}{size}\PYG{p}{(}\PYG{n}{x}\PYG{p}{)}\PYG{p}{)}\PYG{p}{)}
        \PYG{n}{ie} \PYG{o}{=} \PYG{n}{ie}\PYG{p}{[}\PYG{l+m+mi}{1}\PYG{p}{:}\PYG{o}{\PYGZhy{}}\PYG{l+m+mi}{1}\PYG{p}{]} \PYG{o}{\PYGZpc{}} \PYG{l+m+mi}{2}

        \PYG{c+c1}{\PYGZsh{} pesos para intermediarios}
        \PYG{k}{for} \PYG{n}{v} \PYG{o+ow}{in} \PYG{n+nb}{range}\PYG{p}{(}\PYG{n}{np}\PYG{o}{.}\PYG{n}{size}\PYG{p}{(}\PYG{n}{ie}\PYG{p}{)}\PYG{p}{)}\PYG{p}{:}
            \PYG{k}{if} \PYG{n}{ie}\PYG{p}{[}\PYG{n}{v}\PYG{p}{]} \PYG{o}{==} \PYG{l+m+mi}{1}\PYG{p}{:}
                \PYG{n}{ie}\PYG{p}{[}\PYG{n}{v}\PYG{p}{]} \PYG{o}{=} \PYG{l+m+mi}{4}
            \PYG{k}{elif} \PYG{n}{ie}\PYG{p}{[}\PYG{n}{v}\PYG{p}{]} \PYG{o}{==} \PYG{l+m+mi}{0}\PYG{p}{:}
                \PYG{n}{ie}\PYG{p}{[}\PYG{n}{v}\PYG{p}{]} \PYG{o}{=} \PYG{l+m+mi}{1}            
    
        \PYG{c+c1}{\PYGZsh{} concatena para recriar }
        \PYG{n}{ie} \PYG{o}{=} \PYG{n}{np}\PYG{o}{.}\PYG{n}{concatenate}\PYG{p}{(}\PYG{p}{(}\PYG{p}{[}\PYG{l+m+mi}{1}\PYG{p}{]}\PYG{p}{,}\PYG{n}{ie}\PYG{p}{,}\PYG{p}{[}\PYG{l+m+mi}{1}\PYG{p}{]}\PYG{p}{)}\PYG{p}{)}        
        
        \PYG{c+c1}{\PYGZsh{} integral (regra generalizada)}
        \PYG{n}{integral} \PYG{o}{=} \PYG{n}{h}\PYG{p}{[}\PYG{l+m+mi}{0}\PYG{p}{]}\PYG{o}{/}\PYG{l+m+mi}{3}\PYG{o}{*}\PYG{p}{(}\PYG{n}{np}\PYG{o}{.}\PYG{n}{sum}\PYG{p}{(}\PYG{n}{y}\PYG{o}{*}\PYG{n}{ie}\PYG{p}{)}\PYG{p}{)}
        
            
    \PYG{k}{return} \PYG{n}{integral} 
\end{sphinxVerbatim}


\subsection{Exemplo}
\label{\detokenize{lista-5-solucoes:exemplo}}
\sphinxAtStartPar
Integração numérica pela regra do trapézio para a função
\$\(\int_{a=0}^{b=\pi} [ {\rm sen}(3x + 2) + 0.5\pi ] \, dx \approx I_T \)\$

\begin{sphinxVerbatim}[commandchars=\\\{\}]
\PYG{c+c1}{\PYGZsh{} função}
\PYG{n}{a} \PYG{o}{=} \PYG{l+m+mi}{0}
\PYG{n}{b} \PYG{o}{=} \PYG{n}{np}\PYG{o}{.}\PYG{n}{pi}
\PYG{n}{x} \PYG{o}{=} \PYG{n}{np}\PYG{o}{.}\PYG{n}{linspace}\PYG{p}{(}\PYG{n}{a}\PYG{p}{,}\PYG{n}{b}\PYG{p}{,}\PYG{n}{num}\PYG{o}{=}\PYG{l+m+mi}{5}\PYG{p}{,}\PYG{n}{endpoint}\PYG{o}{=}\PYG{k+kc}{True}\PYG{p}{)}
\PYG{n}{f} \PYG{o}{=} \PYG{k}{lambda} \PYG{n}{x}\PYG{p}{:} \PYG{n}{np}\PYG{o}{.}\PYG{n}{sin}\PYG{p}{(}\PYG{l+m+mi}{3}\PYG{o}{*}\PYG{n}{x} \PYG{o}{+} \PYG{l+m+mi}{2}\PYG{p}{)} \PYG{o}{+} \PYG{n}{np}\PYG{o}{.}\PYG{n}{pi}\PYG{o}{/}\PYG{l+m+mi}{2}
\PYG{n}{y} \PYG{o}{=} \PYG{n}{f}\PYG{p}{(}\PYG{n}{x}\PYG{p}{)}

\PYG{c+c1}{\PYGZsh{} integração por trapézio}
\PYG{n}{integral\PYGZus{}newton\PYGZus{}cotes}\PYG{p}{(}\PYG{n}{x}\PYG{p}{,}\PYG{n}{y}\PYG{p}{,}\PYG{l+s+s1}{\PYGZsq{}}\PYG{l+s+s1}{trapezio}\PYG{l+s+s1}{\PYGZsq{}}\PYG{p}{)}
\end{sphinxVerbatim}

\begin{sphinxVerbatim}[commandchars=\\\{\}]
4.799420241706493
\end{sphinxVerbatim}

\begin{sphinxVerbatim}[commandchars=\\\{\}]
\PYG{c+c1}{\PYGZsh{} plotagem dos trapézios}
\PYG{n}{plt}\PYG{o}{.}\PYG{n}{stem}\PYG{p}{(}\PYG{n}{x}\PYG{p}{,}\PYG{n}{y}\PYG{p}{,}\PYG{l+s+s1}{\PYGZsq{}}\PYG{l+s+s1}{\PYGZhy{}ok}\PYG{l+s+s1}{\PYGZsq{}}\PYG{p}{,}\PYG{n}{basefmt}\PYG{o}{=}\PYG{l+s+s1}{\PYGZsq{}}\PYG{l+s+s1}{k\PYGZhy{}}\PYG{l+s+s1}{\PYGZsq{}}\PYG{p}{,}\PYG{n}{use\PYGZus{}line\PYGZus{}collection}\PYG{o}{=}\PYG{k+kc}{True}\PYG{p}{)}
\PYG{n}{plt}\PYG{o}{.}\PYG{n}{plot}\PYG{p}{(}\PYG{n}{x}\PYG{p}{,}\PYG{n}{y}\PYG{p}{,}\PYG{l+s+s1}{\PYGZsq{}}\PYG{l+s+s1}{\PYGZhy{}k}\PYG{l+s+s1}{\PYGZsq{}}\PYG{p}{,}\PYG{n}{label}\PYG{o}{=}\PYG{l+s+s1}{\PYGZsq{}}\PYG{l+s+s1}{\PYGZdl{}I\PYGZus{}T\PYGZdl{}}\PYG{l+s+s1}{\PYGZsq{}}\PYG{p}{)}
\PYG{n}{plt}\PYG{o}{.}\PYG{n}{fill\PYGZus{}between}\PYG{p}{(}\PYG{n}{x}\PYG{p}{,}\PYG{l+m+mi}{0}\PYG{o}{*}\PYG{n}{y}\PYG{p}{,}\PYG{n}{y}\PYG{p}{,}\PYG{n}{facecolor}\PYG{o}{=}\PYG{l+s+s1}{\PYGZsq{}}\PYG{l+s+s1}{r}\PYG{l+s+s1}{\PYGZsq{}}\PYG{p}{,}\PYG{n}{alpha}\PYG{o}{=}\PYG{l+m+mf}{0.1}\PYG{p}{)}

\PYG{c+c1}{\PYGZsh{} \PYGZdq{}imitação\PYGZdq{} da função contínua}
\PYG{n}{xx} \PYG{o}{=} \PYG{n}{np}\PYG{o}{.}\PYG{n}{linspace}\PYG{p}{(}\PYG{n}{a}\PYG{p}{,}\PYG{n}{b}\PYG{p}{,}\PYG{n}{num}\PYG{o}{=}\PYG{l+m+mi}{200}\PYG{p}{,}\PYG{n}{endpoint}\PYG{o}{=}\PYG{k+kc}{True}\PYG{p}{)}
\PYG{n}{plt}\PYG{o}{.}\PYG{n}{plot}\PYG{p}{(}\PYG{n}{xx}\PYG{p}{,}\PYG{n}{f}\PYG{p}{(}\PYG{n}{xx}\PYG{p}{)}\PYG{p}{,}\PYG{n}{color}\PYG{o}{=}\PYG{p}{[}\PYG{l+m+mf}{0.5}\PYG{p}{,}\PYG{l+m+mf}{0.5}\PYG{p}{,}\PYG{l+m+mf}{0.5}\PYG{p}{]}\PYG{p}{,}\PYG{n}{label}\PYG{o}{=}\PYG{l+s+s1}{\PYGZsq{}}\PYG{l+s+s1}{\PYGZdl{}f(x)\PYGZdl{}}\PYG{l+s+s1}{\PYGZsq{}}\PYG{p}{)}
\PYG{n}{plt}\PYG{o}{.}\PYG{n}{legend}\PYG{p}{(}\PYG{p}{)}\PYG{p}{;}
\end{sphinxVerbatim}

\noindent\sphinxincludegraphics{{lista-5-solucoes_51_0}.png}


\subsection{Exemplo}
\label{\detokenize{lista-5-solucoes:id6}}
\sphinxAtStartPar
Idem, para regra 1/3 de Simpson \(I_S\)

\begin{sphinxVerbatim}[commandchars=\\\{\}]
\PYG{c+c1}{\PYGZsh{} integração por 1/3 Simpson }

\PYG{n}{integral\PYGZus{}newton\PYGZus{}cotes}\PYG{p}{(}\PYG{n}{x}\PYG{p}{,}\PYG{n}{y}\PYG{p}{,}\PYG{l+s+s1}{\PYGZsq{}}\PYG{l+s+s1}{simpson13}\PYG{l+s+s1}{\PYGZsq{}}\PYG{p}{)}
\end{sphinxVerbatim}

\begin{sphinxVerbatim}[commandchars=\\\{\}]
4.016218444253924
\end{sphinxVerbatim}


\section{Função residente em Python para integração}
\label{\detokenize{lista-5-solucoes:funcao-residente-em-python-para-integracao}}
\begin{sphinxVerbatim}[commandchars=\\\{\}]
\PYG{k+kn}{import} \PYG{n+nn}{scipy}\PYG{n+nn}{.}\PYG{n+nn}{integrate} \PYG{k}{as} \PYG{n+nn}{sp}
\end{sphinxVerbatim}


\subsection{Exemplo}
\label{\detokenize{lista-5-solucoes:id7}}
\sphinxAtStartPar
Mesma função \(f(x)\), agora calculada com o Scipy.

\begin{sphinxVerbatim}[commandchars=\\\{\}]
\PYG{c+c1}{\PYGZsh{} scipy::quad (erro absoluto)}
\PYG{n}{integral\PYGZus{}} \PYG{o}{=} \PYG{n}{sp}\PYG{o}{.}\PYG{n}{quad}\PYG{p}{(}\PYG{n}{f}\PYG{p}{,}\PYG{n}{a}\PYG{p}{,}\PYG{n}{b}\PYG{p}{)}
\PYG{n}{integral\PYGZus{}}
\end{sphinxVerbatim}

\begin{sphinxVerbatim}[commandchars=\\\{\}]
(4.657370976179917, 1.5031433917569477e\PYGZhy{}13)
\end{sphinxVerbatim}


\subsection{solucao\sphinxhyphen{}L5\sphinxhyphen{}Q9}
\label{\detokenize{lista-5-solucoes:solucao-l5-q9}}
\begin{sphinxVerbatim}[commandchars=\\\{\}]
\PYG{c+c1}{\PYGZsh{} dados}
\PYG{n}{x} \PYG{o}{=} \PYG{n}{np}\PYG{o}{.}\PYG{n}{array}\PYG{p}{(}\PYG{p}{[}\PYG{l+m+mf}{1.00}\PYG{p}{,} \PYG{l+m+mf}{1.05}\PYG{p}{,} \PYG{l+m+mf}{1.10}\PYG{p}{,} \PYG{l+m+mf}{1.15}\PYG{p}{,} \PYG{l+m+mf}{1.20}\PYG{p}{,} \PYG{l+m+mf}{1.25}\PYG{p}{,} \PYG{l+m+mf}{1.30}\PYG{p}{]}\PYG{p}{)}
\PYG{n}{y} \PYG{o}{=} \PYG{n}{np}\PYG{o}{.}\PYG{n}{array}\PYG{p}{(}\PYG{p}{[}\PYG{l+m+mf}{1.0000}\PYG{p}{,} \PYG{l+m+mf}{1.0247}\PYG{p}{,} \PYG{l+m+mf}{1.0488}\PYG{p}{,} \PYG{l+m+mf}{1.0723}\PYG{p}{,} \PYG{l+m+mf}{1.0954}\PYG{p}{,} \PYG{l+m+mf}{1.1180}\PYG{p}{,} \PYG{l+m+mf}{1.1401}\PYG{p}{]}\PYG{p}{)}

\PYG{c+c1}{\PYGZsh{} integral implementada: trapézio}
\PYG{n+nb}{print}\PYG{p}{(}\PYG{n}{integral\PYGZus{}newton\PYGZus{}cotes}\PYG{p}{(}\PYG{n}{x}\PYG{p}{,}\PYG{n}{y}\PYG{p}{,}\PYG{l+s+s1}{\PYGZsq{}}\PYG{l+s+s1}{trapezio}\PYG{l+s+s1}{\PYGZsq{}}\PYG{p}{)}\PYG{p}{)}

\PYG{c+c1}{\PYGZsh{} integral implementada: 1/3 Simpson}
\PYG{n+nb}{print}\PYG{p}{(}\PYG{n}{integral\PYGZus{}newton\PYGZus{}cotes}\PYG{p}{(}\PYG{n}{x}\PYG{p}{,}\PYG{n}{y}\PYG{p}{,}\PYG{l+s+s1}{\PYGZsq{}}\PYG{l+s+s1}{simpson13}\PYG{l+s+s1}{\PYGZsq{}}\PYG{p}{)}\PYG{p}{)}

\PYG{c+c1}{\PYGZsh{} integral scipy: 1/3 Simpson}

\PYG{n}{f} \PYG{o}{=} \PYG{k}{lambda} \PYG{n}{x}\PYG{p}{:} \PYG{n}{np}\PYG{o}{.}\PYG{n}{sqrt}\PYG{p}{(}\PYG{n}{x}\PYG{p}{)} 
\PYG{n}{sp}\PYG{o}{.}\PYG{n}{quad}\PYG{p}{(}\PYG{n}{f}\PYG{p}{,}\PYG{n}{x}\PYG{p}{[}\PYG{l+m+mi}{0}\PYG{p}{]}\PYG{p}{,}\PYG{n}{x}\PYG{p}{[}\PYG{o}{\PYGZhy{}}\PYG{l+m+mi}{1}\PYG{p}{]}\PYG{p}{)}
\end{sphinxVerbatim}

\begin{sphinxVerbatim}[commandchars=\\\{\}]
0.32146250000000026
0.2857383333333336
\end{sphinxVerbatim}

\begin{sphinxVerbatim}[commandchars=\\\{\}]
(0.32148536841925296, 3.569204580991558e\PYGZhy{}15)
\end{sphinxVerbatim}


\subsection{solucao\sphinxhyphen{}L5\sphinxhyphen{}Q10}
\label{\detokenize{lista-5-solucoes:solucao-l5-q10}}
\begin{sphinxVerbatim}[commandchars=\\\{\}]
\PYG{c+c1}{\PYGZsh{} caso h = 0.1}
\PYG{n}{x} \PYG{o}{=} \PYG{n}{np}\PYG{o}{.}\PYG{n}{array}\PYG{p}{(}\PYG{p}{[}\PYG{l+m+mi}{0}\PYG{p}{,} \PYG{l+m+mf}{0.1}\PYG{p}{,} \PYG{l+m+mf}{0.2}\PYG{p}{,} \PYG{l+m+mf}{0.3}\PYG{p}{,} \PYG{l+m+mf}{0.4}\PYG{p}{,} \PYG{l+m+mf}{0.5}\PYG{p}{,} \PYG{l+m+mf}{0.6}\PYG{p}{,} \PYG{l+m+mf}{0.7}\PYG{p}{,} \PYG{l+m+mf}{0.8}\PYG{p}{,} \PYG{l+m+mf}{0.9}\PYG{p}{,} \PYG{l+m+mf}{1.0}\PYG{p}{]}\PYG{p}{)}
\PYG{n}{y} \PYG{o}{=} \PYG{n}{np}\PYG{o}{.}\PYG{n}{array}\PYG{p}{(}\PYG{p}{[}\PYG{l+m+mi}{1}\PYG{p}{,} \PYG{l+m+mf}{0.995}\PYG{p}{,} \PYG{l+m+mf}{0.980}\PYG{p}{,} \PYG{l+m+mf}{0.955}\PYG{p}{,} \PYG{l+m+mf}{0.921}\PYG{p}{,} \PYG{l+m+mf}{0.877}\PYG{p}{,} \PYG{l+m+mf}{0.825}\PYG{p}{,} \PYG{l+m+mf}{0.764}\PYG{p}{,} \PYG{l+m+mf}{0.696}\PYG{p}{,} \PYG{l+m+mf}{0.6216}\PYG{p}{,} \PYG{l+m+mf}{0.5403}\PYG{p}{]}\PYG{p}{)}

\PYG{n+nb}{print}\PYG{p}{(}\PYG{n}{integral\PYGZus{}newton\PYGZus{}cotes}\PYG{p}{(}\PYG{n}{x}\PYG{p}{,}\PYG{n}{y}\PYG{p}{,}\PYG{l+s+s1}{\PYGZsq{}}\PYG{l+s+s1}{trapezio}\PYG{l+s+s1}{\PYGZsq{}}\PYG{p}{)}\PYG{p}{)}

\PYG{c+c1}{\PYGZsh{} integral implementada: 1/3 Simpson}
\PYG{n+nb}{print}\PYG{p}{(}\PYG{n}{integral\PYGZus{}newton\PYGZus{}cotes}\PYG{p}{(}\PYG{n}{x}\PYG{p}{,}\PYG{n}{y}\PYG{p}{,}\PYG{l+s+s1}{\PYGZsq{}}\PYG{l+s+s1}{simpson13}\PYG{l+s+s1}{\PYGZsq{}}\PYG{p}{)}\PYG{p}{)}

\PYG{c+c1}{\PYGZsh{} residente}
\PYG{n}{f} \PYG{o}{=} \PYG{k}{lambda} \PYG{n}{x}\PYG{p}{:} \PYG{n}{np}\PYG{o}{.}\PYG{n}{cos}\PYG{p}{(}\PYG{n}{x}\PYG{p}{)} 
\PYG{n+nb}{print}\PYG{p}{(}\PYG{n}{sp}\PYG{o}{.}\PYG{n}{quad}\PYG{p}{(}\PYG{n}{f}\PYG{p}{,}\PYG{n}{x}\PYG{p}{[}\PYG{l+m+mi}{0}\PYG{p}{]}\PYG{p}{,}\PYG{n}{x}\PYG{p}{[}\PYG{o}{\PYGZhy{}}\PYG{l+m+mi}{1}\PYG{p}{]}\PYG{p}{)}\PYG{p}{)}


\PYG{c+c1}{\PYGZsh{} caso h = 0.2}
\PYG{n}{x2} \PYG{o}{=} \PYG{n}{np}\PYG{o}{.}\PYG{n}{concatenate}\PYG{p}{(}\PYG{p}{(}\PYG{n}{x}\PYG{p}{[}\PYG{l+m+mi}{0}\PYG{p}{:}\PYG{o}{\PYGZhy{}}\PYG{l+m+mi}{1}\PYG{p}{:}\PYG{l+m+mi}{2}\PYG{p}{]}\PYG{p}{,}\PYG{p}{[}\PYG{l+m+mi}{1}\PYG{p}{]}\PYG{p}{)}\PYG{p}{)}
\PYG{n}{y2} \PYG{o}{=} \PYG{n}{np}\PYG{o}{.}\PYG{n}{concatenate}\PYG{p}{(}\PYG{p}{(}\PYG{n}{y}\PYG{p}{[}\PYG{l+m+mi}{0}\PYG{p}{:}\PYG{o}{\PYGZhy{}}\PYG{l+m+mi}{1}\PYG{p}{:}\PYG{l+m+mi}{2}\PYG{p}{]}\PYG{p}{,}\PYG{p}{[}\PYG{l+m+mf}{0.5403}\PYG{p}{]}\PYG{p}{)}\PYG{p}{)}

\PYG{n+nb}{print}\PYG{p}{(}\PYG{n}{integral\PYGZus{}newton\PYGZus{}cotes}\PYG{p}{(}\PYG{n}{x2}\PYG{p}{,}\PYG{n}{y2}\PYG{p}{,}\PYG{l+s+s1}{\PYGZsq{}}\PYG{l+s+s1}{trapezio}\PYG{l+s+s1}{\PYGZsq{}}\PYG{p}{)}\PYG{p}{)} \PYG{c+c1}{\PYGZsh{} TODO \PYGZlt{}=== CHECAR ISTO AQUI!}

\PYG{c+c1}{\PYGZsh{} integral implementada: 1/3 Simpson}
\PYG{n+nb}{print}\PYG{p}{(}\PYG{n}{integral\PYGZus{}newton\PYGZus{}cotes}\PYG{p}{(}\PYG{n}{x2}\PYG{p}{[}\PYG{l+m+mi}{0}\PYG{p}{:}\PYG{o}{\PYGZhy{}}\PYG{l+m+mi}{1}\PYG{p}{]}\PYG{p}{,}\PYG{n}{y2}\PYG{p}{[}\PYG{l+m+mi}{0}\PYG{p}{:}\PYG{o}{\PYGZhy{}}\PYG{l+m+mi}{1}\PYG{p}{]}\PYG{p}{,}\PYG{l+s+s1}{\PYGZsq{}}\PYG{l+s+s1}{simpson13}\PYG{l+s+s1}{\PYGZsq{}}\PYG{p}{)}\PYG{p}{)}

\PYG{c+c1}{\PYGZsh{} residente }
\PYG{n+nb}{print}\PYG{p}{(}\PYG{n}{sp}\PYG{o}{.}\PYG{n}{quad}\PYG{p}{(}\PYG{n}{f}\PYG{p}{,}\PYG{n}{x2}\PYG{p}{[}\PYG{l+m+mi}{0}\PYG{p}{]}\PYG{p}{,}\PYG{n}{x2}\PYG{p}{[}\PYG{o}{\PYGZhy{}}\PYG{l+m+mi}{1}\PYG{p}{]}\PYG{p}{)}\PYG{p}{)}
\end{sphinxVerbatim}

\begin{sphinxVerbatim}[commandchars=\\\{\}]
0.8404750000000001
0.7270900000000001
(0.8414709848078965, 9.34220461887732e\PYGZhy{}15)
0.83843
0.6557999999999999
(0.8414709848078965, 9.34220461887732e\PYGZhy{}15)
\end{sphinxVerbatim}


\subsection{solucao\sphinxhyphen{}L5\sphinxhyphen{}Q11}
\label{\detokenize{lista-5-solucoes:solucao-l5-q11}}
\begin{sphinxVerbatim}[commandchars=\\\{\}]
\PYG{n}{x} \PYG{o}{=} \PYG{n}{np}\PYG{o}{.}\PYG{n}{array}\PYG{p}{(}\PYG{p}{[}\PYG{l+m+mf}{1.2}\PYG{p}{,} \PYG{l+m+mf}{1.3}\PYG{p}{,} \PYG{l+m+mf}{1.4}\PYG{p}{,} \PYG{l+m+mf}{1.5}\PYG{p}{,} \PYG{l+m+mf}{1.6}\PYG{p}{]}\PYG{p}{)}
\PYG{n}{y} \PYG{o}{=} \PYG{n}{np}\PYG{o}{.}\PYG{n}{array}\PYG{p}{(}\PYG{p}{[}\PYG{l+m+mf}{0.93204}\PYG{p}{,} \PYG{l+m+mf}{0.96356}\PYG{p}{,} \PYG{l+m+mf}{0.98545}\PYG{p}{,} \PYG{l+m+mf}{0.99749}\PYG{p}{,} \PYG{l+m+mf}{0.99957}\PYG{p}{]}\PYG{p}{)}

\PYG{n+nb}{print}\PYG{p}{(}\PYG{n}{integral\PYGZus{}newton\PYGZus{}cotes}\PYG{p}{(}\PYG{n}{x}\PYG{p}{,}\PYG{n}{y}\PYG{p}{,}\PYG{l+s+s1}{\PYGZsq{}}\PYG{l+s+s1}{trapezio}\PYG{l+s+s1}{\PYGZsq{}}\PYG{p}{)}\PYG{p}{)}

\PYG{c+c1}{\PYGZsh{} integral implementada: 1/3 Simpson}
\PYG{n+nb}{print}\PYG{p}{(}\PYG{n}{integral\PYGZus{}newton\PYGZus{}cotes}\PYG{p}{(}\PYG{n}{x}\PYG{p}{,}\PYG{n}{y}\PYG{p}{,}\PYG{l+s+s1}{\PYGZsq{}}\PYG{l+s+s1}{simpson13}\PYG{l+s+s1}{\PYGZsq{}}\PYG{p}{)}\PYG{p}{)}

\PYG{c+c1}{\PYGZsh{} residente}
\PYG{n}{f} \PYG{o}{=} \PYG{k}{lambda} \PYG{n}{x}\PYG{p}{:} \PYG{n}{np}\PYG{o}{.}\PYG{n}{sin}\PYG{p}{(}\PYG{n}{x}\PYG{p}{)} 

\PYG{n+nb}{print}\PYG{p}{(}\PYG{n}{sp}\PYG{o}{.}\PYG{n}{quad}\PYG{p}{(}\PYG{n}{f}\PYG{p}{,}\PYG{n}{x}\PYG{p}{[}\PYG{l+m+mi}{0}\PYG{p}{]}\PYG{p}{,}\PYG{n}{x}\PYG{p}{[}\PYG{o}{\PYGZhy{}}\PYG{l+m+mi}{1}\PYG{p}{]}\PYG{p}{)}\PYG{p}{)}
\end{sphinxVerbatim}

\begin{sphinxVerbatim}[commandchars=\\\{\}]
0.39123050000000037
0.35870866666666695
(0.3915572767779624, 4.34715904138419e\PYGZhy{}15)
\end{sphinxVerbatim}


\subsection{solucao L5\sphinxhyphen{}Q17}
\label{\detokenize{lista-5-solucoes:solucao-l5-q17}}
\begin{sphinxVerbatim}[commandchars=\\\{\}]
\PYG{n}{z} \PYG{o}{=} \PYG{n}{np}\PYG{o}{.}\PYG{n}{linspace}\PYG{p}{(}\PYG{l+m+mi}{0}\PYG{p}{,}\PYG{l+m+mi}{10}\PYG{p}{,}\PYG{n}{num}\PYG{o}{=}\PYG{l+m+mi}{50}\PYG{p}{,}\PYG{n}{endpoint}\PYG{o}{=}\PYG{k+kc}{True}\PYG{p}{)}
\PYG{n}{f} \PYG{o}{=} \PYG{k}{lambda} \PYG{n}{z}\PYG{p}{:} \PYG{n}{z}\PYG{o}{/}\PYG{p}{(}\PYG{l+m+mi}{4}\PYG{o}{+}\PYG{n}{z}\PYG{p}{)}\PYG{o}{*}\PYG{n}{np}\PYG{o}{.}\PYG{n}{exp}\PYG{p}{(}\PYG{o}{\PYGZhy{}}\PYG{l+m+mf}{0.5}\PYG{o}{*}\PYG{n}{z}\PYG{p}{)}

\PYG{n}{d} \PYG{o}{=} \PYG{n}{integral\PYGZus{}newton\PYGZus{}cotes}\PYG{p}{(}\PYG{n}{z}\PYG{p}{,}\PYG{n}{z}\PYG{o}{*}\PYG{n}{f}\PYG{p}{(}\PYG{n}{z}\PYG{p}{)}\PYG{p}{,}\PYG{l+s+s1}{\PYGZsq{}}\PYG{l+s+s1}{trapezio}\PYG{l+s+s1}{\PYGZsq{}}\PYG{p}{)}\PYG{o}{/}\PYG{n}{integral\PYGZus{}newton\PYGZus{}cotes}\PYG{p}{(}\PYG{n}{z}\PYG{p}{,}\PYG{n}{f}\PYG{p}{(}\PYG{n}{z}\PYG{p}{)}\PYG{p}{,}\PYG{l+s+s1}{\PYGZsq{}}\PYG{l+s+s1}{trapezio}\PYG{l+s+s1}{\PYGZsq{}}\PYG{p}{)}
\PYG{n+nb}{print}\PYG{p}{(}\PYG{n}{d}\PYG{p}{)}

\PYG{c+c1}{\PYGZsh{} residente}
\PYG{n}{f2} \PYG{o}{=} \PYG{k}{lambda} \PYG{n}{z}\PYG{p}{:} \PYG{p}{(} \PYG{n}{z}\PYG{o}{/}\PYG{p}{(}\PYG{l+m+mi}{4}\PYG{o}{+}\PYG{n}{z}\PYG{p}{)}\PYG{o}{*}\PYG{n}{np}\PYG{o}{.}\PYG{n}{exp}\PYG{p}{(}\PYG{o}{\PYGZhy{}}\PYG{l+m+mf}{0.5}\PYG{o}{*}\PYG{n}{z}\PYG{p}{)} \PYG{p}{)}\PYG{o}{*}\PYG{n}{z}
\PYG{n+nb}{print}\PYG{p}{(}\PYG{n}{sp}\PYG{o}{.}\PYG{n}{quad}\PYG{p}{(}\PYG{n}{f2}\PYG{p}{,}\PYG{l+m+mi}{0}\PYG{p}{,}\PYG{l+m+mi}{10}\PYG{p}{)}\PYG{p}{[}\PYG{l+m+mi}{0}\PYG{p}{]}\PYG{o}{/}\PYG{n}{sp}\PYG{o}{.}\PYG{n}{quad}\PYG{p}{(}\PYG{n}{f}\PYG{p}{,}\PYG{l+m+mi}{0}\PYG{p}{,}\PYG{l+m+mi}{10}\PYG{p}{)}\PYG{p}{[}\PYG{l+m+mi}{0}\PYG{p}{]}\PYG{p}{)}
\end{sphinxVerbatim}

\begin{sphinxVerbatim}[commandchars=\\\{\}]
3.0523917901871935
3.0476097824918496
\end{sphinxVerbatim}

\begin{sphinxVerbatim}[commandchars=\\\{\}]
\PYG{c+c1}{\PYGZsh{} plotagem da força}
\PYG{n}{plt}\PYG{o}{.}\PYG{n}{plot}\PYG{p}{(}\PYG{n}{f}\PYG{p}{(}\PYG{n}{z}\PYG{p}{)}\PYG{p}{,}\PYG{n}{z}\PYG{p}{,}\PYG{n}{label}\PYG{o}{=}\PYG{l+s+s1}{\PYGZsq{}}\PYG{l+s+s1}{\PYGZdl{}f(z)\PYGZdl{}}\PYG{l+s+s1}{\PYGZsq{}}\PYG{p}{)}
\PYG{n}{plt}\PYG{o}{.}\PYG{n}{plot}\PYG{p}{(}\PYG{n}{f}\PYG{p}{(}\PYG{n}{z}\PYG{p}{)}\PYG{p}{,}\PYG{n}{np}\PYG{o}{.}\PYG{n}{ones}\PYG{p}{(}\PYG{n}{np}\PYG{o}{.}\PYG{n}{shape}\PYG{p}{(}\PYG{n}{f}\PYG{p}{(}\PYG{n}{z}\PYG{p}{)}\PYG{p}{)}\PYG{p}{)}\PYG{o}{*}\PYG{n}{d}\PYG{p}{,}\PYG{l+s+s1}{\PYGZsq{}}\PYG{l+s+s1}{\PYGZhy{}\PYGZhy{}}\PYG{l+s+s1}{\PYGZsq{}}\PYG{p}{,}\PYG{n}{label}\PYG{o}{=}\PYG{l+s+s1}{\PYGZsq{}}\PYG{l+s+s1}{\PYGZdl{}d\PYGZdl{}}\PYG{l+s+s1}{\PYGZsq{}}\PYG{p}{)}
\PYG{n}{plt}\PYG{o}{.}\PYG{n}{legend}\PYG{p}{(}\PYG{p}{)}\PYG{p}{;}
\end{sphinxVerbatim}

\noindent\sphinxincludegraphics{{lista-5-solucoes_66_0}.png}


\subsection{QUADRATURA GAUSSIANA}
\label{\detokenize{lista-5-solucoes:quadratura-gaussiana}}
\sphinxAtStartPar
Algumas informações:
\begin{itemize}
\item {} 
\sphinxAtStartPar
A quadratura gaussiana pode ser chamada como

\end{itemize}

\begin{sphinxVerbatim}[commandchars=\\\{\}]
\PYG{k+kn}{from} \PYG{n+nn}{scipy}\PYG{n+nn}{.}\PYG{n+nn}{integrate} \PYG{k+kn}{import} \PYG{n}{quadrature}
\end{sphinxVerbatim}

\sphinxAtStartPar
Depois de importar, veja

\begin{sphinxVerbatim}[commandchars=\\\{\}]
\PYG{n}{help}\PYG{p}{(}\PYG{n}{quadrature}\PYG{p}{)}
\end{sphinxVerbatim}
\begin{itemize}
\item {} 
\sphinxAtStartPar
A tabela de pesos de quadratura pode ser acessada no Numpy através do comando

\end{itemize}

\begin{sphinxVerbatim}[commandchars=\\\{\}]
\PYG{n}{np}\PYG{o}{.}\PYG{n}{polynomial}\PYG{o}{.}\PYG{n}{legendre}\PYG{o}{.}\PYG{n}{leggauss}\PYG{p}{(}\PYG{n+nb}{ord}\PYG{p}{)}
\end{sphinxVerbatim}

\sphinxAtStartPar
Integra exatamente polinômios de grau até

\begin{sphinxVerbatim}[commandchars=\\\{\}]
\PYG{l+m+mi}{2}\PYG{o}{*}\PYG{n+nb}{ord} \PYG{o}{\PYGZhy{}} \PYG{l+m+mi}{1}
\end{sphinxVerbatim}


\subsection{solucao\sphinxhyphen{}L5\sphinxhyphen{}Q18a}
\label{\detokenize{lista-5-solucoes:solucao-l5-q18a}}
\begin{sphinxVerbatim}[commandchars=\\\{\}]
\PYG{k+kn}{from} \PYG{n+nn}{scipy}\PYG{n+nn}{.}\PYG{n+nn}{integrate} \PYG{k+kn}{import} \PYG{n}{quadrature}

\PYG{c+c1}{\PYGZsh{} integração}
\PYG{n}{f} \PYG{o}{=} \PYG{k}{lambda} \PYG{n}{z}\PYG{p}{:} \PYG{n}{z}\PYG{o}{*}\PYG{o}{*}\PYG{l+m+mi}{3} \PYG{o}{+} \PYG{n}{z}\PYG{o}{*}\PYG{o}{*}\PYG{l+m+mi}{2} \PYG{o}{+} \PYG{n}{z} \PYG{o}{+} \PYG{l+m+mi}{1}
\PYG{n}{quadrature}\PYG{p}{(}\PYG{n}{f}\PYG{p}{,} \PYG{o}{\PYGZhy{}}\PYG{l+m+mi}{1}\PYG{p}{,} \PYG{l+m+mi}{1}\PYG{p}{,}\PYG{n}{maxiter}\PYG{o}{=}\PYG{l+m+mi}{3}\PYG{p}{)} \PYG{c+c1}{\PYGZsh{} grau máximo = 3; erro = 0}
\end{sphinxVerbatim}

\begin{sphinxVerbatim}[commandchars=\\\{\}]
(2.6666666666666665, 0.0)
\end{sphinxVerbatim}

\begin{sphinxVerbatim}[commandchars=\\\{\}]
\PYG{c+c1}{\PYGZsh{} Verificando computação numérica x simbólica com Sympy}
\PYG{k+kn}{import} \PYG{n+nn}{sympy} \PYG{k}{as} \PYG{n+nn}{sy}

\PYG{c+c1}{\PYGZsh{} cria símbolo para z}
\PYG{n}{zsym} \PYG{o}{=} \PYG{n}{sy}\PYG{o}{.}\PYG{n}{Symbol}\PYG{p}{(}\PYG{l+s+s1}{\PYGZsq{}}\PYG{l+s+s1}{z}\PYG{l+s+s1}{\PYGZsq{}}\PYG{p}{)} 

\PYG{c+c1}{\PYGZsh{} integração}
\PYG{n}{f} \PYG{o}{=} \PYG{n}{zsym}\PYG{o}{*}\PYG{o}{*}\PYG{l+m+mi}{3} \PYG{o}{+} \PYG{n}{zsym}\PYG{o}{*}\PYG{o}{*}\PYG{l+m+mi}{2} \PYG{o}{+} \PYG{n}{zsym} \PYG{o}{+} \PYG{l+m+mi}{1}
\PYG{n}{val} \PYG{o}{=} \PYG{n}{sy}\PYG{o}{.}\PYG{n}{integrate}\PYG{p}{(}\PYG{n}{f}\PYG{p}{,} \PYG{p}{(}\PYG{n}{zsym}\PYG{p}{,}\PYG{o}{\PYGZhy{}}\PYG{l+m+mi}{1}\PYG{p}{,}\PYG{l+m+mi}{1}\PYG{p}{)}\PYG{p}{)}
\PYG{n+nb}{float}\PYG{p}{(}\PYG{n}{val}\PYG{p}{)}
\end{sphinxVerbatim}

\begin{sphinxVerbatim}[commandchars=\\\{\}]
2.6666666666666665
\end{sphinxVerbatim}


\subsection{solucao\sphinxhyphen{}L5\sphinxhyphen{}Q18b}
\label{\detokenize{lista-5-solucoes:solucao-l5-q18b}}
\begin{sphinxVerbatim}[commandchars=\\\{\}]
\PYG{c+c1}{\PYGZsh{} integração}
\PYG{n}{f} \PYG{o}{=} \PYG{k}{lambda} \PYG{n}{x}\PYG{p}{:} \PYG{n}{x}\PYG{o}{*}\PYG{o}{*}\PYG{l+m+mi}{2} \PYG{o}{\PYGZhy{}}\PYG{l+m+mi}{1}
\PYG{n}{quadrature}\PYG{p}{(}\PYG{n}{f}\PYG{p}{,} \PYG{o}{\PYGZhy{}}\PYG{l+m+mi}{2}\PYG{p}{,} \PYG{l+m+mi}{0}\PYG{p}{,}\PYG{n}{maxiter}\PYG{o}{=}\PYG{l+m+mi}{3}\PYG{p}{)} 
\end{sphinxVerbatim}

\begin{sphinxVerbatim}[commandchars=\\\{\}]
(0.666666666666667, 2.220446049250313e\PYGZhy{}16)
\end{sphinxVerbatim}

\begin{sphinxVerbatim}[commandchars=\\\{\}]
\PYG{c+c1}{\PYGZsh{} integração simbólica}
\PYG{n}{f} \PYG{o}{=} \PYG{n}{zsym}\PYG{o}{*}\PYG{o}{*}\PYG{l+m+mi}{2} \PYG{o}{+} \PYG{o}{\PYGZhy{}}\PYG{l+m+mi}{1}
\PYG{n}{val} \PYG{o}{=} \PYG{n}{sy}\PYG{o}{.}\PYG{n}{integrate}\PYG{p}{(}\PYG{n}{f}\PYG{p}{,} \PYG{p}{(}\PYG{n}{zsym}\PYG{p}{,}\PYG{o}{\PYGZhy{}}\PYG{l+m+mi}{2}\PYG{p}{,}\PYG{l+m+mi}{0}\PYG{p}{)}\PYG{p}{)}
\PYG{n+nb}{float}\PYG{p}{(}\PYG{n}{val}\PYG{p}{)}
\end{sphinxVerbatim}

\begin{sphinxVerbatim}[commandchars=\\\{\}]
0.6666666666666666
\end{sphinxVerbatim}


\subsubsection{Exemplos de integrandos com singularidades}
\label{\detokenize{lista-5-solucoes:exemplos-de-integrandos-com-singularidades}}
\sphinxAtStartPar
Esses casos não são bem manipulados pelo submódulo \sphinxcode{\sphinxupquote{integrate}}.


\subsection{solucao\sphinxhyphen{}L5\sphinxhyphen{}Q18c}
\label{\detokenize{lista-5-solucoes:solucao-l5-q18c}}
\begin{sphinxVerbatim}[commandchars=\\\{\}]
\PYG{c+c1}{\PYGZsh{} função com singularidade}

\PYG{c+c1}{\PYGZsh{} integração}
\PYG{n}{f} \PYG{o}{=} \PYG{k}{lambda} \PYG{n}{x}\PYG{p}{:} \PYG{p}{(}\PYG{p}{(}\PYG{l+m+mi}{1}\PYG{o}{\PYGZhy{}}\PYG{n}{x}\PYG{o}{*}\PYG{o}{*}\PYG{l+m+mi}{2}\PYG{p}{)}\PYG{o}{*}\PYG{o}{*}\PYG{p}{(}\PYG{o}{\PYGZhy{}}\PYG{l+m+mi}{1}\PYG{o}{/}\PYG{l+m+mi}{2}\PYG{p}{)}\PYG{p}{)}\PYG{o}{*}\PYG{n}{x}\PYG{o}{*}\PYG{o}{*}\PYG{l+m+mi}{2}\PYG{o}{*}\PYG{n}{x}\PYG{o}{*}\PYG{o}{*}\PYG{l+m+mi}{2}
\PYG{n}{quadrature}\PYG{p}{(}\PYG{n}{f}\PYG{p}{,} \PYG{o}{\PYGZhy{}}\PYG{l+m+mi}{1}\PYG{p}{,} \PYG{l+m+mi}{1}\PYG{p}{)} 
\end{sphinxVerbatim}

\begin{sphinxVerbatim}[commandchars=\\\{\}]
/Users/gustavo/anaconda3/lib/python3.7/site\PYGZhy{}packages/scipy/integrate/quadrature.py:251: AccuracyWarning: maxiter (50) exceeded. Latest difference = 6.973792e\PYGZhy{}04
  AccuracyWarning)
\end{sphinxVerbatim}

\begin{sphinxVerbatim}[commandchars=\\\{\}]
(1.1436022273829545, 0.0006973792417572788)
\end{sphinxVerbatim}


\subsection{solucao\sphinxhyphen{}L5\sphinxhyphen{}Q18d}
\label{\detokenize{lista-5-solucoes:solucao-l5-q18d}}
\begin{sphinxVerbatim}[commandchars=\\\{\}]
\PYG{c+c1}{\PYGZsh{} função com singularidade }

\PYG{c+c1}{\PYGZsh{} integração}
\PYG{n}{f} \PYG{o}{=} \PYG{k}{lambda} \PYG{n}{x}\PYG{p}{:} \PYG{p}{(}\PYG{n}{x}\PYG{o}{*}\PYG{o}{*}\PYG{l+m+mi}{3} \PYG{o}{+}\PYG{l+m+mi}{2}\PYG{o}{*}\PYG{n}{x}\PYG{o}{*}\PYG{o}{*}\PYG{l+m+mi}{2}\PYG{p}{)}\PYG{o}{/}\PYG{p}{(}\PYG{l+m+mi}{4}\PYG{o}{*}\PYG{p}{(}\PYG{l+m+mi}{4}\PYG{o}{\PYGZhy{}}\PYG{n}{x}\PYG{o}{*}\PYG{o}{*}\PYG{l+m+mi}{2}\PYG{p}{)}\PYG{o}{*}\PYG{o}{*}\PYG{l+m+mi}{1}\PYG{o}{/}\PYG{l+m+mi}{2}\PYG{p}{)}
\PYG{n}{quadrature}\PYG{p}{(}\PYG{n}{f}\PYG{p}{,} \PYG{o}{\PYGZhy{}}\PYG{l+m+mi}{2}\PYG{p}{,} \PYG{l+m+mi}{2}\PYG{p}{,}\PYG{n}{maxiter}\PYG{o}{=}\PYG{l+m+mi}{3}\PYG{p}{)} 
\end{sphinxVerbatim}

\begin{sphinxVerbatim}[commandchars=\\\{\}]
/Users/gustavo/anaconda3/lib/python3.7/site\PYGZhy{}packages/scipy/integrate/quadrature.py:251: AccuracyWarning: maxiter (3) exceeded. Latest difference = 1.333333e+00
  AccuracyWarning)
\end{sphinxVerbatim}

\begin{sphinxVerbatim}[commandchars=\\\{\}]
(3.3333333333333357, 1.333333333333336)
\end{sphinxVerbatim}


\chapter{Lista de Exercícios 6}
\label{\detokenize{lista-6-solucoes:lista-de-exercicios-6}}\label{\detokenize{lista-6-solucoes::doc}}
\sphinxAtStartPar
Solucionário matemático e computacional de exercícios selecionados da Lista de Exercícios 6.

\begin{sphinxVerbatim}[commandchars=\\\{\}]
\PYG{o}{\PYGZpc{}}\PYG{k}{matplotlib} inline 
\end{sphinxVerbatim}

\begin{sphinxVerbatim}[commandchars=\\\{\}]
\PYG{k+kn}{import} \PYG{n+nn}{numpy} \PYG{k}{as} \PYG{n+nn}{np} 
\PYG{k+kn}{import} \PYG{n+nn}{matplotlib}\PYG{n+nn}{.}\PYG{n+nn}{pyplot} \PYG{k}{as} \PYG{n+nn}{plt}
\PYG{k+kn}{from} \PYG{n+nn}{scipy}\PYG{n+nn}{.}\PYG{n+nn}{integrate} \PYG{k+kn}{import} \PYG{n}{odeint}
\end{sphinxVerbatim}


\section{Funções\sphinxhyphen{}base para resolucão}
\label{\detokenize{lista-6-solucoes:funcoes-base-para-resolucao}}
\begin{sphinxVerbatim}[commandchars=\\\{\}]
\PYG{l+s+sd}{\PYGZdq{}\PYGZdq{}\PYGZdq{}}
\PYG{l+s+sd}{Implementação do método de Euler}
\PYG{l+s+sd}{\PYGZsq{}expr\PYGZsq{} deve conter variáveis \PYGZsq{}t\PYGZsq{} e \PYGZsq{}y\PYGZsq{}. }
\PYG{l+s+sd}{Para casos em que são 0, fazer \PYGZsq{}0*t\PYGZsq{} }
\PYG{l+s+sd}{ou \PYGZsq{}0*y\PYGZsq{}.}
\PYG{l+s+sd}{\PYGZdq{}\PYGZdq{}\PYGZdq{}}
\PYG{k}{def} \PYG{n+nf}{met\PYGZus{}euler}\PYG{p}{(}\PYG{n}{expr}\PYG{p}{,}\PYG{n}{n}\PYG{p}{,}\PYG{n}{a}\PYG{p}{,}\PYG{n}{b}\PYG{p}{,}\PYG{n}{y0}\PYG{p}{,}\PYG{n}{mov}\PYG{p}{)}\PYG{p}{:}

    \PYG{c+c1}{\PYGZsh{} expressão}
    \PYG{n}{f} \PYG{o}{=} \PYG{n+nb}{eval}\PYG{p}{(}\PYG{l+s+s1}{\PYGZsq{}}\PYG{l+s+s1}{lambda t,y:}\PYG{l+s+s1}{\PYGZsq{}} \PYG{o}{+} \PYG{n}{expr}\PYG{p}{)}

    \PYG{c+c1}{\PYGZsh{} malha}
    \PYG{n}{t} \PYG{o}{=} \PYG{n}{np}\PYG{o}{.}\PYG{n}{linspace}\PYG{p}{(}\PYG{n}{a}\PYG{p}{,}\PYG{n}{b}\PYG{p}{,}\PYG{n}{num}\PYG{o}{=}\PYG{n}{n}\PYG{p}{,}\PYG{n}{endpoint}\PYG{o}{=}\PYG{k+kc}{True}\PYG{p}{)}

    \PYG{c+c1}{\PYGZsh{} passo }
    \PYG{n}{h} \PYG{o}{=} \PYG{p}{(}\PYG{n}{b}\PYG{o}{\PYGZhy{}}\PYG{n}{a}\PYG{p}{)}\PYG{o}{/}\PYG{p}{(}\PYG{n}{n}\PYG{o}{\PYGZhy{}}\PYG{l+m+mi}{1}\PYG{p}{)}
    \PYG{n+nb}{print}\PYG{p}{(}\PYG{l+s+s1}{\PYGZsq{}}\PYG{l+s+s1}{Tamanho do passo: h = }\PYG{l+s+si}{\PYGZob{}0\PYGZcb{}}\PYG{l+s+s1}{\PYGZsq{}}\PYG{o}{.}\PYG{n}{format}\PYG{p}{(}\PYG{n}{h}\PYG{p}{)}\PYG{p}{)}
    
    \PYG{c+c1}{\PYGZsh{} y(t)}
    \PYG{n}{y} \PYG{o}{=} \PYG{l+m+mi}{0}\PYG{o}{*}\PYG{n}{t}

    \PYG{c+c1}{\PYGZsh{} resolve para frente}
    \PYG{k}{if} \PYG{n}{mov} \PYG{o+ow}{is} \PYG{l+s+s1}{\PYGZsq{}}\PYG{l+s+s1}{front}\PYG{l+s+s1}{\PYGZsq{}}\PYG{p}{:}
        \PYG{c+c1}{\PYGZsh{} cond. inicial}
        \PYG{n}{y}\PYG{p}{[}\PYG{l+m+mi}{0}\PYG{p}{]} \PYG{o}{=} \PYG{n}{y0}

        \PYG{c+c1}{\PYGZsh{} esquema}
        \PYG{k}{for} \PYG{n}{i} \PYG{o+ow}{in} \PYG{n+nb}{range}\PYG{p}{(}\PYG{l+m+mi}{0}\PYG{p}{,}\PYG{n}{n}\PYG{o}{\PYGZhy{}}\PYG{l+m+mi}{1}\PYG{p}{)}\PYG{p}{:}
            \PYG{n}{y}\PYG{p}{[}\PYG{n}{i}\PYG{o}{+}\PYG{l+m+mi}{1}\PYG{p}{]} \PYG{o}{=} \PYG{n}{y}\PYG{p}{[}\PYG{n}{i}\PYG{p}{]} \PYG{o}{+} \PYG{n}{h}\PYG{o}{*}\PYG{n}{f}\PYG{p}{(}\PYG{n}{t}\PYG{p}{[}\PYG{n}{i}\PYG{p}{]}\PYG{p}{,}\PYG{n}{y}\PYG{p}{[}\PYG{n}{i}\PYG{p}{]}\PYG{p}{)}

        \PYG{k}{return} \PYG{n}{t}\PYG{p}{,}\PYG{n}{y}
    
    \PYG{c+c1}{\PYGZsh{} resolve para trás}
    \PYG{k}{elif} \PYG{n}{mov} \PYG{o+ow}{is} \PYG{l+s+s1}{\PYGZsq{}}\PYG{l+s+s1}{back}\PYG{l+s+s1}{\PYGZsq{}}\PYG{p}{:}
        
        \PYG{c+c1}{\PYGZsh{} cond. inicial}
        \PYG{n}{y}\PYG{p}{[}\PYG{o}{\PYGZhy{}}\PYG{l+m+mi}{1}\PYG{p}{]} \PYG{o}{=} \PYG{n}{y0}

        \PYG{c+c1}{\PYGZsh{} esquema}
        \PYG{k}{for} \PYG{n}{i} \PYG{o+ow}{in} \PYG{n+nb}{range}\PYG{p}{(}\PYG{n}{n}\PYG{o}{\PYGZhy{}}\PYG{l+m+mi}{1}\PYG{p}{,}\PYG{l+m+mi}{0}\PYG{p}{,}\PYG{o}{\PYGZhy{}}\PYG{l+m+mi}{1}\PYG{p}{)}\PYG{p}{:}
            \PYG{n}{y}\PYG{p}{[}\PYG{n}{i}\PYG{o}{\PYGZhy{}}\PYG{l+m+mi}{1}\PYG{p}{]} \PYG{o}{=} \PYG{n}{y}\PYG{p}{[}\PYG{n}{i}\PYG{p}{]} \PYG{o}{\PYGZhy{}} \PYG{n}{h}\PYG{o}{*}\PYG{n}{f}\PYG{p}{(}\PYG{n}{t}\PYG{p}{[}\PYG{n}{i}\PYG{p}{]}\PYG{p}{,}\PYG{n}{y}\PYG{p}{[}\PYG{n}{i}\PYG{p}{]}\PYG{p}{)}

        \PYG{k}{return} \PYG{n}{t}\PYG{p}{,}\PYG{n}{y}
    
    
\PYG{k}{def} \PYG{n+nf}{met\PYGZus{}pt\PYGZus{}medio}\PYG{p}{(}\PYG{n}{expr}\PYG{p}{,}\PYG{n}{n}\PYG{p}{,}\PYG{n}{a}\PYG{p}{,}\PYG{n}{b}\PYG{p}{,}\PYG{n}{y0}\PYG{p}{)}\PYG{p}{:}
    
    \PYG{c+c1}{\PYGZsh{} expressão}
    \PYG{n}{f} \PYG{o}{=} \PYG{n+nb}{eval}\PYG{p}{(}\PYG{l+s+s1}{\PYGZsq{}}\PYG{l+s+s1}{lambda t,y:}\PYG{l+s+s1}{\PYGZsq{}} \PYG{o}{+} \PYG{n}{expr}\PYG{p}{)}

    \PYG{c+c1}{\PYGZsh{} malha}
    \PYG{n}{t} \PYG{o}{=} \PYG{n}{np}\PYG{o}{.}\PYG{n}{linspace}\PYG{p}{(}\PYG{n}{a}\PYG{p}{,}\PYG{n}{b}\PYG{p}{,}\PYG{n}{num}\PYG{o}{=}\PYG{n}{n}\PYG{p}{,}\PYG{n}{endpoint}\PYG{o}{=}\PYG{k+kc}{True}\PYG{p}{)}

    \PYG{c+c1}{\PYGZsh{} passo }
    \PYG{n}{h} \PYG{o}{=} \PYG{p}{(}\PYG{n}{b}\PYG{o}{\PYGZhy{}}\PYG{n}{a}\PYG{p}{)}\PYG{o}{/}\PYG{p}{(}\PYG{n}{n}\PYG{o}{\PYGZhy{}}\PYG{l+m+mi}{1}\PYG{p}{)}
    \PYG{n+nb}{print}\PYG{p}{(}\PYG{l+s+s1}{\PYGZsq{}}\PYG{l+s+s1}{Tamanho do passo: h = }\PYG{l+s+si}{\PYGZob{}0\PYGZcb{}}\PYG{l+s+s1}{\PYGZsq{}}\PYG{o}{.}\PYG{n}{format}\PYG{p}{(}\PYG{n}{h}\PYG{p}{)}\PYG{p}{)}
    
    \PYG{c+c1}{\PYGZsh{} y(t)}
    \PYG{n}{y} \PYG{o}{=} \PYG{l+m+mi}{0}\PYG{o}{*}\PYG{n}{t}
    
    \PYG{c+c1}{\PYGZsh{} cond. inicial}
    \PYG{n}{y}\PYG{p}{[}\PYG{l+m+mi}{0}\PYG{p}{]} \PYG{o}{=} \PYG{n}{y0}

    \PYG{c+c1}{\PYGZsh{} esquema}
    \PYG{k}{for} \PYG{n}{i} \PYG{o+ow}{in} \PYG{n+nb}{range}\PYG{p}{(}\PYG{l+m+mi}{0}\PYG{p}{,}\PYG{n}{n}\PYG{o}{\PYGZhy{}}\PYG{l+m+mi}{1}\PYG{p}{)}\PYG{p}{:}
        \PYG{n}{tmed} \PYG{o}{=} \PYG{n}{t}\PYG{p}{[}\PYG{n}{i}\PYG{p}{]} \PYG{o}{+} \PYG{n}{h}\PYG{o}{/}\PYG{l+m+mi}{2}
        \PYG{n}{ymed} \PYG{o}{=} \PYG{n}{y}\PYG{p}{[}\PYG{n}{i}\PYG{p}{]} \PYG{o}{+} \PYG{n}{h}\PYG{o}{/}\PYG{l+m+mi}{2}\PYG{o}{*}\PYG{n}{f}\PYG{p}{(}\PYG{n}{t}\PYG{p}{[}\PYG{n}{i}\PYG{p}{]}\PYG{p}{,}\PYG{n}{y}\PYG{p}{[}\PYG{n}{i}\PYG{p}{]}\PYG{p}{)}
        \PYG{n}{y}\PYG{p}{[}\PYG{n}{i}\PYG{o}{+}\PYG{l+m+mi}{1}\PYG{p}{]} \PYG{o}{=} \PYG{n}{y}\PYG{p}{[}\PYG{n}{i}\PYG{p}{]} \PYG{o}{+} \PYG{n}{h}\PYG{o}{*}\PYG{n}{f}\PYG{p}{(}\PYG{n}{tmed}\PYG{p}{,}\PYG{n}{ymed}\PYG{p}{)}

    \PYG{k}{return} \PYG{n}{t}\PYG{p}{,}\PYG{n}{y}
    
\end{sphinxVerbatim}


\section{Soluções da Lista 6}
\label{\detokenize{lista-6-solucoes:solucoes-da-lista-6}}

\subsection{solucao\sphinxhyphen{}L6\sphinxhyphen{}Q1a}
\label{\detokenize{lista-6-solucoes:solucao-l6-q1a}}
\begin{sphinxVerbatim}[commandchars=\\\{\}]
\PYG{c+c1}{\PYGZsh{} PVI }
\PYG{c+c1}{\PYGZsh{} h = 0.1 =\PYGZgt{} 5 pontos}
\PYG{n}{expr} \PYG{o}{=} \PYG{l+s+s1}{\PYGZsq{}}\PYG{l+s+s1}{y**2 \PYGZhy{} t/2 + 3}\PYG{l+s+s1}{\PYGZsq{}}


\PYG{c+c1}{\PYGZsh{} resolve para a frente}
\PYG{n}{y0} \PYG{o}{=} \PYG{l+m+mi}{1}
\PYG{n}{a} \PYG{o}{=} \PYG{l+m+mf}{1.2}
\PYG{n}{b} \PYG{o}{=} \PYG{l+m+mf}{1.6}
\PYG{n}{n} \PYG{o}{=} \PYG{l+m+mi}{5}
\PYG{n}{mov} \PYG{o}{=} \PYG{l+s+s1}{\PYGZsq{}}\PYG{l+s+s1}{front}\PYG{l+s+s1}{\PYGZsq{}}
\PYG{n}{x1}\PYG{p}{,}\PYG{n}{y1} \PYG{o}{=} \PYG{n}{met\PYGZus{}euler}\PYG{p}{(}\PYG{n}{expr}\PYG{p}{,}\PYG{n}{n}\PYG{p}{,}\PYG{n}{a}\PYG{p}{,}\PYG{n}{b}\PYG{p}{,}\PYG{n}{y0}\PYG{p}{,}\PYG{n}{mov}\PYG{p}{)}


\PYG{c+c1}{\PYGZsh{} resolve para trás}
\PYG{n}{y0} \PYG{o}{=} \PYG{l+m+mi}{1}
\PYG{n}{a} \PYG{o}{=} \PYG{l+m+mf}{0.8}
\PYG{n}{b} \PYG{o}{=} \PYG{l+m+mf}{1.2}
\PYG{n}{n} \PYG{o}{=} \PYG{l+m+mi}{5}
\PYG{n}{mov} \PYG{o}{=} \PYG{l+s+s1}{\PYGZsq{}}\PYG{l+s+s1}{back}\PYG{l+s+s1}{\PYGZsq{}}
\PYG{n}{x2}\PYG{p}{,}\PYG{n}{y2} \PYG{o}{=} \PYG{n}{met\PYGZus{}euler}\PYG{p}{(}\PYG{n}{expr}\PYG{p}{,}\PYG{n}{n}\PYG{p}{,}\PYG{n}{a}\PYG{p}{,}\PYG{n}{b}\PYG{p}{,}\PYG{n}{y0}\PYG{p}{,}\PYG{n}{mov}\PYG{p}{)}

\PYG{c+c1}{\PYGZsh{} aproximação y(0.8)}
\PYG{n+nb}{print}\PYG{p}{(}\PYG{l+s+s1}{\PYGZsq{}}\PYG{l+s+s1}{Aproximação y(0.8) = }\PYG{l+s+si}{\PYGZob{}0\PYGZcb{}}\PYG{l+s+s1}{\PYGZsq{}}\PYG{o}{.}\PYG{n}{format}\PYG{p}{(}\PYG{n}{y2}\PYG{p}{[}\PYG{l+m+mi}{0}\PYG{p}{]}\PYG{p}{)}\PYG{p}{)}

\PYG{c+c1}{\PYGZsh{} aproximação y(1.2)}
\PYG{n+nb}{print}\PYG{p}{(}\PYG{l+s+s1}{\PYGZsq{}}\PYG{l+s+s1}{Aproximação y(1.6) = }\PYG{l+s+si}{\PYGZob{}0\PYGZcb{}}\PYG{l+s+s1}{\PYGZsq{}}\PYG{o}{.}\PYG{n}{format}\PYG{p}{(}\PYG{n}{y1}\PYG{p}{[}\PYG{o}{\PYGZhy{}}\PYG{l+m+mi}{1}\PYG{p}{]}\PYG{p}{)}\PYG{p}{)}

\PYG{c+c1}{\PYGZsh{} plotagem }
\PYG{n}{plt}\PYG{o}{.}\PYG{n}{plot}\PYG{p}{(}\PYG{n}{x1}\PYG{p}{,}\PYG{n}{y1}\PYG{p}{,}\PYG{l+s+s1}{\PYGZsq{}}\PYG{l+s+s1}{o\PYGZhy{}}\PYG{l+s+s1}{\PYGZsq{}}\PYG{p}{,}\PYG{n}{label}\PYG{o}{=}\PYG{l+s+s1}{\PYGZsq{}}\PYG{l+s+s1}{frente}\PYG{l+s+s1}{\PYGZsq{}}\PYG{p}{)}
\PYG{n}{plt}\PYG{o}{.}\PYG{n}{plot}\PYG{p}{(}\PYG{n}{x2}\PYG{p}{,}\PYG{n}{y2}\PYG{p}{,}\PYG{l+s+s1}{\PYGZsq{}}\PYG{l+s+s1}{o\PYGZhy{}}\PYG{l+s+s1}{\PYGZsq{}}\PYG{p}{,}\PYG{n}{label}\PYG{o}{=}\PYG{l+s+s1}{\PYGZsq{}}\PYG{l+s+s1}{trás}\PYG{l+s+s1}{\PYGZsq{}}\PYG{p}{)}
\PYG{n}{plt}\PYG{o}{.}\PYG{n}{axvline}\PYG{p}{(}\PYG{n}{x}\PYG{o}{=}\PYG{l+m+mf}{1.2}\PYG{p}{,}\PYG{n}{color}\PYG{o}{=}\PYG{l+s+s1}{\PYGZsq{}}\PYG{l+s+s1}{k}\PYG{l+s+s1}{\PYGZsq{}}\PYG{p}{,}\PYG{n}{linewidth}\PYG{o}{=}\PYG{l+m+mf}{0.6}\PYG{p}{,}\PYG{n}{linestyle}\PYG{o}{=}\PYG{l+s+s1}{\PYGZsq{}}\PYG{l+s+s1}{\PYGZhy{}\PYGZhy{}}\PYG{l+s+s1}{\PYGZsq{}}\PYG{p}{)}
\PYG{n}{plt}\PYG{o}{.}\PYG{n}{axhline}\PYG{p}{(}\PYG{n}{y}\PYG{o}{=}\PYG{l+m+mf}{1.0}\PYG{p}{,}\PYG{n}{color}\PYG{o}{=}\PYG{l+s+s1}{\PYGZsq{}}\PYG{l+s+s1}{k}\PYG{l+s+s1}{\PYGZsq{}}\PYG{p}{,}\PYG{n}{linewidth}\PYG{o}{=}\PYG{l+m+mf}{0.6}\PYG{p}{,}\PYG{n}{linestyle}\PYG{o}{=}\PYG{l+s+s1}{\PYGZsq{}}\PYG{l+s+s1}{\PYGZhy{}\PYGZhy{}}\PYG{l+s+s1}{\PYGZsq{}}\PYG{p}{)}
\PYG{n}{plt}\PYG{o}{.}\PYG{n}{legend}\PYG{p}{(}\PYG{p}{)}
\end{sphinxVerbatim}

\begin{sphinxVerbatim}[commandchars=\\\{\}]
Tamanho do passo: h = 0.10000000000000003
Tamanho do passo: h = 0.09999999999999998
Aproximação y(0.8) = \PYGZhy{}0.14851547391331865
Aproximação y(1.6) = 3.042921559591501
\end{sphinxVerbatim}

\begin{sphinxVerbatim}[commandchars=\\\{\}]
\PYGZlt{}matplotlib.legend.Legend at 0x1518bfbda0\PYGZgt{}
\end{sphinxVerbatim}

\noindent\sphinxincludegraphics{{lista-6-solucoes_7_2}.png}


\subsection{solucao\sphinxhyphen{}L6\sphinxhyphen{}Q1b}
\label{\detokenize{lista-6-solucoes:solucao-l6-q1b}}
\begin{sphinxVerbatim}[commandchars=\\\{\}]
\PYG{c+c1}{\PYGZsh{} resolve por ponto médio }
\PYG{n}{y0} \PYG{o}{=} \PYG{l+m+mi}{1}
\PYG{n}{a} \PYG{o}{=} \PYG{l+m+mf}{1.2}
\PYG{n}{b} \PYG{o}{=} \PYG{l+m+mf}{1.6}
\PYG{n}{n} \PYG{o}{=} \PYG{l+m+mi}{5}
\PYG{n}{xm}\PYG{p}{,}\PYG{n}{ym} \PYG{o}{=} \PYG{n}{met\PYGZus{}pt\PYGZus{}medio}\PYG{p}{(}\PYG{n}{expr}\PYG{p}{,}\PYG{n}{n}\PYG{p}{,}\PYG{n}{a}\PYG{p}{,}\PYG{n}{b}\PYG{p}{,}\PYG{n}{y0}\PYG{p}{)}

\PYG{c+c1}{\PYGZsh{} resolve por função residente }
\PYG{n}{f} \PYG{o}{=} \PYG{n+nb}{eval}\PYG{p}{(}\PYG{l+s+s1}{\PYGZsq{}}\PYG{l+s+s1}{lambda t,y:}\PYG{l+s+s1}{\PYGZsq{}} \PYG{o}{+} \PYG{n}{expr}\PYG{p}{)}
\PYG{n}{xr} \PYG{o}{=} \PYG{n}{np}\PYG{o}{.}\PYG{n}{linspace}\PYG{p}{(}\PYG{l+m+mf}{1.2}\PYG{p}{,}\PYG{l+m+mf}{1.6}\PYG{p}{,}\PYG{n}{num}\PYG{o}{=}\PYG{l+m+mi}{5}\PYG{p}{,}\PYG{n}{endpoint}\PYG{o}{=}\PYG{k+kc}{True}\PYG{p}{)}
\PYG{n}{yr} \PYG{o}{=} \PYG{n}{odeint}\PYG{p}{(}\PYG{n}{f}\PYG{p}{,}\PYG{l+m+mi}{1}\PYG{p}{,}\PYG{n}{xr}\PYG{p}{)}
\PYG{n}{yr}
\end{sphinxVerbatim}

\begin{sphinxVerbatim}[commandchars=\\\{\}]
Tamanho do passo: h = 0.10000000000000003
\end{sphinxVerbatim}

\begin{sphinxVerbatim}[commandchars=\\\{\}]
array([[1.        ],
       [1.39644385],
       [1.7989217 ],
       [2.20909012],
       [2.62852487]])
\end{sphinxVerbatim}


\subsection{Solução analítica para a EDO do PVI}
\label{\detokenize{lista-6-solucoes:solucao-analitica-para-a-edo-do-pvi}}
\begin{sphinxVerbatim}[commandchars=\\\{\}]
\PYG{c+c1}{\PYGZsh{} Encontrando a solução por computação simbólica }
\PYG{k+kn}{import} \PYG{n+nn}{sympy} \PYG{k}{as} \PYG{n+nn}{sp} 
\PYG{n}{sp}\PYG{o}{.}\PYG{n}{init\PYGZus{}printing}\PYG{p}{(}\PYG{p}{)}

\PYG{c+c1}{\PYGZsh{} variável simbólica}
\PYG{n}{tsym} \PYG{o}{=} \PYG{n}{sp}\PYG{o}{.}\PYG{n}{symbols}\PYG{p}{(}\PYG{l+s+s1}{\PYGZsq{}}\PYG{l+s+s1}{t}\PYG{l+s+s1}{\PYGZsq{}}\PYG{p}{)}

\PYG{c+c1}{\PYGZsh{} função}
\PYG{n}{f} \PYG{o}{=} \PYG{n}{sp}\PYG{o}{.}\PYG{n}{symbols}\PYG{p}{(}\PYG{l+s+s1}{\PYGZsq{}}\PYG{l+s+s1}{f}\PYG{l+s+s1}{\PYGZsq{}}\PYG{p}{,} \PYG{n+nb+bp}{cls}\PYG{o}{=}\PYG{n}{sp}\PYG{o}{.}\PYG{n}{Function}\PYG{p}{)}

\PYG{c+c1}{\PYGZsh{} EDO}
\PYG{n}{edo} \PYG{o}{=} \PYG{n}{sp}\PYG{o}{.}\PYG{n}{Eq}\PYG{p}{(} \PYG{n}{f}\PYG{p}{(}\PYG{n}{tsym}\PYG{p}{)}\PYG{o}{.}\PYG{n}{diff}\PYG{p}{(}\PYG{n}{tsym}\PYG{p}{)}\PYG{p}{,} \PYG{n}{f}\PYG{p}{(}\PYG{n}{tsym}\PYG{p}{)}\PYG{o}{*}\PYG{o}{*}\PYG{l+m+mi}{2} \PYG{o}{\PYGZhy{}} \PYG{n}{tsym}\PYG{o}{/}\PYG{l+m+mi}{2} \PYG{o}{+} \PYG{l+m+mi}{3}\PYG{p}{)}

\PYG{c+c1}{\PYGZsh{} solução }
\PYG{c+c1}{\PYGZsh{} lembre que: t0 = 1.2; y0 = 1}
\PYG{n}{sol} \PYG{o}{=} \PYG{n}{sp}\PYG{o}{.}\PYG{n}{dsolve}\PYG{p}{(}\PYG{n}{edo}\PYG{p}{,}\PYG{n}{f}\PYG{p}{(}\PYG{n}{tsym}\PYG{p}{)}\PYG{p}{,}\PYG{n}{ics}\PYG{o}{=}\PYG{p}{\PYGZob{}}\PYG{n}{f}\PYG{p}{(}\PYG{l+m+mf}{1.2}\PYG{p}{)}\PYG{p}{:}\PYG{l+m+mi}{1}\PYG{p}{\PYGZcb{}}\PYG{p}{)}

\PYG{c+c1}{\PYGZsh{} usa membro direito e remove big\PYGZhy{}oh  }
\PYG{n}{sol} \PYG{o}{=} \PYG{n}{sol}\PYG{o}{.}\PYG{n}{rhs}\PYG{o}{.}\PYG{n}{removeO}\PYG{p}{(}\PYG{p}{)}

\PYG{c+c1}{\PYGZsh{} substitui expressão simbólica por numérica}
\PYG{n}{yt} \PYG{o}{=} \PYG{p}{[}\PYG{n}{sol}\PYG{o}{.}\PYG{n}{subs}\PYG{p}{(}\PYG{n}{tsym}\PYG{p}{,}\PYG{n}{i}\PYG{p}{)} \PYG{k}{for} \PYG{n}{i} \PYG{o+ow}{in} \PYG{n}{xr}\PYG{p}{]}
\PYG{n}{yt} \PYG{o}{=} \PYG{n}{np}\PYG{o}{.}\PYG{n}{asarray}\PYG{p}{(}\PYG{n}{yt}\PYG{p}{)}

\PYG{n+nb}{print}\PYG{p}{(}\PYG{l+s+s2}{\PYGZdq{}}\PYG{l+s+s2}{Solução analítica:}\PYG{l+s+s2}{\PYGZdq{}}\PYG{p}{)}
\PYG{n}{sol}
\end{sphinxVerbatim}

\begin{sphinxVerbatim}[commandchars=\\\{\}]
Solução analítica:
\end{sphinxVerbatim}

\noindent\sphinxincludegraphics{{lista-6-solucoes_11_1}.png}


\subsection{Comparação de soluções}
\label{\detokenize{lista-6-solucoes:comparacao-de-solucoes}}
\begin{sphinxVerbatim}[commandchars=\\\{\}]
\PYG{c+c1}{\PYGZsh{} compara todos os resultados}
\PYG{n}{plt}\PYG{o}{.}\PYG{n}{plot}\PYG{p}{(}\PYG{n}{x1}\PYG{p}{,}\PYG{n}{y1}\PYG{p}{,}\PYG{l+s+s1}{\PYGZsq{}}\PYG{l+s+s1}{o\PYGZhy{}}\PYG{l+s+s1}{\PYGZsq{}}\PYG{p}{,}\PYG{n}{label}\PYG{o}{=}\PYG{l+s+s1}{\PYGZsq{}}\PYG{l+s+s1}{Euler}\PYG{l+s+s1}{\PYGZsq{}}\PYG{p}{)}
\PYG{n}{plt}\PYG{o}{.}\PYG{n}{plot}\PYG{p}{(}\PYG{n}{xm}\PYG{p}{,}\PYG{n}{ym}\PYG{p}{,}\PYG{l+s+s1}{\PYGZsq{}}\PYG{l+s+s1}{o\PYGZhy{}}\PYG{l+s+s1}{\PYGZsq{}}\PYG{p}{,}\PYG{n}{label}\PYG{o}{=}\PYG{l+s+s1}{\PYGZsq{}}\PYG{l+s+s1}{ponto médio}\PYG{l+s+s1}{\PYGZsq{}}\PYG{p}{)}
\PYG{n}{plt}\PYG{o}{.}\PYG{n}{plot}\PYG{p}{(}\PYG{n}{xr}\PYG{p}{,}\PYG{n}{yr}\PYG{o}{.}\PYG{n}{T}\PYG{p}{[}\PYG{l+m+mi}{0}\PYG{p}{]}\PYG{p}{,}\PYG{l+s+s1}{\PYGZsq{}}\PYG{l+s+s1}{o\PYGZhy{}}\PYG{l+s+s1}{\PYGZsq{}}\PYG{p}{,}\PYG{n}{label}\PYG{o}{=}\PYG{l+s+s1}{\PYGZsq{}}\PYG{l+s+s1}{odeint}\PYG{l+s+s1}{\PYGZsq{}}\PYG{p}{)}
\PYG{n}{plt}\PYG{o}{.}\PYG{n}{plot}\PYG{p}{(}\PYG{n}{xr}\PYG{p}{,}\PYG{n}{yt}\PYG{p}{,}\PYG{l+s+s1}{\PYGZsq{}}\PYG{l+s+s1}{o\PYGZhy{}}\PYG{l+s+s1}{\PYGZsq{}}\PYG{p}{,}\PYG{n}{label}\PYG{o}{=}\PYG{l+s+s1}{\PYGZsq{}}\PYG{l+s+s1}{analítica}\PYG{l+s+s1}{\PYGZsq{}}\PYG{p}{)}
\PYG{n}{plt}\PYG{o}{.}\PYG{n}{legend}\PYG{p}{(}\PYG{p}{)}\PYG{p}{;}
\end{sphinxVerbatim}

\noindent\sphinxincludegraphics{{lista-6-solucoes_13_0}.png}

\begin{sphinxVerbatim}[commandchars=\\\{\}]
\PYG{c+c1}{\PYGZsh{} variável simbólica}
\PYG{n}{tsym} \PYG{o}{=} \PYG{n}{sp}\PYG{o}{.}\PYG{n}{symbols}\PYG{p}{(}\PYG{l+s+s1}{\PYGZsq{}}\PYG{l+s+s1}{t}\PYG{l+s+s1}{\PYGZsq{}}\PYG{p}{)}

\PYG{c+c1}{\PYGZsh{} função}
\PYG{n}{f} \PYG{o}{=} \PYG{n}{sp}\PYG{o}{.}\PYG{n}{symbols}\PYG{p}{(}\PYG{l+s+s1}{\PYGZsq{}}\PYG{l+s+s1}{f}\PYG{l+s+s1}{\PYGZsq{}}\PYG{p}{,} \PYG{n+nb+bp}{cls}\PYG{o}{=}\PYG{n}{sp}\PYG{o}{.}\PYG{n}{Function}\PYG{p}{)}

\PYG{c+c1}{\PYGZsh{} EDO}
\PYG{n}{edo} \PYG{o}{=} \PYG{n}{sp}\PYG{o}{.}\PYG{n}{Eq}\PYG{p}{(} \PYG{n}{f}\PYG{p}{(}\PYG{n}{tsym}\PYG{p}{)}\PYG{o}{.}\PYG{n}{diff}\PYG{p}{(}\PYG{n}{tsym}\PYG{p}{)} \PYG{o}{\PYGZhy{}} \PYG{l+m+mi}{4}\PYG{o}{*}\PYG{n}{sp}\PYG{o}{.}\PYG{n}{exp}\PYG{p}{(}\PYG{l+m+mf}{0.8}\PYG{o}{*}\PYG{n}{tsym}\PYG{p}{)} \PYG{o}{+} \PYG{l+m+mf}{0.5}\PYG{o}{*}\PYG{n}{f}\PYG{p}{(}\PYG{n}{tsym}\PYG{p}{)}\PYG{p}{,}\PYG{l+m+mi}{0}\PYG{p}{)}

\PYG{c+c1}{\PYGZsh{} solução }
\PYG{n}{sol} \PYG{o}{=} \PYG{n}{sp}\PYG{o}{.}\PYG{n}{dsolve}\PYG{p}{(}\PYG{n}{edo}\PYG{p}{,}\PYG{n}{f}\PYG{p}{(}\PYG{n}{tsym}\PYG{p}{)}\PYG{p}{,}\PYG{n}{ics}\PYG{o}{=}\PYG{p}{\PYGZob{}}\PYG{n}{f}\PYG{p}{(}\PYG{l+m+mi}{0}\PYG{p}{)}\PYG{p}{:}\PYG{l+m+mi}{2}\PYG{p}{\PYGZcb{}}\PYG{p}{)}

\PYG{c+c1}{\PYGZsh{} usa membro direito e remove big\PYGZhy{}oh  }
\PYG{n}{sol} \PYG{o}{=} \PYG{n}{sol}\PYG{o}{.}\PYG{n}{rhs}\PYG{o}{.}\PYG{n}{removeO}\PYG{p}{(}\PYG{p}{)}

\PYG{c+c1}{\PYGZsh{} substitui expressão simbólica por numérica}
\PYG{n}{yt} \PYG{o}{=} \PYG{p}{[}\PYG{n}{sol}\PYG{o}{.}\PYG{n}{subs}\PYG{p}{(}\PYG{n}{tsym}\PYG{p}{,}\PYG{n}{i}\PYG{p}{)} \PYG{k}{for} \PYG{n}{i} \PYG{o+ow}{in} \PYG{n}{xr}\PYG{p}{]}
\PYG{n}{yt} \PYG{o}{=} \PYG{n}{np}\PYG{o}{.}\PYG{n}{asarray}\PYG{p}{(}\PYG{n}{yt}\PYG{p}{)}

\PYG{n+nb}{print}\PYG{p}{(}\PYG{l+s+s2}{\PYGZdq{}}\PYG{l+s+s2}{Solução analítica:}\PYG{l+s+s2}{\PYGZdq{}}\PYG{p}{)}
\PYG{n}{sol}
\end{sphinxVerbatim}

\begin{sphinxVerbatim}[commandchars=\\\{\}]
Solução analítica:
\end{sphinxVerbatim}

\noindent\sphinxincludegraphics{{lista-6-solucoes_14_1}.png}


\part{Extra}


\chapter{Números em ponto flutuante e seus problemas}
\label{\detokenize{extra/extra-pontoFlutuante:numeros-em-ponto-flutuante-e-seus-problemas}}\label{\detokenize{extra/extra-pontoFlutuante::doc}}
\sphinxAtStartPar
Este capítulo é um compêndio sobre a representação numérica em computadores, sistema de ponto flutuante, bem como notas sobre causas de erros. O texto é essencialmente uma reprodução adaptada do conteúdo encontrado no site \sphinxhref{https://floating-point-gui.de}{floating\sphinxhyphen{}point\sphinxhyphen{}gui.de}.


\section{Perguntas e respostas}
\label{\detokenize{extra/extra-pontoFlutuante:perguntas-e-respostas}}

\subsection{Por que a soma \sphinxstyleliteralintitle{\sphinxupquote{0.1 + 0.2}} não é exatamente 0.3, mas resulta em \sphinxstyleliteralintitle{\sphinxupquote{0.30000000000000004}}?}
\label{\detokenize{extra/extra-pontoFlutuante:por-que-a-soma-0-1-0-2-nao-e-exatamente-0-3-mas-resulta-em-0-30000000000000004}}
\sphinxAtStartPar
Porque, internamente, os computadores usam um formato de ponto flutuante binário limitado que não pode representar com precisão números como 0.1, 0.2 ou 0.3. Quando o código é compilado ou interpretado, \sphinxcode{\sphinxupquote{0.1}} já é arredondado para o número mais próximo consoante este formato, o que resulta em um pequeno erro de arredondamento antes mesmo de o cálculo acontecer.

\begin{sphinxVerbatim}[commandchars=\\\{\}]
\PYG{c+c1}{\PYGZsh{} ?!}
\PYG{l+m+mf}{0.1} \PYG{o}{+} \PYG{l+m+mf}{0.2}
\end{sphinxVerbatim}

\begin{sphinxVerbatim}[commandchars=\\\{\}]
0.30000000000000004
\end{sphinxVerbatim}


\subsection{Este sistema parece um tanto estúpido. Por que os computadores o utilizam para fazer cálculos?}
\label{\detokenize{extra/extra-pontoFlutuante:este-sistema-parece-um-tanto-estupido-por-que-os-computadores-o-utilizam-para-fazer-calculos}}
\sphinxAtStartPar
Não se trata de estupidez. O sistema é apenas diferente. Números decimais não podem representar com precisão qualquer número, a exemplo da fração \sphinxcode{\sphinxupquote{1/3}}. Então, temos que usar algum tipo de arredondamento para algo como \sphinxcode{\sphinxupquote{0.33}}. Afinal, não podemos esperar que a soma \sphinxcode{\sphinxupquote{0.33 + 0.33 + 0.33}} seja exatamente igual a \sphinxcode{\sphinxupquote{1}}, não é mesmo?

\sphinxAtStartPar
Os computadores usam números binários porque são mais rápidos ao lidar com eles e, para a maioria dos cálculos usuais, um pequeno erro na 17a. casa decimal não é relevante, pois, de qualquer maneira, os números com os quais trabalhamos não são “redondos” (ou exatamente “precisos”).


\subsection{O que pode ser feito para evitar este problema?}
\label{\detokenize{extra/extra-pontoFlutuante:o-que-pode-ser-feito-para-evitar-este-problema}}
\sphinxAtStartPar
Isso depende do tipo de cálculo pretendido.

\sphinxAtStartPar
Se a precisão dos resultados for absoluta, especialmente quando se trabalha com cálculos monetários, é melhor usar um tipo de dado especial, a saber o \sphinxcode{\sphinxupquote{decimal}}. Em Python, por exemplo, há um \sphinxhref{https://docs.python.org/3/library/decimal.html}{módulo de mesmo nome} para esta finalidade específica. Por exemplo:

\begin{sphinxVerbatim}[commandchars=\\\{\}]
\PYG{c+c1}{\PYGZsh{} 0.1 + 0.2 }
\PYG{k+kn}{from} \PYG{n+nn}{decimal} \PYG{k+kn}{import} \PYG{n}{Decimal} 

\PYG{n}{a}\PYG{p}{,}\PYG{n}{b} \PYG{o}{=} \PYG{n}{Decimal}\PYG{p}{(}\PYG{l+s+s1}{\PYGZsq{}}\PYG{l+s+s1}{0.1}\PYG{l+s+s1}{\PYGZsq{}}\PYG{p}{)}\PYG{p}{,} \PYG{n}{Decimal}\PYG{p}{(}\PYG{l+s+s1}{\PYGZsq{}}\PYG{l+s+s1}{0.2}\PYG{l+s+s1}{\PYGZsq{}}\PYG{p}{)}
\PYG{n+nb}{print}\PYG{p}{(}\PYG{n}{a}\PYG{o}{+}\PYG{n}{b}\PYG{p}{)}
\end{sphinxVerbatim}

\begin{sphinxVerbatim}[commandchars=\\\{\}]
0.3
\end{sphinxVerbatim}

\sphinxAtStartPar
Se apenas se deseja enxergar algumas casas decimais, pode\sphinxhyphen{}se formatar o resultado para um número fixo de casas decimais que será exibido de forma arredondada.

\begin{sphinxVerbatim}[commandchars=\\\{\}]
\PYG{c+c1}{\PYGZsh{} imprime resultado como float}
\PYG{n+nb}{print}\PYG{p}{(}\PYG{l+m+mf}{0.1} \PYG{o}{+} \PYG{l+m+mf}{0.2}\PYG{p}{)}

\PYG{c+c1}{\PYGZsh{} imprime resultado com 3 casas decimais}
\PYG{n+nb}{print}\PYG{p}{(}\PYG{l+s+sa}{f}\PYG{l+s+s1}{\PYGZsq{}}\PYG{l+s+si}{\PYGZob{}}\PYG{l+m+mf}{0.1} \PYG{o}{+} \PYG{l+m+mf}{0.2}\PYG{l+s+si}{:}\PYG{l+s+s1}{.3f}\PYG{l+s+si}{\PYGZcb{}}\PYG{l+s+s1}{\PYGZsq{}}\PYG{p}{)}
\end{sphinxVerbatim}

\begin{sphinxVerbatim}[commandchars=\\\{\}]
0.30000000000000004
0.300
\end{sphinxVerbatim}


\subsection{Por que outros cálculos como \sphinxstyleliteralintitle{\sphinxupquote{0.1 + 0.4}} funcionam corretamente?}
\label{\detokenize{extra/extra-pontoFlutuante:por-que-outros-calculos-como-0-1-0-4-funcionam-corretamente}}
\sphinxAtStartPar
Nesse caso, o resultado, \sphinxcode{\sphinxupquote{0.5}}, \sphinxstylestrong{pode} ser representado exatamente como um número de ponto flutuante e é possível que haja erro de cancelamento nos números de entrada. Porém, pode não podemos confiar interamente nisto. Por exemplo, quando tais dois números são primeiro armazenados em representações de ponto flutuante de tamanhos diferentes, os erros de arredondamento podem não compensar um com o outro.

\sphinxAtStartPar
Em outros casos, como \sphinxcode{\sphinxupquote{0.1 + 0.3}}, o resultado \sphinxstylestrong{não é realmente} \sphinxcode{\sphinxupquote{0.4}}, mas próximo o suficiente para que \sphinxcode{\sphinxupquote{0.4}} seja o menor número mais próximo do resultado do que qualquer outro número em ponto flutuante. Muitas linguagens de programação exibem esse número em vez de converter o resultado real de volta para a fração decimal mais próxima. Por exemplo:

\begin{sphinxVerbatim}[commandchars=\\\{\}]
\PYG{n+nb}{print}\PYG{p}{(}\PYG{l+m+mf}{0.1} \PYG{o}{+} \PYG{l+m+mf}{0.4}\PYG{p}{)}
\PYG{n+nb}{print}\PYG{p}{(}\PYG{l+m+mf}{0.1} \PYG{o}{+} \PYG{l+m+mf}{0.1} \PYG{o}{+} \PYG{l+m+mf}{0.1} \PYG{o}{\PYGZhy{}} \PYG{l+m+mf}{0.3}\PYG{p}{)} \PYG{c+c1}{\PYGZsh{} 0 !}
\end{sphinxVerbatim}

\begin{sphinxVerbatim}[commandchars=\\\{\}]
0.5
5.551115123125783e\PYGZhy{}17
\end{sphinxVerbatim}


\section{Comparação de números em ponto flutuante}
\label{\detokenize{extra/extra-pontoFlutuante:comparacao-de-numeros-em-ponto-flutuante}}
\sphinxAtStartPar
Devido a erros de arredondamento, a representação da maioria dos números em ponto flutuante torna\sphinxhyphen{}se  imprecisa. Enquanto essa imprecisão permanecer pequena, ela poderá ser geralmente ignorada. No entanto, às vezes, os números que esperamos ser iguais (por exemplo, ao calcular o mesmo resultado por diferentes métodos corretos) diferem ligeiramente de tal forma que um mero teste de igualdade implica em falha. Por exemplo:

\begin{sphinxVerbatim}[commandchars=\\\{\}]
\PYG{n}{a} \PYG{o}{=} \PYG{l+m+mf}{0.15} \PYG{o}{+} \PYG{l+m+mf}{0.15}
\PYG{n}{b} \PYG{o}{=} \PYG{l+m+mf}{0.1} \PYG{o}{+} \PYG{l+m+mf}{0.2}

\PYG{c+c1}{\PYGZsh{} Os resultados abaixo deveriam ser verdadeiros (True),}
\PYG{c+c1}{\PYGZsh{} mas não o são!}
\PYG{n+nb}{print}\PYG{p}{(}\PYG{n}{a} \PYG{o}{==} \PYG{n}{b}\PYG{p}{)} 
\PYG{n+nb}{print}\PYG{p}{(}\PYG{n}{a} \PYG{o}{\PYGZgt{}}\PYG{o}{=} \PYG{n}{b}\PYG{p}{)} 
\end{sphinxVerbatim}

\begin{sphinxVerbatim}[commandchars=\\\{\}]
False
False
\end{sphinxVerbatim}


\subsection{Margens de erro: absoluto x relativo}
\label{\detokenize{extra/extra-pontoFlutuante:margens-de-erro-absoluto-x-relativo}}
\sphinxAtStartPar
Ao compararmos dois números reais, o melhor caminho a seguir não é verificar se os números são exatamente iguais, mas se a \sphinxstyleemphasis{diferença entre ambos é muito pequena}. A margem de erro com a qual a diferença é comparada costuma ser chamada de “epsilon”. A forma mais simples seria por meio do erro absoluto. Por exemplo:

\begin{sphinxVerbatim}[commandchars=\\\{\}]
\PYG{c+c1}{\PYGZsh{} comparação por erro absoluto}
\PYG{k}{if} \PYG{n+nb}{abs}\PYG{p}{(}\PYG{n}{a} \PYG{o}{\PYGZhy{}} \PYG{n}{b}\PYG{p}{)} \PYG{o}{\PYGZlt{}} \PYG{l+m+mf}{1e\PYGZhy{}5}\PYG{p}{:}
    \PYG{n+nb}{print}\PYG{p}{(}\PYG{l+s+s1}{\PYGZsq{}}\PYG{l+s+s1}{OK!}\PYG{l+s+s1}{\PYGZsq{}}\PYG{p}{)}
\end{sphinxVerbatim}

\begin{sphinxVerbatim}[commandchars=\\\{\}]
OK!
\end{sphinxVerbatim}

\sphinxAtStartPar
Isto é, a expressão acima é matematicamente equivalente a \(|a - b| < \epsilon\) para \(\epsilon=10^{-5}\).

\sphinxAtStartPar
Entretanto, essa é uma maneira ruim de comparar números reais, porque um \sphinxstyleemphasis{epsilon} fixo escolhido que parece “pequeno” pode, na verdade, ser muito grande quando os números comparados também forem muito pequenos. A comparação retornaria \sphinxstyleemphasis{verdadeiro} para números bastante diferentes. E quando os números são muito grandes, o \sphinxstyleemphasis{epsilon} pode acabar sendo menor que o menor erro de arredondamento, de forma que a comparação sempre retorna \sphinxstyleemphasis{falso}.

\sphinxAtStartPar
Assim, é razoável verificar se o \sphinxstyleemphasis{erro relativo} é menor que o \sphinxstyleemphasis{epsilon}.

\begin{sphinxVerbatim}[commandchars=\\\{\}]
\PYG{c+c1}{\PYGZsh{} comparação por erro relativo}
\PYG{k}{if} \PYG{n+nb}{abs}\PYG{p}{(}\PYG{p}{(}\PYG{n}{a} \PYG{o}{\PYGZhy{}} \PYG{n}{b}\PYG{p}{)}\PYG{o}{/}\PYG{n}{b}\PYG{p}{)} \PYG{o}{\PYGZlt{}} \PYG{l+m+mf}{1e\PYGZhy{}5}\PYG{p}{:}
    \PYG{n+nb}{print}\PYG{p}{(}\PYG{l+s+s1}{\PYGZsq{}}\PYG{l+s+s1}{OK!}\PYG{l+s+s1}{\PYGZsq{}}\PYG{p}{)}
\end{sphinxVerbatim}

\begin{sphinxVerbatim}[commandchars=\\\{\}]
OK!
\end{sphinxVerbatim}

\sphinxAtStartPar
Mas, esta forma ainda não é inteiramente correta para alguns casos especiais, a saber:
\begin{itemize}
\item {} 
\sphinxAtStartPar
quando tanto \sphinxcode{\sphinxupquote{a}} quanto \sphinxcode{\sphinxupquote{b}} são iguais a zero, a fração resultante \sphinxcode{\sphinxupquote{0.0/0.0}} é uma indefinição do tipo \sphinxstyleemphasis{not a number} (\sphinxcode{\sphinxupquote{NaN}}), a qual gera uma exceção em algumas plataformas, ou retorna \sphinxstyleemphasis{falso} para todas as comparações;

\end{itemize}

\begin{sphinxVerbatim}[commandchars=\\\{\}]
\PYG{n}{a}\PYG{p}{,}\PYG{n}{b} \PYG{o}{=} \PYG{l+m+mi}{0}\PYG{p}{,}\PYG{l+m+mi}{0}
\PYG{n+nb}{abs}\PYG{p}{(}\PYG{p}{(}\PYG{n}{a}\PYG{o}{\PYGZhy{}}\PYG{n}{b}\PYG{p}{)}\PYG{o}{/}\PYG{n}{b}\PYG{p}{)} \PYG{c+c1}{\PYGZsh{} exceção \PYGZsq{}ZeroDivisionError\PYGZsq{}}
\end{sphinxVerbatim}

\begin{sphinxVerbatim}[commandchars=\\\{\}]
\PYG{g+gt}{\PYGZhy{}\PYGZhy{}\PYGZhy{}\PYGZhy{}\PYGZhy{}\PYGZhy{}\PYGZhy{}\PYGZhy{}\PYGZhy{}\PYGZhy{}\PYGZhy{}\PYGZhy{}\PYGZhy{}\PYGZhy{}\PYGZhy{}\PYGZhy{}\PYGZhy{}\PYGZhy{}\PYGZhy{}\PYGZhy{}\PYGZhy{}\PYGZhy{}\PYGZhy{}\PYGZhy{}\PYGZhy{}\PYGZhy{}\PYGZhy{}\PYGZhy{}\PYGZhy{}\PYGZhy{}\PYGZhy{}\PYGZhy{}\PYGZhy{}\PYGZhy{}\PYGZhy{}\PYGZhy{}\PYGZhy{}\PYGZhy{}\PYGZhy{}\PYGZhy{}\PYGZhy{}\PYGZhy{}\PYGZhy{}\PYGZhy{}\PYGZhy{}\PYGZhy{}\PYGZhy{}\PYGZhy{}\PYGZhy{}\PYGZhy{}\PYGZhy{}\PYGZhy{}\PYGZhy{}\PYGZhy{}\PYGZhy{}\PYGZhy{}\PYGZhy{}\PYGZhy{}\PYGZhy{}\PYGZhy{}\PYGZhy{}\PYGZhy{}\PYGZhy{}\PYGZhy{}\PYGZhy{}\PYGZhy{}\PYGZhy{}\PYGZhy{}\PYGZhy{}\PYGZhy{}\PYGZhy{}\PYGZhy{}\PYGZhy{}\PYGZhy{}\PYGZhy{}}
\PYG{n+ne}{ZeroDivisionError}\PYG{g+gWhitespace}{                         }Traceback (most recent call last)
\PYG{o}{\PYGZlt{}}\PYG{n}{ipython}\PYG{o}{\PYGZhy{}}\PYG{n+nb}{input}\PYG{o}{\PYGZhy{}}\PYG{l+m+mi}{175}\PYG{o}{\PYGZhy{}}\PYG{n}{fb6d2c8ce8ce}\PYG{o}{\PYGZgt{}} \PYG{o+ow}{in} \PYG{o}{\PYGZlt{}}\PYG{n}{module}\PYG{o}{\PYGZgt{}}
\PYG{g+gWhitespace}{      }\PYG{l+m+mi}{1} \PYG{n}{a}\PYG{p}{,}\PYG{n}{b} \PYG{o}{=} \PYG{l+m+mi}{0}\PYG{p}{,}\PYG{l+m+mi}{0}
\PYG{n+ne}{\PYGZhy{}\PYGZhy{}\PYGZhy{}\PYGZhy{}\PYGZgt{} }\PYG{l+m+mi}{2} \PYG{n+nb}{abs}\PYG{p}{(}\PYG{p}{(}\PYG{n}{a}\PYG{o}{\PYGZhy{}}\PYG{n}{b}\PYG{p}{)}\PYG{o}{/}\PYG{n}{b}\PYG{p}{)} \PYG{c+c1}{\PYGZsh{} exceção \PYGZsq{}ZeroDivisionError\PYGZsq{}}

\PYG{n+ne}{ZeroDivisionError}: division by zero
\end{sphinxVerbatim}
\begin{itemize}
\item {} 
\sphinxAtStartPar
quando apenas \sphinxcode{\sphinxupquote{b}} é igual a zero, a divisão produz o \sphinxstyleemphasis{infinito} (\(\infty\)), que pode gerar uma exceção ou ser maior do que epsilon mesmo quando \sphinxcode{\sphinxupquote{a}} for menor;

\end{itemize}

\begin{sphinxVerbatim}[commandchars=\\\{\}]
\PYG{n}{a}\PYG{p}{,}\PYG{n}{b} \PYG{o}{=} \PYG{l+m+mf}{1e\PYGZhy{}1}\PYG{p}{,}\PYG{l+m+mi}{0}
\PYG{n+nb}{abs}\PYG{p}{(}\PYG{p}{(}\PYG{n}{a}\PYG{o}{\PYGZhy{}}\PYG{n}{b}\PYG{p}{)}\PYG{o}{/}\PYG{n}{b}\PYG{p}{)} \PYG{c+c1}{\PYGZsh{} exceção \PYGZsq{}ZeroDivisionError\PYGZsq{}}
\end{sphinxVerbatim}

\begin{sphinxVerbatim}[commandchars=\\\{\}]
\PYG{g+gt}{\PYGZhy{}\PYGZhy{}\PYGZhy{}\PYGZhy{}\PYGZhy{}\PYGZhy{}\PYGZhy{}\PYGZhy{}\PYGZhy{}\PYGZhy{}\PYGZhy{}\PYGZhy{}\PYGZhy{}\PYGZhy{}\PYGZhy{}\PYGZhy{}\PYGZhy{}\PYGZhy{}\PYGZhy{}\PYGZhy{}\PYGZhy{}\PYGZhy{}\PYGZhy{}\PYGZhy{}\PYGZhy{}\PYGZhy{}\PYGZhy{}\PYGZhy{}\PYGZhy{}\PYGZhy{}\PYGZhy{}\PYGZhy{}\PYGZhy{}\PYGZhy{}\PYGZhy{}\PYGZhy{}\PYGZhy{}\PYGZhy{}\PYGZhy{}\PYGZhy{}\PYGZhy{}\PYGZhy{}\PYGZhy{}\PYGZhy{}\PYGZhy{}\PYGZhy{}\PYGZhy{}\PYGZhy{}\PYGZhy{}\PYGZhy{}\PYGZhy{}\PYGZhy{}\PYGZhy{}\PYGZhy{}\PYGZhy{}\PYGZhy{}\PYGZhy{}\PYGZhy{}\PYGZhy{}\PYGZhy{}\PYGZhy{}\PYGZhy{}\PYGZhy{}\PYGZhy{}\PYGZhy{}\PYGZhy{}\PYGZhy{}\PYGZhy{}\PYGZhy{}\PYGZhy{}\PYGZhy{}\PYGZhy{}\PYGZhy{}\PYGZhy{}\PYGZhy{}}
\PYG{n+ne}{ZeroDivisionError}\PYG{g+gWhitespace}{                         }Traceback (most recent call last)
\PYG{o}{\PYGZlt{}}\PYG{n}{ipython}\PYG{o}{\PYGZhy{}}\PYG{n+nb}{input}\PYG{o}{\PYGZhy{}}\PYG{l+m+mi}{176}\PYG{o}{\PYGZhy{}}\PYG{l+m+mi}{4}\PYG{n}{bc2e683f9ff}\PYG{o}{\PYGZgt{}} \PYG{o+ow}{in} \PYG{o}{\PYGZlt{}}\PYG{n}{module}\PYG{o}{\PYGZgt{}}
\PYG{g+gWhitespace}{      }\PYG{l+m+mi}{1} \PYG{n}{a}\PYG{p}{,}\PYG{n}{b} \PYG{o}{=} \PYG{l+m+mf}{1e\PYGZhy{}1}\PYG{p}{,}\PYG{l+m+mi}{0}
\PYG{n+ne}{\PYGZhy{}\PYGZhy{}\PYGZhy{}\PYGZhy{}\PYGZgt{} }\PYG{l+m+mi}{2} \PYG{n+nb}{abs}\PYG{p}{(}\PYG{p}{(}\PYG{n}{a}\PYG{o}{\PYGZhy{}}\PYG{n}{b}\PYG{p}{)}\PYG{o}{/}\PYG{n}{b}\PYG{p}{)} \PYG{c+c1}{\PYGZsh{} exceção \PYGZsq{}ZeroDivisionError\PYGZsq{}}

\PYG{n+ne}{ZeroDivisionError}: float division by zero
\end{sphinxVerbatim}
\begin{itemize}
\item {} 
\sphinxAtStartPar
quando \sphinxcode{\sphinxupquote{a}} e \sphinxcode{\sphinxupquote{b}} são muito pequenos, mas estão em lados opostos a zero, a comparação retorna \sphinxstyleemphasis{falso}, ainda que ambos sejam os menores números diferentes de zero possíveis.

\end{itemize}

\begin{sphinxVerbatim}[commandchars=\\\{\}]
\PYG{n}{a}\PYG{p}{,}\PYG{n}{b} \PYG{o}{=} \PYG{o}{\PYGZhy{}}\PYG{l+m+mf}{0.009e\PYGZhy{}20}\PYG{p}{,}\PYG{l+m+mf}{1.2e\PYGZhy{}19}
\PYG{n+nb}{abs}\PYG{p}{(}\PYG{p}{(}\PYG{n}{a}\PYG{o}{\PYGZhy{}}\PYG{n}{b}\PYG{p}{)}\PYG{o}{/}\PYG{n}{b}\PYG{p}{)} \PYG{o}{\PYGZlt{}} \PYG{l+m+mf}{1e\PYGZhy{}5}
\end{sphinxVerbatim}

\begin{sphinxVerbatim}[commandchars=\\\{\}]
False
\end{sphinxVerbatim}

\sphinxAtStartPar
Além disso, o resultado pode não ser sempre comutativo. Isto é, \sphinxcode{\sphinxupquote{abs((a\sphinxhyphen{}b)/b)}} pode ser diferente de  \sphinxcode{\sphinxupquote{abs((b\sphinxhyphen{}a)/b)}}.

\sphinxAtStartPar
É possível escrever uma função – veja abaixo – capaz de passar em vários testes para casos especiais, porém ela usa uma lógica com pouca obviedade. A margem de erro deve ser definida de maneira diferente dependendo dos valores de \sphinxcode{\sphinxupquote{a}} ou \sphinxcode{\sphinxupquote{b}}, porque a definição clássica de erro relativo torna\sphinxhyphen{}se insignificante nesses casos.

\begin{sphinxVerbatim}[commandchars=\\\{\}]
\PYG{k}{def} \PYG{n+nf}{compare\PYGZus{}float}\PYG{p}{(}\PYG{n}{a}\PYG{p}{:}\PYG{n+nb}{float}\PYG{p}{,}\PYG{n}{b}\PYG{p}{:}\PYG{n+nb}{float}\PYG{p}{,}\PYG{n}{eps}\PYG{p}{:}\PYG{n+nb}{float}\PYG{p}{)} \PYG{o}{\PYGZhy{}}\PYG{o}{\PYGZgt{}} \PYG{n+nb}{bool}\PYG{p}{:} \PYG{c+c1}{\PYGZsh{} \PYGZdq{}:\PYGZdq{} e \PYGZdq{}\PYGZhy{}\PYGZgt{}\PYGZdq{} são apenas anotações didáticas    }
    
    \PYG{k+kn}{import} \PYG{n+nn}{sys}
    \PYG{n}{MIN} \PYG{o}{=} \PYG{n}{sys}\PYG{o}{.}\PYG{n}{float\PYGZus{}info}\PYG{o}{.}\PYG{n}{min} \PYG{c+c1}{\PYGZsh{} menor float : \PYGZti{} 1.80e+308}
    \PYG{n}{MAX} \PYG{o}{=} \PYG{n}{sys}\PYG{o}{.}\PYG{n}{float\PYGZus{}info}\PYG{o}{.}\PYG{n}{max} \PYG{c+c1}{\PYGZsh{} maior float : \PYGZti{} \PYGZhy{}2.23e+308}
        
    \PYG{n}{diff} \PYG{o}{=} \PYG{n+nb}{abs}\PYG{p}{(}\PYG{n}{a}\PYG{o}{\PYGZhy{}}\PYG{n}{b}\PYG{p}{)}   
    \PYG{n}{a} \PYG{o}{=} \PYG{n+nb}{abs}\PYG{p}{(}\PYG{n}{a}\PYG{p}{)}
    \PYG{n}{b} \PYG{o}{=} \PYG{n+nb}{abs}\PYG{p}{(}\PYG{n}{b}\PYG{p}{)} 
    
    \PYG{k}{if} \PYG{p}{(}\PYG{n}{a} \PYG{o}{==} \PYG{n}{b}\PYG{p}{)}\PYG{p}{:} \PYG{c+c1}{\PYGZsh{} trata \PYGZsq{}inf\PYGZsq{}}
        \PYG{k}{return} \PYG{k+kc}{True}
    \PYG{k}{elif} \PYG{p}{(}\PYG{n}{a} \PYG{o}{==} \PYG{l+m+mi}{0} \PYG{o+ow}{or} \PYG{n}{b} \PYG{o}{==} \PYG{l+m+mi}{0} \PYG{o+ow}{or} \PYG{p}{(}\PYG{n}{a} \PYG{o}{+} \PYG{n}{b} \PYG{o}{\PYGZlt{}} \PYG{n}{MIN}\PYG{p}{)}\PYG{p}{)}\PYG{p}{:} \PYG{c+c1}{\PYGZsh{} a ou b = 0 ou ambos extremamente próximos de 0}
        \PYG{k}{return} \PYG{n}{diff} \PYG{o}{\PYGZlt{}} \PYG{p}{(}\PYG{n}{eps}\PYG{o}{*}\PYG{n}{MIN}\PYG{p}{)}
    \PYG{k}{else}\PYG{p}{:} \PYG{c+c1}{\PYGZsh{} erro relativo}
        \PYG{k}{return} \PYG{n}{diff}\PYG{o}{/}\PYG{p}{(}\PYG{n+nb}{min}\PYG{p}{(}\PYG{n}{a} \PYG{o}{+} \PYG{n}{b}\PYG{p}{,} \PYG{n}{MAX}\PYG{p}{)}\PYG{p}{)} \PYG{o}{\PYGZlt{}} \PYG{n}{eps}
\end{sphinxVerbatim}

\sphinxAtStartPar
Exemplos:

\begin{sphinxVerbatim}[commandchars=\\\{\}]
\PYG{n+nb}{print}\PYG{p}{(}\PYG{n}{compare\PYGZus{}float}\PYG{p}{(}\PYG{l+m+mf}{1e\PYGZhy{}3}\PYG{p}{,}\PYG{l+m+mf}{1e\PYGZhy{}3}\PYG{p}{,}\PYG{l+m+mf}{1e\PYGZhy{}10}\PYG{p}{)}\PYG{p}{)}
\PYG{n+nb}{print}\PYG{p}{(}\PYG{n}{compare\PYGZus{}float}\PYG{p}{(}\PYG{l+m+mf}{1e\PYGZhy{}3}\PYG{p}{,}\PYG{l+m+mf}{1.1e\PYGZhy{}3}\PYG{p}{,}\PYG{l+m+mf}{0.01}\PYG{p}{)}\PYG{p}{)}
\PYG{n+nb}{print}\PYG{p}{(}\PYG{n}{compare\PYGZus{}float}\PYG{p}{(}\PYG{l+m+mf}{10.111}\PYG{p}{,}\PYG{l+m+mf}{10.1111}\PYG{p}{,}\PYG{l+m+mf}{1e\PYGZhy{}5}\PYG{p}{)}\PYG{p}{)}
\PYG{n+nb}{print}\PYG{p}{(}\PYG{n}{compare\PYGZus{}float}\PYG{p}{(}\PYG{l+m+mf}{10.111}\PYG{p}{,}\PYG{l+m+mf}{10.1111}\PYG{p}{,}\PYG{l+m+mf}{1e\PYGZhy{}6}\PYG{p}{)}\PYG{p}{)}
\end{sphinxVerbatim}

\begin{sphinxVerbatim}[commandchars=\\\{\}]
True
False
True
False
\end{sphinxVerbatim}


\subsection{Referências para estudo}
\label{\detokenize{extra/extra-pontoFlutuante:referencias-para-estudo}}

\subsubsection{Miscelânea}
\label{\detokenize{extra/extra-pontoFlutuante:miscelanea}}\begin{itemize}
\item {} 
\sphinxAtStartPar
\sphinxhref{https://standards.ieee.org/content/ieee-standards/en/standard/754-2019.html}{IEEE 754\sphinxhyphen{}2019 \sphinxhyphen{} IEEE Standard for Floating\sphinxhyphen{}Point Arithmetic}

\item {} 
\sphinxAtStartPar
\sphinxhref{http://download.oracle.com/docs/cd/E19957-01/806-3568/ncg\_goldberg.html}{What Every Computer Scientist Should Know About Floating\sphinxhyphen{}Point Arithmetic}, artigo publicado por David Goldberg na ACM Computing Surveys, Vol. 23 (1), 1991. DOI: \sphinxhref{https://dl.acm.org/doi/abs/10.1145/103162.103163}{10.1145/103162.103163}

\item {} 
\sphinxAtStartPar
\sphinxhref{http://www.cs.berkeley.edu/~wkahan/}{William Kahan’s Homepage (arquiteto do padrão IEEE 754, e vários outros links)}

\item {} 
\sphinxAtStartPar
\sphinxhref{http://speleotrove.com/decimal/decifaq.html}{Decimal Arithmetic FAQ
(Frequently Asked Questions)}

\item {} 
\sphinxAtStartPar
\sphinxhref{https://docs.python.org/3/library/decimal.html}{Python documentation \sphinxhyphen{} \sphinxcode{\sphinxupquote{decimal}} module}

\item {} 
\sphinxAtStartPar
\sphinxhref{https://stackoverflow.com/questions/588004/is-floating-point-math-broken}{Is floating point math broken? @StackOverflow}

\item {} 
\sphinxAtStartPar
\sphinxhref{https://www.lahey.com/float.htm}{The Perils of Floating Point}

\end{itemize}


\subsubsection{Visualizadores interativos de números no padrão IEEE 754}
\label{\detokenize{extra/extra-pontoFlutuante:visualizadores-interativos-de-numeros-no-padrao-ieee-754}}\begin{itemize}
\item {} 
\sphinxAtStartPar
\sphinxhref{https://bartaz.github.io/ieee754-visualization/}{IEEE 754 Visualization}

\item {} 
\sphinxAtStartPar
{[}Float exposed{]}

\end{itemize}


\subsubsection{Exemplos de comparação de números em ponto flutuante}
\label{\detokenize{extra/extra-pontoFlutuante:exemplos-de-comparacao-de-numeros-em-ponto-flutuante}}\begin{itemize}
\item {} 
\sphinxAtStartPar
\sphinxhref{https://randomascii.wordpress.com/2012/02/25/comparing-floating-point-numbers-2012-edition/}{Random ASCII – tech blog of Bruce Dawson}

\end{itemize}


\subsubsection{Livros}
\label{\detokenize{extra/extra-pontoFlutuante:livros}}\begin{itemize}
\item {} 
\sphinxAtStartPar
Modern Computer Architecture and Organization, por Jim Ledin

\item {} 
\sphinxAtStartPar
Computer Organization and Architecture Designing for Performance, por William Stallings

\item {} 
\sphinxAtStartPar
Numerical Computing with IEEE Floating Point Arithmetic, por Michael L. Overton

\end{itemize}


\chapter{Processamento de sinais}
\label{\detokenize{extra/extra-fft:processamento-de-sinais}}\label{\detokenize{extra/extra-fft::doc}}
\sphinxAtStartPar
Na ciência e engenharias, o processamento de sinais é bastante útil, principalmente em áreas como \sphinxstyleemphasis{teoria do controle}. Um sinal pode ser:
\begin{itemize}
\item {} 
\sphinxAtStartPar
\sphinxstyleemphasis{temporal}: uma quantidade que varia no tempo (ex.: sinal de áudio, música etc.)

\item {} 
\sphinxAtStartPar
\sphinxstyleemphasis{espacial}: uma quantidade que varia no espaço (ex.: uma imagem 2D)

\end{itemize}

\sphinxAtStartPar
Sinais são frequentemente funções contínuas. Em aplicações computacionais, entretanto, eles se tornam discretos, visto que são amostrados em um conjunto limitado de pontos separados de distâncias uniformes.

\sphinxAtStartPar
Um algoritmo importantíssimo utilizado para processar sinais é a Transformada Rápida de Fourier, conhecida como \sphinxstyleemphasis{Fast Fourier Transform}, ou simplesmente \sphinxstyleemphasis{FFT}.


\section{Análise espectral}
\label{\detokenize{extra/extra-fft:analise-espectral}}
\sphinxAtStartPar
A análise espectral, como o nome já diz, estuda espectros de frequencias e forma um arcabouço fundamental para aplicação da FFT e de transformadas de Fourier mais gerais. Transformadas são integrais matemáticas que nos permitem transformar um sinal do \sphinxstyleemphasis{domínio temporal} (onde ele é descrito como uma função do tempo) para o \sphinxstyleemphasis{domínio de frequencias} (onde ele é representado como uma função da frequencia).

\sphinxAtStartPar
A representação de um sinal no domínio de frequencias é útil para muitos objetivos, entre os quais podemos citar a extração de frequencias dominantes, a aplicação de filtros e a resolução equações diferenciais.


\section{Transformadas de Fourier}
\label{\detokenize{extra/extra-fft:transformadas-de-fourier}}
\sphinxAtStartPar
A expressão matemática para a transformada de Fourier \(F(v)\) de um sinal contínuo \(f(t)\) é
\begin{equation*}
\begin{split}F(v) = \int_{-\infty}^{+\infty} f(t) e^{-2 \pi i v t} \, dt.\end{split}
\end{equation*}
\sphinxAtStartPar
A transformada de Fourier inversa é dada por:
\begin{equation*}
\begin{split}f(t) = \int_{-\infty}^{+\infty} F(v) e^{2 \pi i v t} \, dv,\end{split}
\end{equation*}
\sphinxAtStartPar
Acima,
\begin{itemize}
\item {} 
\sphinxAtStartPar
\(F(v)\) é o espectro de amplitudes do sinal \(f(t)\) (uma função complexa)

\item {} 
\sphinxAtStartPar
\(v\) é a frequencia

\item {} 
\sphinxAtStartPar
\(F(t)\) é um sinal contínuo com duração infinita

\item {} 
\sphinxAtStartPar
\(t\) é uma coordenada temporal.

\end{itemize}

\sphinxAtStartPar
Usualmente, aplicações computacionais baseiam\sphinxhyphen{}se em amostrar a função \(f(t)\) em \(N\) pontos uniformemente espaçados \(x_0, x_1, \ldots, x_N\) durante um intervalo finito de tempo \(0 \leq t \leq T\), de modo que a transformada contínua anterior seja adaptada para a \sphinxstyleemphasis{Transformada de Fourier Discreta} (DFT), dada por:
\begin{equation*}
\begin{split}X_k = \displaystyle\sum_{n=0}^{N-1}x_n e^{\frac{-2 \pi i n k}{N}},\end{split}
\end{equation*}
\sphinxAtStartPar
cuja DFT inversa é
\begin{equation*}
\begin{split}x_n = \frac{1}{N}\displaystyle\sum_{k=0}^{N-1}X_k e^{\frac{-2 \pi i n k}{N}}.\end{split}
\end{equation*}
\sphinxAtStartPar
Acima,
\begin{itemize}
\item {} 
\sphinxAtStartPar
\(X_k\) é a DFT das amostras \(x_n\);

\item {} 
\sphinxAtStartPar
\(k\) é uma faixa de frequencia (\sphinxstyleemphasis{bin}).

\end{itemize}

\sphinxAtStartPar
O cálculo eficiente da DFT é feito pelo algoritmo conhecido como FFT.


\section{Módulo \sphinxstyleliteralintitle{\sphinxupquote{fftpack}}}
\label{\detokenize{extra/extra-fft:modulo-fftpack}}
\sphinxAtStartPar
Podemos usar o submódulo \sphinxcode{\sphinxupquote{fftpack}} do \sphinxcode{\sphinxupquote{scipy}} para ter acesso a implementações da FFT. Veremos uma aplicação das funções \sphinxcode{\sphinxupquote{fft}} e \sphinxcode{\sphinxupquote{ifft}}, que são, respectivamente, implementações da FFT geral e de sua inversa.


\subsection{Importação de módulos}
\label{\detokenize{extra/extra-fft:importacao-de-modulos}}
\begin{sphinxVerbatim}[commandchars=\\\{\}]
\PYG{k+kn}{from} \PYG{n+nn}{scipy} \PYG{k+kn}{import} \PYG{n}{fftpack}
\PYG{k+kn}{import} \PYG{n+nn}{numpy} \PYG{k}{as} \PYG{n+nn}{np}
\PYG{k+kn}{import} \PYG{n+nn}{matplotlib}\PYG{n+nn}{.}\PYG{n+nn}{pyplot} \PYG{k}{as} \PYG{n+nn}{plt}

\PYG{o}{\PYGZpc{}}\PYG{k}{matplotlib} inline
\end{sphinxVerbatim}

\sphinxAtStartPar
Aqui, criamos um sinal simulado com componentes senoidais puras em 1 Hz e em 22 Hz em cima de um ruído de distribuição normal. A função abaixo gera amostras ruidosas deste sinal:

\begin{sphinxVerbatim}[commandchars=\\\{\}]
\PYG{k}{def} \PYG{n+nf}{amostra\PYGZus{}sinal}\PYG{p}{(}\PYG{n}{t}\PYG{p}{)}\PYG{p}{:}
    \PYG{k}{return} \PYG{p}{(}\PYG{l+m+mi}{2} \PYG{o}{*} \PYG{n}{np}\PYG{o}{.}\PYG{n}{sin}\PYG{p}{(}\PYG{l+m+mi}{2} \PYG{o}{*} \PYG{n}{np}\PYG{o}{.}\PYG{n}{pi} \PYG{o}{*} \PYG{n}{t}\PYG{p}{)} \PYG{o}{+} 
            \PYG{l+m+mi}{3} \PYG{o}{*} \PYG{n}{np}\PYG{o}{.}\PYG{n}{sin}\PYG{p}{(}\PYG{l+m+mi}{22} \PYG{o}{*} \PYG{l+m+mi}{2} \PYG{o}{*} \PYG{n}{np}\PYG{o}{.}\PYG{n}{pi} \PYG{o}{*} \PYG{n}{t}\PYG{p}{)} \PYG{o}{+} 
            \PYG{l+m+mi}{2} \PYG{o}{*} \PYG{n}{np}\PYG{o}{.}\PYG{n}{random}\PYG{o}{.}\PYG{n}{randn}\PYG{p}{(}\PYG{o}{*}\PYG{n}{np}\PYG{o}{.}\PYG{n}{shape}\PYG{p}{(}\PYG{n}{t}\PYG{p}{)}\PYG{p}{)}\PYG{p}{)}
\end{sphinxVerbatim}

\sphinxAtStartPar
Uma vez que a DFT toma amostras discretas como entrada e retorna um espectro de frequencias discretas como saída, para usá\sphinxhyphen{}la em processos que são originalmente contínuos, primeiro devemos reduzir os sinais para valores discretos usando amostragem (\sphinxstyleemphasis{sampling}).

\sphinxAtStartPar
Pelo teorema da amostragem, um sinal contínuo com largura de banda \(B\) (i.e. o sinal não contém frequencias maiores do que \(B\)) pode ser completamente reconstruído a partir de amostras discretas com frequencia de amostragem \(f_s \geq 2B\). Este resultado nos diz sob quais circunstâncias podemos trabalhar com sinais discretos em vez de contínuos. Ele permite\sphinxhyphen{}nos determinar uma taxa de amostragem adequada.

\sphinxAtStartPar
Digamos que estejamos interessados em computar o espectro de frequencias deste sinal até frequencias de 30 Hz. Precisamos escolher a frequencia de amostragem \(f_s = 60\) Hz e, se quisermos obter um espectro de frequencias com resolução de \(\Delta f = 0.01\) Hz, precisamos coletar pelo menos \(N = f_s / \Delta f = 6000\) amostras, correspondendo a um período de amostragem de \(T = N/f_s = 100\) segundos.

\begin{sphinxVerbatim}[commandchars=\\\{\}]
\PYG{n}{B} \PYG{o}{=} \PYG{l+m+mf}{30.0} \PYG{c+c1}{\PYGZsh{} largura de banda [Hz]}
\PYG{n}{f\PYGZus{}s} \PYG{o}{=} \PYG{l+m+mi}{2}\PYG{o}{*}\PYG{n}{B} \PYG{c+c1}{\PYGZsh{} frequencia de amostragem [Hz]}
\PYG{n}{delta\PYGZus{}f} \PYG{o}{=} \PYG{l+m+mf}{0.01} \PYG{c+c1}{\PYGZsh{} espaçamento}
\PYG{n}{N} \PYG{o}{=} \PYG{n+nb}{int}\PYG{p}{(}\PYG{n}{f\PYGZus{}s} \PYG{o}{/} \PYG{n}{delta\PYGZus{}f}\PYG{p}{)} \PYG{c+c1}{\PYGZsh{} número de amostras}
\PYG{n}{T} \PYG{o}{=} \PYG{n}{N} \PYG{o}{/} \PYG{n}{f\PYGZus{}s} \PYG{c+c1}{\PYGZsh{} tempo total de amostragem}
\end{sphinxVerbatim}

\sphinxAtStartPar
Em seguida, amostramos o sinal em \(N\) pontos.

\begin{sphinxVerbatim}[commandchars=\\\{\}]
\PYG{n}{t} \PYG{o}{=} \PYG{n}{np}\PYG{o}{.}\PYG{n}{linspace}\PYG{p}{(}\PYG{l+m+mi}{0}\PYG{p}{,}\PYG{n}{T}\PYG{p}{,}\PYG{n}{N}\PYG{p}{)}
\PYG{n}{f\PYGZus{}t} \PYG{o}{=} \PYG{n}{amostra\PYGZus{}sinal}\PYG{p}{(}\PYG{n}{t}\PYG{p}{)}
\end{sphinxVerbatim}

\sphinxAtStartPar
O sinal é plotado como segue:

\begin{sphinxVerbatim}[commandchars=\\\{\}]
\PYG{n}{fig}\PYG{p}{,} \PYG{n}{ax} \PYG{o}{=} \PYG{n}{plt}\PYG{o}{.}\PYG{n}{subplots}\PYG{p}{(}\PYG{l+m+mi}{1}\PYG{p}{,} \PYG{l+m+mi}{2}\PYG{p}{,} \PYG{n}{figsize}\PYG{o}{=}\PYG{p}{(}\PYG{l+m+mi}{8}\PYG{p}{,} \PYG{l+m+mi}{3}\PYG{p}{)}\PYG{p}{,} \PYG{n}{sharey}\PYG{o}{=}\PYG{k+kc}{True}\PYG{p}{)}
\PYG{n}{ax}\PYG{p}{[}\PYG{l+m+mi}{0}\PYG{p}{]}\PYG{o}{.}\PYG{n}{plot}\PYG{p}{(}\PYG{n}{t}\PYG{p}{,} \PYG{n}{f\PYGZus{}t}\PYG{p}{)}
\PYG{n}{ax}\PYG{p}{[}\PYG{l+m+mi}{0}\PYG{p}{]}\PYG{o}{.}\PYG{n}{set\PYGZus{}xlabel}\PYG{p}{(}\PYG{l+s+s2}{\PYGZdq{}}\PYG{l+s+s2}{tempo (s)}\PYG{l+s+s2}{\PYGZdq{}}\PYG{p}{)}
\PYG{n}{ax}\PYG{p}{[}\PYG{l+m+mi}{0}\PYG{p}{]}\PYG{o}{.}\PYG{n}{set\PYGZus{}ylabel}\PYG{p}{(}\PYG{l+s+s2}{\PYGZdq{}}\PYG{l+s+s2}{sinal}\PYG{l+s+s2}{\PYGZdq{}}\PYG{p}{)}
\PYG{n}{ax}\PYG{p}{[}\PYG{l+m+mi}{1}\PYG{p}{]}\PYG{o}{.}\PYG{n}{plot}\PYG{p}{(}\PYG{n}{t}\PYG{p}{,} \PYG{n}{f\PYGZus{}t}\PYG{p}{)}
\PYG{n}{ax}\PYG{p}{[}\PYG{l+m+mi}{1}\PYG{p}{]}\PYG{o}{.}\PYG{n}{set\PYGZus{}xlim}\PYG{p}{(}\PYG{l+m+mi}{0}\PYG{p}{,} \PYG{l+m+mi}{5}\PYG{p}{)}
\PYG{n}{ax}\PYG{p}{[}\PYG{l+m+mi}{1}\PYG{p}{]}\PYG{o}{.}\PYG{n}{set\PYGZus{}xlabel}\PYG{p}{(}\PYG{l+s+s2}{\PYGZdq{}}\PYG{l+s+s2}{tempo (s)}\PYG{l+s+s2}{\PYGZdq{}}\PYG{p}{)}\PYG{p}{;}
\end{sphinxVerbatim}

\noindent\sphinxincludegraphics{{extra-fft_11_0}.png}

\sphinxAtStartPar
O sinal é muitor ruidoso. À direita, vemos uma plotagem ampliada. Para revelar as componentes senoidais no sinal, podemos usar a FFT para calcular o espectro do sinal, isto é, sua representação no domínio das frequencias.

\begin{sphinxVerbatim}[commandchars=\\\{\}]
\PYG{n}{F} \PYG{o}{=} \PYG{n}{fftpack}\PYG{o}{.}\PYG{n}{fft}\PYG{p}{(}\PYG{n}{f\PYGZus{}t}\PYG{p}{)}
\end{sphinxVerbatim}

\sphinxAtStartPar
\sphinxcode{\sphinxupquote{F}} contém as componentes de frequencia do espectro nas frequencias que são determinadas pela taxa de amostragem e pelo número de amostras. Ao computar estas frequencias, é conveniente usar:

\begin{sphinxVerbatim}[commandchars=\\\{\}]
\PYG{n}{f} \PYG{o}{=} \PYG{n}{fftpack}\PYG{o}{.}\PYG{n}{fftfreq}\PYG{p}{(}\PYG{n}{N}\PYG{p}{,}\PYG{l+m+mf}{1.0}\PYG{o}{/}\PYG{n}{f\PYGZus{}s}\PYG{p}{)}
\end{sphinxVerbatim}

\sphinxAtStartPar
para retornar as frequencias correspondendo a cada \sphinxstyleemphasis{bin} de frequnecia.

\sphinxAtStartPar
\sphinxcode{\sphinxupquote{f}} possui frequencias positivas e negativas. Para selecionar as positivas, fazemos:

\begin{sphinxVerbatim}[commandchars=\\\{\}]
\PYG{n}{mask} \PYG{o}{=} \PYG{n}{np}\PYG{o}{.}\PYG{n}{where}\PYG{p}{(}\PYG{n}{f} \PYG{o}{\PYGZgt{}}\PYG{o}{=} \PYG{l+m+mi}{0}\PYG{p}{)}
\end{sphinxVerbatim}

\sphinxAtStartPar
Em seguida, podemos plotar o espectro com as componentes positivas de frequencia (em escala logarítmica para contrastar sinal e ruído), bem como os dois picos em 1 Hz e 22 Hz que se sobressaem. Ambos correspondem às componentes senoidais existentes no sinal.

\sphinxAtStartPar
Embora o ruído esconda as componentes senoidais no sinal em seu domínio temporal, podemos vê\sphinxhyphen{}las claramente presentes no domínio das frequencias.

\begin{sphinxVerbatim}[commandchars=\\\{\}]
\PYG{n}{fig}\PYG{p}{,} \PYG{n}{axes} \PYG{o}{=} \PYG{n}{plt}\PYG{o}{.}\PYG{n}{subplots}\PYG{p}{(}\PYG{l+m+mi}{3}\PYG{p}{,} \PYG{l+m+mi}{1}\PYG{p}{,} \PYG{n}{figsize}\PYG{o}{=}\PYG{p}{(}\PYG{l+m+mi}{8}\PYG{p}{,} \PYG{l+m+mi}{6}\PYG{p}{)}\PYG{p}{)}
\PYG{n}{axes}\PYG{p}{[}\PYG{l+m+mi}{0}\PYG{p}{]}\PYG{o}{.}\PYG{n}{plot}\PYG{p}{(}\PYG{n}{f}\PYG{p}{[}\PYG{n}{mask}\PYG{p}{]}\PYG{p}{,} \PYG{n}{np}\PYG{o}{.}\PYG{n}{log}\PYG{p}{(}\PYG{n+nb}{abs}\PYG{p}{(}\PYG{n}{F}\PYG{p}{[}\PYG{n}{mask}\PYG{p}{]}\PYG{p}{)}\PYG{p}{)}\PYG{p}{,} \PYG{n}{label}\PYG{o}{=}\PYG{l+s+s2}{\PYGZdq{}}\PYG{l+s+s2}{real}\PYG{l+s+s2}{\PYGZdq{}}\PYG{p}{)}  
\PYG{n}{axes}\PYG{p}{[}\PYG{l+m+mi}{0}\PYG{p}{]}\PYG{o}{.}\PYG{n}{plot}\PYG{p}{(}\PYG{n}{B}\PYG{p}{,} \PYG{l+m+mi}{0}\PYG{p}{,} \PYG{l+s+s1}{\PYGZsq{}}\PYG{l+s+s1}{r*}\PYG{l+s+s1}{\PYGZsq{}}\PYG{p}{,} \PYG{n}{markersize}\PYG{o}{=}\PYG{l+m+mi}{10}\PYG{p}{)}
\PYG{n}{axes}\PYG{p}{[}\PYG{l+m+mi}{0}\PYG{p}{]}\PYG{o}{.}\PYG{n}{set\PYGZus{}ylabel}\PYG{p}{(}\PYG{l+s+s2}{\PYGZdq{}}\PYG{l+s+s2}{\PYGZdl{}}\PYG{l+s+s2}{\PYGZbs{}}\PYG{l+s+s2}{log(|F|)\PYGZdl{}}\PYG{l+s+s2}{\PYGZdq{}}\PYG{p}{,} \PYG{n}{fontsize}\PYG{o}{=}\PYG{l+m+mi}{14}\PYG{p}{)}
\PYG{n}{axes}\PYG{p}{[}\PYG{l+m+mi}{1}\PYG{p}{]}\PYG{o}{.}\PYG{n}{plot}\PYG{p}{(}\PYG{n}{f}\PYG{p}{[}\PYG{n}{mask}\PYG{p}{]}\PYG{p}{,} \PYG{n+nb}{abs}\PYG{p}{(}\PYG{n}{F}\PYG{p}{[}\PYG{n}{mask}\PYG{p}{]}\PYG{p}{)}\PYG{o}{/}\PYG{n}{N}\PYG{p}{,} \PYG{n}{label}\PYG{o}{=}\PYG{l+s+s2}{\PYGZdq{}}\PYG{l+s+s2}{real}\PYG{l+s+s2}{\PYGZdq{}}\PYG{p}{)}
\PYG{n}{axes}\PYG{p}{[}\PYG{l+m+mi}{1}\PYG{p}{]}\PYG{o}{.}\PYG{n}{set\PYGZus{}xlim}\PYG{p}{(}\PYG{l+m+mi}{0}\PYG{p}{,} \PYG{l+m+mi}{2}\PYG{p}{)}
\PYG{n}{axes}\PYG{p}{[}\PYG{l+m+mi}{1}\PYG{p}{]}\PYG{o}{.}\PYG{n}{set\PYGZus{}ylabel}\PYG{p}{(}\PYG{l+s+s2}{\PYGZdq{}}\PYG{l+s+s2}{\PYGZdl{}|F|/N\PYGZdl{}}\PYG{l+s+s2}{\PYGZdq{}}\PYG{p}{,} \PYG{n}{fontsize}\PYG{o}{=}\PYG{l+m+mi}{14}\PYG{p}{)}
\PYG{n}{axes}\PYG{p}{[}\PYG{l+m+mi}{2}\PYG{p}{]}\PYG{o}{.}\PYG{n}{plot}\PYG{p}{(}\PYG{n}{f}\PYG{p}{[}\PYG{n}{mask}\PYG{p}{]}\PYG{p}{,} \PYG{n+nb}{abs}\PYG{p}{(}\PYG{n}{F}\PYG{p}{[}\PYG{n}{mask}\PYG{p}{]}\PYG{p}{)}\PYG{o}{/}\PYG{n}{N}\PYG{p}{,} \PYG{n}{label}\PYG{o}{=}\PYG{l+s+s2}{\PYGZdq{}}\PYG{l+s+s2}{real}\PYG{l+s+s2}{\PYGZdq{}}\PYG{p}{)}
\PYG{n}{axes}\PYG{p}{[}\PYG{l+m+mi}{2}\PYG{p}{]}\PYG{o}{.}\PYG{n}{set\PYGZus{}xlim}\PYG{p}{(}\PYG{l+m+mi}{21}\PYG{p}{,} \PYG{l+m+mi}{23}\PYG{p}{)}
\PYG{n}{axes}\PYG{p}{[}\PYG{l+m+mi}{2}\PYG{p}{]}\PYG{o}{.}\PYG{n}{set\PYGZus{}xlabel}\PYG{p}{(}\PYG{l+s+s2}{\PYGZdq{}}\PYG{l+s+s2}{frequencia (Hz)}\PYG{l+s+s2}{\PYGZdq{}}\PYG{p}{,} \PYG{n}{fontsize}\PYG{o}{=}\PYG{l+m+mi}{14}\PYG{p}{)}
\PYG{n}{axes}\PYG{p}{[}\PYG{l+m+mi}{2}\PYG{p}{]}\PYG{o}{.}\PYG{n}{set\PYGZus{}ylabel}\PYG{p}{(}\PYG{l+s+s2}{\PYGZdq{}}\PYG{l+s+s2}{\PYGZdl{}|F|/N\PYGZdl{}}\PYG{l+s+s2}{\PYGZdq{}}\PYG{p}{,} \PYG{n}{fontsize}\PYG{o}{=}\PYG{l+m+mi}{14}\PYG{p}{)}\PYG{p}{;}
\end{sphinxVerbatim}

\noindent\sphinxincludegraphics{{extra-fft_20_0}.png}


\subsection{Filros no domínio das frequencias}
\label{\detokenize{extra/extra-fft:filros-no-dominio-das-frequencias}}
\sphinxAtStartPar
Podemos também computar o sinal no domínio temporal a partir da representação no domínio das frequencias usando a função FFT inversa.

\sphinxAtStartPar
Por exemplo, a aplicação de um filtro passa\sphinxhyphen{}baixa (\sphinxstyleemphasis{low pass}) de 2 Hz, i.e. que suprime componentes com frequencia maiores do que 2 Hz nos leva a:

\begin{sphinxVerbatim}[commandchars=\\\{\}]
\PYG{n}{F\PYGZus{}filtered} \PYG{o}{=} \PYG{n}{F} \PYG{o}{*} \PYG{p}{(}\PYG{n+nb}{abs}\PYG{p}{(}\PYG{n}{f}\PYG{p}{)} \PYG{o}{\PYGZlt{}} \PYG{l+m+mi}{2}\PYG{p}{)}
\PYG{n}{f\PYGZus{}t\PYGZus{}filtered} \PYG{o}{=} \PYG{n}{fftpack}\PYG{o}{.}\PYG{n}{ifft}\PYG{p}{(}\PYG{n}{F\PYGZus{}filtered}\PYG{p}{)}
\end{sphinxVerbatim}

\sphinxAtStartPar
O cálculo da FFT inversa para o sinal filtrado resulta em um sinal no domínio temporal onde as oscilações de alta frequencia estão ausentes. Este exemplo resume a essência de muitos filtros aplicáveis ao domínio de frequencias.

\begin{sphinxVerbatim}[commandchars=\\\{\}]
\PYG{n}{fig}\PYG{p}{,} \PYG{n}{ax} \PYG{o}{=} \PYG{n}{plt}\PYG{o}{.}\PYG{n}{subplots}\PYG{p}{(}\PYG{n}{figsize}\PYG{o}{=}\PYG{p}{(}\PYG{l+m+mi}{8}\PYG{p}{,} \PYG{l+m+mi}{3}\PYG{p}{)}\PYG{p}{)}
\PYG{n}{ax}\PYG{o}{.}\PYG{n}{plot}\PYG{p}{(}\PYG{n}{t}\PYG{p}{,} \PYG{n}{f\PYGZus{}t}\PYG{p}{,} \PYG{n}{label}\PYG{o}{=}\PYG{l+s+s1}{\PYGZsq{}}\PYG{l+s+s1}{original}\PYG{l+s+s1}{\PYGZsq{}}\PYG{p}{)}
\PYG{n}{ax}\PYG{o}{.}\PYG{n}{plot}\PYG{p}{(}\PYG{n}{t}\PYG{p}{,} \PYG{n}{f\PYGZus{}t\PYGZus{}filtered}\PYG{o}{.}\PYG{n}{real}\PYG{p}{,} \PYG{n}{color}\PYG{o}{=}\PYG{l+s+s2}{\PYGZdq{}}\PYG{l+s+s2}{black}\PYG{l+s+s2}{\PYGZdq{}}\PYG{p}{,} \PYG{n}{lw}\PYG{o}{=}\PYG{l+m+mi}{3}\PYG{p}{,} \PYG{n}{label}\PYG{o}{=}\PYG{l+s+s1}{\PYGZsq{}}\PYG{l+s+s1}{filtrado}\PYG{l+s+s1}{\PYGZsq{}}\PYG{p}{)}
\PYG{n}{ax}\PYG{o}{.}\PYG{n}{set\PYGZus{}xlim}\PYG{p}{(}\PYG{l+m+mi}{0}\PYG{p}{,} \PYG{l+m+mi}{10}\PYG{p}{)}
\PYG{n}{ax}\PYG{o}{.}\PYG{n}{set\PYGZus{}xlabel}\PYG{p}{(}\PYG{l+s+s2}{\PYGZdq{}}\PYG{l+s+s2}{tempo (s)}\PYG{l+s+s2}{\PYGZdq{}}\PYG{p}{)}
\PYG{n}{ax}\PYG{o}{.}\PYG{n}{set\PYGZus{}ylabel}\PYG{p}{(}\PYG{l+s+s2}{\PYGZdq{}}\PYG{l+s+s2}{sinal}\PYG{l+s+s2}{\PYGZdq{}}\PYG{p}{)}
\PYG{n}{ax}\PYG{o}{.}\PYG{n}{legend}\PYG{p}{(}\PYG{p}{)}\PYG{p}{;}
\end{sphinxVerbatim}

\noindent\sphinxincludegraphics{{extra-fft_25_0}.png}


\chapter{Otimização de código}
\label{\detokenize{extra/extra-numba:otimizacao-de-codigo}}\label{\detokenize{extra/extra-numba::doc}}
\sphinxAtStartPar
Python possui diversas bibliotecas para computação numérica que são eficientes e de alto desempenho, tais como \sphinxstyleemphasis{numpy} e \sphinxstyleemphasis{scipy}. A computação de alta performance é atingida não por Python puro, mas pelo aproveitamento de bibliotecas que são compiladas externamente.

\sphinxAtStartPar
Às vezes, temos necessidade de desenvolver código do zero usando puramente Python. Entretanto, essa escolha pode levar a códigos lentos. A solução é escrever rotinas em linguages externas, tais como C, C++ ou Fortran que façam os cálculos que consomem mais tempo e criar interfaces entre elas e o código Puython.

\sphinxAtStartPar
Existem métodos que permitem criar módulos de extensão para Python, tais como:
\begin{itemize}
\item {} 
\sphinxAtStartPar
módulo \sphinxstyleemphasis{ctypes}

\item {} 
\sphinxAtStartPar
API Python para C

\item {} 
\sphinxAtStartPar
CFFI (C foreign function interface)

\end{itemize}

\sphinxAtStartPar
Embora úteis, todas elas exigem conhecimento aprofundado de outras linguagens e são mais úteis para códigos\sphinxhyphen{}fonte já escritos nessas linguagens. Algumas alternativas de desenvolvimento que se aproximam de Python que valem a pena ser consideradas antes de partirmos para uma implementação direta em linguagem complicada existem. Duas delas são: \sphinxstyleemphasis{numba} e \sphinxstyleemphasis{cython}.


\section{\sphinxstyleemphasis{numba}}
\label{\detokenize{extra/extra-numba:numba}}
\sphinxAtStartPar
\sphinxhref{https://numba.pydata.org}{{[}Numba{]}} é um compilador \sphinxstyleemphasis{just\sphinxhyphen{}in\sphinxhyphen{}time} (JIT) para código Python usando \sphinxstyleemphasis{numpy} que produz código de máquina executável de forma mais eficiente do que o código Python original. Para conseguir isso, \sphinxstyleemphasis{numba} aproveita o conjunto de compiladores \sphinxhref{http://llvm.org}{{[}LLVM{]}}, que se tornou muito popular nos últimos anos por seu design e interface modular e reutilizável.


\section{\sphinxstyleemphasis{cython}}
\label{\detokenize{extra/extra-numba:cython}}
\sphinxAtStartPar
\sphinxhref{http://cython.org}{{[}Cython{]}} é um superconjunto da linguagem Python que pode ser traduzido automaticamente para C ou C++ e compilado em um código de máquina executável de modo muito mais rápido do que o código Python. Cython é amplamente utilizado em projetos Python orientados computacionalmente para acelerar partes críticas de tempo de um código que é escrita em Python. Várias bibliotecas dependem muito do Cython. Isso inclui NumPy, SciPy, Pandas e scikit\sphinxhyphen{}learn, apenas para mencionar algumas.

\sphinxAtStartPar
Veremos como usar \sphinxstyleemphasis{numba} e \sphinxstyleemphasis{cython} para acelerar códigos originalmente escritos em Python. É recomendado que se faça um perfilamento de código com o módulo \sphinxcode{\sphinxupquote{cProfile}} ou outras ferramentas para identificar gargalos de cálculo antes de tentarmos otimizar algo.


\section{Exemplos com numba}
\label{\detokenize{extra/extra-numba:exemplos-com-numba}}
\begin{sphinxVerbatim}[commandchars=\\\{\}]
\PYG{k+kn}{import} \PYG{n+nn}{numba}
\PYG{k+kn}{import} \PYG{n+nn}{numpy} \PYG{k}{as} \PYG{n+nn}{np}
\PYG{k+kn}{import} \PYG{n+nn}{matplotlib}\PYG{n+nn}{.}\PYG{n+nn}{pyplot} \PYG{k}{as} \PYG{n+nn}{plt}
\PYG{o}{\PYGZpc{}}\PYG{k}{matplotlib} inline
\end{sphinxVerbatim}

\sphinxAtStartPar
Não precisamos alterar o código\sphinxhyphen{}alvo a ser acelerado. Basta usar o JIT do Numba como “decorador” \sphinxcode{\sphinxupquote{@numba.jit}} para uma função.


\subsection{Soma de elementos em um array}
\label{\detokenize{extra/extra-numba:soma-de-elementos-em-um-array}}
\sphinxAtStartPar
Consideramos o seguinte código simples.

\begin{sphinxVerbatim}[commandchars=\\\{\}]
\PYG{k}{def} \PYG{n+nf}{py\PYGZus{}sum}\PYG{p}{(}\PYG{n}{val}\PYG{p}{)}\PYG{p}{:}
    \PYG{n}{s} \PYG{o}{=} \PYG{l+m+mi}{0}
    \PYG{k}{for} \PYG{n}{v} \PYG{o+ow}{in} \PYG{n}{val}\PYG{p}{:}
        \PYG{n}{s} \PYG{o}{+}\PYG{o}{=} \PYG{n}{v}
    \PYG{k}{return} \PYG{n}{s}
\end{sphinxVerbatim}

\begin{sphinxVerbatim}[commandchars=\\\{\}]
\PYG{n}{val} \PYG{o}{=} \PYG{n}{np}\PYG{o}{.}\PYG{n}{random}\PYG{o}{.}\PYG{n}{rand}\PYG{p}{(}\PYG{l+m+mi}{50000}\PYG{p}{)} \PYG{c+c1}{\PYGZsh{} 50000 aleatórios}
\PYG{o}{\PYGZpc{}}\PYG{k}{timeit} py\PYGZus{}sum(val) 
\end{sphinxVerbatim}

\begin{sphinxVerbatim}[commandchars=\\\{\}]
6.91 ms ± 1.19 ms per loop (mean ± std. dev. of 7 runs, 100 loops each)
\end{sphinxVerbatim}

\sphinxAtStartPar
A versão vetorizada da somatória é mais rápida.

\begin{sphinxVerbatim}[commandchars=\\\{\}]
\PYG{o}{\PYGZpc{}}\PYG{k}{timeit} np.sum(val)
\end{sphinxVerbatim}

\begin{sphinxVerbatim}[commandchars=\\\{\}]
22.6 µs ± 119 ns per loop (mean ± std. dev. of 7 runs, 10000 loops each)
\end{sphinxVerbatim}

\sphinxAtStartPar
Teste da função. Se assert não lança erro, temos a condição válida.

\begin{sphinxVerbatim}[commandchars=\\\{\}]
\PYG{k}{assert} \PYG{n+nb}{abs}\PYG{p}{(}\PYG{n}{py\PYGZus{}sum}\PYG{p}{(}\PYG{n}{val}\PYG{p}{)} \PYG{o}{\PYGZhy{}} \PYG{n}{np}\PYG{o}{.}\PYG{n}{sum}\PYG{p}{(}\PYG{n}{val}\PYG{p}{)}\PYG{p}{)} \PYG{o}{\PYGZlt{}} \PYG{l+m+mf}{1e\PYGZhy{}9}
\end{sphinxVerbatim}

\sphinxAtStartPar
Tratando com numba.

\begin{sphinxVerbatim}[commandchars=\\\{\}]
\PYG{n+nd}{@numba}\PYG{o}{.}\PYG{n}{jit}
\PYG{k}{def} \PYG{n+nf}{jit\PYGZus{}sum}\PYG{p}{(}\PYG{n}{val}\PYG{p}{)}\PYG{p}{:}
    \PYG{n}{s} \PYG{o}{=} \PYG{l+m+mi}{0}
    \PYG{k}{for} \PYG{n}{v} \PYG{o+ow}{in} \PYG{n}{val}\PYG{p}{:}
        \PYG{n}{s} \PYG{o}{+}\PYG{o}{=} \PYG{n}{v}
    \PYG{k}{return} \PYG{n}{s}
\end{sphinxVerbatim}

\begin{sphinxVerbatim}[commandchars=\\\{\}]
\PYG{k}{assert} \PYG{n+nb}{abs}\PYG{p}{(}\PYG{n}{jit\PYGZus{}sum}\PYG{p}{(}\PYG{n}{val}\PYG{p}{)} \PYG{o}{\PYGZhy{}} \PYG{n}{np}\PYG{o}{.}\PYG{n}{sum}\PYG{p}{(}\PYG{n}{val}\PYG{p}{)}\PYG{p}{)} \PYG{o}{\PYGZlt{}} \PYG{l+m+mf}{1e\PYGZhy{}9} \PYG{c+c1}{\PYGZsh{} função numba produz mesmo valor}
\end{sphinxVerbatim}

\begin{sphinxVerbatim}[commandchars=\\\{\}]
\PYG{o}{\PYGZpc{}}\PYG{k}{timeit} jit\PYGZus{}sum(val)
\end{sphinxVerbatim}

\begin{sphinxVerbatim}[commandchars=\\\{\}]
46.5 µs ± 151 ns per loop (mean ± std. dev. of 7 runs, 10000 loops each)
\end{sphinxVerbatim}

\sphinxAtStartPar
Comparando isto com a implementação pura, temos uma grande diferença.


\subsection{Soma cumulativa}
\label{\detokenize{extra/extra-numba:soma-cumulativa}}
\begin{sphinxVerbatim}[commandchars=\\\{\}]
\PYG{k}{def} \PYG{n+nf}{py\PYGZus{}cumsum}\PYG{p}{(}\PYG{n}{val}\PYG{p}{)}\PYG{p}{:}
    \PYG{n}{o} \PYG{o}{=} \PYG{n}{np}\PYG{o}{.}\PYG{n}{zeros\PYGZus{}like}\PYG{p}{(}\PYG{n}{val}\PYG{p}{)}
    \PYG{n}{s}\PYG{o}{=} \PYG{l+m+mi}{0}
    \PYG{k}{for} \PYG{n}{n} \PYG{o+ow}{in} \PYG{n+nb}{range}\PYG{p}{(}\PYG{n+nb}{len}\PYG{p}{(}\PYG{n}{val}\PYG{p}{)}\PYG{p}{)}\PYG{p}{:}
        \PYG{n}{s} \PYG{o}{+}\PYG{o}{=} \PYG{n}{val}\PYG{p}{[}\PYG{n}{n}\PYG{p}{]}
        \PYG{n}{o}\PYG{p}{[}\PYG{n}{n}\PYG{p}{]} \PYG{o}{=} \PYG{n}{s}
    \PYG{k}{return} \PYG{n}{o}
\end{sphinxVerbatim}

\begin{sphinxVerbatim}[commandchars=\\\{\}]
\PYG{o}{\PYGZpc{}}\PYG{k}{timeit} py\PYGZus{}cumsum(val)
\end{sphinxVerbatim}

\begin{sphinxVerbatim}[commandchars=\\\{\}]
16.8 ms ± 3.18 ms per loop (mean ± std. dev. of 7 runs, 10 loops each)
\end{sphinxVerbatim}

\begin{sphinxVerbatim}[commandchars=\\\{\}]
\PYG{o}{\PYGZpc{}}\PYG{k}{timeit} np.cumsum(val)
\end{sphinxVerbatim}

\begin{sphinxVerbatim}[commandchars=\\\{\}]
149 µs ± 4.16 µs per loop (mean ± std. dev. of 7 runs, 10000 loops each)
\end{sphinxVerbatim}

\sphinxAtStartPar
Utilizando numba:

\begin{sphinxVerbatim}[commandchars=\\\{\}]
\PYG{n+nd}{@numba}\PYG{o}{.}\PYG{n}{jit}
\PYG{k}{def} \PYG{n+nf}{jit\PYGZus{}cumsum}\PYG{p}{(}\PYG{n}{val}\PYG{p}{)}\PYG{p}{:}
    \PYG{n}{o} \PYG{o}{=} \PYG{n}{np}\PYG{o}{.}\PYG{n}{zeros\PYGZus{}like}\PYG{p}{(}\PYG{n}{val}\PYG{p}{)}
    \PYG{n}{s}\PYG{o}{=} \PYG{l+m+mi}{0}
    \PYG{k}{for} \PYG{n}{n} \PYG{o+ow}{in} \PYG{n+nb}{range}\PYG{p}{(}\PYG{n+nb}{len}\PYG{p}{(}\PYG{n}{val}\PYG{p}{)}\PYG{p}{)}\PYG{p}{:}
        \PYG{n}{s} \PYG{o}{+}\PYG{o}{=} \PYG{n}{val}\PYG{p}{[}\PYG{n}{n}\PYG{p}{]}
        \PYG{n}{o}\PYG{p}{[}\PYG{n}{n}\PYG{p}{]} \PYG{o}{=} \PYG{n}{s}
    \PYG{k}{return} \PYG{n}{o}
\end{sphinxVerbatim}

\begin{sphinxVerbatim}[commandchars=\\\{\}]
\PYG{o}{\PYGZpc{}}\PYG{k}{timeit} jit\PYGZus{}cumsum(val)
\end{sphinxVerbatim}

\begin{sphinxVerbatim}[commandchars=\\\{\}]
62.3 µs ± 1.31 µs per loop (mean ± std. dev. of 7 runs, 10000 loops each)
\end{sphinxVerbatim}


\subsection{Fractal de Julia}
\label{\detokenize{extra/extra-numba:fractal-de-julia}}
\sphinxAtStartPar
O fractal de Julia exige um número variável de iterações para cada elemento de uma matriz com coordenadas no plano complexo:
\begin{itemize}
\item {} 
\sphinxAtStartPar
Um ponto \(z\) no plano complexo pertence ao conjunto de Julia se a fórmula de iteração \(z \leftarrow z^2 + c\) não diverge após um número grande de iterações.

\end{itemize}

\begin{sphinxVerbatim}[commandchars=\\\{\}]
\PYG{k}{def} \PYG{n+nf}{py\PYGZus{}julia\PYGZus{}fractal}\PYG{p}{(}\PYG{n}{z\PYGZus{}re}\PYG{p}{,} \PYG{n}{z\PYGZus{}im}\PYG{p}{,} \PYG{n}{j}\PYG{p}{)}\PYG{p}{:}
    \PYG{k}{for} \PYG{n}{m} \PYG{o+ow}{in} \PYG{n+nb}{range}\PYG{p}{(}\PYG{n+nb}{len}\PYG{p}{(}\PYG{n}{z\PYGZus{}re}\PYG{p}{)}\PYG{p}{)}\PYG{p}{:}
        \PYG{k}{for} \PYG{n}{n} \PYG{o+ow}{in} \PYG{n+nb}{range}\PYG{p}{(}\PYG{n+nb}{len}\PYG{p}{(}\PYG{n}{z\PYGZus{}im}\PYG{p}{)}\PYG{p}{)}\PYG{p}{:}
            \PYG{n}{z} \PYG{o}{=} \PYG{n}{z\PYGZus{}re}\PYG{p}{[}\PYG{n}{m}\PYG{p}{]} \PYG{o}{+} \PYG{l+m+mi}{1}\PYG{n}{j} \PYG{o}{*} \PYG{n}{z\PYGZus{}im}\PYG{p}{[}\PYG{n}{n}\PYG{p}{]} 
            \PYG{k}{for} \PYG{n}{t} \PYG{o+ow}{in} \PYG{n+nb}{range}\PYG{p}{(}\PYG{l+m+mi}{256}\PYG{p}{)}\PYG{p}{:}
                \PYG{n}{z} \PYG{o}{=} \PYG{n}{z} \PYG{o}{*}\PYG{o}{*} \PYG{l+m+mi}{2} \PYG{o}{\PYGZhy{}} \PYG{l+m+mf}{0.05} \PYG{o}{+} \PYG{l+m+mf}{0.68}\PYG{n}{j} 
                \PYG{k}{if} \PYG{n}{np}\PYG{o}{.}\PYG{n}{abs}\PYG{p}{(}\PYG{n}{z}\PYG{p}{)} \PYG{o}{\PYGZgt{}} \PYG{l+m+mf}{2.0}\PYG{p}{:}
                    \PYG{n}{j}\PYG{p}{[}\PYG{n}{m}\PYG{p}{,} \PYG{n}{n}\PYG{p}{]} \PYG{o}{=} \PYG{n}{t} 
                    \PYG{k}{break}
\end{sphinxVerbatim}

\sphinxAtStartPar
Decorador:

\begin{sphinxVerbatim}[commandchars=\\\{\}]
\PYG{n}{jit\PYGZus{}julia\PYGZus{}fractal} \PYG{o}{=} \PYG{n}{numba}\PYG{o}{.}\PYG{n}{jit}\PYG{p}{(}\PYG{n}{nopython}\PYG{o}{=}\PYG{k+kc}{True}\PYG{p}{)}\PYG{p}{(}\PYG{n}{py\PYGZus{}julia\PYGZus{}fractal}\PYG{p}{)} \PYG{c+c1}{\PYGZsh{} nopython : }
\end{sphinxVerbatim}

\sphinxAtStartPar
Chamada da função:

\begin{sphinxVerbatim}[commandchars=\\\{\}]
\PYG{n}{N} \PYG{o}{=} \PYG{l+m+mi}{1024}
\PYG{n}{j} \PYG{o}{=} \PYG{n}{np}\PYG{o}{.}\PYG{n}{zeros}\PYG{p}{(}\PYG{p}{(}\PYG{n}{N}\PYG{p}{,} \PYG{n}{N}\PYG{p}{)}\PYG{p}{,} \PYG{n}{np}\PYG{o}{.}\PYG{n}{int64}\PYG{p}{)}
\PYG{n}{z\PYGZus{}real} \PYG{o}{=} \PYG{n}{np}\PYG{o}{.}\PYG{n}{linspace}\PYG{p}{(}\PYG{o}{\PYGZhy{}}\PYG{l+m+mf}{1.5}\PYG{p}{,} \PYG{l+m+mf}{1.5}\PYG{p}{,} \PYG{n}{N}\PYG{p}{)} 
\PYG{n}{z\PYGZus{}imag} \PYG{o}{=} \PYG{n}{np}\PYG{o}{.}\PYG{n}{linspace}\PYG{p}{(}\PYG{o}{\PYGZhy{}}\PYG{l+m+mf}{1.5}\PYG{p}{,} \PYG{l+m+mf}{1.5}\PYG{p}{,} \PYG{n}{N}\PYG{p}{)} 
\PYG{n}{jit\PYGZus{}julia\PYGZus{}fractal}\PYG{p}{(}\PYG{n}{z\PYGZus{}real}\PYG{p}{,} \PYG{n}{z\PYGZus{}imag}\PYG{p}{,} \PYG{n}{j}\PYG{p}{)}
\end{sphinxVerbatim}

\sphinxAtStartPar
Visualização do fractal gerado por Numba:

\begin{sphinxVerbatim}[commandchars=\\\{\}]
\PYG{n}{fig}\PYG{p}{,} \PYG{n}{ax} \PYG{o}{=} \PYG{n}{plt}\PYG{o}{.}\PYG{n}{subplots}\PYG{p}{(}\PYG{n}{figsize}\PYG{o}{=}\PYG{p}{(}\PYG{l+m+mi}{8}\PYG{p}{,} \PYG{l+m+mi}{8}\PYG{p}{)}\PYG{p}{)}
\PYG{n}{ax}\PYG{o}{.}\PYG{n}{imshow}\PYG{p}{(}\PYG{n}{j}\PYG{p}{,} \PYG{n}{cmap}\PYG{o}{=}\PYG{n}{plt}\PYG{o}{.}\PYG{n}{cm}\PYG{o}{.}\PYG{n}{RdBu\PYGZus{}r}\PYG{p}{,} \PYG{n}{extent}\PYG{o}{=}\PYG{p}{[}\PYG{o}{\PYGZhy{}}\PYG{l+m+mf}{1.5}\PYG{p}{,} \PYG{l+m+mf}{1.5}\PYG{p}{,} \PYG{o}{\PYGZhy{}}\PYG{l+m+mf}{1.5}\PYG{p}{,} \PYG{l+m+mf}{1.5}\PYG{p}{]}\PYG{p}{)}
\PYG{n}{ax}\PYG{o}{.}\PYG{n}{set\PYGZus{}xlabel}\PYG{p}{(}\PYG{l+s+s2}{\PYGZdq{}}\PYG{l+s+s2}{\PYGZdl{}}\PYG{l+s+s2}{\PYGZbs{}}\PYG{l+s+s2}{mathrm}\PYG{l+s+si}{\PYGZob{}Re\PYGZcb{}}\PYG{l+s+s2}{(z)\PYGZdl{}}\PYG{l+s+s2}{\PYGZdq{}}\PYG{p}{,} \PYG{n}{fontsize}\PYG{o}{=}\PYG{l+m+mi}{18}\PYG{p}{)}
\PYG{n}{ax}\PYG{o}{.}\PYG{n}{set\PYGZus{}ylabel}\PYG{p}{(}\PYG{l+s+s2}{\PYGZdq{}}\PYG{l+s+s2}{\PYGZdl{}}\PYG{l+s+s2}{\PYGZbs{}}\PYG{l+s+s2}{mathrm}\PYG{l+s+si}{\PYGZob{}Im\PYGZcb{}}\PYG{l+s+s2}{(z)\PYGZdl{}}\PYG{l+s+s2}{\PYGZdq{}}\PYG{p}{,} \PYG{n}{fontsize}\PYG{o}{=}\PYG{l+m+mi}{18}\PYG{p}{)}\PYG{p}{;}
\end{sphinxVerbatim}

\noindent\sphinxincludegraphics{{extra-numba_30_0}.png}

\sphinxAtStartPar
Comparação do tempo em cada chamada:

\begin{sphinxVerbatim}[commandchars=\\\{\}]
\PYG{o}{\PYGZpc{}}\PYG{k}{timeit} py\PYGZus{}julia\PYGZus{}fractal(z\PYGZus{}real, z\PYGZus{}imag, j)
\end{sphinxVerbatim}

\begin{sphinxVerbatim}[commandchars=\\\{\}]
56.7 s ± 2.22 s per loop (mean ± std. dev. of 7 runs, 1 loop each)
\end{sphinxVerbatim}

\begin{sphinxVerbatim}[commandchars=\\\{\}]
\PYG{o}{\PYGZpc{}}\PYG{k}{timeit} jit\PYGZus{}julia\PYGZus{}fractal(z\PYGZus{}real, z\PYGZus{}imag, j)
\end{sphinxVerbatim}

\begin{sphinxVerbatim}[commandchars=\\\{\}]
141 ms ± 6.84 ms per loop (mean ± std. dev. of 7 runs, 10 loops each)
\end{sphinxVerbatim}


\subsection{\sphinxstyleliteralintitle{\sphinxupquote{numba.vectorize}}}
\label{\detokenize{extra/extra-numba:numba-vectorize}}
\sphinxAtStartPar
Aqui, vamos construir a função de Heaviside:

\begin{sphinxVerbatim}[commandchars=\\\{\}]
\PYG{k}{def} \PYG{n+nf}{py\PYGZus{}Heaviside}\PYG{p}{(}\PYG{n}{x}\PYG{p}{)}\PYG{p}{:}
    \PYG{k}{if} \PYG{n}{x} \PYG{o}{==} \PYG{l+m+mf}{0.0}\PYG{p}{:} 
        \PYG{k}{return} \PYG{l+m+mf}{0.5}
    \PYG{k}{if} \PYG{n}{x} \PYG{o}{\PYGZlt{}} \PYG{l+m+mf}{0.0}\PYG{p}{:} 
        \PYG{k}{return} \PYG{l+m+mf}{0.0}    
    \PYG{k}{else}\PYG{p}{:}
        \PYG{k}{return} \PYG{l+m+mf}{1.0}
\end{sphinxVerbatim}

\begin{sphinxVerbatim}[commandchars=\\\{\}]
\PYG{n}{x} \PYG{o}{=} \PYG{n}{np}\PYG{o}{.}\PYG{n}{linspace}\PYG{p}{(}\PYG{o}{\PYGZhy{}}\PYG{l+m+mi}{2}\PYG{p}{,} \PYG{l+m+mi}{2}\PYG{p}{,} \PYG{l+m+mi}{50001}\PYG{p}{)}
\PYG{o}{\PYGZpc{}}\PYG{k}{timeit} [py\PYGZus{}Heaviside(xx) for xx in x]
\end{sphinxVerbatim}

\begin{sphinxVerbatim}[commandchars=\\\{\}]
17.2 ms ± 443 µs per loop (mean ± std. dev. of 7 runs, 10 loops each)
\end{sphinxVerbatim}

\sphinxAtStartPar
Para ter a função aplicada elemento a elemento em um array, fazemos:

\begin{sphinxVerbatim}[commandchars=\\\{\}]
\PYG{n}{np\PYGZus{}vec\PYGZus{}Heaviside} \PYG{o}{=} \PYG{n}{np}\PYG{o}{.}\PYG{n}{vectorize}\PYG{p}{(}\PYG{n}{py\PYGZus{}Heaviside}\PYG{p}{)}
\PYG{n}{np\PYGZus{}vec\PYGZus{}Heaviside}\PYG{p}{(}\PYG{n}{x}\PYG{p}{)}
\end{sphinxVerbatim}

\begin{sphinxVerbatim}[commandchars=\\\{\}]
array([0., 0., 0., ..., 1., 1., 1.])
\end{sphinxVerbatim}

\sphinxAtStartPar
A função \sphinxcode{\sphinxupquote{vectorize}} não resolve o problema com a performance.

\begin{sphinxVerbatim}[commandchars=\\\{\}]
\PYG{o}{\PYGZpc{}}\PYG{k}{timeit} np\PYGZus{}vec\PYGZus{}Heaviside(x)
\end{sphinxVerbatim}

\begin{sphinxVerbatim}[commandchars=\\\{\}]
8.25 ms ± 283 µs per loop (mean ± std. dev. of 7 runs, 100 loops each)
\end{sphinxVerbatim}

\sphinxAtStartPar
Melhor performance é atingida com:

\begin{sphinxVerbatim}[commandchars=\\\{\}]
\PYG{k}{def} \PYG{n+nf}{np\PYGZus{}Heaviside}\PYG{p}{(}\PYG{n}{x}\PYG{p}{)}\PYG{p}{:}
    \PYG{k}{return} \PYG{p}{(}\PYG{n}{x} \PYG{o}{\PYGZgt{}} \PYG{l+m+mf}{0.0}\PYG{p}{)} \PYG{o}{+} \PYG{p}{(}\PYG{n}{x} \PYG{o}{==} \PYG{l+m+mf}{0.0}\PYG{p}{)}\PYG{o}{/}\PYG{l+m+mf}{2.0}
\end{sphinxVerbatim}

\begin{sphinxVerbatim}[commandchars=\\\{\}]
\PYG{o}{\PYGZpc{}}\PYG{k}{timeit} np\PYGZus{}Heaviside(x)
\end{sphinxVerbatim}

\begin{sphinxVerbatim}[commandchars=\\\{\}]
194 µs ± 4.91 µs per loop (mean ± std. dev. of 7 runs, 1000 loops each)
\end{sphinxVerbatim}

\sphinxAtStartPar
Um desempenho ainda melhor pode ser alcançado usando Numba e o decorador \sphinxcode{\sphinxupquote{vectorize}}, que obtém uma lista de assinaturas de função para a qual gerar o código compilado por JIT.

\sphinxAtStartPar
Aqui, escolhemos gerar funções vetorizadas para duas assinaturas \sphinxhyphen{} uma que recebe matrizes de números de ponto flutuante de 32 bits como entrada e saída, definidos como \sphinxcode{\sphinxupquote{numba.float32}}, e um que recebe matrizes de números de ponto flutuante de 64 bits como entrada e saída, definido como \sphinxcode{\sphinxupquote{numba.float64}}:

\begin{sphinxVerbatim}[commandchars=\\\{\}]
\PYG{n+nd}{@numba}\PYG{o}{.}\PYG{n}{vectorize}\PYG{p}{(}\PYG{p}{[}\PYG{n}{numba}\PYG{o}{.}\PYG{n}{float32}\PYG{p}{(}\PYG{n}{numba}\PYG{o}{.}\PYG{n}{float32}\PYG{p}{)}\PYG{p}{,} \PYG{n}{numba}\PYG{o}{.}\PYG{n}{float64}\PYG{p}{(}\PYG{n}{numba}\PYG{o}{.}\PYG{n}{float64}\PYG{p}{)}\PYG{p}{]}\PYG{p}{)}
\PYG{k}{def} \PYG{n+nf}{jit\PYGZus{}Heaviside}\PYG{p}{(}\PYG{n}{x}\PYG{p}{)}\PYG{p}{:}
    \PYG{k}{if} \PYG{n}{x} \PYG{o}{==} \PYG{l+m+mf}{0.0}\PYG{p}{:} 
        \PYG{k}{return} \PYG{l+m+mf}{0.5}
    \PYG{k}{if} \PYG{n}{x} \PYG{o}{\PYGZlt{}} \PYG{l+m+mf}{0.0}\PYG{p}{:} 
        \PYG{k}{return} \PYG{l+m+mf}{0.0}    
    \PYG{k}{else}\PYG{p}{:}
        \PYG{k}{return} \PYG{l+m+mf}{1.0}
\end{sphinxVerbatim}

\begin{sphinxVerbatim}[commandchars=\\\{\}]
\PYG{o}{\PYGZpc{}}\PYG{k}{timeit} jit\PYGZus{}Heaviside(x)
\end{sphinxVerbatim}

\begin{sphinxVerbatim}[commandchars=\\\{\}]
31.4 µs ± 650 ns per loop (mean ± std. dev. of 7 runs, 10000 loops each)
\end{sphinxVerbatim}


\chapter{Malhas numéricas}
\label{\detokenize{extra/extra-malhasNumericas:malhas-numericas}}\label{\detokenize{extra/extra-malhasNumericas::doc}}
\sphinxAtStartPar
Ao transferir a informação de nossos modelos matemáticos do universo \sphinxstyleemphasis{contínuo} para o universo \sphinxstyleemphasis{discreto} do computador, precisamos criar uma \sphinxstyleemphasis{malha numérica} (a construção Cartesiana em gradeado recebe o nome de \sphinxstyleemphasis{grade numérica}), que pode ser estruturada por uma sucessão de pontos tal como uma imitação de uma reta “furada”. Em uma dimensão, uma malha numérica \sphinxstyleemphasis{uniforme}, cujo espaçamento entre seus pontos é igual, pode ser definida como uma progressão aritmética:
\begin{equation*}
\begin{split}t_n = t_0 \pm n h, \ \ n = 1,2,\ldots,N,\end{split}
\end{equation*}
\sphinxAtStartPar
onde \(h\) é conhecido como \sphinxstyleemphasis{passo}.


\section{Malhas uniformes}
\label{\detokenize{extra/extra-malhasNumericas:malhas-uniformes}}
\sphinxAtStartPar
A figura abaixo mostra os pontos de uma malha numérica uniforme. A linha pontilhada foi desenhada meramente para representar a porção do contínuo que o computador \sphinxstylestrong{não} captura, isto é, a informação que é perdida.

\begin{sphinxVerbatim}[commandchars=\\\{\}]
\PYG{c+c1}{\PYGZsh{} desenha uma malha numerica uniforme}

\PYG{k+kn}{import} \PYG{n+nn}{numpy} \PYG{k}{as} \PYG{n+nn}{np}
\PYG{k+kn}{import} \PYG{n+nn}{matplotlib}\PYG{n+nn}{.}\PYG{n+nn}{pyplot} \PYG{k}{as} \PYG{n+nn}{plt} 

\PYG{c+c1}{\PYGZsh{} pontos}
\PYG{n}{x} \PYG{o}{=} \PYG{n}{np}\PYG{o}{.}\PYG{n}{arange}\PYG{p}{(}\PYG{l+m+mi}{9}\PYG{p}{)}
\PYG{n}{y} \PYG{o}{=} \PYG{l+m+mi}{0}\PYG{o}{*}\PYG{n}{x}

\PYG{c+c1}{\PYGZsh{} configuracoes}
\PYG{n}{plt}\PYG{o}{.}\PYG{n}{figure}\PYG{p}{(}\PYG{n}{figsize}\PYG{o}{=}\PYG{p}{(}\PYG{l+m+mf}{6.4}\PYG{p}{,}\PYG{l+m+mf}{0.5}\PYG{p}{)}\PYG{p}{)}
\PYG{n}{plt}\PYG{o}{.}\PYG{n}{plot}\PYG{p}{(}\PYG{n}{x}\PYG{p}{,}\PYG{n}{y}\PYG{p}{,}\PYG{l+s+s1}{\PYGZsq{}}\PYG{l+s+s1}{ok}\PYG{l+s+s1}{\PYGZsq{}}\PYG{p}{,}\PYG{n}{markersize}\PYG{o}{=}\PYG{l+m+mi}{4}\PYG{p}{)}
\PYG{n}{plt}\PYG{o}{.}\PYG{n}{plot}\PYG{p}{(}\PYG{n}{x}\PYG{p}{,}\PYG{n}{y}\PYG{p}{,}\PYG{l+s+s1}{\PYGZsq{}}\PYG{l+s+s1}{:k}\PYG{l+s+s1}{\PYGZsq{}}\PYG{p}{,}\PYG{n}{linewidth}\PYG{o}{=}\PYG{l+m+mi}{1}\PYG{p}{)}
\PYG{n}{plt}\PYG{o}{.}\PYG{n}{ylim}\PYG{p}{(}\PYG{p}{(}\PYG{o}{\PYGZhy{}}\PYG{l+m+mf}{0.02}\PYG{p}{,}\PYG{l+m+mf}{0.1}\PYG{p}{)}\PYG{p}{)}
\PYG{n}{plt}\PYG{o}{.}\PYG{n}{box}\PYG{p}{(}\PYG{k+kc}{False}\PYG{p}{)}
\PYG{n}{locs}\PYG{p}{,} \PYG{n}{labels} \PYG{o}{=} \PYG{n}{plt}\PYG{o}{.}\PYG{n}{xticks}\PYG{p}{(}\PYG{p}{)}
\PYG{n}{plt}\PYG{o}{.}\PYG{n}{xticks}\PYG{p}{(}\PYG{n}{x}\PYG{p}{,} \PYG{p}{(}\PYG{l+s+s1}{\PYGZsq{}}\PYG{l+s+s1}{\PYGZdl{}t\PYGZus{}}\PYG{l+s+s1}{\PYGZob{}}\PYG{l+s+s1}{\PYGZhy{}N\PYGZcb{}\PYGZdl{}}\PYG{l+s+s1}{\PYGZsq{}}\PYG{p}{,}\PYG{l+s+s1}{\PYGZsq{}}\PYG{l+s+s1}{\PYGZdl{}}\PYG{l+s+s1}{\PYGZbs{}}\PYG{l+s+s1}{ldots\PYGZdl{}}\PYG{l+s+s1}{\PYGZsq{}}\PYG{p}{,}\PYG{l+s+s1}{\PYGZsq{}}\PYG{l+s+s1}{\PYGZdl{}t\PYGZus{}}\PYG{l+s+s1}{\PYGZob{}}\PYG{l+s+s1}{\PYGZhy{}2\PYGZcb{}\PYGZdl{}}\PYG{l+s+s1}{\PYGZsq{}}\PYG{p}{,}\PYG{l+s+s1}{\PYGZsq{}}\PYG{l+s+s1}{\PYGZdl{}t\PYGZus{}}\PYG{l+s+s1}{\PYGZob{}}\PYG{l+s+s1}{\PYGZhy{}1\PYGZcb{}\PYGZdl{}}\PYG{l+s+s1}{\PYGZsq{}}\PYG{p}{,}\PYG{l+s+s1}{\PYGZsq{}}\PYG{l+s+s1}{\PYGZdl{}t\PYGZus{}}\PYG{l+s+si}{\PYGZob{}0\PYGZcb{}}\PYG{l+s+s1}{\PYGZdl{}}\PYG{l+s+s1}{\PYGZsq{}}\PYG{p}{,}\PYG{l+s+s1}{\PYGZsq{}}\PYG{l+s+s1}{\PYGZdl{}t\PYGZus{}}\PYG{l+s+si}{\PYGZob{}1\PYGZcb{}}\PYG{l+s+s1}{\PYGZdl{}}\PYG{l+s+s1}{\PYGZsq{}}\PYG{p}{,}\PYG{l+s+s1}{\PYGZsq{}}\PYG{l+s+s1}{\PYGZdl{}t\PYGZus{}}\PYG{l+s+si}{\PYGZob{}2\PYGZcb{}}\PYG{l+s+s1}{\PYGZdl{}}\PYG{l+s+s1}{\PYGZsq{}}\PYG{p}{,}\PYG{l+s+s1}{\PYGZsq{}}\PYG{l+s+s1}{\PYGZdl{}}\PYG{l+s+s1}{\PYGZbs{}}\PYG{l+s+s1}{ldots\PYGZdl{}}\PYG{l+s+s1}{\PYGZsq{}}\PYG{p}{,}\PYG{l+s+s1}{\PYGZsq{}}\PYG{l+s+s1}{\PYGZdl{}t\PYGZus{}}\PYG{l+s+si}{\PYGZob{}N\PYGZcb{}}\PYG{l+s+s1}{\PYGZdl{}}\PYG{l+s+s1}{\PYGZsq{}}\PYG{p}{)} \PYG{p}{)}
\PYG{n}{plt}\PYG{o}{.}\PYG{n}{tick\PYGZus{}params}\PYG{p}{(}\PYG{n}{axis}\PYG{o}{=}\PYG{l+s+s1}{\PYGZsq{}}\PYG{l+s+s1}{both}\PYG{l+s+s1}{\PYGZsq{}}\PYG{p}{,}\PYG{n}{width}\PYG{o}{=}\PYG{l+m+mf}{0.0}\PYG{p}{,}\PYG{n}{labelleft}\PYG{o}{=}\PYG{k+kc}{False}\PYG{p}{)}
\end{sphinxVerbatim}

\noindent\sphinxincludegraphics{{extra-malhasNumericas_1_0}.png}


\section{Malhas não uniformes}
\label{\detokenize{extra/extra-malhasNumericas:malhas-nao-uniformes}}
\sphinxAtStartPar
Uma malha numérica \sphinxstyleemphasis{não uniforme} é aquela para a qual o tamanho do passo não é constante, como
vemos na Figura a seguir. Neste caso, podemos ter comprimentos arbitrários \(h_0 \neq h_1 \neq \ldots \neq h_{N-1}\) todos distintos, tanto à esquerda quanto à direita de \(t_0\).

\begin{sphinxVerbatim}[commandchars=\\\{\}]
\PYG{c+c1}{\PYGZsh{} desenha uma malha numerica nao\PYGZhy{}uniforme}

\PYG{c+c1}{\PYGZsh{} pontos}
\PYG{n}{x} \PYG{o}{=} \PYG{n}{np}\PYG{o}{.}\PYG{n}{array}\PYG{p}{(}\PYG{p}{[}\PYG{o}{\PYGZhy{}}\PYG{l+m+mf}{0.01}\PYG{p}{,}\PYG{l+m+mf}{0.06}\PYG{p}{,}\PYG{l+m+mf}{0.15}\PYG{p}{,}\PYG{l+m+mf}{0.21}\PYG{p}{,}\PYG{l+m+mf}{0.28}\PYG{p}{,}\PYG{l+m+mf}{0.45}\PYG{p}{,}\PYG{l+m+mf}{0.65}\PYG{p}{,}\PYG{l+m+mf}{0.9}\PYG{p}{,}\PYG{l+m+mf}{1.2}\PYG{p}{]}\PYG{p}{)}
\PYG{n}{y} \PYG{o}{=} \PYG{l+m+mi}{0}\PYG{o}{*}\PYG{n}{x}

\PYG{c+c1}{\PYGZsh{} configuracoes}
\PYG{n}{plt}\PYG{o}{.}\PYG{n}{figure}\PYG{p}{(}\PYG{n}{figsize}\PYG{o}{=}\PYG{p}{(}\PYG{l+m+mf}{6.4}\PYG{p}{,}\PYG{l+m+mf}{0.5}\PYG{p}{)}\PYG{p}{)}
\PYG{n}{plt}\PYG{o}{.}\PYG{n}{plot}\PYG{p}{(}\PYG{n}{x}\PYG{p}{,}\PYG{n}{y}\PYG{p}{,}\PYG{l+s+s1}{\PYGZsq{}}\PYG{l+s+s1}{ok}\PYG{l+s+s1}{\PYGZsq{}}\PYG{p}{,}\PYG{n}{markersize}\PYG{o}{=}\PYG{l+m+mi}{4}\PYG{p}{)}
\PYG{n}{plt}\PYG{o}{.}\PYG{n}{plot}\PYG{p}{(}\PYG{n}{x}\PYG{p}{,}\PYG{n}{y}\PYG{p}{,}\PYG{l+s+s1}{\PYGZsq{}}\PYG{l+s+s1}{:k}\PYG{l+s+s1}{\PYGZsq{}}\PYG{p}{,}\PYG{n}{linewidth}\PYG{o}{=}\PYG{l+m+mi}{1}\PYG{p}{)}
\PYG{n}{plt}\PYG{o}{.}\PYG{n}{ylim}\PYG{p}{(}\PYG{p}{(}\PYG{o}{\PYGZhy{}}\PYG{l+m+mf}{0.02}\PYG{p}{,}\PYG{l+m+mf}{0.1}\PYG{p}{)}\PYG{p}{)}
\PYG{n}{plt}\PYG{o}{.}\PYG{n}{box}\PYG{p}{(}\PYG{k+kc}{False}\PYG{p}{)}
\PYG{n}{locs}\PYG{p}{,} \PYG{n}{labels} \PYG{o}{=} \PYG{n}{plt}\PYG{o}{.}\PYG{n}{xticks}\PYG{p}{(}\PYG{p}{)}
\PYG{n}{plt}\PYG{o}{.}\PYG{n}{xticks}\PYG{p}{(}\PYG{n}{x}\PYG{p}{,} \PYG{p}{(}\PYG{l+s+s1}{\PYGZsq{}}\PYG{l+s+s1}{\PYGZdl{}t\PYGZus{}}\PYG{l+s+s1}{\PYGZob{}}\PYG{l+s+s1}{\PYGZhy{}N\PYGZcb{}\PYGZdl{}}\PYG{l+s+s1}{\PYGZsq{}}\PYG{p}{,}\PYG{l+s+s1}{\PYGZsq{}}\PYG{l+s+s1}{\PYGZdl{}}\PYG{l+s+s1}{\PYGZbs{}}\PYG{l+s+s1}{ldots\PYGZdl{}}\PYG{l+s+s1}{\PYGZsq{}}\PYG{p}{,}\PYG{l+s+s1}{\PYGZsq{}}\PYG{l+s+s1}{\PYGZdl{}t\PYGZus{}}\PYG{l+s+s1}{\PYGZob{}}\PYG{l+s+s1}{\PYGZhy{}2\PYGZcb{}\PYGZdl{}}\PYG{l+s+s1}{\PYGZsq{}}\PYG{p}{,}\PYG{l+s+s1}{\PYGZsq{}}\PYG{l+s+s1}{\PYGZdl{}t\PYGZus{}}\PYG{l+s+s1}{\PYGZob{}}\PYG{l+s+s1}{\PYGZhy{}1\PYGZcb{}\PYGZdl{}}\PYG{l+s+s1}{\PYGZsq{}}\PYG{p}{,}\PYG{l+s+s1}{\PYGZsq{}}\PYG{l+s+s1}{\PYGZdl{}t\PYGZus{}}\PYG{l+s+si}{\PYGZob{}0\PYGZcb{}}\PYG{l+s+s1}{\PYGZdl{}}\PYG{l+s+s1}{\PYGZsq{}}\PYG{p}{,}\PYG{l+s+s1}{\PYGZsq{}}\PYG{l+s+s1}{\PYGZdl{}t\PYGZus{}}\PYG{l+s+si}{\PYGZob{}1\PYGZcb{}}\PYG{l+s+s1}{\PYGZdl{}}\PYG{l+s+s1}{\PYGZsq{}}\PYG{p}{,}\PYG{l+s+s1}{\PYGZsq{}}\PYG{l+s+s1}{\PYGZdl{}t\PYGZus{}}\PYG{l+s+si}{\PYGZob{}2\PYGZcb{}}\PYG{l+s+s1}{\PYGZdl{}}\PYG{l+s+s1}{\PYGZsq{}}\PYG{p}{,}\PYG{l+s+s1}{\PYGZsq{}}\PYG{l+s+s1}{\PYGZdl{}}\PYG{l+s+s1}{\PYGZbs{}}\PYG{l+s+s1}{ldots\PYGZdl{}}\PYG{l+s+s1}{\PYGZsq{}}\PYG{p}{,}\PYG{l+s+s1}{\PYGZsq{}}\PYG{l+s+s1}{\PYGZdl{}t\PYGZus{}}\PYG{l+s+si}{\PYGZob{}N\PYGZcb{}}\PYG{l+s+s1}{\PYGZdl{}}\PYG{l+s+s1}{\PYGZsq{}}\PYG{p}{)} \PYG{p}{)}
\PYG{n}{plt}\PYG{o}{.}\PYG{n}{tick\PYGZus{}params}\PYG{p}{(}\PYG{n}{axis}\PYG{o}{=}\PYG{l+s+s1}{\PYGZsq{}}\PYG{l+s+s1}{both}\PYG{l+s+s1}{\PYGZsq{}}\PYG{p}{,}\PYG{n}{width}\PYG{o}{=}\PYG{l+m+mf}{0.0}\PYG{p}{,}\PYG{n}{labelleft}\PYG{o}{=}\PYG{k+kc}{False}\PYG{p}{)}
\end{sphinxVerbatim}

\noindent\sphinxincludegraphics{{extra-malhasNumericas_3_0}.png}


\section{Refinamento de malha}
\label{\detokenize{extra/extra-malhasNumericas:refinamento-de-malha}}
\sphinxAtStartPar
Quando uma malha não uniforme possui uma ou mais regiões onde há uma acumulação de pontos, dizemos que ela está \sphinxstyleemphasis{refinada} nessas regiões. Na figura, os pontos \(t_{-2}, t_{-1}\) e \(t_{0}\), por exemplo, acumulam\sphinxhyphen{}se próximos um do outro, ao passo que os pontos \(t_1, t_2\) e \(t_N\) afastam\sphinxhyphen{}se cada vez mais um do outro e da região de refinamento. Então, poderíamos dizer que a malha foi refinada entre \(t_{-2}\) e \(t_0\).

\sphinxAtStartPar
O refinamento faz mais sentido apenas quando queremos capturar informações localizadas com mais precisão. Esta situação ocorre, por exemplo, em problemas práticos que envolvem variações ou mudanças abruptas de propriedades, como é o caso da massa específica nas proximidades de uma interface de dois fluidos imiscíveis.

\sphinxAtStartPar
Nos esquemas acima, os nós com índice negativo são apenas ilustrativos e têm pouco efeito prático. Podemos trabalhar apenas com a indexação positiva sem problema algum.


\section{Malha numérica bidimensional uniforme}
\label{\detokenize{extra/extra-malhasNumericas:malha-numerica-bidimensional-uniforme}}
\sphinxAtStartPar
Em particular, uma \sphinxstyleemphasis{malha numérica} bidimensional dependerá de cópias de \sphinxstyleemphasis{arrays} unidimensionais que formam as coordenadas do plano Cartesiano em cada dimensão

\begin{sphinxVerbatim}[commandchars=\\\{\}]
\PYG{k+kn}{import} \PYG{n+nn}{numpy} \PYG{k}{as} \PYG{n+nn}{np}
\PYG{k+kn}{import} \PYG{n+nn}{matplotlib}\PYG{n+nn}{.}\PYG{n+nn}{pyplot} \PYG{k}{as} \PYG{n+nn}{plt}

\PYG{c+c1}{\PYGZsh{} limites do domínio:}
\PYG{c+c1}{\PYGZsh{} região do plano [a,b] x [c,d]}
\PYG{n}{a}\PYG{p}{,} \PYG{n}{b} \PYG{o}{=} \PYG{o}{\PYGZhy{}}\PYG{l+m+mf}{4.0}\PYG{p}{,} \PYG{l+m+mf}{5.0}
\PYG{n}{c}\PYG{p}{,} \PYG{n}{d} \PYG{o}{=} \PYG{o}{\PYGZhy{}}\PYG{l+m+mf}{2.0}\PYG{p}{,} \PYG{l+m+mf}{3.0}

\PYG{c+c1}{\PYGZsh{} no. de pontos em cada direção}
\PYG{n}{nx}\PYG{p}{,} \PYG{n}{ny} \PYG{o}{=} \PYG{l+m+mi}{10}\PYG{p}{,} \PYG{l+m+mi}{20} 

\PYG{c+c1}{\PYGZsh{} distribuição dos pontos}
\PYG{n}{x} \PYG{o}{=} \PYG{n}{np}\PYG{o}{.}\PYG{n}{linspace}\PYG{p}{(}\PYG{n}{a}\PYG{p}{,}\PYG{n}{b}\PYG{p}{,}\PYG{n}{nx}\PYG{p}{)}
\PYG{n}{y} \PYG{o}{=} \PYG{n}{np}\PYG{o}{.}\PYG{n}{linspace}\PYG{p}{(}\PYG{n}{c}\PYG{p}{,}\PYG{n}{d}\PYG{p}{,}\PYG{n}{ny}\PYG{p}{)}

\PYG{c+c1}{\PYGZsh{} grade numérica 2D}
\PYG{p}{[}\PYG{n}{X}\PYG{p}{,}\PYG{n}{Y}\PYG{p}{]} \PYG{o}{=} \PYG{n}{np}\PYG{o}{.}\PYG{n}{meshgrid}\PYG{p}{(}\PYG{n}{x}\PYG{p}{,}\PYG{n}{y}\PYG{p}{)}

\PYG{c+c1}{\PYGZsh{} plotando pontos da grade numérica}
\PYG{n}{plt}\PYG{o}{.}\PYG{n}{scatter}\PYG{p}{(}\PYG{n}{X}\PYG{p}{,}\PYG{n}{Y}\PYG{p}{,}\PYG{n}{s}\PYG{o}{=}\PYG{l+m+mi}{3}\PYG{p}{,}\PYG{n}{c}\PYG{o}{=}\PYG{l+s+s1}{\PYGZsq{}}\PYG{l+s+s1}{k}\PYG{l+s+s1}{\PYGZsq{}}\PYG{p}{)}\PYG{p}{;}
\PYG{n}{plt}\PYG{o}{.}\PYG{n}{title}\PYG{p}{(}\PYG{l+s+s1}{\PYGZsq{}}\PYG{l+s+s1}{Grade numérica 2D: pontos (x,y)}\PYG{l+s+s1}{\PYGZsq{}}\PYG{p}{)}
\PYG{n}{plt}\PYG{o}{.}\PYG{n}{xlabel}\PYG{p}{(}\PYG{l+s+s1}{\PYGZsq{}}\PYG{l+s+s1}{x}\PYG{l+s+s1}{\PYGZsq{}}\PYG{p}{)}\PYG{p}{;} \PYG{n}{plt}\PYG{o}{.}\PYG{n}{ylabel}\PYG{p}{(}\PYG{l+s+s1}{\PYGZsq{}}\PYG{l+s+s1}{y}\PYG{l+s+s1}{\PYGZsq{}}\PYG{p}{)}\PYG{p}{;}
\end{sphinxVerbatim}

\noindent\sphinxincludegraphics{{extra-malhasNumericas_6_0}.png}


\subsection{Plotando curvas de nível}
\label{\detokenize{extra/extra-malhasNumericas:plotando-curvas-de-nivel}}
\sphinxAtStartPar
Plotaremos as curvas de nível 0 de funções não\sphinxhyphen{}lineares é útil para realizarmos análise gráfica e escolher vetores de estimativa inicial.

\sphinxAtStartPar
Para plotar curvas de nível das funções sobre a grade numérica anterior, fazemos o seguinte:

\begin{sphinxVerbatim}[commandchars=\\\{\}]
\PYG{c+c1}{\PYGZsh{} funções definidas sobre a grade 2D}
\PYG{n}{F} \PYG{o}{=} \PYG{n}{X}\PYG{o}{*}\PYG{o}{*}\PYG{l+m+mi}{2} \PYG{o}{+} \PYG{n}{Y}\PYG{o}{*}\PYG{o}{*}\PYG{l+m+mi}{2} \PYG{o}{\PYGZhy{}} \PYG{l+m+mi}{2}
\PYG{n}{G} \PYG{o}{=} \PYG{n}{X}\PYG{o}{*}\PYG{o}{*}\PYG{l+m+mi}{2} \PYG{o}{\PYGZhy{}} \PYG{n}{Y}\PYG{o}{*}\PYG{o}{*}\PYG{l+m+mi}{2} \PYG{o}{\PYGZhy{}} \PYG{l+m+mi}{1}

\PYG{c+c1}{\PYGZsh{} contorno de nível 0}
\PYG{n}{plt}\PYG{o}{.}\PYG{n}{contour}\PYG{p}{(}\PYG{n}{X}\PYG{p}{,}\PYG{n}{Y}\PYG{p}{,}\PYG{n}{F}\PYG{p}{,}\PYG{n}{colors}\PYG{o}{=}\PYG{l+s+s1}{\PYGZsq{}}\PYG{l+s+s1}{red}\PYG{l+s+s1}{\PYGZsq{}}\PYG{p}{,}\PYG{n}{levels}\PYG{o}{=}\PYG{l+m+mi}{0}\PYG{p}{)}\PYG{p}{;}
\PYG{n}{plt}\PYG{o}{.}\PYG{n}{contour}\PYG{p}{(}\PYG{n}{X}\PYG{p}{,}\PYG{n}{Y}\PYG{p}{,}\PYG{n}{G}\PYG{p}{,}\PYG{n}{colors}\PYG{o}{=}\PYG{l+s+s1}{\PYGZsq{}}\PYG{l+s+s1}{blue}\PYG{l+s+s1}{\PYGZsq{}}\PYG{p}{,}\PYG{n}{levels}\PYG{o}{=}\PYG{l+m+mi}{0}\PYG{p}{)}\PYG{p}{;}
\PYG{n}{plt}\PYG{o}{.}\PYG{n}{grid}\PYG{p}{(}\PYG{p}{)}
\end{sphinxVerbatim}

\noindent\sphinxincludegraphics{{extra-malhasNumericas_8_0}.png}

\sphinxAtStartPar
Por que a figura está meio “tosca”? Porque temos poucos pontos na grade. Vamos aumentar o número de pontos. Este processo é conhecido como \sphinxstyleemphasis{refinamento de malha}.

\begin{sphinxVerbatim}[commandchars=\\\{\}]
\PYG{c+c1}{\PYGZsh{} refinando a malha numérica}
\PYG{n}{nx2}\PYG{p}{,} \PYG{n}{ny2} \PYG{o}{=} \PYG{l+m+mi}{100}\PYG{p}{,} \PYG{l+m+mi}{200}

\PYG{c+c1}{\PYGZsh{} redistribuição dos pontos}
\PYG{n}{x2} \PYG{o}{=} \PYG{n}{np}\PYG{o}{.}\PYG{n}{linspace}\PYG{p}{(}\PYG{n}{a}\PYG{p}{,}\PYG{n}{b}\PYG{p}{,}\PYG{n}{nx2}\PYG{p}{)}
\PYG{n}{y2} \PYG{o}{=} \PYG{n}{np}\PYG{o}{.}\PYG{n}{linspace}\PYG{p}{(}\PYG{n}{c}\PYG{p}{,}\PYG{n}{d}\PYG{p}{,}\PYG{n}{ny2}\PYG{p}{)}

\PYG{c+c1}{\PYGZsh{} grade numérica 2D refinada}
\PYG{p}{[}\PYG{n}{X2}\PYG{p}{,}\PYG{n}{Y2}\PYG{p}{]} \PYG{o}{=} \PYG{n}{np}\PYG{o}{.}\PYG{n}{meshgrid}\PYG{p}{(}\PYG{n}{x2}\PYG{p}{,}\PYG{n}{y2}\PYG{p}{)}

\PYG{c+c1}{\PYGZsh{} plotando pontos da grade numérica}
\PYG{n}{plt}\PYG{o}{.}\PYG{n}{scatter}\PYG{p}{(}\PYG{n}{X2}\PYG{p}{,}\PYG{n}{Y2}\PYG{p}{,}\PYG{n}{s}\PYG{o}{=}\PYG{l+m+mf}{0.1}\PYG{p}{,}\PYG{n}{c}\PYG{o}{=}\PYG{l+s+s1}{\PYGZsq{}}\PYG{l+s+s1}{b}\PYG{l+s+s1}{\PYGZsq{}}\PYG{p}{)}\PYG{p}{;}
\PYG{n}{plt}\PYG{o}{.}\PYG{n}{title}\PYG{p}{(}\PYG{l+s+s1}{\PYGZsq{}}\PYG{l+s+s1}{Grade numérica 2D refinada: muitos pontos (x,y)}\PYG{l+s+s1}{\PYGZsq{}}\PYG{p}{)}
\PYG{n}{plt}\PYG{o}{.}\PYG{n}{xlabel}\PYG{p}{(}\PYG{l+s+s1}{\PYGZsq{}}\PYG{l+s+s1}{x}\PYG{l+s+s1}{\PYGZsq{}}\PYG{p}{)}\PYG{p}{;} \PYG{n}{plt}\PYG{o}{.}\PYG{n}{ylabel}\PYG{p}{(}\PYG{l+s+s1}{\PYGZsq{}}\PYG{l+s+s1}{y}\PYG{l+s+s1}{\PYGZsq{}}\PYG{p}{)}\PYG{p}{;}
\end{sphinxVerbatim}

\noindent\sphinxincludegraphics{{extra-malhasNumericas_10_0}.png}

\sphinxAtStartPar
Vamos plotar novamente as curvas de nível das funções sobre a grade numérica refinada.

\begin{sphinxVerbatim}[commandchars=\\\{\}]
\PYG{c+c1}{\PYGZsh{} funções definidas sobre a grade 2D refinada}
\PYG{n}{F2} \PYG{o}{=} \PYG{n}{X2}\PYG{o}{*}\PYG{o}{*}\PYG{l+m+mi}{2} \PYG{o}{+} \PYG{n}{Y2}\PYG{o}{*}\PYG{o}{*}\PYG{l+m+mi}{2} \PYG{o}{\PYGZhy{}} \PYG{l+m+mi}{2}
\PYG{n}{G2} \PYG{o}{=} \PYG{n}{X2}\PYG{o}{*}\PYG{o}{*}\PYG{l+m+mi}{2} \PYG{o}{\PYGZhy{}} \PYG{n}{Y2}\PYG{o}{*}\PYG{o}{*}\PYG{l+m+mi}{2} \PYG{o}{\PYGZhy{}} \PYG{l+m+mi}{1}

\PYG{c+c1}{\PYGZsh{} contorno de nível 0 na malha refinada}
\PYG{n}{plt}\PYG{o}{.}\PYG{n}{contour}\PYG{p}{(}\PYG{n}{X2}\PYG{p}{,}\PYG{n}{Y2}\PYG{p}{,}\PYG{n}{F2}\PYG{p}{,}\PYG{n}{colors}\PYG{o}{=}\PYG{l+s+s1}{\PYGZsq{}}\PYG{l+s+s1}{red}\PYG{l+s+s1}{\PYGZsq{}}\PYG{p}{,}\PYG{n}{levels}\PYG{o}{=}\PYG{l+m+mi}{0}\PYG{p}{)}\PYG{p}{;}
\PYG{n}{plt}\PYG{o}{.}\PYG{n}{contour}\PYG{p}{(}\PYG{n}{X2}\PYG{p}{,}\PYG{n}{Y2}\PYG{p}{,}\PYG{n}{G2}\PYG{p}{,}\PYG{n}{colors}\PYG{o}{=}\PYG{l+s+s1}{\PYGZsq{}}\PYG{l+s+s1}{blue}\PYG{l+s+s1}{\PYGZsq{}}\PYG{p}{,}\PYG{n}{levels}\PYG{o}{=}\PYG{l+m+mi}{0}\PYG{p}{)}\PYG{p}{;}
\PYG{n}{plt}\PYG{o}{.}\PYG{n}{grid}\PYG{p}{(}\PYG{p}{)}
\end{sphinxVerbatim}

\noindent\sphinxincludegraphics{{extra-malhasNumericas_12_0}.png}


\chapter{Campos de direção}
\label{\detokenize{extra/extra-camposDirecao-edo:campos-de-direcao}}\label{\detokenize{extra/extra-camposDirecao-edo::doc}}
\sphinxAtStartPar
\sphinxstyleemphasis{Campos de direção} são úteis para entender o comportamento das soluções de uma EDO. O gráfico de uma solução da equação \(y' = f(t,y)\) é aquele que, para todo ponto \((t,y)\) do plano, conhecemos a inclinação da curva \(y(t)\), solução da EDO. Campos de direção podem ser plotados em Python através das funções \sphinxcode{\sphinxupquote{meshgrid}}, do pacote \sphinxcode{\sphinxupquote{numpy}}, e \sphinxcode{\sphinxupquote{quiver}}, do pacote \sphinxcode{\sphinxupquote{matplotlib}}.

\sphinxAtStartPar
\sphinxstylestrong{Exemplo} Consideremos a EDO \(y'= y\). A inclinação é dada por \(f(t,y) = y\) e é independente de \(t\). Vamos gerar o diagrama do campo de direções para esta EDO pelo código a seguir.

\begin{sphinxVerbatim}[commandchars=\\\{\}]
\PYG{k+kn}{import} \PYG{n+nn}{numpy} \PYG{k}{as} \PYG{n+nn}{np} 
\PYG{k+kn}{import} \PYG{n+nn}{matplotlib}\PYG{n+nn}{.}\PYG{n+nn}{pyplot} \PYG{k}{as} \PYG{n+nn}{plt} 

\PYG{c+c1}{\PYGZsh{} parametros}
\PYG{n}{t0}\PYG{p}{,} \PYG{n}{tb} \PYG{o}{=} \PYG{l+m+mi}{0}\PYG{p}{,} \PYG{l+m+mi}{1}
\PYG{n}{y0}\PYG{p}{,} \PYG{n}{yb} \PYG{o}{=} \PYG{o}{\PYGZhy{}}\PYG{l+m+mi}{6}\PYG{p}{,} \PYG{l+m+mi}{6}
\PYG{n}{nt}\PYG{p}{,} \PYG{n}{ny} \PYG{o}{=} \PYG{l+m+mi}{10}\PYG{p}{,} \PYG{l+m+mi}{20}

\PYG{c+c1}{\PYGZsh{} dominio (t,y)}
\PYG{n}{t} \PYG{o}{=} \PYG{n}{np}\PYG{o}{.}\PYG{n}{linspace}\PYG{p}{(}\PYG{n}{t0}\PYG{p}{,}\PYG{n}{tb}\PYG{p}{,}\PYG{n}{nt}\PYG{p}{)}
\PYG{n}{y} \PYG{o}{=} \PYG{n}{np}\PYG{o}{.}\PYG{n}{linspace}\PYG{p}{(}\PYG{n}{y0}\PYG{p}{,}\PYG{n}{yb}\PYG{p}{,}\PYG{n}{ny}\PYG{p}{)}
\PYG{p}{[}\PYG{n}{T}\PYG{p}{,}\PYG{n}{Y}\PYG{p}{]} \PYG{o}{=} \PYG{n}{np}\PYG{o}{.}\PYG{n}{meshgrid}\PYG{p}{(}\PYG{n}{t}\PYG{p}{,}\PYG{n}{y}\PYG{p}{)}

\PYG{c+c1}{\PYGZsh{} EDO }
\PYG{n}{dt} \PYG{o}{=} \PYG{n}{np}\PYG{o}{.}\PYG{n}{ones}\PYG{p}{(}\PYG{n}{T}\PYG{o}{.}\PYG{n}{shape}\PYG{p}{)}
\PYG{n}{dy} \PYG{o}{=} \PYG{n}{Y}

\PYG{c+c1}{\PYGZsh{} campo}
\PYG{n}{f} \PYG{o}{=} \PYG{n}{plt}\PYG{o}{.}\PYG{n}{figure}\PYG{p}{(}\PYG{n}{figsize}\PYG{o}{=}\PYG{p}{(}\PYG{l+m+mi}{6}\PYG{p}{,}\PYG{l+m+mi}{5}\PYG{p}{)}\PYG{p}{)}

\PYG{n}{plt}\PYG{o}{.}\PYG{n}{quiver}\PYG{p}{(}\PYG{n}{T}\PYG{p}{,}\PYG{n}{Y}\PYG{p}{,}\PYG{n}{dt}\PYG{p}{,}\PYG{n}{dy}\PYG{p}{,}\PYG{n}{color}\PYG{o}{=}\PYG{l+s+s1}{\PYGZsq{}}\PYG{l+s+s1}{red}\PYG{l+s+s1}{\PYGZsq{}}\PYG{p}{)}

\PYG{c+c1}{\PYGZsh{} solucoes particulares }
\PYG{n}{y\PYGZus{}part} \PYG{o}{=} \PYG{k}{lambda} \PYG{n}{c}\PYG{p}{:} \PYG{n}{c}\PYG{o}{*}\PYG{n}{np}\PYG{o}{.}\PYG{n}{exp}\PYG{p}{(}\PYG{n}{t}\PYG{p}{)}

\PYG{k}{for} \PYG{n}{c} \PYG{o+ow}{in} \PYG{p}{[}\PYG{l+m+mf}{0.6}\PYG{p}{,}\PYG{l+m+mf}{0.2}\PYG{p}{,}\PYG{o}{\PYGZhy{}}\PYG{l+m+mf}{0.3}\PYG{p}{,}\PYG{o}{\PYGZhy{}}\PYG{l+m+mf}{0.7}\PYG{p}{]}\PYG{p}{:}
    \PYG{n}{aux} \PYG{o}{=} \PYG{n}{y\PYGZus{}part}\PYG{p}{(}\PYG{n}{c}\PYG{p}{)}
    \PYG{n}{plt}\PYG{o}{.}\PYG{n}{plot}\PYG{p}{(}\PYG{n}{t}\PYG{p}{,}\PYG{n}{y\PYGZus{}part}\PYG{p}{(}\PYG{n}{aux}\PYG{p}{)}\PYG{p}{)}    
\end{sphinxVerbatim}

\noindent\sphinxincludegraphics{{extra-camposDirecao-edo_1_0}.png}

\sphinxAtStartPar
A solução geral desta EDO é \(y(t) = ce^{t}\). Na figura, destacamos quatro soluções particulares, para valores \(c \in \{0.6,0.2,-0.3,-0.7\}\).

\sphinxAtStartPar
\sphinxstylestrong{Exemplo:} Vamos gerar o diagrama do campo de direções para a EDO \(y' = 1/(1-t^2)\).

\begin{sphinxVerbatim}[commandchars=\\\{\}]
\PYG{k+kn}{import} \PYG{n+nn}{numpy} \PYG{k}{as} \PYG{n+nn}{np} 
\PYG{k+kn}{import} \PYG{n+nn}{matplotlib}\PYG{n+nn}{.}\PYG{n+nn}{pyplot} \PYG{k}{as} \PYG{n+nn}{plt} 

\PYG{c+c1}{\PYGZsh{} parametros}
\PYG{n}{t0}\PYG{p}{,} \PYG{n}{tb} \PYG{o}{=} \PYG{o}{\PYGZhy{}}\PYG{l+m+mi}{5}\PYG{p}{,} \PYG{l+m+mi}{5}
\PYG{n}{y0}\PYG{p}{,} \PYG{n}{yb} \PYG{o}{=} \PYG{o}{\PYGZhy{}}\PYG{l+m+mi}{6}\PYG{p}{,} \PYG{l+m+mi}{6}
\PYG{n}{nt}\PYG{p}{,} \PYG{n}{ny} \PYG{o}{=} \PYG{l+m+mi}{10}\PYG{p}{,} \PYG{l+m+mi}{20}

\PYG{c+c1}{\PYGZsh{} dominio (t,y)}
\PYG{n}{t} \PYG{o}{=} \PYG{n}{np}\PYG{o}{.}\PYG{n}{linspace}\PYG{p}{(}\PYG{n}{t0}\PYG{p}{,}\PYG{n}{tb}\PYG{p}{,}\PYG{n}{nt}\PYG{p}{)}
\PYG{n}{y} \PYG{o}{=} \PYG{n}{np}\PYG{o}{.}\PYG{n}{linspace}\PYG{p}{(}\PYG{n}{y0}\PYG{p}{,}\PYG{n}{yb}\PYG{p}{,}\PYG{n}{ny}\PYG{p}{)}
\PYG{p}{[}\PYG{n}{T}\PYG{p}{,}\PYG{n}{Y}\PYG{p}{]} \PYG{o}{=} \PYG{n}{np}\PYG{o}{.}\PYG{n}{meshgrid}\PYG{p}{(}\PYG{n}{t}\PYG{p}{,}\PYG{n}{y}\PYG{p}{)}

\PYG{c+c1}{\PYGZsh{} EDO }
\PYG{n}{dt} \PYG{o}{=} \PYG{n}{np}\PYG{o}{.}\PYG{n}{ones}\PYG{p}{(}\PYG{n}{T}\PYG{o}{.}\PYG{n}{shape}\PYG{p}{)}
\PYG{n}{dy} \PYG{o}{=} \PYG{l+m+mf}{1.}\PYG{o}{/}\PYG{p}{(}\PYG{l+m+mi}{1}\PYG{o}{\PYGZhy{}}\PYG{n}{T}\PYG{o}{*}\PYG{o}{*}\PYG{l+m+mi}{2}\PYG{p}{)}

\PYG{c+c1}{\PYGZsh{} campo}
\PYG{n}{f} \PYG{o}{=} \PYG{n}{plt}\PYG{o}{.}\PYG{n}{figure}\PYG{p}{(}\PYG{n}{figsize}\PYG{o}{=}\PYG{p}{(}\PYG{l+m+mi}{6}\PYG{p}{,}\PYG{l+m+mi}{5}\PYG{p}{)}\PYG{p}{)}
\PYG{n}{plt}\PYG{o}{.}\PYG{n}{quiver}\PYG{p}{(}\PYG{n}{T}\PYG{p}{,}\PYG{n}{Y}\PYG{p}{,}\PYG{n}{dt}\PYG{p}{,}\PYG{n}{dy}\PYG{p}{,}\PYG{n}{color}\PYG{o}{=}\PYG{l+s+s1}{\PYGZsq{}}\PYG{l+s+s1}{b}\PYG{l+s+s1}{\PYGZsq{}}\PYG{p}{)}\PYG{p}{;}
\end{sphinxVerbatim}

\noindent\sphinxincludegraphics{{extra-camposDirecao-edo_4_0}.png}


\chapter{Problemas}
\label{\detokenize{extra/extra-camposDirecao-edo:problemas}}\begin{enumerate}
\sphinxsetlistlabels{\arabic}{enumi}{enumii}{}{.}%
\item {} 
\sphinxAtStartPar
Use o Python para plotar os campos de direção para a família de soluções de cada PVI do Problema 1 da Aula 1.

\item {} 
\sphinxAtStartPar
Em cada caso, plote as soluções particulares que você encontrou com a substituição de \(c\).

\end{enumerate}


\chapter{Variações do Método de Euler}
\label{\detokenize{extra/extra-eulerMelhorado:variacoes-do-metodo-de-euler}}\label{\detokenize{extra/extra-eulerMelhorado::doc}}

\section{Método de Euler Melhorado}
\label{\detokenize{extra/extra-eulerMelhorado:metodo-de-euler-melhorado}}
\sphinxAtStartPar
O \sphinxstyleemphasis{Método de Euler Melhorado} (MEM) é uma técnica numérica explícita de passo simples usada para resolver EDOs que modifica o método explícito de Euler. O MEM usa uma \sphinxstyleemphasis{inclinação ponderada} com derivadas computadas em \(t_i\) e \(t_{i+1}\). No início do intervalo, a inclinação é
\begin{equation*}
\begin{split}\dfrac{dy}{dt}\Bigg|_{t = t_i} = f(t_i,y_i),\end{split}
\end{equation*}
\sphinxAtStartPar
a mesma assumida no Método de Euler Explícito. Todavia, existe uma diferença fundamental na estimativa da inclinação em \(t_{i+1}\). Primeiramente, um valor aproximado para \(y_{i+1}\) é determinado como
\begin{equation*}
\begin{split}\tilde{y}_{i+1} = y_i + h f(t_i,y_i),\end{split}
\end{equation*}
\sphinxAtStartPar
que corresponde à estimativa calculada pelo Método de Euler (reta com inclinação constante \(f(t_i,y_i)\)). Em seguida, usamos este valor para determinar uma nova inclinação que será usada para a integração da EDO, a saber:
\begin{equation*}
\begin{split}\dfrac{dy}{dt}\Bigg|_{t = t_{i+1}, y = \tilde{y}_{i+1}} = f(t_{i+1},\tilde{y}_i),\end{split}
\end{equation*}
\sphinxAtStartPar
O valor \(\tilde{y}_{i+1}\) é usado como um \sphinxstyleemphasis{preditor}. Então, dispondo dessas duas inclinações, uma inclinação ponderada pela média resulta da seguinte equação:
\begin{equation*}
\begin{split}\overline{\dfrac{dy}{dt}}\Bigg|_{t = t_i} = \dfrac{1}{2}\left( \dfrac{dy}{dt}\Bigg|_{t = t_i} + \dfrac{dy}{dt}\Bigg|_{t = t_{i+1}, y = \tilde{y}_{i+1}} \right)\end{split}
\end{equation*}
\sphinxAtStartPar
Por sua vez, construímos a \sphinxstyleemphasis{equação de correção} (ou \sphinxstyleemphasis{corretor}) como:
\begin{equation*}
\begin{split}y_{i+1} = y_i + \phi(t_i,y_i)h = y_i + \frac{f(t_i,y_i) + f(t_{i+1},\tilde{y}_{i+1})}{2}h\end{split}
\end{equation*}
\sphinxAtStartPar
Na realidade, o MEM é um algoritmo da família \sphinxstyleemphasis{preditor\sphinxhyphen{}corretor}. A rigor de notação, a aproximação para a solução do PVI é dada pelo processo iterativo a seguir:
\begin{align*}
w_0 &= \alpha \\
\tilde{w}_{i+1} &= w_i + h f(t_i,w_i) \\
		{w}_{i+1} &= w_i + \frac{h}{2}[f(t_i,w_i) + f(t_{i+1}, \tilde{w}_{i+1})], \quad i = 0,1,\dots,N-1.
\end{align*}

\subsection{Implementação computacional}
\label{\detokenize{extra/extra-eulerMelhorado:implementacao-computacional}}
\sphinxAtStartPar
O seguinte código implementa o MEM:

\begin{sphinxVerbatim}[commandchars=\\\{\}]
\PYG{k+kn}{from} \PYG{n+nn}{numpy} \PYG{k+kn}{import} \PYG{o}{*}

\PYG{k}{def} \PYG{n+nf}{euler\PYGZus{}melh}\PYG{p}{(}\PYG{n}{t0}\PYG{p}{,}\PYG{n}{tf}\PYG{p}{,}\PYG{n}{y0}\PYG{p}{,}\PYG{n}{h}\PYG{p}{,}\PYG{n}{fun}\PYG{p}{)}\PYG{p}{:}
    \PYG{l+s+sd}{\PYGZdq{}\PYGZdq{}\PYGZdq{}}
\PYG{l+s+sd}{    Resolve o PVI y\PYGZsq{} = f(t,y), t0 \PYGZlt{}= t \PYGZlt{}= tf, y(t0) = y0}
\PYG{l+s+sd}{    com passo h usando o metodo de Euler melhorado. }
\PYG{l+s+sd}{    }
\PYG{l+s+sd}{    Entrada: }
\PYG{l+s+sd}{        t0  \PYGZhy{} tempo inicial}
\PYG{l+s+sd}{        tf  \PYGZhy{} tempo final }
\PYG{l+s+sd}{        y0  \PYGZhy{} condicao inicial }
\PYG{l+s+sd}{        h   \PYGZhy{} passo }
\PYG{l+s+sd}{        fun \PYGZhy{} funcao f(t,y) (anonima)}
\PYG{l+s+sd}{        }
\PYG{l+s+sd}{    Saida:}
\PYG{l+s+sd}{        t   \PYGZhy{} nos da malha numerica }
\PYG{l+s+sd}{        y   \PYGZhy{} solucao aproximada}
\PYG{l+s+sd}{    \PYGZdq{}\PYGZdq{}\PYGZdq{}}
    
    \PYG{n}{n} \PYG{o}{=} \PYG{n+nb}{round}\PYG{p}{(}\PYG{p}{(}\PYG{n}{tf} \PYG{o}{\PYGZhy{}} \PYG{n}{t0}\PYG{p}{)}\PYG{o}{/}\PYG{n}{h} \PYG{o}{+} \PYG{l+m+mi}{1}\PYG{p}{)}
    \PYG{n}{t} \PYG{o}{=} \PYG{n}{linspace}\PYG{p}{(}\PYG{n}{t0}\PYG{p}{,}\PYG{n}{t0}\PYG{o}{+}\PYG{p}{(}\PYG{n}{n}\PYG{o}{\PYGZhy{}}\PYG{l+m+mi}{1}\PYG{p}{)}\PYG{o}{*}\PYG{n}{h}\PYG{p}{,}\PYG{n}{n}\PYG{p}{)}
    \PYG{n}{y} \PYG{o}{=} \PYG{n}{linspace}\PYG{p}{(}\PYG{n}{t0}\PYG{p}{,}\PYG{n}{t0}\PYG{o}{+}\PYG{p}{(}\PYG{n}{n}\PYG{o}{\PYGZhy{}}\PYG{l+m+mi}{1}\PYG{p}{)}\PYG{o}{*}\PYG{n}{h}\PYG{p}{,}\PYG{n}{n}\PYG{p}{)}
    \PYG{n}{y} \PYG{o}{=} \PYG{n}{zeros}\PYG{p}{(}\PYG{p}{(}\PYG{n}{n}\PYG{p}{,}\PYG{p}{)}\PYG{p}{)}
    
    \PYG{n}{y}\PYG{p}{[}\PYG{l+m+mi}{0}\PYG{p}{]} \PYG{o}{=} \PYG{n}{y0}

    \PYG{k}{for} \PYG{n}{i} \PYG{o+ow}{in} \PYG{n+nb}{range}\PYG{p}{(}\PYG{l+m+mi}{1}\PYG{p}{,}\PYG{n}{n}\PYG{p}{)}\PYG{p}{:}        
        \PYG{n}{ytilde} \PYG{o}{=} \PYG{n}{y}\PYG{p}{[}\PYG{n}{i}\PYG{o}{\PYGZhy{}}\PYG{l+m+mi}{1}\PYG{p}{]} \PYG{o}{+} \PYG{n}{h}\PYG{o}{*}\PYG{n}{f}\PYG{p}{(}\PYG{n}{t}\PYG{p}{[}\PYG{n}{i}\PYG{o}{\PYGZhy{}}\PYG{l+m+mi}{1}\PYG{p}{]}\PYG{p}{,}\PYG{n}{y}\PYG{p}{[}\PYG{n}{i}\PYG{o}{\PYGZhy{}}\PYG{l+m+mi}{1}\PYG{p}{]}\PYG{p}{)}
        \PYG{n}{ymean} \PYG{o}{=} \PYG{l+m+mf}{0.5}\PYG{o}{*}\PYG{p}{(} \PYG{n}{f}\PYG{p}{(}\PYG{n}{t}\PYG{p}{[}\PYG{n}{i}\PYG{o}{\PYGZhy{}}\PYG{l+m+mi}{1}\PYG{p}{]}\PYG{p}{,}\PYG{n}{y}\PYG{p}{[}\PYG{n}{i}\PYG{o}{\PYGZhy{}}\PYG{l+m+mi}{1}\PYG{p}{]}\PYG{p}{)} \PYG{o}{+} \PYG{n}{f}\PYG{p}{(}\PYG{n}{t}\PYG{p}{[}\PYG{n}{i}\PYG{p}{]}\PYG{p}{,}\PYG{n}{ytilde}\PYG{p}{)} \PYG{p}{)}
        \PYG{n}{y}\PYG{p}{[}\PYG{n}{i}\PYG{p}{]} \PYG{o}{=} \PYG{n}{y}\PYG{p}{[}\PYG{n}{i}\PYG{o}{\PYGZhy{}}\PYG{l+m+mi}{1}\PYG{p}{]} \PYG{o}{+} \PYG{n}{h}\PYG{o}{*}\PYG{n}{ymean}                
    
    \PYG{k}{return} \PYG{p}{(}\PYG{n}{t}\PYG{p}{,}\PYG{n}{y}\PYG{p}{)}

\PYG{c+c1}{\PYGZsh{}\PYGZsh{}\PYGZsh{} COPIA  }

\PYG{k}{def} \PYG{n+nf}{euler\PYGZus{}expl}\PYG{p}{(}\PYG{n}{t0}\PYG{p}{,}\PYG{n}{tf}\PYG{p}{,}\PYG{n}{y0}\PYG{p}{,}\PYG{n}{h}\PYG{p}{,}\PYG{n}{fun}\PYG{p}{)}\PYG{p}{:}
    \PYG{l+s+sd}{\PYGZdq{}\PYGZdq{}\PYGZdq{}}
\PYG{l+s+sd}{    Resolve o PVI y\PYGZsq{} = f(t,y), t0 \PYGZlt{}= t \PYGZlt{}= tf, y(t0) = y0}
\PYG{l+s+sd}{    com passo h usando o metodo de Euler explicito. }
\PYG{l+s+sd}{    }
\PYG{l+s+sd}{    Entrada: }
\PYG{l+s+sd}{        t0  \PYGZhy{} tempo inicial}
\PYG{l+s+sd}{        tf  \PYGZhy{} tempo final }
\PYG{l+s+sd}{        y0  \PYGZhy{} condicao inicial }
\PYG{l+s+sd}{        h   \PYGZhy{} passo }
\PYG{l+s+sd}{        fun \PYGZhy{} funcao f(t,y) (anonima)}
\PYG{l+s+sd}{        }
\PYG{l+s+sd}{    Saida:}
\PYG{l+s+sd}{        t   \PYGZhy{} nos da malha numerica }
\PYG{l+s+sd}{        y   \PYGZhy{} solucao aproximada}
\PYG{l+s+sd}{    \PYGZdq{}\PYGZdq{}\PYGZdq{}}
    
    \PYG{n}{n} \PYG{o}{=} \PYG{n+nb}{round}\PYG{p}{(}\PYG{p}{(}\PYG{n}{tf} \PYG{o}{\PYGZhy{}} \PYG{n}{t0}\PYG{p}{)}\PYG{o}{/}\PYG{n}{h} \PYG{o}{+} \PYG{l+m+mi}{1}\PYG{p}{)}
    \PYG{n}{t} \PYG{o}{=} \PYG{n}{linspace}\PYG{p}{(}\PYG{n}{t0}\PYG{p}{,}\PYG{n}{t0}\PYG{o}{+}\PYG{p}{(}\PYG{n}{n}\PYG{o}{\PYGZhy{}}\PYG{l+m+mi}{1}\PYG{p}{)}\PYG{o}{*}\PYG{n}{h}\PYG{p}{,}\PYG{n}{n}\PYG{p}{)}
    \PYG{n}{y} \PYG{o}{=} \PYG{n}{linspace}\PYG{p}{(}\PYG{n}{t0}\PYG{p}{,}\PYG{n}{t0}\PYG{o}{+}\PYG{p}{(}\PYG{n}{n}\PYG{o}{\PYGZhy{}}\PYG{l+m+mi}{1}\PYG{p}{)}\PYG{o}{*}\PYG{n}{h}\PYG{p}{,}\PYG{n}{n}\PYG{p}{)}
    \PYG{n}{y} \PYG{o}{=} \PYG{n}{zeros}\PYG{p}{(}\PYG{p}{(}\PYG{n}{n}\PYG{p}{,}\PYG{p}{)}\PYG{p}{)}
    
    \PYG{n}{y}\PYG{p}{[}\PYG{l+m+mi}{0}\PYG{p}{]} \PYG{o}{=} \PYG{n}{y0}

    \PYG{k}{for} \PYG{n}{i} \PYG{o+ow}{in} \PYG{n+nb}{range}\PYG{p}{(}\PYG{l+m+mi}{1}\PYG{p}{,}\PYG{n}{n}\PYG{p}{)}\PYG{p}{:}
        \PYG{n}{y}\PYG{p}{[}\PYG{n}{i}\PYG{p}{]} \PYG{o}{=} \PYG{n}{y}\PYG{p}{[}\PYG{n}{i}\PYG{o}{\PYGZhy{}}\PYG{l+m+mi}{1}\PYG{p}{]} \PYG{o}{+} \PYG{n}{h}\PYG{o}{*}\PYG{n}{f}\PYG{p}{(}\PYG{n}{t}\PYG{p}{[}\PYG{n}{i}\PYG{o}{\PYGZhy{}}\PYG{l+m+mi}{1}\PYG{p}{]}\PYG{p}{,}\PYG{n}{y}\PYG{p}{[}\PYG{n}{i}\PYG{o}{\PYGZhy{}}\PYG{l+m+mi}{1}\PYG{p}{]}\PYG{p}{)}

    \PYG{k}{return} \PYG{p}{(}\PYG{n}{t}\PYG{p}{,}\PYG{n}{y}\PYG{p}{)}
\end{sphinxVerbatim}

\sphinxAtStartPar
O exemplo a seguir é o mesmo que resolvemos com o Método de Euler Explícito.

\sphinxAtStartPar
\sphinxstylestrong{Exemplo:} Resolva numericamente
\begin{equation*}
\begin{split}\begin{cases}
y'(t) = \frac{y(t) + t^2 - 2}{t+1}\\
y(0) = 2 \\
0 \le t \le 6 \\ 
h = 0.5
\end{cases}\end{split}
\end{equation*}
\sphinxAtStartPar
Defina \(y_{mem}(t)\) como a solução numérica obtida pelo método de Euler melhorado e \(y_{mee}(t)\) como a solução numérica obtida pelo método de Euler explícito. Plote o gráfico das funções aproximadas juntamente com o da solução exata \(y(t) = t^2 + 2t + 2 - 2(t+1)\ln(t+1)\)

\begin{sphinxVerbatim}[commandchars=\\\{\}]
\PYG{k+kn}{import} \PYG{n+nn}{matplotlib}\PYG{n+nn}{.}\PYG{n+nn}{pyplot} \PYG{k}{as} \PYG{n+nn}{plt} 

\PYG{c+c1}{\PYGZsh{} define funcao}
\PYG{n}{f} \PYG{o}{=} \PYG{k}{lambda} \PYG{n}{t}\PYG{p}{,}\PYG{n}{y}\PYG{p}{:} \PYG{p}{(}\PYG{n}{y} \PYG{o}{+} \PYG{n}{t}\PYG{o}{*}\PYG{o}{*}\PYG{l+m+mi}{2} \PYG{o}{\PYGZhy{}} \PYG{l+m+mi}{2}\PYG{p}{)}\PYG{o}{/}\PYG{p}{(}\PYG{n}{t}\PYG{o}{+}\PYG{l+m+mi}{1}\PYG{p}{)}

\PYG{c+c1}{\PYGZsh{} invoca metodo}
\PYG{n}{t0} \PYG{o}{=} \PYG{l+m+mf}{0.0}
\PYG{n}{tf} \PYG{o}{=} \PYG{l+m+mf}{6.0}
\PYG{n}{y0} \PYG{o}{=} \PYG{l+m+mf}{2.0}
\PYG{n}{h} \PYG{o}{=} \PYG{l+m+mf}{1.5}
\PYG{n}{t}\PYG{p}{,}\PYG{n}{ymem} \PYG{o}{=} \PYG{n}{euler\PYGZus{}melh}\PYG{p}{(}\PYG{n}{t0}\PYG{p}{,}\PYG{n}{tf}\PYG{p}{,}\PYG{n}{y0}\PYG{p}{,}\PYG{n}{h}\PYG{p}{,}\PYG{n}{f}\PYG{p}{)}
\PYG{n}{t}\PYG{p}{,}\PYG{n}{ymee} \PYG{o}{=} \PYG{n}{euler\PYGZus{}expl}\PYG{p}{(}\PYG{n}{t0}\PYG{p}{,}\PYG{n}{tf}\PYG{p}{,}\PYG{n}{y0}\PYG{p}{,}\PYG{n}{h}\PYG{p}{,}\PYG{n}{f}\PYG{p}{)}

\PYG{c+c1}{\PYGZsh{} plota funcoes }
\PYG{n}{yex} \PYG{o}{=} \PYG{n}{t}\PYG{o}{*}\PYG{o}{*}\PYG{l+m+mi}{2} \PYG{o}{+} \PYG{l+m+mi}{2}\PYG{o}{*}\PYG{n}{t} \PYG{o}{+} \PYG{l+m+mi}{2} \PYG{o}{\PYGZhy{}} \PYG{l+m+mi}{2}\PYG{o}{*}\PYG{p}{(}\PYG{n}{t}\PYG{o}{+}\PYG{l+m+mi}{1}\PYG{p}{)}\PYG{o}{*}\PYG{n}{log}\PYG{p}{(}\PYG{n}{t}\PYG{o}{+}\PYG{l+m+mi}{1}\PYG{p}{)}
\PYG{n}{plt}\PYG{o}{.}\PYG{n}{plot}\PYG{p}{(}\PYG{n}{t}\PYG{p}{,}\PYG{n}{ymem}\PYG{p}{,}\PYG{l+s+s1}{\PYGZsq{}}\PYG{l+s+s1}{b\PYGZhy{}\PYGZhy{}}\PYG{l+s+s1}{\PYGZsq{}}\PYG{p}{,}\PYG{n}{label}\PYG{o}{=}\PYG{l+s+s1}{\PYGZsq{}}\PYG{l+s+s1}{\PYGZdl{}y\PYGZus{}}\PYG{l+s+si}{\PYGZob{}mem\PYGZcb{}}\PYG{l+s+s1}{(t)\PYGZdl{}}\PYG{l+s+s1}{\PYGZsq{}}\PYG{p}{)}
\PYG{n}{plt}\PYG{o}{.}\PYG{n}{plot}\PYG{p}{(}\PYG{n}{t}\PYG{p}{,}\PYG{n}{ymee}\PYG{p}{,}\PYG{l+s+s1}{\PYGZsq{}}\PYG{l+s+s1}{go\PYGZhy{}}\PYG{l+s+s1}{\PYGZsq{}}\PYG{p}{,}\PYG{n}{label}\PYG{o}{=}\PYG{l+s+s1}{\PYGZsq{}}\PYG{l+s+s1}{\PYGZdl{}y\PYGZus{}}\PYG{l+s+si}{\PYGZob{}mee\PYGZcb{}}\PYG{l+s+s1}{(t)\PYGZdl{}}\PYG{l+s+s1}{\PYGZsq{}}\PYG{p}{)}
\PYG{n}{plt}\PYG{o}{.}\PYG{n}{plot}\PYG{p}{(}\PYG{n}{t}\PYG{p}{,}\PYG{n}{yex}\PYG{p}{,}\PYG{l+s+s1}{\PYGZsq{}}\PYG{l+s+s1}{k}\PYG{l+s+s1}{\PYGZsq{}}\PYG{p}{,}\PYG{n}{label}\PYG{o}{=}\PYG{l+s+s1}{\PYGZsq{}}\PYG{l+s+s1}{\PYGZdl{}y(t)\PYGZdl{}}\PYG{l+s+s1}{\PYGZsq{}}\PYG{p}{)}
\PYG{n}{plt}\PYG{o}{.}\PYG{n}{grid}\PYG{p}{(}\PYG{k+kc}{True}\PYG{p}{)}
\PYG{n}{plt}\PYG{o}{.}\PYG{n}{legend}\PYG{p}{(}\PYG{p}{)}
\end{sphinxVerbatim}

\begin{sphinxVerbatim}[commandchars=\\\{\}]
\PYGZlt{}matplotlib.legend.Legend at 0x7f83ac799f10\PYGZgt{}
\end{sphinxVerbatim}

\noindent\sphinxincludegraphics{{extra-eulerMelhorado_5_1}.png}


\section{Método do Ponto Médio (ou Ponto Central)}
\label{\detokenize{extra/extra-eulerMelhorado:metodo-do-ponto-medio-ou-ponto-central}}
\sphinxAtStartPar
O \sphinxstyleemphasis{Método do Ponto Médio} (MPM) é outra variação do Método de Euler e sua dedução é feita de maneira semelhante à anterior. Neste caso, a inclinação utilizada para o avanço é calculada no ponto médio \(t_{i+1/2}\) do passo discreto e dada pela expressão
\begin{equation*}
\begin{split}\dfrac{dy}{dt}\Bigg|_{t = t_i + 1/2} = f( t_{i + 1/2}, y_{i+1/2})\end{split}
\end{equation*}
\sphinxAtStartPar
O MPM pode ser formalmente escrito pelo processo iterativo:
\begin{align*}
w_0 &= \alpha \\ 
t_m &= t_i + \frac{h}{2} \\
w_m &= w_i + \frac{h}{2} f(t_i,w_i) \\
w_{i+1} &=w_i + h f(t_m, w_m), \quad i = 0,1,\dots,N-1
\end{align*}
\sphinxAtStartPar
O MPM é assim chamado pela relação com a fórmula de quadratura de mesmo nome. Juntamente com o MEM, exige maior esforço computacional, mas são recompensados com redução de erro e melhor acurácia. Além disso, ambos são casos particulares da família de métodos de Runge\sphinxhyphen{}Kutta.

\sphinxAtStartPar
\sphinxstylestrong{Exercício:} usando os códigos anteriores como bases, implemente o método do Ponto Médio com o Python.


\chapter{Estabilidade Numérica para o Método de Euler}
\label{\detokenize{extra/extra-estabilidadeEuler:estabilidade-numerica-para-o-metodo-de-euler}}\label{\detokenize{extra/extra-estabilidadeEuler::doc}}
\sphinxAtStartPar
A seguir, aplicaremos um resultado derivado do contexto da estabilidade teórica do PVI ao método de Euler e estudar alguns aspectos de sua estabilidade numérica.

\sphinxAtStartPar
Definamos uma solução numérica \(\{z_n\}\) tal que
\begin{equation*}
\begin{split}z_{n+1} = z_n + h f(t_n,z_n), \quad n = 0,1,\ldots,N(h)-1, \qquad z_0 = y_0 + \epsilon.\end{split}
\end{equation*}
\sphinxAtStartPar
De modo análogo ao que sabemos para o PVI, usamos uma condição inicial perturbada para verificar que \(\{z_n\}\) comporta\sphinxhyphen{}se como uma segunda solução \(\{ y_n \}\) à medida que \(h \to 0\).

\sphinxAtStartPar
Tomemos o erro \(e_n = z_n - y_n, \quad n \geq 0\). Então, \(e_0 = \epsilon\). Subtraindo \(y_{n+1} = y_n + h f(t_n,y_n)\) da anterior, obtemos
\begin{equation*}
\begin{split}e_{n+1} = e_n + h [ f(t_n,z_n) - f(t_n,y_n).\end{split}
\end{equation*}
\sphinxAtStartPar
Isto tem exatamente o mesmo formato que o erro total para o método de Euler considerando nulo o erro de truncamento. A partir do teorema do limite do erro, podemos concluir que (consultar demonstração):
\begin{equation*}
\begin{split}\max_{0 \leq n \leq N(h)} |z_n - y_n| \leq e^{(b-t_0)K}|\epsilon| \leq \kappa |\epsilon|,\end{split}
\end{equation*}
\sphinxAtStartPar
para uma constante \(\kappa \geq 0\). Vale relembrar que \(K\) é a constante de Lipschitz. Este resultado diz\sphinxhyphen{}nos que, ressalvadas as condições teóricas do PVI, a estabilidade numérica do método é encontrada.


\section{Notas sobre análise de estabilidade}
\label{\detokenize{extra/extra-estabilidadeEuler:notas-sobre-analise-de-estabilidade}}
\sphinxAtStartPar
Consideremos o PVI
\begin{equation*}
\begin{split}\begin{cases}
y'(t) = -100y(t)\\
y(0) = y_0
\end{cases}\end{split}
\end{equation*}
\sphinxAtStartPar
cuja solução analítica é \(y(t) = y_0e^{-100t}\). Se \(y_0 > 0\), a solução é positiva e decai rapidamente com \(t \to \infty\). Se o método de Euler explícito for aplicado a este PVI, teremos
\begin{equation*}
\begin{split}w_{n+1} = (1 - 100h)w_n.\end{split}
\end{equation*}
\sphinxAtStartPar
Com um passo \(h = 0.1\), a aproximação numérica \(w_{n+1} = -9w_n\) cresce de acordo com \(w_n = (-9)^n w_0\) e oscilla com uma amplitude crescente sem, de fato, aproximar a solução verdadeira. Ao se reduzir o passo para \(h = 0.001\), a solução numérica transforma\sphinxhyphen{}se para  \(w_{n+1} = 0.9w_n\), de modo que \(w_n = 0.9^n w_0\) significa um decaimento suave.

\sphinxAtStartPar
Reutilizemos nosso código do método de Euler para realizar este teste numérico.

\begin{sphinxVerbatim}[commandchars=\\\{\}]
\PYG{k+kn}{from} \PYG{n+nn}{numpy} \PYG{k+kn}{import} \PYG{o}{*}

\PYG{k}{def} \PYG{n+nf}{euler\PYGZus{}expl}\PYG{p}{(}\PYG{n}{t0}\PYG{p}{,}\PYG{n}{tf}\PYG{p}{,}\PYG{n}{y0}\PYG{p}{,}\PYG{n}{h}\PYG{p}{,}\PYG{n}{fun}\PYG{p}{)}\PYG{p}{:}
    \PYG{l+s+sd}{\PYGZdq{}\PYGZdq{}\PYGZdq{}}
\PYG{l+s+sd}{    Resolve o PVI y\PYGZsq{} = f(t,y), t0 \PYGZlt{}= t \PYGZlt{}= tf, y(t0) = y0}
\PYG{l+s+sd}{    com passo h usando o metodo de Euler explicito. }
\PYG{l+s+sd}{    }
\PYG{l+s+sd}{    Entrada: }
\PYG{l+s+sd}{        t0  \PYGZhy{} tempo inicial}
\PYG{l+s+sd}{        tf  \PYGZhy{} tempo final }
\PYG{l+s+sd}{        y0  \PYGZhy{} condicao inicial }
\PYG{l+s+sd}{        h   \PYGZhy{} passo }
\PYG{l+s+sd}{        fun \PYGZhy{} funcao f(t,y) (anonima)}
\PYG{l+s+sd}{        }
\PYG{l+s+sd}{    Saida:}
\PYG{l+s+sd}{        t   \PYGZhy{} nos da malha numerica }
\PYG{l+s+sd}{        y   \PYGZhy{} solucao aproximada}
\PYG{l+s+sd}{    \PYGZdq{}\PYGZdq{}\PYGZdq{}}
    
    \PYG{n}{n} \PYG{o}{=} \PYG{n+nb}{round}\PYG{p}{(}\PYG{p}{(}\PYG{n}{tf} \PYG{o}{\PYGZhy{}} \PYG{n}{t0}\PYG{p}{)}\PYG{o}{/}\PYG{n}{h} \PYG{o}{+} \PYG{l+m+mi}{1}\PYG{p}{)}
    \PYG{n}{t} \PYG{o}{=} \PYG{n}{linspace}\PYG{p}{(}\PYG{n}{t0}\PYG{p}{,}\PYG{n}{t0}\PYG{o}{+}\PYG{p}{(}\PYG{n}{n}\PYG{o}{\PYGZhy{}}\PYG{l+m+mi}{1}\PYG{p}{)}\PYG{o}{*}\PYG{n}{h}\PYG{p}{,}\PYG{n}{n}\PYG{p}{)}
    \PYG{n}{y} \PYG{o}{=} \PYG{n}{linspace}\PYG{p}{(}\PYG{n}{t0}\PYG{p}{,}\PYG{n}{t0}\PYG{o}{+}\PYG{p}{(}\PYG{n}{n}\PYG{o}{\PYGZhy{}}\PYG{l+m+mi}{1}\PYG{p}{)}\PYG{o}{*}\PYG{n}{h}\PYG{p}{,}\PYG{n}{n}\PYG{p}{)}
    \PYG{n}{y} \PYG{o}{=} \PYG{n}{zeros}\PYG{p}{(}\PYG{p}{(}\PYG{n}{n}\PYG{p}{,}\PYG{p}{)}\PYG{p}{)}
    
    \PYG{n}{y}\PYG{p}{[}\PYG{l+m+mi}{0}\PYG{p}{]} \PYG{o}{=} \PYG{n}{y0}

    \PYG{k}{for} \PYG{n}{i} \PYG{o+ow}{in} \PYG{n+nb}{range}\PYG{p}{(}\PYG{l+m+mi}{1}\PYG{p}{,}\PYG{n}{n}\PYG{p}{)}\PYG{p}{:}
        \PYG{n}{y}\PYG{p}{[}\PYG{n}{i}\PYG{p}{]} \PYG{o}{=} \PYG{n}{y}\PYG{p}{[}\PYG{n}{i}\PYG{o}{\PYGZhy{}}\PYG{l+m+mi}{1}\PYG{p}{]} \PYG{o}{+} \PYG{n}{h}\PYG{o}{*}\PYG{n}{f}\PYG{p}{(}\PYG{n}{t}\PYG{p}{[}\PYG{n}{i}\PYG{o}{\PYGZhy{}}\PYG{l+m+mi}{1}\PYG{p}{]}\PYG{p}{,}\PYG{n}{y}\PYG{p}{[}\PYG{n}{i}\PYG{o}{\PYGZhy{}}\PYG{l+m+mi}{1}\PYG{p}{]}\PYG{p}{)}

    \PYG{k}{return} \PYG{p}{(}\PYG{n}{t}\PYG{p}{,}\PYG{n}{y}\PYG{p}{)}
\end{sphinxVerbatim}

\begin{sphinxVerbatim}[commandchars=\\\{\}]
\PYG{k+kn}{import} \PYG{n+nn}{matplotlib}\PYG{n+nn}{.}\PYG{n+nn}{pyplot} \PYG{k}{as} \PYG{n+nn}{plt}
\PYG{n}{f} \PYG{o}{=} \PYG{k}{lambda} \PYG{n}{t}\PYG{p}{,}\PYG{n}{y}\PYG{p}{:} \PYG{o}{\PYGZhy{}}\PYG{l+m+mi}{100}\PYG{o}{*}\PYG{n}{y}

\PYG{c+c1}{\PYGZsh{} h = 0.1}
\PYG{n}{plt}\PYG{o}{.}\PYG{n}{subplot}\PYG{p}{(}\PYG{l+m+mi}{121}\PYG{p}{)}
\PYG{k}{for} \PYG{n}{y0} \PYG{o+ow}{in} \PYG{p}{[}\PYG{l+m+mf}{0.5}\PYG{p}{,}\PYG{l+m+mi}{1}\PYG{p}{,}\PYG{l+m+mi}{2}\PYG{p}{,}\PYG{l+m+mi}{5}\PYG{p}{]}\PYG{p}{:}
    \PYG{p}{(}\PYG{n}{t}\PYG{p}{,}\PYG{n}{ynum}\PYG{p}{)} \PYG{o}{=} \PYG{n}{euler\PYGZus{}expl}\PYG{p}{(}\PYG{l+m+mi}{0}\PYG{p}{,}\PYG{l+m+mi}{2}\PYG{p}{,}\PYG{n}{y0}\PYG{p}{,}\PYG{l+m+mf}{0.1}\PYG{p}{,}\PYG{n}{f}\PYG{p}{)}    
    \PYG{n}{plt}\PYG{o}{.}\PYG{n}{plot}\PYG{p}{(}\PYG{n}{t}\PYG{p}{,}\PYG{n}{ynum}\PYG{p}{)}
    \PYG{n}{plt}\PYG{o}{.}\PYG{n}{title}\PYG{p}{(}\PYG{l+s+s1}{\PYGZsq{}}\PYG{l+s+s1}{\PYGZdl{}h=0.1\PYGZdl{}}\PYG{l+s+s1}{\PYGZsq{}}\PYG{p}{)}

\PYG{c+c1}{\PYGZsh{} h = 0.001}
\PYG{n}{plt}\PYG{o}{.}\PYG{n}{subplot}\PYG{p}{(}\PYG{l+m+mi}{122}\PYG{p}{)}    
\PYG{k}{for} \PYG{n}{y0} \PYG{o+ow}{in} \PYG{p}{[}\PYG{l+m+mf}{0.5}\PYG{p}{,}\PYG{l+m+mi}{1}\PYG{p}{,}\PYG{l+m+mi}{2}\PYG{p}{,}\PYG{l+m+mi}{5}\PYG{p}{]}\PYG{p}{:}
    \PYG{p}{(}\PYG{n}{t}\PYG{p}{,}\PYG{n}{ynum}\PYG{p}{)} \PYG{o}{=} \PYG{n}{euler\PYGZus{}expl}\PYG{p}{(}\PYG{l+m+mi}{0}\PYG{p}{,}\PYG{l+m+mi}{2}\PYG{p}{,}\PYG{n}{y0}\PYG{p}{,}\PYG{l+m+mf}{0.001}\PYG{p}{,}\PYG{n}{f}\PYG{p}{)}    
    \PYG{n}{plt}\PYG{o}{.}\PYG{n}{plot}\PYG{p}{(}\PYG{n}{t}\PYG{p}{,}\PYG{n}{ynum}\PYG{p}{)}
    \PYG{n}{plt}\PYG{o}{.}\PYG{n}{title}\PYG{p}{(}\PYG{l+s+s1}{\PYGZsq{}}\PYG{l+s+s1}{\PYGZdl{}h=0.001\PYGZdl{}}\PYG{l+s+s1}{\PYGZsq{}}\PYG{p}{)}
\end{sphinxVerbatim}

\noindent\sphinxincludegraphics{{extra-estabilidadeEuler_2_0}.png}

\sphinxAtStartPar
\sphinxstylestrong{Exercício:} Repita o mesmo experimento numérico anterior para o PVI
\begin{equation*}
\begin{split}\begin{cases}
y'(t) = \lambda(y - \textrm{sen}(t)) + \cos(t) \\
y(\pi/4) = 1/2
\end{cases}\end{split}
\end{equation*}
\sphinxAtStartPar
para \(\lambda = 2\) e \(\lambda = - 0.2\). Compare as soluções numéricas obtidas pelo método de Euler para \(h = \pi/10\) e \(h = \pi/20\) para cada valor de \(\lambda\).


\subsection{Região de estabilidade}
\label{\detokenize{extra/extra-estabilidadeEuler:regiao-de-estabilidade}}
\sphinxAtStartPar
Considere o PVI mais geral
\begin{equation*}
\begin{split}\begin{cases}
y'(t) = \lambda y(t)\\
y(0) = y_0
\end{cases},\end{split}
\end{equation*}
\sphinxAtStartPar
em que \(\lambda \in \mathbb{C}\). A EDO teste é estável, i.e. a taxa de crescimento da perturbação é limitada quando \(\Re\{\lambda\} = \alpha < 0\). Neste, a solução numérica pelo método de Euler explícito corresponde ao processo
\begin{equation*}
\begin{split}w_{n+1} = (1 + h\lambda)w_n\end{split}
\end{equation*}
\sphinxAtStartPar
e a solução numérica será estável se, e somente se, o \sphinxstyleemphasis{fator de amplificação} for limitado por 1, ou seja,
\begin{equation*}
\begin{split}|1 + h\lambda| \leq 1.\end{split}
\end{equation*}
\sphinxAtStartPar
Assim, diz\sphinxhyphen{}se que o método de Euler será \sphinxstylestrong{estável} para esta EDO teste para valores de \(h\) que satisfizerem a condição acima (para \(\lambda\) real e negativo, \(h \leq -2/\lambda\)). O conjunto
\begin{equation*}
\begin{split}\mathcal{S} = \{ h \lambda \in \mathbb{C}; |1 + h\lambda| \leq 1 \}\end{split}
\end{equation*}
\sphinxAtStartPar
é chamado de \sphinxstyleemphasis{região de estabilidade} para o método de Euler e trata\sphinxhyphen{}se de um disco de raio unitário centrado em \((-1,0)\) no plano de Argand\sphinxhyphen{}Gauss.


\chapter{Método de Euler Implícito}
\label{\detokenize{extra/extra-eulerImplicito:metodo-de-euler-implicito}}\label{\detokenize{extra/extra-eulerImplicito::doc}}
\sphinxAtStartPar
A forma do \sphinxstyleemphasis{Método de Euler Implícito} (MEI) é similar àquela do MEE, exceto por uma característica distintiva. Em vez de a inclinação utilizada para avanço ser tomada em \(t_i\), ela é tomada em \(t_{i+1}\). Isto é, o processo numérico avança com a estimativa \(f(t_{i+1},y_{i+1})\) e não \(f(t_i,y_i)\). O fato de usar esta inclinação ainda desconhecida é a razão de o método ser denominado “implícito”. O esquema numérico resultante é o processo iterativo:
\begin{align*}
w_0 &= \alpha \\
w_{i+1} &= w_i + h f(t_{i+1},w_{i+1}) \ \ i = 0,1,\dots,N-1.
\end{align*}
\sphinxAtStartPar
Note, entretanto, que \(w_{i+1}\) aparece não apenas no lado esquerdo, mas também no lado direito da equaçõe. Esta incógnita nem sempre é obtenível de modo explícito, isto é, por isolamento. Quando este é o caso, \(f\) é linear ou uma função simples, mas, em geral, \(f\) é não\sphinxhyphen{}linear e dependente do valor futuro \(w_{i+1}\).


\chapter{Código preditor/corretor}
\label{\detokenize{extra/extra-eulerImplicito:codigo-preditor-corretor}}
\begin{sphinxVerbatim}[commandchars=\\\{\}]
\PYG{k+kn}{import} \PYG{n+nn}{matplotlib}\PYG{n+nn}{.}\PYG{n+nn}{pyplot} \PYG{k}{as} \PYG{n+nn}{plt}
\end{sphinxVerbatim}

\begin{sphinxVerbatim}[commandchars=\\\{\}]
\PYG{c+c1}{\PYGZsh{} MEI: preditor/corretor}

\PYG{k}{def} \PYG{n+nf}{euler\PYGZus{}impl}\PYG{p}{(}\PYG{n}{t0}\PYG{p}{,}\PYG{n}{tf}\PYG{p}{,}\PYG{n}{y0}\PYG{p}{,}\PYG{n}{h}\PYG{p}{,}\PYG{n}{f}\PYG{p}{,}\PYG{n}{tol}\PYG{p}{)}\PYG{p}{:}
    
    \PYG{n}{n} \PYG{o}{=} \PYG{n+nb}{round}\PYG{p}{(}\PYG{p}{(}\PYG{n}{tf} \PYG{o}{\PYGZhy{}} \PYG{n}{t0}\PYG{p}{)}\PYG{o}{/}\PYG{n}{h}\PYG{p}{)} \PYG{o}{+} \PYG{l+m+mi}{1}
    \PYG{n}{t} \PYG{o}{=} \PYG{n}{linspace}\PYG{p}{(}\PYG{n}{t0}\PYG{p}{,}\PYG{n}{t0}\PYG{o}{+}\PYG{p}{(}\PYG{n}{n}\PYG{o}{\PYGZhy{}}\PYG{l+m+mi}{1}\PYG{p}{)}\PYG{o}{*}\PYG{n}{h}\PYG{p}{,}\PYG{n}{n}\PYG{p}{)}
    \PYG{n}{y} \PYG{o}{=} \PYG{n}{zeros}\PYG{p}{(}\PYG{n}{n}\PYG{p}{)}
    \PYG{n}{y}\PYG{p}{[}\PYG{l+m+mi}{0}\PYG{p}{]} \PYG{o}{=} \PYG{n}{y0}
    
    \PYG{n}{i} \PYG{o}{=} \PYG{l+m+mi}{1}
    \PYG{c+c1}{\PYGZsh{} iteracoes }
    \PYG{k}{while} \PYG{n}{i} \PYG{o}{\PYGZlt{}} \PYG{n}{n}\PYG{p}{:}
        
        \PYG{c+c1}{\PYGZsh{} preditor (MEE)}
        \PYG{n}{yf} \PYG{o}{=} \PYG{n}{y}\PYG{p}{[}\PYG{n}{i}\PYG{o}{\PYGZhy{}}\PYG{l+m+mi}{1}\PYG{p}{]} \PYG{o}{+} \PYG{n}{h}\PYG{o}{*}\PYG{n}{f}\PYG{p}{(}\PYG{n}{t}\PYG{p}{[}\PYG{n}{i}\PYG{o}{\PYGZhy{}}\PYG{l+m+mi}{1}\PYG{p}{]}\PYG{p}{,}\PYG{n}{y}\PYG{p}{[}\PYG{n}{i}\PYG{o}{\PYGZhy{}}\PYG{l+m+mi}{1}\PYG{p}{]}\PYG{p}{)}
        
        \PYG{c+c1}{\PYGZsh{} iteracoes internas (maximo 10)}
        \PYG{n}{count} \PYG{o}{=} \PYG{l+m+mi}{0}
        \PYG{n}{diff} \PYG{o}{=} \PYG{l+m+mf}{1.0}
        \PYG{k}{while} \PYG{n}{diff} \PYG{o}{\PYGZgt{}} \PYG{n}{tol} \PYG{o+ow}{and} \PYG{n}{count} \PYG{o}{\PYGZlt{}} \PYG{l+m+mi}{10}\PYG{p}{:}
            
            \PYG{c+c1}{\PYGZsh{} corretor (MEI)}
            \PYG{n}{yf2} \PYG{o}{=} \PYG{n}{y}\PYG{p}{[}\PYG{n}{i}\PYG{o}{\PYGZhy{}}\PYG{l+m+mi}{1}\PYG{p}{]} \PYG{o}{+} \PYG{n}{h}\PYG{o}{*}\PYG{n}{f}\PYG{p}{(}\PYG{n}{t}\PYG{p}{[}\PYG{n}{i}\PYG{p}{]}\PYG{p}{,}\PYG{n}{yf}\PYG{p}{)}
            
            \PYG{n}{diff} \PYG{o}{=} \PYG{n+nb}{abs}\PYG{p}{(}\PYG{n}{yf2}\PYG{o}{\PYGZhy{}}\PYG{n}{yf}\PYG{p}{)}
            \PYG{n}{yf} \PYG{o}{=} \PYG{n}{yf2}
            \PYG{n}{count} \PYG{o}{+}\PYG{o}{=} \PYG{l+m+mi}{1}
        
        \PYG{k}{if} \PYG{n}{count} \PYG{o}{\PYGZgt{}}\PYG{o}{=} \PYG{l+m+mi}{10}\PYG{p}{:}
            \PYG{n+nb}{print}\PYG{p}{(}\PYG{l+s+s1}{\PYGZsq{}}\PYG{l+s+s1}{Nao convergindo apos 10 passos em t = }\PYG{l+s+si}{\PYGZob{}0:f\PYGZcb{}}\PYG{l+s+s1}{\PYGZsq{}}\PYG{o}{.}\PYG{n}{format}\PYG{p}{(}\PYG{n}{t}\PYG{p}{[}\PYG{n}{i}\PYG{p}{]}\PYG{p}{)}\PYG{p}{)}
        
        \PYG{n}{y}\PYG{p}{[}\PYG{n}{i}\PYG{p}{]} \PYG{o}{=} \PYG{n}{yf2}
        \PYG{n}{i} \PYG{o}{+}\PYG{o}{=} \PYG{l+m+mi}{1}
    
    \PYG{k}{return} \PYG{n}{t}\PYG{p}{,}\PYG{n}{y}
\end{sphinxVerbatim}

\begin{sphinxVerbatim}[commandchars=\\\{\}]
\PYG{k+kn}{from} \PYG{n+nn}{numpy} \PYG{k+kn}{import} \PYG{n}{linspace}\PYG{p}{,} \PYG{n}{zeros}

\PYG{k}{def} \PYG{n+nf}{euler\PYGZus{}expl}\PYG{p}{(}\PYG{n}{t0}\PYG{p}{,}\PYG{n}{tf}\PYG{p}{,}\PYG{n}{y0}\PYG{p}{,}\PYG{n}{h}\PYG{p}{,}\PYG{n}{fun}\PYG{p}{)}\PYG{p}{:}
    \PYG{l+s+sd}{\PYGZdq{}\PYGZdq{}\PYGZdq{}}
\PYG{l+s+sd}{    Resolve o PVI y\PYGZsq{} = f(t,y), t0 \PYGZlt{}= t \PYGZlt{}= tf, y(t0) = y0}
\PYG{l+s+sd}{    com passo h usando o metodo de Euler explicito. }
\PYG{l+s+sd}{    }
\PYG{l+s+sd}{    Entrada: }
\PYG{l+s+sd}{        t0  \PYGZhy{} tempo inicial}
\PYG{l+s+sd}{        tf  \PYGZhy{} tempo final }
\PYG{l+s+sd}{        y0  \PYGZhy{} condicao inicial }
\PYG{l+s+sd}{        h   \PYGZhy{} passo }
\PYG{l+s+sd}{        fun \PYGZhy{} funcao f(t,y) (anonima)}
\PYG{l+s+sd}{        }
\PYG{l+s+sd}{    Saida:}
\PYG{l+s+sd}{        t   \PYGZhy{} nos da malha numerica }
\PYG{l+s+sd}{        y   \PYGZhy{} solucao aproximada}
\PYG{l+s+sd}{    \PYGZdq{}\PYGZdq{}\PYGZdq{}}
    
    \PYG{n}{n} \PYG{o}{=} \PYG{n+nb}{round}\PYG{p}{(}\PYG{p}{(}\PYG{n}{tf} \PYG{o}{\PYGZhy{}} \PYG{n}{t0}\PYG{p}{)}\PYG{o}{/}\PYG{n}{h}\PYG{p}{)} \PYG{o}{+} \PYG{l+m+mi}{1}
    \PYG{n}{t} \PYG{o}{=} \PYG{n}{linspace}\PYG{p}{(}\PYG{n}{t0}\PYG{p}{,}\PYG{n}{t0}\PYG{o}{+}\PYG{p}{(}\PYG{n}{n}\PYG{o}{\PYGZhy{}}\PYG{l+m+mi}{1}\PYG{p}{)}\PYG{o}{*}\PYG{n}{h}\PYG{p}{,}\PYG{n}{n}\PYG{p}{)}
    \PYG{n}{y} \PYG{o}{=} \PYG{n}{linspace}\PYG{p}{(}\PYG{n}{t0}\PYG{p}{,}\PYG{n}{t0}\PYG{o}{+}\PYG{p}{(}\PYG{n}{n}\PYG{o}{\PYGZhy{}}\PYG{l+m+mi}{1}\PYG{p}{)}\PYG{o}{*}\PYG{n}{h}\PYG{p}{,}\PYG{n}{n}\PYG{p}{)}
    \PYG{n}{y} \PYG{o}{=} \PYG{n}{zeros}\PYG{p}{(}\PYG{p}{(}\PYG{n}{n}\PYG{p}{,}\PYG{p}{)}\PYG{p}{)}
    
    \PYG{n}{y}\PYG{p}{[}\PYG{l+m+mi}{0}\PYG{p}{]} \PYG{o}{=} \PYG{n}{y0}

    \PYG{k}{for} \PYG{n}{i} \PYG{o+ow}{in} \PYG{n+nb}{range}\PYG{p}{(}\PYG{l+m+mi}{1}\PYG{p}{,}\PYG{n}{n}\PYG{p}{)}\PYG{p}{:}
        \PYG{n}{y}\PYG{p}{[}\PYG{n}{i}\PYG{p}{]} \PYG{o}{=} \PYG{n}{y}\PYG{p}{[}\PYG{n}{i}\PYG{o}{\PYGZhy{}}\PYG{l+m+mi}{1}\PYG{p}{]} \PYG{o}{+} \PYG{n}{h}\PYG{o}{*}\PYG{n}{f}\PYG{p}{(}\PYG{n}{t}\PYG{p}{[}\PYG{n}{i}\PYG{o}{\PYGZhy{}}\PYG{l+m+mi}{1}\PYG{p}{]}\PYG{p}{,}\PYG{n}{y}\PYG{p}{[}\PYG{n}{i}\PYG{o}{\PYGZhy{}}\PYG{l+m+mi}{1}\PYG{p}{]}\PYG{p}{)}

    \PYG{k}{return} \PYG{p}{(}\PYG{n}{t}\PYG{p}{,}\PYG{n}{y}\PYG{p}{)}
\end{sphinxVerbatim}

\begin{sphinxVerbatim}[commandchars=\\\{\}]
\PYG{k+kn}{from} \PYG{n+nn}{numpy} \PYG{k+kn}{import} \PYG{n}{log} 

\PYG{c+c1}{\PYGZsh{} define funcao}
\PYG{n}{f} \PYG{o}{=} \PYG{k}{lambda} \PYG{n}{t}\PYG{p}{,}\PYG{n}{y}\PYG{p}{:} \PYG{p}{(}\PYG{n}{y} \PYG{o}{+} \PYG{n}{t}\PYG{o}{*}\PYG{o}{*}\PYG{l+m+mi}{2} \PYG{o}{\PYGZhy{}} \PYG{l+m+mi}{2}\PYG{p}{)}\PYG{o}{/}\PYG{p}{(}\PYG{n}{t}\PYG{o}{+}\PYG{l+m+mi}{1}\PYG{p}{)}

\PYG{c+c1}{\PYGZsh{} invoca metodo}
\PYG{n}{t0} \PYG{o}{=} \PYG{l+m+mf}{0.0}
\PYG{n}{tf} \PYG{o}{=} \PYG{l+m+mf}{6.0}
\PYG{n}{y0} \PYG{o}{=} \PYG{l+m+mf}{2.0}
\PYG{n}{h} \PYG{o}{=} \PYG{l+m+mf}{0.5}
\PYG{n}{tol} \PYG{o}{=} \PYG{l+m+mf}{1e\PYGZhy{}3}
\PYG{n}{t}\PYG{p}{,}\PYG{n}{y1} \PYG{o}{=} \PYG{n}{euler\PYGZus{}expl}\PYG{p}{(}\PYG{n}{t0}\PYG{p}{,}\PYG{n}{tf}\PYG{p}{,}\PYG{n}{y0}\PYG{p}{,}\PYG{n}{h}\PYG{p}{,}\PYG{n}{f}\PYG{p}{)}
\PYG{n}{t}\PYG{p}{,}\PYG{n}{y2} \PYG{o}{=} \PYG{n}{euler\PYGZus{}impl}\PYG{p}{(}\PYG{n}{t0}\PYG{p}{,}\PYG{n}{tf}\PYG{p}{,}\PYG{n}{y0}\PYG{p}{,}\PYG{n}{h}\PYG{p}{,}\PYG{n}{f}\PYG{p}{,}\PYG{n}{tol}\PYG{p}{)}


\PYG{c+c1}{\PYGZsh{} plota funcoes }
\PYG{n}{yex} \PYG{o}{=} \PYG{n}{t}\PYG{o}{*}\PYG{o}{*}\PYG{l+m+mi}{2} \PYG{o}{+} \PYG{l+m+mi}{2}\PYG{o}{*}\PYG{n}{t} \PYG{o}{+} \PYG{l+m+mi}{2} \PYG{o}{\PYGZhy{}} \PYG{l+m+mi}{2}\PYG{o}{*}\PYG{p}{(}\PYG{n}{t}\PYG{o}{+}\PYG{l+m+mi}{1}\PYG{p}{)}\PYG{o}{*}\PYG{n}{log}\PYG{p}{(}\PYG{n}{t}\PYG{o}{+}\PYG{l+m+mi}{1}\PYG{p}{)}
\PYG{n}{plt}\PYG{o}{.}\PYG{n}{plot}\PYG{p}{(}\PYG{n}{t}\PYG{p}{,}\PYG{n}{y1}\PYG{p}{,}\PYG{l+s+s1}{\PYGZsq{}}\PYG{l+s+s1}{b\PYGZhy{}}\PYG{l+s+s1}{\PYGZsq{}}\PYG{p}{,}\PYG{n}{label}\PYG{o}{=}\PYG{l+s+s1}{\PYGZsq{}}\PYG{l+s+s1}{\PYGZdl{}y\PYGZus{}}\PYG{l+s+si}{\PYGZob{}MEE\PYGZcb{}}\PYG{l+s+s1}{(t)\PYGZdl{}}\PYG{l+s+s1}{\PYGZsq{}}\PYG{p}{)}
\PYG{n}{plt}\PYG{o}{.}\PYG{n}{plot}\PYG{p}{(}\PYG{n}{t}\PYG{p}{,}\PYG{n}{y2}\PYG{p}{,}\PYG{l+s+s1}{\PYGZsq{}}\PYG{l+s+s1}{r\PYGZhy{}}\PYG{l+s+s1}{\PYGZsq{}}\PYG{p}{,}\PYG{n}{label}\PYG{o}{=}\PYG{l+s+s1}{\PYGZsq{}}\PYG{l+s+s1}{\PYGZdl{}y\PYGZus{}}\PYG{l+s+si}{\PYGZob{}MEI\PYGZcb{}}\PYG{l+s+s1}{(t)\PYGZdl{}}\PYG{l+s+s1}{\PYGZsq{}}\PYG{p}{)}
\PYG{n}{plt}\PYG{o}{.}\PYG{n}{plot}\PYG{p}{(}\PYG{n}{t}\PYG{p}{,}\PYG{n}{yex}\PYG{p}{,}\PYG{l+s+s1}{\PYGZsq{}}\PYG{l+s+s1}{k\PYGZhy{}\PYGZhy{}}\PYG{l+s+s1}{\PYGZsq{}}\PYG{p}{,}\PYG{n}{label}\PYG{o}{=}\PYG{l+s+s1}{\PYGZsq{}}\PYG{l+s+s1}{\PYGZdl{}y(t)\PYGZdl{}}\PYG{l+s+s1}{\PYGZsq{}}\PYG{p}{)}
\PYG{n}{plt}\PYG{o}{.}\PYG{n}{legend}\PYG{p}{(}\PYG{p}{)}\PYG{p}{;}
\end{sphinxVerbatim}

\noindent\sphinxincludegraphics{{extra-eulerImplicito_4_0}.png}


\chapter{Método Trapezoidal}
\label{\detokenize{extra/extra-eulerImplicito:metodo-trapezoidal}}
\begin{sphinxVerbatim}[commandchars=\\\{\}]
\PYG{k}{def} \PYG{n+nf}{trapezoidal}\PYG{p}{(}\PYG{n}{t0}\PYG{p}{,}\PYG{n}{tf}\PYG{p}{,}\PYG{n}{y0}\PYG{p}{,}\PYG{n}{h}\PYG{p}{,}\PYG{n}{f}\PYG{p}{,}\PYG{n}{tol}\PYG{p}{)}\PYG{p}{:}
    
    \PYG{n}{n} \PYG{o}{=} \PYG{n+nb}{round}\PYG{p}{(}\PYG{p}{(}\PYG{n}{tf} \PYG{o}{\PYGZhy{}} \PYG{n}{t0}\PYG{p}{)}\PYG{o}{/}\PYG{n}{h}\PYG{p}{)} \PYG{o}{+} \PYG{l+m+mi}{1}
    \PYG{n}{t} \PYG{o}{=} \PYG{n}{linspace}\PYG{p}{(}\PYG{n}{t0}\PYG{p}{,}\PYG{n}{t0}\PYG{o}{+}\PYG{p}{(}\PYG{n}{n}\PYG{o}{\PYGZhy{}}\PYG{l+m+mi}{1}\PYG{p}{)}\PYG{o}{*}\PYG{n}{h}\PYG{p}{,}\PYG{n}{n}\PYG{p}{)}
    \PYG{n}{y} \PYG{o}{=} \PYG{n}{zeros}\PYG{p}{(}\PYG{n}{n}\PYG{p}{)}
    \PYG{n}{y}\PYG{p}{[}\PYG{l+m+mi}{0}\PYG{p}{]} \PYG{o}{=} \PYG{n}{y0}
    
    \PYG{n}{i} \PYG{o}{=} \PYG{l+m+mi}{1}
    \PYG{c+c1}{\PYGZsh{} iteracoes }
    \PYG{k}{while} \PYG{n}{i} \PYG{o}{\PYGZlt{}} \PYG{n}{n}\PYG{p}{:}
        
        \PYG{c+c1}{\PYGZsh{} f(tn,yn)        }
        \PYG{n}{fyt} \PYG{o}{=} \PYG{n}{f}\PYG{p}{(}\PYG{n}{t}\PYG{p}{[}\PYG{n}{i}\PYG{o}{\PYGZhy{}}\PYG{l+m+mi}{1}\PYG{p}{]}\PYG{p}{,}\PYG{n}{y}\PYG{p}{[}\PYG{n}{i}\PYG{o}{\PYGZhy{}}\PYG{l+m+mi}{1}\PYG{p}{]}\PYG{p}{)}

        \PYG{c+c1}{\PYGZsh{} Euler}
        \PYG{n}{yt1} \PYG{o}{=} \PYG{n}{y}\PYG{p}{[}\PYG{n}{i}\PYG{o}{\PYGZhy{}}\PYG{l+m+mi}{1}\PYG{p}{]} \PYG{o}{+} \PYG{n}{h}\PYG{o}{*}\PYG{n}{fyt}
                
        \PYG{c+c1}{\PYGZsh{} iteracoes internas (maximo 10)}
        \PYG{n}{count} \PYG{o}{=} \PYG{l+m+mi}{0}
        \PYG{n}{diff} \PYG{o}{=} \PYG{l+m+mf}{1.0}
        \PYG{k}{while} \PYG{n}{diff} \PYG{o}{\PYGZgt{}} \PYG{n}{tol} \PYG{o+ow}{and} \PYG{n}{count} \PYG{o}{\PYGZlt{}} \PYG{l+m+mi}{10}\PYG{p}{:}
            
            \PYG{c+c1}{\PYGZsh{} corretor (Trapezoidal)}
            \PYG{n}{yt2} \PYG{o}{=} \PYG{n}{y}\PYG{p}{[}\PYG{n}{i}\PYG{o}{\PYGZhy{}}\PYG{l+m+mi}{1}\PYG{p}{]} \PYG{o}{+} \PYG{l+m+mf}{0.5}\PYG{o}{*}\PYG{n}{h}\PYG{o}{*}\PYG{p}{(} \PYG{n}{fyt} \PYG{o}{+} \PYG{n}{f}\PYG{p}{(}\PYG{n}{t}\PYG{p}{[}\PYG{n}{i}\PYG{p}{]}\PYG{p}{,}\PYG{n}{yt1}\PYG{p}{)} \PYG{p}{)}
            \PYG{n}{diff} \PYG{o}{=} \PYG{n+nb}{abs}\PYG{p}{(}\PYG{n}{yt2}\PYG{o}{\PYGZhy{}}\PYG{n}{yt1}\PYG{p}{)}
            \PYG{n}{yt1} \PYG{o}{=} \PYG{n}{yt2}
            \PYG{n}{count} \PYG{o}{+}\PYG{o}{=} \PYG{l+m+mi}{1}
        
        \PYG{k}{if} \PYG{n}{count} \PYG{o}{\PYGZgt{}}\PYG{o}{=} \PYG{l+m+mi}{10}\PYG{p}{:}
            \PYG{n+nb}{print}\PYG{p}{(}\PYG{l+s+s1}{\PYGZsq{}}\PYG{l+s+s1}{Nao convergindo apos 10 passos em t = }\PYG{l+s+si}{\PYGZob{}0:f\PYGZcb{}}\PYG{l+s+s1}{\PYGZsq{}}\PYG{o}{.}\PYG{n}{format}\PYG{p}{(}\PYG{n}{t}\PYG{p}{[}\PYG{n}{i}\PYG{p}{]}\PYG{p}{)}\PYG{p}{)}
        
        \PYG{n}{y}\PYG{p}{[}\PYG{n}{i}\PYG{p}{]} \PYG{o}{=} \PYG{n}{yt2}
        \PYG{n}{i} \PYG{o}{+}\PYG{o}{=} \PYG{l+m+mi}{1}
    
    \PYG{k}{return} \PYG{n}{t}\PYG{p}{,}\PYG{n}{y}
\end{sphinxVerbatim}

\begin{sphinxVerbatim}[commandchars=\\\{\}]
\PYG{c+c1}{\PYGZsh{} define funcao}
\PYG{n}{f} \PYG{o}{=} \PYG{k}{lambda} \PYG{n}{t}\PYG{p}{,}\PYG{n}{y}\PYG{p}{:} \PYG{p}{(}\PYG{n}{y} \PYG{o}{+} \PYG{n}{t}\PYG{o}{*}\PYG{o}{*}\PYG{l+m+mi}{2} \PYG{o}{\PYGZhy{}} \PYG{l+m+mi}{2}\PYG{p}{)}\PYG{o}{/}\PYG{p}{(}\PYG{n}{t}\PYG{o}{+}\PYG{l+m+mi}{1}\PYG{p}{)}

\PYG{c+c1}{\PYGZsh{} invoca metodo}
\PYG{n}{t0} \PYG{o}{=} \PYG{l+m+mf}{0.0}
\PYG{n}{tf} \PYG{o}{=} \PYG{l+m+mf}{6.0}
\PYG{n}{y0} \PYG{o}{=} \PYG{l+m+mf}{2.0}
\PYG{n}{h} \PYG{o}{=} \PYG{l+m+mf}{1.0}
\PYG{n}{tol} \PYG{o}{=} \PYG{l+m+mf}{1e\PYGZhy{}3}
\PYG{n}{t}\PYG{p}{,}\PYG{n}{y1} \PYG{o}{=} \PYG{n}{euler\PYGZus{}expl}\PYG{p}{(}\PYG{n}{t0}\PYG{p}{,}\PYG{n}{tf}\PYG{p}{,}\PYG{n}{y0}\PYG{p}{,}\PYG{n}{h}\PYG{p}{,}\PYG{n}{f}\PYG{p}{)}
\PYG{n}{t}\PYG{p}{,}\PYG{n}{y2} \PYG{o}{=} \PYG{n}{euler\PYGZus{}impl}\PYG{p}{(}\PYG{n}{t0}\PYG{p}{,}\PYG{n}{tf}\PYG{p}{,}\PYG{n}{y0}\PYG{p}{,}\PYG{n}{h}\PYG{p}{,}\PYG{n}{f}\PYG{p}{,}\PYG{n}{tol}\PYG{p}{)}
\PYG{n}{t}\PYG{p}{,}\PYG{n}{y3} \PYG{o}{=} \PYG{n}{trapezoidal}\PYG{p}{(}\PYG{n}{t0}\PYG{p}{,}\PYG{n}{tf}\PYG{p}{,}\PYG{n}{y0}\PYG{p}{,}\PYG{n}{h}\PYG{p}{,}\PYG{n}{f}\PYG{p}{,}\PYG{n}{tol}\PYG{p}{)}


\PYG{c+c1}{\PYGZsh{} plota funcoes }
\PYG{n}{yex} \PYG{o}{=} \PYG{n}{t}\PYG{o}{*}\PYG{o}{*}\PYG{l+m+mi}{2} \PYG{o}{+} \PYG{l+m+mi}{2}\PYG{o}{*}\PYG{n}{t} \PYG{o}{+} \PYG{l+m+mi}{2} \PYG{o}{\PYGZhy{}} \PYG{l+m+mi}{2}\PYG{o}{*}\PYG{p}{(}\PYG{n}{t}\PYG{o}{+}\PYG{l+m+mi}{1}\PYG{p}{)}\PYG{o}{*}\PYG{n}{log}\PYG{p}{(}\PYG{n}{t}\PYG{o}{+}\PYG{l+m+mi}{1}\PYG{p}{)}
\PYG{n}{plt}\PYG{o}{.}\PYG{n}{plot}\PYG{p}{(}\PYG{n}{t}\PYG{p}{,}\PYG{n}{y1}\PYG{p}{,}\PYG{l+s+s1}{\PYGZsq{}}\PYG{l+s+s1}{b\PYGZhy{}}\PYG{l+s+s1}{\PYGZsq{}}\PYG{p}{,}\PYG{n}{label}\PYG{o}{=}\PYG{l+s+s1}{\PYGZsq{}}\PYG{l+s+s1}{\PYGZdl{}y\PYGZus{}}\PYG{l+s+si}{\PYGZob{}MEE\PYGZcb{}}\PYG{l+s+s1}{(t)\PYGZdl{}}\PYG{l+s+s1}{\PYGZsq{}}\PYG{p}{)}
\PYG{n}{plt}\PYG{o}{.}\PYG{n}{plot}\PYG{p}{(}\PYG{n}{t}\PYG{p}{,}\PYG{n}{y2}\PYG{p}{,}\PYG{l+s+s1}{\PYGZsq{}}\PYG{l+s+s1}{r\PYGZhy{}}\PYG{l+s+s1}{\PYGZsq{}}\PYG{p}{,}\PYG{n}{label}\PYG{o}{=}\PYG{l+s+s1}{\PYGZsq{}}\PYG{l+s+s1}{\PYGZdl{}y\PYGZus{}}\PYG{l+s+si}{\PYGZob{}MEI\PYGZcb{}}\PYG{l+s+s1}{(t)\PYGZdl{}}\PYG{l+s+s1}{\PYGZsq{}}\PYG{p}{)}
\PYG{n}{plt}\PYG{o}{.}\PYG{n}{plot}\PYG{p}{(}\PYG{n}{t}\PYG{p}{,}\PYG{n}{y3}\PYG{p}{,}\PYG{l+s+s1}{\PYGZsq{}}\PYG{l+s+s1}{g\PYGZhy{}}\PYG{l+s+s1}{\PYGZsq{}}\PYG{p}{,}\PYG{n}{label}\PYG{o}{=}\PYG{l+s+s1}{\PYGZsq{}}\PYG{l+s+s1}{\PYGZdl{}y\PYGZus{}}\PYG{l+s+si}{\PYGZob{}MT\PYGZcb{}}\PYG{l+s+s1}{(t)\PYGZdl{}}\PYG{l+s+s1}{\PYGZsq{}}\PYG{p}{)}
\PYG{n}{plt}\PYG{o}{.}\PYG{n}{plot}\PYG{p}{(}\PYG{n}{t}\PYG{p}{,}\PYG{n}{yex}\PYG{p}{,}\PYG{l+s+s1}{\PYGZsq{}}\PYG{l+s+s1}{k\PYGZhy{}\PYGZhy{}}\PYG{l+s+s1}{\PYGZsq{}}\PYG{p}{,}\PYG{n}{label}\PYG{o}{=}\PYG{l+s+s1}{\PYGZsq{}}\PYG{l+s+s1}{\PYGZdl{}y(t)\PYGZdl{}}\PYG{l+s+s1}{\PYGZsq{}}\PYG{p}{)}
\PYG{n}{plt}\PYG{o}{.}\PYG{n}{legend}\PYG{p}{(}\PYG{p}{)}\PYG{p}{;}
\end{sphinxVerbatim}

\begin{sphinxVerbatim}[commandchars=\\\{\}]
Nao convergindo apos 10 passos em t = 1.000000
\end{sphinxVerbatim}

\noindent\sphinxincludegraphics{{extra-eulerImplicito_7_1}.png}


\chapter{Métodos de Adams\sphinxhyphen{}Bashfort}
\label{\detokenize{extra/extra-multistep-adamsBashfort:metodos-de-adams-bashfort}}\label{\detokenize{extra/extra-multistep-adamsBashfort::doc}}
\begin{sphinxVerbatim}[commandchars=\\\{\}]
\PYG{o}{\PYGZpc{}}\PYG{k}{matplotlib} inline 
\PYG{k+kn}{from} \PYG{n+nn}{numpy} \PYG{k+kn}{import} \PYG{o}{*}
\PYG{k+kn}{from} \PYG{n+nn}{matplotlib}\PYG{n+nn}{.}\PYG{n+nn}{pyplot} \PYG{k+kn}{import} \PYG{o}{*}
\end{sphinxVerbatim}

\begin{sphinxVerbatim}[commandchars=\\\{\}]
\PYG{c+c1}{\PYGZsh{} Metodo de Adams\PYGZhy{}Bashfort de 2a. ordem}

\PYG{k}{def} \PYG{n+nf}{adams\PYGZus{}bashfort\PYGZus{}2nd\PYGZus{}order}\PYG{p}{(}\PYG{n}{t0}\PYG{p}{,}\PYG{n}{tf}\PYG{p}{,}\PYG{n}{y0}\PYG{p}{,}\PYG{n}{h}\PYG{p}{,}\PYG{n}{fun}\PYG{p}{)}\PYG{p}{:}
    \PYG{l+s+sd}{\PYGZsq{}\PYGZsq{}\PYGZsq{}}
\PYG{l+s+sd}{    Resolve o PVI y\PYGZsq{} = f(t,y), t0 \PYGZlt{}= t \PYGZlt{}= b, y(t0)=y0}
\PYG{l+s+sd}{    usando a formula de Adams\PYGZhy{}Bashfort de ordem 2 }
\PYG{l+s+sd}{    com passo h. O metodo de Euler eh usado para }
\PYG{l+s+sd}{    obter y1. A funcao f(t,y) deve ser definida }
\PYG{l+s+sd}{    pelo usuario. }
\PYG{l+s+sd}{  }
\PYG{l+s+sd}{    Saida: }
\PYG{l+s+sd}{  }
\PYG{l+s+sd}{    A rotina AB2 retorna dois vetores, t e y,  }
\PYG{l+s+sd}{    contendo, nesta orde, os pontos nodais e }
\PYG{l+s+sd}{    a solucao numerica. }
\PYG{l+s+sd}{    \PYGZsq{}\PYGZsq{}\PYGZsq{}}

    \PYG{c+c1}{\PYGZsh{} malha numerica  }
    \PYG{n}{n} \PYG{o}{=} \PYG{n+nb}{round}\PYG{p}{(}\PYG{p}{(}\PYG{n}{tf} \PYG{o}{\PYGZhy{}} \PYG{n}{t0}\PYG{p}{)}\PYG{o}{/}\PYG{n}{h}\PYG{p}{)} \PYG{o}{+} \PYG{l+m+mi}{1}
    \PYG{n}{t} \PYG{o}{=} \PYG{n}{linspace}\PYG{p}{(}\PYG{n}{t0}\PYG{p}{,}\PYG{n}{t0}\PYG{o}{+}\PYG{p}{(}\PYG{n}{n}\PYG{o}{\PYGZhy{}}\PYG{l+m+mi}{1}\PYG{p}{)}\PYG{o}{*}\PYG{n}{h}\PYG{p}{,}\PYG{n}{n}\PYG{p}{)}
    \PYG{n}{y} \PYG{o}{=} \PYG{n}{np}\PYG{o}{.}\PYG{n}{zeros}\PYG{p}{(}\PYG{n}{n}\PYG{p}{)}

    \PYG{n}{y}\PYG{p}{[}\PYG{l+m+mi}{0}\PYG{p}{]} \PYG{o}{=} \PYG{n}{y0} \PYG{c+c1}{\PYGZsh{} condicao inicial }

    \PYG{n}{f1} \PYG{o}{=} \PYG{n}{fun}\PYG{p}{(}\PYG{n}{t}\PYG{p}{[}\PYG{l+m+mi}{0}\PYG{p}{]}\PYG{p}{,}\PYG{n}{y}\PYG{p}{[}\PYG{l+m+mi}{0}\PYG{p}{]}\PYG{p}{)} \PYG{c+c1}{\PYGZsh{} f(t\PYGZus{}i,y\PYGZus{}i)}
    \PYG{n}{y}\PYG{p}{[}\PYG{l+m+mi}{1}\PYG{p}{]} \PYG{o}{=} \PYG{n}{y}\PYG{p}{[}\PYG{l+m+mi}{0}\PYG{p}{]} \PYG{o}{+} \PYG{n}{h}\PYG{o}{*}\PYG{n}{f1} \PYG{c+c1}{\PYGZsh{} Euler }

    \PYG{k}{for} \PYG{n}{i} \PYG{o+ow}{in} \PYG{n+nb}{range}\PYG{p}{(}\PYG{l+m+mi}{2}\PYG{p}{,}\PYG{n}{n}\PYG{p}{)}\PYG{p}{:}
        \PYG{n}{f2} \PYG{o}{=} \PYG{n}{fun}\PYG{p}{(}\PYG{n}{t}\PYG{p}{[}\PYG{n}{i}\PYG{o}{\PYGZhy{}}\PYG{l+m+mi}{1}\PYG{p}{]}\PYG{p}{,}\PYG{n}{y}\PYG{p}{[}\PYG{n}{i}\PYG{o}{\PYGZhy{}}\PYG{l+m+mi}{1}\PYG{p}{]}\PYG{p}{)} \PYG{c+c1}{\PYGZsh{} f(t\PYGZus{}i\PYGZhy{}1,y\PYGZus{}i\PYGZhy{}1)}
        \PYG{n}{y}\PYG{p}{[}\PYG{n}{i}\PYG{p}{]} \PYG{o}{=} \PYG{n}{y}\PYG{p}{[}\PYG{n}{i}\PYG{o}{\PYGZhy{}}\PYG{l+m+mi}{1}\PYG{p}{]} \PYG{o}{+} \PYG{n}{h}\PYG{o}{*}\PYG{p}{(}\PYG{l+m+mi}{3}\PYG{o}{*}\PYG{n}{f2} \PYG{o}{\PYGZhy{}} \PYG{n}{f1}\PYG{p}{)}\PYG{o}{/}\PYG{l+m+mi}{2} \PYG{c+c1}{\PYGZsh{} esquema AB2}
        \PYG{n}{f1} \PYG{o}{=} \PYG{n}{f2} \PYG{c+c1}{\PYGZsh{} atualiza }
        
    \PYG{k}{return} \PYG{n}{t}\PYG{p}{,}\PYG{n}{y}


\PYG{k}{def} \PYG{n+nf}{tab\PYGZus{}erro\PYGZus{}rel}\PYG{p}{(}\PYG{n}{t}\PYG{p}{,}\PYG{n}{y\PYGZus{}n}\PYG{p}{,}\PYG{n}{y\PYGZus{}e}\PYG{p}{)}\PYG{p}{:}

    \PYG{c+c1}{\PYGZsh{} erro relativo}
    \PYG{n}{e\PYGZus{}r} \PYG{o}{=} \PYG{n+nb}{abs}\PYG{p}{(}\PYG{n}{y\PYGZus{}n} \PYG{o}{\PYGZhy{}} \PYG{n}{y\PYGZus{}e}\PYG{p}{)}\PYG{o}{/}\PYG{n+nb}{abs}\PYG{p}{(}\PYG{n}{y\PYGZus{}e}\PYG{p}{)}

    \PYG{n+nb}{print}\PYG{p}{(}\PYG{l+s+s1}{\PYGZsq{}}\PYG{l+s+s1}{i }\PYG{l+s+se}{\PYGZbs{}t}\PYG{l+s+s1}{ t }\PYG{l+s+se}{\PYGZbs{}t}\PYG{l+s+s1}{ y\PYGZus{}ex }\PYG{l+s+se}{\PYGZbs{}t}\PYG{l+s+s1}{ y\PYGZus{}num }\PYG{l+s+se}{\PYGZbs{}t}\PYG{l+s+s1}{ e\PYGZus{}r}\PYG{l+s+s1}{\PYGZsq{}}\PYG{p}{)}
    \PYG{k}{for} \PYG{n}{i} \PYG{o+ow}{in} \PYG{n+nb}{range}\PYG{p}{(}\PYG{n+nb}{len}\PYG{p}{(}\PYG{n}{e\PYGZus{}r}\PYG{p}{)}\PYG{p}{)}\PYG{p}{:} 
        
        \PYG{k}{if} \PYG{n}{i} \PYG{o}{\PYGZpc{}} \PYG{l+m+mi}{10} \PYG{o}{==} \PYG{l+m+mi}{0}\PYG{p}{:}
            \PYG{n+nb}{print}\PYG{p}{(}\PYG{l+s+s1}{\PYGZsq{}}\PYG{l+s+si}{\PYGZob{}0:d\PYGZcb{}}\PYG{l+s+s1}{ }\PYG{l+s+se}{\PYGZbs{}t}\PYG{l+s+s1}{ }\PYG{l+s+si}{\PYGZob{}1:f\PYGZcb{}}\PYG{l+s+s1}{ }\PYG{l+s+se}{\PYGZbs{}t}\PYG{l+s+s1}{ }\PYG{l+s+si}{\PYGZob{}2:f\PYGZcb{}}\PYG{l+s+s1}{ }\PYG{l+s+se}{\PYGZbs{}t}\PYG{l+s+s1}{ }\PYG{l+s+si}{\PYGZob{}3:f\PYGZcb{}}\PYG{l+s+s1}{ }\PYG{l+s+se}{\PYGZbs{}t}\PYG{l+s+s1}{ }\PYG{l+s+si}{\PYGZob{}4:e\PYGZcb{}}\PYG{l+s+s1}{ }\PYG{l+s+se}{\PYGZbs{}n}\PYG{l+s+s1}{\PYGZsq{}}\PYG{o}{.}\PYG{n}{format}\PYG{p}{(}\PYG{n}{i}\PYG{p}{,}\PYG{n}{t}\PYG{p}{[}\PYG{n}{i}\PYG{p}{]}\PYG{p}{,}\PYG{n}{y\PYGZus{}e}\PYG{p}{[}\PYG{n}{i}\PYG{p}{]}\PYG{p}{,}\PYG{n}{y\PYGZus{}n}\PYG{p}{[}\PYG{n}{i}\PYG{p}{]}\PYG{p}{,}\PYG{n}{e\PYGZus{}r}\PYG{p}{[}\PYG{n}{i}\PYG{p}{]}\PYG{p}{)}\PYG{p}{)}
            
\PYG{k}{def} \PYG{n+nf}{plot\PYGZus{}fig}\PYG{p}{(}\PYG{n}{t}\PYG{p}{,}\PYG{n}{y\PYGZus{}n}\PYG{p}{,}\PYG{n}{y\PYGZus{}e}\PYG{p}{,}\PYG{n}{h}\PYG{p}{)}\PYG{p}{:}    
    \PYG{n}{plot}\PYG{p}{(}\PYG{n}{t}\PYG{p}{,}\PYG{n}{y\PYGZus{}e}\PYG{p}{,}\PYG{l+s+s1}{\PYGZsq{}}\PYG{l+s+s1}{r\PYGZhy{}}\PYG{l+s+s1}{\PYGZsq{}}\PYG{p}{)}
    \PYG{n}{plot}\PYG{p}{(}\PYG{n}{t}\PYG{p}{,}\PYG{n}{y\PYGZus{}n}\PYG{p}{,}\PYG{l+s+s1}{\PYGZsq{}}\PYG{l+s+s1}{bo}\PYG{l+s+s1}{\PYGZsq{}}\PYG{p}{)}
    \PYG{n}{title}\PYG{p}{(}\PYG{l+s+s1}{\PYGZsq{}}\PYG{l+s+s1}{Adams\PYGZhy{}Bashfort, 2a. ordem: \PYGZdl{}h=}\PYG{l+s+s1}{\PYGZsq{}} \PYG{o}{+} \PYG{n+nb}{str}\PYG{p}{(}\PYG{n}{h}\PYG{p}{)} \PYG{o}{+} \PYG{l+s+s1}{\PYGZsq{}}\PYG{l+s+s1}{\PYGZdl{}}\PYG{l+s+s1}{\PYGZsq{}}\PYG{p}{)}
    \PYG{n}{legend}\PYG{p}{(}\PYG{p}{[}\PYG{l+s+s1}{\PYGZsq{}}\PYG{l+s+s1}{\PYGZdl{}y\PYGZus{}}\PYG{l+s+si}{\PYGZob{}exata\PYGZcb{}}\PYG{l+s+s1}{\PYGZdl{}}\PYG{l+s+s1}{\PYGZsq{}}\PYG{p}{,}\PYG{l+s+s1}{\PYGZsq{}}\PYG{l+s+s1}{\PYGZdl{}y\PYGZus{}}\PYG{l+s+si}{\PYGZob{}num\PYGZcb{}}\PYG{l+s+s1}{\PYGZdl{}}\PYG{l+s+s1}{\PYGZsq{}}\PYG{p}{]}\PYG{p}{)}
\end{sphinxVerbatim}

\sphinxAtStartPar
\sphinxstylestrong{Exemplo:} Use o esquema de Adams\sphinxhyphen{}Bashfort de 2a. ordem para resolver o PVI.
\begin{equation*}
\begin{split}\begin{cases}
y'(t) = -y(t) + 2 \cos(t) \\
y(0) = 1 \\
0 \le t \le 18 \\ 
\end{cases}\end{split}
\end{equation*}
\sphinxAtStartPar
\sphinxstyleemphasis{Solução exata}: \(y(t) = {\rm sen}(t) + \cos(t)\)

\begin{sphinxVerbatim}[commandchars=\\\{\}]
\PYG{c+c1}{\PYGZsh{} define funcao}
\PYG{n}{f} \PYG{o}{=} \PYG{k}{lambda} \PYG{n}{t}\PYG{p}{,}\PYG{n}{y}\PYG{p}{:} \PYG{o}{\PYGZhy{}}\PYG{n}{y} \PYG{o}{+} \PYG{l+m+mi}{2}\PYG{o}{*}\PYG{n}{cos}\PYG{p}{(}\PYG{n}{t}\PYG{p}{)}

\PYG{c+c1}{\PYGZsh{} parametros}
\PYG{n}{t0} \PYG{o}{=} \PYG{l+m+mf}{0.0}
\PYG{n}{tf} \PYG{o}{=} \PYG{l+m+mf}{18.0}
\PYG{n}{y0} \PYG{o}{=} \PYG{l+m+mf}{1.0}
\PYG{n}{h} \PYG{o}{=} \PYG{l+m+mf}{0.5}

\PYG{c+c1}{\PYGZsh{} solucao numerica }
\PYG{n}{t}\PYG{p}{,}\PYG{n}{y\PYGZus{}num} \PYG{o}{=} \PYG{n}{adams\PYGZus{}bashfort\PYGZus{}2nd\PYGZus{}order}\PYG{p}{(}\PYG{n}{t0}\PYG{p}{,}\PYG{n}{tf}\PYG{p}{,}\PYG{n}{y0}\PYG{p}{,}\PYG{n}{h}\PYG{p}{,}\PYG{n}{f}\PYG{p}{)}
    
\PYG{c+c1}{\PYGZsh{} solucao exata }
\PYG{n}{y\PYGZus{}ex} \PYG{o}{=} \PYG{n}{sin}\PYG{p}{(}\PYG{n}{t}\PYG{p}{)} \PYG{o}{+} \PYG{n}{cos}\PYG{p}{(}\PYG{n}{t}\PYG{p}{)}

\PYG{n}{plot\PYGZus{}fig}\PYG{p}{(}\PYG{n}{t}\PYG{p}{,}\PYG{n}{y\PYGZus{}num}\PYG{p}{,}\PYG{n}{y\PYGZus{}ex}\PYG{p}{,}\PYG{n}{h}\PYG{p}{)}

\PYG{n}{tab\PYGZus{}erro\PYGZus{}rel}\PYG{p}{(}\PYG{n}{t}\PYG{p}{,}\PYG{n}{y\PYGZus{}num}\PYG{p}{,}\PYG{n}{y\PYGZus{}ex}\PYG{p}{)}
\end{sphinxVerbatim}

\begin{sphinxVerbatim}[commandchars=\\\{\}]
i 	 t 	 y\PYGZus{}ex 	 y\PYGZus{}num 	 e\PYGZus{}r
0 	 0.000000 	 1.000000 	 1.000000 	 0.000000e+00 

10 	 5.000000 	 \PYGZhy{}0.675262 	 \PYGZhy{}0.670163 	 7.550607e\PYGZhy{}03 

20 	 10.000000 	 \PYGZhy{}1.383093 	 \PYGZhy{}1.477388 	 6.817733e\PYGZhy{}02 

30 	 15.000000 	 \PYGZhy{}0.109400 	 \PYGZhy{}0.167908 	 5.348066e\PYGZhy{}01 
\end{sphinxVerbatim}

\noindent\sphinxincludegraphics{{extra-multistep-adamsBashfort_4_1}.png}

\begin{sphinxVerbatim}[commandchars=\\\{\}]
\PYG{n}{h} \PYG{o}{=} \PYG{l+m+mf}{0.1}

\PYG{c+c1}{\PYGZsh{} solucao numerica }
\PYG{n}{t}\PYG{p}{,}\PYG{n}{y\PYGZus{}num} \PYG{o}{=} \PYG{n}{adams\PYGZus{}bashfort\PYGZus{}2nd\PYGZus{}order}\PYG{p}{(}\PYG{n}{t0}\PYG{p}{,}\PYG{n}{tf}\PYG{p}{,}\PYG{n}{y0}\PYG{p}{,}\PYG{n}{h}\PYG{p}{,}\PYG{n}{f}\PYG{p}{)}
    
\PYG{c+c1}{\PYGZsh{} solucao exata }
\PYG{n}{y\PYGZus{}ex} \PYG{o}{=} \PYG{n}{sin}\PYG{p}{(}\PYG{n}{t}\PYG{p}{)} \PYG{o}{+} \PYG{n}{cos}\PYG{p}{(}\PYG{n}{t}\PYG{p}{)}

\PYG{n}{plot}\PYG{p}{(}\PYG{n}{t}\PYG{p}{,}\PYG{n}{y\PYGZus{}ex}\PYG{p}{,}\PYG{l+s+s1}{\PYGZsq{}}\PYG{l+s+s1}{r\PYGZhy{}}\PYG{l+s+s1}{\PYGZsq{}}\PYG{p}{)}
\PYG{n}{plot}\PYG{p}{(}\PYG{n}{t}\PYG{p}{,}\PYG{n}{y\PYGZus{}num}\PYG{p}{,}\PYG{l+s+s1}{\PYGZsq{}}\PYG{l+s+s1}{bo}\PYG{l+s+s1}{\PYGZsq{}}\PYG{p}{)}
\PYG{n}{title}\PYG{p}{(}\PYG{l+s+s1}{\PYGZsq{}}\PYG{l+s+s1}{Adams\PYGZhy{}Bashfort, 2a. ordem: \PYGZdl{}h=0.5\PYGZdl{}}\PYG{l+s+s1}{\PYGZsq{}}\PYG{p}{)}
\PYG{n}{legend}\PYG{p}{(}\PYG{p}{[}\PYG{l+s+s1}{\PYGZsq{}}\PYG{l+s+s1}{\PYGZdl{}y\PYGZus{}}\PYG{l+s+si}{\PYGZob{}exata\PYGZcb{}}\PYG{l+s+s1}{\PYGZdl{}}\PYG{l+s+s1}{\PYGZsq{}}\PYG{p}{,}\PYG{l+s+s1}{\PYGZsq{}}\PYG{l+s+s1}{\PYGZdl{}y\PYGZus{}}\PYG{l+s+si}{\PYGZob{}num\PYGZcb{}}\PYG{l+s+s1}{\PYGZdl{}}\PYG{l+s+s1}{\PYGZsq{}}\PYG{p}{]}\PYG{p}{)}

\PYG{n}{tab\PYGZus{}erro\PYGZus{}rel}\PYG{p}{(}\PYG{n}{t}\PYG{p}{,}\PYG{n}{y\PYGZus{}num}\PYG{p}{,}\PYG{n}{y\PYGZus{}ex}\PYG{p}{)}

\PYG{n}{plot\PYGZus{}fig}\PYG{p}{(}\PYG{n}{t}\PYG{p}{,}\PYG{n}{y\PYGZus{}num}\PYG{p}{,}\PYG{n}{y\PYGZus{}ex}\PYG{p}{,}\PYG{n}{h}\PYG{p}{)}
\end{sphinxVerbatim}

\begin{sphinxVerbatim}[commandchars=\\\{\}]
i 	 t 	 y\PYGZus{}ex 	 y\PYGZus{}num 	 e\PYGZus{}r
0 	 0.000000 	 1.000000 	 1.000000 	 0.000000e+00 

10 	 1.000000 	 1.381773 	 1.384518 	 1.986208e\PYGZhy{}03 

20 	 2.000000 	 0.493151 	 0.491752 	 2.835038e\PYGZhy{}03 

30 	 3.000000 	 \PYGZhy{}0.848872 	 \PYGZhy{}0.852907 	 4.752486e\PYGZhy{}03 

40 	 4.000000 	 \PYGZhy{}1.410446 	 \PYGZhy{}1.413326 	 2.041670e\PYGZhy{}03 

50 	 5.000000 	 \PYGZhy{}0.675262 	 \PYGZhy{}0.674309 	 1.410789e\PYGZhy{}03 

60 	 6.000000 	 0.680755 	 0.684675 	 5.758689e\PYGZhy{}03 

70 	 7.000000 	 1.410889 	 1.414177 	 2.330243e\PYGZhy{}03 

80 	 8.000000 	 0.843858 	 0.843492 	 4.337392e\PYGZhy{}04 

90 	 9.000000 	 \PYGZhy{}0.499012 	 \PYGZhy{}0.502694 	 7.379922e\PYGZhy{}03 

100 	 10.000000 	 \PYGZhy{}1.383093 	 \PYGZhy{}1.386706 	 2.612468e\PYGZhy{}03 

110 	 11.000000 	 \PYGZhy{}0.995565 	 \PYGZhy{}0.995786 	 2.227766e\PYGZhy{}04 

120 	 12.000000 	 0.307281 	 0.310655 	 1.097903e\PYGZhy{}02 

130 	 13.000000 	 1.327614 	 1.331481 	 2.913030e\PYGZhy{}03 

140 	 14.000000 	 1.127345 	 1.128150 	 7.144782e\PYGZhy{}04 

150 	 15.000000 	 \PYGZhy{}0.109400 	 \PYGZhy{}0.112397 	 2.739478e\PYGZhy{}02 

160 	 16.000000 	 \PYGZhy{}1.245563 	 \PYGZhy{}1.249607 	 3.246745e\PYGZhy{}03 

170 	 17.000000 	 \PYGZhy{}1.236561 	 \PYGZhy{}1.237934 	 1.110338e\PYGZhy{}03 

180 	 18.000000 	 \PYGZhy{}0.090671 	 \PYGZhy{}0.088110 	 2.823799e\PYGZhy{}02 
\end{sphinxVerbatim}

\noindent\sphinxincludegraphics{{extra-multistep-adamsBashfort_5_1}.png}

\begin{sphinxVerbatim}[commandchars=\\\{\}]
\PYG{n}{h} \PYG{o}{=} \PYG{l+m+mf}{0.01}

\PYG{c+c1}{\PYGZsh{} solucao numerica }
\PYG{n}{t}\PYG{p}{,}\PYG{n}{y\PYGZus{}num} \PYG{o}{=} \PYG{n}{adams\PYGZus{}bashfort\PYGZus{}2nd\PYGZus{}order}\PYG{p}{(}\PYG{n}{t0}\PYG{p}{,}\PYG{n}{tf}\PYG{p}{,}\PYG{n}{y0}\PYG{p}{,}\PYG{n}{h}\PYG{p}{,}\PYG{n}{f}\PYG{p}{)}
    
\PYG{c+c1}{\PYGZsh{} solucao exata }
\PYG{n}{y\PYGZus{}ex} \PYG{o}{=} \PYG{n}{sin}\PYG{p}{(}\PYG{n}{t}\PYG{p}{)} \PYG{o}{+} \PYG{n}{cos}\PYG{p}{(}\PYG{n}{t}\PYG{p}{)}

\PYG{n}{plot}\PYG{p}{(}\PYG{n}{t}\PYG{p}{,}\PYG{n}{y\PYGZus{}ex}\PYG{p}{,}\PYG{l+s+s1}{\PYGZsq{}}\PYG{l+s+s1}{r\PYGZhy{}}\PYG{l+s+s1}{\PYGZsq{}}\PYG{p}{)}
\PYG{n}{plot}\PYG{p}{(}\PYG{n}{t}\PYG{p}{,}\PYG{n}{y\PYGZus{}num}\PYG{p}{,}\PYG{l+s+s1}{\PYGZsq{}}\PYG{l+s+s1}{bo}\PYG{l+s+s1}{\PYGZsq{}}\PYG{p}{)}
\PYG{n}{title}\PYG{p}{(}\PYG{l+s+s1}{\PYGZsq{}}\PYG{l+s+s1}{Adams\PYGZhy{}Bashfort, 2a. ordem: \PYGZdl{}h=0.1\PYGZdl{}}\PYG{l+s+s1}{\PYGZsq{}}\PYG{p}{)}
\PYG{n}{legend}\PYG{p}{(}\PYG{p}{[}\PYG{l+s+s1}{\PYGZsq{}}\PYG{l+s+s1}{\PYGZdl{}y\PYGZus{}}\PYG{l+s+si}{\PYGZob{}exata\PYGZcb{}}\PYG{l+s+s1}{\PYGZdl{}}\PYG{l+s+s1}{\PYGZsq{}}\PYG{p}{,}\PYG{l+s+s1}{\PYGZsq{}}\PYG{l+s+s1}{\PYGZdl{}y\PYGZus{}}\PYG{l+s+si}{\PYGZob{}num\PYGZcb{}}\PYG{l+s+s1}{\PYGZdl{}}\PYG{l+s+s1}{\PYGZsq{}}\PYG{p}{]}\PYG{p}{)}

\PYG{c+c1}{\PYGZsh{}tab\PYGZus{}erro\PYGZus{}rel(t,y\PYGZus{}num,y\PYGZus{}ex)}

\PYG{n}{plot\PYGZus{}fig}\PYG{p}{(}\PYG{n}{t}\PYG{p}{,}\PYG{n}{y\PYGZus{}num}\PYG{p}{,}\PYG{n}{y\PYGZus{}ex}\PYG{p}{,}\PYG{n}{h}\PYG{p}{)}
\end{sphinxVerbatim}

\noindent\sphinxincludegraphics{{extra-multistep-adamsBashfort_6_0}.png}


\chapter{EDOs de ordem superior (redução de ordem)}
\label{\detokenize{extra/extra-edo-superior:edos-de-ordem-superior-reducao-de-ordem}}\label{\detokenize{extra/extra-edo-superior::doc}}
\sphinxAtStartPar
EDOs de ordem superior podem ser reescritas como sistemas de EDOs de primeira ordem. A fim de verificarmos como isto pode ser feito, consideremos, inicialmente o caso geral de uma EDO de segunda ordem.
\begin{equation*}
\begin{split}\begin{cases}
y''(t) = f(t,y,y') \\
y(t_0) = y_0 \\ 
y'(t_0) = y_0'
\end{cases}\end{split}
\end{equation*}
\sphinxAtStartPar
Se fizermos \(y_1(t) = y(t)\) e \(y_2(t) = y'(t)\), o PVI acima pode ser reformulado como um sistema de duas EDOs de primeira ordem como:
\begin{equation*}
\begin{split}\begin{cases}
y_1'(t) = y_2(t) \\
y_2'(t) = f(t,y_1,y_2) \\
y_1(t_0) = y_0 \\
y_2(t_0) = y_0'
\end{cases}\end{split}
\end{equation*}

\section{Movimento pendular simples}
\label{\detokenize{extra/extra-edo-superior:movimento-pendular-simples}}
\sphinxAtStartPar
Como exemplo (simplificado) prático derivado de uma EDO de segunda ordem, temos o movimento de um pêndulo submetido apenas à força gravitacional como mostra a figura abaixo. O pêndulo de massa \(m\) fica pendurado por uma corda de comprimento \(l\) e se move em relação ao eixo \(\theta = 0\). O movimento é completamente descrito com a especificação da posição inicial \(\theta(0) = \theta_0\) e a velocidade angular inicial \(\theta'(0) = \theta_0'\).

\begin{sphinxVerbatim}[commandchars=\\\{\}]
\PYG{k+kn}{import} \PYG{n+nn}{numpy} \PYG{k}{as} \PYG{n+nn}{np}
\PYG{k+kn}{import} \PYG{n+nn}{matplotlib}\PYG{n+nn}{.}\PYG{n+nn}{pyplot} \PYG{k}{as} \PYG{n+nn}{plt} 

\PYG{c+c1}{\PYGZsh{}plt.autoscale(enable=True, axis=\PYGZsq{}x\PYGZsq{}, tight=True)}
\PYG{n}{plt}\PYG{o}{.}\PYG{n}{figure}\PYG{p}{(}\PYG{n}{figsize}\PYG{o}{=}\PYG{p}{(}\PYG{l+m+mi}{3}\PYG{p}{,}\PYG{l+m+mi}{3}\PYG{p}{)}\PYG{p}{)}

\PYG{n}{theta} \PYG{o}{=} \PYG{n}{np}\PYG{o}{.}\PYG{n}{pi}\PYG{o}{/}\PYG{l+m+mi}{10}
\PYG{n}{x} \PYG{o}{=} \PYG{n}{np}\PYG{o}{.}\PYG{n}{linspace}\PYG{p}{(}\PYG{l+m+mi}{0}\PYG{p}{,}\PYG{l+m+mi}{1}\PYG{p}{,}\PYG{l+m+mi}{10}\PYG{p}{)}
\PYG{n}{plt}\PYG{o}{.}\PYG{n}{axvline}\PYG{p}{(}\PYG{n}{x}\PYG{o}{=}\PYG{l+m+mf}{0.5}\PYG{p}{,}\PYG{n}{ymin}\PYG{o}{=}\PYG{l+m+mi}{0}\PYG{p}{,}\PYG{n}{ymax}\PYG{o}{=}\PYG{l+m+mf}{1.0}\PYG{p}{,}\PYG{n}{linestyle}\PYG{o}{=}\PYG{l+s+s1}{\PYGZsq{}}\PYG{l+s+s1}{\PYGZhy{}\PYGZhy{}}\PYG{l+s+s1}{\PYGZsq{}}\PYG{p}{,}\PYG{n}{linewidth}\PYG{o}{=}\PYG{l+m+mf}{1.0}\PYG{p}{)}
\PYG{n}{plt}\PYG{o}{.}\PYG{n}{plot}\PYG{p}{(}\PYG{p}{[}\PYG{l+m+mf}{0.5}\PYG{p}{,}\PYG{l+m+mf}{0.5} \PYG{o}{+} \PYG{n}{np}\PYG{o}{.}\PYG{n}{sin}\PYG{p}{(}\PYG{n}{theta}\PYG{p}{)}\PYG{p}{]}\PYG{p}{,}\PYG{p}{[}\PYG{l+m+mf}{1.0}\PYG{p}{,}\PYG{l+m+mf}{1.0}\PYG{o}{\PYGZhy{}}\PYG{n}{np}\PYG{o}{.}\PYG{n}{cos}\PYG{p}{(}\PYG{n}{theta}\PYG{p}{)}\PYG{p}{]}\PYG{p}{,}\PYG{n}{linewidth}\PYG{o}{=}\PYG{l+m+mf}{1.0}\PYG{p}{)}
\PYG{n}{plt}\PYG{o}{.}\PYG{n}{scatter}\PYG{p}{(}\PYG{p}{[}\PYG{l+m+mf}{0.5} \PYG{o}{+} \PYG{n}{np}\PYG{o}{.}\PYG{n}{sin}\PYG{p}{(}\PYG{n}{theta}\PYG{p}{)}\PYG{p}{]}\PYG{p}{,}\PYG{p}{[}\PYG{l+m+mf}{1.0}\PYG{o}{\PYGZhy{}}\PYG{n}{np}\PYG{o}{.}\PYG{n}{cos}\PYG{p}{(}\PYG{n}{theta}\PYG{p}{)}\PYG{p}{]}\PYG{p}{,}\PYG{n}{s}\PYG{o}{=}\PYG{l+m+mi}{300}\PYG{p}{)}
\PYG{n}{plt}\PYG{o}{.}\PYG{n}{box}\PYG{p}{(}\PYG{k+kc}{False}\PYG{p}{)}\PYG{p}{;} \PYG{n}{locs}\PYG{p}{,} \PYG{n}{labels} \PYG{o}{=} \PYG{n}{plt}\PYG{o}{.}\PYG{n}{xticks}\PYG{p}{(}\PYG{p}{)}\PYG{p}{;} \PYG{n}{plt}\PYG{o}{.}\PYG{n}{tick\PYGZus{}params}\PYG{p}{(}\PYG{n}{axis}\PYG{o}{=}\PYG{l+s+s1}{\PYGZsq{}}\PYG{l+s+s1}{both}\PYG{l+s+s1}{\PYGZsq{}}\PYG{p}{,}\PYG{n}{width}\PYG{o}{=}\PYG{l+m+mf}{0.0}\PYG{p}{,}\PYG{n}{labelleft}\PYG{o}{=}\PYG{k+kc}{False}\PYG{p}{,}\PYG{n}{labelbottom}\PYG{o}{=}\PYG{k+kc}{False}\PYG{p}{)}
\PYG{n}{x1}\PYG{p}{,}\PYG{n}{x2} \PYG{o}{=} \PYG{n}{plt}\PYG{o}{.}\PYG{n}{xlim}\PYG{p}{(}\PYG{p}{[}\PYG{l+m+mi}{0}\PYG{p}{,}\PYG{l+m+mi}{1}\PYG{p}{]}\PYG{p}{)}
\PYG{n}{y1}\PYG{p}{,}\PYG{n}{y2} \PYG{o}{=} \PYG{n}{plt}\PYG{o}{.}\PYG{n}{ylim}\PYG{p}{(}\PYG{p}{[}\PYG{o}{\PYGZhy{}}\PYG{l+m+mf}{0.3}\PYG{p}{,}\PYG{l+m+mi}{1}\PYG{p}{]}\PYG{p}{)}

\PYG{n}{plt}\PYG{o}{.}\PYG{n}{annotate}\PYG{p}{(}\PYG{l+s+s1}{\PYGZsq{}}\PYG{l+s+s1}{\PYGZdl{}}\PYG{l+s+se}{\PYGZbs{}\PYGZbs{}}\PYG{l+s+s1}{theta\PYGZdl{}}\PYG{l+s+s1}{\PYGZsq{}}\PYG{p}{,}\PYG{n}{xy}\PYG{o}{=}\PYG{p}{(}\PYG{l+m+mf}{0.52}\PYG{p}{,}\PYG{l+m+mf}{0.72}\PYG{p}{)}\PYG{p}{,}\PYG{n}{fontsize}\PYG{o}{=}\PYG{l+m+mi}{14}\PYG{p}{)}\PYG{p}{;}
\end{sphinxVerbatim}

\noindent\sphinxincludegraphics{{extra-edo-superior_2_0}.png}

\sphinxAtStartPar
Uma vez que o comprimento do arco realizado pelo pêndulo é \(s = l\theta\) e a força\sphinxhyphen{}peso tangente ao arco oposta ao movimento é determinada em função de \(\theta\) por \(-mg{\rm sen}(\theta)\), a segunda lei de Newton diz\sphinxhyphen{}nos que
\begin{equation*}
\begin{split}m\dfrac{d^2s}{dt^2} = -mg{\rm sen}(\theta) \Rightarrow \dfrac{d^2 \theta}{dt^2} + \dfrac{g}{l}{\rm sen}(\theta) = 0\end{split}
\end{equation*}
\sphinxAtStartPar
Tomando a EDO acima juntamente com as condições iniciais e valendo\sphinxhyphen{}se das substituições \(y_1(t) = \theta(t)\) e \(y_2(t) = \theta'(t)\), obtemos um sistema de EDOs de primeira ordem que nos conduz ao seguinte PVI:
\$\(\begin{cases}
y_1'(t) = y_2(t) \\
y_2'(t) = - \dfrac{g}{l}{\rm sen}(\theta) \\
y_1(0) = \theta(0) = \theta_0 \\
y_2(0) = \theta'(0) = \theta_0'
\end{cases}\)\$


\section{Hipótese de ângulos pequenos}
\label{\detokenize{extra/extra-edo-superior:hipotese-de-angulos-pequenos}}
\sphinxAtStartPar
No movimento pendular, se \(\theta\) for considerado muito pequeno, podemos dizer que \(\theta \approx {\rm sen}(\theta)\). Logo, a EDO de segunda ordem reduz\sphinxhyphen{}se a
\begin{equation*}
\begin{split}\dfrac{d^2 \theta}{dt^2} + \dfrac{g}{l}\theta = 0,\end{split}
\end{equation*}
\sphinxAtStartPar
de maneira que, analogamente ao caso anterior, temos um novo sistema levemente modificado na segunda equação:
\$\(\begin{cases}
y_1'(t) = y_2(t) \\
y_2'(t) = - \dfrac{g}{l}y_1(t) \\
y_1(0) = \theta(0) = \theta_0 \\
y_2(0) = \theta'(0) = \theta_0'
\end{cases}\)\$

\sphinxAtStartPar
Vale dizer que a solução analítica para este caso é dada por \(\theta(t) = \theta_0 \cos \left( \sqrt{\dfrac{g}{l}} t \right)\).


\section{Caso geral de ordem \protect\(m\protect\)}
\label{\detokenize{extra/extra-edo-superior:caso-geral-de-ordem-m}}
\sphinxAtStartPar
De modo similar, uma EDO de ordem \(m\) pode gerar um sistema de EDOs de ordem 1. Considerando o PVI generalizado
\$\(\begin{cases}
\dfrac{d^m y(t)}{dt^m} = f(t,y,y',\ldots,y^{(m-1)}) \\
y(t_0) = y_0, \ \ y'(t_0) = y_0', \ldots y^{(m-1)}(t_0) = y_0^{(m-1)},
\end{cases}\)\$
usamos as substituições
\begin{equation*}
\begin{split}\begin{align}
y_1(t) &= y(t) \\
y_2(t) &= y'(t) \\
\vdots &\vdots  \vdots \\
y_m(t) &= y^{(m-1)}(t) \\
\end{align}\end{split}
\end{equation*}
\sphinxAtStartPar
para reformular o PVI original no sistema:
\begin{equation*}
\begin{split}\begin{cases}
y_1'(t) = y_2(t) \\
y_2'(t) = y_3(t) \\
\vdots \\
y_{m-1}'(t) = y_m(t) \\
y_m'(t) = f(t,y_1,\ldots,y_m) \\
y_1(t_0) = y_0 \\
y_2(t_0) = y_0'' \\
\vdots \\
y_m(t_0) = y_0^{(m-1)} \\
\end{cases}\end{split}
\end{equation*}
\sphinxAtStartPar
\sphinxstylestrong{Exemplo:} O PVI
\begin{equation*}
\begin{split}\begin{cases}
y'''(t) + 3y''(t) + 3y'(t) + y(t) = -4{\rm sen}(t) \\
y(0) = y'(0) = 1, y''(0) = -1
\end{cases}\end{split}
\end{equation*}
\sphinxAtStartPar
pode ser reformulado no sistema
\begin{equation*}
\begin{split}\begin{cases}
y_1'(t) = y_2(t) \\
y_2'(t) = y_3(t) \\
y_3'(t) = - y_1(t) -3y_2(t) - 3y_3(t) -4{\rm sen}(t) \\
y_1(0) = 1 \\
y_2(0) = 1 \\
y_3(0) = -1.
\end{cases}\end{split}
\end{equation*}
\sphinxAtStartPar
\sphinxstylestrong{Exercício resolvido:} utilizando os parâmetros \(g=9.867 \, m/s^2\), \(\theta_0 = \pi/25\) e \(\theta_0' = 0.5 \, m/s\), resolva numericamente o sistema de EDOs do movimento pendular sob a hipótese de ângulos pequenos.

\sphinxAtStartPar
Para resolver o exercício, utilizaremos o método de Euler que implementamos anteriormente. Basta definirmos uma nova função \sphinxcode{\sphinxupquote{fsys}} e os parâmetros necessários para o funcionamento da função \sphinxcode{\sphinxupquote{euler\_sys}}.

\begin{sphinxVerbatim}[commandchars=\\\{\}]
\PYG{k+kn}{from} \PYG{n+nn}{numpy} \PYG{k+kn}{import} \PYG{o}{*}

\PYG{k}{def} \PYG{n+nf}{euler\PYGZus{}sys}\PYG{p}{(}\PYG{n}{t0}\PYG{p}{,}\PYG{n}{tf}\PYG{p}{,}\PYG{n}{y0}\PYG{p}{,}\PYG{n}{h}\PYG{p}{,}\PYG{n}{f}\PYG{p}{)}\PYG{p}{:}
    \PYG{l+s+sd}{\PYGZdq{}\PYGZdq{}\PYGZdq{}}
\PYG{l+s+sd}{    Resolve o PVI de um sistema de EDOs escalares}
\PYG{l+s+sd}{    y’ = f(t,y), t0 \PYGZlt{}= t \PYGZlt{}= b, y(t0)=y0 pelo metodo de Euler com tamanho de passo h.}
\PYG{l+s+sd}{    O usuario deve fornecer um vetor f contendo as funcoes a serem avaliadas como }
\PYG{l+s+sd}{    membro direito.}
\PYG{l+s+sd}{    }
\PYG{l+s+sd}{    Entrada: }
\PYG{l+s+sd}{        t0  \PYGZhy{} tempo inicial}
\PYG{l+s+sd}{        tf  \PYGZhy{} tempo final }
\PYG{l+s+sd}{        y0  \PYGZhy{} condicao inicial }
\PYG{l+s+sd}{        h   \PYGZhy{} passo }
\PYG{l+s+sd}{        f   \PYGZhy{} vetor de funcoes f(t,y) (anonima)}
\PYG{l+s+sd}{    }
\PYG{l+s+sd}{    Saída:}
\PYG{l+s+sd}{        t   \PYGZhy{} vetor contendo os valores nodais t[i], i = 1,2,...,n}
\PYG{l+s+sd}{        Y   \PYGZhy{} matriz de dimensoes n x m, com m sendo o numero de EDOs}
\PYG{l+s+sd}{             (a i\PYGZhy{}esima linha y[i,:] traz as estimativas de todas }
\PYG{l+s+sd}{              as funcoes y\PYGZus{}j no tempo t[i])}
\PYG{l+s+sd}{    \PYGZdq{}\PYGZdq{}\PYGZdq{}}    
    
    \PYG{n}{m} \PYG{o}{=} \PYG{n}{y0}\PYG{o}{.}\PYG{n}{size}
    \PYG{n}{n} \PYG{o}{=} \PYG{n+nb}{round}\PYG{p}{(}\PYG{p}{(}\PYG{n}{tf} \PYG{o}{\PYGZhy{}} \PYG{n}{t0}\PYG{p}{)}\PYG{o}{/}\PYG{n}{h} \PYG{o}{+} \PYG{l+m+mi}{1}\PYG{p}{)}
    \PYG{n}{t} \PYG{o}{=} \PYG{n}{linspace}\PYG{p}{(}\PYG{n}{t0}\PYG{p}{,}\PYG{n}{t0}\PYG{o}{+}\PYG{p}{(}\PYG{n}{n}\PYG{o}{\PYGZhy{}}\PYG{l+m+mi}{1}\PYG{p}{)}\PYG{o}{*}\PYG{n}{h}\PYG{p}{,}\PYG{n}{n}\PYG{p}{)}
    \PYG{n}{Y} \PYG{o}{=} \PYG{n}{zeros}\PYG{p}{(}\PYG{p}{(}\PYG{n}{n}\PYG{p}{,}\PYG{n}{m}\PYG{p}{)}\PYG{p}{)}
    
    \PYG{n}{Y}\PYG{p}{[}\PYG{l+m+mi}{0}\PYG{p}{,}\PYG{p}{:}\PYG{p}{]} \PYG{o}{=} \PYG{n}{y0}
    
    
    \PYG{k}{for} \PYG{n}{i} \PYG{o+ow}{in} \PYG{n+nb}{range}\PYG{p}{(}\PYG{l+m+mi}{1}\PYG{p}{,}\PYG{n}{n}\PYG{p}{)}\PYG{p}{:}  
        \PYG{n}{Y}\PYG{p}{[}\PYG{n}{i}\PYG{p}{,}\PYG{p}{:}\PYG{p}{]} \PYG{o}{=} \PYG{n}{Y}\PYG{p}{[}\PYG{n}{i}\PYG{o}{\PYGZhy{}}\PYG{l+m+mi}{1}\PYG{p}{,}\PYG{p}{:}\PYG{p}{]} \PYG{o}{+} \PYG{n}{h}\PYG{o}{*}\PYG{n}{f}\PYG{p}{(}\PYG{n}{t}\PYG{p}{[}\PYG{n}{i}\PYG{o}{\PYGZhy{}}\PYG{l+m+mi}{1}\PYG{p}{]}\PYG{p}{,}\PYG{n}{Y}\PYG{p}{[}\PYG{n}{i}\PYG{o}{\PYGZhy{}}\PYG{l+m+mi}{1}\PYG{p}{,}\PYG{p}{:}\PYG{p}{]}\PYG{p}{)}     
        
    \PYG{k}{return} \PYG{n}{t}\PYG{p}{,}\PYG{n}{Y}
\end{sphinxVerbatim}

\begin{sphinxVerbatim}[commandchars=\\\{\}]
\PYG{c+c1}{\PYGZsh{} define funcao f(t,y)}

\PYG{k}{def} \PYG{n+nf}{params}\PYG{p}{(}\PYG{n}{g}\PYG{p}{,}\PYG{n}{l}\PYG{p}{)}\PYG{p}{:}
    \PYG{k}{return} \PYG{n}{g}\PYG{p}{,}\PYG{n}{l}

\PYG{k}{def} \PYG{n+nf}{fsys}\PYG{p}{(}\PYG{n}{t}\PYG{p}{,}\PYG{n}{y}\PYG{p}{)}\PYG{p}{:}
    
    \PYG{n}{aux} \PYG{o}{=} \PYG{n}{params}\PYG{p}{(}\PYG{n}{g}\PYG{p}{,}\PYG{n}{l}\PYG{p}{)} 
    \PYG{c+c1}{\PYGZsh{} considera g e l como parâmetros}
    \PYG{n}{F} \PYG{o}{=} \PYG{n}{array}\PYG{p}{(}\PYG{p}{[}\PYG{n}{y}\PYG{p}{[}\PYG{l+m+mi}{1}\PYG{p}{]}\PYG{p}{,}\PYG{o}{\PYGZhy{}}\PYG{n}{aux}\PYG{p}{[}\PYG{l+m+mi}{0}\PYG{p}{]}\PYG{o}{/}\PYG{n}{aux}\PYG{p}{[}\PYG{l+m+mi}{1}\PYG{p}{]}\PYG{o}{*}\PYG{n}{y}\PYG{p}{[}\PYG{l+m+mi}{0}\PYG{p}{]}\PYG{p}{]}\PYG{p}{)}
    
    \PYG{k}{return} \PYG{n}{F}
\end{sphinxVerbatim}

\begin{sphinxVerbatim}[commandchars=\\\{\}]
\PYG{k+kn}{from} \PYG{n+nn}{matplotlib}\PYG{n+nn}{.}\PYG{n+nn}{pyplot} \PYG{k+kn}{import} \PYG{n}{plot}\PYG{p}{,}\PYG{n}{legend}


\PYG{c+c1}{\PYGZsh{} parametros}
\PYG{n}{t0}\PYG{p}{,}\PYG{n}{tf} \PYG{o}{=} \PYG{l+m+mi}{0}\PYG{p}{,}\PYG{l+m+mi}{5}
\PYG{n}{h} \PYG{o}{=} \PYG{l+m+mf}{0.05}

\PYG{n}{theta0} \PYG{o}{=} \PYG{n}{np}\PYG{o}{.}\PYG{n}{pi}\PYG{o}{/}\PYG{l+m+mi}{25}
\PYG{n}{dtheta0} \PYG{o}{=} \PYG{l+m+mf}{0.5}
\PYG{n}{y0} \PYG{o}{=} \PYG{n}{array}\PYG{p}{(}\PYG{p}{[}\PYG{n}{theta0}\PYG{p}{,}\PYG{n}{dtheta0}\PYG{p}{]}\PYG{p}{)}
\PYG{n}{g}\PYG{p}{,} \PYG{n}{l} \PYG{o}{=} \PYG{l+m+mf}{9.867}\PYG{p}{,} \PYG{l+m+mf}{1.0}

\PYG{c+c1}{\PYGZsh{} solucao numerica do sistema}
\PYG{n}{t}\PYG{p}{,}\PYG{n}{Y} \PYG{o}{=} \PYG{n}{euler\PYGZus{}sys}\PYG{p}{(}\PYG{n}{t0}\PYG{p}{,}\PYG{n}{tf}\PYG{p}{,}\PYG{n}{y0}\PYG{p}{,}\PYG{n}{h}\PYG{p}{,}\PYG{n}{fsys}\PYG{p}{)}
\PYG{n}{y1\PYGZus{}num} \PYG{o}{=} \PYG{n}{Y}\PYG{p}{[}\PYG{p}{:}\PYG{p}{,}\PYG{l+m+mi}{0}\PYG{p}{]}
\PYG{n}{y2\PYGZus{}num} \PYG{o}{=} \PYG{n}{Y}\PYG{p}{[}\PYG{p}{:}\PYG{p}{,}\PYG{l+m+mi}{1}\PYG{p}{]}

\PYG{c+c1}{\PYGZsh{} solucao analitica}
\PYG{n}{y1\PYGZus{}an} \PYG{o}{=} \PYG{n}{theta0}\PYG{o}{*}\PYG{n}{np}\PYG{o}{.}\PYG{n}{cos}\PYG{p}{(}\PYG{n}{np}\PYG{o}{.}\PYG{n}{sqrt}\PYG{p}{(}\PYG{n}{g}\PYG{o}{/}\PYG{n}{l}\PYG{p}{)}\PYG{o}{*}\PYG{n}{t}\PYG{p}{)}

\PYG{c+c1}{\PYGZsh{} plotagem}
\PYG{n}{plot}\PYG{p}{(}\PYG{n}{t}\PYG{p}{,}\PYG{n}{y1\PYGZus{}num}\PYG{p}{,}\PYG{l+s+s1}{\PYGZsq{}}\PYG{l+s+s1}{r\PYGZhy{}\PYGZhy{}}\PYG{l+s+s1}{\PYGZsq{}}\PYG{p}{,}\PYG{n}{label}\PYG{o}{=}\PYG{l+s+s1}{\PYGZsq{}}\PYG{l+s+s1}{\PYGZdl{}y\PYGZus{}}\PYG{l+s+s1}{\PYGZob{}}\PYG{l+s+s1}{1,h\PYGZcb{}(t)\PYGZdl{}}\PYG{l+s+s1}{\PYGZsq{}}\PYG{p}{)}
\PYG{c+c1}{\PYGZsh{}plot(t,y2\PYGZus{}num,\PYGZsq{}b\PYGZhy{}\PYGZhy{}\PYGZsq{},label=\PYGZsq{}\PYGZdl{}y\PYGZus{}\PYGZob{}2,h\PYGZcb{}(t)\PYGZdl{}\PYGZsq{})}

\PYG{n}{plot}\PYG{p}{(}\PYG{n}{t}\PYG{p}{,}\PYG{n}{y1\PYGZus{}an}\PYG{p}{,}\PYG{l+s+s1}{\PYGZsq{}}\PYG{l+s+s1}{r}\PYG{l+s+s1}{\PYGZsq{}}\PYG{p}{,}\PYG{n}{label}\PYG{o}{=}\PYG{l+s+s1}{\PYGZsq{}}\PYG{l+s+s1}{\PYGZdl{}y\PYGZus{}1(t)\PYGZdl{}}\PYG{l+s+s1}{\PYGZsq{}}\PYG{p}{)}
\PYG{c+c1}{\PYGZsh{}plot(t,y2\PYGZus{}an,\PYGZsq{}b\PYGZsq{},label=\PYGZsq{}\PYGZdl{}y\PYGZus{}2(t)\PYGZdl{}\PYGZsq{})}

\PYG{c+c1}{\PYGZsh{} erro absoluto}
\PYG{n}{plot}\PYG{p}{(}\PYG{n}{t}\PYG{p}{,}\PYG{n+nb}{abs}\PYG{p}{(}\PYG{n}{y1\PYGZus{}an} \PYG{o}{\PYGZhy{}} \PYG{n}{y1\PYGZus{}num}\PYG{p}{)}\PYG{p}{,}\PYG{l+s+s1}{\PYGZsq{}}\PYG{l+s+s1}{r:}\PYG{l+s+s1}{\PYGZsq{}}\PYG{p}{,}\PYG{n}{label}\PYG{o}{=}\PYG{l+s+s1}{\PYGZsq{}}\PYG{l+s+s1}{\PYGZdl{}EA\PYGZus{}1(t)\PYGZdl{}}\PYG{l+s+s1}{\PYGZsq{}}\PYG{p}{)}
\PYG{c+c1}{\PYGZsh{}plot(t,abs(y2\PYGZus{}an \PYGZhy{} y2\PYGZus{}num),\PYGZsq{}b:\PYGZsq{},label=\PYGZsq{}\PYGZdl{}EA\PYGZus{}2(t)\PYGZdl{}\PYGZsq{})}

\PYG{n}{legend}\PYG{p}{(}\PYG{n}{loc}\PYG{o}{=}\PYG{l+m+mi}{3}\PYG{p}{,}\PYG{n}{fontsize}\PYG{o}{=}\PYG{l+m+mi}{10}\PYG{p}{)}
\end{sphinxVerbatim}

\begin{sphinxVerbatim}[commandchars=\\\{\}]
\PYGZlt{}matplotlib.legend.Legend at 0x7fb2d222e100\PYGZgt{}
\end{sphinxVerbatim}

\noindent\sphinxincludegraphics{{extra-edo-superior_10_1}.png}

\sphinxAtStartPar
\sphinxstylestrong{Exercício:} Converta as EDOs de ordem superior abaixo para sistemas de EDOs escalares.
\begin{enumerate}
\sphinxsetlistlabels{\arabic}{enumi}{enumii}{}{.}%
\item {} 
\end{enumerate}
\begin{equation*}
\begin{split}\begin{cases}
y'''(t) + 4y''(t) + 5y'(t) + 2y(t) = 2t^2 + 10t + 8 \\
y(0) = 1, y'(0) = -1, y''(0) = 3
\end{cases}\end{split}
\end{equation*}
\sphinxAtStartPar
Solução analítica: \(y(t) = \exp(-t) + t^2\).
\begin{enumerate}
\sphinxsetlistlabels{\arabic}{enumi}{enumii}{}{.}%
\item {} 
\end{enumerate}
\begin{equation*}
\begin{split}\begin{cases}
y''(t) + 4y'(t) + 13y(t) = 40\cos(t) \\
y(0) = 3, y'(0) = 4, y''(0) = 3
\end{cases}\end{split}
\end{equation*}
\sphinxAtStartPar
Solução analítica: \(y(t) = 3\cos(t) + {\rm sen}(t) + \exp(-2t){\rm sen}(3t)\).


\chapter{Problemas}
\label{\detokenize{extra/extra-edo-superior:problemas}}
\sphinxAtStartPar
\sphinxstylestrong{Problema:} Converta o seguinte sistema de EDOs de 2a. ordem para um sistema maior de EDOs de primeira ordem. Este sistema surge do estudo da atração gravitacional de uma massa por outra:
\begin{equation*}
\begin{split}\begin{cases}
x''(t) = \dfrac{-cx(t)}{r(t)^3} \\
y''(t) = \dfrac{-cy(t)}{r(t)^3} \\
z''(t) = \dfrac{-cz(t)}{r(t)^3},
\end{cases}\end{split}
\end{equation*}
\sphinxAtStartPar
onde \(c\) é uma constante positiva e \(r(t) = [x(t)^2 + y(t)^2 + z(t)^2]^{1/2}\), com \(t\) denotando o tempo.

\sphinxAtStartPar
\sphinxstylestrong{Problema:} Resolva o problema do pêndulo numericamente pelo método de Euler. Considere \(l = 1\) e \(g = 32.2 \, pés/s^2\). Como valores iniciais, escolha \(0 < \theta(0) \le \pi/2\) e \(\theta'(0) = 1\). Experimente vários valores de \(h\) de modo a obter erros pequenos. Plote os gráficos \(t \, vs. \, \theta(t)\), \(t \, vs. \theta'(t)\) e \(\theta(t) \, vs. \theta'(t)\). O movimento parece periódico no tempo? (Use a hipótese de pequenos ângulos.)


\chapter{Sistemas de EDOs}
\label{\detokenize{extra/extra-sistemas-edo:sistemas-de-edos}}\label{\detokenize{extra/extra-sistemas-edo::doc}}
\sphinxAtStartPar
Embora EDOs escalares sejam responsáveis por descrever uma vasta quantidade de fenômenos naturais, muitas aplicações são melhor descritas através de um sistema de EDOs escalares ou de ordem superior. O tratamento numérico de EDOs de alta ordem baseia\sphinxhyphen{}se em uma conversão a sistemas de primeira ordem. Vejamos como escrever a forma geral de um sistema de duas EDOS de primeira ordem:
\begin{equation*}
\begin{split}\begin{cases}
y_1'(t) &=& f_1(t, y_1(t), y_2(t)) \\
y_2'(t) &=& f_2(t, y_1(t), y_2(t))
\end{cases}\end{split}
\end{equation*}
\sphinxAtStartPar
As funções \(f_1(t,z_1,z_2)\) e \(f_2(t,z_1,z_2)\) definem EDOs e as incógnitas são as funções \(y_1(t)\) e \(y_2(t)\). O problema de valor inicial então consiste de resolver o sistema anterior, sujeito às condições iniciais \(y_1(t_0) = y_{1,0}\) e \(y_2(t_0) = y_{2,0}\).

\sphinxAtStartPar
\sphinxstylestrong{Exemplo:} O PVI
\$\(\begin{cases}
y_1'(t) &=& y_1(t) − 2y_2(t) + 4\cos(t) − 2{\rm sen}(t) \\
y_2'(t) &=& 3y_1(t) − 4y_2(t) + 5\cos(t) − 5{\rm sen}(t) \\
y_1(0)  &=& 1 \\ 
y_2(0)  &=& 2
\end{cases}\)\$

\sphinxAtStartPar
tem por solução as funções \(y_1(t) = \cos(t) + {\rm sen}(t)\) e \(y_2(t) = 2\cos(t)\).

\sphinxAtStartPar
\sphinxstylestrong{Exemplo (Modelo de Lotka\sphinxhyphen{}Volterra):} O sistema de EDOs dado por
\$\(\begin{cases}
y_1'(t) &=& A y_1(t)[1 − B y_2(t)] \\
y_2'(t) &=& C y_2(t)[D y_1(t) − 1] \\
y_1(0)  &=& y_{1,0} \\ 
y_2(0)  &=& y_{2,0}
\end{cases}\)\$

\sphinxAtStartPar
com constantes \(A\), \(B\), \(C\), \(D>0\) é conhecido como modelo \sphinxstyleemphasis{predador\sphinxhyphen{}presa}. A variável \(t\) é o tempo, \(y_1(t)\) o número de presas no tempo \(t\) (e.g. coelhos) e \(y_2(t)\) o número de predadores (e.g. raposas). Para apenas um tipo de predador e de presa, este modelo é uma aproximação razoável da realidade.


\section{Sistema com \protect\(m\protect\) EDOs de primeira ordem}
\label{\detokenize{extra/extra-sistemas-edo:sistema-com-m-edos-de-primeira-ordem}}
\sphinxAtStartPar
Um PVI com \(m\) EDOs escalares é dado por
\begin{equation*}
\begin{split}\begin{cases}
y_1'(t) &=& f_1(t, y_1(t), y_2(t),\ldots,y_m(t)) \\
y_2'(t) &=& f_2(t, y_1(t), y_2(t),\ldots,y_m(t)) \\
\vdots  &\vdots& \vdots \\ 
y_m'(t) &=& f_m(t, y_1(t), y_2(t),\ldots,y_m(t)) \\ 
y_1(t_0)  &=& y_{1,0} \\ 
y_2(t_0)  &=& y_{2,0} \\
\vdots  &\vdots& \vdots \\ 
y_m(t_0)  &=& y_{m,0} \\
t_0 \leq t \leq b,
\end{cases}\end{split}
\end{equation*}
\sphinxAtStartPar
cuja solução procurada são as funções \(y_1(t), y_2(t),\ldots,y_m(t)\). Entretanto, a forma anterior não é computacionalmente adequada para se trabalhar. Assim, simplificamo\sphinxhyphen{}na para a forma vetorial
\begin{equation*}
\begin{split}\begin{cases}
{\bf y}'(t) &=& {\bf f}(t,{\bf y}(t)) \\
{\bf y}(t_0) &=& {\bf y}_0,
\end{cases}\end{split}
\end{equation*}
\sphinxAtStartPar
onde
\begin{equation*}
\begin{split}{\bf y}(t) = 
\begin{bmatrix}
y_1(t) \\
\vdots \\
y_m(t) \\
\end{bmatrix}, 
\quad
{\bf y}_0 = 
\begin{bmatrix}
y_{1,0} \\
\vdots \\
y_{m,0} \\
\end{bmatrix}, 
\quad
{\bf f}(t,{\bf y}) = 
\begin{bmatrix}
f_1(t, y_1, y_2,\ldots,y_m) \\
\vdots \\
f_m(t, y_1, y_2,\ldots,y_m) \\
\end{bmatrix}, \end{split}
\end{equation*}
\sphinxAtStartPar
com \({\bf y} = [y_1,y_2,\ldots,y_m]^T\).

\sphinxAtStartPar
\sphinxstylestrong{Exemplo:} O primeiro PVI pode ser reescrito como
\begin{equation*}
\begin{split}\begin{cases}
{\bf y}'(t) &=& {\bf A}{\bf y}(t) + {\bf G}(t) \\
{\bf y}(0) &=& {\bf y}_0, \\
\end{cases}\end{split}
\end{equation*}
\sphinxAtStartPar
com
\begin{equation*}
\begin{split}{\bf y} = 
\begin{bmatrix}
y_1 \\
y_2
\end{bmatrix}, 
\quad
{\bf A} = 
\begin{bmatrix}
1 & -2 \\ 
3 & 4 
\end{bmatrix}, 
\quad
{\bf G}(t) = 
\begin{bmatrix}
4\cos(t) − 2{\rm sen}(t) \\
5\cos(t) − 5{\rm sen}(t)
\end{bmatrix}, 
\quad
{\bf y}_0 = 
\begin{bmatrix}
1 \\
2
\end{bmatrix}\end{split}
\end{equation*}

\section{Métodos numéricos para sistemas}
\label{\detokenize{extra/extra-sistemas-edo:metodos-numericos-para-sistemas}}
\sphinxAtStartPar
Tanto o método de Euler quanto outros métodos numéricos podem ser usados de forma similar para sistemas de EDOs quando aplicados a cada EDO individual. Para isto, lança\sphinxhyphen{}se mão da forma matriz/vetor, em que a derivação é essencialmente a mesma feita para o caso individual.

\sphinxAtStartPar
Lembremos que a série de Taylor desenvolvida para o método de Euler é dada por:
\begin{equation*}
\begin{split}y_j(t_{n+1}) = y_j(t_{n}) + hy_j'(t_{n}) + \dfrac{h^2}{2}y_j''(\xi_{n,j}), \ \ t_n \leq \xi_{n,j} \leq t_{n+1}, \quad j = 1,\ldots,m\end{split}
\end{equation*}
\sphinxAtStartPar
para \(m\) EDOs. Desprezando\sphinxhyphen{}se os termos de erro, o método de Euler em forma vetorial é escrito como
\begin{equation*}
\begin{split}{\bf y}_{n+1} = {\bf y}_{n} + h{\bf f}(t_n,{\bf y}_n), \ \ \ {\bf y}_0 = {\bf y}(0)\end{split}
\end{equation*}

\subsection{Implementação computacional}
\label{\detokenize{extra/extra-sistemas-edo:implementacao-computacional}}
\sphinxAtStartPar
O seguinte código é uma implementação do método de Euler para resolver sistemas de EDOs. O usuário necessita especificar uma função adicional para determinar as EDOs como componentes de um vetor. Isto é feito pela função

\begin{sphinxVerbatim}[commandchars=\\\{\}]
\PYG{k}{def} \PYG{n+nf}{fsys}\PYG{p}{(}\PYG{n}{t}\PYG{p}{,}\PYG{n}{y}\PYG{p}{)}\PYG{p}{:}
    \PYG{p}{(}\PYG{o}{.}\PYG{o}{.}\PYG{o}{.}\PYG{p}{)}
    \PYG{k}{return} \PYG{n}{F}
\end{sphinxVerbatim}

\begin{sphinxVerbatim}[commandchars=\\\{\}]
\PYG{k+kn}{from} \PYG{n+nn}{numpy} \PYG{k+kn}{import} \PYG{o}{*}

\PYG{k}{def} \PYG{n+nf}{euler\PYGZus{}sys}\PYG{p}{(}\PYG{n}{t0}\PYG{p}{,}\PYG{n}{tf}\PYG{p}{,}\PYG{n}{y0}\PYG{p}{,}\PYG{n}{h}\PYG{p}{,}\PYG{n}{f}\PYG{p}{)}\PYG{p}{:}
    \PYG{l+s+sd}{\PYGZdq{}\PYGZdq{}\PYGZdq{}}
\PYG{l+s+sd}{    Resolve o PVI de um sistema de EDOs escalares}
\PYG{l+s+sd}{    y’ = f(t,y), t0 \PYGZlt{}= t \PYGZlt{}= b, y(t0)=y0 pelo metodo de Euler com tamanho de passo h.}
\PYG{l+s+sd}{    O usuario deve fornecer um vetor f contendo as funcoes a serem avaliadas como }
\PYG{l+s+sd}{    membro direito.}
\PYG{l+s+sd}{    }
\PYG{l+s+sd}{    Entrada: }
\PYG{l+s+sd}{        t0  \PYGZhy{} tempo inicial}
\PYG{l+s+sd}{        tf  \PYGZhy{} tempo final }
\PYG{l+s+sd}{        y0  \PYGZhy{} condicao inicial }
\PYG{l+s+sd}{        h   \PYGZhy{} passo }
\PYG{l+s+sd}{        f   \PYGZhy{} vetor de funcoes f(t,y) (anonima)}
\PYG{l+s+sd}{    }
\PYG{l+s+sd}{    Saída:}
\PYG{l+s+sd}{        t   \PYGZhy{} vetor contendo os valores nodais t[i], i = 1,2,...,n}
\PYG{l+s+sd}{        Y   \PYGZhy{} matriz de dimensoes n x m, com m sendo o numero de EDOs}
\PYG{l+s+sd}{             (a i\PYGZhy{}esima linha y[i,:] traz as estimativas de todas }
\PYG{l+s+sd}{              as funcoes y\PYGZus{}j no tempo t[i])}
\PYG{l+s+sd}{    \PYGZdq{}\PYGZdq{}\PYGZdq{}}    
    
    \PYG{n}{m} \PYG{o}{=} \PYG{n}{y0}\PYG{o}{.}\PYG{n}{size}
    \PYG{n}{n} \PYG{o}{=} \PYG{n+nb}{round}\PYG{p}{(}\PYG{p}{(}\PYG{n}{tf} \PYG{o}{\PYGZhy{}} \PYG{n}{t0}\PYG{p}{)}\PYG{o}{/}\PYG{n}{h} \PYG{o}{+} \PYG{l+m+mi}{1}\PYG{p}{)}
    \PYG{n}{t} \PYG{o}{=} \PYG{n}{linspace}\PYG{p}{(}\PYG{n}{t0}\PYG{p}{,}\PYG{n}{t0}\PYG{o}{+}\PYG{p}{(}\PYG{n}{n}\PYG{o}{\PYGZhy{}}\PYG{l+m+mi}{1}\PYG{p}{)}\PYG{o}{*}\PYG{n}{h}\PYG{p}{,}\PYG{n}{n}\PYG{p}{)}
    \PYG{n}{Y} \PYG{o}{=} \PYG{n}{zeros}\PYG{p}{(}\PYG{p}{(}\PYG{n}{n}\PYG{p}{,}\PYG{n}{m}\PYG{p}{)}\PYG{p}{)}
    
    \PYG{n}{Y}\PYG{p}{[}\PYG{l+m+mi}{0}\PYG{p}{,}\PYG{p}{:}\PYG{p}{]} \PYG{o}{=} \PYG{n}{y0}
    
    
    \PYG{k}{for} \PYG{n}{i} \PYG{o+ow}{in} \PYG{n+nb}{range}\PYG{p}{(}\PYG{l+m+mi}{1}\PYG{p}{,}\PYG{n}{n}\PYG{p}{)}\PYG{p}{:}  
        \PYG{n}{Y}\PYG{p}{[}\PYG{n}{i}\PYG{p}{,}\PYG{p}{:}\PYG{p}{]} \PYG{o}{=} \PYG{n}{Y}\PYG{p}{[}\PYG{n}{i}\PYG{o}{\PYGZhy{}}\PYG{l+m+mi}{1}\PYG{p}{,}\PYG{p}{:}\PYG{p}{]} \PYG{o}{+} \PYG{n}{h}\PYG{o}{*}\PYG{n}{f}\PYG{p}{(}\PYG{n}{t}\PYG{p}{[}\PYG{n}{i}\PYG{o}{\PYGZhy{}}\PYG{l+m+mi}{1}\PYG{p}{]}\PYG{p}{,}\PYG{n}{Y}\PYG{p}{[}\PYG{n}{i}\PYG{o}{\PYGZhy{}}\PYG{l+m+mi}{1}\PYG{p}{,}\PYG{p}{:}\PYG{p}{]}\PYG{p}{)}     
        
    \PYG{k}{return} \PYG{n}{t}\PYG{p}{,}\PYG{n}{Y}
\end{sphinxVerbatim}

\sphinxAtStartPar
\sphinxstylestrong{Exercício resolvido:} Resolva o PVI abaixo pelo método de Euler:
\begin{equation*}
\begin{split}\begin{cases}
{\bf y}'(t) &=& {\bf A}{\bf y}(t) + {\bf G}(t) \\
{\bf y}(0) &=& {\bf y}_0, \\
\end{cases}\end{split}
\end{equation*}
\sphinxAtStartPar
com
\begin{equation*}
\begin{split}{\bf y} = 
\begin{bmatrix}
y_1 \\
y_2
\end{bmatrix}, 
\quad
{\bf A} = 
\begin{bmatrix}
1 & -2 \\ 
3 & -4 
\end{bmatrix}, 
\quad
{\bf G}(t) = 
\begin{bmatrix}
4\cos(t) - 2{\rm sen}(t) \\
5\cos(t) - 5{\rm sen}(t) 
\end{bmatrix}, 
\quad
{\bf y}_0 = 
\begin{bmatrix}
1 \\
2
\end{bmatrix}.\end{split}
\end{equation*}
\sphinxAtStartPar
Resolver numericamente pelo método de Euler para \(0 < t \le 5\) e \(h = 0.5\).

\begin{sphinxVerbatim}[commandchars=\\\{\}]
\PYG{c+c1}{\PYGZsh{} define funcao vetorial f(t,y)}
\PYG{k}{def} \PYG{n+nf}{fsys}\PYG{p}{(}\PYG{n}{t}\PYG{p}{,}\PYG{n}{y}\PYG{p}{)}\PYG{p}{:}
    \PYG{l+s+sd}{\PYGZdq{}\PYGZdq{}\PYGZdq{}}
\PYG{l+s+sd}{    Função definida pelo usuario para montar vetor das m funcoes fm(t,(y1,...,ym))}
\PYG{l+s+sd}{    Em cada componente prescrevemos a funcao da EDO correspondente no sistema }
\PYG{l+s+sd}{    de m EDOs. }
\PYG{l+s+sd}{    }
\PYG{l+s+sd}{    Isto e, }
\PYG{l+s+sd}{    }
\PYG{l+s+sd}{    F = [f1(t,y1,...,ym),f2(t,y1,...,ym),...,fm(t,y1,...,ym)]\PYGZca{}T }
\PYG{l+s+sd}{    }
\PYG{l+s+sd}{    Neste exemplo, o sistema possui 2 EDOs, com:}
\PYG{l+s+sd}{    }
\PYG{l+s+sd}{    f1(t,y1,y2) = y1 \PYGZhy{} 2y2 + 4cos(t) \PYGZhy{} 2sin(t)}
\PYG{l+s+sd}{    f2(t,y1,y2) = 3y1 \PYGZhy{} 4y2 + 5cos(t) \PYGZhy{} 5sin(t)    }
\PYG{l+s+sd}{    \PYGZdq{}\PYGZdq{}\PYGZdq{}}    
    \PYG{n}{F} \PYG{o}{=} \PYG{n}{array}\PYG{p}{(}\PYG{p}{[}\PYG{n}{y}\PYG{p}{[}\PYG{l+m+mi}{0}\PYG{p}{]}\PYG{o}{\PYGZhy{}}\PYG{l+m+mi}{2}\PYG{o}{*}\PYG{n}{y}\PYG{p}{[}\PYG{l+m+mi}{1}\PYG{p}{]}\PYG{o}{+}\PYG{l+m+mi}{4}\PYG{o}{*}\PYG{n}{cos}\PYG{p}{(}\PYG{n}{t}\PYG{p}{)}\PYG{o}{\PYGZhy{}}\PYG{l+m+mi}{2}\PYG{o}{*}\PYG{n}{sin}\PYG{p}{(}\PYG{n}{t}\PYG{p}{)}\PYG{p}{,}\PYG{l+m+mi}{3}\PYG{o}{*}\PYG{n}{y}\PYG{p}{[}\PYG{l+m+mi}{0}\PYG{p}{]}\PYG{o}{\PYGZhy{}}\PYG{l+m+mi}{4}\PYG{o}{*}\PYG{n}{y}\PYG{p}{[}\PYG{l+m+mi}{1}\PYG{p}{]}\PYG{o}{+}\PYG{l+m+mi}{5}\PYG{o}{*}\PYG{n}{cos}\PYG{p}{(}\PYG{n}{t}\PYG{p}{)}\PYG{o}{\PYGZhy{}}\PYG{l+m+mi}{5}\PYG{o}{*}\PYG{n}{sin}\PYG{p}{(}\PYG{n}{t}\PYG{p}{)}\PYG{p}{]}\PYG{p}{)}
    
    \PYG{k}{return} \PYG{n}{F}
\end{sphinxVerbatim}

\begin{sphinxVerbatim}[commandchars=\\\{\}]
\PYG{k+kn}{from} \PYG{n+nn}{matplotlib}\PYG{n+nn}{.}\PYG{n+nn}{pyplot} \PYG{k+kn}{import} \PYG{n}{plot}\PYG{p}{,}\PYG{n}{legend}


\PYG{c+c1}{\PYGZsh{} parâmetros}
\PYG{n}{t0}\PYG{p}{,}\PYG{n}{tf} \PYG{o}{=} \PYG{l+m+mi}{0}\PYG{p}{,}\PYG{l+m+mi}{5}
\PYG{n}{h} \PYG{o}{=} \PYG{l+m+mf}{0.05}
\PYG{n}{y0} \PYG{o}{=} \PYG{n}{array}\PYG{p}{(}\PYG{p}{[}\PYG{l+m+mi}{1}\PYG{p}{,}\PYG{l+m+mi}{2}\PYG{p}{]}\PYG{p}{)}

\PYG{c+c1}{\PYGZsh{} solução numérica do sistema}
\PYG{n}{t}\PYG{p}{,}\PYG{n}{Y} \PYG{o}{=} \PYG{n}{euler\PYGZus{}sys}\PYG{p}{(}\PYG{n}{t0}\PYG{p}{,}\PYG{n}{tf}\PYG{p}{,}\PYG{n}{y0}\PYG{p}{,}\PYG{n}{h}\PYG{p}{,}\PYG{n}{fsys}\PYG{p}{)}
\PYG{n}{y1\PYGZus{}num} \PYG{o}{=} \PYG{n}{Y}\PYG{p}{[}\PYG{p}{:}\PYG{p}{,}\PYG{l+m+mi}{0}\PYG{p}{]}
\PYG{n}{y2\PYGZus{}num} \PYG{o}{=} \PYG{n}{Y}\PYG{p}{[}\PYG{p}{:}\PYG{p}{,}\PYG{l+m+mi}{1}\PYG{p}{]}

\PYG{c+c1}{\PYGZsh{} solução analítica}
\PYG{n}{y1\PYGZus{}an} \PYG{o}{=} \PYG{n}{cos}\PYG{p}{(}\PYG{n}{t}\PYG{p}{)} \PYG{o}{+} \PYG{n}{sin}\PYG{p}{(}\PYG{n}{t}\PYG{p}{)}
\PYG{n}{y2\PYGZus{}an} \PYG{o}{=} \PYG{l+m+mi}{2}\PYG{o}{*}\PYG{n}{cos}\PYG{p}{(}\PYG{n}{t}\PYG{p}{)}

\PYG{c+c1}{\PYGZsh{} plotagem}
\PYG{n}{plot}\PYG{p}{(}\PYG{n}{t}\PYG{p}{,}\PYG{n}{y1\PYGZus{}num}\PYG{p}{,}\PYG{l+s+s1}{\PYGZsq{}}\PYG{l+s+s1}{r\PYGZhy{}\PYGZhy{}}\PYG{l+s+s1}{\PYGZsq{}}\PYG{p}{,}\PYG{n}{label}\PYG{o}{=}\PYG{l+s+s1}{\PYGZsq{}}\PYG{l+s+s1}{\PYGZdl{}y\PYGZus{}}\PYG{l+s+s1}{\PYGZob{}}\PYG{l+s+s1}{1,h\PYGZcb{}(t)\PYGZdl{}}\PYG{l+s+s1}{\PYGZsq{}}\PYG{p}{)}
\PYG{n}{plot}\PYG{p}{(}\PYG{n}{t}\PYG{p}{,}\PYG{n}{y2\PYGZus{}num}\PYG{p}{,}\PYG{l+s+s1}{\PYGZsq{}}\PYG{l+s+s1}{b\PYGZhy{}\PYGZhy{}}\PYG{l+s+s1}{\PYGZsq{}}\PYG{p}{,}\PYG{n}{label}\PYG{o}{=}\PYG{l+s+s1}{\PYGZsq{}}\PYG{l+s+s1}{\PYGZdl{}y\PYGZus{}}\PYG{l+s+s1}{\PYGZob{}}\PYG{l+s+s1}{2,h\PYGZcb{}(t)\PYGZdl{}}\PYG{l+s+s1}{\PYGZsq{}}\PYG{p}{)}

\PYG{n}{plot}\PYG{p}{(}\PYG{n}{t}\PYG{p}{,}\PYG{n}{y1\PYGZus{}an}\PYG{p}{,}\PYG{l+s+s1}{\PYGZsq{}}\PYG{l+s+s1}{r}\PYG{l+s+s1}{\PYGZsq{}}\PYG{p}{,}\PYG{n}{label}\PYG{o}{=}\PYG{l+s+s1}{\PYGZsq{}}\PYG{l+s+s1}{\PYGZdl{}y\PYGZus{}1(t)\PYGZdl{}}\PYG{l+s+s1}{\PYGZsq{}}\PYG{p}{)}
\PYG{n}{plot}\PYG{p}{(}\PYG{n}{t}\PYG{p}{,}\PYG{n}{y2\PYGZus{}an}\PYG{p}{,}\PYG{l+s+s1}{\PYGZsq{}}\PYG{l+s+s1}{b}\PYG{l+s+s1}{\PYGZsq{}}\PYG{p}{,}\PYG{n}{label}\PYG{o}{=}\PYG{l+s+s1}{\PYGZsq{}}\PYG{l+s+s1}{\PYGZdl{}y\PYGZus{}2(t)\PYGZdl{}}\PYG{l+s+s1}{\PYGZsq{}}\PYG{p}{)}

\PYG{c+c1}{\PYGZsh{} erro absoluto}
\PYG{n}{plot}\PYG{p}{(}\PYG{n}{t}\PYG{p}{,}\PYG{n+nb}{abs}\PYG{p}{(}\PYG{n}{y1\PYGZus{}an} \PYG{o}{\PYGZhy{}} \PYG{n}{y1\PYGZus{}num}\PYG{p}{)}\PYG{p}{,}\PYG{l+s+s1}{\PYGZsq{}}\PYG{l+s+s1}{r:}\PYG{l+s+s1}{\PYGZsq{}}\PYG{p}{,}\PYG{n}{label}\PYG{o}{=}\PYG{l+s+s1}{\PYGZsq{}}\PYG{l+s+s1}{\PYGZdl{}EA\PYGZus{}1(t)\PYGZdl{}}\PYG{l+s+s1}{\PYGZsq{}}\PYG{p}{)}
\PYG{n}{plot}\PYG{p}{(}\PYG{n}{t}\PYG{p}{,}\PYG{n+nb}{abs}\PYG{p}{(}\PYG{n}{y2\PYGZus{}an} \PYG{o}{\PYGZhy{}} \PYG{n}{y2\PYGZus{}num}\PYG{p}{)}\PYG{p}{,}\PYG{l+s+s1}{\PYGZsq{}}\PYG{l+s+s1}{b:}\PYG{l+s+s1}{\PYGZsq{}}\PYG{p}{,}\PYG{n}{label}\PYG{o}{=}\PYG{l+s+s1}{\PYGZsq{}}\PYG{l+s+s1}{\PYGZdl{}EA\PYGZus{}2(t)\PYGZdl{}}\PYG{l+s+s1}{\PYGZsq{}}\PYG{p}{)}

\PYG{n}{legend}\PYG{p}{(}\PYG{n}{loc}\PYG{o}{=}\PYG{l+m+mi}{3}\PYG{p}{,}\PYG{n}{fontsize}\PYG{o}{=}\PYG{l+m+mi}{10}\PYG{p}{)}\PYG{p}{;}
\end{sphinxVerbatim}

\noindent\sphinxincludegraphics{{extra-sistemas-edo_5_0}.png}

\sphinxAtStartPar
\sphinxstylestrong{Exercício:} Considere o PVI
\begin{equation*}
\begin{split}\begin{cases}
{\bf y}'(t) &=& {\bf A}{\bf y}(t) + {\bf G}(t) \\
{\bf y}(0) &=& {\bf y}_0, \\
\end{cases}\end{split}
\end{equation*}
\sphinxAtStartPar
com
\begin{equation*}
\begin{split}{\bf y} = 
\begin{bmatrix}
y_1 \\
y_2
\end{bmatrix}, 
\quad
{\bf A} = 
\begin{bmatrix}
1 & -2 \\ 
2 & 1 
\end{bmatrix}, 
\quad
{\bf G}(t) = 
\begin{bmatrix}
-2\exp(-t) + 2 \\
-2\exp(-t) + 1 \\
\end{bmatrix}, 
\quad
{\bf y}_0 = 
\begin{bmatrix}
1 \\
1
\end{bmatrix}.\end{split}
\end{equation*}
\sphinxAtStartPar
para \(t \in (0,t_f]\) e \( 0 < h < 1\) convenientemente escolhidos. Verifique que a sua solução analítica é dada por \({\bf y} = [\exp(-t), 1]^T\), resolva\sphinxhyphen{}o numericamente, plote os gráficos das curvas da solução numérica e do \sphinxstyleemphasis{erro relativo} para cada uma. Compare os gráficos com aqueles das curvas da solução analítica.


\section{Problemas}
\label{\detokenize{extra/extra-sistemas-edo:problemas}}
\sphinxAtStartPar
\sphinxstylestrong{Problema:} considere o modelo de Lotka\sphinxhyphen{}Volterra com os parâmetros \(A = 4\), \(B = 1\), \(C = 3\) e \(D = 1\). Usando o método de Euler, resolva\sphinxhyphen{}o para \(0 \le t \le 5\). Use passos de \(h = 0.001\), \(0.0005\) e \(0.00025\) e a condição inicial values \(y_1(0) = 3\), \(y_2(0) = 5\). Plote as funções \(y_1\) e \(y_2\) em função de \(t\) e, depois, plote o gráfico de \(y_1\) \sphinxstyleemphasis{versus} \(y_2\). Comente os resultados.

\sphinxAtStartPar
\sphinxstylestrong{Problema:} Considere o esquema numérico
\begin{equation*}
\begin{split}y_{n+1} =  y_n + \frac{h}{2}[f(t_n,y_n) + f(t_{n+1},y_n + h f(t_n,y_n))]\end{split}
\end{equation*}
\sphinxAtStartPar
Tente adaptá\sphinxhyphen{}lo para resolver o PVI do \sphinxstylestrong{Exercício resolvido}.


\chapter{Método das Linhas}
\label{\detokenize{extra/extra-metodo-linhas:metodo-das-linhas}}\label{\detokenize{extra/extra-metodo-linhas::doc}}
\begin{sphinxVerbatim}[commandchars=\\\{\}]
\PYG{k+kn}{from} \PYG{n+nn}{numpy} \PYG{k+kn}{import} \PYG{o}{*}
\PYG{k+kn}{import} \PYG{n+nn}{sympy} \PYG{k}{as} \PYG{n+nn}{sy} 
\PYG{k+kn}{import} \PYG{n+nn}{matplotlib}\PYG{n+nn}{.}\PYG{n+nn}{pyplot} \PYG{k}{as} \PYG{n+nn}{plt}
\PYG{k+kn}{from} \PYG{n+nn}{mpl\PYGZus{}toolkits}\PYG{n+nn}{.}\PYG{n+nn}{mplot3d}\PYG{n+nn}{.}\PYG{n+nn}{axes3d} \PYG{k+kn}{import} \PYG{n}{Axes3D}
\end{sphinxVerbatim}

\sphinxAtStartPar
O código abaixo usa o método de Euler explícito para resolver a EDP do calor unidimensional.

\begin{sphinxVerbatim}[commandchars=\\\{\}]
\PYG{k}{def} \PYG{n+nf}{met\PYGZus{}linhas\PYGZus{}euler}\PYG{p}{(}\PYG{n}{d0}\PYG{p}{,}\PYG{n}{d1}\PYG{p}{,}\PYG{n}{f}\PYG{p}{,}\PYG{n}{G}\PYG{p}{,}\PYG{n}{t0}\PYG{p}{,}\PYG{n}{T}\PYG{p}{,}\PYG{n}{h}\PYG{p}{,}\PYG{n}{m}\PYG{p}{)}\PYG{p}{:}
    \PYG{l+s+sd}{\PYGZdq{}\PYGZdq{}\PYGZdq{}}
\PYG{l+s+sd}{    Usa o metodo das linhas para resolver}
\PYG{l+s+sd}{    ut = uxx + G(x,t), 0 \PYGZlt{} x \PYGZlt{} 1, 0 \PYGZlt{} t \PYGZlt{} T}
\PYG{l+s+sd}{    com C.C. }
\PYG{l+s+sd}{    u(0,t) = d0(t), u(1,t) = d1(t)}
\PYG{l+s+sd}{    e C.I.}
\PYG{l+s+sd}{    u(x,0) = f(x).}
\PYG{l+s+sd}{    }
\PYG{l+s+sd}{    Usa o metodo de Euler para resolver o }
\PYG{l+s+sd}{    sistema de EDOs. Para a discretizacao, }
\PYG{l+s+sd}{    usa passo espacial delta = 1/m e passo }
\PYG{l+s+sd}{    temporal h. Para estabilidade numérica, }
\PYG{l+s+sd}{    usa h = 1/(2*m**2) ou menor.}
\PYG{l+s+sd}{    \PYGZdq{}\PYGZdq{}\PYGZdq{}}
    \PYG{n}{x} \PYG{o}{=} \PYG{n}{linspace}\PYG{p}{(}\PYG{l+m+mi}{0}\PYG{p}{,}\PYG{l+m+mi}{1}\PYG{p}{,}\PYG{n}{m}\PYG{o}{+}\PYG{l+m+mi}{1}\PYG{p}{)}
    \PYG{n}{delta} \PYG{o}{=} \PYG{l+m+mi}{1}\PYG{o}{/}\PYG{n}{m}
    \PYG{n}{delta\PYGZus{}sqr} \PYG{o}{=} \PYG{n}{delta}\PYG{o}{*}\PYG{o}{*}\PYG{l+m+mi}{2}
        
    \PYG{n}{N} \PYG{o}{=} \PYG{n+nb}{round}\PYG{p}{(}\PYG{p}{(}\PYG{n}{T} \PYG{o}{\PYGZhy{}} \PYG{n}{t0}\PYG{p}{)}\PYG{o}{/}\PYG{n}{h}\PYG{p}{)} \PYG{o}{+} \PYG{l+m+mi}{1}
    \PYG{n}{t} \PYG{o}{=} \PYG{n}{linspace}\PYG{p}{(}\PYG{n}{t0}\PYG{p}{,}\PYG{p}{(}\PYG{n}{N}\PYG{o}{\PYGZhy{}}\PYG{l+m+mi}{1}\PYG{p}{)}\PYG{o}{*}\PYG{n}{h}\PYG{p}{,}\PYG{n}{N}\PYG{p}{)}
    
    \PYG{c+c1}{\PYGZsh{} inicializa u }
    \PYG{n}{u} \PYG{o}{=} \PYG{n}{zeros}\PYG{p}{(}\PYG{p}{(}\PYG{n}{m}\PYG{o}{+}\PYG{l+m+mi}{1}\PYG{p}{,}\PYG{n}{N}\PYG{p}{)}\PYG{p}{)}

    \PYG{c+c1}{\PYGZsh{} construir vetores, se funcao constante}
    \PYG{k}{if} \PYG{n+nb}{isinstance}\PYG{p}{(}\PYG{n}{d0}\PYG{p}{,}\PYG{n+nb}{float}\PYG{p}{)}\PYG{p}{:}
        \PYG{n}{d0} \PYG{o}{=} \PYG{n}{d0}\PYG{o}{*}\PYG{n}{ones}\PYG{p}{(}\PYG{n}{N}\PYG{p}{)}
        \PYG{n}{u}\PYG{p}{[}\PYG{l+m+mi}{0}\PYG{p}{,}\PYG{p}{:}\PYG{p}{]} \PYG{o}{=} \PYG{n}{d0}
    \PYG{k}{else}\PYG{p}{:}
        \PYG{n}{u}\PYG{p}{[}\PYG{l+m+mi}{0}\PYG{p}{,}\PYG{p}{:}\PYG{p}{]} \PYG{o}{=} \PYG{n}{d0}\PYG{p}{(}\PYG{n}{t}\PYG{p}{)}
        
    \PYG{k}{if} \PYG{n+nb}{isinstance}\PYG{p}{(}\PYG{n}{d1}\PYG{p}{,}\PYG{n+nb}{float}\PYG{p}{)}\PYG{p}{:}
        \PYG{n}{d1} \PYG{o}{=} \PYG{n}{d1}\PYG{o}{*}\PYG{n}{ones}\PYG{p}{(}\PYG{n}{N}\PYG{p}{)}
        \PYG{n}{u}\PYG{p}{[}\PYG{n}{m}\PYG{p}{,}\PYG{p}{:}\PYG{p}{]} \PYG{o}{=} \PYG{n}{d1}
    \PYG{k}{else}\PYG{p}{:}
        \PYG{n}{u}\PYG{p}{[}\PYG{n}{m}\PYG{p}{,}\PYG{p}{:}\PYG{p}{]} \PYG{o}{=} \PYG{n}{d1}\PYG{p}{(}\PYG{n}{t}\PYG{p}{)}
        
    \PYG{k}{if} \PYG{n+nb}{isinstance}\PYG{p}{(}\PYG{n}{f}\PYG{p}{,}\PYG{n+nb}{float}\PYG{p}{)}\PYG{p}{:}
        \PYG{n}{f} \PYG{o}{=} \PYG{n}{f}\PYG{o}{*}\PYG{n}{ones}\PYG{p}{(}\PYG{n}{N}\PYG{p}{)}
        \PYG{n}{u}\PYG{p}{[}\PYG{p}{:}\PYG{p}{,}\PYG{l+m+mi}{0}\PYG{p}{]} \PYG{o}{=} \PYG{n}{f}
    \PYG{k}{else}\PYG{p}{:}
        \PYG{n}{u}\PYG{p}{[}\PYG{p}{:}\PYG{p}{,}\PYG{l+m+mi}{0}\PYG{p}{]} \PYG{o}{=} \PYG{n}{f}\PYG{p}{(}\PYG{n}{x}\PYG{p}{)}
            
    \PYG{c+c1}{\PYGZsh{} Resolve para u usando Euler }
    \PYG{k}{for} \PYG{n}{n} \PYG{o+ow}{in} \PYG{n+nb}{range}\PYG{p}{(}\PYG{l+m+mi}{1}\PYG{p}{,}\PYG{n}{N}\PYG{p}{)}\PYG{p}{:}
        \PYG{n}{g} \PYG{o}{=} \PYG{n}{G}\PYG{p}{(}\PYG{n}{x}\PYG{p}{[}\PYG{l+m+mi}{1}\PYG{p}{:}\PYG{n}{m}\PYG{o}{\PYGZhy{}}\PYG{l+m+mi}{1}\PYG{p}{]}\PYG{p}{,}\PYG{n}{t}\PYG{p}{[}\PYG{n}{n}\PYG{o}{\PYGZhy{}}\PYG{l+m+mi}{1}\PYG{p}{]}\PYG{p}{)}
        \PYG{n}{u}\PYG{p}{[}\PYG{l+m+mi}{1}\PYG{p}{:}\PYG{n}{m}\PYG{p}{,}\PYG{n}{n}\PYG{p}{]} \PYG{o}{=} \PYG{n}{u}\PYG{p}{[}\PYG{l+m+mi}{1}\PYG{p}{:}\PYG{n}{m}\PYG{p}{,}\PYG{n}{n}\PYG{o}{\PYGZhy{}}\PYG{l+m+mi}{1}\PYG{p}{]} \PYG{o}{+} \PYG{p}{(}\PYG{n}{h}\PYG{o}{/}\PYG{n}{delta\PYGZus{}sqr}\PYG{p}{)}\PYG{o}{*}\PYG{p}{(}\PYG{n}{u}\PYG{p}{[}\PYG{l+m+mi}{0}\PYG{p}{:}\PYG{p}{(}\PYG{n}{m}\PYG{o}{\PYGZhy{}}\PYG{l+m+mi}{1}\PYG{p}{)}\PYG{p}{,}\PYG{n}{n}\PYG{p}{]}\PYG{p}{)}
        \PYG{o}{\PYGZhy{}} \PYG{l+m+mi}{2}\PYG{o}{*}\PYG{n}{u}\PYG{p}{[}\PYG{l+m+mi}{1}\PYG{p}{:}\PYG{n}{m}\PYG{o}{\PYGZhy{}}\PYG{l+m+mi}{1}\PYG{p}{,}\PYG{n}{n}\PYG{o}{\PYGZhy{}}\PYG{l+m+mi}{1}\PYG{p}{]} \PYG{o}{+} \PYG{n}{u}\PYG{p}{[}\PYG{l+m+mi}{2}\PYG{p}{:}\PYG{n}{m}\PYG{p}{,}\PYG{n}{n}\PYG{o}{\PYGZhy{}}\PYG{l+m+mi}{1}\PYG{p}{]} \PYG{o}{+} \PYG{n}{h}\PYG{o}{*}\PYG{n}{g}
            
    \PYG{k}{return} \PYG{n}{x}\PYG{p}{,}\PYG{n}{t}\PYG{p}{,}\PYG{n}{u}
\end{sphinxVerbatim}

\sphinxAtStartPar
A solução exata para a EDP é \(u(x,t) = \exp(-0.1t){\rm sen}(\pi x)\). A partir desta função obteremos todos os demais termos da EDP. Abaixo, encontraremos as derivadas parciais em relação ao tempo e ao espaço.

\begin{sphinxVerbatim}[commandchars=\\\{\}]
\PYG{c+c1}{\PYGZsh{} u(x,t) = exp(\PYGZhy{}0.1*t)*sin(pi*x)}

\PYG{c+c1}{\PYGZsh{} variaveis simbólicas}
\PYG{n}{xsym}\PYG{p}{,}\PYG{n}{tsym} \PYG{o}{=} \PYG{n}{sy}\PYG{o}{.}\PYG{n}{symbols}\PYG{p}{(}\PYG{l+s+s1}{\PYGZsq{}}\PYG{l+s+s1}{x,t}\PYG{l+s+s1}{\PYGZsq{}}\PYG{p}{)}

\PYG{c+c1}{\PYGZsh{} u(x,t)}
\PYG{n}{u} \PYG{o}{=} \PYG{n}{sy}\PYG{o}{.}\PYG{n}{exp}\PYG{p}{(}\PYG{o}{\PYGZhy{}}\PYG{l+m+mf}{0.1}\PYG{o}{*}\PYG{n}{tsym}\PYG{p}{)}\PYG{o}{*}\PYG{n}{sy}\PYG{o}{.}\PYG{n}{sin}\PYG{p}{(}\PYG{n}{sy}\PYG{o}{.}\PYG{n}{pi}\PYG{o}{*}\PYG{n}{xsym}\PYG{p}{)}

\PYG{c+c1}{\PYGZsh{} dudt }
\PYG{n}{dudt} \PYG{o}{=} \PYG{n}{sy}\PYG{o}{.}\PYG{n}{diff}\PYG{p}{(}\PYG{n}{u}\PYG{p}{,}\PYG{n}{tsym}\PYG{p}{)}

\PYG{c+c1}{\PYGZsh{} d2udx2}
\PYG{n}{d2udx2} \PYG{o}{=} \PYG{n}{sy}\PYG{o}{.}\PYG{n}{diff}\PYG{p}{(}\PYG{n}{u}\PYG{p}{,}\PYG{n}{xsym}\PYG{p}{,}\PYG{l+m+mi}{2}\PYG{p}{)}

\PYG{c+c1}{\PYGZsh{} G(x,t)}
\PYG{n}{Gxt} \PYG{o}{=} \PYG{n}{dudt} \PYG{o}{\PYGZhy{}} \PYG{n}{d2udx2}

\PYG{n+nb}{print}\PYG{p}{(}\PYG{n}{Gxt}\PYG{p}{)}
\end{sphinxVerbatim}

\begin{sphinxVerbatim}[commandchars=\\\{\}]
\PYGZhy{}0.1*exp(\PYGZhy{}0.1*t)*sin(pi*x) + pi**2*exp(\PYGZhy{}0.1*t)*sin(pi*x)
\end{sphinxVerbatim}

\begin{sphinxVerbatim}[commandchars=\\\{\}]
\PYG{c+c1}{\PYGZsh{} d0 = u(0,t) = 0}
\PYG{c+c1}{\PYGZsh{} d1 = u(1,t) = 0}
\PYG{c+c1}{\PYGZsh{} f = sin(pi*x)}
\PYG{c+c1}{\PYGZsh{} \PYGZhy{}0.1*exp(\PYGZhy{}0.1*t)*sin(pi*x) + pi**2*exp(\PYGZhy{}0.1*t)*sin(pi*x)}
\PYG{n}{d0} \PYG{o}{=} \PYG{l+m+mf}{0.0}
\PYG{n}{d1} \PYG{o}{=} \PYG{l+m+mf}{0.0}
\PYG{n}{f} \PYG{o}{=} \PYG{k}{lambda} \PYG{n}{x}\PYG{p}{:} \PYG{n}{sin}\PYG{p}{(}\PYG{n}{pi}\PYG{o}{*}\PYG{n}{x}\PYG{p}{)}
\PYG{n}{G} \PYG{o}{=} \PYG{k}{lambda} \PYG{n}{x}\PYG{p}{,}\PYG{n}{t}\PYG{p}{:} \PYG{o}{\PYGZhy{}}\PYG{l+m+mf}{0.1}\PYG{o}{*}\PYG{n}{exp}\PYG{p}{(}\PYG{o}{\PYGZhy{}}\PYG{l+m+mf}{0.1}\PYG{o}{*}\PYG{n}{t}\PYG{p}{)}\PYG{o}{*}\PYG{n}{sin}\PYG{p}{(}\PYG{n}{pi}\PYG{o}{*}\PYG{n}{x}\PYG{p}{)} \PYG{o}{+} \PYG{n}{pi}\PYG{o}{*}\PYG{o}{*}\PYG{l+m+mi}{2}\PYG{o}{*}\PYG{n}{exp}\PYG{p}{(}\PYG{o}{\PYGZhy{}}\PYG{l+m+mf}{0.1}\PYG{o}{*}\PYG{n}{t}\PYG{p}{)}\PYG{o}{*}\PYG{n}{sin}\PYG{p}{(}\PYG{n}{pi}\PYG{o}{*}\PYG{n}{x}\PYG{p}{)}
\PYG{n}{t0} \PYG{o}{=} \PYG{l+m+mf}{0.0}
\PYG{n}{T} \PYG{o}{=} \PYG{l+m+mf}{1.0}
\PYG{n}{h} \PYG{o}{=} \PYG{l+m+mf}{0.078}
\PYG{n}{m} \PYG{o}{=} \PYG{l+m+mi}{8}

\PYG{c+c1}{\PYGZsh{} sol. numerica}
\PYG{n}{x}\PYG{p}{,}\PYG{n}{t}\PYG{p}{,}\PYG{n}{un} \PYG{o}{=} \PYG{n}{met\PYGZus{}linhas\PYGZus{}euler}\PYG{p}{(}\PYG{n}{d0}\PYG{p}{,}\PYG{n}{d1}\PYG{p}{,}\PYG{n}{f}\PYG{p}{,}\PYG{n}{G}\PYG{p}{,}\PYG{n}{t0}\PYG{p}{,}\PYG{n}{T}\PYG{p}{,}\PYG{n}{h}\PYG{p}{,}\PYG{n}{m}\PYG{p}{)}

\PYG{n}{TT}\PYG{p}{,}\PYG{n}{X} \PYG{o}{=} \PYG{n}{meshgrid}\PYG{p}{(}\PYG{n}{t}\PYG{p}{,}\PYG{n}{x}\PYG{p}{)}

\PYG{c+c1}{\PYGZsh{} sol. exata}
\PYG{n}{ue} \PYG{o}{=} \PYG{n}{exp}\PYG{p}{(}\PYG{o}{\PYGZhy{}}\PYG{l+m+mf}{0.1}\PYG{o}{*}\PYG{n}{TT}\PYG{p}{)}\PYG{o}{*}\PYG{n}{sin}\PYG{p}{(}\PYG{n}{pi}\PYG{o}{*}\PYG{n}{X}\PYG{p}{)}

\PYG{c+c1}{\PYGZsh{} plotagem}
\PYG{n}{fig} \PYG{o}{=} \PYG{n}{plt}\PYG{o}{.}\PYG{n}{figure}\PYG{p}{(}\PYG{n}{figsize}\PYG{o}{=}\PYG{p}{(}\PYG{l+m+mi}{8}\PYG{p}{,}\PYG{l+m+mi}{6}\PYG{p}{)}\PYG{p}{)}
\PYG{n}{ax} \PYG{o}{=} \PYG{n}{fig}\PYG{o}{.}\PYG{n}{add\PYGZus{}subplot}\PYG{p}{(}\PYG{l+m+mi}{1}\PYG{p}{,}\PYG{l+m+mi}{1}\PYG{p}{,}\PYG{l+m+mi}{1}\PYG{p}{,} \PYG{n}{projection}\PYG{o}{=}\PYG{l+s+s1}{\PYGZsq{}}\PYG{l+s+s1}{3d}\PYG{l+s+s1}{\PYGZsq{}}\PYG{p}{)}

\PYG{n}{ax}\PYG{o}{.}\PYG{n}{plot\PYGZus{}surface}\PYG{p}{(}\PYG{n}{X}\PYG{p}{,} \PYG{n}{TT}\PYG{p}{,} \PYG{n}{un}\PYG{p}{,} \PYG{n}{alpha}\PYG{o}{=}\PYG{l+m+mf}{0.4}\PYG{p}{)}
\PYG{n}{ax}\PYG{o}{.}\PYG{n}{plot\PYGZus{}surface}\PYG{p}{(}\PYG{n}{X}\PYG{p}{,} \PYG{n}{TT}\PYG{p}{,} \PYG{n}{ue}\PYG{p}{,} \PYG{n}{alpha}\PYG{o}{=}\PYG{l+m+mf}{0.7}\PYG{p}{)}

\PYG{n}{plt}\PYG{o}{.}\PYG{n}{xlabel}\PYG{p}{(}\PYG{l+s+s1}{\PYGZsq{}}\PYG{l+s+s1}{X}\PYG{l+s+s1}{\PYGZsq{}}\PYG{p}{)}
\PYG{n}{plt}\PYG{o}{.}\PYG{n}{ylabel}\PYG{p}{(}\PYG{l+s+s1}{\PYGZsq{}}\PYG{l+s+s1}{T}\PYG{l+s+s1}{\PYGZsq{}}\PYG{p}{)}
\PYG{n}{plt}\PYG{o}{.}\PYG{n}{title}\PYG{p}{(}\PYG{l+s+s1}{\PYGZsq{}}\PYG{l+s+s1}{comparativo}\PYG{l+s+s1}{\PYGZsq{}}\PYG{p}{)}
\PYG{n}{ax}\PYG{o}{.}\PYG{n}{view\PYGZus{}init}\PYG{p}{(}\PYG{l+m+mi}{45}\PYG{p}{,} \PYG{l+m+mi}{45}\PYG{p}{)}
\end{sphinxVerbatim}

\noindent\sphinxincludegraphics{{extra-metodo-linhas_6_0}.png}

\begin{sphinxVerbatim}[commandchars=\\\{\}]
\PYG{n}{fig} \PYG{o}{=} \PYG{n}{plt}\PYG{o}{.}\PYG{n}{figure}\PYG{p}{(}\PYG{n}{figsize}\PYG{o}{=}\PYG{p}{(}\PYG{l+m+mi}{8}\PYG{p}{,}\PYG{l+m+mi}{6}\PYG{p}{)}\PYG{p}{)}
\PYG{n}{ax} \PYG{o}{=} \PYG{n}{fig}\PYG{o}{.}\PYG{n}{add\PYGZus{}subplot}\PYG{p}{(}\PYG{l+m+mi}{1}\PYG{p}{,}\PYG{l+m+mi}{1}\PYG{p}{,}\PYG{l+m+mi}{1}\PYG{p}{,} \PYG{n}{projection}\PYG{o}{=}\PYG{l+s+s1}{\PYGZsq{}}\PYG{l+s+s1}{3d}\PYG{l+s+s1}{\PYGZsq{}}\PYG{p}{)}
\PYG{n}{ax}\PYG{o}{.}\PYG{n}{plot\PYGZus{}surface}\PYG{p}{(}\PYG{n}{X}\PYG{p}{,} \PYG{n}{TT}\PYG{p}{,} \PYG{n+nb}{abs}\PYG{p}{(}\PYG{n}{ue}\PYG{o}{\PYGZhy{}}\PYG{n}{un}\PYG{p}{)}\PYG{p}{,} \PYG{n}{alpha}\PYG{o}{=}\PYG{l+m+mf}{1.}\PYG{p}{)}
\PYG{n}{plt}\PYG{o}{.}\PYG{n}{xlabel}\PYG{p}{(}\PYG{l+s+s1}{\PYGZsq{}}\PYG{l+s+s1}{X}\PYG{l+s+s1}{\PYGZsq{}}\PYG{p}{)}
\PYG{n}{plt}\PYG{o}{.}\PYG{n}{ylabel}\PYG{p}{(}\PYG{l+s+s1}{\PYGZsq{}}\PYG{l+s+s1}{T}\PYG{l+s+s1}{\PYGZsq{}}\PYG{p}{)}
\PYG{n}{plt}\PYG{o}{.}\PYG{n}{title}\PYG{p}{(}\PYG{l+s+s1}{\PYGZsq{}}\PYG{l+s+s1}{erro}\PYG{l+s+s1}{\PYGZsq{}}\PYG{p}{)}
\PYG{n}{ax}\PYG{o}{.}\PYG{n}{view\PYGZus{}init}\PYG{p}{(}\PYG{l+m+mi}{45}\PYG{p}{,} \PYG{l+m+mi}{45}\PYG{p}{)}
\end{sphinxVerbatim}

\noindent\sphinxincludegraphics{{extra-metodo-linhas_7_0}.png}







\renewcommand{\indexname}{Index}
\printindex
\end{document}